\part{线性代数}
\begingroup
\def\x{\mat{x}}
\def\y{\mat{y}}
\def\a{\mat{\alpha}}
\def\b{\mat{\beta}}
\def\e{\mat{\epsilon}}
\def\g{\mat{\gamma}}
\def\X#1{\x_{#1}}
\def\A{\mat{A}}
\def\wA{\widetilde{\A}}
\def\B{\mat{B}}
\def\wB{\widetilde{\B}}
\def\C{\mat{C}}
\def\D{\mat{D}}
\def\P{\mat{P}}
\def\Q{\mat{Q}}
\def\l{\lambda}
\def\L#1{\l_{#1}}
\def\z{\mat{0}}
\def\E{\mat{E}}
\def\V{\mat{\Lambda}}

%线性代数的核心内容是研究有限维线性空间的结构和线性空间的线性变换.
%由于数域\(P\)上的任意一个\(n\)维线性空间\(V\)都与\(n\)维向量空间\(P^n\)同构,
%\(V\)上的全体线性变换构成的集合\(L(V,V)\)%
%与数域\(P\)上的全体\(n \times n\)矩阵构成的集合\(P^{n \times n}\)同构,
%因此,在本书中线性代数主要研究:
%矩阵理论、\(n\)维向量的线性关系、线性方程组、行列式、二次型、矩阵的特征值与特征向量、内积等内容.

\chapter{线性方程组}
\section{线性方程组}
\subsection{线性方程组的概念}
我们把含有\(n\)个未知量\(\AutoTuple{x}{n}\)的一次方程组
\begin{equation}\label[equation-system]{equation:线性方程组.线性方程组的代数形式}
	\left\{ \begin{array}{l}
		a_{11} x_1 + a_{12} x_2 + \dotsb + a_{1n} x_n = b_1, \\
		a_{21} x_1 + a_{22} x_2 + \dotsb + a_{2n} x_n = b_2, \\
		\hdotsfor{1} \\
		a_{s1} x_1 + a_{s2} x_2 + \dotsb + a_{sn} x_n = b_s
	\end{array} \right.
\end{equation}
称为“\(n\)元\DefineConcept{线性方程组}(\emph{linear system of equations} in \(n\) variables)”.
这里,\(s\)为方程的个数\footnote{%
在\cref{equation:线性方程组.线性方程组的代数形式} 中,
方程的数目\(s\)可以等于未知数的数目\(n\),
也可以不相等(即\(s<n\)或\(s>n\)).
};
我们把数
\[
	a_{ij}
	\quad(i=1,2,\dotsc,s; j=1,2,\dotsc,n)
\]
称为“第\(i\)个方程中\(x_j\)的\DefineConcept{系数}(coefficient)”;
把数
\[
	b_i
	\quad(i=1,2,\dotsc,s)
\]
叫做“第\(i\)个方程的\DefineConcept{常数项}(constant term)”.

\subsection{线性方程组的表示}
\begin{definition}
将\cref{equation:线性方程组.线性方程组的代数形式} 的系数按原位置构成的\(s \times n\)矩阵\[
	\A = \begin{bmatrix}
		a_{11} & a_{12} & \dots & a_{1n} \\
		a_{21} & a_{22} & \dots & a_{2n} \\
		\vdots & \vdots & & \vdots \\
		a_{n1} & a_{n2} & \dots & a_{nn}
	\end{bmatrix}
\]叫做\DefineConcept{系数矩阵}(coefficient matrix).

特别地,如果\(s = n\)(即系数矩阵\(\A\)是一个方阵),
则系数矩阵的行列式\(\abs{\A}\)叫做\DefineConcept{系数行列式}.
\end{definition}

为使表述简明,常用向量、矩阵表示线性方程组.
若记\[
	\x=\begin{bmatrix}
		x_1 \\ x_2 \\ \vdots \\ x_n
	\end{bmatrix},
	\quad
	\b=\begin{bmatrix}
		b_1 \\ b_2 \\ \vdots \\ b_n
	\end{bmatrix},
	\quad
	\a_j=\begin{bmatrix}
		a_{1j} \\ a_{2j} \\ \vdots \\ a_{sj}
	\end{bmatrix},
	\quad
	j=1,2,\dotsc,n.
\]
\(\A\)的列分块阵为\(\A = (\a_1,\a_2,\dotsc,\a_n)\),
则\cref{equation:线性方程组.线性方程组的代数形式} 有以下两种等价表示:
\begin{enumerate}
	\item 矩阵形式\[
		\A \x = \b.
	\]
	\item 向量形式\[
		x_1 \a_1 + x_2 \a_2 + \dotsb + x_n \a_n = \b.
	\]
\end{enumerate}

\begin{definition}
我们把常数项全为零的线性方程组
称为\DefineConcept{齐次线性方程组},
把常数项中有非零数的线性方程组
称为\DefineConcept{非齐次线性方程组}.
\end{definition}

\begin{definition}
如果存在\(n\)个数\(\AutoTuple{c}{n}\)满足\cref{equation:线性方程组.线性方程组的代数形式},即\[
a_{i1} c_1 + a_{12} c_2 + \dotsb + a_{in} c_n \equiv b_i
\quad(i=1,2,\dotsc,s),
\]
则称“\cref{equation:线性方程组.线性方程组的代数形式} 有解”,
或“\cref{equation:线性方程组.线性方程组的代数形式} 是相容的”;
否则,称“\cref{equation:线性方程组.线性方程组的代数形式} 无解”,
或“\cref{equation:线性方程组.线性方程组的代数形式} 是不相容的”.

称这\(n\)个数\(\AutoTuple{c}{n}\)构成的列向量\((\AutoTuple{c}{n})^T\)为%
\cref{equation:线性方程组.线性方程组的代数形式} 的一个\DefineConcept{解}(solution)%
或\DefineConcept{解向量}(solution vector).
\cref{equation:线性方程组.线性方程组的代数形式} 的解的全体构成的集合,
称为“\cref{equation:线性方程组.线性方程组的代数形式} 的\DefineConcept{解集}”.
\end{definition}

\begin{definition}
元素全为零的解向量称为\DefineConcept{零解}.
元素不全为零的解向量称为\DefineConcept{非零解}.
\end{definition}

\begin{theorem}
任意一个齐次线性方程组都有零解.
\end{theorem}

\begin{definition}
解集相等的两个线性方程组称为\DefineConcept{同解方程组}.
\end{definition}

\begin{example}
设\(\A\in M_{s \times n}(\mathbb{R})\).
证明:齐次线性方程组\(\A\x=\z\)与\((\A^T\A)\x=\z\)同解.
\begin{proof}
\def\a{\vb{\xi}}
\def\b{\vb{\eta}}
设\(\a\)是\(\A\x=\z\)的任意一个解,
则\(\A\a=\z\),于是\[
	(\A^T\A)\a=\A^T(\A\a)=\A^T\z=\z,
\]
这就是说\(\a\)是\((\A^T\A)\x=\z\)的一个解.

又设\(\b\)是\((\A^T\A)\x=\z\)的任意一个解,
则\[
	(\A^T\A)\b=\z.
	\eqno(1)
\]
在(1)式等号两边同时左乘\(\b^T\)得\[
	\b^T(\A^T\A)\b=(\A\b)^T(\A\b)=0.
	\eqno(2)
\]

假设\(\A\b=(\AutoTuple{c}{s})^T\).
由(2)式有\[
	(\AutoTuple{c}{s}) (\AutoTuple{c}{s})^T
	= \AutoTuple{c}{n}[+][2]
	= 0.
\]
由于\(\AutoTuple{c}{n}\in\mathbb{R}\),
所以\(\AutoTuple{c}{n}[=]=0\),
\(\A\b=\z\),
这就是说\(\b\)是\(\A\x=\z\)的一个解.

综上所述,\((\A^T\A)\x=\z\)与\(\A\x=\z\)同解.
\end{proof}
\end{example}

需要注意的是,线性方程组的解集与数域有关.

\begin{theorem}
\(n\)元线性方程组的解的情况有且只有三种可能:
无解、有唯一解、有无穷多个解.
\end{theorem}

\section{克拉默法则}
\begin{theorem}[克拉默法则]\label{theorem:线性方程组.克拉默法则}
%@see: https://mathworld.wolfram.com/CramersRule.html
给定一个未知量个数与方程个数相同的线性方程组\[
	\left\{ \begin{array}{l}
		a_{11}x_1 + a_{12}x_2 + \dotsb + a_{1n}x_n = b_1, \\
		a_{21}x_1 + a_{22}x_2 + \dotsb + a_{2n}x_n = b_2, \\
		\hdotsfor{1} \\
		a_{n1}x_1 + a_{n2}x_2 + \dotsb + a_{nn}x_n = b_n.
	\end{array} \right.
	\eqno(1)
\]
如果它的系数行列式满足\[
	d
	=\begin{vmatrix}
	a_{11} & a_{12} & \dots & a_{1n} \\
	a_{21} & a_{22} & \dots & a_{2n} \\
	\vdots & \vdots & \ddots & \vdots \\
	a_{n1} & a_{n2} & \dots & a_{nn}
	\end{vmatrix}
	\neq 0,
\]
则线性方程组(1)存在唯一解:\[
	\x_0
	= \left( \frac{d_1}{d},\frac{d_2}{d},\dotsc,\frac{d_n}{d} \right)^T,
\]
其中\[
	d_k
	= \begin{vmatrix}
		a_{11} & \dots & a_{1\ k-1} & b_1 & a_{1\ k+1} & \dots & a_{1n} \\
		a_{21} & \dots & a_{2\ k-1} & b_2 & a_{2\ k+1} & \dots & a_{2n} \\
		\vdots & & \vdots & \vdots & \vdots & & \vdots \\
		a_{n1} & \dots & a_{n\ k-1} & b_n & a_{n\ k+1} & \dots & a_{nn}
	\end{vmatrix},
	\quad k=1,2,\dotsc,n.
\]
%TODO 在不借助矩阵记号的情况下可能需要数学归纳法证明?
% \begin{proof}
% 上述原线性方程组可以改写为矩阵形式\(\A \x = \b\).
% 因为\(d = \abs{\A} \neq 0\),故\(\A\)可逆,那么线性方程组有唯一解\[
% 	\X0 = \A^{-1} \b = \frac{1}{d} \A^* \b,
% \]其中\(\A^*\)是\(\A\)的伴随矩阵.
% 设\(A_{ij}\)表示\(\A\)中元素\(a_{ij}\ (i,j=1,2,\dotsc,n)\)的代数余子式,
% 则行列式\(d_k\)可按第\(k\)列展开,得\[
% 	d_k = b_1 A_{1k} + b_2 A_{2k} + \dotsb + b_n A_{nk},
% 	\quad k=1,2,\dotsc,n,
% \]
% 则\[
% 	\X0 = \frac{1}{d} \A^* \b
% 	= \frac{1}{d} \begin{bmatrix}
% 		A_{11} & A_{21} & \dots & A_{n1} \\
% 		A_{12} & A_{22} & \dots & A_{n2} \\
% 		\vdots & \vdots & \ddots & \vdots \\
% 		A_{1n} & A_{2n} & \dots & A_{nn}
% 	\end{bmatrix} \begin{bmatrix} b_1 \\ b_2 \\ \vdots \\ b_n \end{bmatrix}
% 	= \frac{1}{d} \begin{bmatrix} d_1 \\ d_2 \\ \vdots \\ d_n \end{bmatrix}.
% 	\qedhere
% \]
% \end{proof}
\end{theorem}

\section{高斯--若尔当消元法}
\begin{definition}
在数域\(P\)上,设矩阵\(\A \in P^{s \times n}\),
向量\(\x \in P^n\),
\(\b \in P^s\).
由\(n\)元线性方程组\[
	\A \x = \b
\]的系数和常数项按原位置构成的\(s \times (n+1)\)矩阵\((\A,\b)\),
称为该\(n\)元线性方程组的\DefineConcept{增广矩阵}(augmented matrix),
记作\(\wA\).
\end{definition}

\begin{theorem}
对线性方程组\(\A \x = \b\)的增广矩阵\(\wA = (\A,\b)\)作初等行变换,
变为\(\widetilde{\C} = (\C,\g)\),
则相应的线性方程组\(\C \x = \g\)与原线性方程组同解.
\begin{proof}
显然存在可逆矩阵\(\P\),
使得\(\wA \to \widetilde{\C} = \P \wA = (\P\A,\P\b)\),
\(\C = \P\A\),\(\g = \P\b\).
如果\(\X0\)是原线性方程组的解,即\(\A \X0 = \b\),
用\(\P\)左乘等式两端得\(\C \X0 = \g\);
反之,若\(\X0\)满足\(\C \X0 = \g\),
用\(\P^{-1}\)左乘等式两端得\(\A \X0 = \b\),
故两方程组同解.
\end{proof}
\end{theorem}
这就是消元法解线性方程组的理论根据.
具体化简\(\wA\)时,
可用一系列初等行变换将其变成一个较为简单的“阶梯形矩阵”(或更简单的“若尔当阶梯形矩阵”).

\begin{definition}
称如下形式的\(s \times n\)矩阵\[
	\A = \begin{bmatrix}
		0 & \dots & a_{1 j_1} & \dots & a_{1 j_2} & \dots & a_{1 j_r} & \dots & a_{1n} \\
		0 & \dots & 0 & \dots & a_{2 j_2} & \dots & a_{2 j_r} & \dots & a_{2n} \\
		\vdots & & \vdots & & \vdots & & \vdots & & \vdots \\
		0 & \dots & 0 & \dots & 0 & \dots & a_{r j_r} & \dots & a_{rn} \\
		0 & \dots & 0 & \dots & 0 & \dots & 0 & \dots & 0 \\
		\vdots & & \vdots & & \vdots & & \vdots & & \vdots \\
		0 & \dots & 0 & \dots & 0 & \dots & 0 & \dots & 0 \\
	\end{bmatrix}
\]为\DefineConcept{阶梯形矩阵}(ladder matrix),
其中\(a_{1 j_1},a_{2 j_2},\dotsc,a_{r j_r}\)均不为零,
\(j_1 < j_2 < \dotsb < j_r\),
\(\A\)的后\(s-r\)行全为零.

元素全为\(0\)的行,称为\DefineConcept{零行}.
元素不全为\(0\)的行,称为\DefineConcept{非零行}.

在非零行中,从左数起,第一个不为\(0\)的元素\(a_{i j_i}\ (i=1,2,\dotsc,r)\),
称为\DefineConcept{主元}(pivot)或\DefineConcept{非零首元}.

以主元为系数的未知量\(x_{j_1},x_{j_2},\dotsc,x_{j_r}\),
称为\DefineConcept{主变量};
不以主元为系数的未知量,称为\DefineConcept{自由未知量}.
\end{definition}

\begin{definition}
若阶梯形矩阵\(\A\)的非零行的非零首元全为\(1\),
它们所在列的其余元素全为零,
则称\(\A\)为\DefineConcept{若尔当阶梯形矩阵}或\DefineConcept{行约化矩阵}.
\end{definition}
% 在Mathematica中可以用RowReduce对矩阵进行初等行变换化为若尔当阶梯形矩阵.

\begin{lemma}
任何一个非零矩阵都可经初等行变换化为阶梯形矩阵.
\begin{proof}
设\(\A_{s \times n} \neq \z\),则\(\A\)经0次或1次交换两行的变换化为\(\B\),即\[
	\A \to \B = \begin{bmatrix}
		0 & \dots & 0 & b_{1 j_1} & \dots & b_{1n} \\
		0 & \dots & 0 & b_{2 j_1} & \dots & b_{2n} \\
		\vdots & & \vdots & \vdots & & \vdots \\
		0 & \dots & 0 & b_{s j_1} & \dots & b_{sn}
	\end{bmatrix},
\]
其中\(b_{1 j_1} \neq 0\).
分别将\(\B\)的第一行的\(-b_{i j_1}/b_{1 j_1}\)倍加到第\(i\ (i=2,3,\dotsc,s)\)行,则\[
	\B \to \C = \begin{bmatrix}
		\z & b_{1 j_1} & \B_1 \\
		\z & \z & \C_1
	\end{bmatrix},
\]
其中\(\C_1\)是\((s-1)\times(n-j_1)\)矩阵,
再对\(\C\)的后面\(s-1\)行作类似的初等行变换化简.
因为矩阵行数有限,这样下去,最后总可化为阶梯形矩阵.
\end{proof}
\end{lemma}

\begin{corollary}\label{theorem:线性方程组.非零矩阵可经初等行变换化为若尔当阶梯形矩阵}
任何一个非零矩阵都可经初等行变换化为若尔当阶梯形矩阵.
\end{corollary}

\begin{theorem}
设\(\A\)为\(n\)阶方阵,
则齐次线性方程组\(\A\x = \z\)有非零解的充要条件是:
\(\abs{\A} = 0\).
\begin{proof}
必要性.
给定\(\X0 \neq \z\)满足\(\A \X0 = \z\).
假设\(\abs{\A} \neq 0\),
由克拉默法则可知方程组有唯一解\(\X0 = \A^{-1} \z = \z\),
矛盾,故\(\abs{\A} = 0\).

充分性.
用数学归纳法证明.
给定\(\abs{\A} = 0\).
当\(n=1\)时,\(\A = \z_{1 \times 1} = 0\),
\(0 \cdot x_1 = 0\)有非零解;
假设当\(n=k-1\geq1\)时,结论成立;
那么当\(n=k\)时,
设\(\A\)经初等行变换\(\P\)化为阶梯形矩阵\[
	\B = \begin{bmatrix}
		b & \g \\
		\z & \C \\
	\end{bmatrix} = \P \A,
\]
其中\(\C\)是\(n-1\)阶方阵,\(\P\)是\(n\)阶可逆矩阵.
取行列式得
\[
	\abs{\B} = \abs{\P} \abs{\A} = 0 = b \abs{\C}.
\]
解同解方程组\(\B \x = \z\).
若\(b = 0\)则\((1,0,\dotsc,0)^T\)是一个非零解;
若\(b \neq 0\),则\(\abs{\C} = 0\),由归纳假设,齐次线性方程组\[
	\C \begin{bmatrix} x_2 \\ x_3 \\ \vdots \\ x_n \end{bmatrix} = \z
\]有非零解\((k_2,k_3,\dotsc,k_n)^T\),
代入\(\B \x = \z\)的第一个方程,
因为\(x_1\)的系数\(b \neq 0\),可解出\(x_1\).
于是\((\AutoTuple{k}{n})^T\)是\(\A \x = \z\)的一个非零解.
\end{proof}
\end{theorem}

\begin{corollary}\label{theorem:线性方程组.方程个数少于未知量个数的齐次线性方程组必有非零解}
方程个数少于未知量个数的齐次线性方程组必有非零解.
\begin{proof}
设数域\(P\)上的线性方程组\(\A \x = \z\),
它的系数矩阵\(\A \in M_{s \times n}(P)\ (s<n)\).
在原方程组的后面添加\(n-s\)个\(0=0\)的方程,解不变,新方程组的系数矩阵为:
\[
	\B_n = \begin{bmatrix} \A_{s \times n} \\ \z_{(n-s) \times n} \end{bmatrix},
\]
由于\(s < n\),有\(\abs{\B} = 0\),故\(\B \x = \z\)有非零解,从而\(\A \x = \z\)有非零解.
\end{proof}
\end{corollary}
注:上述推论的逆命题不成立,即方程组有非零解,不能得出未知量个数与方程个数的关系.


\chapter{向量、矩阵及其基本运算}
线性代数的核心内容是研究有限维线性空间的结构和线性空间的线性变换.由于数域\(P\)上的任意一个\(n\)维线性空间\(V\)都与\(n\)维向量空间\(P^n\)同构,\(V\)上的全体线性变换构成的集合\(L(V,V)\)与数域\(P\)上的全体\(n \times n\)矩阵构成的集合\(P^{n \times n}\)同构,因此,在本书中线性代数主要研究矩阵理论、\(n\)维向量的线性关系、线性方程组、行列式、二次型、矩阵的特征值与特征向量、内积等内容.

\section{向量}
\subsection{向量的概念}
\begin{definition}
由\(n\)个数构成的有序数组,若排成一行记作\[
\a = (\alpha_1, \alpha_2, \dotsc, \alpha_n),
\]则称为\(n\)维\DefineConcept{行向量};
若排成一列记作\[
\a = \begin{bmatrix}
a_1 \\ a_2 \\ \vdots \\ a_n
\end{bmatrix},
\]则称作\(n\)维\DefineConcept{列向量}.
并称数\(a_i\)为\(\a\)的第\(i\)个\DefineConcept{分量}(\(i=1,2,\dotsc,n\)).

\(n\)维行向量和\(n\)维列向量统称为\(n\)维\DefineConcept{向量}(vector),常用小写黑体字母表示.
\end{definition}

\subsection{向量的关系与运算}
\begin{definition}
对于\(n\)维向量\(\a = (\v{a}{n})\)和\(\b = (\v{b}{n})\).如果这两个向量对应的分量全部相等,即\(a_i = b_i\)(\(i=1,2,\dotsc,n\)),则称向量\(\a\)与\(\b\)\DefineConcept{相等},记作\[
\a = \b.
\]
\end{definition}

\begin{definition}
对于\(n\)维向量\(\a = (\v{a}{n})\)和\(\b = (\v{b}{n})\).
\begin{enumerate}
\item {\bf 加法}:称向量\((a_1+b_1,a_2+b_2,\dotsc,a_n+b_n)\)为\(\a\)与\(\b\)的\DefineConcept{和},记作\[
\a+\b=(a_1+b_1,a_2+b_2,\dotsc,a_n+b_n)
\]
\item {\bf 数量乘法}:设\(k\)为数,称向量\((k a_1, k a_2, \dotsc, k a_n)\)为\(k\)与\(\a\)的\DefineConcept{数乘},记作\[
k\a = (k a_1, k a_2, \dotsc, k a_n)
\]
\item 分量全为零的向量\((0,0,\dotsc,0)\)称为\DefineConcept{零向量},记作\(\z\).
\item 称\((-a_1,-a_2,\dotsc,-a_n)\)为\(\a\)的\DefineConcept{负向量},记作\(-\a\).
\end{enumerate}

向量的加法、数乘统称为向量的\DefineConcept{线性运算}.
\end{definition}

\begin{theorem}
由上述定义可知,向量的线性运算满足下面八条运算规律:
\begin{enumerate}
\item \(\a + \b = \b + \a\)
\item \((\a + \b) + \g = \a + (\b + \g)\)
\item \(\a + \z = \a\)
\item \(\a + (-\a) = \z\)
\item \(1 \a = \a\)
\item \(k(l \a) = (kl) \a\)
\item \(k(\a + \b) = k\a + k\b\)
\item \((k+l)\a = k\a + l\a\)
\end{enumerate}
\end{theorem}

\begin{property}
向量的运算还满足以下性质:
\begin{enumerate}
\item \(0 \a = \z\)
\item \((-1) \a = -\a\)
\item \(k \z = \z\)
\item \(k \a = \z \implies k = 0 \lor \a = \z\)
\end{enumerate}
\end{property}

\begin{definition}
对于\(n\)维向量\(\a = (\v{a}{n})\)和\(\b = (\v{b}{n})\).
称向量\(\a\)与向量\(\b\)的负向量\(-\b\)的和为向量\(\a\)与向量\(\b\)的\DefineConcept{差},即\[
\a - \b = \a + (-\b).
\]
\end{definition}

\begin{definition}
两个\(n\)维复向量\(\a=(\v{a}{n})^T\),\(\b=(\v{b}{n})^T\)的\DefineConcept{内积}(inner product)定义为复数\[
a_1b_1 + a_2b_2 + \dotsb + a_nb_n
\]记作\(\vectorinnerproduct{\a}{\b}\).
\end{definition}

\begin{definition}
若向量\(\a\)与\(\b\)满足\(\vectorinnerproduct{\a}{\b}=0\),则称\(\a\)与\(\b\)正交(orthogonal),记作\(\a\perp\b\).
\end{definition}

\begin{property}
向量内积具有以下性质:
\begin{enumerate}
\item \(\vectorinnerproduct{\a}{\b} = \vectorinnerproduct{\b}{\a}\);
\item \(\vectorinnerproduct{(\a+\b)}{\g} = \vectorinnerproduct{\a}{\g} + \vectorinnerproduct{\b}{\g}\);
\item \(\vectorinnerproduct{(k\a)}{\b} = k (\vectorinnerproduct{\a}{\b})\)(\(k\in\mathbb{R}\));
\item \(\a\neq\z \iff \vectorinnerproduct{\a}{\a} > 0\);\(\a=\z \iff \vectorinnerproduct{\a}{\a} = 0\);
\item \(\vectorinnerproduct{\z}{\a} = 0\);
\end{enumerate}
\end{property}

\subsection{向量的长度(模、范数)与单位向量}
\begin{definition}
设\(n\)维向量\(\a = (\v{a}{n})\).
定义向量的\DefineConcept{长度}为\[
\sqrt{\vectorinnerproduct{\a}{\a}} = \sqrt{a_1^2+a_2^2+\dotsb+a_n^2}.
\]同样地可以定义\(n\)维列向量的长度.
2维向量、3维向量的长度常被称作向量的\DefineConcept{模}(module),记作\(\abs{\a}\).
高维(\(n > 3\))向量的长度常被称作向量的\DefineConcept{范数}(norm),记作\(\norm{\a}\).
\end{definition}

\begin{property}
显然有向量的长度为非负实数,即\(\abs{\a}\geqslant0\).
\end{property}

\begin{definition}
长度为1的向量被称为\DefineConcept{单位向量}.
\end{definition}

\begin{definition}
\def\f{\frac{1}{\abs{\a}}}
设\(\a\)满足\(\abs{\a}>0\).用\(\f\)数乘\(\a\),称为将\(\a\)\DefineConcept{单位化},得单位向量\(\f\a\).
\end{definition}

尽管我们通常出于几何(特别是欧式几何)的考量,像上面一样将向量\(\a\)的模(或范数)定义为\(\sqrt{\vectorinnerproduct{\a}{\a}}\),不过我们还可以定义其他形式的模(或范数).
观察上面的模(或范数)的定义,我们可以发现,向量的模(或范数)实际上是满足以下3条性质的映射\(f: P^n \to P, \mat{x} \mapsto m\):
\begin{enumerate}
\item {\bf 非负性},即\(\forall \mat{x} \in P^n : f(\mat{x}) \geqslant 0\);
\item {\bf 齐次性},即\(\forall \mat{x} \in P^n, \forall c \in P : f(c \mat{x}) = \abs{c} f(\mat{x})\);
\item {\bf 三角不等式},即\(\forall \mat{x},\mat{y} \in P^n : f(\mat{x}+\mat{y}) \leqslant f(\mat{x}) + f(\mat{y})\).
\end{enumerate}

据此我们可以定义p范数如下:
\begin{definition}\label{definition:向量与矩阵.p范数}
形如\[
f: \mathbb{R}^n \to \mathbb{R},
\mat{x} = \opair{\v{x}{n}}
\mapsto
\sqrt[p]{\abs{x_1}^p + \abs{x_2}^p + \dotsb + \abs{x_n}^p}
\]的这一类映射\(f(p)\ (p\geqslant1)\),称为\DefineConcept{p范数},记作\(\norm{\mat{x}}_p\).
\end{definition}

易证\[
\begin{array}{lll}
\norm{\mat{x}}_1 &=& \abs{x_1} + \abs{x_2} + \dotsb + \abs{x_n}, \\
\norm{\mat{x}}_2 &=& \sqrt{x_1^2 + x_2^2 + \dotsb + x_n^2}, \\
\norm{\mat{x}}_\infty &=& \max\{\abs{x_1},\abs{x_2},\dotsc,\abs{x_n}\}.
\end{array}
\]

\section{矩阵的概念与运算}
\subsection{矩阵的基本概念}
\begin{definition}[矩阵]
数域\(P\)中\(s \times n\)个数排成的\(s\)行\(n\)列的矩形表,加上方括号:\[
\begin{bmatrix}
a_{11} & a_{12} & \dots & a_{1n} \\
a_{21} & a_{22} & \dots & a_{2n} \\
\vdots & \vdots & \ddots & \vdots \\
a_{s1} & a_{s2} & \dots & a_{sn}
\end{bmatrix}
\]称为数域\(P\)上的\(s \times n\)矩阵(matrix),通常用一个大写黑体字母如\(\A\)或\(\A_{s \times n}\)表示,其中\(s\)称为矩阵的\DefineConcept{行数},\(n\)称为矩阵的\DefineConcept{列数}.
有时矩阵也记作\((a_{ij})_{s \times n}\),其中\(a_{ij}\)(\(i=1,2,\dotsc,s\);\(j=1,2,\dotsc,n\))称为矩阵\(\A\)的第\(i\)行第\(j\)列\DefineConcept{元素},或\(\opair{i,j}\)\DefineConcept{元素}(entry).
“\(\A\)是数域\(P\)上的\(s \times n\)矩阵”可以记作\(\A \in M_{s \times n}(P)\)或\(\A \in P^{s \times n}\).

当数域\(P\)上的矩阵\(\A_{s \times n}\)的行数等于列数(即\(s=n\))时,称\[
\A = \begin{bmatrix}
a_{11} & a_{12} & \dots & a_{1n} \\
a_{21} & a_{22} & \dots & a_{2n} \\
\vdots & \vdots & \ddots & \vdots \\
a_{n1} & a_{n2} & \dots & a_{nn}
\end{bmatrix}
\]为\(n\)阶矩阵或\(n\)阶方阵,记作\(\A \in M_n(P)\).称\(a_{11},a_{22},\dotsc,a_{nn}\)为\(\A\)的\DefineConcept{主对角线}上的元素.

如果两个矩阵行数相同且列数相同,则称两者\DefineConcept{同型}.

特别地,可以视\(n\)维行向量为\(1 \times n\)矩阵,\(n\)维列向量为\(n \times 1\)矩阵.
\end{definition}

\subsection{矩阵的线性运算}
\begin{definition}
设\(\A=(a_{ij})_{s \times n}\)和\(\B=(b_{ij})_{s \times n}\)是(数域\(P\)上)两个\(s \times n\)(同型)矩阵.
如果它们对应的元素分别相等,即\(a_{ij} = b_{ij}\)(\(i=1,2,\dotsc,s\),\(j=1,2,\dotsc,n\)),则称\(\A\)与\(\B\)\DefineConcept{相等},记作\(\A=\B\).
\end{definition}

\begin{definition}
设\(\A=(a_{ij})_{s \times n}\)和\(\B=(b_{ij})_{s \times n}\)是(数域\(P\)上)两个\(s \times n\)(同型)矩阵.
\begin{enumerate}
\item 称矩阵\[
(a_{ij} + b_{ij})_{s \times n} = \begin{bmatrix}
a_{11}+b_{11} & a_{12}+b_{12} & \dots & a_{1n}+b_{1n} \\
a_{21}+b_{21} & a_{22}+b_{22} & \dots & a_{2n}+b_{2n} \\
\vdots & \vdots & \ddots & \vdots \\
a_{s1}+b_{s1} & a_{s2}+b_{s2} & \dots & a_{sn}+b_{sn}
\end{bmatrix}
\]为\(\A\)与\(\B\)的\DefineConcept{和}(sum),记作\(\A+\B\).
\item 设\(k\)为(数域\(P\)中的)数,称矩阵\[
(ka_{ij})_{s \times n} = \begin{bmatrix}
ka_{11} & ka_{12} & \dots & ka_{1n} \\
ka_{21} & ka_{22} & \dots & ka_{2n} \\
\vdots & \vdots & \ddots & \vdots \\
ka_{s1} & ka_{s2} & \dots & ka_{sn}
\end{bmatrix}
\]为数\(k\)与矩阵\(\A\)的\DefineConcept{数乘},记作\(k\A\).
\item 称元素全为零的矩阵为\DefineConcept{零矩阵}(zero matrix),记作\(\z\).
\item 称矩阵\[
(-a_{ij})_{s \times n}=\begin{bmatrix}
-a_{11} & -a_{12} & \dots & -a_{1n} \\
-a_{21} & -a_{22} & \dots & -a_{2n} \\
\vdots & \vdots & \ddots & \vdots \\
-a_{s1} & -a_{s2} & \dots & -a_{sn}
\end{bmatrix}
\]为\(\A\)的\DefineConcept{负矩阵},记作\(-\A\).
\end{enumerate}
\end{definition}

\begin{theorem}
由上述定义可知,矩阵的线性运算满足下面八条运算规律:
\begin{enumerate}
\item \(\A + \B = \B + \A\)
\item \((\A + \B) + \C = \A + (\B + \C)\)
\item \(\A + \z = \A\)
\item \(\A + (-\A) = \z\)
\item \(1 \A = \A\)
\item \(k(l \A) = (kl) \A\)
\item \(k(\A + \B) = k\A + k\B\)
\item \((k+l)\A = k\A + l\A\)
\end{enumerate}
\end{theorem}

\begin{property}
矩阵的运算还满足以下性质:
\begin{enumerate}
\item \(0\A = \z\)
\item \((-1)\A = -\A\)
\item \(k\z = \z\)
\item \(k\A = \z \implies k = 0 \lor \A = \z\)
\end{enumerate}
\end{property}

\subsection{矩阵的乘法}
\begin{definition}
设\(\A = (a_{ij})_{s \times n}\),\(\B = (b_{ij})_{n \times m}\),\(\C = (c_{ij})_{s \times m}\),如果满足\[
c_{ij} = \sum\limits_{k=1}^n {a_{ik} b_{kj}},
\]其中\(i=1,2,\dotsc,s\),\(j=1,2,\dotsc,m\),则称矩阵\(\C\)是\(\A\)与\(\B\)的\DefineConcept{乘积},记作\(\C = \A \B\).
\end{definition}

\begin{example}
设\(\A = (\v{a}{n})^T\),\(\B = (\v{b}{n})\),计算\(\A\B\)和\(\B\A\).
\begin{solution}
\[
\A\B = \begin{bmatrix}
a_1 b_1 & a_1 b_2 & \dots & a_1 b_n \\
a_2 b_1 & a_2 b_2 & \dots & a_2 b_n \\
\vdots & \vdots & \ddots & \vdots \\
a_n b_1 & a_n b_2 & \dots & a_n b_n \\
\end{bmatrix},
\]\[
\B\A = \sum\limits_{i=1}^n{b_i a_i}.
\]
\end{solution}
\end{example}

\begin{definition}
一般地,矩阵乘法不满足交换律.但如果同阶矩阵\(\A = (a_{ij})_n\)和\(\B = (b_{ij})_n\)的乘积满足交换律,即\[
\A \B = \B \A,
\]则称矩阵\(\A\)与\(\B\)\DefineConcept{可交换}.
\end{definition}

\begin{example}
举例说明非零矩阵的乘积可能是零矩阵.
\begin{solution}
矩阵\[
\A = \begin{bmatrix}
0 & 0 & 0 \\
a_{21} & 0 & 0 \\
a_{31} & a_{32} & 0 \\
\end{bmatrix}
\]和\[
\B = \begin{bmatrix}
b_{11} & b_{12} & b_{13} \\
0 & b_{22} & b_{23} \\
0 & 0 & b_{33} \\
\end{bmatrix}
\]都不是零矩阵,但他们的乘积是零矩阵.
\end{solution}
\end{example}

\begin{example}
举例说明矩阵方程\(\A \B = \A \C\)与\(\A = \z \lor \B = \C\)不等价,即消去律不成立.
\end{example}

\begin{theorem}
矩阵乘法满足结合律,%
即对\(\A = (a_{ij})_{s \times n}\),%
\(\B = (b_{ij})_{n \times m}\),%
\(\C = (c_{ij})_{m \times p}\),%
总有\((\A\B)\C = \A(\B\C)\).
\end{theorem}

\begin{definition}
若对角矩阵\(\A=(a_{ij})_n\)总有\(a_{ii} = 1\)(\(i=1,2,\dotsc,n\))成立,则称\(\A\)为\DefineConcept{单位矩阵},记作\(\E\)或\(\E_n\)或\(\mat{I}\).
\end{definition}

\begin{property}
矩阵的乘法满足以下性质:
\begin{enumerate}
\item \(k(\A\B) = (k\A)\B = \A(k\B)\)(\(k \in P\))
\item 左分配律\(\A(\B+\C) = \A\B + \A\C\)
\item 右分配律\((\A+\B)\C = \A\C + \B\C\)
\item \(\z_{q \times s} \A_{s \times n} = \z_{q \times n}\),\(\A_{s \times n} \z_{n \times p} = \z_{s \times p}\)
\item \(\E_s \A_{s \times n} = \A\),\(\A_{s \times n} \E_n = \A\)
\end{enumerate}
\end{property}

\begin{theorem}
单位矩阵\(\E_n\)与任意\(n\)阶方阵可交换.
\end{theorem}

\begin{definition}
设\(\A\)为\(n\)阶方阵,由乘法结合律,可定义\(\A\)的乘幂.规定
\begin{align*}
\A^0 &= \E \\
\A^k &= \underbrace{\A\A\dotsm\A}_{k\text{个}}
\end{align*}
\end{definition}

\begin{theorem}
指数律\(\A^k\A^l = \A^{k+l}\),\((\A^k)^l = \A^{kl}\)(\(k,l \in \mathbb{N}\))成立.
\end{theorem}


对于同阶方阵\(\A\)、\(\B\),恒有\((\A\B)^k = \A(\B\A)^{k-1}\B\)(\(k \geqslant 1\)).
注意,当同阶方阵\(\A\)、\(\B\)不可交换时,通常有\((\A\B)^k \neq \A^k\B^k\).

\begin{example}
设\(\A,\B,\C \in M_n(P)\),则\[
(\A + \B + \C)^2 = \A^2 + \B^2 + \C^2 + \A\B + \B\A + \A\C + \C\A + \B\C + \C\B.
\]
\end{example}

\begin{theorem}
方阵\(\A\)与\(\A^k\)可交换,与\(f(\A) = \sum\limits_{i=0}^n a_i \A^i\)可交换.
\end{theorem}

\begin{theorem}
如果\(g(x)\)和\(h(x)\)是两个多项式,设\(l(x) = g(x) + h(x)\),\(m(x) = g(x) h(x)\),则\[
l(\A) = g(\A) + h(\A),
\quad
m(\A) = g(\A) h(\A)
\]
\end{theorem}

\begin{example}
设\(\A,\B,\x \in M_n(P)\).证明:若\(\A\x=\x\B\),则对任意多项式\[
f(x) = a_0 + a_1 x + a_2 x^2 + \dotsb + a_k x^k,
\quad
a_0,a_1,a_2,\dotsc,a_k \in P,
\]总有\[
f(\A) \x = \x f(\B).
\]
\begin{proof}
因为\[
\A\x = \x\B,
\]所以\[
\A^2 \x = \A(\A\x) = \A(\x\B),
\]\[
\x \B^2 = (\x\B)\B = (\A\x)\B.
\]以此类推,可证\[
\A^n \x = \x \B^n,
\quad n=1,2,\dotsc.
\]

因为\[
f(\A) = a_0 \E + a_1 \A + a_2 \A^2 + \dotsb + a_k \A^k,
\]根据左分配律,有\begin{gather}
f(\A) \x = a_0 \x + a_1 \A\x + a_2 \A^2 \x + \dotsb + a_k \A^k \x. \tag1
\end{gather}同理,根据右分配律,有\begin{gather}
\x f(\B) = a_0 \x + a_1 \x\B + a_2 \x \B^2 + \dotsb + a_k \x \B^k. \tag2
\end{gather}
因为式1与式2中各项逐项相等,故\(f(\A) \x = \x f(\B)\).
\end{proof}
\end{example}

\subsection{矩阵的转置}
\begin{definition}[矩阵的转置]
将矩阵\(\A=(a_{ij})_{s \times n}\)的行、列互换得到矩阵\(\B=(b_{ij})_{n \times s}\),其中\(b_{ij} = a_{ji}\),那么称矩阵\(\B\)为\(\A\)的\DefineConcept{转置矩阵}(简称\DefineConcept{转置},transpose),记为\(\A^T\)(或\(\A'\)).
\end{definition}
矩阵的转置运算可以看作集合\(P^{s \times n}\)到集合\(P^{n \times s}\)的一个映射.

\begin{definition}[厄米矩阵]
设矩阵\(\A = (a_{ij})_{s \times n} \in \mathbb{C}^{s \times n}\).
将\(\A\)转置后,再对各元素取共轭,称为\(\A\)的\DefineConcept{厄米矩阵},记作\(\A^H\),即\[
\A^H = \overline{\A^T} = \overline{\A}^T.
\]
\end{definition}


\begin{property}
矩阵的转置满足以下性质:
\begin{enumerate}
\item \((\A^T)^T = \A\);
\item \((\A+\B)^T = \A^T + \B^T\);
\item \((k\A)^T = k\A^T \quad(k \in P)\).
\end{enumerate}
\end{property}

\begin{theorem}\label{theorem:矩阵.矩阵乘积的转置}
对于矩阵\(\A \in P^{s \times n}\)和矩阵\(\B \in P^{n \times m}\),总有\[
(\A\B)^T = \B^T \A^T.
\]
\begin{proof}
\begin{align*}
(\A \B)^T_{\opair{i,j}}
&= (\A \B)_{\opair{j,i}}
= \A_{\opair{j,*}} \B_{\opair{*,i}}
= \sum\limits_{k=1}^n a_{jk} b_{ki}, \\
(\B^T \A^T)_{\opair{i,j}}
&= \B^T_{\opair{i,*}} \A^T_{\opair{*,j}}
= (\B_{\opair{*,i}})^T (\A_{\opair{j,*}})^T
= \sum\limits_{k=1}^n b_{ki} a_{jk}
= \sum\limits_{k=1}^n a_{jk} b_{ki}.
\qedhere
\end{align*}
\end{proof}
\end{theorem}

\begin{corollary}
\((\A_1 \A_2 \dotsb \A_n)^T = \A_n^T \dots \A_2^T \A_1^T\).
\end{corollary}

\begin{example}
设\(\A=\diag(\v{a}{n})\),\(\B=\diag(\v{b}{n})\),试证:\[
\A \B = \diag(a_1 b_1,a_2 b_2,\dotsc,a_n b_n).
\]
\end{example}

\begin{example}
设二阶矩阵\(\A=\begin{bmatrix} 1 & \lambda \\ 0 & 1 \end{bmatrix}\),试证\(\A^k=\begin{bmatrix} 1 & k\lambda \\ 0 & 1 \end{bmatrix}\).
\end{example}

\begin{example}
设\[
\A = \begin{bmatrix}
\cos t & \sin t \\
-\sin t & \cos t
\end{bmatrix}.
\]令\[
\B = \begin{bmatrix}
\cos t & 0 \\
0 & \cos t
\end{bmatrix}, \qquad
\C = \begin{bmatrix}
0 & \sin t \\
-\sin t & 0
\end{bmatrix},
\]则\(\A=\B+\C\).
因为\[
\B\C = \begin{bmatrix}
\cos t & 0 \\
0 & \cos t
\end{bmatrix} \begin{bmatrix}
0 & \sin t \\
-\sin t & 0
\end{bmatrix} = \begin{bmatrix}
0 & \cos t \sin t \\
-\cos t \sin t & 0
\end{bmatrix},
\]\[
\C\B = \begin{bmatrix}
0 & \sin t \\
-\sin t & 0
\end{bmatrix} \begin{bmatrix}
\cos t & 0 \\
0 & \cos t
\end{bmatrix} = \begin{bmatrix}
0 & \cos t \sin t \\
-\cos t \sin t & 0
\end{bmatrix},
\]所以\(\B\C=\C\B\),\(\B\)与\(\C\)可交换.
由牛顿二项式定理有,\[
\A^n=(\B+\C)^n
=\sum\limits_{k=0}^n C_n^k \B^{n-k} \C^k.
\]
\end{example}

\subsection{矩阵的分块}
\begin{definition}
设\(\A\)是\(s \times n\)矩阵,用一些水平线与垂直线(通常不画出来)将\(\A\)分成若干块小矩阵,每一块称为\(\A\)的\DefineConcept{子块},以子块为元素的矩阵叫\DefineConcept{分块阵}.

将\(\A\)按行分块,\[
\A=\begin{bmatrix} \A_1 \\ \A_2 \\ \vdots \\ \A_s \end{bmatrix},
\]其中\(\A_i=(a_{i1},a_{i2},\dotsc,a_{in})\)为\(\A\)的第\(i\)(\(i=1,2,\dotsc,s\))行(向量).

将\(\A\)按列分块,\[
\A=\begin{bmatrix} \a_1 & \a_2 & \dots & \a_n \end{bmatrix},
\]其中\(\a_j=(a_{1j},a_{2j},\dotsc,a_{sj})^T\)为\(\A\)的第\(j\)(\(j=1,2,\dotsc,n\))列(向量).

\(\A\)的分块矩阵的一般形式为\[
\begin{matrix}
& \begin{matrix} n_1 & n_2 & \dots & n_r \end{matrix} \\
\begin{matrix} s_1 \\ s_2 \\ \vdots \\ s_t \end{matrix} & \begin{bmatrix}
\A_{11} & \A_{12} & \dots & \A_{1r} \\
\A_{21} & \A_{22} & \dots & \A_{2r} \\
\vdots & \vdots & \ddots & \vdots \\
\A_{t1} & \A_{t2} & \dots & \A_{tr}
\end{bmatrix}
\end{matrix}
\]
\end{definition}

\begin{theorem}
分块阵的运算服从以下规律:
\begin{enumerate}
\item {\bf 分块阵的加法}

设\(\A,\B \in P^{s \times n}\),
若将\(\A\)、\(\B\)按同样的规则分块为
\[
    \A=(\A_{ij})_{t \times r}, \qquad
    \B=(\B_{ij})_{t \times r},
\]
其中\(\A_{ij}\)、\(\B_{ij}\)都是\(s_i \times n_j\)矩阵
(\(i=1,2,\dotsc,t;\;j=1,2,\dotsc,r\)),
则\[
    \A+\B=(\A_{ij}+\B_{ij})_{t \times r}.
\]

\item {\bf 分块阵的数乘}
\[
    k\A=(k\A_{ij})_{t \times r}.
\]

\item {\bf 分块阵的转置}
设\(\A=(\A_{ij})_{t \times r}\),则\[
    \A^T=(\A_{ji}^T)_{r \times t}.
\]
这就是说,在转置分块阵时,要将每个子块转置.

\item {\bf 分块阵的乘法}
设\(\A \in P^{s \times n}\),\(\B \in P^{n \times m}\),
若将\(\A\)、\(\B\)分别分块为
\[
    \A=(\A_{ij})_{t \times r}, \qquad
    \B=(\B_{jk})_{r \times p},
\]
且\(\A\)的列的分块法与\(\B\)的行的分块法一致,即
\begin{align*}
    \A = \begin{matrix}
        & \begin{matrix} n_1 & n_2 & \dots & n_r \end{matrix} \\
        \begin{matrix} s_1 \\ s_2 \\ \vdots \\ s_t \end{matrix} & \begin{bmatrix}
        \A_{11} & \A_{12} & \dots & \A_{1r} \\
        \A_{21} & \A_{22} & \dots & \A_{2r} \\
        \vdots & \vdots & \ddots & \vdots \\
        \A_{t1} & \A_{t2} & \dots & \A_{tr}
        \end{bmatrix}
    \end{matrix}
    ,\qquad
    \B = \begin{matrix}
        & \begin{matrix} m_1 & m_2 & \dots & m_p \end{matrix} \\
        \begin{matrix} n_1 \\ n_2 \\ \vdots \\ n_r \end{matrix} & \begin{bmatrix}
        \B_{11} & \B_{12} & \dots & \B_{1p} \\
        \B_{21} & \B_{22} & \dots & \B_{2p} \\
        \vdots & \vdots & \ddots & \vdots \\
        \B_{r1} & \B_{r2} & \dots & \B_{rp}
        \end{bmatrix},
    \end{matrix}
\end{align*}
则
\begin{align*}
    \A\B = \begin{matrix}
        & \begin{matrix} m_1 & m_2 & \dots & m_p \end{matrix} \\
        \begin{matrix} s_1 \\ s_2 \\ \vdots \\ s_t \end{matrix} & \begin{bmatrix}
        \C_{11} & \C_{12} & \dots & \C_{1p} \\
        \C_{21} & \C_{22} & \dots & \C_{2p} \\
        \vdots & \vdots & \ddots & \vdots \\
        \C_{t1} & \C_{t2} & \dots & \C_{tp}
        \end{bmatrix}
    \end{matrix}.
\end{align*}
其中\(\C_{ij}=\sum\limits_{k=1}^r \A_{ik} \B_{kj}\ (i=1,2,\dotsc,t;\;j=1,2,\dotsc,p)\).
\end{enumerate}
\end{theorem}

\begin{example}
设\(\A\)、\(\B\)都是\(n\)阶上三角阵,证明:\(\A\B\)是上三角阵.
\begin{proof}
利用数学归纳法.当\(n=1\)时,\(\A=a\),\(\B=b\),\(\A\B = ab\),结论成立.
假设\(n=k\)时上三角阵的乘积是上三角阵.当\(n=k+1\)时,对矩阵\(\A\)和\(b\)分块如下:\[
\A = \begin{bmatrix}
a_{11} & \A_2 \\
\z & \A_4
\end{bmatrix},
\qquad
\B = \begin{bmatrix}
b_{11} & \B_2 \\
\z & \B_4
\end{bmatrix},
\]其中\(\A_4\)和\(\B_4\)都是\(k\)阶上三角阵,由归纳假设,\(\A_4 \B_4\)是\(k\)阶上三角阵,则\[
\A\B = \begin{bmatrix}
a_{11} b_{11} & a_{11} \B_2 + \A_2 \B_4 \\
\z & \A_4 \B_4
\end{bmatrix},
\]即\(\A\B\)是\(k+1\)阶上三角阵.
\end{proof}
\end{example}

\section{特殊矩阵}
\subsection{三角矩阵}
\begin{definition}
设方阵\(\A=(a_{ij})_n\).
如果\(\A\)满足当\(i>j\)时必有\(a_{ij} = 0\)成立,称之为\DefineConcept{上三角形矩阵}(或\DefineConcept{上三角阵}).
如果\(\A\)满足当\(i<j\)时必有\(a_{ij} = 0\)成立,称之为\DefineConcept{下三角形矩阵}(或\DefineConcept{下三角阵}).
\end{definition}

\subsection{对角矩阵}
\begin{definition}
若方阵\(\A=(a_{ij})_n\)满足:
\begin{enumerate}
\item 当\(i \neq j\)时,必有\(a_{ij} = 0\)成立;
\item 存在\(i\)使得\(a_{ii} \neq 0\).
\end{enumerate}
则称\(\A\)为\DefineConcept{对角矩阵},记作\(\diag(a_{11},a_{22},\dotsc,a_{nn})\).
\end{definition}

\subsection{对称矩阵}
\begin{definition}
若方阵\(\A\)满足\(\A^T = \A\),则称\(\A\)为\DefineConcept{对称矩阵}(symmetric matrix).
\end{definition}

\begin{example}
设矩阵\(\A = (a_{ij})_n \in \mathbb{R}^{n \times n}\),\(\A \neq \z\),试证:\(\A \A^T\)为对称矩阵,且\(\A \A^T \neq \z\).
\begin{proof}
因为\((\A \A^T)^T = (\A^T)^T \A^T = \A \A^T\),所以\(\A \A^T\)是对称矩阵.设\[
\B = (b_{ij})_n = \A \A^T = \begin{bmatrix}
a_{11} & a_{12} & a_{13} \\
a_{21} & a_{22} & a_{23} \\
a_{31} & a_{32} & a_{33} \\
\end{bmatrix} \begin{bmatrix}
a_{11} & a_{21} & a_{31} \\
a_{12} & a_{22} & a_{32} \\
a_{13} & a_{23} & a_{33} \\
\end{bmatrix},
\]则\[
b_{ij} = a_{i1} a_{j1} + a_{i2} a_{j2} + a_{i3} a_{j3},
\quad i,j=1,2,3.
\]特别地,\(\B\)主对角线上的元素\(b_{ii}\)(\(i=1,2,3\))是实数的平方和,即\[
b_{ii} = a_{i1}^2 + a_{i2}^2 + a_{i3}^2 \geqslant 0,
\quad i=1,2,3.
\]因为\(\A \neq \z\),所以存在\(a_{kl} \neq 0\),从而有\(b_{kk} > 0\),故\(\B = \A \A^T \neq \z\).
\end{proof}
\end{example}

\begin{example}
设\(\A\)和\(\B\)是同阶对称矩阵,试证:\(\A\B\)是对称矩阵的充要条件是\(\A\B = \B\A\).
\begin{proof}
因为\(\A\)和\(\B\)都是对称矩阵,所以\(\A^T = \A\),\(\B^T = \B\).
又因为\(\A\B = \B\A\),\((\A\B)^T = \B^T \A^T = \B\A = \A\B\),即\(\A\B\)是对称矩阵.
\end{proof}
\end{example}

\subsection{反对称矩阵}
\begin{definition}
若方阵\(\A\)满足条件\(\A^T = -\A\),则称\(\A\)为\DefineConcept{反对称矩阵}(antisymmetric matrix)或\DefineConcept{斜对称矩阵}.
\end{definition}

\begin{property}
反对称矩阵主对角线上的元素全为零.
\end{property}

\begin{example}
零矩阵\(\z\)是唯一一个既是实对称矩阵又是实反对称矩阵的矩阵.
\begin{proof}
\(\A^T = \A = -\A \implies 2\A = \A+\A = \z \implies \A = \z\).
\end{proof}
\end{example}

\begin{example}
设\(\A\)是一个方阵,证明:\(\A+\A^T\)为对称矩阵,\(\A-\A^T\)为反对称矩阵.
\begin{proof}
因为\((\A+\A^T)^T = \A^T+\A\),而\((\A-\A^T)^T = \A^T - \A = -(\A-\A^T)\),所以\(\A+\A^T\)为对称矩阵,\(\A-\A^T\)为反对称矩阵.
显然有\(\A = \frac{\A + \A^T}{2} + \frac{\A - \A^T}{2}\).
\end{proof}
\end{example}

\begin{example}
设\(\A\)是3阶实对称矩阵,\(\B\)是3阶实反对称矩阵,\(\A^2 = \B^2\),试证:\(\A = \B = \z\).
\begin{proof}
设\(\A = (a_{ij})_n\),\(\B = (b_{ij})_n\).
因为\(\A = \A^T\),\(\A^2 = \A^T \A\),所以\(\A^2\)的\(\opair{i,j}\)元素为\(a_{1i} a_{1j} + a_{2i} a_{2j} + \dotsb + a_{ni} a_{nj}\).
因为\(\B = -\B^T\),\(\B^2 = -\B^T \B\),所以\(\B^2\)的\(\opair{i,j}\)元素为\(-(b_{1i} b_{1j} + b_{2i} b_{2j} + \dotsb + b_{ni} b_{nj})\).
因为\(\A^2 = \B^2\),所以\[
a_{1i} a_{1j} + a_{2i} a_{2j} + \dotsb + a_{ni} a_{nj}
= -(b_{1i} b_{1j} + b_{2i} b_{2j} + \dotsb + b_{ni} b_{nj}).
\]

当\(i=j\)时,上式变为\(
a_{1i}^2 + a_{2i}^2 + \dotsb + a_{ni}^2
= -(b_{1i}^2 + b_{2i}^2 + \dotsb + b_{ni}^2)
\),又由\(a_{ij},b_{ij} \in \mathbb{R}\)可知\(a_{1i}^2 + a_{2i}^2 + \dotsb + a_{ni}^2 \geqslant 0\),\(-(b_{1i}^2 + b_{2i}^2 + \dotsb + b_{ni}^2) \leqslant 0\),所以\[
a_{1i}^2 + a_{2i}^2 + \dotsb + a_{ni}^2
= -(b_{1i}^2 + b_{2i}^2 + \dotsb + b_{ni}^2) = 0,
\]进而有\[
a_{1i} = a_{2i} = \dotsb = a_{ni} = b_{1i} = b_{2i} = \dotsb = b_{ni} = 0.
\qedhere
\]
\end{proof}
\end{example}

\subsection{幂零矩阵}
\begin{definition}
设矩阵\(\A \in M_n(K)\).
若\(\exists m\in\mathbb{N}^+\),
使得\(\A^m = \z\),
则称“\(\A\)是\DefineConcept{幂零矩阵}”;
称使得\(\A^m = \z\)成立的最小正整数\[
    \min\Set{ m\in\mathbb{N}^+ \given \A^m = \z }
\]为“\(\A\)的\DefineConcept{幂零指数}”.
\end{definition}

\section{特殊矩阵}
\subsection{三角矩阵}
\begin{definition}
设\(\A=(a_{ij})_n\).
\begin{enumerate}
	\item 如果\(\A\)满足\[
		a_{ij} = 0
		\quad(i>j),
	\]
	则称之为\DefineConcept{上三角形矩阵},
	简称\DefineConcept{上三角阵}.

	\item 如果\(\A\)满足\[
		a_{ij} = 0
		\quad(i<j),
	\]
	则称之为\DefineConcept{下三角形矩阵},
	简称\DefineConcept{下三角阵}.
\end{enumerate}
\end{definition}

\begin{example}
设\(\A,\B\)都是\(n\)阶上三角阵,证明:\(\A\B\)是上三角阵.
\begin{proof}
利用数学归纳法.
当\(n=1\)时,\(\A=a\),\(\B=b\),\(\A\B = ab\),结论成立.

假设\(n=k\)时上三角阵的乘积是上三角阵.
当\(n=k+1\)时,对矩阵\(\A\)和\(b\)分块如下:\[
	\A = \begin{bmatrix}
		a_{11} & \A_2 \\
		\z & \A_4
	\end{bmatrix},
	\qquad
	\B = \begin{bmatrix}
		b_{11} & \B_2 \\
		\z & \B_4
	\end{bmatrix},
\]
其中\(\A_4\)和\(\B_4\)都是\(k\)阶上三角阵,
由归纳假设,\(\A_4 \B_4\)是\(k\)阶上三角阵,
则\[
	\A\B = \begin{bmatrix}
		a_{11} b_{11} & a_{11} \B_2 + \A_2 \B_4 \\
		\z & \A_4 \B_4
	\end{bmatrix},
\]
即\(\A\B\)是\(k+1\)阶上三角阵.
\end{proof}
\end{example}

\subsection{对角矩阵}
\begin{definition}
若方阵\(\A=(a_{ij})_n\)满足:\[
	[i \neq j \implies a_{ij} = 0]
	\land
	(\exists i)[a_{ii}\neq0].
\]
则称\(\A\)为\DefineConcept{对角矩阵},
记作\(\diag(a_{11},a_{22},\dotsc,a_{nn})\).
\end{definition}

\subsection{对称矩阵,厄米矩阵}
\begin{definition}
若矩阵\(\A \in M_n(K)\)满足\[
    \A^T = \A,
\]
则称\(\A\)为\DefineConcept{对称矩阵}(symmetric matrix).
\end{definition}

\begin{definition}
如果矩阵\(\A \in M_n(K)\)满足\[
    \A^H = \A,
\]
那么把\(\A\)称为\DefineConcept{厄米矩阵}(Hermitian matrix).
\end{definition}

\begin{example}
设矩阵\(\A = (a_{ij})_n\).
试证:\(\A\A^T\)为对称矩阵.
\begin{proof}
因为\((\A \A^T)^T = (\A^T)^T \A^T = \A \A^T\),所以\(\A \A^T\)是对称矩阵.
\end{proof}
\end{example}

\begin{example}
设\(\A\)和\(\B\)是同阶对称矩阵.
试证:\(\A\B\)是对称矩阵的充要条件是\(\A\B = \B\A\).
\begin{proof}
因为\(\A\)和\(\B\)都是对称矩阵,所以\[
	\A^T = \A, \qquad
	\B^T = \B.
\]
又因为\(\A\B = \B\A\),\[
	(\A\B)^T = \B^T \A^T = \B\A = \A\B,
\]
所以\(\A\B\)是对称矩阵.
\end{proof}
\end{example}

\subsection{反对称矩阵}
\begin{definition}
若方阵\(\A\)满足条件\(\A^T = -\A\),
则称\(\A\)为\DefineConcept{反对称矩阵}(antisymmetric matrix)或\DefineConcept{斜对称矩阵}.
\end{definition}

\begin{property}
反对称矩阵主对角线上的元素全为零.
\end{property}

\begin{example}
零矩阵\(\z\)是唯一一个既是实对称矩阵又是实反对称矩阵的矩阵.
\begin{proof}
\(\A^T = \A = -\A \implies 2\A = \A+\A = \z \implies \A = \z\).
\end{proof}
\end{example}

\begin{example}
设\(\A\)是一个方阵,证明:\(\A+\A^T\)为对称矩阵,\(\A-\A^T\)为反对称矩阵.
\begin{proof}
因为\((\A+\A^T)^T = \A^T+\A\),而\((\A-\A^T)^T = \A^T - \A = -(\A-\A^T)\),所以\(\A+\A^T\)为对称矩阵,\(\A-\A^T\)为反对称矩阵.
显然有\(\A = \frac{\A + \A^T}{2} + \frac{\A - \A^T}{2}\).
\end{proof}
\end{example}

\begin{example}
设\(\A\)是3阶实对称矩阵,\(\B\)是3阶实反对称矩阵,\(\A^2 = \B^2\),试证:\(\A = \B = \z\).
\begin{proof}
设\(\A = (a_{ij})_n\),\(\B = (b_{ij})_n\).
因为\(\A = \A^T\),\(\A^2 = \A^T \A\),所以\(\A^2\)的\(\opair{i,j}\)元素为\(a_{1i} a_{1j} + a_{2i} a_{2j} + \dotsb + a_{ni} a_{nj}\).
因为\(\B = -\B^T\),\(\B^2 = -\B^T \B\),所以\(\B^2\)的\(\opair{i,j}\)元素为\(-(b_{1i} b_{1j} + b_{2i} b_{2j} + \dotsb + b_{ni} b_{nj})\).
因为\(\A^2 = \B^2\),所以\[
a_{1i} a_{1j} + a_{2i} a_{2j} + \dotsb + a_{ni} a_{nj}
= -(b_{1i} b_{1j} + b_{2i} b_{2j} + \dotsb + b_{ni} b_{nj}).
\]

当\(i=j\)时,上式变为\(
a_{1i}^2 + a_{2i}^2 + \dotsb + a_{ni}^2
= -(b_{1i}^2 + b_{2i}^2 + \dotsb + b_{ni}^2)
\),又由\(a_{ij},b_{ij} \in \mathbb{R}\)可知\(a_{1i}^2 + a_{2i}^2 + \dotsb + a_{ni}^2 \geq 0\),\(-(b_{1i}^2 + b_{2i}^2 + \dotsb + b_{ni}^2) \leq 0\),所以\[
a_{1i}^2 + a_{2i}^2 + \dotsb + a_{ni}^2
= -(b_{1i}^2 + b_{2i}^2 + \dotsb + b_{ni}^2) = 0,
\]进而有\[
a_{1i} = a_{2i} = \dotsb = a_{ni} = b_{1i} = b_{2i} = \dotsb = b_{ni} = 0.
\qedhere
\]
\end{proof}
\end{example}

\subsection{幂零矩阵}
\begin{definition}
设矩阵\(\A \in M_n(K)\).
若\(\exists m\in\mathbb{N}^+\),
使得\(\A^m = \z\),
则称“\(\A\)是\DefineConcept{幂零矩阵}”;
称使得\(\A^m = \z\)成立的最小正整数\[
    \min\Set{ m\in\mathbb{N}^+ \given \A^m = \z }
\]为“\(\A\)的\DefineConcept{幂零指数}”.
\end{definition}

\section{矩阵的乘法}
\begin{definition}
设\(\A = (a_{ij})_{s \times n}\),
\(\B = (b_{ij})_{n \times m}\),
\(\C = (c_{ij})_{s \times m}\).
如果满足\[
	c_{ij} = \sum\limits_{k=1}^n {a_{ik} b_{kj}},
	\quad
	i=1,2,\dotsc,s;j=1,2,\dotsc,m,
\]
则称矩阵\(\C\)是\(\A\)与\(\B\)的\DefineConcept{乘积},
记作\(\C = \A \B\).
\end{definition}

可以注意到,如果我们分别对\(\A\)和\(\B\)做行分块和列分块,得\[
	\A=(\AutoTuple{\a}{s}[,][T])^T, \qquad
	\B=(\AutoTuple{\b}{m}),
\]
那么有\[
	c_{ij} = \a_i^T \b_j,
	\quad
	i=1,2,\dotsc,s;j=1,2,\dotsc,m.
\]
如果我们分别对\(\A\)和\(\B\)做列分块和行分块,
\begingroup%
\def\mx{\mat{\xi}}%
\def\mz{\mat{\zeta}}%
得\[
	\A=(\AutoTuple{\mx}{n}), \qquad
	\B=(\AutoTuple{\mz}{n}[,][T])^T,
\]
那么有\[
	\mx_i \mz_i^T
	= \begin{bmatrix}
		a_{1i} b_{i1} & a_{1i} b_{i2} & \dots & a_{1i} b_{im} \\
		a_{2i} b_{i1} & a_{2i} b_{i2} & \dots & a_{2i} b_{im} \\
		\vdots & \vdots & & \vdots \\
		a_{si} b_{i1} & a_{si} b_{i2} & \dots & a_{si} b_{im} \\
	\end{bmatrix},
	\quad
	i=1,2,\dotsc,n,
\]
于是\[
	\A\B=\sum\limits_{i=1}^n \mx_i \mz_i^T.
\]
\endgroup%

一般地,矩阵乘法不满足交换律.
但如果同阶矩阵\(\A = (a_{ij})_n\)和\(\B = (b_{ij})_n\)的乘积满足交换律,
即\[
	\A \B = \B \A,
\]则称“\(\A\)与\(\B\) \DefineConcept{可交换}”.

\begin{example}
举例说明非零矩阵的乘积可能是零矩阵.
\begin{solution}
矩阵\[
	\A = \begin{bmatrix}
		0 & 0 & 0 \\
		a_{21} & 0 & 0 \\
		a_{31} & a_{32} & 0 \\
	\end{bmatrix}
	\quad\text{和}\quad
	\B = \begin{bmatrix}
		b_{11} & b_{12} & b_{13} \\
		0 & b_{22} & b_{23} \\
		0 & 0 & b_{33} \\
	\end{bmatrix}
\]可以都不是零矩阵,
但他们的乘积\(\A\B\)一定是零矩阵.
\end{solution}
\end{example}

\begin{example}
举例说明矩阵方程\(\A \B = \A \C\)与\(\A = \z \lor \B = \C\)不等价,即消去律不成立.
\end{example}

\begin{theorem}
矩阵乘法满足结合律.
\begin{proof}
设\(\A = (a_{ij})_{s \times n},
\B = (b_{ij})_{n \times m},
\C = (c_{ij})_{m \times r}\).
显然\((\A\B)\C\)与\(\A(\B\C)\)同型,都是\(s \times r\)矩阵.
由于\begin{align*}
	\MatrixEntry{((\A\B)\C)}{i,j}
	&= \sum\limits_{l=1}^m (\MatrixEntry{(\A\B)}{i,l}) \cdot c_{lj} \\
	&= \sum\limits_{l=1}^m \left( \sum\limits_{k=1}^n a_{ik} b_{kl} \right) c_{lj} \\
	&= \sum\limits_{l=1}^m \left( \sum\limits_{k=1}^n a_{ik} b_{kl} c_{lj} \right), \\
	\MatrixEntry{(\A(\B\C))}{i,j}
	&= \sum\limits_{k=1}^n a_{ik} \cdot (\MatrixEntry{(\B\C)}{k,j}) \\
	&= \sum\limits_{k=1}^n a_{ik} \left( \sum\limits_{l=1}^m b_{kl} c_{lj} \right) \\
	&= \sum\limits_{k=1}^n \left( \sum\limits_{l=1}^m a_{ik} b_{kl} c_{lj} \right) \\
	&= \sum\limits_{l=1}^m \left( \sum\limits_{k=1}^n a_{ik} b_{kl} c_{lj} \right),
\end{align*}
于是\((\A\B)\C = \A(\B\C)\).
\end{proof}
\end{theorem}

\begin{definition}
设\(\A\in M_n(K)\).
若有\[
	\A(i,j) = \left\{ \begin{array}{cl}
		1, & i=j, \\
		0, & i\neq j,
	\end{array} \right.
\]
则称“\(\A\)是\DefineConcept{单位矩阵}(identity matrix)”,记作\(\E\).
%@see: https://mathworld.wolfram.com/IdentityMatrix.html
\end{definition}

\begin{property}
矩阵的乘法满足以下性质:
\begin{gather}
	(\forall\A,\B\in M_{s\times n}(K))(\forall k\in K)[k(\A\B) = (k\A)\B = \A(k\B)], \\
	(\forall\A,\B,\C\in M_{s\times n}(K))[\A(\B+\C) = \A\B + \A\C], \\
	(\forall\A,\B,\C\in M_{s\times n}(K))[(\A+\B)\C = \A\C + \B\C], \\
	(\forall\A\in M_{s\times n}(K))[\z_{q \times s} \A = \z_{q \times n}], \\
	(\forall\A\in M_{s\times n}(K))[\A \z_{n \times p} = \z_{s \times p}], \\
	(\forall\A\in M_{s\times n}(K))[\E_s \A = \A], \\
	(\forall\A\in M_{s\times n}(K))[\A \E_n = \A].
\end{gather}
\end{property}

\begin{definition}
设\(\A\)为\(n\)阶方阵,由乘法结合律,可定义\(\A\)的乘幂.
规定
\begin{align*}
	\A^0 &\defeq \E, \\
	\A^k &\defeq \underbrace{\A\A\dotsm\A}_{k\text{个}}.
\end{align*}
\end{definition}

\begin{theorem}
指数律成立,即
\begin{gather}
	\A^k\A^l = \A^{k+l}, \\
	(\A^k)^l = \A^{kl}, \quad k,l \in \mathbb{N}.
\end{gather}
\end{theorem}

对于同阶方阵\(\A\)、\(\B\),恒有\begin{equation}
	(\A\B)^k = \A(\B\A)^{k-1}\B,
	\quad k \geq 1.
\end{equation}
注意,当同阶方阵\(\A\)、\(\B\)不可交换时,通常有\[
	(\A\B)^k \neq \A^k\B^k.
\]

\begin{example}
设\(\A,\B,\C \in M_n(P)\),则\[
(\A + \B + \C)^2 = \A^2 + \B^2 + \C^2 + \A\B + \B\A + \A\C + \C\A + \B\C + \C\B.
\]
\end{example}

\begin{theorem}
方阵\(\A\)与\(\A^k\)可交换,与\(f(\A) = \sum\limits_{i=0}^n a_i \A^i\)可交换.
\end{theorem}

\begin{theorem}
如果\(g(x)\)和\(h(x)\)是两个多项式,设\(l(x) = g(x) + h(x)\),\(m(x) = g(x) h(x)\),则\[
l(\A) = g(\A) + h(\A),
\quad
m(\A) = g(\A) h(\A)
\]
\end{theorem}

\begin{example}
设\(\A,\B,\x \in M_n(P)\).证明:若\(\A\x=\x\B\),则对任意多项式\[
f(x) = a_0 + a_1 x + a_2 x^2 + \dotsb + a_k x^k,
\quad
a_0,a_1,a_2,\dotsc,a_k \in P,
\]总有\[
f(\A) \x = \x f(\B).
\]
\begin{proof}
因为\[
\A\x = \x\B,
\]所以\[
\A^2 \x = \A(\A\x) = \A(\x\B),
\]\[
\x \B^2 = (\x\B)\B = (\A\x)\B.
\]以此类推,可证\[
\A^n \x = \x \B^n,
\quad n=1,2,\dotsc.
\]

因为\[
f(\A) = a_0 \E + a_1 \A + a_2 \A^2 + \dotsb + a_k \A^k,
\]根据左分配律,有\begin{gather}
f(\A) \x = a_0 \x + a_1 \A\x + a_2 \A^2 \x + \dotsb + a_k \A^k \x. \tag1
\end{gather}同理,根据右分配律,有\begin{gather}
\x f(\B) = a_0 \x + a_1 \x\B + a_2 \x \B^2 + \dotsb + a_k \x \B^k. \tag2
\end{gather}
因为式1与式2中各项逐项相等,故\(f(\A) \x = \x f(\B)\).
\end{proof}
\end{example}

\begin{theorem}\label{theorem:矩阵.矩阵乘积的转置}
设\(\A\in M_{s\times n}(K),
\B\in M_{n \times t}(K)\),
则有\[
	(\A\B)^T = \B^T \A^T.
\]
\begin{proof}
假设\[
	\A=(a_{ij})_{s \times n}
	=(\AutoTuple{\a}{s})^T, \qquad
	\B=(b_{ij})_{n \times t}
	=(\AutoTuple{\b}{t}),
\]
其中\(\a_i\in K^n\ (i=1,2,\dotsc,s)\)是行向量,
\(\b_j\in K^n\ (j=1,2,\dotsc,t)\)是列向量.
又假设\[
	\A\B=(c_{ij})_{s \times t}, \qquad
	\B^T\A^T=(d_{ij})_{t \times s}.
\]
那么\[
	c_{ij}%\(\A\)的第\(i\)行,\(\B\)的第\(j\)列
	= \a_i\cdot\b_j
	= \sum\limits_{k=1}^n a_{ik}b_{kj},
\]\[
	d_{ij}%\(\B^T\)的第\(i\)行,\(\A^T\)的第\(j\)列
	= \b_i\cdot\a_j%相当于\(\B\)的第\(i\)列,\(\A\)的第\(j\)行
	= \sum\limits_{k=1}^n a_{jk}b_{ki},
\]
可见\(c_{ij}=d_{ji}\ (i=1,2,\dotsc,s;j=1,2,\dotsc,t)\).
因此,\((\A\B)^T = \B^T \A^T\).
\end{proof}
\end{theorem}

\begin{corollary}
\((\A_1 \A_2 \dotsb \A_n)^T = \A_n^T \dots \A_2^T \A_1^T\).
\end{corollary}

\begin{example}
设\(\A=\diag(\AutoTuple{a}{n}),
\B=\diag(\AutoTuple{b}{n})\).
那么\[
	\A\B = \diag(a_1b_1,a_2b_2,\dotsc,a_nb_n).
\]
\end{example}

\begin{example}
设二阶矩阵\(\A=\begin{bmatrix} 1 & \lambda \\ 0 & 1 \end{bmatrix}\).
试证\(\A^k=\begin{bmatrix} 1 & k\lambda \\ 0 & 1 \end{bmatrix}\).
\end{example}

\begin{example}
设\[
	\A = \begin{bmatrix}
	\cos t & \sin t \\
	-\sin t & \cos t
	\end{bmatrix}.
\]
令\[
	\B = \begin{bmatrix}
		\cos t & 0 \\
		0 & \cos t
	\end{bmatrix},
	\qquad
	\C = \begin{bmatrix}
		0 & \sin t \\
		-\sin t & 0
	\end{bmatrix},
\]
则\(\A=\B+\C\).
因为\[
	\B\C = \begin{bmatrix}
		\cos t & 0 \\
		0 & \cos t
	\end{bmatrix}
	\begin{bmatrix}
		0 & \sin t \\
		-\sin t & 0
	\end{bmatrix}
	= \begin{bmatrix}
		0 & \cos t \sin t \\
		-\cos t \sin t & 0
	\end{bmatrix},
\]\[
	\C\B = \begin{bmatrix}
		0 & \sin t \\
		-\sin t & 0
	\end{bmatrix}
	\begin{bmatrix}
		\cos t & 0 \\
		0 & \cos t
	\end{bmatrix}
	= \begin{bmatrix}
		0 & \cos t \sin t \\
		-\cos t \sin t & 0
	\end{bmatrix},
\]
所以\(\B\C=\C\B\),\(\B\)与\(\C\)可交换.
由牛顿二项式定理有,\[
	\A^n=(\B+\C)^n
	=\sum\limits_{k=0}^n C_n^k \B^{n-k} \C^k.
\]
\end{example}

\section{向量的内积}
\begin{definition}
设\(\a=(\AutoTuple{a}{n})^T\)和\(\b=(\AutoTuple{b}{n})^T\)都是\(n\)维复向量.
我们把复数\[
	a_1b_1 + a_2b_2 + \dotsb + a_nb_n
\]
称为“\(\a\)与\(\b\)的\DefineConcept{内积}(inner product)”,
记作\(\vectorinnerproduct{\a}{\b}\).
\end{definition}

\begin{definition}
若向量\(\a\)与\(\b\)满足\(\vectorinnerproduct{\a}{\b}=0\),
则称\(\a\)与\(\b\)正交(orthogonal),
记作\(\a\perp\b\).
\end{definition}

\begin{property}
向量内积具有以下性质:
\begin{enumerate}
	\item \(\vectorinnerproduct{\a}{\b} = \vectorinnerproduct{\b}{\a}\);
	\item \(\vectorinnerproduct{(\a+\b)}{\g} = \vectorinnerproduct{\a}{\g} + \vectorinnerproduct{\b}{\g}\);
	\item \(\vectorinnerproduct{(k\a)}{\b} = k (\vectorinnerproduct{\a}{\b})\ (k\in\mathbb{R})\);
	\item \(\a\neq\z \iff \vectorinnerproduct{\a}{\a} > 0\);\(\a=\z \iff \vectorinnerproduct{\a}{\a} = 0\);
	\item \(\vectorinnerproduct{\z}{\a} = 0\);
\end{enumerate}
\end{property}

\subsection{向量的长度(模、范数)与单位向量}
\begin{definition}
设\(n\)维向量\(\a = (\AutoTuple{a}{n})\).
定义向量的\DefineConcept{长度}为\[
	\sqrt{\vectorinnerproduct{\a}{\a}} = \sqrt{a_1^2+a_2^2+\dotsb+a_n^2}.
\]同样地可以定义\(n\)维列向量的长度.
2维向量、3维向量的长度常被称作向量的\DefineConcept{模}(module),记作\(\abs{\a}\).
高维(\(n > 3\))向量的长度常被称作向量的\DefineConcept{范数}(norm),记作\(\norm{\a}\).
\end{definition}

\begin{property}
显然有向量的长度为非负实数,即\(\abs{\a}\geq0\).
\end{property}

\begin{definition}
长度为1的向量被称为\DefineConcept{单位向量}.
\end{definition}

\begin{definition}
\def\f{\frac{1}{\abs{\a}}}
设\(\a\)满足\(\abs{\a}>0\).
用\(\f\)数乘\(\a\),
称为“将\(\a\) \DefineConcept{单位化}”,
得单位向量\(\f\a\).
\end{definition}

尽管我们通常出于几何(特别是欧式几何)的考量,
像上面一样将向量\(\a\)的模(或范数)定义为\(\sqrt{\vectorinnerproduct{\a}{\a}}\),
不过我们还可以定义其他形式的模(或范数).
观察上面的模(或范数)的定义,我们可以发现,
向量的模(或范数)实际上是满足以下3条性质的映射
\begingroup%
\def\x{\vb{x}}%
\def\y{\vb{y}}%
\(f\colon K^n \to K, \x \mapsto m\):
\begin{enumerate}
	\item {\bf 非负性},
	即\((\forall \x \in K^n)[f(\x) \geq 0]\);
	\item {\bf 齐次性},
	即\((\forall \x \in K^n)(\forall c \in P)[f(c \x) = \abs{c} f(\x)]\);
	\item {\bf 三角不等式},
	即\((\forall \x,\y \in K^n)[f(\x+\y) \leq f(\x) + f(\y)]\).
\end{enumerate}

\begin{definition}\label{definition:向量与矩阵.p范数}
形如\[
	f\colon\mathbb{R}^n \to \mathbb{R},
	\x = \opair{\AutoTuple{x}{n}}
	\mapsto
	\sqrt[p]{\abs{x_1}^p + \abs{x_2}^p + \dotsb + \abs{x_n}^p}
\]的这一类映射,
称为 \DefineConcept{\(p\)范数},
记作\(\norm{\x}_p\).
\end{definition}

易见
\begin{gather}
	\norm{\x}_1 = \abs{x_1} + \abs{x_2} + \dotsb + \abs{x_n}, \\
	\norm{\x}_2 = \sqrt{x_1^2 + x_2^2 + \dotsb + x_n^2}, \\
	\norm{\x}_\infty = \max\{\abs{x_1},\abs{x_2},\dotsc,\abs{x_n}\}.
\end{gather}
\endgroup%

\section{分块矩阵的运算}
分块阵的运算服从以下规律:
\begin{enumerate}
	\item {\bf 分块阵的加法}

	设\(\A,\B \in M_{s \times n}(K)\),
	若将\(\A\)和\(\B\)按同样的规则分块为\[
		\A=(\A_{ij})_{t \times r}, \qquad
		\B=(\B_{ij})_{t \times r},
	\]
	其中\(\A_{ij},\B_{ij}\in M_{s_i \times n_j}(K)\ (i=1,2,\dotsc,t;j=1,2,\dotsc,r)\),
	则\[
		\A+\B=(\A_{ij}+\B_{ij})_{t \times r}.
	\]

	\item {\bf 分块阵的数乘}

	设\(\A\in M_{s \times n}(K)\),
	若将\(\A\)分块为\[
		\A=(\A_{ij})_{t \times r},
	\]
	其中\(\A_{ij}\in M_{s_i \times n_j}(K)\ (i=1,2,\dotsc,t;j=1,2,\dotsc,r)\),
	则\[
		k\A=(k\A_{ij})_{t \times r}.
	\]

	\item {\bf 分块阵的转置}

	设\(\A\in M_{s \times n}(K)\),
	若将\(\A\)分块为\[
		\A=(\A_{ij})_{t \times r},
	\]
	其中\(\A_{ij}\in M_{s_i \times n_j}(K)\ (i=1,2,\dotsc,t;j=1,2,\dotsc,r)\),
	则\[
		\A^T=(\A_{ji}^T)_{r \times t}.
	\]
	这就是说,在转置分块阵时,要将每个子块转置.

	\item {\bf 分块阵的乘法}

	设\(\A\in M_{s \times n}(K),
	\B\in M_{n \times m}(K)\),
	若将\(\A\)、\(\B\)分别分块为\[
		\A=(\A_{ij})_{t \times r}, \qquad
		\B=(\B_{jk})_{r \times p},
	\]
	且\(\A\)的列的分块法与\(\B\)的行的分块法一致,即\[
		\A = \begin{matrix}
			& \begin{matrix} n_1 & n_2 & \dots & n_r \end{matrix} \\
			\begin{matrix} s_1 \\ s_2 \\ \vdots \\ s_t \end{matrix} & \begin{bmatrix}
			\A_{11} & \A_{12} & \dots & \A_{1r} \\
			\A_{21} & \A_{22} & \dots & \A_{2r} \\
			\vdots & \vdots & & \vdots \\
			\A_{t1} & \A_{t2} & \dots & \A_{tr}
			\end{bmatrix}
		\end{matrix},
		\qquad
		\B = \begin{matrix}
			& \begin{matrix} m_1 & m_2 & \dots & m_p \end{matrix} \\
			\begin{matrix} n_1 \\ n_2 \\ \vdots \\ n_r \end{matrix} & \begin{bmatrix}
			\B_{11} & \B_{12} & \dots & \B_{1p} \\
			\B_{21} & \B_{22} & \dots & \B_{2p} \\
			\vdots & \vdots & & \vdots \\
			\B_{r1} & \B_{r2} & \dots & \B_{rp}
			\end{bmatrix},
		\end{matrix}
	\]
	则\[
		\A\B = \begin{matrix}
			& \begin{matrix} m_1 & m_2 & \dots & m_p \end{matrix} \\
			\begin{matrix} s_1 \\ s_2 \\ \vdots \\ s_t \end{matrix} & \begin{bmatrix}
			\C_{11} & \C_{12} & \dots & \C_{1p} \\
			\C_{21} & \C_{22} & \dots & \C_{2p} \\
			\vdots & \vdots & & \vdots \\
			\C_{t1} & \C_{t2} & \dots & \C_{tp}
			\end{bmatrix}
		\end{matrix}.
	\]
	其中\(\C_{ij}=\sum_{k=1}^r \A_{ik} \B_{kj}\ (i=1,2,\dotsc,t;j=1,2,\dotsc,p)\).
\end{enumerate}

\section{初等矩阵}
\subsection{初等变换}
\begin{definition}
对矩阵施行以下变换,称为矩阵的\DefineConcept{初等行变换}(elementary row operation):
\begin{enumerate}
	\item 互换两行的位置;
	\item 用一非零数\(c\)乘以某行;
	\item 将某行的\(k\)倍加到另一行.
\end{enumerate}
类似地,可以定义矩阵的\DefineConcept{初等列变换}(elementary column operation):
\begin{enumerate}
	\item 互换两列的位置;
	\item 用一非零数\(c\)乘以某列;
	\item 将某列的\(k\)倍加到另一列.
\end{enumerate}
矩阵的初等行变换、初等列变换统称为矩阵的\DefineConcept{初等变换}(elementary operation).
\end{definition}

%我们约定:
%矩阵\(\A\)经过一次初等行变换\(\sigma_1\)化为矩阵\(\B\)的过程
%可以表示为在连接矩阵\(\A\)和\(\B\)的箭头上方标记\(\sigma_1\),即\[
%	\A \xlongrightarrow{\sigma_1} \B;
%\]而矩阵\(\A\)经过一次初等列变换\(\sigma_2\)化为矩阵\(\B\)的过程
%可以表示为在连接矩阵\(\A\)和\(\B\)的箭头下方标记\(\sigma_2\),即\[
%	\A \xlongrightarrow[\sigma_2]{} \B.
%\]

\subsection{初等矩阵的概念}
\begin{definition}\label{definition:逆矩阵.矩阵等价}
若矩阵\(\A\)可以经过一系列初等变换化为矩阵\(\B\),
则称“\(\A\)与\(\B\) \DefineConcept{等价}(equivalent)”,
或“\(\A\)与\(\B\) \DefineConcept{相抵}”,
记作\(\A\cong\B\).
\end{definition}

\begin{definition}
由\(n\)阶单位矩阵\(\E\)经过\emph{一次}初等变换所得矩阵称为\(n\)阶\DefineConcept{初等矩阵}(elementary matrix).
\end{definition}

对应于矩阵的三类初等变换,有三种类型的初等矩阵:
\begin{enumerate}
	\item 互换\(\E\)的\(i\),\(j\)两行(列)所得的矩阵\[
		\P(i,j) = \begin{bmatrix}
			\E_{i-1} & & & \\
			& 0 & & 1 & \\
			& & \E_{j-i-1} & & \\
			& 1 & & 0 & \\
			& & & & \E_{n-j}
		\end{bmatrix}_n;
	\]
	\item 用非零数\(c\)乘以\(\E\)的第\(i\)行(列)所得的矩阵\[
		\P(i(c)) = \begin{bmatrix}
			\E_{i-1} & & \\
			& c & \\
			& & \E_{n-i}
		\end{bmatrix}_n;
	\]
	\item 把\(\E\)的第\(j\)行(第\(i\)列)的\(k\)倍加到第\(i\)行(第\(j\)列)所得的矩阵\[
		\P(i,j(k)) = \begin{bmatrix}
			\E_{i-1} & & & \\
			& 1 & & k & \\
			& & \E_{j-i-1} & & \\
			& 0 & & 1 & \\
			& & & & \E_{n-j}
		\end{bmatrix}_n.
	\]
\end{enumerate}

\subsection{初等矩阵的性质}
\begin{property}\label{theorem:逆矩阵.初等矩阵的性质1}
初等矩阵具有以下性质:
\begin{enumerate}
	\item \(\abs{\P(i,j)}=-1\);
	\item \(\abs{\P(i(c))}=c\);
	\item \(\abs{\P(i,j(k))}=1\);
	\item \(\P(i,j)^T=\P(i,j)\);
	\item \(\P(i(c))^T=\P(i(c))\);
	\item \(\P(i,j(k))^T=\P(j,i(k))\);
	\item \(\P(i,j)^{-1}=\P(i,j)\);
	\item \(\P(i(c))^{-1}=\P(i(c^{-1}))\);
	\item \(\P(i,j(k))^{-1}=\P(i,j(-k))\).
\end{enumerate}
\end{property}

\begin{property}\label{theorem:逆矩阵.初等矩阵的性质2}
对\(n \times t\)矩阵\(\A\)施行一次初等行变换,相当于用一个相应的\(n\)阶初等矩阵左乘\(\A\);
对\(\A\)施行一次初等列变换,相当于用一个相应的\(t\)阶初等矩阵右乘\(\A\).
\begin{proof}
用\(n\)阶矩阵\(\P(i,j)\)左乘\(\A\),将矩阵\(\A\)作相应分块,有\[
	\P(i,j) \A = \begin{bmatrix}
		\E_{i-1} \\
		& 0 & & 1 \\
		& & \E_{j-i-1} \\
		& 1 & & 0 \\
		& & & & \E_{n-j}
	\end{bmatrix}
	\begin{bmatrix}
		\A_1 \\ \a_i \\ \A_2 \\ \a_j \\ \A_3
	\end{bmatrix}
	= \begin{bmatrix}
		\A_1 \\ \a_j \\ \A_2 \\ \a_i \\ \A_3
	\end{bmatrix},
\]
即\(\A\)交换\(i\)、\(j\)两行.

用\(n\)阶矩阵\(\P(i(c))\)左乘\(\A\),将矩阵\(\A\)作相应分块,有\[
	\P(i(c)) \A = \begin{bmatrix}
		\E_{i-1} \\
		& c \\
		& & \E_{n-i}
	\end{bmatrix}
	\begin{bmatrix}
		\A_1 \\ \a_i \\ \A_2
	\end{bmatrix}
	= \begin{bmatrix}
		\A_1 \\ c \a_i \\ a_2
	\end{bmatrix},
\]
即用一非零数\(c\)乘以第\(i\)行.

用\(n\)阶矩阵\(\P(i,j(k))\ (i < j)\)左乘\(\A\),将矩阵\(\A\)作相应分块,有\[
	\P(i,j(k)) \A = \begin{bmatrix}
		\E_{i-1} \\
		& 1 & & k \\
		& & \E_{j-i-1} \\
		& 0 & & 1 \\
		& & & & \E_{n-j}
	\end{bmatrix}
	\begin{bmatrix}
		\A_1 \\ \a_i \\ \A_2 \\ \a_j \\ \A_3
	\end{bmatrix}
	= \begin{bmatrix}
		\A_1 \\ \a_i + k \a_j \\ \A_2 \\ \a_j \\ \A_3
	\end{bmatrix},
\]
即把\(\A\)第\(j\)行的\(k\)倍加到第\(i\)行.
\end{proof}
\end{property}

\section{广义初等变换}
\subsection{广义初等变换的概念}
\begin{definition}
\def\originalmatrix{%
	\begin{bmatrix}%
	\A & \B \\
	\C & \D
	\end{bmatrix}%
}%
分块矩阵有以下三种\DefineConcept{广义初等行变换}:
\def\originalmatrixTail{%
	\originalmatrix \begin{matrix} m \\ n \end{matrix}%
}%
\begin{enumerate}
\item 交换两行,\[
\originalmatrixTail \to \begin{bmatrix}
\C & \D \\
\A & \B
\end{bmatrix} = {\color{red} \begin{bmatrix}
\z & \E_n \\
\E_m & \z
\end{bmatrix}} \originalmatrix
\]

\item 用一个可逆矩阵\(\P_m\)左乘某一行,\[
\originalmatrixTail \to \begin{bmatrix}
\P\A & \P\B \\
\C & \D
\end{bmatrix} = {\color{red} \begin{bmatrix}
\P & \z \\
\z & \E_n
\end{bmatrix}} \originalmatrix
\]

\item 用一个矩阵\(\Q_{n \times m}\)左乘某一行后加到另一行,\[
\originalmatrixTail \to \begin{bmatrix}
\A & \B \\
\C+\Q\A & \D+\Q\B
\end{bmatrix} = {\color{red} \begin{bmatrix}
\E_m & \z \\
\Q & \E_n
\end{bmatrix}} \originalmatrix
\]
\end{enumerate}

类似地,有\DefineConcept{广义初等列变换}:
\def\originalmatrixHead{%
	\overset{\begin{matrix} s & t \end{matrix}}{ \originalmatrix }%
}%
\begin{enumerate}
\item 交换两列,\[
\originalmatrixHead \to \begin{bmatrix}
\B & \A \\
\D & \C
\end{bmatrix} = \originalmatrix {\color{red} \begin{bmatrix}
\z & \E_s \\
\E_t & \z
\end{bmatrix}}
\]

\item 用一个可逆矩阵\(\P_t\)右乘某一列,\[
\originalmatrixHead \to \begin{bmatrix}
\A & \B \P \\
\C & \D \P
\end{bmatrix} = \originalmatrix {\color{red} \begin{bmatrix}
\E_s & \z \\
\z & \P
\end{bmatrix}}
\]

\item 用一个矩阵\(\Q_{t \times s}\)右乘某一列后加到另一列,\[
\originalmatrixHead \to \begin{bmatrix}
\A + \B \Q & \B \\
\C + \D \Q & \D
\end{bmatrix} = \originalmatrix {\color{red} \begin{bmatrix}
\E_s & \z \\
\Q & \E_t
\end{bmatrix}}
\]
\end{enumerate}

广义初等行变换与广义初等列变换统称为\DefineConcept{广义初等变换}.
\end{definition}

类比于初等矩阵,我们定义分块初等矩阵如下:
\begin{definition}
将\(n\)阶单位矩阵\(\E\)分为\(m\)块后,进行\emph{一次}广义初等变换所得的矩阵称为\DefineConcept{分块初等矩阵}.
\end{definition}

\begin{property}
分块初等矩阵都是可逆矩阵.
\end{property}

\subsection{舒尔定理}
\begin{theorem}[舒尔定理]\label{theorem:逆矩阵.舒尔定理}
设\(\A\)是\(m\)阶可逆矩阵,
\(\B,\C,\D\)分别是\(m \times p, n \times m, n \times p\)矩阵,
则有\begin{gather}
	\begin{bmatrix}
		\E_m & \z \\
		-\C\A^{-1} & \E_n
	\end{bmatrix}
	\begin{bmatrix}
		\A & \B \\
		\C & \D
	\end{bmatrix}
	= \begin{bmatrix}
	\A & \B \\
	\z & \D - \C \A^{-1} \B
	\end{bmatrix},
	\\
	\begin{bmatrix}
		\A & \B \\
		\C & \D
	\end{bmatrix}
	\begin{bmatrix}
		\E_m & -\A^{-1} \B \\
		\z & \E_p
	\end{bmatrix}
	= \begin{bmatrix}
		\A & \z \\
		\C & \D - \C \A^{-1} \B
	\end{bmatrix},
	\\
	\begin{bmatrix}
		\E_m & \z \\
		-\C \A^{-1} & \E_n
	\end{bmatrix}
	\begin{bmatrix}
		\A & \B \\
		\C & \D
	\end{bmatrix}
	\begin{bmatrix}
		\E_m & -\A^{-1} \B \\
		\z & \E_p
	\end{bmatrix}
	= \begin{bmatrix}
		\A & \z \\
		\z & \D - \C \A^{-1} \B
	\end{bmatrix},
\end{gather}
\rm
其中\(\D - \C \A^{-1} \B\)%
称为矩阵\(\begin{bmatrix} \A & \B \\ \C & \D \end{bmatrix}\)%
关于\(\A\)的\DefineConcept{舒尔补}(Schur complement).
\end{theorem}
可以看到只要\(\A\)可逆,就能通过初等分块矩阵直接对分块矩阵进行分块对角化,而在操作过程中就把一些原本不为0的分块矩阵变成了零矩阵,这个过程可以形象地称为“矩阵打洞”,即让矩阵出现尽可能多的0.

利用\hyperref[theorem:逆矩阵.舒尔定理]{舒尔定理}可以证明下述行列式降阶定理:
\begin{theorem}[行列式第一降阶定理]\label{theorem:逆矩阵.行列式第一降阶定理}
\def\M{\mat{M}}
设\(\M = \begin{bmatrix}
\A & \B \\
\C & \D
\end{bmatrix}\)是方阵.\begin{enumerate}
\item 若\(\A\)可逆,则\[
\abs{\M} = \abs{\A} \abs{\D - \C \A^{-1} \B};
\]

\item 若\(\D\)可逆,则\[
\abs{\M} = \abs{\D} \abs{\A - \B \D^{-1} \C};
\]

\item 若\(\A,\D\)均可逆,则\[
\abs{\A} \abs{\D - \C \A^{-1} \B}
= \abs{\D} \abs{\A - \B \D^{-1} \C}.
\]
\end{enumerate}
\end{theorem}

\begin{example}\label{example:逆矩阵.行列式降阶定理的重要应用1}
设\(\A \in M_{s \times n}(K),
\B \in M_{n \times s}(K)\).
证明:\[
	\begin{vmatrix}
		\E_n & \B \\
		\A & \E_s
	\end{vmatrix} = \abs{\E_s - \A\B}.
\]
\begin{proof}
由于\(\E_n,\E_s\)都是单位矩阵,必可逆,
那么根据\cref{theorem:逆矩阵.行列式第一降阶定理},有\[
	\begin{vmatrix}
		\E_n & \B \\
		\A & \E_s
	\end{vmatrix} = \abs{\E_n} \abs{\E_s - \A(\E_n)^{-1}\B}
	= \abs{\E_s - \A\B}.
	\qedhere
\]
\end{proof}
\end{example}

\begin{example}
\def\M{\mat{M}}
求解行列式\(\det \M\),其中\[
\M = \begin{bmatrix}
1+a_1 b_1 & a_1 b_2 & \dots & a_1 b_n \\
a_2 b_1 & 1+a_2 b_2 & \dots & a_2 b_n \\
\vdots & \vdots & & \vdots \\
a_n b_1 & a_n b_2 & \dots & 1+a_n b_n
\end{bmatrix}.
\]
\begin{solution}
记\(\a = (\AutoTuple{a}{n})^T, \b = (\AutoTuple{b}{n})\),则\(\M = \E_n + \a\b\).

注意到\(\M = \E_n + \a\b\)形似某个矩阵的舒尔补,因此考虑下面的矩阵:\[
\A = \begin{bmatrix}
1 & -\b \\
\a & \E_n
\end{bmatrix}.
\]由于\[
\begin{bmatrix}
1 & \z \\
-\a & \E_n
\end{bmatrix} \A
= \begin{bmatrix}
1 & \z \\
-\a & \E_n
\end{bmatrix} \begin{bmatrix}
1 & -\b \\
\a & \E_n
\end{bmatrix} = \begin{bmatrix}
1 & -\b \\
\z & \M
\end{bmatrix},
\]且\[
\begin{bmatrix}
1 & \b \\
\z & \E_n
\end{bmatrix} \A
= \begin{bmatrix}
1 & \b \\
\z & \E_n
\end{bmatrix} \begin{bmatrix}
1 & -\b \\
\a & \E_n
\end{bmatrix} = \begin{bmatrix}
1+\a\b & \z \\
\a & \E_n
\end{bmatrix},
\]所以\[
\abs{\A} = \abs{\M} = \abs{1+\a\b}
= 1+\a\b = 1 + \sum\limits_{k=1}^n a_k b_k.
\]
\end{solution}
\end{example}


\chapter{行列式}
\section{排列与逆序数}
\begin{definition}
\(n\)个不同的自然数按一定顺序排列组成的一个有序数组\[
	(\AutoTuple{k}{n})
\]称为一个\(n\)阶\DefineConcept{排列}.

当\(1 \leq i<j \leq n\)时,
如果\(k_i>k_j\),则称“\(k_i,k_j\)构成一个\DefineConcept{逆序}”;
如果\(k_i<k_j\),则称“\(k_i,k_j\)构成一个\DefineConcept{顺序}”.
此排列中的逆序的总数叫它的\DefineConcept{逆序数},记作\(\tau(\AutoTuple{k}{n})\).

逆序数为偶数的排列叫\DefineConcept{偶排列},逆序数为奇数的排列叫\DefineConcept{奇排列}.
\end{definition}

计算一个排列的逆序数时,排列中的逆序不能重复计算,也不能漏掉.
可按公式\[
	\tau(\AutoTuple{k}{n})=m_1+m_2+\dotsb+m_n
\]计算,其中\(m_i\)为排列中排在\(k_i\)后面比它小的数的个数.
\begin{example}
在4阶排列2341中,2与3形成的数对\((2,3)\),小的数在前,大的数在后,这一对数构成一个顺序;
而2与1形成的数对\((2,1)\),大的数在前,小的数在后,这一对数构成一个逆序.

\begin{figure}[ht]
	\centering
	\begin{tikzpicture}
		\foreach \j in {0,...,3} {
			\fill[ballblue](\j cm+1pt,1pt)rectangle(\j cm+1cm-1pt,1cm-1pt);
		}
		\foreach \i in {0,...,3} {
			\fill[orangepeel](-1pt,-\i cm-1pt)rectangle(-1cm+1pt,-\i cm-1cm+1pt);
		}
		\tiny
		\draw[black](-.5,5pt)node{\(k_i\)}
				(-5pt,.5)node{\(k_j\)};
		\normalsize
		\draw(-.5,-.5)node{2}
			(-.5,-1.5)node{3}
			(-.5,-2.5)node{4}
			(-.5,-3.5)node{1};
		\draw(.5,.5)node{2}
			(1.5,.5)node{3}
			(2.5,.5)node{4}
			(3.5,.5)node{1};
		\draw[applegreen](1.5,-.5)node{(2,3)}
			(2.5,-.5)node{(2,4)}
			(2.5,-1.5)node{(3,4)};
		\draw[tangelo](3.5,-.5)node{(2,1)}
			(3.5,-1.5)node{(3,1)}
			(3.5,-2.5)node{(4,1)};
		\draw[black!30,dashed](0,0)--(4,-4);
	\end{tikzpicture}
	\caption{}
	\label{figure:行列式.4阶排列2341的所有数对}
\end{figure}

如\cref{figure:行列式.4阶排列2341的所有数对},
构成逆序的数对有\((2,1),(3,1),(4,1)\),
构成顺序的数对有\((2,3),(2,4),(3,4)\),
这个4阶排列的逆序数是3,即\(\tau(2341)=3\),
它是一个奇排列.
\end{example}

\begin{property}
\(n\)阶排列共有\(n!\)个.
\end{property}

\begin{definition}
排列\(1,2,\dotsc,n\)由小到大按自然顺序排列,叫做\(n\)阶\DefineConcept{自然排列}.
\end{definition}

\begin{property}
自然排列中没有逆序,即\begin{equation}
	\tau(1,2,\dotsc,n)=0.
\end{equation}
\end{property}

\begin{example}
证明:\begin{equation}
	\tau(n,n-1,\dotsc,1)=\frac{n(n-1)}{2}.
\end{equation}
\begin{proof}
由于对于排列中的每一个数来说,其后的所有数都比它小,所以\[
	\tau(n,n-1,\dotsc,1)
	= (n-1) + (n-2) + \dotsb + 1 + 0
	= \frac{(n-1)n}{2}.
	\qedhere
\]
\end{proof}
\end{example}

\begin{definition}
把排列中的两个数的位置互换,其余数字不动,得到另一个排列;
像这样的变换称为\DefineConcept{对换}.
\end{definition}

\begin{theorem}
排列经一次对换后奇偶性改变.
\begin{proof}
我们首先讨论对换的两个数在\(n\)阶排列中相邻的情形:
排列\(k_1,\dotsc,k_i,k_{i+1},\dotsc,k_n\)
对换\(k_i\)与\(k_{i+1}\)这两个数会得到\(k_1,\dotsc,k_{i+1},k_i,\dotsc,k_n\);
除\(k_i,k_{i+1}\)以外的数构成的数对是顺序还是逆序,
在变换前与变换后是一样的;
\(k_i\)和\(k_{i+1}\)以外的数与\(k_i\)或\(k_{i+1}\)构成的数对是顺序还是逆序,
在变换前后也是一样的.
只有数对\((k_i,k_{i+1})\),
如果它在变换前是顺序,那么它在变换后是逆序,
这时变换后排列的逆序数比变换前排列的逆序数多1,即\[
	\tau(k_1,\dotsc,k_{i+1},k_i,\dotsc,k_n)
	= \tau(k_1,\dotsc,k_i,k_{i+1},\dotsc,k_n) + 1;
\]
如果它在变换前是逆序,那么它在变换后是顺序,
这时变换后排列的逆序数比变换前排列的逆序数少1,即\[
	\tau(k_1,\dotsc,k_{i+1},k_i,\dotsc,k_n)
	= \tau(k_1,\dotsc,k_i,k_{i+1},\dotsc,k_n) - 1.
\]
因此,在对换的两个数在\(n\)阶排列中相邻的情形下,变换前后排列的奇偶性相反.

再讨论一般情形:
排列\(k_1,\dotsc,k_{i-1},k_i,k_{i+1},\dotsc,k_{j-1},k_j,k_{j+1},\dotsc,k_n\)
对换\(k_i,k_j\)这两个数会得到
\(k_1,\dotsc,k_{i-1},k_j,k_{i+1},\dotsc,k_{j-1},k_i,k_{j+1},\dotsc,k_n\);
由于这次对换可以视作一系列相邻两数的对换,
即“对换\(k_i\)与\(k_{i+1}\)”“对换\(k_{i+1}\)与\(k_{i+2}\)”%
......%
“对换\(k_{j-2}\)与\(k_{j-1}\)”“对换\(k_{j-1}\)与\(k_j\)”,
而这就是作了\(s+1+s=2s+1\)次相邻两数的对换,
像这样奇数次相邻两数的对换回改变排列的奇偶性,也就是说,变换前后排列的奇偶性相反.
\end{proof}
\end{theorem}

\begin{corollary}
排列经奇数次对换后奇偶性改变,经偶数次对换后奇偶性不变.
\end{corollary}
我们可以把这个推论表述为如下形式:
设对换次数为\(s\),变换前后的排列分别为\[
	\AutoTuple{\mu}{n}
	\quad\text{和}\quad
	\AutoTuple{\nu}{n},
\]
则\[
	(-1)^{\tau(\AutoTuple{\nu}{n})} = (-1)^s (-1)^{\tau(\AutoTuple{\mu}{n})}.
\]

\begin{theorem}\label{theorem:行列式.任意排列可化为自然序}
任意一个\(n\)阶排列\(\AutoTuple{k}{n}\)都可经一系列对换变成自然顺序排列,
且对换的次数\(s\)与\(\tau(\AutoTuple{k}{n})\)同奇偶,即\[
	(-1)^s = (-1)^{\tau(\AutoTuple{k}{n})}.
\]
\begin{proof}
设\(n\)阶排列\(\AutoTuple{k}{n}\)经过\(s\)次对换变成\(1,2,\dotsc,n\).
考虑到\(1,2,\dotsc,n\)是偶排列,
因此,如果\(\AutoTuple{k}{n}\)是奇排列,则\(s\)必为奇数,才能把奇排列变成偶排列;
如果\(\AutoTuple{k}{n}\)是偶排列,则\(s\)必为偶数,才能保持排列的奇偶性不变.
显然,如果\(n\)阶排列\(\AutoTuple{k}{n}\)经过\(s\)次对换变成自然排列\(1,2,\dotsc,n\),
那么\(1,2,\dotsc,n\)经过上述\(n\)次对换(次序相反)就变成排列\(\AutoTuple{k}{n}\).
\end{proof}
\end{theorem}

\begin{example}
证明:在全部\(n\)阶排列中,奇偶排列各占一半.
%TODO
\end{example}

\begin{example}
试证:\(\tau(\AutoTuple{i}{n})+\tau(i_n,i_{n-1},\dotsc,i_1)=\frac{n(n-1)}{2}\).
\begin{proof}
记\(I=\Set{i_1,i_2,\dotsc,i_{n-1},i_n}\).
对于\(\forall p,q \in I\),
根据排列的定义必有\(p \neq q\),即有\(p<q\)或\(p>q\)成立.
因此,对于数对\((i_k,i_{k+1})\)和\((i_{k+1},i_k)\),
有且仅有以下两种情况:\begin{itemize}
	\item \(i_k,i_{k+1}\)构成一个顺序,\(i_{k+1},i_k\)构成一个逆序;
	\item \(i_k,i_{k+1}\)构成一个逆序,\(i_{k+1},i_k\)构成一个顺序.
\end{itemize}
不论是哪种情况,都有\(\tau(i_k,i_{k+1})+\tau(i_{k+1},i_k)=1\).

由上可知,排列\((i_1,i_2,\dotsc,i_{n-1},i_n)\)
与\((i_n,i_{n-1},\dotsc,i_2,i_1)\)的逆序数之和
相当于是从集合\(I\)中任取两个数构成一个逆序的取法,
那么\[
	\tau(i_1,i_2,\dotsc,i_{n-1},i_n)+\tau(i_n,i_{n-1},\dotsc,i_2,i_1)
	= C_n^2
	= \frac{n(n-1)}{2}.
	\qedhere
\]
\end{proof}
\end{example}

\section{行列式}
\subsection{行列式的概念}
\begin{definition}
设
\[
\A = \begin{bmatrix}
a_{11} & a_{12} & \dots & a_{1n} \\
a_{21} & a_{22} & \dots & a_{2n} \\
\vdots & \vdots & \ddots & \vdots \\
a_{n1} & a_{n2} & \dots & a_{nn}
\end{bmatrix}
\]
是数域\(P\)上的一个\(n\)阶方阵.
从矩阵\(\A\)中取出不同行又不同列的\(n\)个元素作乘积
\begin{equation}\label{equation:行列式.行列式的项1}
	(-1)^{\tau(\AutoTuple{j}{n})}
	a_{1 j_1} a_{2 j_2} \dotsm a_{n j_n},
\end{equation}
构成一项;%
我们可以像这样构造\(n!\)项,
并且称这\(n!\)项之和\[
\sum\limits_{\AutoTuple{j}{n}}
(-1)^{\tau(\AutoTuple{j}{n})}
a_{1 j_1} a_{2 j_2} \dotsm a_{n j_n}
\]为“矩阵\(\A\)的行列式(determinant)”,
记作\[
\begin{vmatrix}
a_{11} & a_{12} & \dots & a_{1n} \\
a_{21} & a_{22} & \dots & a_{2n} \\
\vdots & \vdots & & \vdots \\
a_{n1} & a_{n2} & \dots & a_{nn}
\end{vmatrix},
\]或\(\det\A\),或\(\abs{\A}\);
即
\begin{equation}\label{equation:行列式.行列式的定义式}
\begin{vmatrix}
a_{11} & a_{12} & \dots & a_{1n} \\
a_{21} & a_{22} & \dots & a_{2n} \\
\vdots & \vdots & & \vdots \\
a_{n1} & a_{n2} & \dots & a_{nn}
\end{vmatrix}
\defeq
\sum\limits_{\AutoTuple{j}{n}}
(-1)^{\tau(\AutoTuple{j}{n})}
a_{1 j_1} a_{2 j_2} \dotsm a_{n j_n}.
\end{equation}
这里,求和指标\(\AutoTuple{j}{n}\)表示遍取所有\(n\)阶排列.

我们称\cref{equation:行列式.行列式的定义式}
为“行列式\(\abs{A}\)的\DefineConcept{完全展开式}”.
\end{definition}

特别地,
一阶行列式为
\begin{equation}
	\begin{vmatrix} a \end{vmatrix} = a.
\end{equation}

二阶行列式为
\begin{equation}
	\begin{vmatrix}
		a_{11} & a_{12} \\
		a_{21} & a_{22}
	\end{vmatrix}
	= a_{11} a_{22} - a_{12} a_{21}.
\end{equation}

三阶行列式为
\begin{equation}
	\begin{vmatrix}
		a_{11} & a_{12} & a_{13} \\
		a_{21} & a_{22} & a_{23} \\
		a_{31} & a_{32} & a_{33}
	\end{vmatrix}
	= \begin{array}[t]{l}
		(a_{11} a_{22} a_{33} + a_{12} a_{23} a_{31} + a_{13} a_{21} a_{32} \\
		\hspace{20pt}
		- a_{13} a_{22} a_{31} - a_{12} a_{21} a_{33} - a_{11} a_{23} a_{32})
	\end{array}.
\end{equation}

我们还可以用数学归纳法证明以下两条公式:
\begin{gather}
	\begin{vmatrix}
		a_{11} & a_{12} & \dots & a_{1n} \\
		& a_{22} & \dots & a_{2n} \\
		& & \ddots & \vdots \\
		& & & a_{nn}
	\end{vmatrix}
	= a_{11} a_{22} \dotsm a_{nn}, \\%
	\begin{vmatrix}
		& & & & a_{1n} \\
		& & & a_{2,n-1} & a_{2n} \\
		& & & \vdots & \vdots \\
		& a_{n-1,2} & \dots & a_{n-1,n-1} & a_{n-1,n} \\
		a_{n1} & a_{n2} & \dots & a_{n,n-1} & a_{nn}
	\end{vmatrix}
	=(-1)^{\frac{1}{2}n(n-1)} a_{1n} a_{2,n-1} \dotsm a_{n-1,2} a_{n1}.
\end{gather}

\begin{lemma}
设\(\A=(a_{ij})_n\),而\(\AutoTuple{i}{n}\)与\(\AutoTuple{j}{n}\)是两个\(n\)阶排列,则
\begin{equation}\label{equation:行列式.行列式的项2}
	(-1)^{\tau(\AutoTuple{i}{n})+\tau(\AutoTuple{j}{n})}
	a_{i_1j_1} a_{i_2j_2} \dotsm a_{i_nj_n}
\end{equation}
是\(\abs{\A}\)的项.
\begin{proof}
由乘法交换律,\cref{equation:行列式.行列式的项2} 可以经过\(s\)次互换两个因子的次序写成\[
(-1)^{\tau(\AutoTuple{i}{n})+\tau(\AutoTuple{j}{n})}
	a_{1 l_1} a_{2 l_2} \dotsm a_{n l_n},
\]其中,\(\AutoTuple{l}{n}\)是一个\(n\)阶排列.

同时,行标排列\(\AutoTuple{i}{n}\)与列标排列\(\AutoTuple{j}{n}\)
分别经过\(s\)次对换变到\(1,2,\dotsc,n\)与\(\AutoTuple{l}{n}\),
而它们的奇偶性都分别改变了\(s\)次,总共改变了\(2s\)次(偶数次),故\[
	(-1)^{\tau(\AutoTuple{i}{n})+\tau(\AutoTuple{j}{n})}
	= (-1)^{\tau(1,2,\dotsc,n)+\tau(\AutoTuple{l}{n})}
	= (-1)^{\tau(\AutoTuple{l}{n})},
\]这说明\cref{equation:行列式.行列式的项2} 是行列式\(\abs{\A}\)的项.
\end{proof}
\end{lemma}

\begin{corollary}
给定行指标的一个排列\(\AutoTuple{i}{n}\),则\(n\)阶矩阵\(\A\)的行列式为
\begin{equation}\label{equation:行列式.给定行指标排列下的行列式的完全展开式}
\abs{\A}
= \sum\limits_{\AutoTuple{j}{n}}
(-1)^{\tau(\AutoTuple{i}{n})+\tau(\AutoTuple{j}{n})}
a_{i_1 j_1} a_{i_2 j_2} \dotsm a_{i_n j_n};
\end{equation}
或者给定列指标的一个排列\(\AutoTuple{j}{n}\),则\(n\)阶矩阵\(\A\)的行列式为
\begin{equation}\label{equation:行列式.给定列指标排列下的行列式的完全展开式}
	\abs{\A}
	= \sum\limits_{\AutoTuple{i}{n}}
	(-1)^{\tau(\AutoTuple{i}{n})+\tau(\AutoTuple{j}{n})}
	a_{i_1 j_1} a_{i_2 j_2} \dotsm a_{i_n j_n}.
\end{equation}

特别地,\(n\)阶行列式\(\abs{\A}\)的每一项可以按列指标成自然序排好位置,
这时用行指标所成排列的奇偶性来决定该项前面所带的符号,即
\begin{equation}\label{equation:行列式.给定列指标为自然序下行列式的完全展开式}
	\abs{\A} =
	\sum\limits_{\AutoTuple{i}{n}}
	(-1)^{\tau(\AutoTuple{i}{n})}
	a_{i_1 1} a_{i_2 2} \dotsm a_{i_n n}.
\end{equation}
\end{corollary}
\cref{equation:行列式.行列式的定义式} 和\cref{equation:行列式.给定列指标为自然序下行列式的完全展开式}
表明,行列式中行与列的地位是平等的,即\(\abs{\A} = \abs{\A^T}\).

\begin{example}
若\(n\)阶行列式\(\det\A\)中为零的元多于\(n^2-n\)个,证明:\(\det\A=0\).
% TODO
\end{example}

\begin{example}
证明:如果\(n\ (n\geq2)\)阶矩阵\(\A\)的元素为\(1\)或\(-1\),则\(\abs{\A}\)必为偶数.
% TODO
\end{example}

\subsection{行列式的性质}
\begin{property}\label{theorem:行列式.性质1}
设\(\A=(a_{ij})_n\),则\(\det \A = \det \A^T\).
\end{property}
这就说明,行列互换,行列式的值不变.

\begin{property}\label{theorem:行列式.性质2}
\(\det(\a_1,\dotsc,k\a_j,\dotsc,\a_n)
=k \cdot \det(\a_1,\dotsc,\a_j,\dotsc,\a_n)\).
\end{property}
这就说明,行列式某列(行)各元素的公因子可以提到行列式外.

\begin{corollary}\label{theorem:行列式.性质2.推论1}
\(\det(\a_1,\dotsc,\z,\dotsc,\a_n) = 0\).
\end{corollary}
也就是说,如果行列式中某一列(行)元素全为零,则行列式等于零.

\begin{corollary}\label{theorem:行列式.性质2.推论2}
\(\det(k\A) = k^n \det \A\).
\end{corollary}

应该注意到,一般说来,\(\det(k\A) \neq k \det\A\).

\begin{property}\label{theorem:行列式.性质3}
\(\det(\a_1,\dotsc,\b + \g,\dotsc,\a_n)
= \det(\a_1,\dotsc,\b,\dotsc,\a_n)
+ \det(\a_1,\dotsc,\g,\dotsc,\a_n)\).
\begin{proof}
直接计算得
\begin{align*}
	\det(\a_1,\dotsc,\b + \g,\dotsc,\a_n)
	&= \sum\limits_{\AutoTuple{i}{n}}{
	(-1)^{\tau(\AutoTuple{i}{n})}
	a_{i_1 1} \dotsm (b_{i_j} + c_{i_j}) \dotsm a_{i_n n}
	} \\
	&= \sum\limits_{\AutoTuple{i}{n}}{
	(-1)^{\tau(\AutoTuple{i}{n})}
	a_{i_1 1} \dotsm b_{i_j} \dotsm a_{i_n n}
	} \\
	&\qquad+ \sum\limits_{\AutoTuple{i}{n}}{
	(-1)^{\tau(\AutoTuple{i}{n})}
	a_{i_1 1} \dotsm c_{i_j} \dotsm a_{i_n n}
	} \\
	&= \det(\a_1,\dotsc,\b,\dotsc,\a_n)
	+ \det(\a_1,\dotsc,\g,\dotsc,\a_n).
	\qedhere
\end{align*}
\end{proof}
\end{property}
注:一般地,\(\det(\a_1+\b_1,\a_2+\b_2,\dotsc,\a_n+\b_n)\)可以拆成\(2^n\)个行列式之和.

\begin{property}\label{theorem:行列式.性质4}
\(\det(\a_1,\dotsc,\a_s,\dotsc,\a_t,\dotsc,\a_n)
=-\det(\a_1,\dotsc,\a_t,\dotsc,\a_s,\dotsc,\a_n)\).
\end{property}
也就是说,交换两列(行),行列式变号.

\begin{property}\label{theorem:行列式.性质5}
\(\det(\a_1,\dotsc,k\a,\dotsc,l\a,\dotsc,\a_n) = 0\).
\end{property}
这就说明,行列式中若有两列(行)成比例,则行列式等于零.

\begin{property}\label{theorem:行列式.性质6}
\(\det(\a_1,\dotsc,\a_s,\dotsc,\a_t,\dotsc,\a_n)
= \det(\a_1,\dotsc,\a_s,\dotsc,\a_t + k\a_s,\dotsc,\a_n)\).
\end{property}
这说明,将一列的\(k\)倍加到另一列,行列式的值不变.

\begin{example}
设\(\A\)为奇数阶反对称矩阵,即\(\A^T = -\A\),则\(\abs{\A}=0\).
\begin{proof}
设\(\A = (a_{ij})_n\).因为\(\A^T = -\A\),根据行列式的性质,有\[
\abs{\A} = \abs{\A^T} = \abs{-\A} = (-1)^n \abs{\A} = -\abs{\A},
\]于是\(\abs{\A} = 0\).
\end{proof}
\end{example}

\begin{example}
计算\(n\)阶行列式\[
D_n = \begin{vmatrix}
	k & \lambda & \lambda & \dots & \lambda \\
	\lambda & k & \lambda & \dots & \lambda \\
	\lambda & \lambda & k & \dots & \lambda \\
	\vdots & \vdots & \vdots & & \vdots \\
	\lambda & \lambda & \lambda & \dots & k
\end{vmatrix},
\quad k\neq\lambda.
\]
\begin{solution}
当\(n>1\)时,有\begin{align*}
	D_n &= \begin{vmatrix}
		k+(n-1)\lambda & \lambda & \lambda & \dots & \lambda \\
		k+(n-1)\lambda & k & \lambda & \dots & \lambda \\
		k+(n-1)\lambda & \lambda & k & \dots & \lambda \\
		\vdots & \vdots & \vdots & & \vdots \\
		k+(n-1)\lambda & \lambda & \lambda & \dots & k
	\end{vmatrix} \\
	&= [k+(n-1)\lambda] \begin{vmatrix}
		1 & \lambda & \lambda & \dots & \lambda \\
		1 & k & \lambda & \dots & \lambda \\
		1 & \lambda & k & \dots & \lambda \\
		\vdots & \vdots & \vdots & & \vdots \\
		1 & \lambda & \lambda & \dots & k
	\end{vmatrix} \\
	&= [k+(n-1)\lambda] \begin{vmatrix}
		1 & \lambda & \lambda & \dots & \lambda \\
		0 & k-\lambda & 0 & \dots & 0 \\
		0 & 0 & k-\lambda & \dots & 0 \\
		\vdots & \vdots & \vdots & & \vdots \\
		0 & 0 & 0 & \dots & k-\lambda
	\end{vmatrix} \\
	&= [k+(n-1)\lambda] (k-\lambda)^{n-1}.
	\tag1
\end{align*}

当\(n=1\)时,\(D_1 = k\)符合(1)式.
\end{solution}
\end{example}

\begin{example}\label{example:行列式.两个向量的乘积矩阵的行列式}
设\(\a=(\AutoTuple{a}{n})^T,\b=(\AutoTuple{b}{n})^T\)是\(n\)维列向量.
求:\(\abs{\a\b^T}\).
\begin{solution}
根据\cref{theorem:行列式.性质2},
有\begin{align*}
	\abs{\a\b^T} = \begin{vmatrix}
		a_1 b_1 & a_1 b_2 & \dots & a_1 b_n \\
		a_2 b_1 & a_2 b_2 & \dots & a_2 b_n \\
		\vdots & \vdots & & \vdots \\
		a_n b_1 & a_n b_2 & \dots & a_n b_n
	\end{vmatrix}
	= a_1 a_2 \dotsm a_n \cdot \begin{vmatrix}
		b_1 & b_2 & \dots & b_n \\
		b_1 & b_2 & \dots & b_n \\
		\vdots & \vdots & & \vdots \\
		b_1 & b_2 & \dots & b_n
	\end{vmatrix}.
\end{align*}
而\[
\begin{vmatrix}
	b_1 & b_2 & \dots & b_n \\
	b_1 & b_2 & \dots & b_n \\
	\vdots & \vdots & & \vdots \\
	b_1 & b_2 & \dots & b_n
\end{vmatrix}
\]各行成比例,
根据\cref{theorem:行列式.性质5},那么该行列式等于0,可知\(\abs{\a\b^T} = 0\).
\end{solution}
\end{example}

\section{行列式按行(或列)展开及其计算}
\subsection{子式}
\begin{definition}
在矩阵\(\A=(a_{ij})_{s \times n}\)中,
任取\(k\)行\(k\)列,
位于这些行与列交叉处的\(k^2\)个元素,按原顺序排成的\(k\)阶矩阵的行列式\[
	\begin{vmatrix}
		a_{i_1,j_1} & a_{i_1,j_2} & \dots & a_{i_1,j_k} \\
		a_{i_2,j_1} & a_{i_2,j_2} & \dots & a_{i_2,j_k} \\
		\vdots & \vdots & & \vdots \\
		a_{i_k,j_1} & a_{i_k,j_2} & \dots & a_{i_k,j_k}
	\end{vmatrix},
	\quad
	\begin{array}{c}
		1 \leq i_1 < i_2 < \dotsb < i_k \leq s; \\
		1 \leq j_1 < j_2 < \dotsb < j_k \leq n
	\end{array}
\]称为“\(\A\)的一个\(k\)阶\DefineConcept{子式}(minor)”,记作\[
	\MatrixMinor\A{
		\AutoTuple{i}{k} \\
		\AutoTuple{j}{k}
	}.
\]
\end{definition}

\begin{property}
设矩阵\(\A = (a_{ij})_{s \times n}\).
如果存在\(r < \min\{s,n\}\),
使得所有\(r\)阶子式都等于零,
则对任意\(k > r\)有\(\A\)的所有\(k\)阶子式全为零.
\end{property}

\subsection{主子式,顺序主子式}
\begin{definition}
设\(\A=(a_{ij})_n\),
\(k\)阶子式\[
	\MatrixMinor\A{
		i_1,i_2,\dotsc,i_k \\
		i_1,i_2,\dotsc,i_k
	}
\]
称为“\(\A\)的\(k\)阶\DefineConcept{主子式}(principal minor)”.

\(\A\)位于左上角的\(k\)阶主子式\[
	\MatrixMinor\A{
		1,2,\dotsc,k \\
		1,2,\dotsc,k
	}
\]称为“\(\A\)的\(k\)阶\DefineConcept{顺序主子式}(ordinal principal minor)”.
\end{definition}

\subsection{余子式、代数余子式}
\begin{definition}
在\(n\)阶矩阵\(\A=(a_{ij})_n\)中,
称子式\[
	\MatrixMinor\A{
		\AutoTuple{\mu}{n-k} \\
		\AutoTuple{\nu}{n-k}
	},
\]为“子式\(\MatrixMinor\A{
	\AutoTuple{i}{k} \\
	\AutoTuple{j}{k}
}\)的\DefineConcept{余子式}(cofactor)”,
其中\[
	\Set{ \AutoTuple{\mu}{n-k} } = \Set{ 1,2,\dotsc,n } - \Set{ \AutoTuple{i}{k} },
\]\[
	\Set{ \AutoTuple{\nu}{n-k} } = \Set{ 1,2,\dotsc,n } - \Set{ \AutoTuple{j}{k} },
\]
且\(\mu_1<\mu_2<\dotsb<\mu_{n-k},
\nu_1<\nu_2<\dotsb<\nu_{n-k}\).

把\[
	(-1)^{i_1+\dotsb+i_k+j_1+\dotsb+j_k}
	\MatrixMinor\A{
		\AutoTuple{\mu}{n-k} \\
		\AutoTuple{\nu}{n-k}
	}
\]称作“子式\(\MatrixMinor\A{
	\AutoTuple{i}{k} \\
	\AutoTuple{j}{k}
}\)的\DefineConcept{代数余子式}(algebraic cofactor)”.

特别地,称子式\[
	\MatrixMinor\A{
		1,\dotsc,i-1,i+1,\dotsc,n \\
		1,\dotsc,j-1,j+1,\dotsc,n
	}
\]为“元素\(a_{ij}\)的\DefineConcept{余子式}”,记作\(M_{ij}\).
又称\[
(-1)^{i+j} M_{ij}
\]为“\(a_{ij}\)的\DefineConcept{代数余子式}”,记作\(A_{ij}\).
\end{definition}

\subsection{伴随矩阵}
\begin{definition}\label{definition:伴随矩阵.伴随矩阵的定义}
设\(\A=(a_{ij})_n\),
\(A_{ij}\)为元素\(a_{ij}\ (i,j=1,2,\dotsc,n)\)的代数余子式.
以\(A_{ij}\)作为第\(j\)行第\(i\)列元素构成的\(n\)阶矩阵,
称为“\(\A\)的\DefineConcept{伴随矩阵}(adjoint, adjugate matrix)”,
记为\(\A^*\),即\[
	\A^*
	\defeq
	\begin{bmatrix}
		A_{11} & A_{21} & \dots & A_{n1} \\
		A_{12} & A_{22} & \dots & A_{n2} \\
		\vdots & \vdots & & \vdots \\
		A_{1n} & A_{2n} & \dots & A_{nn}
	\end{bmatrix}.
\]
\end{definition}

\begin{example}
设\(\A=(a_{ij})_n\)的伴随矩阵为\(\A^*\),求\((k\A)^*\).
\begin{solution}
设\(\A\)的元素\(a_{ij}\)的代数余子式是\(A_{ij}\),
那么矩阵\(k\A = (b_{ij})_n\)的元素\(b_{ij} = k a_{ij}\)的代数余子式是\[
	B_{ij}
	= (-1)^{i+j}
	\begin{vmatrix}
		k a_{11} & \dots & k a_{1,j-1} & k a_{1,j+1} & \dots & k a_{1n} \\
		\vdots & & \vdots & \vdots & & \vdots \\
		k a_{i-1,1} & \dots & k a_{i-1,j-1} & k a_{i-1,j+1} & \dots & k a_{i-1,n} \\
		k a_{i+1,1} & \dots & k a_{i+1,j-1} & k a_{i+1,j+1} & \dots & k a_{i+1,n} \\
		\vdots & & \vdots & \vdots & & \vdots \\
		k a_{n1} & \dots & k a_{n,j-1} & k a_{n,j+1} & \dots & k a_{nn} \\
	\end{vmatrix}
	= k^{n-1} A_{ij}.
\]
因此,\(k\A\)的伴随矩阵是\((B_{ji})_n
= k^{n-1} (A_{ji})_n
= k^{n-1} \A^*\).
\end{solution}
\end{example}


\subsection{行列式按一行(或一列)展开}
\begin{theorem}
设\(\A=(a_{ij})_n\),
\(A_{ij}\)为\(a_{ij}\ (i,j=1,2,\dotsc,n)\)的代数余子式.
\begin{enumerate}
	\item 行列式等于它的任一行的各元与其代数余子式乘积之和,
	即\begin{equation}
		\abs{\A} = \sum_{j=1}^n a_{ij} A_{ij}
		\quad(i=1,2,\dotsc,n).
	\end{equation}

	\item 行列式的任一行的各元与另一行对应元素的代数余子式乘积之和为零,
	即\begin{equation}
		\sum_{j=1}^n a_{ij} A_{kj} = 0
		\quad(i \neq k;
		i,k=1,2,\dotsc,n).
	\end{equation}
\end{enumerate}
\begin{proof}
注意\[
	\tau(i,1,2,\dotsc,i-1,i+1,\dotsc,n) = i-1,
\]\[
	\tau(j,j_1,j_2,\dotsc,j_{i-1},j_{i+1},\dotsc,j_n)
	= j-1+\tau(j_1,j_2,\dotsc,j_{i-1},j_{i+1},\dotsc,j_n),
\]于是有\begin{align*}
	\abs{\A}
	&= \sum_{j,j_1,j_2,\dotsc,j_{i-1},j_{i+1},\dotsc,j_n}
		(-1)^{\tau(i,1,2,\dotsc,i-1,i+1,\dotsc,n) + \tau(j,j_1,j_2,\dotsc,j_{i-1},j_{i+1},\dotsc,j_n)}
		a_{ij} \prod_{\substack{k=1 \\ k \neq i}}^n a_{k j_k} \\
	&= \sum_{j=1}^n a_{ij} (-1)^{(i-1)+(j-1)}
		\sum_{j_1,j_2,\dotsc,j_{i-1},j_{i+1},\dotsc,j_n}
			(-1)^{\tau(j_1,j_2,\dotsc,j_{i-1},j_{i+1},\dotsc,j_n)}
				\prod_{\substack{k=1 \\ k \neq i}}^n a_{k j_k} \\
	&= \sum_{j=1}^n a_{ij} (-1)^{i+j} M_{ij}
	= \sum_{j=1}^n a_{ij} A_{ij}.
	\qedhere
\end{align*}
\end{proof}
\end{theorem}

由于行列式中行与列的地位平等,因此又可以行列式按某一列展开.
\begin{theorem}
设\(\A=(a_{ij})_n\),\(A_{ij}\)为\(a_{ij}\ (i,j=1,2,\dotsc,n)\)的代数余子式.
\begin{enumerate}
	\item 行列式等于它的任一列的各元与其代数余子式乘积之和,即\begin{equation}
		\abs{\A} = \sum_{i=1}^n a_{ij} A_{ij}
		\quad(j=1,2,\dotsc,n).
	\end{equation}

	\item 行列式的任一列的各元与另一列对应元素的代数余子式乘积之和为零,即\begin{equation}
		\sum_{i=1}^n a_{ij} A_{ik} = 0
		\quad(j \neq k;
		j,k=1,2,\dotsc,n).
	\end{equation}
\end{enumerate}
\end{theorem}

我们可以将上述两个定理中的公式分别改写成以下形式:
\begin{gather}
	\sum_{j=1}^n a_{ij} A_{kj} = \left\{ \begin{array}{cl}
		\abs{\A}, & k = i, \\
		0, & k \neq i;
	\end{array} \right.
	\quad(i=1,2,\dotsc,n) \\%
	\sum_{i=1}^n a_{ij} A_{ik} = \left\{ \begin{array}{cl}
		\abs{\A}, & k = j, \\
		0, & k \neq j.
	\end{array} \right.
	\quad(j=1,2,\dotsc,n)
\end{gather}

\begin{theorem}
设方阵\(\A\)的伴随矩阵为\(\A^*\),
则\begin{gather}
	\A \A^* = \A^* \A = \abs{\A} \E, \label{equation:行列式.伴随矩阵.恒等式1} \\
	(\A^*)^T = (\A^T)^*, \label{equation:行列式.伴随矩阵.恒等式2} \\
	(\A\B)^* = \B^* \A^*. \label{equation:行列式.伴随矩阵.恒等式3}
\end{gather}
\begin{proof}
这里只证\cref{equation:行列式.伴随矩阵.恒等式1}.
设\(\A=(a_{ij})_n\),\(\A^*=(\A_{ji})_n\).
那么\begin{align*}
	\A\A^*
	&= \begin{bmatrix}
		a_{11} & a_{12} & \dots & a_{1n} \\
		a_{21} & a_{22} & \dots & a_{2n} \\
		\vdots & \vdots & & \vdots \\
		a_{n1} & a_{n2} & \dots & a_{nn} \\
	\end{bmatrix}
	\begin{bmatrix}
		A_{11} & A_{21} & \dots & A_{n1} \\
		A_{12} & A_{22} & \dots & A_{n2} \\
		\vdots & \vdots & & \vdots \\
		A_{1n} & A_{2n} & \dots & A_{nn}
	\end{bmatrix} \\
	&= \begin{bmatrix}
		\abs{\A} & 0 & \dots & 0 \\
		0 & \abs{\A} & \dots & 0 \\
		\vdots & \vdots & & \vdots \\
		0 & 0 & \dots & \abs{\A}
	\end{bmatrix}
	= \abs{\A} \E.
\end{align*}
利用对称性,立即可得\(\A^*\A\)也等于\(\abs{\A} \E\).
\end{proof}
\end{theorem}

\begin{example}
%@see: 《线性代数》(张慎语、周厚隆) P34. 例3
试证范德蒙德行列式:
\begin{equation}\label{equation:行列式.范德蒙德行列式}
	V_n = \begin{vmatrix}
		1 & 1 & 1 & \dots & 1 \\
		x_1 & x_2 & x_3 & \dots & x_n \\
		x_1^2 & x_2^2 & x_3^2 & \dots & x_n^2 \\
		\vdots & \vdots & \vdots& & \vdots \\
		x_1^{n-1} & x_2^{n-1} & x_3^{n-1} & \dots & x_n^{n-1}
	\end{vmatrix}
	= \prod_{1 \leq j < i \leq n}(x_i-x_j).
\end{equation}
\begin{proof}
利用数学归纳法.
当\(n=2\)时,\(V_2 = \begin{vmatrix}
	1 & 1 \\ x_1 & x_2
\end{vmatrix} = x_2 - x_1\),结论成立.

假设\(n=k-1\)时,结论成立;
那么当\(n=k\)时,在\(V_k\)中,
依次将第\(k-1\)行的\(-x_1\)倍加到第\(k\)行,
将第\(k-2\)行的\(-x_1\)倍加到第\(k-1\)行,
以此类推,直到最后将第1行的\(-x_1\)倍加到第\(2\)行,得\[
	V_k = \begin{vmatrix}
		1 & 1 & 1 & \dots & 1 \\
		0 & x_2 - x_1 & x_3 - x_1 & \dots & x_k - x_1 \\
		0 & x_2(x_2 - x_1) & x_3(x_3 - x_1) & \dots & x_k(x_k - x_1) \\
		\vdots & \vdots & \vdots & & \vdots \\
		0 & x_2^{k-2}(x_2 - x_1) & x_3^{k-2}(x_3 - x_1) & \dots & x_k^{k-2}(x_k - x_1) \\
	\end{vmatrix};
\]
按第1列展开,得\[
	V_k = 1 \cdot (-1)^{1+1} \cdot \begin{vmatrix}
		x_2 - x_1 & x_3 - x_1 & \dots & x_k - x_1 \\
		x_2(x_2 - x_1) & x_3(x_3 - x_1) & \dots & x_k(x_k - x_1) \\
		\vdots & \vdots & & \vdots \\
		x_2^{k-2}(x_2 - x_1) & x_3^{k-2}(x_3 - x_1) & \dots & x_k^{k-2}(x_k - x_1) \\
	\end{vmatrix};
\]
提取各列公因子,得
\begin{align*}
	V_k &= (x_2 - x_1)(x_3 - x_1)\dotsm(x_k - x_1) \begin{vmatrix}
		1 & 1 & \dots & 1 \\
		x_2 & x_3 & \dots & x_k \\
		\vdots & \vdots & & \vdots \\
		x_2^{k-2} & x_3^{k-2} & \dots & x_k^{k-2} \\
	\end{vmatrix} \\
	&= (x_2 - x_1)(x_3 - x_1)\dotsm(x_k - x_1)
		\prod_{2 \leq j < i \leq k}(x_i - x_j) \\
	&= \prod_{1 \leq j < i \leq k}(x_i - x_j).
	\qedhere
\end{align*}
\end{proof}
\end{example}
我们可以从范德蒙德行列式的表达式 \labelcref{equation:行列式.范德蒙德行列式} 看出,
\(n\)阶范德蒙德行列式\(V_n\)等于零的充分必要条件是:
\(\AutoTuple{x}{n}\)这\(n\)个数中至少有两个相等,即\[
	(\exists b,c\in\Set{\AutoTuple{x}{n}})[b=c].
\]
因此,如果\(\AutoTuple{x}{n}\)两两不等,则范德蒙德行列式不等于零.

\begin{example}
%@see: 《线性代数》(张慎语、周厚隆) P35. 例4
计算\(n\)阶三对角行列式\[
	D_n = \begin{vmatrix}
		a+b & ab & \\
		1 & a+b & ab & \\
		& 1 & a + b & \ddots & \\
		& & \ddots & \ddots & ab \\
		& & & 1 & a+b \\
	\end{vmatrix}_n.
\]
\begin{proof}
将\(D_n\)按第一行展开,得\begin{align*}
D_n &= (a+b) D_{n-1} - ab \begin{vmatrix}
1 & ab \\
0 & a+b & ab \\
 & 1 & a+b & \ddots \\
 & & \ddots & \ddots & ab \\
 & & & 1 & a+b \\
\end{vmatrix}_{n-1} \\
&= (a+b) D_{n-1} - ab D_{n-2},
\end{align*}
把上式改写为\(D_n - a D_{n-1} = b(D_{n-1} - a D_{n-2})\),继续递推下去,得\begin{align*}
D_n - a D_{n-1} &= b(D_{n-1} - a D_{n-2}) = b^2(D_{n-2} - a D_{n-3}) \\
&= \dotsb = b^{n-2}(D_2 - a D_1) \\
&= b^{n-2} [(a^2 + ab + b^2) - a(a+b)] = b^n,
\end{align*}所以\[
D_n - a D_{n-1} = b^n,
\]又由\(a\)和\(b\)的对称性可得\[
D_n - b D_{n-1} = a^n.
\]

当\(a \neq b\)时,可解得\[
D_n = \frac{a^{n+1} - b^{n+1}}{a - b}
= a^n + a^{n-1} b + a^{n-2} b^2 + \dotsb + a b^{n-1} + b^n.
\]当\(a = b\)时,由\[
D_n - a D_{n-1} = a^n
\]可继续递推得\begin{align*}
D_n &= a D_{n-1} + a^n
= a(a D_{n-2} + a^{n-1}) + a^n
= a^2 D_{n-2} + 2 a^n \\
&= a^3 D_{n-3} + 3 a^n
= \dotsb
= (n+1) a^n.
\end{align*}
综上所述,对任意\(a,b\in\mathbb{R}\)都有\[
D_n = a^n + a^{n-1} b + a^{n-2} b^2 + \dotsb + a b^{n-1} + b^n.
\qedhere
\]
\end{proof}
\end{example}

\begin{example}
设矩阵\(\A = (a_{ij})_n\),\(A_{ij}\)为\(a_{ij}\)的代数余子式.
把\(\A\)的每个元素都加上同一个数\(t\),
得到的矩阵记作\(\A(t) = (a_{ij} + t)_n\).
证明:\[
	\abs{\A(t)}
	= \abs{\A} + t \sum_{i=1}^n \sum_{j=1}^n A_{ij}.
\]
\begin{proof}
对\(\A\)列分块得\[
	\A = (\AutoTuple{\a}{n}),
\]
又令\(n\)维列向量\(\b=(t,t,\dotsc,t)^T\),
那么根据\cref{theorem:行列式.性质3,theorem:行列式.性质5} 有
\begin{align*}
	\abs{\A(t)}
	&= \abs{(\a_1+\b,\a_2+\b,\dotsc,\a_n+\b)} \\
	&= \abs{(\a_1,\a_2+\b,\dotsc,\a_n+\b)}
		+ \abs{(\b,\a_2+\b,\dotsc,\a_n+\b)} \\
	&= \dotsb = \abs{(\a_1,\a_2,\dotsc,\a_n)}
		\begin{aligned}[t]
			&+ \abs{(\b,\a_2,\dotsc,\a_{n-1},\a_n)} \\
			&+ \abs{(\a_1,\b,\dotsc,\a_{n-1},\a_n)} \\
			&+ \dotsb
			+ \abs{(\a_1,\a_2,\dotsc,\a_{n-1},\b)}
		\end{aligned}
\end{align*}
%TODO
%直接计算得\begin{align*}
%	\abs{\A(t)}
%	&=\begin{vmatrix}
%		a_{11} + t & a_{12} + t & \dots & a_{1n} + t \\
%		\vdots & \vdots & & \vdots \\
%		a_{n1} + t & a_{n2} + t & \dots & a_{nn} + t
%	\end{vmatrix} \\
%	&= \begin{vmatrix}
%		a_{11} + t & a_{12} - a_{11} & \dots & a_{1n} - a_{11} \\
%		\vdots & \vdots & & \vdots \\
%		a_{n1} + t & a_{n2} - a_{n1} & \dots & a_{nn} - a_{n1}
%	\end{vmatrix} \tag{\cref{theorem:行列式.性质6}} \\
%	&= \begin{vmatrix}
%		a_{11} & a_{12} - a_{11} & \dots & a_{1n} - a_{11} \\
%		\vdots & \vdots & & \vdots \\
%		a_{n1} & a_{n2} - a_{n1} & \dots & a_{nn} - a_{n1}
%	\end{vmatrix} + \begin{vmatrix}
%		t & a_{12} - a_{11} & \dots & a_{1n} - a_{11} \\
%		\vdots & \vdots & & \vdots \\
%		t & a_{n2} - a_{n1} & \dots & a_{nn} - a_{n1}
%	\end{vmatrix} \tag{\cref{theorem:行列式.性质3}} \\
%	&= \begin{vmatrix}
%		a_{11} & a_{12} & \dots & a_{1n} \\
%		\vdots & \vdots & & \vdots \\
%		a_{n1} & a_{n2} & \dots & a_{nn}
%	\end{vmatrix} + \begin{vmatrix}
%		t & a_{12} - a_{11} & \dots & a_{1n} - a_{11} \\
%		\vdots & \vdots & & \vdots \\
%		t & a_{n2} - a_{n1} & \dots & a_{nn} - a_{n1}
%	\end{vmatrix} \tag{\cref{theorem:行列式.性质6}} \\
%	&= \abs{\A} + t \begin{vmatrix}
%		1 & a_{12} - a_{11} & \dots & a_{1n} - a_{11} \\
%		\vdots & \vdots & & \vdots \\
%		1 & a_{n2} - a_{n1} & \dots & a_{nn} - a_{n1}
%	\end{vmatrix}, \tag{\cref{theorem:行列式.性质2}}
%\end{align*}
%因此只需证\[
%	\begin{vmatrix}
%		1 & a_{12} - a_{11} & \dots & a_{1n} - a_{11} \\
%		\vdots & \vdots & & \vdots \\
%		1 & a_{n2} - a_{n1} & \dots & a_{nn} - a_{n1}
%	\end{vmatrix} = \sum_{i=1}^n \sum_{j=1}^n A_{ij}.
%\]
\end{proof}
\end{example}

\subsection{行列式按\texorpdfstring{\(k\)}{k}行(或\texorpdfstring{\(k\)}{k}列)展开}
\begin{theorem}[拉普拉斯定理]\label{theorem:行列式.拉普拉斯定理}
%@see: 《高等代数创新教材(上册)》(丘维声) P68. 定理1(Laplace定理)
在\(n\)阶矩阵\(\A\)中,
取定第\(\AutoTuple{i}{k}\)行
(\(i_1<i_2<\dotsb<i_k\)
且\(1 \leq k < n\)),
则这\(k\)行元素形成的所有\(k\)阶子式与它们自己的代数余子式的乘积之和等于\(\abs{\A}\),
即\begin{equation}
	\abs{\A} =
	\sum_{1 \leq j_1 < j_2 < \dotsb < j_k \leq n}
	\MatrixMinor\A{
		\AutoTuple{i}{k} \\
		\AutoTuple{j}{k}
	}
	(-1)^{(i_1+\dotsb+i_k)+(j_1+\dotsb+j_k)}
	\MatrixMinor\A{
		\AutoTuple{\mu}{n-k} \\
		\AutoTuple{\nu}{n-k}
	},
\end{equation}
其中\[
	\Set{ \AutoTuple{\mu}{n-k} } = \Set{ 1,2,\dotsc,n } - \Set{ \AutoTuple{i}{k} },
\]\[
	\Set{ \AutoTuple{\nu}{n-k} } = \Set{ 1,2,\dotsc,n } - \Set{ \AutoTuple{j}{k} },
\]
且\(\mu_1<\mu_2<\dotsb<\mu_{n-k},
\nu_1<\nu_2<\dotsb<\nu_{n-k}\).
\begin{proof}
根据\cref{equation:行列式.给定行指标排列下的行列式的完全展开式},
给定\(\abs{\A}\)的行指标排列\(\AutoTuple{i}{k},\AutoTuple{\mu}{n-k}\),
\(\abs{\A}\)的表达式为\begin{align*}
	\abs{\A}
	&= \sum_{\AutoTuple{p}{k},\AutoTuple{q}{n-k}}
	(-1)^{\tau(\AutoTuple{i}{k},\AutoTuple{\mu}{n-k}) + \tau(\AutoTuple{p}{k},\AutoTuple{q}{n-k})}
	a_{i_1 p_1} \dotsm a_{i_k p_k}
	a_{\mu_1 q_1} \dotsm a_{\mu_{n-k} q_{n-k}} \\
	&= \sum_{\AutoTuple{p}{k},\AutoTuple{q}{n-k}}
	(-1)^{(i_1+\dotsb+i_k) - \frac{1}{2}k(k+1) + \tau(\AutoTuple{p}{k},\AutoTuple{q}{n-k})}
	a_{i_1 p_1} \dotsm a_{i_k p_k}
	a_{\mu_1 q_1} \dotsm a_{\mu_{n-k} q_{n-k}}.
\end{align*}

通过任意给定\(\AutoTuple{j}{k}\),
其中\(1 \leq j_1 < j_2 < \dotsb < j_k \leq n\),
可以把\(n!\)个\(n\)阶排列分成\(C_n^k\)组,
对应于\(\AutoTuple{j}{k}\)这一组中的\(n\)阶排列形如\(\AutoTuple{p}{k},\AutoTuple{q}{n-k}\),
其中\(\AutoTuple{p}{k}\)是\(\AutoTuple{j}{k}\)形成的\(k\)阶排列,
\(\AutoTuple{q}{n-k}\)是\(\AutoTuple{\mu}{n-k}\)形成的\(n-k\)阶排列.
于是对于第\(\AutoTuple{i}{k}\)行,
通过任意取定\(k\)列,例如第\(\AutoTuple{j}{k}\)列,
可以把\(\abs{\A}\)的表达式中的\(n!\)项分成\(C_n^k\)组;
再根据\cref{theorem:行列式.任意排列可化为自然序} 有
\begin{align*}
	&\hspace{-20pt}
	(-1)^{\tau(\AutoTuple{p}{k},\AutoTuple{q}{n-k})}
	= (-1)^{\tau(\AutoTuple{p}{k})}
	(-1)^{\tau(\AutoTuple{i}{k},\AutoTuple{q}{n-k})} \\
	&= (-1)^{\tau(\AutoTuple{p}{k})}
	(-1)^{[(i_1-1)+(i_2-2)+\dotsb+(i_k-k)] + \tau(\AutoTuple{q}{n-k})} \\
	&= (-1)^{(i_1+\dotsb+i_k) - \frac{1}{2}k(k+1)}
	(-1)^{\tau(\AutoTuple{p}{k}) + \tau(\AutoTuple{q}{n-k})},
\end{align*}
于是\begin{align*}
	\abs{\A}
	&= \sum_{i \leq j_1 < \dotsb < j_k \leq n}
			\sum_{\AutoTuple{p}{k}}
			\sum_{\AutoTuple{q}{n-k}}
			(-1)^{(i_1+\dotsb+i_k) - \frac{1}{2}k(k+1)}
			(-1)^{(j_1+\dotsb+j_k) - \frac{1}{2}k(k+1)} \\
		&\hspace{40pt}\cdot(-1)^{\tau(\AutoTuple{p}{k}) + \tau(\AutoTuple{q}{n-k})}
			a_{i_1 p_1} \dotsm a_{i_k p_k}
			a_{\mu_1 q_1} \dotsm a_{\mu_{n-k} q_{n-k}} \\
	&= \sum_{i \leq j_1 < \dotsb < j_k \leq n}
		(-1)^{(i_1+\dotsb+i_k)+(j_1+\dotsb+j_k)}
		\biggl\{
			\biggl[
				\sum_{\AutoTuple{p}{k}}
				(-1)^{\tau(\AutoTuple{p}{k})}
				a_{i_1 p_1} \dotsm a_{i_k p_k}
			\biggr] \\
			&\hspace{40pt}\cdot\biggl[
				\sum_{\AutoTuple{q}{n-k}}
				(-1)^{\tau(\AutoTuple{q}{n-k})}
				a_{\mu_1 q_1} \dotsm a_{\mu_{n-k} q_{n-k}}
			\biggr]
		\biggr\} \\
	&= \sum_{i \leq j_1 < \dotsb < j_k \leq n}
		(-1)^{(i_1+\dotsb+i_k)+(j_1+\dotsb+j_k)}
		\MatrixMinor\A{
			\AutoTuple{i}{k} \\
			\AutoTuple{j}{k}
		}
		\MatrixMinor\A{
			\AutoTuple{\mu}{n-k} \\
			\AutoTuple{\nu}{n-k}
		}.
	\qedhere
\end{align*}
\end{proof}
\end{theorem}
\cref{theorem:行列式.拉普拉斯定理} 称为“拉普拉斯定理”或“行列式按\(k\)行展开定理”.

由于行列式中行与列的地位平等,因此也有行列式按\(k\)列展开的定理:
\begin{theorem}\label{theorem:行列式.行列式按k列展开}
%@see: 《高等代数(第三版)》(丘维声) P52. 定理2
\(n\)阶行列式\(\abs{\A}\)中,取定\(k\ (1 \leq k < n)\)列,
则这\(k\)列元素形成的所有\(k\)阶子式与它们自己的代数余子式的乘积之和等于\(\abs{\A}\).
\end{theorem}

\section{矩阵乘积的行列式}
\begin{lemma}
设\(\A\in M_n^*(K),
\B\in M_m^*(K),
\C\in M_{m\times n}(K),
\D\in M_{n\times m}(K)\),
则\begin{align}
	\begin{vmatrix}
		\A & \z \\
		\C & \B
	\end{vmatrix}
	&= \begin{vmatrix}
		\A & \D \\
		\z & \B
	\end{vmatrix}
	= \abs{\A} \abs{\B}, \label{equation:行列式.广义三角阵的行列式1} \\
	\begin{vmatrix}
		\z & \A \\
		\B & \C
	\end{vmatrix}
	&= \begin{vmatrix}
		\D & \A \\
		\B & \z
	\end{vmatrix}
	= (-1)^{mn} \abs{\A} \abs{\B}. \label{equation:行列式.广义三角阵的行列式2}
\end{align}
\end{lemma}

\begin{theorem}[矩阵乘积的行列式定理]\label{theorem:行列式.矩阵乘积的行列式}
设\(\A,\B\in M_n(K)\),
则\begin{equation}
\abs{\A\B} = \abs{\A} \abs{\B}.
\end{equation}
\end{theorem}

\begin{example}
设\(\A\)是\(n\)阶矩阵,且\(\A\A^T = \E\),\(\abs{\A}<0\).证明:\(\abs{\E+\A}=0\).
\begin{proof}
等式\(\A\A^T=\E\)的两端取行列式,\(\abs{\A} \abs{\A^T} = \abs{\E}\),由\(\abs{\A} = \abs{\A^T}\)与\(\abs{\A} < 0\),得\(\abs{\A}^2 = 1\),\(\abs{\A} = -1\).故\begin{align*}
\abs{\E+\A}
&= \abs{\A\A^T+\A}
= \abs{\A(\A^T+\E)}
= \abs{\A} \abs{\A^T+\E} \\
&= -\abs{(\A^T+\E)^T}
= -\abs{\A+\E},
\end{align*}所以\(\abs{\E+\A} = -\abs{\E+\A}\),\(\abs{\E+\A}=0\).
\end{proof}
\end{example}

\begin{example}
设\(\A^*\)是\(n\)阶矩阵\(\A\)的伴随矩阵,若\(\abs{\A} \neq 0\),证明:\(\abs{\A^*} \neq 0\)且\(\abs{\A^*}=\abs{\A}^{n-1}\).
\begin{proof}
因为\(\A \A^* = \abs{\A} \E\),%
所以\(\abs{\A \A^*} = \abs{\abs{\A} \E} = \abs{\A}^n \abs{\E} = \abs{\A}^n\).
又因为\(\abs{\A \A^*} = \abs{\A} \abs{\A^*}\),%
所以\(\abs{\A^*} = \abs{\A}^{n-1} \neq 0\).
\end{proof}
\end{example}

\begin{example}
设矩阵\(\A,\B\)满足\(\abs{\A}=3,\abs{\B}=2,\abs{\A^{-1}+\B}=2\),试求\(\abs{\A+\B^{-1}}\).
\begin{solution}
由于\(\abs{\E+\A\B} = \abs{\A(\A^{-1}+\B)} = \abs{\A} \abs{\A^{-1}+\B} = 6\),所以\[
\abs{\A+\B^{-1}}
= \frac{\abs{(\A+\B^{-1})\B}}{\abs{\B}}
= \frac{\abs{\A\B+\E}}{\abs{\B}}
= 3.
\]
\end{solution}
\end{example}

\begin{example}
用\(\abs{\A}^2 = \abs{\A} \abs{\A^T}\)的方法计算行列式\[
\abs{\A} = \begin{vmatrix}
a & b & c & d \\
-b & a & d & -c \\
-c & -d & a & b \\
-d & c & -b & a
\end{vmatrix}.
\]
\begin{solution}
因为\begin{align*}
\abs{\A}^2 &= \abs{\A} \abs{\A^T} = \abs{\A \A^T} \\
&= \abs{\begin{bmatrix}
a & b & c & d \\
-b & a & d & -c \\
-c & -d & a & b \\
-d & c & -b & a
\end{bmatrix} \begin{bmatrix}
a & -b & -c & -d \\
b & a & -d & c \\
c & d & a & -b \\
d & -c & b & a
\end{bmatrix}} \\
&= \abs{(a^2+b^2+c^2+d^2) \E}_4
= (a^2+b^2+c^2+d^2)^4,
\end{align*}所以\(\abs{\A} = \pm(a^2+b^2+c^2+d^2)^2\),再由\(\abs{\A}\)中含有项\(a^4\),得\[
\abs{\A} = (a^2+b^2+c^2+d^2)^2.
\]
\end{solution}
\end{example}

\begin{example}
计算:\[
	D = \begin{vmatrix}
		a & a & a & a \\
		a & a & -a & -a \\
		a & -a & a & -a \\
		a & -a & -a & a
	\end{vmatrix}.
\]
\begin{solution}
因为\[
	\begin{bmatrix}
		1 & 0 & 0 & 0 \\
		0 & 0 & 1 & 0 \\
		0 & 1 & 0 & 0 \\
		0 & 0 & 0 & 1
	\end{bmatrix} \begin{bmatrix}
		a & a & a & a \\
		a & a & -a & -a \\
		a & -a & a & -a \\
		a & -a & -a & a
	\end{bmatrix}
	= \begin{bmatrix}
		1 & 0 & 0 & 0 \\
		1 & 1 & 0 & 0 \\
		1 & 0 & 1 & 0 \\
		1 & 1 & 1 & 1
	\end{bmatrix} \begin{bmatrix}
		a & a & a & a \\
		0 & -2 a & 0 & -2 a \\
		0 & 0 & -2 a & -2 a \\
		0 & 0 & 0 & 4 a
	\end{bmatrix},
\]而\[
	\begin{vmatrix}
		1 & 0 & 0 & 0 \\
		0 & 0 & 1 & 0 \\
		0 & 1 & 0 & 0 \\
		0 & 0 & 0 & 1
	\end{vmatrix} = -1,
	\qquad
	\begin{vmatrix}
		1 & 0 & 0 & 0 \\
		1 & 1 & 0 & 0 \\
		1 & 0 & 1 & 0 \\
		1 & 1 & 1 & 1
	\end{vmatrix} = 1,
\]\[
	\begin{vmatrix}
		a & a & a & a \\
		0 & -2 a & 0 & -2 a \\
		0 & 0 & -2 a & -2 a \\
		0 & 0 & 0 & 4 a
	\end{vmatrix}
	= a\cdot(-2a)\cdot(-2a)\cdot(4a) = 16a^2,
\]
所以\[
	\begin{vmatrix}
		a & a & a & a \\
		a & a & -a & -a \\
		a & -a & a & -a \\
		a & -a & -a & a
	\end{vmatrix}
	= -16a^2.
\]
\end{solution}
\end{example}

\begin{example}
设\(s_k = a_1^k + a_2^k + a_3^k + a_4^k\ (k=1,2,3,4,5,6)\).
计算:\[
	D = \begin{vmatrix}
		4 & s_1 & s_2 & s_3 \\
		s_1 & s_2 & s_3 & s_4 \\
		s_2 & s_3 & s_4 & s_5 \\
		s_3 & s_4 & s_5 & s_6
	\end{vmatrix}.
\]
\begin{solution}
令矩阵\[
	\A = \begin{bmatrix}
		1 & 1 & 1 & 1 \\
		a_1 & a_2 & a_3 & a_4 \\
		a_1^2 & a_2^2 & a_3^2 & a_4^2 \\
		a_1^3 & a_2^3 & a_3^3 & a_4^3
	\end{bmatrix},
\]
显然\[
	D = \det(\A\A^T) = \abs{\A}^2.
\]
而根据\cref{equation:行列式.范德蒙德行列式},
\(\abs{\A}
= \prod\limits_{1 \leq j < i \leq n} (a_i - a_j)\),
故\(D = \prod\limits_{1 \leq j < i \leq n} (a_i - a_j)^2\).
\end{solution}
\end{example}

\section{柯西--比内公式}
\begin{theorem}
%@see: 《高等代数(第三版 上册)》(丘维声) P141. 定理1
已知数域\(K\).
设矩阵\(\A \in M_{m \times n}(K),
\B \in M_{n \times m}(K)\).
如果\(m < n\),
那么\begin{equation}\label{equation:线性方程组.柯西比内公式}
	\abs{\A\B}
	= \sum_{1 \leq i_1 < i_2 < \dotsb < i_m \leq n}
	\MatrixMinor\A{
		1,2,\dotsc,m \\
		i_1,i_2,\dotsc,i_m
	}
	\MatrixMinor\B{
		i_1,i_2,\dotsc,i_m \\
		1,2,\dotsc,m
	}.
\end{equation}
\begin{proof}
考虑\(m+n\)阶分块矩阵\[
	\begin{bmatrix}
		\E_n & \B \\
		\z & \A\B
	\end{bmatrix}.
\]
由于\[
	\begin{vmatrix}
		\E_n & \B \\
		\z & \A\B
	\end{vmatrix}
	= \abs{\E_n} \abs{\A\B}
	= \abs{\A\B},
\]
所以\[
	\begin{bmatrix}
		\E_n & \B \\
		\z & \A\B
	\end{bmatrix}
	\to
	\begin{bmatrix}
		\E_n & \B \\
		-\A & \z
	\end{bmatrix}
	= \begin{bmatrix}
		\E_n & \z \\
		-\A & \E_m
	\end{bmatrix} \begin{bmatrix}
		\E_n & \B \\
		\z & \A\B
	\end{bmatrix},
\]\[
	\begin{vmatrix}
		\E_n & \B \\
		-\A & \z
	\end{vmatrix}
	= \begin{vmatrix}
		\E_n & \z \\
		-\A & \E_m
	\end{vmatrix} \begin{vmatrix}
		\E_n & \B \\
		\z & \A\B
	\end{vmatrix}
	= \begin{vmatrix}
		\E_n & \B \\
		\z & \A\B
	\end{vmatrix},
\]
根据\hyperref[theorem:行列式.拉普拉斯定理]{拉普拉斯定理},
把上式最左端行列式按后\(m\)行展开得\[
	\begin{vmatrix}
		\E_n & \B \\
		-\A & \z
	\end{vmatrix}
	= \sum_{1 \leq i_1 < \dotsb < i_m \leq n}
	\MatrixMinor{(-\A)}{
		1,2,\dotsc,m \\
		i_1,i_2,\dotsc,i_m
	}
	(-1)^{[(n+1)+\dotsb+(n+m)]+(i_1+\dotsb+i_m)}
	\abs{(\e_{\mu_1},\dotsc,\e_{\mu_{n-m}},\B)},
\]
其中\(\Set{\mu_1,\dotsc,\mu_{n-m}}
= \Set{1,\dotsc,n}-\Set{i_1,\dotsc,i_s}\),
且\(\mu_1<\dotsb<\mu_{n-m}\).

把\(\abs{(\e_{\mu_1},\dotsc,\e_{\mu_{n-m}},\B)}\)
按前\(n-m\)行展开得\[
	\abs{(\e_{\mu_1},\dotsc,\e_{\mu_{n-m}},\B)}
	= \abs{\E_{n-m}}
	(-1)^{(\mu_1+\dotsb+\mu_{n-m})+[1+\dotsb+(n-m)]}
	\MatrixMinor{\B}{
		i_1,i_2,\dotsc,i_m \\
		1,2,\dotsc,m
	}.
\]
因此\begin{align*}
	\begin{vmatrix}
		\E_n & \B \\
		-\A & \z
	\end{vmatrix}
	&= \sum_{1 \leq i_1 < \dotsb < i_m \leq n}
	(-1)^{m+m^2+n+n^2}
	\MatrixMinor\A{
		1,2,\dotsc,m \\
		i_1,i_2,\dotsc,i_m
	}
	\MatrixMinor\B{
		i_1,i_2,\dotsc,i_m \\
		1,2,\dotsc,m
	} \\
	&= \sum_{1 \leq i_1 < \dotsb < i_m \leq n}
	\MatrixMinor\A{
		1,2,\dotsc,m \\
		i_1,i_2,\dotsc,i_m
	}
	\MatrixMinor\B{
		i_1,i_2,\dotsc,i_m \\
		1,2,\dotsc,m
	}.
\end{align*}
综上所述,\[
	\abs{\A\B}
	= \sum_{1 \leq i_1 \leq i_2 \leq \dotsb \leq i_m \leq n}
	\MatrixMinor\A{
		1,2,\dotsc,m \\
		i_1,i_2,\dotsc,i_m
	}
	\MatrixMinor\B{
		i_1,i_2,\dotsc,i_m \\
		1,2,\dotsc,m
	}.
\]
\end{proof}
\end{theorem}
\cref{equation:线性方程组.柯西比内公式} 称为\DefineConcept{柯西--比内公式}.


\begin{example}
设\(\A = (\B,\C) \in M_{n \times m}(\mathbb{R})\),
其中\(\B \in M_{n \times s}(\mathbb{R})\),
\(\C \in M_{n \times (m-s)}(\mathbb{R})\).
证明:\begin{equation}
\abs{\A^T \A} \leq \abs{\B^T \B} \abs{\C^T \C}.
\end{equation}
%TODO
\end{example}

\begin{example}
设\(\A = (a_{ij})_n \in M_n(\mathbb{R})\).
证明:\begin{equation}\label{equation:线性方程组.Hadamard不等式}
	\abs{\A}^2 \leq \prod_{j=1}^n \sum_{i=1}^n a_{ij}^2.
\end{equation}
%TODO
\end{example}

\begin{example}
设\(\A = (a_{ij})_n \in M_n(\mathbb{R})\),
且\(\abs{a_{ij}} < M\ (i,j=1,2,\dotsc,n)\).
证明:\begin{equation}
	\abs{\det\A} \leq M^n n^{n/2}.
\end{equation}
%TODO
\end{example}

\section{本章总结}
我们可以采用以下方法计算行列式:
\begin{enumerate}
	\item 利用初等变换,将行列式化为上三角形,如此行列式就等于主对角线上元素之积.
	\item 拆成若干个行列式的和.
	\item 按行(或列)展开.
	\item 数学归纳法.
	\item 递推关系法.
	\item 加边法(即升阶法).
	\item 降阶法.
	\item 利用\hyperref[equation:行列式.范德蒙德行列式]{范德蒙德行列式}等特殊行列式计算.
\end{enumerate}


\chapter{矩阵的逆}
\section{可逆矩阵}
\begin{definition}\label{definition:可逆矩阵.可逆矩阵的定义}
%@see: 《线性代数》(张慎语、周厚隆) P43. 定义1
%@see: 《高等代数(第三版 上册)》(丘维声) P128. 定义1
对于数域\(K\)上的矩阵\(\A\),
如果存在数域\(K\)上的矩阵\(\B\),使得\[
	\A\B=\B\A=\E,
\]
则称“\(\A\)是一个\DefineConcept{可逆矩阵}(\(\A\) is an invertible matrix)”,
或称“\(\A\) \DefineConcept{可逆}(\(\A\) is invertible)”;
并称“\(\B\)是\(\A\)的\DefineConcept{逆矩阵}(inverse matrix)”,
记作\(\A^{-1}\),
即\[
	\A^{-1} = \B \defiff \A\B=\B\A=\E.
\]
\end{definition}

从定义可知,如果矩阵\(\A,\B\)满足\(\A\B=\B\A=\E\),
那么这两个矩阵都是可逆矩阵,且两者互为逆矩阵.

\begin{definition}
设\(\A \in M_n(K)\).
\begin{enumerate}
	\item 若\(\abs{\A}=0\),
	则称“\(\A\)是\DefineConcept{奇异矩阵}(singular matrix)”.
	\item 若\(\abs{\A} \neq 0\),
	则称“\(\A\)是\DefineConcept{非奇异矩阵}(non-singular matrix)”.
\end{enumerate}
\end{definition}

\begin{definition}
若\(\A\)是可逆矩阵,那么规定:
对于正整数\(k\),有
\begin{equation}
	\A^{-k} = (\A^{-1})^k
	= \underbrace{\A^{-1}\A^{-1}\dotsm\A^{-1}}_{k\ \text{个}}.
\end{equation}
\end{definition}

\begin{theorem}\label{theorem:逆矩阵.矩阵可逆的充分必要条件1}
%@see: 《线性代数》(张慎语、周厚隆) P43. 定理1
设\(\A\)是\(n\)阶方阵,则“\(\A\)可逆”的充分必要条件是“\(\A\)是非奇异矩阵”.
\begin{proof}
必要性.
假设矩阵\(\A\)可逆,那么存在\(n\)阶方阵\(\B\),使得\(\A\B=\E\),于是\(\abs{\A\B}=\abs{\E}\);
而根据\cref{theorem:行列式.矩阵乘积的行列式},
\(\abs{\A\B}=\abs{\A}\abs{\B}=1\),\(\abs{\A}\neq0\).

充分性.
设\(\abs{\A}\neq0\),\(\A^*\)是\(\A\)的伴随矩阵.
根据\cref{equation:行列式.伴随矩阵.恒等式1},
若令\[
	\B=\frac{1}{\abs{\A}} \A^*,
\]
则有\(\A\B = \B\A = \E\),
故由可逆矩阵的定义可知,矩阵\(\A\)可逆,
且有\(\A^{-1} = \B\).
\end{proof}
\end{theorem}

\begin{property}\label{theorem:逆矩阵.逆矩阵的唯一性}
设\(\A\)是可逆矩阵,则它的逆矩阵存在且唯一,且有\begin{equation}
	\A^{-1}=\frac{1}{\abs{\A}}\A^*.
\end{equation}
\begin{proof}
存在性.
在\cref{theorem:逆矩阵.矩阵可逆的充分必要条件1} 的证明过程中,
我们看到矩阵\(\B=\abs{\A}^{-1} \A^*\)是可逆矩阵\(\A\)的一个逆矩阵,即\(\A\B=\E\).

唯一性.
设矩阵\(\C\)也是\(\A\)的逆矩阵,即\(\C\A=\E\),于是\[
	\C=\C\E=\C(\A\B)=(\C\A)\B=\E\B=\B.
	\qedhere
\]
\end{proof}
\end{property}

\begin{property}\label{theorem:逆矩阵.单位矩阵可逆}
单位矩阵\(\E\)可逆,且\(\E^{-1}=\E\).
\end{property}

\begin{property}\label{theorem:逆矩阵.逆矩阵的行列式}
%@see: 《线性代数》(张慎语、周厚隆) P44. 性质3
设\(\A\)可逆,则\(\abs{\A^{-1}}=\abs{\A}^{-1}\).
\end{property}

\begin{property}\label{theorem:逆矩阵.逆矩阵的逆}
%@see: 《线性代数》(张慎语、周厚隆) P44. 性质4
设\(\A\)可逆,则\(\A^{-1}\)可逆,且\((\A^{-1})^{-1}=\A\).
\end{property}

\begin{property}\label{theorem:逆矩阵.矩阵乘积的逆1}
%@see: 《线性代数》(张慎语、周厚隆) P44. 性质5
设\(\A\)、\(\B\)都是\(n\)阶可逆矩阵,则\(\A\B\)可逆,且\begin{equation}
(\A \B)^{-1} = \B^{-1} \A^{-1}.
\end{equation}
\begin{proof}
因为\(\A,\B\)都可逆,可设它们的逆矩阵分别为\(\A^{-1},\B^{-1}\),
于是\[
	(\A\B)(\B^{-1}\A^{-1})
	= \A(\B\B^{-1})\A^{-1}
	= \A\E\A^{-1}
	= \A\A^{-1}
	= \E.
	\qedhere
\]
\end{proof}
\end{property}

\cref{theorem:逆矩阵.矩阵乘积的逆1} 可以推广到有限个\(n\)阶可逆矩阵乘积的情形.
\begin{property}\label{theorem:逆矩阵.矩阵乘积的逆2}
设\(\AutoTuple{\A}{n}\)都是\(n\)阶可逆矩阵,
则\(\A_1 \A_2 \dotsm \A_{n-1} \A_n\)可逆,且\begin{equation}
(\A_1 \A_2 \dotsm \A_{n-1} \A_n)^{-1}
= \A_n^{-1} \A_{n-1}^{-1} \dotsm \A_2^{-1} \A_1^{-1}.
\end{equation}
\end{property}

\begin{property}\label{theorem:逆矩阵.数与矩阵乘积的逆}
%@see: 《线性代数》(张慎语、周厚隆) P44. 性质6
设数域\(K\)上的\(n\)阶矩阵\(\A\)可逆,
\(k \in K-\{0\}\),则\(k\A\)可逆,且
\begin{equation}
	(k\A)^{-1}=k^{-1}\A^{-1}.
\end{equation}
\begin{proof}
由\cref{theorem:行列式.性质2.推论2},
\(\abs{k\A} = k^n\abs{\A}\).
因为\(\A\)可逆,所以\(\abs{\A}\neq0\);
又因为\(k\neq0\),所以\(\abs{k\A}\neq0\),即\(k\A\)可逆.
因此\[
	(k^{-1}\A^{-1})(k\A)
	= (k^{-1} \cdot k)(\A^{-1}\A)
	= 1 \E = \E,
\]
也就是说\(k^{-1}\A^{-1}\)是\(k\A\)的逆矩阵.
\end{proof}
\end{property}

\begin{property}\label{theorem:逆矩阵.转置矩阵的逆与逆矩阵的转置}
%@see: 《线性代数》(张慎语、周厚隆) P44. 性质7
设\(\A\)可逆,则\(\A^T\)可逆,且\begin{equation}
	(\A^T)^{-1}=(\A^{-1})^T.
\end{equation}
\begin{proof}
由\cref{theorem:行列式.性质1},
\(\abs{\A^T}=\abs{\A}\neq0\),
于是\(\A^T\)可逆.
由\cref{theorem:矩阵.矩阵乘积的转置},
\((\A \A^{-1})^T = (\A^{-1})^T \A^T\).
既然\(\A \A^{-1} = \E, \E^T = \E\),
于是\((\A^{-1})^T \A^T = \E\),
那么由逆矩阵的定义可知,
\((\A^T)^{-1}=(\A^{-1})^T\).
\end{proof}
\end{property}

\begin{example}
若\(\A\)、\(\B\)可交换,证明:\(\A^{-1}\)与\(\B\)可交换.
\begin{proof}
因为\(\A\B = \B\A\),在等式两边同时左乘\(\A^{-1}\),得\[
\B = (\A^{-1}\A)\B = \A^{-1}(\A\B) = \A^{-1}(\B\A);
\]再在等式两边右乘\(\A^{-1}\),得\[
\B\A^{-1} = (\A^{-1}\B\A)\A^{-1} = \A^{-1}\B(\A\A^{-1}) = \A^{-1}\B.
\qedhere
\]
\end{proof}
\end{example}

\begin{example}
下面看一些常见矩阵的逆矩阵:
\[
	\begin{bmatrix}
		\lambda_1 \\
		& \lambda_2 \\
		&& \ddots \\
		&&& \lambda_n
	\end{bmatrix}^{-1}
	= \begin{bmatrix}
		\lambda_1^{-1} \\
		& \lambda_2^{-1} \\
		&& \ddots \\
		&&& \lambda_n^{-1}
	\end{bmatrix},
\]\[
	\begin{bmatrix}
		& & & & \lambda_1 \\
		& & & \lambda_2 \\
		& & \iddots \\
		& \lambda_{n-1} \\
		\lambda_n
	\end{bmatrix}^{-1}
	= \begin{bmatrix}
		& & & & \lambda_n^{-1} \\
		& & & \lambda_{n-1}^{-1} \\
		& & \iddots \\
		& \lambda_2^{-1} \\
		\lambda_1^{-1}
	\end{bmatrix}.
\]
\end{example}

\begin{example}
%@see: 《线性代数》(张慎语、周厚隆) P46. 例4
设\(\A\)可逆.
证明:\(\A\)的伴随矩阵\(\A^*\)可逆,且
\begin{equation}
	(\A^*)^{-1}=\frac{1}{\abs{\A}}\A=(\A^{-1})^*.
\end{equation}
\begin{proof}
由\cref{theorem:逆矩阵.逆矩阵的唯一性,theorem:逆矩阵.逆矩阵的逆,theorem:逆矩阵.数与矩阵乘积的逆} 有\[
	\A
	= (\A^{-1})^{-1}
	= \left( \frac{1}{\abs{\A}} \A^* \right)^{-1}
	= \abs{\A} (\A^*)^{-1}.
\]
又因为\(\A\)可逆,\(\abs{\A}\neq0\),\[
	(\A^*)^{-1} = \frac{1}{\abs{\A}} \A.
\]
另外,\[
	(\A^{-1})^* = \abs{\A^{-1}} (\A^{-1})^{-1}
	= \frac{1}{\abs{\A}} \A,
\]
所以\[
	(\A^*)^{-1}=\frac{1}{\abs{\A}}\A=(\A^{-1})^*.
	\qedhere
\]
\end{proof}
\end{example}

\begin{example}\label{example:可逆矩阵.分块上三角矩阵的逆}
%@see: 《线性代数》(张慎语、周厚隆) P46. 例5
设\(\A \in M_s(K),
\B \in M_n(K),
\C \in M_{s \times n}(K)\),
\(\A\)和\(\B\)都可逆.
证明:矩阵\[
	\vb{M} = \begin{bmatrix}
		\A & \C \\
		\vb0 & \B
	\end{bmatrix}
\]可逆,且\[
	\vb{M}^{-1} = \begin{bmatrix}
		\A^{-1} & -\A^{-1} \C \B^{-1} \\
		\vb0 & \B^{-1}
	\end{bmatrix}.
\]
\begin{proof}
因为\(\A\)、\(\B\)为可逆矩阵,\(\abs{\A} \neq 0\),\(\abs{\B} \neq 0\).
所以\(\abs{\vb{M}}=\abs{\A}\abs{\B} \neq 0\),即\(\vb{M}\)可逆.

令\(\vb{M}\x=\E\),即\[
	\begin{bmatrix}
		\A & \C \\
		\vb0 & \B
	\end{bmatrix}
	\begin{bmatrix}
		\x_1 & \x_2 \\
		\x_3 & \x_4
	\end{bmatrix}
	= \begin{bmatrix}
		\E & \vb0 \\
		\vb0 & \E
	\end{bmatrix}
\]则\[
	\begin{bmatrix}
		\A\x_1+\C\x_3 & \A\x_2+\C\x_4 \\
		\B\x_3 & \B\x_4
	\end{bmatrix}
	= \begin{bmatrix}
		\E & \vb0 \\
		\vb0 & \E
	\end{bmatrix}
\]
进而有\[
	\left\{ \begin{array}{l}
		\A\x_1+\C\x_3 = \E \\
		\A\x_2+\C\x_4 = \vb0 \\
		\B\x_3 = \vb0 \\
		\B\x_4 = \E
	\end{array} \right.
\]
由第4式可得\(\x_4 = \B^{-1}\).
代入第2式得\(\A\x_2=-\C\B^{-1}\),
\(\x_2=-\A^{-1}\C\B^{-1}\).
用\(\B^{-1}\)左乘第3式左右两端,\(\B^{-1}\B\x_3=\x_3=\vb0\).
则第1式化为\(\A\x_1=\E\),显然\(\x_1=\A^{-1}\),所以\[
	\vb{M}^{-1} = \x = \begin{bmatrix}
		\A^{-1} & -\A^{-1}\C\B^{-1} \\
		\vb0 & \B^{-1}
	\end{bmatrix}.
	\qedhere
\]
\end{proof}
\end{example}

\begin{remark}
从\cref{example:可逆矩阵.分块上三角矩阵的逆} 的结论\[
	\begin{bmatrix}
		\A & \C \\
		\vb0 & \B
	\end{bmatrix}^{-1}
	= \begin{bmatrix}
		\A^{-1} & -\A^{-1} \C \B^{-1} \\
		\vb0 & \B^{-1}
	\end{bmatrix}
\]出发,
令\(\D \in M_{n \times s}(K)\),
我们还可以得到\[
	\begin{bmatrix}
		\A & \vb0 \\
		\D & \B
	\end{bmatrix}^{-1}
	= \begin{bmatrix}
		\A^{-1} & \vb0 \\
		-\B^{-1} \D \A^{-1} & \B^{-1}
	\end{bmatrix},
\]\[
	\begin{bmatrix}
		\C & \A \\
		\B & \vb0
	\end{bmatrix}^{-1}
	= \begin{bmatrix}
		\vb0 & \B^{-1} \\
		\A^{-1} & -\A^{-1}\C\B^{-1}
	\end{bmatrix},
\]\[
	\begin{bmatrix}
		\vb0 & \A \\
		\B & \D
	\end{bmatrix}^{-1}
	= \begin{bmatrix}
		-\B^{-1}\D\A^{-1} & \B^{-1} \\
		\A^{-1} & \vb0
	\end{bmatrix}.
\]

我们还可以进一步利用
\(\A^* = \abs{\A} \A^{-1}\),
得到\[
	\begin{bmatrix}
		\A & \C \\
		\vb0 & \B
	\end{bmatrix}^*
	= \begin{bmatrix}
		\abs{\B} \A^* & -\A^*\C\B^* \\
		\vb0 & \abs{\A} \B^*
	\end{bmatrix},
\]\[
	\begin{bmatrix}
		\A & \vb0 \\
		\D & \B
	\end{bmatrix}^*
	= \begin{bmatrix}
		\abs{\B} \A^* & \vb0 \\
		-\B^* \D \A^* & \abs{\A} \B^*
	\end{bmatrix},
\]\[
	\begin{bmatrix}
		\C & \A \\
		\B & \vb0
	\end{bmatrix}^*
	= (-1)^{sn} \begin{bmatrix}
		\vb0 & \abs{\A} \B^* \\
		\abs{\B} \A^* & -\A^*\C\B^*
	\end{bmatrix},
\]\[
	\begin{bmatrix}
		\vb0 & \A \\
		\B & \D
	\end{bmatrix}^*
	= (-1)^{sn} \begin{bmatrix}
		-\B^*\D\A^* & \abs{\A} \B^* \\
		\abs{\B} \A^* & \vb0
	\end{bmatrix}.
\]
\end{remark}

\section{应用初等变换求解逆矩阵}
\begin{property}\label{theorem:逆矩阵.初等矩阵的性质3}
初等矩阵可逆.
\end{property}

\begin{theorem}\label{theorem:逆矩阵.可逆矩阵与初等矩阵的关系}
设\(\A=(a_{ij})_n\),则\(\A\)可逆的充分必要条件是:
\(\A\)可经一系列初等行变换化为单位矩阵\(\E_n\),
即\(\A \cong \E_n\).
\begin{proof}
\def\Ps{\P_t \P_{t-1} \dotsm \P_2 \P_1}
存在与\(t\)次初等行变换对应的\(t\)个初等矩阵\(\P_t,\P_{t-1},\dotsc,\P_2,\P_1\),使\[
	\A \to \E_n = \Ps \A,
\]
则\(\A\)可逆且\(\A^{-1} = \Ps\).

对矩阵\((\A,\E_n)\)作以上初等行变换,则\begin{align*}
	(\A,\E_n) \to &\Ps(\A,\E_n) = \A^{-1}(\A,\E_n) \\
	&= (\A^{-1}\A,\A^{-1}\E_n) = (\E_n,\A^{-1}).
	\qedhere
\end{align*}
\end{proof}
\end{theorem}

\begin{corollary}\label{theorem:逆矩阵.计算逆矩阵的方法}
如果方阵\(\A\)经\(t\)次初等行变换为\(\E_n\),
那么同样的初等行变换会将\(\E_n\)变为\(\A^{-1}\).
\end{corollary}

\begin{corollary}
可逆矩阵\(\A\)可以表示成若干个初等矩阵的乘积.
\end{corollary}

\begin{corollary}
\(n\)阶方阵\(\A\)可逆的充分必要条件是:
\(\A\)可经过一系列初等列变换变为\(\E_n\),
且同样的初等列变换将\(\begin{bmatrix}\A\\\E_n\end{bmatrix}\)变为
\(\begin{bmatrix}\E_n\\\A^{-1}\end{bmatrix}\).
\end{corollary}

\begin{theorem}
设\(\A\)与\(\B\)都是\(s \times n\)矩阵,
则\(\A\)与\(\B\)等价的充分必要条件是:
存在\(s\)阶可逆矩阵\(\P\)与\(n\)阶可逆矩阵\(\Q\),使得\(\B=\P\A\Q\).
\end{theorem}

\begin{example}
初等矩阵的逆:
\begin{enumerate}
	\item \([\P(i,j)]^{-1}=\P(i,j)\)
	\item \([\P(i(c))]^{-1}=\P(i(c^{-1}))\)
	\item \([\P(i,j(k))]^{-1}=\P(i,j(-k))\)
\end{enumerate}
\end{example}

\section{舒尔定理}
\begin{theorem}\label{theorem:逆矩阵.舒尔定理}
设\(\A\)是\(m\)阶可逆矩阵,
\(\B,\C,\D\)分别是\(m \times p, n \times m, n \times p\)矩阵,
则有\begin{gather}
	\begin{bmatrix}
		\E_m & \z \\
		-\C\A^{-1} & \E_n
	\end{bmatrix}
	\begin{bmatrix}
		\A & \B \\
		\C & \D
	\end{bmatrix}
	= \begin{bmatrix}
		\A & \B \\
		\z & \D - \C \A^{-1} \B
	\end{bmatrix},
	\\
	\begin{bmatrix}
		\A & \B \\
		\C & \D
	\end{bmatrix}
	\begin{bmatrix}
		\E_m & -\A^{-1} \B \\
		\z & \E_p
	\end{bmatrix}
	= \begin{bmatrix}
		\A & \z \\
		\C & \D - \C \A^{-1} \B
	\end{bmatrix},
	\\
	\begin{bmatrix}
		\E_m & \z \\
		-\C \A^{-1} & \E_n
	\end{bmatrix}
	\begin{bmatrix}
		\A & \B \\
		\C & \D
	\end{bmatrix}
	\begin{bmatrix}
		\E_m & -\A^{-1} \B \\
		\z & \E_p
	\end{bmatrix}
	= \begin{bmatrix}
		\A & \z \\
		\z & \D - \C \A^{-1} \B
	\end{bmatrix},
\end{gather}
\rm
其中\(\D - \C \A^{-1} \B\)
称为矩阵\(\begin{bmatrix}
	\A & \B \\
	\C & \D
\end{bmatrix}\)
关于\(\A\)的\DefineConcept{舒尔补}(Schur complement).
\end{theorem}
我们把\cref{theorem:逆矩阵.舒尔定理} 称为“舒尔定理”.

可以看到只要\(\A\)可逆,
就能通过初等分块矩阵直接对分块矩阵进行分块对角化,
而在操作过程中就把一些原本不为0的分块矩阵变成了零矩阵,
这个过程可以形象地称为“矩阵打洞”,
即让矩阵出现尽可能多的0.

利用\hyperref[theorem:逆矩阵.舒尔定理]{舒尔定理}可以证明下述行列式降阶定理:
\begin{theorem}[行列式第一降阶定理]\label{theorem:逆矩阵.行列式第一降阶定理}
\def\M{\vb{M}}
设\(\M = \begin{bmatrix}
	\A & \B \\
	\C & \D
\end{bmatrix}\)是方阵.
\begin{enumerate}
	\item 若\(\A\)可逆,则\[
		\abs{\M} = \abs{\A} \abs{\D - \C \A^{-1} \B};
	\]

	\item 若\(\D\)可逆,则\[
		\abs{\M} = \abs{\D} \abs{\A - \B \D^{-1} \C};
	\]

	\item 若\(\A,\D\)均可逆,则\[
		\abs{\A} \abs{\D - \C \A^{-1} \B}
		= \abs{\D} \abs{\A - \B \D^{-1} \C}.
	\]
\end{enumerate}
\end{theorem}

\begin{example}\label{example:逆矩阵.行列式降阶定理的重要应用1}
设\(\A \in M_{s \times n}(K),
\B \in M_{n \times s}(K)\).
证明:\[
	\begin{vmatrix}
		\E_n & \B \\
		\A & \E_s
	\end{vmatrix} = \abs{\E_s - \A\B}.
\]
\begin{proof}
由于\(\E_n,\E_s\)都是单位矩阵,必可逆,
那么根据\cref{theorem:逆矩阵.行列式第一降阶定理},有\[
	\begin{vmatrix}
		\E_n & \B \\
		\A & \E_s
	\end{vmatrix}
	= \abs{\E_n} \abs{\E_s - \A(\E_n)^{-1}\B}
	= \abs{\E_s - \A\B}.
	\qedhere
\]
\end{proof}
\end{example}

\begin{example}
\def\M{\vb{M}}
求解行列式\(\det \M\),其中\[
	\M = \begin{bmatrix}
		1+a_1 b_1 & a_1 b_2 & \dots & a_1 b_n \\
		a_2 b_1 & 1+a_2 b_2 & \dots & a_2 b_n \\
		\vdots & \vdots & & \vdots \\
		a_n b_1 & a_n b_2 & \dots & 1+a_n b_n
	\end{bmatrix}.
\]
\begin{solution}
记\(\a = (\AutoTuple{a}{n})^T,
\b = (\AutoTuple{b}{n})\),
则\(\M = \E_n + \a\b\).

注意到\(\M = \E_n + \a\b\)形似某个矩阵的舒尔补,因此考虑下面的矩阵:\[
	\A = \begin{bmatrix}
		1 & -\b \\
		\a & \E_n
	\end{bmatrix}.
\]由于\[
	\begin{bmatrix}
		1 & \z \\
		-\a & \E_n
	\end{bmatrix} \A
	= \begin{bmatrix}
		1 & \z \\
		-\a & \E_n
	\end{bmatrix}
	\begin{bmatrix}
		1 & -\b \\
		\a & \E_n
	\end{bmatrix}
	= \begin{bmatrix}
		1 & -\b \\
		\z & \M
	\end{bmatrix},
\]
且\[
	\begin{bmatrix}
		1 & \b \\
		\z & \E_n
	\end{bmatrix} \A
	= \begin{bmatrix}
		1 & \b \\
		\z & \E_n
	\end{bmatrix}
	\begin{bmatrix}
		1 & -\b \\
		\a & \E_n
	\end{bmatrix}
	= \begin{bmatrix}
		1+\a\b & \z \\
		\a & \E_n
	\end{bmatrix},
\]
所以\[
	\abs{\A}
	= \abs{\M}
	= \abs{1+\a\b}
	= 1+\a\b
	= 1 + \sum_{k=1}^n a_k b_k.
\]
\end{solution}
\end{example}

\section{广义逆矩阵}
对于线性方程组\(\A\x=\b\),如果\(\A\)可逆,那么它有\(\x=\A^{-1}\b\).
如果\(\A\)不可逆,但\(\A\x=\b\)有解,那么它的解是否也可表达为类似的简洁公式呢?
我们接下来带着这个问题,开始对\(\A^{-1}\)的性质的分析.

如果\(\A\)可逆,那么\(\A\A^{-1}=\E\).
显然,只要在等式两端同时右乘\(\A\),便得\(\A\A^{-1}\A=\A\).
这就表明:当\(\A\)可逆时,\(\A^{-1}\)是矩阵方程\(\A\x\A=\A\)的一个解.
受此启发,当\(\A\)不可逆时,为了找到\(\A^{-1}\)的替代物,应当去找矩阵方程\(\A\x\A=\A\)的解.

\begin{theorem}[广义逆存在定理]\label{theorem:线性方程组.广义逆1}
设\(\A\)是数域\(K\)上的\(s \times n\)非零矩阵,
则矩阵方程
\begin{equation}\label{equation:线性方程组.广义逆1矩阵方程}
	\A\x\A = \A
\end{equation}一定有解.
如果\(\rank\A=r\),并且\[
\A = \P \begin{bmatrix} \E_r & \z \\ \z & \z \end{bmatrix} \Q,
\]其中\(\P,\Q\)分别是\(K\)上\(s\)阶、\(n\)阶可逆矩阵,
那么矩阵方程 \labelcref{equation:线性方程组.广义逆1矩阵方程} 的通解为
\begin{equation}\label{equation:线性方程组.广义逆1矩阵方程的通解}
	\x = \Q^{-1} \begin{bmatrix} \E_r & \B \\ \C & \D \end{bmatrix} \P^{-1},
\end{equation}
其中\(\B\in M_{r\times(s-r)}(K),
\C\in M_{(n-r)\times r}(K),
\D\in M_{(n-r)\times(s-r)}(K)\).
\end{theorem}

\begin{definition}
设\(\A\)是数域\(K\)上的\(s \times n\)矩阵,
矩阵方程\(\A\x\A = \A\)的每一个解都称为
“\(\A\)的一个\DefineConcept{广义逆矩阵}(generalized inverse)”,
简称“\(\A\)的广义逆”,记作\(\A^-\).
\end{definition}

\begin{property}\label{theorem:线性方程组.广义逆的性质1}
广义逆满足以下性质:
\begin{enumerate}
	\item \(\A\A^-\A=\A\).
	\item \(\A^-\A\A^-=\A^-\).
	\item \((\A\A^-)^H=\A\A^-\).
	\item \((\A^-\A)^H=\A^-\A\).
\end{enumerate}
\end{property}

\begin{property}\label{theorem:线性方程组.广义逆的性质2}
任意一个\(n \times s\)矩阵都是\(\z_{s \times n}\)的广义逆.
\end{property}

\begin{theorem}[非齐次线性方程组的相容性定理]\label{theorem:线性方程组.非齐次线性方程组的相容性定理}
非齐次线性方程组\(\A\x=\b\)有解的充要条件是\(\b=\A\A^-\b\).
\begin{proof}
必要性.设\(\a\)是\(\A\x=\b\)的一个解,则\[
\b = \A\a = (\A\A^-\A)\a = \A\A^-\b.
\]

充分性.设\(\b=\A\A^-\b\),则\(\A^-\b\)是\(\A\x=\b\)的解.
\end{proof}
\end{theorem}

\begin{theorem}[非齐次线性方程组的解的结构定理]\label{theorem:线性方程组.非齐次线性方程组的解的结构定理}
非齐次线性方程组\(\A\x=\b\)有解时,它的通解为\begin{equation}\label{equation:线性方程组.非齐次线性方程组的通解1}
\x=\A^-\b.
\end{equation}
\end{theorem}
从\cref{theorem:线性方程组.非齐次线性方程组的解的结构定理} 看出,任意非齐次线性方程组\(\A\x=\b\)有解时,它的通解有简洁漂亮的形式 \labelcref{equation:线性方程组.非齐次线性方程组的通解1}.

\begin{theorem}[齐次线性方程组的解的结构定理]\label{theorem:线性方程组.齐次线性方程组的解的结构定理}
数域\(K\)上\(n\)元齐次线性方程组\(\A\x=\z\)的通解为\begin{equation}\label{equation:线性方程组.齐次线性方程组的通解}
\x=(\E_n - \A^- \A) \vb{Z},
\end{equation}
其中\(\A^-\)是\(\A\)的任意给定的一个广义逆,
\(\vb{Z}\)取遍\(K^n\)中任意列向量.
\end{theorem}

\begin{corollary}\label{theorem:线性方程组.齐次线性方程组的解的结构定理.推论1}
设数域\(K\)上\(n\)元非齐次线性方程组\(\A\x=\b\)有解,则它的通解为\begin{equation}\label{equation:线性方程组.非齐次线性方程组的通解2}
\x = \A^-\b + (\E_n - \A^- \A) \vb{Z},
\end{equation}
其中\(\A^-\)是\(\A\)的任意给定的一个广义逆,\(\vb{Z}\)取遍\(K^n\)中任意列向量.
\end{corollary}

一般情况下,矩阵方程\(\A\x\A=\A\)的解不唯一,从而\(\A\)的广义逆不唯一.
但是我们有时候希望\(\A\)的满足特殊条件的广义逆是唯一的,这就引出以下概念:
\begin{definition}
设\(\A \in M_{s \times n}(\mathbb{C})\).
关于矩阵\(\x\)的矩阵方程组\begin{equation}\label{equation:线性方程组.彭罗斯方程组}
	\begin{cases}
		\A\x\A=\A, \\
		\x\A\x=\x, \\
		(\A\x)^H = \A\x, \\
		(\x\A)^H = \x\A
	\end{cases}
\end{equation}
称为\(\A\)的\DefineConcept{彭罗斯方程组},
它的解称为\(\A\)的\DefineConcept{穆尔--彭罗斯广义逆},记作\(\A^+\).
%@see: https://mathworld.wolfram.com/Moore-PenroseMatrixInverse.html
\end{definition}

\begin{theorem}[穆尔--彭罗斯广义逆的唯一性]\label{theorem:线性方程组.穆尔--彭罗斯广义逆的唯一性}
如果\(\A \in M_{s \times n}(\mathbb{C})\),则\(\A\)的彭罗斯方程组 \labelcref{equation:线性方程组.彭罗斯方程组} 总是有解,并且它的解唯一.

设\(\A=\B\C\),其中\(\B\)、\(\C\)分别是列满秩矩阵、行满秩矩阵,则方程组 \labelcref{equation:线性方程组.彭罗斯方程组} 的唯一解是
\begin{equation}\label{equation:线性方程组.彭罗斯方程组的唯一解}
\x = \C^H (\C \C^H)^{-1} (\B^H \B)^{-1} \B^H.
\end{equation}
\begin{proof}
首先考虑\(\A\neq\z\).
把\cref{equation:线性方程组.彭罗斯方程组的唯一解} 代入彭罗斯方程组 \labelcref{equation:线性方程组.彭罗斯方程组} 的每一个方程,不难验证每一个方程都将变成恒等式\footnote{由于\(\rank(\C\C^H)=\rank\C=r\),所以\(\C\C^H\)是\(r\)阶满秩矩阵,可逆;同理\(\B^H\B\)也可逆.},由此可知\cref{equation:线性方程组.彭罗斯方程组的唯一解} 的确是彭罗斯方程组 \labelcref{equation:线性方程组.彭罗斯方程组} 的解.

要证明这种广义逆的唯一性,先设\(\X1\)和\(\X2\)都是彭罗斯方程组 \labelcref{equation:线性方程组.彭罗斯方程组} 的解,则\begin{align*}
\X1
&= \X1\A\X1
= \X1(\A\X2\A)\X1
= \X1(\A\X2)(\A\X1)
= \X1(\A\X2)^H(\A\X1)^H \\
&= \X1(\A\X1\A\X2)^H
= \X1(\A\X2)^H
= \X1\A\X2
= \X1(\A\X2\A)\X2 \\
&= (\X1\A)(\X2\A)\X2
= (\X1\A)^H(\X2\A)^H\X2
= (\X2\A\X1\A)^H \X2 \\
&= (\X2\A)^H \X2
= \X2\A\X2
= \X2.
\end{align*}
这就证明了彭罗斯方程组 \labelcref{equation:线性方程组.彭罗斯方程组} 的解的唯一性.

现在考虑\(\A=\z\).
设\(\X0\)是零矩阵\(\z\)的穆尔--彭罗斯广义逆,则\[
\X0 = \X0 \z \X0 = \z.
\]显然\(\z\)是零矩阵的彭罗斯方程组的解,因此零矩阵的穆尔--彭罗斯广义逆是零矩阵本身.

综上,对任意复矩阵\(\A\),它的穆尔--彭罗斯广义逆存在且唯一.
\end{proof}
\end{theorem}


\chapter{向量空间}
\section{向量空间}
为了直接用线性方程组的系数和常数项判断方程组有没有解,有多少解,
我们再前面给出了用系数行列式判断\(n\)个方程的\(n\)元线性方程组有唯一解的充要条件.
这一判定方法只适用于方程数目与未知量数目相等的线性方程组;
而且,当系数行列式等于零时,只能得出方程组无解或有无穷多解的结论,
没有办法区分什么时候无解,什么时候有无穷多解.
对于任意的线性方程组,有没有这样一种判定方法:
直接依据它的系数和常数项,给出它有没有解,有多少解呢?
为此我们需要探讨和建立线性方程组的进一步的理论.
这一理论还将使我们弄清楚线性方程组有无穷多个解时解的结构.

\subsection{向量空间}
设\(K\)是数域,\(n\)是任意给定的一个正整数.
令\[
	K^n \defeq \Set{ (\AutoTuple{a}{n}) \given a_i \in K\ (i=1,2,\dotsc,n) }.
\]

如果\[
	a_i=b_i
	\quad(i=1,2,\dotsc,n),
\]
则称“\(K^n\)中的两个元素\((\AutoTuple{a}{n})\)与\((\AutoTuple{b}{n})\)相等”.

在\(K^n\)中规定“加法”运算如下:
\begin{equation}\label{equation:向量空间.向量的加法.定义式}
	(\AutoTuple{a}{n}) + (\AutoTuple{b}{n})
	\defeq (a_1+b_1,a_2+b_2,\dotsc,a_n+b_n).
\end{equation}

在\(K\)的元素与\(K^n\)的元素之间规定“数量乘法”运算如下:
\begin{equation}\label{equation:向量空间.向量的数量乘法.定义式}
	k (\AutoTuple{a}{n})
	\defeq (k a_1,k a_2,\dotsc,k a_n).
\end{equation}

容易验证,上述加法和数量乘法满足下述8条运算法则:
\begin{enumerate}
	\item 加法交换律,即\((\forall \a,\b \in K^n)[\a+\b=\b+\a]\).

	\item 加法结合律,即\((\forall \a,\b,\g \in K^n)[(\a+\b)+\g=\a+(\b+\g)]\).

	\item 记\(\z=(0,0,\dotsc,0)\),\[
		(\forall\a \in K^n)[\a+\z = \z+\a = \a].
	\]
	称\(\z\)为“\(K^n\)的\DefineConcept{零元}(zero element)”.

	\item \(\forall\a=(\AutoTuple{a}{n}) \in K^n\),令\[
		-\a \defeq (\AutoTuple{-a}{n}),
	\]
	则\(-\a \in K^n\)且\[
		\a+-\a
		= -\a+\a
		= \z;
	\]
	称\(-\a\)为“\(\a\)的\DefineConcept{负元}(negative element)”.

	\item \((\forall \a \in K^n)[1 \a=\a]\).

	\item \((\forall \a \in K^n)(\forall k,l \in K)[k (l \a)=(kl) \a]\).

	\item \((\forall \a \in K^n)(\forall k,l \in K)[(k+l) \a=k \a+l \a]\).

	\item \((\forall \a,\b \in K^n)(\forall k \in K)[k (\a+\b)=k \a+k \b]\).
\end{enumerate}

\begin{definition}
数域\(K\)上全体\(n\)元组组成的集合\(K^n\),
连同定义在它上面的加法运算和数量乘法运算,
及其满足的8条运算法则一起,
称为“数域\(K\)上的一个\(n\)维\DefineConcept{向量空间}(vector space)”.
\(K^n\)的元素称为“\(n\)维\DefineConcept{向量}(vector)”.

对于\(K^n\)中的任意一个向量\(\a=(\AutoTuple{a}{n})\),
称数\[
	a_i\quad(i=1,2,\dotsc,n)
\]为“\(\a\)的第\(i\)个\DefineConcept{分量}”.
\end{definition}

在\(n\)维向量空间\(K^n\)中,我们可以额外定义减法运算如下:
\begin{equation}\label{equation:向量空间.向量的减法.定义式}
	\a-\b \defeq \a+(-\b).
\end{equation}

在\(n\)维向量空间\(K^n\)中,容易验证下述4条性质:
\begin{property}
\((\forall\a \in K^n)[0\cdot\a=\z]\).
\end{property}

\begin{property}
\((\forall\a \in K^n)[(-1)\cdot\a=-\a]\).
\end{property}

\begin{property}
\((\forall k \in K)[k\z=\z]\).
\end{property}

\begin{property}
\(k\a=\z \implies k=0 \lor \a=\z\).
\end{property}

把\(n\)元组写成一行,得\[
	(\AutoTuple{a}{n})
	\quad\text{或}\quad
	\begin{bmatrix}
		a_1 & a_2 & \dots & a_n
	\end{bmatrix},
\]
称之为“\(n\)维\DefineConcept{行向量}(row vector)”.

把\(n\)元组写成一列,得\[
	\begin{bmatrix} a_1 \\ a_2 \\ \vdots \\ a_n \end{bmatrix},
\]
称之为“\(n\)维\DefineConcept{列向量}(column vector)”;
不过,我们有时候会为了方便排版,把列向量写成\[
	(a_1,a_2,\dotsc,a_n)^T.
\]

\(K^n\)可以看成是\(n\)维行向量组成的向量空间,也可以看成是\(n\)维列向量组成的向量空间.
两者并没有本质的区别,只是它们的元素的写法不同而已.


由有限个\(n\)维行向量构成的集合,称为“\(n\)维\DefineConcept{行向量组}”.
由有限个\(n\)维列向量构成的集合,称为“\(n\)维\DefineConcept{列向量组}”.
\(n\)维行向量组和\(n\)维列向量组统称\(n\)维\DefineConcept{向量组},是\(n\)维向量空间的子集.

称满足
\[
	e_{ij} = \left\{ \begin{array}{ll}
		1, & i=j, \\
		0, & i \neq j
	\end{array} \right.
\]
的向量组
\[
	\e_i = \begin{bmatrix}
		e_{1i} \\ e_{2i} \\ \vdots \\ e_{ni}
	\end{bmatrix}
	\quad(i=1,2,\dotsc,n)
\]为“\(K^n\)的\DefineConcept{基本向量组}”.

\subsection{线性组合,线性表出}
%@see: 《线性代数》(张慎语、周厚隆) P67 定义5
\begin{definition}\label{definition:向量空间.线性组合}
在\(K^n\)中,给定向量组\(\AutoTuple{\a}{s}\),
任给\(K\)中一组数\(\AutoTuple{k}{s}\),
我们把\[
	k_1 \a_1 + k_2 \a_2 + \dotsb + k_s \a_s
\]
称为“向量组\(\AutoTuple{\a}{s}\)的一个\DefineConcept{线性组合}(linear combination)”,
把\(\AutoTuple{k}{s}\)称为\DefineConcept{系数}.
\end{definition}

\begin{definition}\label{definition:向量空间.线性表出1}
对于\(\b \in K^n\),
如果存在\(K\)中一组数\(\AutoTuple{c}{s}\),使得\[
	\b = c_1 \a_1 + c_2 \a_2 + \dotsb + c_s \a_s,
\]
则称“\(\b\)可以由\(\AutoTuple{\a}{s}\) \DefineConcept{线性表出}”.
\end{definition}

现在,利用向量的加法运算和数量乘法运算,
我们可以把数域\(K\)上\(n\)元线性方程组 \labelcref{equation:线性方程组.线性方程组的代数形式}
写成
\begin{equation}\label{equation:线性方程组.线性方程组的向量形式}
	x_1 \a_1 + x_2 \a_2 + \dotsb + x_n \a_n = \b,
\end{equation}
其中\[
	\a_j=(a_{1j},a_{2j},\dotsc,a_{sj})^T,
	\quad
	j=1,2,\dotsc,n.
\]
于是,\begin{align*}
	&\text{数域\(K\)上线性方程组\(x_1 \a_1 + x_2 \a_2 + \dotsb + x_n \a_n = \b\)有解} \\
	&\iff \text{\(K\)中存在一组数\(\AutoTuple{c}{n}\),使得\(c_1 \a_1 + c_2 \a_2 + \dotsb + c_n \a_n = \b\)成立} \\
	&\iff \text{\(\b\)可以由\(\AutoTuple{\a}{n}\)线性表出}.
\end{align*}
这样我们把线性方程组有没有解的问题归结为:
常数项列向量\(\b\)能不能由系数矩阵的列向量组线性表出.
这个结论有两方面的意义:
一方面,为了从理论上研究线性方程组有没有解,
就需要去研究\(\b\)能否由\(\AutoTuple{\a}{n}\)线性表出;
另一方面,对于\(K^n\)中给定的向量组\(\AutoTuple{\a}{n}\),
以及给定的\(\b\),
为了判断\(\b\)能否由\(\AutoTuple{\a}{n}\)线性表出,
就可以去判断线性方程组\(x_1 \a_1 + x_2 \a_2 + \dotsb + x_n \a_n = \b\)是否有解.

\subsection{线性子空间}
在\(K^n\)中,从理论上如何判断任一向量\(\b\)能否由向量组\(\AutoTuple{\a}{n}\)线性表出?
从线性表出的定义知道,这需要考察\(\b\)是否等于\(\AutoTuple{\a}{n}\)的某一个线性组合.
为此,我们把\(\AutoTuple{\a}{n}\)的所有线性组合组成一个集合\(W\),即\[
	W \defeq \Set{ k_1 \a_1 + k_2 \a_2 + \dotsb + k_s \a_s \given k_i \in K, i=1,2,\dotsc,s }.
\]
如果我们能够把\(W\)的结构研究清楚,那么就比较容易判断\(\b\)是否属于\(W\),
也就是判断\(\b\)能否由\(\AutoTuple{\a}{n}\)线性表出.

现在我们来研究\(W\)的结构.
任取\(\a,\g\in W\),设\[
	\a=a_1\a_1+a_2\a_2+\dotsb+a_s\a_s, \qquad
	\g=b_1\a_1+b_2\a_2+\dotsb+b_s\a_s,
\]
则\begin{align*}
	\a+\g
	&=(a_1\a_1+a_2\a_2+\dotsb+a_s\a_s)+(b_1\a_1+b_2\a_2+\dotsb+b_s\a_s) \\
	&=(a_1+b_1)\a_1+(a_2+b_2)\a_2+\dotsb+(a_s+b_s)\a_s,
\end{align*}
从而\(\a+\g\in W\).

再任取\(k\in W\),则\begin{align*}
	k\a
	&=k(a_1\a_1+a_2\a_2+\dotsb+a_s\a_s) \\
	&=(ka_1)\a_1+(ka_2)\a_2+\dotsb+(ka_s)\a_s,
\end{align*}
从而\(k\a\in W\).

受此启发,我们引出如下概念.
\begin{definition}
\(K^n\)的一个非空子集\(U\)如果满足:
\begin{enumerate}
	\item \(U\)对\(K^n\)的加法封闭,即\[
		\a,\b \in U \implies \a+\b \in U;
	\]
	\item \(U\)对\(K^n\)的数量乘法封闭,即\[
		\a \in U, k \in K \implies k\a \in U;
	\]
\end{enumerate}
那么称\(U\)是“\(K^n\)的一个\DefineConcept{线性子空间}(linear subspace)”,
简称为\DefineConcept{子空间}(subspace).
\end{definition}
\(\{\z\}\)是\(K^n\)的一个子空间,称之为\DefineConcept{零子空间}(zero subspace).
\(K^n\)也是其自身的一个子空间.

从上面的讨论知道,在\(K^n\)中,
向量组\(\AutoTuple{\a}{s}\)的所有线性组合组成的集合\(W\)是\(K^n\)的一个子空间,
称它为“\(\AutoTuple{\a}{s}\)生成的子空间”,
记作\[
	\opair{\AutoTuple{\a}{s}}.
\]

于是,我们得出结论,以下三个命题等价:
\begin{enumerate}
	\item 数域\(K\)上的\(n\)元线性方程组\(x_1 \a_1 + x_2 \a_2 + \dotsb + x_n \a_n = \b\)有解.
	\item \(\b\)可以由\(\AutoTuple{\a}{n}\)线性表出.
	\item \(\b\in\opair{\AutoTuple{\a}{n}}\).
\end{enumerate}

\begin{example}
设\(1 \leqslant r < n\).
证明:集合\[
	U = \Set{ (\AutoTuple{a}{r},0,\dotsc,0) \given a_i \in K, i=1,2,\dotsc,r }
\]是\(K^n\)的子空间.
\begin{proof}
在\(K\)中任取一个数\(k\),%
再在\(U\)中任取两个向量
\[
	\a = (\AutoTuple{a}{r},0,\dotsc,0),
	\qquad
	\b = (\AutoTuple{b}{r},0,\dotsc,0),
\]
有\[
	\a+\b = (a_1+b_1,a_2+b_2,\dotsc,a_r+b_r,0,\dotsc,0) \in U,
\]\[
	k \a = (k a_1,k a_2,\dotsc,k a_r,0,\dotsc,0) \in U,
\]
因此\(U\)是\(K^n\)的一个子空间.
\end{proof}
\end{example}

%\begin{theorem}
%\(K^n\)中任一向量都可由基本向量组唯一地线性表出.
%\begin{proof}
%对于任意一个向量\(\a=(\AutoTuple{a}{n})^T\),%
%线性方程组\(x_1 \e_1 + x_2 \e_2 + \dotsb + x_n \e_n = \a\)的系数行列式为
%\[
%\begin{vmatrix}
%	1 & 0 & \dots & 0 \\
%	0 & 1 & \dots & 0 \\
%	\vdots & \vdots & & \vdots \\
%	0 & 0 & \dots & 1
%\end{vmatrix}
%= 1 \neq 0,
%\]
%那么,根据克拉默法则,这个线性方程组有唯一解,%
%于是\(K^n\)中任一向量\(\a\)都能由基本向量组线性表出,且表出方式唯一.
%事实上,由于
%\[
%a_1 \begin{bmatrix}
%1 \\ 0 \\ 0 \\ \vdots \\ 0
%\end{bmatrix}
%+ a_2 \begin{bmatrix}
%0 \\ 1 \\ 0 \\ \vdots \\ 0
%\end{bmatrix}
%+ \dotsb + a_n \begin{bmatrix}
%0 \\ 0 \\ 0 \\ \vdots \\ 1
%\end{bmatrix}
%= \begin{bmatrix}
%a_1 \\ a_2 \\ a_3 \\ \vdots \\ a_n
%\end{bmatrix},
%\]
%因此,用基本向量组标出向量\(\a\)的方式为
%\[
%\a = a_1 \e_1 + a_2 \e_2 + \dotsb + a_n \e_n.
%\qedhere
%\]
%\end{proof}
%\end{theorem}

\section{向量组的线性相关性}
在上一节,我们把线性方程组有没有解的问题归结为:
常数项列向量\(\b\)能否由系数矩阵的列向量组\(\AutoTuple{\a}{s}\)线性表出.
那么,如何研究\(K^n\)中一个向量能不能由一个向量组线性表出呢?

\subsection{线性相关性的概念}
我们首先回顾\cref{theorem:解析几何.两向量共线的充要条件1,%
theorem:解析几何.三向量共面的充要条件1},
以及\cref{theorem:解析几何.两向量不共线的充要条件1,%
theorem:解析几何.三向量不共面的充要条件1}.

受此启发,我们提出以下两个概念.
\begin{definition}\label{definition:线性方程组.线性相关与线性无关的定义}
%@see: 《线性代数》(张慎语、周厚隆) P68 定义6
设\(A=\Set{\AutoTuple{\a}{s}}\)是\(n\)维向量空间\(K^n\)中的一个向量组.

如果\(K\)中存在不全为零的数\(\AutoTuple{k}{s}\),使得\[
	k_1 \a_1 + k_2 \a_2 + \dotsb + k_s \a_s = \z,
\]
则称“向量组\(A\) \DefineConcept{线性相关}(linearly dependent)”;
否则,称“向量组\(A\) \DefineConcept{线性无关}(linearly independent)”.
%@see: https://mathworld.wolfram.com/LinearlyIndependent.html
\end{definition}

显然,从\cref{definition:线性方程组.线性相关与线性无关的定义} 立即可得
\[
	\text{向量组\(A\)线性无关}
	\iff
	[k_1 \a_1 + k_2 \a_2 + \dotsb + k_s \a_s = \z
	\implies
	(\AutoTuple{k}{s}) = \z].
\]

特别地,我们规定:空集\(\emptyset\)线性无关.

\subsection{线性相关性的判定条件}
根据线性相关、线性无关的定义和解析几何的结论,
在几何空间中,共线的两个向量是线性相关的,
共面的三个向量是线性相关的,
不共面的三个向量是线性无关的,
不共线的两个向量是线性无关的.

下面我们再来看几个例子.
\begin{example}\label{example:线性方程组.含有零向量的向量组线性相关}
%@see: 《线性代数》(张慎语、周厚隆) P68 例1
向量空间\(K^n\)中的零向量可以由任意向量组\(\AutoTuple{\b}{t}\)线性表出,
这是因为恒等式\[
	0\b_1+0\b_2+\dotsb+0\b_t=\z.
\]
进一步,
含有零向量\(\z\)的向量组\[
	\Set{\z,\AutoTuple{\a}{s}}
\]总是线性相关的,
这是因为\[
	1 \z + 0 \a_1 + 0 \a_2 + \dotsb + 0 \a_s = \z.
\]
\end{example}

\begin{example}\label{example:线性方程组.基本向量组线性无关}
%@see: 《线性代数》(张慎语、周厚隆) P68 例2
\(K^n\)的基本向量组\(\AutoTuple{\e}{n}\)线性无关.
\begin{proof}
令\(k_1 \e_1 + k_2 \e_2 + \dotsb + k_n \e_n = \z\),即\[
	k_1 (1,0,\dotsc,0)^T + k_2 (0,1,\dotsc,0)^T + \dotsb k_n (0,0,\dotsc,1)^T = \z.
\]
进一步,有\[
	(\AutoTuple{k}{n})^T = (0,\dotsc,0)^T,
\]
于是\(k_1 = k_2 = \dotsb = k_n = 0\),
因此\(\e_1,\e_2,\dotsc,\e_n\)线性无关.
\end{proof}
\end{example}

\begin{example}\label{example:线性方程组.单向量组线性相关的充要条件}
证明:一个向量\(\a\)组成的向量组\(\{\a\}\)线性相关的充要条件是\(\a=\z\).
\begin{proof}
必要性.
设\(\{\a\}\)线性相关,存在数\(k \neq 0\)使得\(k\a = \z\),可得\(\a = \z\).

充分性.
设\(\a = \z\),则\(1\a = \z\),而数\(1 \neq 0\),故\(\{\a\}\)线性相关.
\end{proof}
\end{example}
我们还可以给出逆否命题:\(\text{向量组\(\{\a\}\)线性无关} \iff \a\neq\z\).

\begin{theorem}\label{theorem:线性方程组.向量组线性相关的充要条件1}
向量组\(A=\{\AutoTuple{\a}{s}\}\ (s>1)\)线性相关的充要条件是:
\(A\)中至少有一个向量可由其余\(s-1\)个向量线性表出,即\[
	(\exists \a\in A)[\a \in \Span(A-\{\a\})].
\]
\begin{proof}
必要性.
\(A\)线性相关,则存在不全为零的数\(\AutoTuple{k}{s}\),使得\[
	k_1 \a_1 + k_2 \a_2 + \dotsb + k_s \a_s = \z.
\]
设\(k_i\neq0\ (1 \leqslant i \leqslant s)\),于是\[
	\a_i = -\frac{1}{k_i} (
		k_1 \a_1 + k_2 \a_2 + \dotsb
		+ k_{i-1} \a_{i-1} + k_{i+1} \a_{i+1}
		+ \dotsb + k_s \a_s
	),
\]
即\(\a_i\)可由其余\(s-1\)个向量线性表出.

充分性.
若\(\a_j \in A\)可由其余\(s-1\)个向量线性表出,即\[
	\a_j = l_1 \a_1 + \dotsb + l_{j-1} \a_{j-1} + l_{j+1} \a_{j+1} + \dotsb + l_s \a_s,
\]
移项得\[
	l_1 \a_1 + \dotsb
	+ l_{j-1} \a_{j-1} + (-1) \a_j + l_{j+1} \a_{j+1}
	+ \dotsb + l_s \a_s = \z,
\]
上式等号左边的系数中至少有一个数\(-1\neq0\),
因此\(A\)线性相关.
\end{proof}
\end{theorem}

根据\cref{theorem:线性方程组.向量组线性相关的充要条件1},我们立即有它的逆否命题成立:
\begin{corollary}
向量组\(A=\{\AutoTuple{\a}{s}\}\ (s>1)\)线性无关的充要条件是:
\(A\)中每一个向量都不能由其余向量线性表出.
\end{corollary}

\begin{example}
%@see: 《线性代数》(张慎语、周厚隆) P68 例4
设向量组\(A=\{\AutoTuple{\a}{s}\}\)线性无关,
\(B=\{\AutoTuple{\a}{s},\b\}\)线性相关.
证明:\(\b\)可由\(A\)线性表出.
\begin{proof}
由于向量组\(B\)线性相关,
则存在不全为零的数\(\AutoTuple{k}{s},k\)使得\[
	k_1 \a_1 + k_2 \a_2 + \dotsb + k_s \a_s + k \b = \z.
\]
假设\(k = 0\),
则\(\AutoTuple{k}{s}\)不全为零,
且有\(k_1 \a_1 + k_2 \a_2 + \dotsb + k_s \a_s = \z\),
即\(A\)线性相关,
与题设矛盾,说明\(k \neq 0\).
于是\[
	\b = -\frac{1}{k} (k_1 \a_1 + k_2 \a_2 + \dotsb + k_s \a_s).
	\qedhere
\]
\end{proof}
利用\cref{theorem:线性方程组.向量组线性相关的充要条件1}
可以证明,“\(\b\)可由\(A\)线性表出”蕴含“\(B\)线性相关”.
因此,“\(\b\)可由\(A\)线性表出”是“\(B\)线性相关”成立的充要条件.

再则我们还能得到如下结论:
设\(A\)线性无关,
则“\(\b\)不能由\(A\)线性表出”的充要条件是“\(B\)线性无关”.
\end{example}

\begin{theorem}\label{theorem:线性方程组.部分组线性相关则全组线性相关}
若向量组\(A=\{\AutoTuple{\a}{s}\}\)的一个部分组线性相关,则\(A\)线性相关.
\begin{proof}
假设\(A\)的部分组\(B=\{\AutoTuple{\a}{t}\}\ (t \leqslant s)\)线性相关,
即存在不全为零的数\(\AutoTuple{k}{t}\)使得\[
	k_1 \a_1 + k_2 \a_2 + \dotsb + k_t \a_t = \z;
\]
从而有\[
	k_1 \a_1 + k_2 \a_2 + \dotsb + k_t \a_t + 0 \a_{t+1} + \dotsb + 0 \a_s = \z;
\]
由于上式等号左边的系数\(\AutoTuple{k}{t},0,\dotsc,0\)不全为零,
因此向量组\(A\)线性相关.
\end{proof}
\end{theorem}

由\cref{theorem:线性方程组.部分组线性相关则全组线性相关} 立即得到:
\begin{corollary}\label{theorem:线性方程组.全组线性无关则任一部分组线性无关}
如果向量组\(A\)线性无关,
那么\(A\)的任意一个部分组也线性无关.
\end{corollary}

给定\(n\)维向量组\(\AutoTuple{\a}{s}\),
为其中的每个向量都添上\(m\)个分量,
所添分量的位置对于每个向量都一样,
把得到的\(n+m\)维向量组\(\AutoTuple{\b}{s}\)称为%
“\(\AutoTuple{\a}{s}\)的\DefineConcept{延伸组}”;
反过来,把\(\AutoTuple{\a}{s}\)称为%
“\(\AutoTuple{\b}{s}\)的\DefineConcept{缩短组}”.

如果向量组线性无关,那么它的延伸组也线性无关.
如果向量组线性相关,那么它的缩短组也线性相关.

\begin{theorem}
\(n\)个\(n\)维列向量\(\AutoTuple{\a}{n}\)线性相关的充要条件是:\[
	\det(\AutoTuple{\a}{n})=0.
\]
\end{theorem}

\begin{corollary}
\(n\)个\(n\)维列向量\(\AutoTuple{\a}{n}\)线性无关的充要条件是:\[
	\det(\AutoTuple{\a}{n})\neq0.
\]
\end{corollary}

需要注意的是,当\(s \neq n\)时,
\(s\)个\(n\)维向量\(\AutoTuple{\a}{s}\)不能构成行列式,
只能用其他方法判断其线性相关性.



\begin{theorem}[替换定理]
设向量组\(\AutoTuple{\a}{s}\)线性无关,
\(\b=b_1\a_1+\dotsb+b_s\a_s\).
如果\(b_j\neq0\),
那么用\(\b\)替换\(\a_j\)以后得到的向量组
\(\AutoTuple{\a}{j-1},\b,\AutoTuple{\a}[j+1]{s}\)
也线性无关.
\end{theorem}

\begin{example}
设\(\A\)是3阶矩阵,\(\a_1,\a_2,\a_3\)为3维列向量组,
若\(\A\a_1,\A\a_2,\A\a_3\)线性无关,
证明:\(\a_1,\a_2,\a_3\)线性无关,且\(\A\)为可逆矩阵.
\begin{proof}
因为\(\A\a_1,\A\a_2,\A\a_3\)线性无关,所以齐次线性方程组\[
	x_1 \A\a_1 + x_2 \A\a_2 + x_3 \A\a_3
	= (\A\a_1,\A\a_2,\A\a_3) (x_1,x_2,x_3)^T
	= \z
\]
只有零解(即\(x_1 = x_2 = x_3 = 0\)),根据克拉默法则,有\[
	\det(\A\a_1,\A\a_2,\A\a_3) \neq 0.
\]

因为\(\det(\A\a_1,\A\a_2,\A\a_3) = \abs{\A} \cdot \det(\a_1,\a_2,\a_3) \neq 0\),
所以\(\abs{\A} \neq 0\)(即\(\A\)是可逆矩阵),
且\(\det(\a_1,\a_2,\a_3) \neq 0\)%
(即齐次线性方程组\(x_1 \a_1 + x_2 \a_2 + x_3 \a_3 = \z\)只有零解,
向量组\(\a_1,\a_2,\a_3\)线性无关).
\end{proof}
\end{example}

%\begin{example}
%证明:\(\mathbb{R}^n\)中的任意正交组线性无关.
%\begin{proof}
%设\(A=\{\AutoTuple{\a}{m}\}\)是\(\mathbb{R}^n\)的一个正交组,令\[
%	k_1 \a_1 + k_2 \a_2 + \dotsb + k_m \a_m = \z,
%\]
%两端分别与\(\a_1\)作内积,
%即\[
%	\vectorinnerproduct{(k_1 \a_1 + k_2 \a_2 + \dotsb + k_m \a_m)}{\a_1}
%	= \vectorinnerproduct{\z}{\a_1};
%\]
%由内积性质,\[
%	k_1 (\vectorinnerproduct{\a_1}{\a_1})
%	+ k_2 (\vectorinnerproduct{\a_2}{\a_1})
%	+ \dotsb
%	+ k_m (\vectorinnerproduct{\a_m}{\a_1})
%	= \vectorinnerproduct{\z}{\a_1},
%\]
%其中\(\vectorinnerproduct{\z}{\a_1} = 0\),
%\(\vectorinnerproduct{\a_j}{\a_1} = 0\ (j=2,3,\dotsc,m)\),
%故\(k_1 \vectorinnerproduct{\a_1}{\a_1} = 0\),
%而\(\vectorinnerproduct{\a_1}{\a_1} > 0\),
%所以\(k_1=0\).
%同理可得\(k_2=k_3=\dotsb=k_m=0\),从而\(A\)线性无关.
%\end{proof}
%\end{example}

\section{向量组的秩}
\subsection{向量组的等价关系}
\begin{definition}
在\(K^n\)中,如果向量组\[
	A=\{\AutoTuple{\a}{s}\}
\]的每个向量都可由向量组\[
	B=\{\AutoTuple{\b}{t}\}
\]线性表出,
即\[
	(\forall \a \in A)[\a \in \opair{\AutoTuple{\b}{t}}],
\]
则称“\(A\)可由\(B\) \DefineConcept{线性表出}”.
\end{definition}

\begin{theorem}
非空向量组\(A\)总可由它本身线性表出.
\begin{proof}
设\(A=\{\AutoTuple{\a}{s}\}\),
显然\[
	\a_i=0\a_1+\dotsb+0\a_{i-1}+1\a_i+0\a_{i+1}+\dotsb+0\a_s,
	\quad i=1,2,\dotsc,s;
\]
这就是说\(\a_i\ (i=1,2,\dotsc,s)\)可由\(A\)线性表出,即\[
	\a_i\in\opair{\AutoTuple{\a}{s}},
	\quad i=1,2,\dotsc,s.
\]
于是\(A\)可由\(A\)线性表出,
即\(\{\AutoTuple{\a}{s}\}\subseteq\opair{\AutoTuple{\a}{s}}\).
\end{proof}
\end{theorem}

\begin{theorem}
如果向量组\(A\)可由\(B\)线性表出,
那么\(\opair{\AutoTuple{\a}{s}}\)是\(\opair{\AutoTuple{\b}{t}}\)的子集.
\begin{proof}
任意取定\(\a\in\opair{\AutoTuple{\a}{s}}\),
不妨设\(k_i\in K\ (i=1,2,\dotsc,s)\)满足\[
	\a = \sum\limits_{i=1}^s k_i \a_i;
\]
又设\(l_{ij}\in K\ (i=1,2,\dotsc,s;j=1,2,\dotsc,t)\)满足\[
	\a_i = \sum\limits_{j=1}^t l_{ij} \b_j,
	\quad i=1,2,\dotsc,s;
\]
那么\[
	\a = \sum\limits_{i=1}^s k_i \left(
		\sum\limits_{j=1}^t l_{ij} \b_j
	\right)
	= \sum\limits_{j=1}^t \left(
		\sum\limits_{i=1}^s k_i l_{ij}
	\right) \b_j,
\]
这就是说\(\a\in\opair{\AutoTuple{\b}{t}}\),
于是\(\opair{\AutoTuple{\a}{s}}\subseteq\opair{\AutoTuple{\b}{t}}\).
\end{proof}
\end{theorem}

\begin{theorem}
如果\(\opair{\AutoTuple{\a}{s}}\subseteq\opair{\AutoTuple{\b}{t}}\),
那么向量组\(\AutoTuple{\a}{s}\)可由\(\AutoTuple{\b}{t}\)线性表出.
\begin{proof}
因为\(\opair{\AutoTuple{\a}{s}}\subseteq\opair{\AutoTuple{\b}{t}}\),
对于任意\(\a\in\opair{\AutoTuple{\a}{s}}\),总有\(\a\in\opair{\AutoTuple{\b}{t}}\);
又因为\(\AutoTuple{\a}{s}\in\opair{\AutoTuple{\a}{s}}\),
所以\(\AutoTuple{\a}{s}\in\opair{\AutoTuple{\b}{t}}\).
\end{proof}
\end{theorem}

于是,我们可以说:\[
	\text{向量组\(\AutoTuple{\a}{s}\)可由\(\AutoTuple{\b}{t}\)线性表出}
	\iff
	\opair{\AutoTuple{\a}{s}}\subseteq\opair{\AutoTuple{\b}{t}}.
\]
从而“线性表出”和集合的“包含”关系一样,具有自反性和传递性:
\begin{enumerate}
	\item \(\opair{\AutoTuple{\a}{s}}\subseteq\opair{\AutoTuple{\a}{s}}\).
	\item \(\opair{\AutoTuple{\a}{s}}\subseteq\opair{\AutoTuple{\b}{r}}\subseteq\opair{\AutoTuple{\g}{m}}
	\implies\opair{\AutoTuple{\a}{s}}\subseteq\opair{\AutoTuple{\g}{m}}\).
\end{enumerate}
具体来说,
假设向量组\(A=\Set{\AutoTuple{\a}{s}}\)可以由向量组\(B=\Set{\AutoTuple{\b}{r}}\)线性表出,
且\(B\)可以由向量组\(C=\Set{\AutoTuple{\g}{m}}\)线性表出.
在向量组\(A\)中任取一个向量\(\a_i\),则\[
	\a_i = \sum\limits_{j=1}^r k_j \b_j.
\]
又由于\(\b_j\)可以由向量组\(C\)线性表出,因此\[
	\b_j = \sum\limits_{t=1}^m l_{jt} \g_t,
	\quad j=1,\dotsc,r.
\]
从而\[
	\a_i = \sum\limits_{j=1}^r k_j \b_j
	= \sum\limits_{j=1}^r k_j \left(
		\sum\limits_{t=1}^m l_{jt} \g_t
	\right)
	= \sum\limits_{t=1}^m \left(
		\sum\limits_{j=1}^r k_j l_{jt}
	\right) \g_t.
\]
于是\(\a_i\)可以由\(C\)线性表出,
从而\(A\)可以由\(C\)线性表出.


\begin{definition}
如果\(A\)与\(B\)可以相互线性表出,
即\[
	\opair{\AutoTuple{\a}{s}}\subseteq\opair{\AutoTuple{\b}{t}}
	\land
	\opair{\AutoTuple{\b}{t}}\subseteq\opair{\AutoTuple{\a}{s}},
\]
则称\(A\)与\(B\) \DefineConcept{等价},
记作\(A \cong B\).
\end{definition}

“向量组的等价”是向量组之间的一种等价关系,
这是因为对于任意向量组\(A,B,C\subseteq K^n\)来说,
\begin{enumerate}
	\item 它具有自反性,即\(A \cong A\);
	\item 它具有对称性,即\(A \cong B \implies B \cong A\);
	\item 它具有传递性,即\(A \cong B \land B \cong C \implies A \cong C\).
\end{enumerate}

\begin{theorem}\label{theorem:线性方程组.部分组可由全组线性表出}
部分组可由全组线性表出.
\begin{proof}
设数域\(K\)上一个向量组\(A=\{\AutoTuple{\a}{s}\}\),
从中任取\(t\ (t \leqslant s)\)个向量组成向量组\[
	B=\{\AutoTuple{\a}{t}\}.
\]
欲证部分组可由全组线性表出,
即证\(\forall \a_j \in B\),
\(\exists \AutoTuple{k}{j},\dotsc,k_s \in P\),
使得\[
	\a_j = k_1 \a_1 + k_2 \a_2 + \dotsb + k_j \a_j + \dotsb + k_s \a_s.
\]
显然只要取\[
	k_i = \left\{ \begin{array}{cl}
		1, & i=j, \\
		0, & i \neq j,
	\end{array} \right.
\]
便可令上式成立.
\end{proof}
\end{theorem}

\begin{theorem}
\(\alpha=\{\AutoTuple{\a}{s}\}\ (s>1)\)%
线性相关的充要条件是:
\(\alpha\)可由某个部分组%
\[\a_1,\dotsc,\a_{i-1},\a_{i+1},\dotsc,\a_s\]
线性表出.
\end{theorem}

\begin{theorem}
设向量组\(A=\{\AutoTuple{\a}{s}\}\)可由%
\(B=\{\AutoTuple{\b}{t}\}\)线性表出;
如果\(s>t\),则\(A\)线性相关.
\begin{proof}
欲证\(A\)线性相关,须找到不全为零的\(s\)个数\(\AutoTuple{k}{s}\)使得\[
	k_1 \a_1 + k_2 \a_2 + \dotsb + k_s \a_s = \z.
\]
因为向量组\(A\)可由\(B\)线性表出,即有\[
	\left\{ \begin{array}{l}
		\a_1 = c_{11} \b_1 + c_{21} \b_2 + \dotsb + c_{t1} \b_t, \\
		\a_2 = c_{12} \b_1 + c_{22} \b_2 + \dotsb + c_{t2} \b_t, \\
		\hdotsfor{1} \\
		\a_s = c_{1s} \b_1 + c_{2s} \b_2 + \dotsb + c_{ts} \b_t.
	\end{array} \right.
\]代入可得\[
	\sum\limits_{j=1}^s k_j \a_j
	=\sum\limits_{j=1}^s k_j \sum\limits_{i=1}^t c_{ij} \b_i
	=\sum\limits_{j=1}^s \sum\limits_{i=1}^t k_j c_{ij} \b_i
	=\sum\limits_{i=1}^t \b_i \sum\limits_{j=1}^s k_j c_{ij}
	=\z.
\]
如此只需证存在不全为零的\(s\)个数\(\AutoTuple{k}{s}\)
使得对于任意\(i=1,2,\dotsc,t\)都有\[
	\sum\limits_{j=1}^s k_j c_{ij} = 0.
\]
而关于\(k_i\ (i=1,2,\dotsc,s)\)的齐次线性方程组
\[
	\left\{ \begin{array}{l}
		c_{11} k_1 + c_{12} k_2 + \dotsb + c_{1s} k_s = 0, \\
		c_{21} k_1 + c_{22} k_2 + \dotsb + c_{2s} k_s = 0, \\
		\hdotsfor{1} \\
		c_{t1} k_1 + c_{t2} k_2 + \dotsb + c_{ts} k_s = 0.
	\end{array} \right.
\]中方程数\(t\)小于未知量个数\(s\),必有非零解.
\end{proof}
\end{theorem}

\begin{corollary}
任意\(n+1\)个\(n\)维向量线性相关.
换言之,向量个数大于维数的向量组线性相关.
\begin{proof}
\(P^n\)中任意\(n+1\)个\(n\)维向量\(\alpha = \{ \AutoTuple{\a}{n+1} \}\)
可由基本向量组\(\AutoTuple{\e}{n}\)线性表出,
这两个向量组中的向量个数满足\(n+1 > n\),向量组\(\alpha\)线性相关.
\end{proof}
\end{corollary}

\begin{corollary}
若线性无关向量组\[
\alpha=\{\AutoTuple{\a}{s}\}
\]可由向量组\[
\beta=\{\b_1,\b_2,\dotsc,\b_t\}
\]线性表出,则\(s \leqslant t\).
\begin{proof}
假设\(s > t\),因为向量组\(\alpha\)可由\(\beta\)线性表出,所以向量组\(\alpha\)线性相关,矛盾,故\(s \leqslant t\).
\end{proof}
\end{corollary}

\begin{corollary}
两个等价的线性无关向量组含有相同的向量个数.
\begin{proof}
设\(A=\{\AutoTuple{\a}{s}\}\)
与\(B=\{\AutoTuple{\b}{t}\}\)
都线性无关,且\(A \cong B\).
因为\(A \cong B\),%
所以\(A\)可由\(B\)线性表出,%
从而\(s \leqslant t\);
同理可得\(t \leqslant s\);
综上所述,\(s = t\).
\end{proof}
\end{corollary}

\begin{example}
在数域\(K\)上,满足\[
\abs{a_{ii}} > \sum\limits_{\substack{1 \leqslant j \leqslant n \\ j \neq i}} \abs{a_{ij}}
\quad (i=1,2,\dotsc,n)
\]的\(n\)阶矩阵\(\A = (a_{ij})_n\)称为\DefineConcept{主对角占优矩阵}.
证明:\(\A\)的列向量组\(\AutoTuple{\a}{n}\)的秩等于\(n\).
\begin{proof}
假设\(\AutoTuple{\a}{n}\)线性相关,则在\(K\)中有一组不全为0的数\(\AutoTuple{k}{n}\),使得\[
k_1 \a_1 + k_2 \a_2 + \dotsb + k_n \a_n = \z.
\]不妨设\(\abs{k_l} = \max\{\abs{k_1},\abs{k_2},\dotsc,\abs{k_n}\}\neq0\).
由\[
k_1 a_{l1} + k_2 a_{l2} + \dotsb + k_l a_{ll} + \dotsb + k_n a_{ln} = 0,
\]可得\[
a_{ll} = -\frac{1}{k_l} (k_1 a_{l1} + \dotsb + k_{l-1} a_{l,l-1} + k_{l+1} a_{l,l+1} + \dotsb + k_n a_{ln})
= - \sum\limits_{\substack{1 \leqslant j \leqslant n \\ j \neq l}} \frac{k_j}{k_l} a_{lj},
\]\[
\abs{a_{ll}} \leqslant \sum\limits_{\substack{1 \leqslant j \leqslant n \\ j \neq l}} \frac{\abs{k_j}}{\abs{k_l}} \abs{a_{lj}}
\leqslant \sum\limits_{\substack{1 \leqslant j \leqslant n \\ j \neq l}} \abs{a_{lj}}.
\]这与已知条件矛盾!
因此\(\AutoTuple{\a}{n}\)线性无关,\(\rank\{\AutoTuple{\a}{n}\} = n\).
\end{proof}
\end{example}

\subsection{极大线性无关组的概念}
\begin{definition}
在\(K^n\)中,设\(B\)是\(A\)的一个部分组.
如果\begin{enumerate}
	\item \(B\)线性无关,
	\item \(A\)可由\(B\)线性表出,
\end{enumerate}
则称“\(B\)是\(A\)的一个\DefineConcept{极大线性无关组}(maximally linearly independent subset)”.
%@see: https://mathworld.wolfram.com/MaximallyLinearlyIndependent.html
\end{definition}

\begin{property}
向量组与其极大无关组等价.
\begin{proof}
因为作为部分组,极大无关组可由全组线性表出,
同时根据极大无关组的定义全组可由极大无关组线性表出,
则根据向量组等价的定义可知原向量组与极大无关组等价.
\end{proof}
\end{property}

\begin{corollary}
向量组的任何两个极大无关组等价,且包含相同个数的向量.
\begin{proof}
设\(A=\{\AutoTuple{\a}{r}\}\)
与\(B=\{\AutoTuple{\b}{t}\}\)
是某个向量组的两个极大无关组.
因为向量组的任意向量可由其极大无关组线性表出,%
所以\(A\)可由\(B\)线性表出,%
\(B\)也可由\(A\)线性表出,%
即\(A \cong B\),%
进而这两个等价的线性无关向量组含有相同的向量个数.
\end{proof}
\end{corollary}

\begin{theorem}
在\(P^n\)中,任意向量组的极大无关组的向量个数不大于\(n\)个.
\begin{proof}
根据定义,任意向量组的极大无关组是线性无关的,而向量个数大于维数的向量组总是线性相关,故任意向量组的极大无关组的向量个数总是不大于其维数\(n\)的.
\end{proof}
\end{theorem}

\begin{example}
求向量组\(\{\a\}\)的极大无关组.
\begin{solution}
显然有\[
\Powerset\{\a\} = \{ \emptyset, \{\a\} \},
\]即\(\{\a\}\)的部分组只有\(\emptyset\)和\(\{\a\}\),从而它的极大无关组也只能是这两者中的一个.

当\(\a=\z\)时,\(\{\a\}\)线性相关,不能满足极大无关组的定义,故\(\{\a\}\)的极大无关组是\(\emptyset\).

当\(\a\neq\z\)时,\(\{\a\}\)线性无关,所以\(\{\a\}\)的极大无关组是它本身.
\end{solution}
\end{example}

\subsection{向量组的秩}
\begin{definition}
向量组的极大无关组所含向量的个数,称为向量组的\DefineConcept{秩}(rank).
记\[
A = \{\AutoTuple{\a}{s}\}
\]的秩为\(\rank A\)或\(r\{\AutoTuple{\a}{s}\}\).
\end{definition}

\begin{property}
空集\(\emptyset\)的秩为零,即\(\rank\emptyset = 0\).
\end{property}

\begin{property}
零向量组的秩为零,即\(r\{\z,\z,\dotsc,\z\}=0\).
\begin{proof}
因为向量组本质是向量的集合,所以\[
\{\z,\z,\dotsc,\z\} = \{\z\},
\]而由上例知道\(\{\z\}\)的极大无关组是\(\emptyset\),故\[
r\{\z,\z,\dotsc,\z\}
= r\{\z\}
= \abs{\emptyset}
= 0.
\qedhere
\]
\end{proof}
\end{property}

\begin{corollary}
设向量组\(A=\{\AutoTuple{\a}{s}\}\).
如果\(\rank{A}<s\),则向量组\(A\)线性相关;
如果\(\rank{A}=s\),则向量组\(A\)线性无关.
\end{corollary}

\begin{corollary}
设向量组
\(A=\{\AutoTuple{\a}{s}\}\)
可由向量组
\(B=\{\AutoTuple{\b}{t}\}\)
线性表出,%
则\(\rank A \leqslant \rank B\).
\begin{proof}
设\(\rank A = r\),\(\rank B = u\).因为\(A\)可由\(B\)线性表出,即\[
\a_k = \sum\limits_{i=1}^t l_{ki} \b_i,
\quad k=1,2,\dotsc,s.
\]
设\(A'=\{\AutoTuple{\a}{r}\}\)
和\(B'=\{\AutoTuple{\b}{u}\}\)
分别是\(A\)和\(B\)的极大无关组,%
则\(B\)可由\(B'\)线性表出,即\[
\b_i = \sum\limits_{j=1}^u b_{ij} \b_j,
\quad i=1,2,\dotsc,t;
\]所以有\[
\a_k = \sum\limits_{i=1}^t l_{ki} \sum\limits_{j=1}^u b_{ij} \b_j
= \sum\limits_{i=1}^t \sum\limits_{j=1}^u l_{ki} b_{ij} \b_j
= \sum\limits_{j=1}^u \b_j \sum\limits_{i=1}^t l_{ki} b_{ij},
\quad k=1,2,\dotsc,s.
\]

特别地,\(A'\)可由\(B'\)线性表出,%
则有\(r \leqslant u\),即\(\rank A \leqslant \rank B\).
\end{proof}
\end{corollary}

\begin{corollary}
等价向量组的秩相等.秩相等的向量组却不一定等价.
\begin{proof}
设向量组\(A\)与\(B\)等价,则\(A\)可由\(B\)线性表出,那么\(\rank A \leqslant \rank B\);同理可得\(\rank A \geqslant \rank B\),所以\(\rank A = \rank B\).

设\(A=\{ (0,1) \}\),\(B=\{ (1,0) \}\),虽然\(\rank A = \rank B = 1\),但\(A\)与\(B\)不等价.
\end{proof}
\end{corollary}

\begin{example}
证明:在秩为\(r\)的向量组中,任意\(r+1\)个向量必线性相关.
\begin{proof}
设向量组\(\AutoTuple{\a}{s}\)的秩为\(r\).
假设部分组\(\AutoTuple{\a}{r+1}\)线性无关,%
那么\[
\rank\{\AutoTuple{\a}{r+1}\} = r+1.
\]
因为部分组总可由全组线性表出,所以部分组的秩总是小于或等于全组的秩,即\[
r+1 = \rank\{\AutoTuple{\a}{r+1}\} \leqslant \rank\{\AutoTuple{\a}{s}\} = r,
\]矛盾,所以部分组\(\AutoTuple{\a}{r+1}\)一定线性相关.
\end{proof}
\end{example}

\begin{example}
设向量组\(\AutoTuple{\a}{s}\)的秩为\(r\).
如果\(\AutoTuple{\a}{r}\)线性无关,证明:
\(\AutoTuple{\a}{r}\)
是\(\AutoTuple{\a}{s}\)的一个极大无关组.
\begin{proof}
设\[
A=\{\AutoTuple{\a}{s}\},
\qquad
B=\{\AutoTuple{\a}{r}\}.
\]要证\(B\)是\(A\)的一个极大无关组,须证\(A\)的任意向量可由\(B\)线性表出.

\begin{enumerate}
\item 显然地,\(\a_i\ (i=1,2,\dotsc,r)\)可由\(B\)线性表出.

\item 根据上例,在秩为\(r\)的向量组中,任意\(r+1\)个向量必线性相关,那么向量组\[
A_i = \{\AutoTuple{\a}{r},\a_i\}\quad(i=r+1,\dotsc,s)
\]必线性相关.

又因为\(B\)线性无关,所以\(\a_i\ (i=r+1,\dotsc,s)\)可由\(B\)线性表出.
\end{enumerate}

综上所述,\(A\)的任意向量可由\(B\)线性表出,且\(B\)线性无关,根据极大无关组的定义,\(B\)是\(A\)的一个极大无关组.
\end{proof}
\end{example}

\begin{example}
向量组
\(\AutoTuple{\a}{r+1}\)
与部分组
\(\AutoTuple{\a}{r}\)
的秩相等.
证明:\(\a_{r+1}\)可由
\(\AutoTuple{\a}{r}\)
线性表出.
\begin{proof}
记\(A=\{\AutoTuple{\a}{r+1}\}\),
\(B=\{\AutoTuple{\a}{r}\}\).
设\(B\)的极大无关组为
\[
B'=\{\a_1,\a_2,\dotsc,\a_t\},
\quad 0 \leqslant t \leqslant r.
\]
由题意有
\(\rank A = \rank B = \rank B' = \abs{B'} = t\).

由上例可知,因为\(\rank A = t\),%
而\(B'\)线性无关,%
所以\(B'\)是\(A\)的一个极大无关组.
那么向量组\(A\)中的向量\(\a_{r+1}\)可以由极大无关组\(B'\)线性表出.
又由于\(B'\)是\(B\)的部分组,故\(B'\)可由\(B\)线性表出.
总而言之,\(A\)可由\(B\)线性表出.
\end{proof}
\end{example}

\subsection{极大无关组的求解}
\begin{theorem}
设矩阵\[
\A=(\AutoTuple{\a}{m})
\]经一系列初等行变换化为矩阵\[
\B=(\AutoTuple{\b}{m}),
\]则\(\a_{j1},\a_{j2},\dotsc,\a_{jk}\)为\(\A\)的列极大无关组的充要条件是:
\(\b_{j1},\b_{j2},\dotsc,\b_{jk}\)为\(\B\)的列极大无关组.
\begin{proof}
矩阵\((\a_{j1},\a_{j2},\dotsc,\a_{jk},\a_l)\)经相同的初等行变换化为
\[
(\b_{j1},\b_{j2},\dotsc,\b_{jk},\b_l) \quad(l=1,2,\dotsc,m).
\]
考虑以下四个向量形式的线性方程组
\begin{gather}
x_1 \a_{j1} + x_2 \a_{j2} + \dotsb + x_k \a_{jk} = \z, \tag1 \\
x_1 \b_{j1} + x_2 \b_{j2} + \dotsb + x_k \b_{jk} = \z, \tag2 \\
y_1 \a_{j1} + y_2 \a_{j2} + \dotsb + y_k \a_{jk} = \a_l, \tag3 \\
y_1 \b_{j1} + y_2 \b_{j2} + \dotsb + y_k \b_{jk} = \b_l, \tag4
\end{gather}
其中(1)与(2)同解,(3)与(4)同解.

必要性.
当\(\a_{j1},\a_{j2},\dotsc,\a_{jk}\)是\(\A\)的列极大无关组时,%
(1)仅有零解,(3)有解.于是(2)仅有零解,(4)有解,%
从而\(\b_{j1},\b_{j2},\dotsc,\b_{jk}\)线性无关,%
\(\b_l\ (l=1,2,\dotsc,m)\)可由其线性表出;
由极大无关组定义,%
\(\b_{j1},\b_{j2},\dotsc,\b_{jk}\)是\(\B\)的列极大无关组.

同理可证充分性.
\end{proof}
\end{theorem}

\begin{example}
求列向量组\[
\a_1 = \begin{bmatrix} -1 \\ 1 \\ 0 \\ 0 \end{bmatrix},
\a_2 = \begin{bmatrix} -1 \\ 2 \\ -1 \\ 1 \end{bmatrix},
\a_3 = \begin{bmatrix} 0 \\ -1 \\ 1 \\ -1 \end{bmatrix},
\a_4 = \begin{bmatrix} 1 \\ -1 \\ 2 \\ 3 \end{bmatrix},
\a_5 = \begin{bmatrix} 2 \\ -6 \\ 4 \\ 1 \end{bmatrix}
\]的秩与一个极大无关组.
\begin{solution}
对矩阵\(\A = (\AutoTuple{\a}{5})\)作初等行变换化为阶梯形矩阵:
\begin{align*}
\A &= \begin{bmatrix}
-1 & -1 & 0 & 1 & 2 \\
1 & 2 & -1 & -3 & -6 \\
0 & -1 & 1 & 2 & 4 \\
0 & 1 & -1 & 3 & 1 \\
\end{bmatrix}
\xlongrightarrow{\begin{array}{c}
(2\text{行}) \addeq 1 \times (1\text{行}) \\
(4\text{行}) \addeq (3\text{行})
\end{array}}
\begin{bmatrix}
-1 & -1 & 0 & 1 & 2 \\
0 & 1 & -1 & -2 & -4 \\
0 & -1 & 1 & 2 & 4 \\
0 & 0 & 0 & 5 & 5 \\
\end{bmatrix} \\
&\xlongrightarrow{\begin{array}{c}
(3\text{行}) \addeq (2\text{行}) \\
(4\text{行}) \diveq 5
\end{array}}
\begin{bmatrix}
-1 & -1 & 0 & 1 & 2 \\
0 & 1 & -1 & -2 & -4 \\
0 & 0 & 0 & 0 & 0 \\
0 & 0 & 0 & 1 & 1 \\
\end{bmatrix} \\
&\xlongrightarrow{\begin{array}{c} \text{交换(3行)与(4行)} \end{array}}
\begin{bmatrix}
-1 & -1 & 0 & 1 & 2 \\
0 & 1 & -1 & -2 & -4 \\
0 & 0 & 0 & 1 & 1 \\
0 & 0 & 0 & 0 & 0 \\
\end{bmatrix} = \B.
\end{align*}
若按列分块有\(\B = (\AutoTuple{\b}{5})\).
阶梯形矩阵\(\B\)有3行不为零,故\[
\rank\{\AutoTuple{\a}{5}\}=3.
\]又因为\(\B\)的非零首元分别位于1、2、4列,%
则\(\b_1,\b_2,\b_4\)是\(\B\)的一个列极大无关组,%
相应地,\(\a_1,\a_2,\a_4\)是\(\A\)的一个列极大无关组,%
即\(\{\AutoTuple{\a}{5}\}\)的极大无关组.
\end{solution}
\end{example}

\section{向量空间及其子空间的基与维数}
\begin{definition}
%@see: 《高等代数(第三版 上册)》(丘维声) P77. 定义1
设\(U\)是\(K^n\)的一个子空间.
如果\(A=\{\AutoTuple{\a}{r}\}\subseteq U\)满足\begin{enumerate}
	\item \(A\)线性无关,
	\item \(U\)中的每一个向量都可以由\(A\)线性表出,
\end{enumerate}
那么称\(A\)是\(U\)的一个\DefineConcept{基}.
\end{definition}
%由于基\(\AutoTuple{\a}{r}\)线性无关,
%因此如果\(\a\)可以由\(\AutoTuple{\a}{r}\)线性表出,
%那么表法唯一.

在\(K^n\)中,基本向量组\(\AutoTuple{\e}{n}\)线性无关,
并且根据\cref{theorem:向量空间.任一向量可由基本向量组唯一线性表出},
每一个向量\(\a=(\AutoTuple{a}{n})^T\)可由基本向量组线性表出,
于是基本向量组是\(K^n\)的一个基,
称之为\(K^n\)的\DefineConcept{标准基}.

\begin{theorem}\label{theorem:线性方程组.向量空间1}
%@see: 《高等代数(第三版 上册)》(丘维声) P77. 定理1
\(K^n\)的任一非零子空间\(U\)都有一个基.
\begin{proof}
因为\(U\neq\{\z\}\),
所以\(U\)中至少有一个非零向量\(\a_1\).
由\cref{example:线性方程组.单向量组线性相关的充要条件} 可知,
向量组\(\{\a_1\}\)是线性无关的.
若\(\opair{\a_1} \neq U\),
则\((\exists \a_2 \in U)[\a_2 \notin \opair{\a_1}]\).
于是\(\a_2\)不能由\(\a_1\)线性表出,
由\cref{theorem:向量空间.增加一个向量对线性相关性的影响2},
\(\{\a_1,\a_2\}\)线性无关.
若\(\opair{\a_1,\a_2} \neq U\),
则\((\exists \a_3 \in U)[\a_3 \notin \opair{\a_1,\a_2}]\).
同理\(\{\a_1,\a_2,\a_3\}\)线性无关.
以此类推,
根据\cref{theorem:向量空间.线性无关向量组的基数不大于可以线性表出它的任意向量组的基数},
由于\(K^n\)的任一线性无关向量组所含向量个数不超过\(n\),
因此上述过程不能无限进行下去,到某一步必定终止.
即将我们得到了\(U\)中一个线性无关向量组\(\{\AutoTuple{\a}{s}\}\)以后,
有\(\opair{\AutoTuple{\a}{s}} = U\),
则\(\{\AutoTuple{\a}{s}\}\)就是\(U\)的一个基.
\end{proof}
\end{theorem}
\cref{theorem:线性方程组.向量空间1} 的证明过程也表明,
从子空间\(U\)的一个非零向量出发,可以扩充成\(U\)的一个基.

\begin{theorem}\label{theorem:线性方程组.向量空间2}
\(K^n\)的非零子空间\(U\)的任意两个基所含向量的个数相等.
\begin{proof}
等价的线性无关的向量组含有相同个数的向量.
\end{proof}
\end{theorem}

\begin{definition}
设\(U\)是\(K^n\)的一个非零子空间.
\(U\)的一个基所含向量的个数称为“\(U\)的\DefineConcept{维数}(dimension)”,
记作\(\dim_K U\),或简记为\(\dim U\).

特别地,规定零空间的维数为0.
\end{definition}

\begin{property}
\(\dim K^n = n\).
\begin{proof}
基本向量组\(\AutoTuple{\e}{n}\)是\(K^n\)的一个基.
\end{proof}
\end{property}

设\(\AutoTuple{\a}{r}\)是\(K^n\)的子空间\(U\)的一个基,则\(U\)的每一个向量\(\a\)都可以由\(\AutoTuple{\a}{r}\)唯一地线性表出:\[
\a = x_1 \a_1 + x_2 \a_2 + \dotsb + x_r \a_r.
\]把元组\(\opair{x_1,x_2,\dotsc,x_r}\)称为\(\a\)在基\(\AutoTuple{\a}{r}\)下的\DefineConcept{坐标}.

\begin{example}
设\(\dim U = r\),证明:\(U\)中任意\(r+1\)个向量都线性相关.
\end{example}

\begin{example}
设\(\dim U = r\),证明:\(U\)中任意\(r\)个线性无关的向量都是\(U\)的一个基.
\end{example}

\begin{example}
设\(\dim U = r\),\(\AutoTuple{\a}{r} \in U\).
证明:如果\(U\)中每一个向量都可以由\(\AutoTuple{\a}{r}\)线性表出,那么\(\AutoTuple{\a}{r}\)是\(U\)的一个基.
\end{example}

\begin{example}
设\(U\)和\(W\)都是\(K^n\)的非零子空间.
证明:如果\(U \subseteq W\),那么\(\dim U \leqslant \dim W\).
\end{example}

\begin{example}
设\(U\)和\(W\)都是\(K^n\)的非零子空间,且\(U \subseteq W\).
证明:如果\(\dim U = \dim W\),那么\(U = W\).
\end{example}

\begin{theorem}
向量组\(\AutoTuple{\a}{s}\)的一个极大线性无关组是这个向量组生成的子空间\(\opair{\AutoTuple{\a}{s}}\)的一个基,从而\begin{equation}\label{equation:线性方程组.子空间的维数与向量组的秩的联系}
\dim\opair{\AutoTuple{\a}{s}} = \rank\{\AutoTuple{\a}{s}\}.
\end{equation}
\end{theorem}
这里要注意区分“子空间的维数\(\dim\opair{\AutoTuple{\a}{s}}\)”
和“向量组的秩\(\rank\{\AutoTuple{\a}{s}\}\)”这两个概念:
维数是对子空间而言,秩是对向量组而言;
在子空间\(\opair{\AutoTuple{\a}{s}}\)这个集合中有无穷多个向量,
而向量组\(\{\AutoTuple{\a}{s}\}\)这个集合中只有有限的\(s\)个向量.

数域\(K\)上任一\(s \times n\)矩阵\(\A\)的列向量组\(\AutoTuple{\a}{n}\)%
生成的子空间称为\(\A\)的\DefineConcept{列空间};
\(\A\)的行向量组\(\AutoTuple{\g}{s}\)生成的子空间称为\(\A\)的\DefineConcept{行空间}.
易知,\(\A\)的列(行)空间的维数等于\(\A\)的列(行)向量组的秩.

\section{矩阵的秩}
\subsection{矩阵的行秩与列秩}
为了求解向量组的秩,我们可以把向量组看成矩阵的行向量组或列向量组,
利用矩阵的性质,得出这个矩阵的行向量组、列向量组的秩,
最后得到所求向量组的秩.

\begin{definition}\label{definition:线性方程组.行秩与列秩的定义}
%@see: 《线性代数》(张慎语、周厚隆) P76. 定义11(1)
设\(\A\)是向量.
\begin{enumerate}
	\item \(\A\)的行向量组生成的子空间称为
	“\(\A\)的\DefineConcept{行空间}(row space)”,记为\(\Span_R\A\).
	\item \(\A\)的列向量组生成的子空间称为
	“\(\A\)的\DefineConcept{列空间}(column space)”,记为\(\Span_C\A\).
	\item \(\A\)的行向量组的秩称为
	“\(\A\)的\DefineConcept{行秩}(row rank)”,记为\(\rank_R\A\).
	\item \(\A\)的列向量组的秩称为
	“\(\A\)的\DefineConcept{列秩}(column rank)”,记为\(\rank_C\A\).
\end{enumerate}
%@see: https://mathworld.wolfram.com/RowSpace.html
%@see: https://mathworld.wolfram.com/ColumnSpace.html
\end{definition}

矩阵的列秩等于它的列空间的维数,它的行秩等于它的行空间的维数,即\[
	\rank_C\A=\dim(\Span_C\A), \qquad
	\rank_R\A=\dim(\Span_R\A).
\]

\begin{theorem}
设\(\A\)是矩阵,那么\begin{gather}
	\rank_R\A=\rank_C\A^T, \\
	\rank_C\A=\rank_R\A^T.
\end{gather}
\begin{proof}
显然\(\A\)的行向量组就是\(\A^T\)的列向量组,
而\(\A\)的列向量组就是\(\A^T\)的行向量组.
\end{proof}
\end{theorem}

\begin{definition}
设矩阵\(\A = (a_{ij})_{s \times n}\).
\begin{enumerate}
	\item 若\(\rank_R\A = s\),
	则称“\(\A\)为\DefineConcept{行满秩矩阵}(row full rank matrix)”.
	\item 若\(\rank_C\A = n\),
	则称“\(\A\)为\DefineConcept{列满秩矩阵}(column full rank matrix)”.
\end{enumerate}
\end{definition}

现在我们来研究一个问题:矩阵的列秩与行秩之间有什么联系?

\subsection{矩阵的秩}
\begin{definition}\label{definition:线性方程组.矩阵的秩的定义}
%@see: 《线性代数》(张慎语、周厚隆) P76. 定义11(2)
\(\A\)的非零子式的最高阶数称为
“矩阵\(\A\)的\DefineConcept{秩}(rank)”,
记为\(\rank\A\).
%@see: https://mathworld.wolfram.com/MatrixRank.html
\end{definition}

\begin{property}\label{theorem:线性方程组.矩阵的秩的性质1}
零矩阵的秩为零,即\(\rank\z = 0\).
\end{property}

\begin{property}\label{theorem:线性方程组.矩阵的秩的性质2}
设\(\A\)是\(s \times n\)矩阵,则\(0 \leq \rank\A \leq \min\{s,n\}\).
\end{property}

\begin{theorem}
设\(\A\in M_{s \times n}(K)\).
如果\(\A\)有一个\(k\)阶非零子式,那么\(\rank\A\geqslant k\).
\begin{proof}
用反证法.
假设\(\rank\A<k\),
也就是说\(\A\)的非零子式的最高阶数\(r\)小于\(k\),
但是根据前提条件,\(\A\)有一个\(k\)阶非零子式,\(k>r\),
这就和\hyperref[definition:线性方程组.矩阵的秩的定义]{矩阵的秩的定义}矛盾!
因此\(\rank\A\geqslant k\).
\end{proof}
\end{theorem}

\begin{example}
求矩阵\(\A = \begin{bmatrix} 1 & 7 \\ 2 & 6 \\ -3 & 1 \end{bmatrix}\)的秩、行秩与列秩.
\begin{solution}
因为至少存在一个\(\A\)的二阶子式不为零,即\[
	\begin{vmatrix} 1 & 7 \\ 2 & 6 \end{vmatrix} = -8 \neq 0,
\]
且\(\A\)没有三阶子式,所以\(\rank\A = 2\).

写出\(\A\)的行向量组\[
	(1,7), \qquad
	(2,6), \qquad
	(-3,1),
\]
分别转置后,列成一个矩阵(显然就是\(\A^T\)),
作初等行变换化成阶梯形矩阵\[
	\A^T = \begin{bmatrix}
		1 & 2 & -3 \\
		7 & 6 & 1
	\end{bmatrix} \to \begin{bmatrix}
		1 & 0 & 17/2 \\
		0 & 4 & -11
	\end{bmatrix} = \B_1.
\]阶梯形矩阵\(\B_1\)有2行不为零,故矩阵\(\A\)的行秩为2.

写出\(\A\)的列向量组\[
	(1,2,-3)^T,
	(7,6,1)^T,
\]列成一个矩阵(显然就是\(\A\)本身),
作初等行变换化成阶梯形矩阵\[
	\A = \begin{bmatrix} 1 & 7 \\ 2 & 6 \\ -3 & 1 \end{bmatrix}
	\to \begin{bmatrix} 1 & 0 \\ 0 & 1 \\ 0 & 0 \end{bmatrix} = \B_2.
\]
阶梯形矩阵\(\B_2\)有2行不为零,故矩阵\(\A\)的列秩为2.
\end{solution}
\end{example}

\subsection{矩阵的秩、行秩、列秩的关系}
\begin{lemma}\label{theorem:向量空间.矩阵的秩与行秩和列秩的关系.引理}
%@see: 《线性代数》(张慎语、周厚隆) P77. 引理
设矩阵\(\A = (a_{ij})_{s \times n}\)的列秩等于\(\A\)的列数\(n\),
则\(\A\)行秩、秩都等于\(n\).
\begin{proof}
对\(\A\)分别进行列分块与行分块:\[
	\A = (\AutoTuple{\a}{n})
	= (\AutoTuple{\A}{s}[,][T])^T.
\]
根据\cref{theorem:向量空间.秩与线性相关性的关系},
由于\(\rank_C\A=n\),
所以\(\A\)的列向量组线性无关,
也就是说齐次线性方程组\[
	x_1 \a_1 + x_2 \a_2 + \dotsb + x_n \a_n = \z
\]只有零解,或者说\(\A\x=\z\)只有零解.
进而根据\cref{theorem:线性方程组.方程个数少于未知量个数的齐次线性方程组必有非零解},
在\(\A\x=\z\)中,方程个数\(s\)不少于未知量个数\(n\),即\(s \geq n\).

设\(\rank_R\A=t\).
根据\cref{theorem:向量空间.子空间维数.命题4},
因为\(\A\)的行向量的维数是\(n\),
所以\(t \leq n\).

设\(\A\)的行极大线性无关组是
\(\A_{i_1},\A_{i_2},\dotsc,\A_{i_t}\),
其中\(1 \leq i_1 < i_2 < \dotsb < i_t \leq s\).

假设\(t < n\),
则\(\A\)可通过初等行变换化为\[
	\begin{bmatrix}
		\A_{i_1} \\ \A_{i_2} \\ \vdots \\ \A_{i_t} \\ \z_{(s-t) \times n}
	\end{bmatrix},
\]
于是,\(\A\x=\z\)表示的齐次线性方程组中,
非零方程的个数\(t\)小于未知量个数\(n\),
那么根据\cref{theorem:线性方程组.方程个数少于未知量个数的齐次线性方程组必有非零解},
\(\A\x=\z\)有非零解,
这与我们前面得到的结论“\(\A\x=\z\)只有零解”矛盾,
所以假设\(t < n\)不成立,
因此\(t = n\).
这就是说\[
	\rank_R\A=\rank_C\A=n.
\]

既然\(t=n\),
那么\(\A\)的行极大线性无关组恰好可以构成一个方阵,
而这个方阵的行列式为\[
	d=\begin{vmatrix} \A_{i_1} \\ \A_{i_2} \\ \vdots \\ \A_{i_n} \end{vmatrix}.
\]
根据\cref{theorem:线性方程组.克拉默法则},
因为\(\A\x=\z\)只有零解作为它的唯一解,
所以\(d\neq0\).
于是\(d\)是\(\A\)的一个\(n\)阶非零子式.
\(\A\)是一个\(s \times n\)矩阵,
不可能有阶数大于\(n\)的子式,
因此\(\rank\A = n\).
\end{proof}
\end{lemma}

\begin{theorem}\label{theorem:向量空间.矩阵的秩与行秩和列秩的关系.定理}
%@see: 《线性代数》(张慎语、周厚隆) P77. 定理5
矩阵的行秩、列秩、秩都相等.
\begin{proof}
如果矩阵\(\A = \z\),
则\(\rank\A=\rank_R\A=\rank_C\A=0\),
结论成立.

当\(\A\neq\z\)时,
设\(\rank\A=r\),
则根据\hyperref[definition:线性方程组.矩阵的秩的定义]{矩阵的秩的定义},
\(\A\)的所有\(t\ (t > r)\)阶子式全为零;
且\(\A\)有一个\(r\)阶子式不为零,
从而该\(r\)阶子式的列向量组线性无关,
它们的延伸组也线性无关,
\(\A\)有\(r\)列线性无关,
于是\(\rank_C\A=p \geq r\);
由\cref{theorem:向量空间.矩阵的秩与行秩和列秩的关系.引理},
\(\A\)的列极大线性无关组构成的矩阵有一非零\(p\)阶子式,
也是\(\A\)的子式,
故\(p \leq r\);
综上得\(p = r\).
因此\(\rank\A=\rank_C\A\).
同样地,将这一结论用于\(\A^T\),
得\(\rank\A^T=\rank_C\A^T=\rank_R\A\).
\end{proof}
\end{theorem}

\begin{theorem}
%@see: 《高等代数(第三版 上册)》(丘维声) P83. 推论6
设\(\A\)是矩阵,那么\(\rank\A = \rank\A^T\).
\begin{proof}
\(\rank\A=\rank_C\A=\rank_R\A^T=\rank\A^T\).
\end{proof}
\end{theorem}

\begin{theorem}\label{theorem:线性方程组.初等变换不变秩}
%@see: 《高等代数(第三版 上册)》(丘维声) P81. 定理2
%@see: 《高等代数(第三版 上册)》(丘维声) P82. 定理3
%@see: 《高等代数(第三版 上册)》(丘维声) P83. 推论7
初等变换不改变矩阵的秩.
\begin{proof}
由\cref{theorem:向量空间.利用初等行变换求取列极大线性无关组的依据},
初等行变换不改变矩阵的列秩;
同理,初等列变换不改变矩阵的行秩;
再由\cref{theorem:向量空间.矩阵的秩与行秩和列秩的关系.定理},
初等变换不改变矩阵的秩.
\end{proof}
\end{theorem}

\subsection{满秩矩阵}
\begin{definition}
%@see: 《线性代数》(张慎语、周厚隆) P76.
设矩阵\(\A = (a_{ij})_n\).
若\(\rank\A = n\),
则称“\(\A\)是\DefineConcept{满秩矩阵}(full rank matrix)”,
或“\(\A\)是\DefineConcept{非退化矩阵}(non-degenerate matrix)”.
\end{definition}

%应该注意到,“可逆矩阵”“非奇异矩阵”“满秩矩阵”“非退化矩阵”是从不同侧重点对同一类矩阵的四种称谓.

\begin{theorem}\label{theorem:向量空间.满秩方阵的行列式非零}
设\(\A\in M_n(K)\).
\(\A\)是满秩矩阵的充要条件是\(\abs{\A}\neq0\).
\end{theorem}

\begin{example}
计算行列式\(\det\D\),其中\[
\D = \begin{bmatrix}
	1 & \cos(\alpha_1-\alpha_2) & \cos(\alpha_1-\alpha_3) & \dots & \cos(\alpha_1-\alpha_n) \\
	\cos(\alpha_1-\alpha_2) & 1 & \cos(\alpha_2-\alpha_3) & \dots & \cos(\alpha_2-\alpha_n) \\
	\cos(\alpha_1-\alpha_3) & \cos(\alpha_2-\alpha_3) & 1 & \dots & \cos(\alpha_3-\alpha_n) \\
	\vdots & \vdots & \vdots & & \vdots \\
	\cos(\alpha_1-\alpha_n) & \cos(\alpha_2-\alpha_n) & \cos(\alpha_3-\alpha_n) & \dots & 1
\end{bmatrix}.
\]
\begin{solution}
记\(\D = (d_{ij})_n\).
由\cref{equation:函数.三角函数.和积互化公式2} 可知\[
	d_{ij} = \cos(\alpha_i-\alpha_j)
	= \cos\alpha_i\cos\alpha_j+\sin\alpha_i\sin\alpha_j,
	\quad i,j=1,2,\dotsc,n.
\]
记\[
	\A = \begin{bmatrix}
		\cos\alpha_1 & \cos\alpha_2 & \dots & \cos\alpha_n \\
		\sin\alpha_1 & \sin\alpha_2 & \dots & \sin\alpha_n
	\end{bmatrix},
\]那么\(\D = \A^T \A\).

由\cref{theorem:线性方程组.矩阵的秩的性质2} 可知,
\(\rank\A = \rank\A^T \leq \min\{n,2\}\).
由\cref{theorem:线性方程组.矩阵乘积的秩} 可知,\[
	\rank(\A^T\A) \leq \min\left\{\rank\A^T,\rank\A\right\} = \rank\A.
\]

当\(n=1\)时,\(\abs{\D}=1\).

当\(n=2\)时,\[
	\abs{\D}
	= \begin{vmatrix}
		1 & \cos(\alpha_1-\alpha_2) \\
		\cos(\alpha_1-\alpha_2) & 1
	\end{vmatrix}
	= 1 - \cos^2(\alpha_1-\alpha_2).
\]

当\(n>2\)时,
\(\rank(\A^T\A) \leq \rank\A \leq2\),
\(\D = \A^T \A\)不满秩,
故由\cref{theorem:向量空间.满秩方阵的行列式非零} 有\(\abs{\D}=0\).
\end{solution}
\end{example}

\begin{example}
设\(\a_1=(1,2,-1,0)^T,\a_2=(1,1,0,2)^T,\a_3=(2,1,1,a)^T\).
若\(\dim\opair{\AutoTuple{\a}{3}}=2\),求\(a\).
\begin{solution}
除了利用\cref{equation:线性方程组.子空间的维数与向量组的秩的联系} 在将矩阵\[
	\A = (\a_1,\a_2,\a_3)
	= \begin{bmatrix}
		1 & 1 & 2 \\
		2 & 1 & 1 \\
		-1 & 0 & 1 \\
		0 & 2 & a
	\end{bmatrix}
\]化为阶梯形矩阵以后,
根据\(\rank\{\AutoTuple{\a}{3}\}=2\)求出\(a\)的值这种方法以外,
我们还可以利用本节\hyperref[definition:线性方程组.矩阵的秩的定义]{矩阵的秩的定义},
得出\(\rank\A=\dim\opair{\AutoTuple{\a}{3}}=2\)这一结论,
从而根据\cref{definition:线性方程组.矩阵的秩的定义} 可知,
\(\A\)中任意3阶子式全都为零.
对于\(\A\)这么一个\(4\times3\)矩阵,
任意去掉不含\(a\)的一行(不妨去掉第一行)得到一个行列式必为零:\[
	\begin{vmatrix}
	2 & 1 & 1 \\
	-1 & 0 & 1 \\
	0 & 2 & a
	\end{vmatrix}
	= a - 6 = 0,
\]
解得\(a = 6\).
\end{solution}
\end{example}

\begin{example}
%@see: 《高等代数(第三版 上册)》(丘维声) P122. 命题2
设\(\A \in M_{m \times n}(\mathbb{R})\).
求证:\begin{equation}
	\rank\A = \rank(\A\A^T).
\end{equation}
\begin{proof}
设\(\rank\A=r\),
由\hyperref[definition:线性方程组.矩阵的秩的定义]{矩阵的秩的定义}可知,
\(\A\)有\(r\)阶子式不等于零,即\[
	\MatrixMinor\A{
		\AutoTuple{i}{r} \\
		\AutoTuple{j}{r}
	} \neq 0.
	\eqno(1)
\]
由\hyperref[equation:线性方程组.柯西比内公式]{柯西--比内公式}可知
\begin{align*}
	\MatrixMinor{(\A\A^T)}{
		\AutoTuple{i}{r} \\
		\AutoTuple{i}{r}
	}
	&= \sum\limits_{1 \leq k_1 < k_2 < \dotsb < k_r \leq n}
	\MatrixMinor\A{
		\AutoTuple{i}{r} \\
		\AutoTuple{k}{r}
	}
	\MatrixMinor{\A^T}{
		\AutoTuple{k}{r} \\
		\AutoTuple{i}{r}
	} \\
	&= \sum\limits_{1 \leq k_1 < k_2 < \dotsb < k_r \leq n}
	\left[
		\MatrixMinor\A{
			\AutoTuple{i}{r} \\
			\AutoTuple{k}{r}
		}
	\right]^2.
	\tag2
\end{align*}
因为\(\A \in M_{m \times n}(\mathbb{R})\),
所以\(\A\)的任意子式都是实数,于是有
\[
	\left[
		\MatrixMinor\A{
			\AutoTuple{i}{r} \\
			\AutoTuple{k}{r}
		}
	\right]^2
	\geq 0.
	\eqno(3)
\]
由(1)式可知,
\[
	\left[
		\MatrixMinor\A{
			\AutoTuple{i}{r} \\
			\AutoTuple{j}{r}
		}
	\right]^2
	> 0.
	\eqno(4)
\]
由于在(2)式中,
求和指标可以\((\AutoTuple{k}{r})\)取到\((\AutoTuple{j}{r})\),
所以(4)式是(2)式中的一项.
综上所述
\[
	\MatrixMinor{(\A\A^T)}{
		\AutoTuple{i}{r} \\
		\AutoTuple{i}{r}
	}
	= \sum\limits_{1 \leq k_1 < k_2 < \dotsb < k_r \leq n}
	\left[
		\MatrixMinor\A{
			\AutoTuple{i}{r} \\
			\AutoTuple{k}{r}
		}
	\right]^2
	> 0,
\]
这就说明\(\rank(\A\A^T) \geq r\).
又因为\(\rank(\A\A^T) \leq \rank\A = r\),
所以\(\rank(\A\A^T) = \rank\A = r\).
\end{proof}
\end{example}

\section{线性方程组有解的充要条件}
现在我们可以来回答直接根据线性方程组的系数和常数项判断方程组有没有解,有多少解的问题.

\begin{theorem}\label{theorem:向量空间.线性方程组有解判别定理}
%@see: 《高等代数(第三版 上册)》(丘维声) P87. 定理1
线性方程组\[
	x_1 \a_1 + x_2 \a_2 + \dotsb + x_n \a_n = \b
\]有解的充要条件是:
它的系数矩阵与增广矩阵有相同的秩.
\begin{proof}
记系数矩阵\(\A=(\AutoTuple{\a}{n})\),
增广矩阵\(\wA=(\A,\b)\),
那么
\begin{align*}
	&\hspace{-20pt}
	\text{线性方程组\(x_1 \a_1 + x_2 \a_2 + \dotsb + x_n \a_n = \b\)有解} \\
	&\iff \b\in\opair{\AutoTuple{\a}{n}} \\
	&\iff \opair{\AutoTuple{\a}{n},\b}\subseteq\opair{\AutoTuple{\a}{n}} \\
	&\iff \opair{\AutoTuple{\a}{n},\b}=\opair{\AutoTuple{\a}{n}} \\
	&\iff \dim\opair{\AutoTuple{\a}{n},\b}=\dim\opair{\AutoTuple{\a}{n}} \\
	&\iff \rank\A=\rank\wA.
	\qedhere
\end{align*}
\end{proof}
\end{theorem}

从\cref{theorem:向量空间.线性方程组有解判别定理} 看出,
判断线性方程组有没有解,只要去比较它的系数矩阵与增广矩阵的秩是否相等.
这种判别方法有几种优越之处:
首先,求矩阵的秩有多种方法,不一定要把系数矩阵和增广矩阵化成阶梯形矩阵.
其次,有时不用求出系数矩阵的秩和增广矩阵的秩,也能比较它们的秩是否相等.
由于系数矩阵\(\A\)是增广矩阵\(\wA\)的子矩阵,总有\(\rank\A\leq\rank\wA\),
那么只要能够证明\(\rank\wA\leq\rank\A\),就能得出\(\rank\A=\rank\wA\).

现在我们想知道,当线性方程组\(x_1 \a_1 + x_2 \a_2 + \dotsb + x_n \a_n = \b\)有解时,
能不能用系数矩阵的秩去判别它有唯一解,还是有无穷多个解?

\begin{theorem}\label{theorem:向量空间.有解的非齐次线性方程组的解的个数定理}
%@see: 《高等代数(第三版 上册)》(丘维声) P88. 定理2
设线性方程组\[
	x_1 \a_1 + x_2 \a_2 + \dotsb + x_n \a_n = \b
\]有解,\(\A\)是它的系数矩阵.
那么\begin{enumerate}
	\item 如果\(\rank\A=n\),则这个方程组有唯一解.
	\item 如果\(\rank\A<n\),则这个方程组有无穷多个解.
\end{enumerate}
\end{theorem}

把\cref{theorem:向量空间.有解的非齐次线性方程组的解的个数定理}
应用到齐次线性方程组上,便得出以下推论.
\begin{corollary}\label{theorem:线性方程组.齐次线性方程组有非零解的充要条件}
%@see: 《高等代数(第三版 上册)》(丘维声) P88. 推论3
%@see: 《线性代数》(张慎语、周厚隆) P80. 定理8
齐次线性方程组有非零解的充要条件是:它的系数矩阵的秩小于未知量的数目.
\end{corollary}
\cref{theorem:线性方程组.齐次线性方程组有非零解的充要条件}
的逆否命题“齐次线性方程组\(\A\x=\z\)只有零解的充要条件为\(\rank\A=n\)”也成立.

\section{齐次线性方程组的解集的结构}
\begin{proposition}\label{theorem:线性方程组.齐次线性方程组的解的线性组合也是解}
%@see: 《高等代数(第三版 上册)》(丘维声) P90. 性质1
%@see: 《高等代数(第三版 上册)》(丘维声) P90. 性质2
%@see: 《线性代数》(张慎语、周厚隆) P80. 性质1
%@see: 《线性代数》(张慎语、周厚隆) P81. 推论
齐次线性方程组\(\A\x=\z\)的解的任意线性组合也是解.
\begin{proof}
设\(\X1\)与\(\X2\)是齐次线性方程组\(\A\x=\z\)的任意两个解,
即\[
	\A\X1=\z, \qquad
	\A\X2=\z.
\]
又设\(k\)是任意常数,那么有\[
	\A (\X1 + \X2) = \A \X1 + \A \X2 = \z + \z = \z,
\]\[
	\A (k \X1) = k (\A \X1) = k \z = \z,
\]
所以\(\X1 + \X2\)与\(k \X1\)都是\(\A \x = \z\)的解.
\end{proof}
\end{proposition}

\cref{theorem:线性方程组.齐次线性方程组的解的线性组合也是解} 表明,
\(n\)元齐次线性方程组\(x_1\a_1+x_2\a_2+\dotsb+x_n\a_n=\z\)的解集\(W\)是\(K^n\)的一个子空间.
我们把它称为这个方程组的\DefineConcept{解空间}(space of solution).
如果这个方程组只有零解,那么\(W\)是零子空间.
如果这个方程组有非零解,那么\(W\)是非零子空间,从而\(W\)有基.
我们把解空间\(W\)的一个基称为这个方程组的一个\DefineConcept{基础解系}(basic set of solutions).

如果我们找到了齐次线性方程组的一个基础解系\(\{\AutoTuple{\x}{t}\}\),
\def\tongjie{k_1\x_1+k_2\x_2+\dotsb+k_t\x_t}%
那么这个方程组的解集为\[
	W = \Set{ \tongjie \given \AutoTuple{k}{t} \in K }.
\]
我们把表达式\((\tongjie)\)称为这个方程组的\DefineConcept{通解}(general solution).

如何找出齐次线性方程组的一个基础解系?
解空间\(W\)的维数是多少?

\begin{theorem}\label{theorem:线性方程组.齐次线性方程组的解向量个数}
%@see: 《高等代数(第三版 上册)》(丘维声) P91. 定理1
数域\(K\)上\(n\)元齐次线性方程组\(\A\x=\vb0\)的解空间\(W\)的维数为\begin{equation}
	\dim W = n - r,
\end{equation}
其中\(r\)是方程组的系数矩阵\(\A\)的秩\(\rank\A\).
当方程组有非零解时,它的每一个基础解系所含的解向量的数目都等于\(\dim W\).
\begin{proof}
设\(\A\)经一系列初等行变换化为阶梯形矩阵\(\B\),
那么由\cref{theorem:线性方程组.初等变换不变秩} 可知\[
	\rank\B = \rank\A = r,
\]
也就是说\(\B\)的前\(r\)行向量不为零.

不失一般性,
设\(\B\)的第\(i\)行非零首元为\(b_{ii}\ (i=1,2,\dotsc,r)\),则\[
	\A \to \B = \begin{bmatrix}
		\B_1 & \B_2 \\
		\z & \z
	\end{bmatrix},
\]
其中\[
	\B_1 = \begin{bmatrix}
		b_{11} & b_{12} & \dots & b_{1r} \\
		& b_{22} & \dots & b_{2r} \\
		& & \ddots & \vdots \\
		& & & b_{rr}
	\end{bmatrix},
	\qquad
	\B_2 = \begin{bmatrix}
		b_{1,r+1} & \dots & b_{1n} \\
		b_{2,r+1} & \dots & b_{2n} \\
		\vdots & & \vdots \\
		b_{r,r+1} & \dots & b_{rn}
	\end{bmatrix}.
\]

记\[
	\x = (x_1,x_2,\dotsc,x_r,x_{r+1},\dotsc,x_n)^T,
\]
将自由未知量\(x_{r+1},x_{r+2},\dotsc,x_n\)的一组值\((1,0,\dotsc,0)\)代入\[
	\B \x = \z,
\]
去掉\(0 = 0\)的等式,
移项得线性方程组\[
	\begin{bmatrix}
		b_{11} & b_{12} & \dots & b_{1r} \\
		& b_{22} & \dots & b_{2r} \\
		& & \ddots & \vdots \\
		& & & b_{rr}
	\end{bmatrix}
	\begin{bmatrix}
		x_1 \\ x_2 \\ \vdots \\ x_r
	\end{bmatrix}
	= \begin{bmatrix}
		-b_{1,r+1} \\
		-b_{2,r+1} \\
		\vdots \\
		-b_{r,r+1}
	\end{bmatrix}.
	\eqno(1)
\]
系数行列式\(D = b_{11} b_{22} \dotsm b_{rr} \neq 0\).

由克拉默法则,(1)式有唯一解,于是得\(\A\x=\z\)的一个解\[
\X1 = (c_{11},c_{21},\dotsc,c_{r1},1,0,\dotsc,0)^T.
\]

同理,分别将\(x_{r+1},x_{r+2},\dotsc,x_n\)的值\((0,1,\dotsc,0),\dotsc,(0,0,\dotsc,1)\)代入\[
	\B \x = \z,
\]
求出\(\A\x=\z\)的相应的解\[
	\begin{array}{rcl}
		\X2 &=& (c_{12},c_{22},\dotsc,c_{r2},0,1,\dotsc,0)^T, \\
		&\vdots& \\
		\X{n-r} &=& (c_{1,n-r},c_{2,n-r},\dotsc,c_{r,n-r},0,0,\dotsc,1)^T.
	\end{array}
\]

易见,\begin{enumerate}
	\item \(\AutoTuple{\x}{n-r}\)是\(\A\x=\z\)的解;

	\item 考虑向量方程\(k_1\X1+k_2\X2+\dotsb+k_{n-r}\X{n-r}=\z\),即\[
		(l_1,l_2,\dotsc,l_r,k_1,k_2,\dotsc,k_{n-r})^T
		= (0,0,\dotsc,0,0,\dotsc,0)^T,
	\]
	有\[
		k_1 = k_2 = \dotsb = k_{n-r} = 0,
	\]
	即\(\AutoTuple{\x}{n-r}\)线性无关;

	\item 设\(\x=(c_1,c_2,\dotsc,c_r,k_1,k_2,\dotsc,k_{n-r})^T\)是方程组\(\A\x=\z\)的任意一个解,
	则\[
		\x - (k_1 \X1 + k_2 \X2 + \dotsb + k_{n-r} \X{n-r})
		= (d_1,d_2,\dotsc,d_r,0,0,\dotsc,0)^T
	\]是齐次方程组的解,
	代入\(\B\x=\z\),去掉\(0 = 0\)的等式,得\[
		\begin{bmatrix}
			b_{11} & b_{12} & \dots & b_{1r} \\
			& b_{22} & \dots & b_{2r} \\
			& & \ddots & \vdots \\
			& & & b_{rr}
		\end{bmatrix}
		\begin{bmatrix}
			d_1 \\ d_2 \\ \vdots \\ d_r
		\end{bmatrix}
		= \begin{bmatrix}
			0 \\ 0 \\ \vdots \\ 0
		\end{bmatrix}.
	\]
	因为系数行列式\(\abs{\B_1} \neq 0\),
	所以\(d_1 = d_2 = \dotsb = d_r = 0\).
	于是\[
		\x - (k_1 \X1 + k_2 \X2 + \dotsb + k_{n-r} \X{n-r}) = \z,
	\]或\[
		\x = k_1 \X1 + k_2 \X2 + \dotsb + k_{n-r} \X{n-r}.
	\]
\end{enumerate}

综上所述,\(\AutoTuple{\x}{n-r}\)是\(\A\x=\z\)的一个基础解系,含有\(n-r\)个解向量.
\end{proof}
\end{theorem}

\begin{corollary}
设齐次线性方程组\(\A\x=\z\)的系数矩阵\(\A\)是\(s \times n\)矩阵,若\(\rank\A = r < n\),
则\begin{enumerate}
	\item \(\A\x=\z\)的每个基础解系都含有\(n-r\)个解向量;
	\item \(\A\x=\z\)的任意\(n-r+1\)个解向量线性相关;
	\item \(\A\x=\z\)的任意\(n-r\)个线性无关的解都是它的一个基础解系.
\end{enumerate}
\end{corollary}

\begin{proposition}\label{theorem:线性方程组.同解方程组的系数矩阵的秩相同}
设\(\A,\B \in M_{s \times n}(K)\),
则“方程组\(\A\x=\vb0\)与\(\B\x=\vb0\)同解”
是“\(\rank\A=\rank\B\)”的充分不必要条件.
\begin{proof}
假设\(\A\x=\vb0\)与\(\B\x=\vb0\)同解,
那么方程\(\A\x=\vb0\)的解空间与\(\B\x=\vb0\)的解空间相同,
那么由\cref{theorem:线性方程组.齐次线性方程组的解向量个数}
有\(n-\rank\A=n-\rank\B\),\(\rank\A=\rank\B\).

反过来,取\(\A = (1,0)\),\(\B = (0,1)\).
显然\(\rank\A = \rank\B = 1\).
但是线性方程组\(\A\x=\vb0\)的解是\(k(0,1)^T\ (\text{$k$是常数})\),
而\(\B\x=\vb0\)的解是\(k(1,0)^T\ (\text{$k$是常数})\).
\end{proof}
\end{proposition}

\begin{proposition}
设\(\A \in M_{s \times n}(K),
\B \in M_{n \times m}(K)\),
则\[
	\text{$(\A\B)\x=\vb0$与$\B\x=\vb0$同解}
	\iff
	\rank(\A\B)=\rank\B.
\]
\begin{proof}
充分性.
由\cref{theorem:线性方程组.同解方程组的系数矩阵的秩相同} 可知\[
	\text{$(\A\B)\x=\vb0$与$\B\x=\vb0$同解}
	\implies
	\rank(\A\B)=\rank\B.
\]

必要性.
设\(\vb\xi\)是\(\B\x=\vb0\)的一个解,
即\(\B\vb\xi=\vb0\),
那么左乘\(\A\)便得\(\A\B\vb\xi=\vb0\),
这就说明\(\vb\xi\)也是\((\A\B)\x=\vb0\)的一个解.
由于\(\vb\xi\)的任意性,
所以\(\B\x=\vb0\)的解都是\((\A\B)\x=\vb0\)的解,
也就是说\[
	(\text{$\B\x=\vb0$的解空间})
	\subseteq(\text{$(\A\B)\x=\vb0$的解空间}).
\]
因为\(\rank(\A\B)=\rank\B\),
所以由\cref{theorem:线性方程组.齐次线性方程组的解向量个数} 可知\[
	\dim(\text{$(\A\B)\x=\vb0$的解空间})
	=\dim(\text{$\B\x=\vb0$的解空间}).
\]
因此由\cref{theorem:向量空间.两个非零子空间的关系2} 可知\[
	(\text{$\B\x=\vb0$的解空间})
	=(\text{$(\A\B)\x=\vb0$的解空间}),
\]
也就是说\(\A\x=\vb0\)与\(\B\x=\vb0\)同解.
\end{proof}
\end{proposition}

\begin{example}
求齐次线性方程组\[
	\left\{ \begin{array}{*{11}{r}}
		x_1 &-& 2 x_2 &-& x_3 &+& 2 x_4 &+& 4 x_5 &=& 0 \\
		2 x_1 &-& 2 x_2 &-& 3 x_3 && &+& 2 x_5 &=& 0 \\
		4 x_1 &-& 2 x_2 &-& 7 x_3 &-& 4 x_4 &-& 2 x_5 &=& 0
	\end{array} \right.
\]的通解.
\begin{solution}
写出系数矩阵\(\A\),并作初等行变换化简
\begin{align*}
	\A &= \begin{bmatrix}
		1 & -2 & -1 & 2 & 4 \\
		2 & -2 & -3 & 0 & 2 \\
		4 & -2 & -7 & -4 & -2
	\end{bmatrix} \\
	&\xlongrightarrow{\begin{array}{c}
		-2\times\text{(1行)}+\text{(2行)} \\
		-4\times\text{(1行)}+\text{(3行)}
	\end{array}}
	\begin{bmatrix}
		1 & -2 & -1 & 2 & 4 \\
		0 & 2 & -1 & -4 & -6 \\
		0 & 6 & -3 & -12 & -18
	\end{bmatrix} \\
	&\xlongrightarrow{\begin{array}{c}
		-3\times\text{(2行)}+\text{(3行)} \\
		1\times\text{(2行)}+\text{(1行)}
	\end{array}}
	\begin{bmatrix}
		1 & 0 & -2 & -2 & -2 \\
		0 & 2 & -1 & -4 & -6 \\
		0 & 0 & 0 & 0 & 0
	\end{bmatrix}
	= \B,
\end{align*}
因为\(\rank\A=\rank\B=2\),所以基础解系含\(5-2=3\)个向量.
分别将\(x_3,x_4,x_5\)的3组值\((2,0,0),(0,1,0),(0,0,1)\)代入\(\B\x=\z\),
得基础解系:\[
	\X1 = (4,1,2,0,0)^T, \quad
	\X2 = (2,2,0,1,0)^T, \quad
	\X3 = (2,3,0,0,1)^T.
\]
原方程组的通解为\(k_1 \X1 + k_2 \X2 + k_3 \X3\),其中\(k_1,k_2,k_3\)为任意常数.
\end{solution}
\end{example}

\cref{theorem:线性方程组.齐次线性方程组的解向量个数}
揭示矩阵\(\A\)的秩与方程\(\A\x=\z\)的基础解系所含向量个数的关系.
它不仅对\(\A\x=\z\)的求解有重要意义,而且可以用来解决矩阵的秩的一些问题.
\begin{example}
设\(\A\)为\(s \times n\)矩阵,\(\B\)为\(n \times m\)矩阵,\(\A\B=\z\).
证明:\(\rank\A + \rank\B \leq n\).
\begin{proof}
设矩阵\(\B\)与\(\A\B=\z\)右端的零矩阵列分块矩阵分别为\[
	\B=(\AutoTuple{\b}{m}),
	\quad
	\z=(\z,\dotsc,\z)
\]
那么\[
	\A(\AutoTuple{\b}{m})
	= (\A\b_1,\A\b_2,\dotsc,\A\b_m)
	= (\z,\z,\dotsc,\z)
\]或\[
	\A\b_j = \z \quad(j=1,2,\dotsc,m)
\]
即\(\beta=\{\AutoTuple{\b}{m}\}\)是齐次线性方程组\(\A\x=\z\)解向量组.

若\(\rank\A=n\),则\(\A\x=\z\)只有零解,\(\B=\z\),\(\rank\B=0=n-\rank\A\).

若\(\rank\A<n\),则\(X=\{\AutoTuple{\x}{n-r}\}\)是\(\A\x=\z\)的一个基础解系,
则\(\beta\)可由\(X\)线性表出,\(\rank\beta \leq \rank X\).
而\(\rank\beta=\rank\B\),\(\rank X=n-\rank\A\).

综上所述,\(\rank\A+\rank\B \leq n\)成立.
\end{proof}
\end{example}

\begin{example}
已知向量\(\a_1 = \begin{bmatrix}
	1 \\ 2 \\ 3
\end{bmatrix},
\a_2 = \begin{bmatrix}
	2 \\ 1 \\ 1
\end{bmatrix},
\b_1 = \begin{bmatrix}
	2 \\ 5 \\ 9
\end{bmatrix},
\b_2 = \begin{bmatrix}
	1 \\ 0 \\ 1
\end{bmatrix}\).
若向量\(\g\)既可由\(\a_1,\a_2\)线性表出,
也可由\(\b_1,\b_2\)线性表出,求\(\g\).
\begin{solution}
设\(\g = x_1 \a_1 + x_2 \a_2 = x_3 \b_1 + x_4 \b_2\),
则\(x_1 \a_1 + x_2 \a_2 - x_3 \b_1 - x_4 \b_2 = \z\).
又由于\[
	(\a_1,\a_2,-\b_1,-\b_2) = \begin{bmatrix}
		1 & 2 & -2 & -1 \\
		2 & 1 & -5 & 0 \\
		3 & 1 & -9 & -1
	\end{bmatrix}
	\to \begin{bmatrix}
		1 & 0 & 0 & 3 \\
		0 & 1 & 0 & -1 \\
		0 & 0 & 1 & 1
	\end{bmatrix},
\]
解得\((x_1,x_2,x_3,x_4)^T = c (-3,1,-1,1)^T\ (c\in\mathbb{R})\).
因此\[
	\g = c (x_3 \b_1 + x_4 \b_2)
	= c \begin{bmatrix}
		-1 \\ -5 \\ -8
	\end{bmatrix}
	= k \begin{bmatrix}
		1 \\ 5 \\ 8
	\end{bmatrix}
	\quad(k\in\mathbb{R}).
\]
\end{solution}
\end{example}

\section{非齐次线性方程组的解集的结构}
数域\(K\)上\(n\)元非齐次线性方程组
\begin{equation}\label[equation-system]{equation:向量空间.非齐次线性方程组}
	x_1\a_1+x_2\a_2+\dotsb+x_n\a_n=\b
\end{equation}
的每一个解都是\(K^n\)中的一个向量,
我们称这个向量为该方程组的一个解向量.
因此这个\(n\)元非齐次线性方程组的解集\(U\)是\(K^n\)的一个子集.
当该方程组无解时,\(U\)是空集.
现在我们想要知道,当该方程组有无穷多个解时,解集\(U\)的结构如何?

我们把齐次线性方程组\(\A\x=\z\)
称为非齐次线性方程组\(\A\x=\b\)的\DefineConcept{导出组}.

\begin{proposition}
%@see: 《高等代数(第三版 上册)》(丘维声) P97. 性质1
非齐次线性方程组\(\A\x=\b\)的两个解的差是它的导出组的一个解.
\begin{proof}
设\(\X1,\X2\)是非齐次线性方程组\(\A\x=\b\)的任意两个解,即\[
	\A\X1=\b, \qquad
	\A\X2=\b,
\]
则\[
	\A(\X1-\X2)=\A\X1-\A\X2=\b-\b=\z;
\]
所以\(\X1-\X2\)是\(\A\x=\z\)的解.
\end{proof}
\end{proposition}

\begin{proposition}
%@see: 《高等代数(第三版 上册)》(丘维声) P97. 性质2
非齐次线性方程组\(\A\x=\b\)的一个解与它的导出组的一个解之和仍是它的一个解.
\begin{proof}
\def\ma{\vb{\xi}}
\def\mb{\vb{\zeta}}
设\(\ma\)是\(\A\x=\b\)的一个解,\(\mb\)是\(\A\x=\z\)的一个解,即\[
	\A\ma=\b, \qquad
	\A\mb=\z,
\]
则\[
	\A(\ma+\mb)=\A\ma+\A\mb=\b+\z=\b;
\]
所以\(\ma+\mb\)是\(\A\x=\b\)的解.
\end{proof}
\end{proposition}

\begin{theorem}\label{theorem:向量空间.非齐次线性方程组的解集的结构}
%@see: 《高等代数(第三版 上册)》(丘维声) P97. 定理1
数域\(K\)上\(n\)元非齐次线性方程组 \labelcref{equation:向量空间.非齐次线性方程组} 有解,
则它的解集\(U\)为
\begin{equation}
	\Set{ \x_0+\x \given \x \in W },
\end{equation}
其中\(\x_0\)是这个非齐次线性方程组的一个解,
\(W\)是它的导出组的解空间.
\end{theorem}

我们把\cref{theorem:向量空间.非齐次线性方程组的解集的结构} 中的解向量\(\x_0\)
称为“\cref{equation:向量空间.非齐次线性方程组} 的\DefineConcept{特解}(particular solution)”.

\begin{corollary}
%@see: 《高等代数(第三版 上册)》(丘维声) P98. 推论2
如果\(n\)元非齐次线性方程组 \labelcref{equation:向量空间.非齐次线性方程组} 有解,
则它的解唯一的充分必要条件是它的导出组只有零解.
\end{corollary}

于是当非齐次线性方程组有无穷多个解时,
它的导出组必有非零解.
此时取导出组的一个基础解系\(\AutoTuple{k}{n-r}\),
其中\(r\)是导出组的系数矩阵的秩,
则非齐次线性方程组的解集为\[
	U = \Set{ \x_0+k_1\x_1+k_2\x_2+ \dotsb +k_{n-r}\x_{n-r} \given \AutoTuple{k}{n-r} \in K },
\]
其中\(\x_0\)是非齐次线性方程组的一个特解.

解集\(U\)的代表元素\[
	\x_0+k_1\x_1+k_2\x_2+ \dotsb +k_{n-r}\x_{n-r}
	\quad(\AutoTuple{k}{n-r} \in K)
\]
称为“\cref{equation:向量空间.非齐次线性方程组} 的通解”.

\begin{example}
求线性方程组\[
	\left\{ \begin{array}{*{9}{r}}
		x_1 &-& 2 x_2 &-& x_3 &+& 2 x_4 &=& 4 \\
		2 x_1 &-& 2 x_2 &-& 3 x_3 && &=& 2 \\
		4 x_1 &-& 2 x_2 &-& 7 x_3 &-& 4 x_4 &=& -2
	\end{array} \right.
\]的通解.
\begin{solution}
写出增广矩阵\(\wA\),并作初等行变换化简
\begin{align*}
	\wA
	&= \begin{bmatrix}
		1 & -2 & -1 & 2 & 4 \\
		2 & -2 & -3 & 0 & 2 \\
		4 & -2 & -7 & -4 & -2
	\end{bmatrix}
	\xlongrightarrow{\begin{array}{c}
		-2\times\text{(1行)}+\text{(2行)} \\
		-4\times\text{(1行)}+\text{(3行)}
	\end{array}}
	\begin{bmatrix}
		1 & -2 & -1 & 2 & 4 \\
		0 & 2 & -1 & -4 & -6 \\
		0 & 6 & -3 & -12 & -18
	\end{bmatrix} \\
	&\to \begin{bmatrix}
		1 & -2 & -1 & 2 & 4 \\
		0 & 2 & -1 & -4 & -6 \\
		0 & 0 & 0 & 0 & 0
	\end{bmatrix}
	\to \begin{bmatrix}
		1 & 0 & -2 & -2 & -2 \\
		0 & 2 & -2 & -4 & -6 \\
		0 & 0 & 0 & 0 & 0
	\end{bmatrix}
	= \wB
	= (\B,\g),
\end{align*}
故\(\rank\A = \rank\wA = 2\),于是原方程组有解.

解同解方程组\[
	\left\{ \begin{array}{*{9}{c}}
		x_1 && &-& 2 x_3 &-& 2 x_4 &=& -2 \\
		&& 2 x_2 &-& x_3 &-& 4 x_4 &=& -6
	\end{array} \right..
\]
令\(x_3 = 0\),\(x_4 = 0\),
解得\(x_1 = -2\),\(x_2 = -3\),得特解\[
	\x_0 = (-2,-3,0,0)^T.
\]

又因为导出组的基础解系含\(4 - \rank\A = 2\)个向量.
将\(x_3,x_4\)的两组值\((2,0),(0,1)\)分别代入\[
	\left\{ \begin{array}{*{9}{c}}
		x_1 && &-& 2 x_3 &-& 2 x_4 &=& 0 \\
		&& 2 x_2 &-& x_3 &-& 4 x_4 &=& 0
	\end{array} \right.
\]
得基础解系
\(\x_1 = (4,1,2,0)^T\),
\(\x_2 = (2,2,0,1)^T\).

于是原方程的通解为\[
	\x = \x_0 + k_1 \x_1 + k_2 \x_2
	= \begin{bmatrix} -2 \\ -3 \\ 0 \\ 0 \end{bmatrix}
	+ k_1 \begin{bmatrix} 4 \\ 1 \\ 2 \\ 0 \end{bmatrix}
	+ k_2 \begin{bmatrix} 2 \\ 2 \\ 0 \\ 1 \end{bmatrix},
\]
其中\(k_1,k_2\)是任意常数.
\end{solution}
\end{example}

\begin{example}
写出线性方程组\[
	\left\{ \begin{array}{l}
		x_1 - x_2 = a_1 \\
		x_2 - x_3 = a_2 \\
		x_3 - x_4 = a_3 \\
		x_4 - x_1 = a_4
	\end{array} \right.
\]有解的充分必要条件,并求解.
\begin{solution}
对增广矩阵\(\wA\)作初等行变换化简
\begin{align*}
	\wA
	&= \begin{bmatrix}
		1 & -1 & 0 & 0 & a_1 \\
		0 & 1 & -1 & 0 & a_2 \\
		0 & 0 & 1 & -1 & a_3 \\
		-1 & 0 & 0 & 0 & a_4
	\end{bmatrix} \to \begin{bmatrix}
		1 & -1 & 0 & 0 & a_1 \\
		0 & 1 & -1 & 0 & a_2 \\
		0 & 0 & 1 & -1 & a_3 \\
		0 & 0 & 0 & 0 & \sum_{i=1}^4 a_i
	\end{bmatrix} \\
	&\to \begin{bmatrix}
		1 & 0 & 0 & -1 & a_1 + a_2 + a_3 \\
		0 & 1 & 0 & -1 & a_2 + a_3 \\
		0 & 0 & 1 & -1 & a_3 \\
		0 & 0 & 0 & 0 & \sum_{i=1}^4 a_i
	\end{bmatrix}.
\end{align*}
可见\(\rank\A = 3\).
方程组有解的充分必要条件是\(\rank\wA = \rank\A = 3\),
那么充分必要条件就是\(\sum_{i=1}^4 a_i = 0\).

当方程组有解时,通解为\[
	\begin{bmatrix}
		a_1 + a_2 + a_3 \\ a_2 + a_3 \\ a_3 \\ 0
	\end{bmatrix}
	+ k \begin{bmatrix}
		1 \\ 1 \\ 1 \\ 1
	\end{bmatrix},
\]
其中\(k\)为任意常数.
\end{solution}
\end{example}

至此,我们讨论了线性方程组的解的存在性、解的性质、解的结构及求解方法,建立起了线性方程组的完整理论.
解线性方程组是线性代数的基本问题之一,现代科学技术方面用到的数学问题也有很多要归结到解线性方程组.

\begin{example}
设\(\X0\)是非齐次线性方程组\(\A\x=\b\)的一个解,
\(\AutoTuple{\x}{n-r}\)是其导出组\(\A\x=\z\)的一个基础解系.
证明:\(\X0,\AutoTuple{\x}{n-r}\)线性无关.
\begin{proof}
因为\(\AutoTuple{\x}{n-r}\)是其导出组\(\A\x=\z\)的一个基础解系,
根据基础解系的定义,显然有\(\AutoTuple{\x}{n-r}\)线性无关.
假设\(\X0,\AutoTuple{\x}{n-r}\)线性相关,
那么\(\X0\)可由\(\AutoTuple{\x}{n-r}\)线性表出,
即存在数\(k_1,k_2,\dotsc,k_{n-r}\)使得\[
	\X0 = k_1 \X1 + k_2 \X2 + \dotsb + k_{n-r} \X{n-r},
\]
进而有\begin{align*}
	&\A\X0 = \A(k_1 \X1 + k_2 \X2 + \dotsb + k_{n-r} \X{n-r}) \\
	&= k_1 \A\X1 + k_2 \A\X2 + \dotsb + k_{n-r} \A\X{n-r}
	= \z + \z + \dotsb + \z = \z,
\end{align*}
即\(\X0\)是\(\A\x=\z\)的一个解,
这与\(\X0\)是\(\A\x=\b\neq\z\)的一个解矛盾,
所以\(\X0,\AutoTuple{\x}{n-r}\)线性无关.
\end{proof}
\end{example}

\begin{example}
设线性方程组\(\A\x=\b\)的增广矩阵\(\wA = (\A,\b)\)是一个\(n\)阶可逆矩阵.
证明:方程组无解.
\begin{proof}
因为\(\wA\)是\(n\)阶方阵,
所以\(\A\)是\(n \times (n-1)\)矩阵,
从而\(\rank\A \leq \min\{n-1,n\} = n-1\).
又因为\(\wA\)可逆,所以\(\rank\wA = n\).
因为\(\rank\wA = n > n-1 \geq \rank\A\),
所以方程组\(\A\x=\b\)无解.
\end{proof}
\end{example}

\section{矩阵乘积的秩}
\begin{theorem}
设矩阵\(\A \in M_{m \times n}(K),
\B \in M_{n \times m}(K)\).
如果\(m > n\),
那么\(\abs{\A\B} = 0\).
\begin{proof}
当\(m>n\)时,
根据\cref{theorem:线性方程组.矩阵的秩的性质2},有\[
	\rank\A,\rank\B \leq \min\{m,n\} = n;
\]
再根据\cref{theorem:线性方程组.矩阵乘积的秩},有\[
	\rank(\A\B) \leq \min\{\rank\A,\rank\B\} = n < m,
\]
也就是说,矩阵\(\A\B\)不满秩;
那么根据\cref{theorem:向量空间.满秩方阵的行列式非零} 可知\(\abs{\A\B} = 0\).
\end{proof}
\end{theorem}

\begin{theorem}\label{theorem:线性方程组.矩阵乘积的秩}
%@see: 《线性代数》(张慎语、周厚隆) P78. 定理7
%@see: 《高等代数(第三版 上册)》(丘维声) P121. 定理1
设\(\A\in M_{s\times n}(K),
\B\in M_{n\times m}(K)\),
那么\[
	\rank(\A\B) \leq \min\{\rank\A,\rank\B\}.
\]
\begin{proof}
记\(\C = \A\B\),
显然\(\rank(\A\B)=\rank_C\C\).
将\(\C\)、\(\B\)分别进行列分块得\[
	\C = (\AutoTuple{\g}{n}),
	\qquad
	\B = (\AutoTuple{\b}{n}),
\]
则\[
	(\AutoTuple{\g}{n}) = \A (\AutoTuple{\b}{n}) = (\AutoTuple{\A\b}{n}),
\]
于是\(\g_i = \A \b_i\ (i=1,2,\dotsc,n)\).

假设\(\rank\B = t<m\),
那么由\cref{example:向量空间.若部分组向量个数多于全组的秩则部分组必线性相关},
\(\B\)的任意\(t+1\)个列向量\(\b_{k_1},\b_{k_2},\dotsc,\b_{k_{t+1}}\)线性相关,
也就是说,存在不全为零的数\(l_1,l_2,\dotsc,l_{t+1}\in K\),使得\[
	l_1 \b_{k_1} + l_2 \b_{k_2} + \dotsb + l_{t+1} \b_{k_{t+1}} = \z.
\]
因此\[
	l_1 \g_{k_1} + l_2 \g_{k_2} + \dotsb + l_{t+1} \g_{k_{t+1}}
	= \A(l_1 \b_{k_1} + l_2 \b_{k_2} + \dotsb + l_{t+1} \b_{k_{t+1}})
	= \z,
\]
这就是说\(\C\)的任意\(t+1\)个列向量也线性相关,
那么\(\rank_C\C \ngeq t+1\),
\(\rank(\A\B)\leq t\),
即\(\rank(\A\B) \leq \rank\B\).
利用这个结论,我们还可以得到\[
	\rank(\A\B)
	= \rank(\A\B)^T
	= \rank(\B^T \A^T)
	\leq \rank \A^T
	= \rank \A.
\]
综上所述\(\rank(\A\B) \leq \min\{\rank\A,\rank\B\}\).
\end{proof}
\end{theorem}

\begin{corollary}
设\(\A\)是一个\(s \times n\)矩阵,\(\P\)、\(\Q\)分别是\(s\)阶和\(n\)阶可逆矩阵,则\[
\rank\A = \rank(\P\A) = \rank(\A\Q) = \rank(\P\A\Q).
\]
\begin{proof}
因为\(\A = (\P^{-1} \P) \A = \P^{-1} (\P \A)\),由\cref{theorem:线性方程组.矩阵乘积的秩},有\[
\rank\A = \rank(\P^{-1}(\P\A)) \leq \rank(\P\A) \leq \rank\A,
\]所以\(\rank\A = \rank(\P\A)\);同理可得\(\rank\A = \rank(\A\Q) = \rank(\P\A\Q)\).
\end{proof}
\end{corollary}

\begin{theorem}
矩阵\(\A\)满足\(\rank\A=r\)的充要条件是:存在可逆矩阵\(\P,\Q\),使得\[
\P \A \Q = \begin{bmatrix}
\E_r & \z \\ \z & \z
\end{bmatrix} = \B.
\]\rm
称矩阵\(\B\)为\(\A\)的\DefineConcept{等价标准型}.
\begin{proof}
充分性.如果可逆矩阵\(\P,\Q\)使得\[
\P\A\Q = \begin{bmatrix}
\E_r & \z \\ \z & \z
\end{bmatrix},
\]
把上式等号左边的可逆矩阵\(\P\)、\(\Q\)分别视作对矩阵\(\A\)的初等行变换和初等列变换,
那么,根据\cref{theorem:线性方程组.初等变换不变秩},
所得矩阵\(\B\)的秩与原矩阵\(\A\)相同,
即\[
\rank\A = \rank\B = r.
\qedhere
\]
%\cref{theorem:线性方程组.非零矩阵可经初等行变换化为若尔当阶梯形矩阵}
\end{proof}
\end{theorem}

\begin{theorem}
设\(\A\)与\(\B\)都是\(s \times n\)矩阵,则\(\A \cong \B\)的充要条件是:\(\rank\A = \rank\B\).
\begin{proof}
必要性.因为\(\A\)可经一系列初等变换化为\(\B\),根据\cref{theorem:线性方程组.初等变换不变秩},初等变换不改变矩阵的秩,所以\(\rank\A = \rank\B\).

充分性.已知\(\rank\A = \rank\B = r\).对\(\A\)作初等行变换可将其化简为仅前\(r\)行不为零的阶梯形矩阵\(\C\),同样对\(\C\)作初等列变换可化简为\(\A\)的等价标准型.
对\(\B\)也可作初等变换化为等价标准型.
那么存在\(s\)阶可逆矩阵\(\P_1\)和\(\P_2\),存在\(n\)阶可逆矩阵\(\Q_1\)和\(\Q_2\),使得\[
\P_1 \A \Q_1 = \P_2 \A \Q_2 = \begin{bmatrix} \E_r & \z \\ \z & \z \end{bmatrix},
\]令\(\P = \P_2^{-1} \P_1\),\(\Q = \Q_1 \Q_2^{-1}\),则\(\P\)和\(\Q\)可逆,\(\P \A \Q = \B\),从而\(\A \cong \B\).
\end{proof}
\end{theorem}

\begin{example}
试证:\begin{equation}
	\rank\begin{bmatrix} \A & \z \\ \z & \B \end{bmatrix} = \rank\A + \rank\B.
	\label{equation:矩阵的秩.分块矩阵的秩的等式1}
\end{equation}
\begin{proof}
设\(\rank\A = r_1, \rank\B = r_2\),则存在可逆矩阵\(\P_1,\Q_1,\P_2,\Q_2\),使得\[
\P_1 \A \Q_1 = \begin{bmatrix}
\E_{r_1} & \z \\
\z & \z
\end{bmatrix},
\qquad
\P_2 \B \Q_2 = \begin{bmatrix}
\E_{r_2} & \z \\
\z & \z
\end{bmatrix},
\]于是\[
\begin{bmatrix}
\P_1 & \z \\
\z & \P_2
\end{bmatrix} \begin{bmatrix}
\A & \z \\
\z & \B
\end{bmatrix} \begin{bmatrix}
\Q_1 & \z \\
\z & \Q_2
\end{bmatrix}
= \begin{bmatrix}
\P_1 \A \Q_1 & \z \\
\z & \P_2 \B \Q_2
\end{bmatrix}
= \begin{bmatrix}
\E_{r_1} \\
& \z \\
& & \E_{r_2} \\
& & & \z
\end{bmatrix}.
\]显然上式最右端矩阵的秩为\(r_1+r_2\),而最左端的矩阵\(\begin{bmatrix}
\P_1 & \z \\
\z & \P_2
\end{bmatrix}\)和\(\begin{bmatrix}
\Q_1 & \z \\
\z & \Q_2
\end{bmatrix}\)都是可逆矩阵,与之相乘不改变矩阵\(\begin{bmatrix} \A & \z \\ \z & \B \end{bmatrix}\)的秩,说明\(\rank\begin{bmatrix} \A & \z \\ \z & \B \end{bmatrix} = r_1 + r_2 = \rank\A + \rank\B\).
\end{proof}
\end{example}

\begin{example}
试证:\begin{gather}
	\rank\begin{bmatrix} \A & \C \\ \z & \B \end{bmatrix} \geq \rank\A + \rank\B,
	\label{equation:矩阵的秩.分块矩阵的秩的不等式1} \\
	\rank\begin{bmatrix} \A & \z \\ \D & \B \end{bmatrix} \geq \rank\A + \rank\B.
	\label{equation:矩阵的秩.分块矩阵的秩的不等式2}
\end{gather}
\end{example}

\begin{example}
设\(\A,\B\in M_n(K)\),\(\A\B=\B\A\),证明:\[
	\rank(\A+\B)\leq\rank\A+\rank\B-\rank(\A\B).
\]
\begin{proof}
考虑\[
	\begin{bmatrix}
		\E & \E \\
		\z & \E
	\end{bmatrix}
	\begin{bmatrix}
		\A & \z \\
		\z & \B
	\end{bmatrix}
	\begin{bmatrix}
		\E & -\B \\
		\E & \A
	\end{bmatrix}
	= \begin{bmatrix}
		\A+\B & -\A\B+\B\A \\
		\B & \B\A
	\end{bmatrix}
	= \begin{bmatrix}
		\A+\B & \z \\
		\B & \A\B
	\end{bmatrix}.
\]
由\cref{equation:矩阵的秩.分块矩阵的秩的等式1,%
equation:矩阵的秩.分块矩阵的秩的不等式2} 可知\[
	\rank\A+\rank\B
	= \rank\begin{bmatrix}
		\A & \z \\
		\z & \B
	\end{bmatrix}
	\geq \rank\begin{bmatrix}
		\A+\B & \z \\
		\B & \B\A
	\end{bmatrix}
	\geq \rank(\A+\B) + \rank(\A\B).
	\qedhere
\]
\end{proof}
\end{example}

\begin{example}
设矩阵\(\A \in M_{s \times n}(P)\),矩阵\(\B \in M_{s \times m}(P)\),则
\begin{equation}
	\rank(\A,\B) \leq \rank\A + \rank\B.
\end{equation}
\begin{proof}
\def\as{\AutoTuple{\a}{n}}
\def\bs{\AutoTuple{\b}{m}}
\def\asi{\a_{i_1},\dotsc,\a_{i_r}}
\def\bsj{\b_{j_1},\dotsc,\b_{j_t}}
设\(\rank\A = r\),\(\rank\B = t\).
对\(\A\)、\(\B\)分别按列分块得\[
	\A = (\as),
	\qquad
	\B = (\bs),
\]
则\[
	(\A,\B) = (\as,\bs),
\]且\[
	\rank\{\as\} = r,
	\quad
	\rank\{\bs\} = t.
\]

由于\(\as\)可由其极大线性无关组\[
	\asi
\]线性表出,
\(\bs\)可由其极大线性无关组\[
	\bsj
\]线性表出,
故\[
	V_1=\{\as,\bs\}
\]可由向量组\[
	V_2=\{\asi,\bsj\}
\]线性表出,
则\[
	\rank V_1
	\leq
	\rank V_2
	\leq
	r+t;
\]
于是\(\rank(\A,\B) = \rank V_1 \leq r+t\).
\end{proof}
\end{example}

\begin{example}
设\(\A\)、\(\B\)都是\(s \times n\)矩阵,证明:\(\rank(\A+\B) \leq \rank\A + \rank\B\).
\begin{proof}
\def\asi{\a_{i_1},\a_{i_2},\dotsc,\a_{i_r}}
\def\bsj{\b_{j_1},\b_{j_2},\dotsc,\b_{j_t}}
设\(\rank\A = r\),\(\rank\B = t\).对\(\A\)、\(\B\)分别按列分块得\[
\A = (\AutoTuple{\a}{n}), \qquad
\B = (\AutoTuple{\b}{m}),
\]则\[
\A + \B = (\a_1 + \b_1,\a_2 + \b_2,\dotsc,\a_n + \b_n).
\]
由于\(\AutoTuple{\a}{n}\)可由其极大线性无关组\(\asi\)线性表出,
\(\AutoTuple{\b}{m}\)可由其极大线性无关组\(\bsj\)线性表出,
故\[
\a_1 + \b_1,\a_2 + \b_2,\dotsc,\a_n + \b_n
\]可由向量组\[
\asi,\bsj
\]线性表出,
结论显然成立.
\end{proof}
\end{example}

\begin{example}
设\(\A\)是\(n\)阶矩阵,证明:\(n \leq \rank(\A + \E) + \rank(\A - \E)\).
\begin{proof}
可以证
\begin{align*}
\rank(\A + \E) + \rank(\A - \E)
&= \rank(\A + \E) + \rank[-(\A - \E)] \\
&\geq \rank\{(\A + \E) + [-(\A - \E)]\} \\
&= \rank(2\E) = \rank\E = n.
\qedhere
\end{align*}
\end{proof}
\end{example}

\begin{example}
设\(\A\)是\(s \times n\)矩阵(\(s \neq n\)),证明:\(\det(\A \A^T) \det(\A^T \A) = 0\).
\begin{proof}
由题有,\(\A^T \A\)是\(n\)阶矩阵,\(\A \A^T\)是\(s\)阶矩阵.

当\(s < n\)时,\(\rank(\A^T \A) \leq \min\{ \rank\A, \rank\A^T \} = \rank\A = \rank\A^T \leq \min\{s,n\} = s < n\),\(n\)阶方阵\(\A^T \A\)不满秩,那么\(\det(\A^T \A) = 0\).

同理,当\(n < s\)时,\(\rank(\A \A^T) \leq \min\{s,n\} = n < s\),那么\(\det(\A \A^T) = 0\).

综上所述,\(s \neq n\)时总有\(\det(\A \A^T) \det(\A^T \A) = 0\).
\end{proof}
\end{example}

\begin{example}
设\(\A\)是\(m \times n\)矩阵,\(\B\)是\(n \times m\)矩阵,\(\E\)是\(m\)阶单位矩阵.
已知\(\A\B = \E\),求\(\rank\A,\rank\B\).
\begin{solution}
假设\(m > n\),则由\cref{theorem:线性方程组.矩阵的秩的性质2} 可知\[
\rank\A,\rank\B \leq \min\{m,n\} = n.
\]再由\cref{theorem:线性方程组.矩阵乘积的秩} 可知\[
\rank(\A\B) \leq \min\{\rank\A,\rank\B\} \leq n < m.
\]但\(\rank(\A\B) = \rank\E = m\),矛盾!
由此可知,必有\(m \leq n\),那么\[
\rank\A,\rank\B \leq \min\{m,n\} = m,
\]因此\begin{align*}
m = \rank(\A\B) \leq \min\{\rank\A,\rank\B\} \leq m
&\implies
\min\{\rank\A,\rank\B\} = m \\
&\implies
m \leq \rank\A,\rank\B \leq m \\
&\implies \rank\A=\rank\B=m.
\end{align*}
\end{solution}
\end{example}



\subsection{重要不等式}
\begin{theorem}[西尔维斯特不等式]
设\(\A\)是\(m \times n\)矩阵,\(\B\)是\(n \times t\)矩阵.证明:\begin{equation}\label{equation:线性方程组.西尔维斯特不等式}
\rank\A + \rank\B - n \leq \rank(\A \B).
\end{equation}
\begin{proof}
\def\AA{\P_1 \begin{bmatrix} \E_r & \z \\ \z & \z \end{bmatrix} \Q_1}
\def\BB{\P_2 \begin{bmatrix} \E_s & \z \\ \z & \z \end{bmatrix} \Q_2}
\def\CC#1{\C_{#1}}
设\(\rank\A = r\),\(\rank\B = s\),则存在可逆矩阵\(\P_1,\P_2,\Q_1,\Q_2\)使得\[
\A = \AA,
\quad
\B = \BB,
\]所以\[
\A \B = \AA \BB.
\]若\(\Q_1 \P_2 = \begin{bmatrix} \CC1 & \CC2 \\ \CC3 & \CC4 \end{bmatrix}\)(其中\(\CC1\)是\(r \times s\)矩阵),则\[
\A \B = \P_1 \begin{bmatrix} \CC1 & \z \\ \z & \z \end{bmatrix} \Q_2.
\]
注意到从任意一个矩阵中划去一行(或一列),\(\A\)的秩至多减少1,那么可以把\(\CC1\)看作将一个\(n\)阶方阵划去\(n-r\)行,并划去\(n-s\)列得到的,即\[
\rank\CC1 \geq n - (n-r) - (n-s) = r+s-n,
\]所以\(\rank\A + \rank\B - n \leq \rank(\A \B) = \rank\CC1\).
\end{proof}
\end{theorem}

上面的结果和\cref{theorem:线性方程组.矩阵乘积的秩} 一起可以确定矩阵乘积的秩的区间:\[
\rank\A + \rank\B - n \leq \rank(\A\B) \leq \min\{\rank\A,\rank\B\}.
\]
进一步,如果有\(\A\B=\z\),那么\[
\rank\A + \rank\B \leq n.
\]

我们还可以将\hyperref[equation:线性方程组.西尔维斯特不等式]{西尔维斯特不等式}进行如下的推广:\begin{equation}\label{equation:线性方程组.弗罗贝尼乌斯不等式}
\rank(\A\B\C) \geq \rank(\A\B) + \rank(\B\C) - \rank\B.
\end{equation}称其为\DefineConcept{弗罗贝尼乌斯不等式}.


\begin{example}
计算行列式\(\det\D\),其中\[
\D = \begin{bmatrix}
	1 & \cos(\alpha_1-\alpha_2) & \cos(\alpha_1-\alpha_3) & \dots & \cos(\alpha_1-\alpha_n) \\
	\cos(\alpha_1-\alpha_2) & 1 & \cos(\alpha_2-\alpha_3) & \dots & \cos(\alpha_2-\alpha_n) \\
	\cos(\alpha_1-\alpha_3) & \cos(\alpha_2-\alpha_3) & 1 & \dots & \cos(\alpha_3-\alpha_n) \\
	\vdots & \vdots & \vdots & & \vdots \\
	\cos(\alpha_1-\alpha_n) & \cos(\alpha_2-\alpha_n) & \cos(\alpha_3-\alpha_n) & \dots & 1
\end{bmatrix}.
\]
\begin{solution}
记\(\D = (d_{ij})_n\).
由\cref{equation:函数.三角函数.和积互化公式2} 可知\[
	d_{ij} = \cos(\alpha_i-\alpha_j)
	= \cos\alpha_i\cos\alpha_j+\sin\alpha_i\sin\alpha_j,
	\quad i,j=1,2,\dotsc,n.
\]
记\[
	\A = \begin{bmatrix}
		\cos\alpha_1 & \cos\alpha_2 & \dots & \cos\alpha_n \\
		\sin\alpha_1 & \sin\alpha_2 & \dots & \sin\alpha_n
	\end{bmatrix},
\]那么\(\D = \A^T \A\).

由\cref{theorem:线性方程组.矩阵的秩的性质2} 可知,
\(\rank\A = \rank\A^T \leq \min\{n,2\}\).
由\cref{theorem:线性方程组.矩阵乘积的秩} 可知,\[
	\rank(\A^T\A) \leq \min\left\{\rank\A^T,\rank\A\right\} = \rank\A.
\]

当\(n=1\)时,\(\abs{\D}=1\).

当\(n=2\)时,\[
	\abs{\D}
	= \begin{vmatrix}
		1 & \cos(\alpha_1-\alpha_2) \\
		\cos(\alpha_1-\alpha_2) & 1
	\end{vmatrix}
	= 1 - \cos^2(\alpha_1-\alpha_2).
\]

当\(n>2\)时,
\(\rank(\A^T\A) \leq \rank\A \leq2\),
\(\D = \A^T \A\)不满秩,
故由\cref{theorem:向量空间.满秩方阵的行列式非零} 有\(\abs{\D}=0\).
\end{solution}
\end{example}

\begin{example}
设\(\a_1=(1,2,-1,0)^T,\a_2=(1,1,0,2)^T,\a_3=(2,1,1,a)^T\).
若\(\dim\opair{\AutoTuple{\a}{3}}=2\),求\(a\).
\begin{solution}
除了利用\cref{equation:线性方程组.子空间的维数与向量组的秩的联系} 在将矩阵\[
	\A = (\a_1,\a_2,\a_3)
	= \begin{bmatrix}
		1 & 1 & 2 \\
		2 & 1 & 1 \\
		-1 & 0 & 1 \\
		0 & 2 & a
	\end{bmatrix}
\]化为阶梯形矩阵以后,
根据\(\rank\{\AutoTuple{\a}{3}\}=2\)求出\(a\)的值这种方法以外,
我们还可以利用本节\hyperref[definition:线性方程组.矩阵的秩的定义]{矩阵的秩的定义},
得出\(\rank\A=\dim\opair{\AutoTuple{\a}{3}}=2\)这一结论,
从而根据\cref{definition:线性方程组.矩阵的秩的定义} 可知,
\(\A\)中任意3阶子式全都为零.
对于\(\A\)这么一个\(4\times3\)矩阵,
任意去掉不含\(a\)的一行(不妨去掉第一行)得到一个行列式必为零:\[
	\begin{vmatrix}
	2 & 1 & 1 \\
	-1 & 0 & 1 \\
	0 & 2 & a
	\end{vmatrix}
	= a - 6 = 0,
\]
解得\(a = 6\).
\end{solution}
\end{example}

\begin{example}
%@see: 《高等代数(第三版 上册)》(丘维声) P122. 命题2
设\(\A \in M_{m \times n}(\mathbb{R})\).
求证:\begin{equation}
	\rank\A = \rank(\A\A^T).
\end{equation}
\begin{proof}
设\(\rank\A=r\),
由\hyperref[definition:线性方程组.矩阵的秩的定义]{矩阵的秩的定义}可知,
\(\A\)有\(r\)阶子式不等于零,即\[
	\MatrixMinor\A{
		\AutoTuple{i}{r} \\
		\AutoTuple{j}{r}
	} \neq 0.
	\eqno(1)
\]
由\hyperref[equation:线性方程组.柯西比内公式]{柯西--比内公式}可知
\begin{align*}
	\MatrixMinor{(\A\A^T)}{
		\AutoTuple{i}{r} \\
		\AutoTuple{i}{r}
	}
	&= \sum\limits_{1 \leq k_1 < k_2 < \dotsb < k_r \leq n}
	\MatrixMinor\A{
		\AutoTuple{i}{r} \\
		\AutoTuple{k}{r}
	}
	\MatrixMinor{\A^T}{
		\AutoTuple{k}{r} \\
		\AutoTuple{i}{r}
	} \\
	&= \sum\limits_{1 \leq k_1 < k_2 < \dotsb < k_r \leq n}
	\left[
		\MatrixMinor\A{
			\AutoTuple{i}{r} \\
			\AutoTuple{k}{r}
		}
	\right]^2.
	\tag2
\end{align*}
因为\(\A \in M_{m \times n}(\mathbb{R})\),
所以\(\A\)的任意子式都是实数,于是有
\[
	\left[
		\MatrixMinor\A{
			\AutoTuple{i}{r} \\
			\AutoTuple{k}{r}
		}
	\right]^2
	\geq 0.
	\eqno(3)
\]
由(1)式可知,
\[
	\left[
		\MatrixMinor\A{
			\AutoTuple{i}{r} \\
			\AutoTuple{j}{r}
		}
	\right]^2
	> 0.
	\eqno(4)
\]
由于在(2)式中,
求和指标可以\((\AutoTuple{k}{r})\)取到\((\AutoTuple{j}{r})\),
所以(4)式是(2)式中的一项.
综上所述
\[
	\MatrixMinor{(\A\A^T)}{
		\AutoTuple{i}{r} \\
		\AutoTuple{i}{r}
	}
	= \sum\limits_{1 \leq k_1 < k_2 < \dotsb < k_r \leq n}
	\left[
		\MatrixMinor\A{
			\AutoTuple{i}{r} \\
			\AutoTuple{k}{r}
		}
	\right]^2
	> 0,
\]
这就说明\(\rank(\A\A^T) \geq r\).
又因为\(\rank(\A\A^T) \leq \rank\A = r\),
所以\(\rank(\A\A^T) = \rank\A = r\).
\end{proof}
\end{example}

\section{西尔维斯特不等式}
\begin{theorem}
设\(\A \in M_{s \times n}(K),
\B \in M_{n \times t}(K)\).
证明:\begin{equation}\label{equation:线性方程组.西尔维斯特不等式}
	\rank\A + \rank\B - n \leq \rank(\A\B).
\end{equation}
\begin{proof}
由\cref{equation:矩阵的秩.分块矩阵的秩的等式1},\[
	\rank\begin{bmatrix}
		\E_n & \z \\
		\z & \A\B
	\end{bmatrix}
	= n + \rank(\A\B).
	\eqno(1)
\]
又因为\[
	\begin{bmatrix}
		\B & \E_n \\
		\z & \A
	\end{bmatrix}
	= \begin{bmatrix}
		\E_n & \z \\
		\A & \E_s
	\end{bmatrix}
	\begin{bmatrix}
		\E_n & \z \\
		\z & \A\B
	\end{bmatrix}
	\begin{bmatrix}
		\E_n & -\B \\
		\z & \E_t
	\end{bmatrix}
	\begin{bmatrix}
		\z & \E_s \\
		-\E_t & \z
	\end{bmatrix},
\]
而\[
	\begin{bmatrix}
		\E_n & \z \\
		\A & \E_s
	\end{bmatrix}, \qquad
	\begin{bmatrix}
		\E_n & -\B \\
		\z & \E_t
	\end{bmatrix},
	\quad\text{和}\quad
	\begin{bmatrix}
		\z & \E_s \\
		-\E_t & \z
	\end{bmatrix}
\]这三个矩阵都是满秩矩阵,
所以\[
	\rank\begin{bmatrix}
		\E_n & \z \\
		\z & \A\B
	\end{bmatrix}
	= \rank\begin{bmatrix}
		\B & \E_n \\
		\z & \A
	\end{bmatrix}.
	\eqno(2)
\]
再由\cref{equation:矩阵的秩.分块矩阵的秩的不等式} 有\[
	\rank\begin{bmatrix}
		\B & \E_n \\
		\z & \A
	\end{bmatrix}
	\geq \rank\A+\rank\B.
	\eqno(3)
\]
因此,\(\rank\A + \rank\B \leq n + \rank(\A\B)\).
%@see: https://math.stackexchange.com/a/2414197/591741
%@see: http://www.m-hikari.com/imf-password2009/33-36-2009/luIMF33-36-2009.pdf
\end{proof}
\end{theorem}

我们把\cref{equation:线性方程组.西尔维斯特不等式} 称为“西尔维斯特不等式”.

进一步,如果有\(\A\B=\z\),那么\[
	\rank\A + \rank\B \leq n.
\]

我们还可以将\hyperref[equation:线性方程组.西尔维斯特不等式]{西尔维斯特不等式}进行如下的推广.
\begin{theorem}
\begin{equation}\label{equation:线性方程组.弗罗贝尼乌斯不等式}
	\rank(\A\B\C) \geq \rank(\A\B) + \rank(\B\C) - \rank\B.
\end{equation}
\begin{proof}
利用初等变换,有\[
	\begin{bmatrix}
		\B & \z \\
		\z & \A\B\C
	\end{bmatrix}
	\to \begin{bmatrix}
		\B & \z \\
		\A\B & \A\B\C
	\end{bmatrix}
	\to \begin{bmatrix}
		\B & -\B\C \\
		\A\B & \z
	\end{bmatrix}
	\to \begin{bmatrix}
		\B\C & \B \\
		\z & \A\B
	\end{bmatrix},
\]
于是\[
	\rank\B + \rank(\A\B\C)
	= \rank\begin{bmatrix}
		\B & \z \\
		\z & \A\B\C
	\end{bmatrix}
	= \rank\begin{bmatrix}
		\B\C & \B \\
		\z & \A\B
	\end{bmatrix}
	\geq \rank(\A\B) + \rank(\B\C).
	\qedhere
\]
\end{proof}
\end{theorem}

我们把\cref{equation:线性方程组.弗罗贝尼乌斯不等式} 称为“弗罗贝尼乌斯不等式”.

\begin{example}
%@see: 《高等代数(第三版 上册)》(丘维声) P143. 习题4.5 4.
证明:如果\(n\)阶矩阵\(\A\)是\DefineConcept{对合矩阵},
即满足\(\A^2=\E\),
则\[
	\rank(\E+\A)+\rank(\E-\A)=n.
\]
%TODO
\end{example}

\begin{example}
%@see: 《高等代数(第三版 上册)》(丘维声) P143. 习题4.5 5.
证明:如果\(n\)阶矩阵\(\A\)是\DefineConcept{幂等矩阵}(idempotent matrix),
即满足\(\A^2=\A\),
则\[
	\rank\A+\rank(\E-\A)=n.
\]
\begin{proof}
由于\[
	\A^2=\A
	\iff
	\A^2-\A=\z
	\iff
	\rank(\A^2-\A)=0.
\]
又因为利用初等变换可以得到\[
	\begin{bmatrix}
		\A & \z \\
		\z & \E_n-\A
	\end{bmatrix}
	\to \begin{bmatrix}
		\A & \z \\
		\A & \E_n-\A
	\end{bmatrix}
	\to \begin{bmatrix}
		\A & \A \\
		\A & \E_n
	\end{bmatrix}
	\to \begin{bmatrix}
		\A-\A^2 & \z \\
		\A & \E_n
	\end{bmatrix}
	\to \begin{bmatrix}
		\A-\A^2 & \z \\
		\z & \E_n
	\end{bmatrix},
\]
所以\(\rank\A+\rank(\E_n-\A)
=\rank(\A-\A^2)+n
=n\).
\end{proof}
\end{example}

\begin{example}
设\(\A \in M_{m \times n}(K)\).
证明:\[
	\rank(\E_m-\A\A^T)-\rank(\E_n-\A^T\A)=m-n.
\]
\begin{proof}
因为利用初等变换可以得到\begin{align*}
	\begin{bmatrix}
		\E_m-\A\A^T & \z \\
		\z & \E_n
	\end{bmatrix}
	&\to \begin{bmatrix}
		\E_m-\A\A^T & \A \\
		\z & \E_n
	\end{bmatrix}
	\to \begin{bmatrix}
		\E_m & \A \\
		\A^T & \E_n
	\end{bmatrix} \\
	&\to \begin{bmatrix}
		\E_m & \z \\
		\A^T & \E_n-\A^T\A
	\end{bmatrix}
	\to \begin{bmatrix}
		\E_m & \z \\
		\z & \E_n-\A^T\A
	\end{bmatrix},
\end{align*}
所以\[
	\rank(\E_m-\A\A^T)+n=m+\rank(\E_n-\A^T\A),
\]
移项便得\(\rank(\E_m-\A\A^T)-\rank(\E_n-\A^T\A)=m-n\).
\end{proof}
\end{example}

\begin{example}
设\(\A,\B,\C \in M_n(K)\),
\(\rank\C=n\),
\(\A(\B\A+\C)=\z\).
证明:\[
	\rank(\B\A+\C)=n-\rank\A.
\]
\begin{proof}
因为\(\A(\B\A+\C)=\z\),
所以根据\cref{equation:线性方程组.西尔维斯特不等式} 有\[
	\rank(\B\A+\C)+\rank\A \leq n.
\]
又利用初等变换可以得到\[
	\begin{bmatrix}
		\B\A+\C & \z \\
		\z & \A
	\end{bmatrix}
	\to \begin{bmatrix}
		\B\A+\C & \B\A \\
		\z & \A
	\end{bmatrix}
	\to \begin{bmatrix}
		\C & \B\A \\
		-\A & \A
	\end{bmatrix},
\]
于是\[
	\rank(\B\C+\C)
	+\rank\A
	= \rank\begin{bmatrix}
		\B\A+\C & \z \\
		\z & \A
	\end{bmatrix}
	= \rank\begin{bmatrix}
		\C & \B\A \\
		-\A & \A
	\end{bmatrix}
	\geq \rank\C = n.
	\qedhere
\]
\end{proof}
\end{example}


\chapter{特征值,特征向量}
\section{矩阵的迹}
\begin{definition}
矩阵\(\A=(a_{ij})_{s \times n}\)
主对角线上元素之和称为\(\A\)的\DefineConcept{迹}(trace),
记作\(\tr\A\),即\[
	\tr\A \defeq \sum\limits_{i=1}^m a_{ii},
\]
其中\(m = \min\{s,n\}\).
\end{definition}

\begin{property}
已知矩阵\(\A \in M_{s \times n}(K)\),
则\(\tr\A = \tr(\A^T)\).
\end{property}

\begin{property}
已知矩阵\(\A,\B \in M_{s \times n}(K)\),则
\begin{enumerate}
	\item \(\tr(\A+\B) = \tr\A + \tr\B\);
	\item \(\tr(k \A) = k \tr\A\ (k \in P)\).
\end{enumerate}
\begin{proof}
设\(\A=(a_{ij})_{s \times n},
\B=(b_{ij})_{s \times n}\),
取\(m = \min\{s,n\}\),
那么\[
	\tr(\A+\B) = \sum\limits_{i=1}^m (a_{ii}+b_{ii})
	= \sum\limits_{i=1}^m a_{ii}
	+ \sum\limits_{i=1}^m b_{ii}
	= \tr\A + \tr\B,
\]\[
	\tr(k \A) = \sum\limits_{i=1}^m (k a_{ii})
	= k \sum\limits_{i=1}^m a_{ii}
	= k \tr\A.
\]
这也说明“矩阵的迹”具有“线性性”.
\end{proof}
\end{property}

\begin{property}
设\(\A\)是可逆矩阵,
则\(\tr(\A^*) = \abs{\A} \cdot \tr(\A^{-1})\).
\end{property}

\begin{property}
已知矩阵\(\A,\B \in M_n(K)\),
则\[
	\tr(\A\B) = \tr(\B\A).
\]
\end{property}

\begin{property}
已知矩阵\(\A \in M_{s \times n}(K)\),
则\(\tr(\A\A^T) = \tr(\A^T\A)\).
\end{property}

\begin{property}
已知矩阵\(\A,\B \in M_n(K)\),且\(\A,\B\)均为实对称矩阵,则\[
	\tr(\A\B)^2 \leq \tr(\A^2\B^2).
\]
\end{property}

\section{矩阵的特征值与特征向量}
\subsection{特征值、特征向量的概念与性质}
\begin{definition}
%@see: 《高等代数(第三版 上册)》(丘维声) P172. 定义1
设矩阵\(\A \in M_n(K)\).
如果\[
	(\exists \lambda_0 \in K)
	(\exists \x_0 \in K^n - \{\vb0\})
	[\A\x_0 = \lambda_0\x_0],
\]
则称“\(\lambda_0\)是\(\A\)的一个\DefineConcept{特征值}(eigenvalue)”,
称“\(\x_0\)是\(\A\)的属于特征值\(\lambda_0\)的一个\DefineConcept{特征向量}(eigenvector)”.
\end{definition}

容易观察到,当特征值\(\l=0\)时,\(\l\)对应的特征向量都是齐次方程\(\A\x=\z\)的解.
当\(\abs{\A}\neq0\)时,这个方程只有零解,因此,一个矩阵有特征值\(\l=0\)说明它不满秩.

\begin{property}\label{theorem:矩阵的特征值与特征向量.性质1}
若\(\X1\),\(\X2\)是\(\A\)属于同一个特征值\(\lambda_0\)的特征向量,
且\(\X1 + \X2 \neq 0\),则\(\X1 + \X2\)也是\(\A\)的属于\(\lambda_0\)的特征向量.
\begin{proof}
\(\A(\X1+\X2)=\A\X1+\A\X2=\L0\X1+\L0\X2=\L0(\X1+\X2)\).
\end{proof}
\end{property}

\begin{property}\label{theorem:矩阵的特征值与特征向量.性质2}
若\(\x_0\)是\(\A\)属于特征值\(\lambda_0\)的特征向量,\(k\)为任意非零常数,
则\(k\X0\)也是\(\A\)的属于\(\lambda_0\)的特征向量.
\begin{proof}
因为\(k\X0\neq\z\),\(\A(k\X0)=k(\A\X0)=k(\L0\X0)=\L0(k\X0)\),
所以\(k\X0\)也是\(\A\)的属于\(\lambda_0\)的特征向量.
\end{proof}
\end{property}

\begin{remark}
由\cref{theorem:矩阵的特征值与特征向量.性质1,theorem:矩阵的特征值与特征向量.性质2} 可知,
\(\A\)的属于同一个特征值\(\lambda_0\)的
特征向量\(\AutoTuple{\x}{t}\)的非零线性组合
\(\sum_{i=1}^t{k_i \X i}\)
也是\(\A\)的属于\(\lambda_0\)的特征向量.
\end{remark}

\begin{example}
设\(\A \in M_n(\mathbb{R})\),
\(\A\)的各行元素之和均为\(\lambda_0\).
证明:\(\lambda_0\)是\(\A\)的特征值,
\(n\)维列向量\(\x_0=(1,\dotsc,1)^T\)是\(\A\)属于\(\lambda_0\)的特征向量.
\begin{proof}
记\[
	\A
	= \begin{bmatrix}
		a_{11} & a_{12} & \dots & a_{1n} \\
		a_{21} & a_{22} & \dots & a_{2n} \\
		\vdots & \vdots & & \vdots \\
		a_{n1} & a_{n2} & \dots & a_{nn}
	\end{bmatrix},
\]
那么\[
	\A \x_0
	= \begin{bmatrix}
		a_{11} & a_{12} & \dots & a_{1n} \\
		a_{21} & a_{22} & \dots & a_{2n} \\
		\vdots & \vdots & & \vdots \\
		a_{n1} & a_{n2} & \dots & a_{nn}
	\end{bmatrix}
	\begin{bmatrix}
		1 \\ 1 \\ \vdots \\ 1
	\end{bmatrix}
	= \begin{bmatrix}
		a_{11} + a_{12} + \dotsb + a_{1n} \\
		a_{21} + a_{22} + \dotsb + a_{2n} \\
		\vdots \\
		a_{n1} + a_{n2} + \dotsb + a_{nn}
	\end{bmatrix}
	= \begin{bmatrix}
		\lambda_0 \\ \lambda_0 \\ \vdots \\ \lambda_0
	\end{bmatrix}
	= \lambda_0
	\begin{bmatrix}
		1 \\ 1 \\ \vdots \\ 1
	\end{bmatrix},
\]
由此可见,\(\lambda_0\)是\(\A\)的特征值,
\(n\)维列向量\(\x_0=(1,\dotsc,1)^T\)是\(\A\)属于\(\lambda_0\)的特征向量.
\end{proof}
\end{example}

\begin{property}
设\(\lambda_0\)是\(\A\)的特征值.
当\(m\in\mathbb{N}\)时,\(\L0^m\)是\(\A^m\)的特征值.
若\(\A\)可逆,则当\(m\in\mathbb{Z}\)时,\(\L0^m\)是\(\A^m\)的特征值.
\begin{proof}
由定义,存在\(\X0\neq\z\),
使得\(\A\X0 = \L0\X0\),则\[
	\A^2\X0 = \A(\A\X0)
	=\A(\L0\X0)
	=\L0(\A\X0)
	=\L0(\L0\X0)
	=\L0^2\X0.
\]
设\(\A^{m-1}\X0 = \L0^{m-1}\X0\)成立,则\[
	\A^m\X0 = \A(\A^{m-1}\X0)
	= \A(\L0^{m-1}\X0)
	= \L0^{m-1}(\A\X0)
	= \L0^{m-1}(\L0\X0)
	= \L0^m\X0.
\]

当\(\A\)可逆时,\(\L0\neq0\),由\(\L0\X0 = \A\X0\)可得\[
	\L0(\A^{-1}\X0)
	= \A^{-1}(\L0\X0)
	= \A^{-1}(\A\X0)
	= (\A^{-1}\A)\X0
	= \E\X0
	= \X0,
\]
从而有\(\A^{-1}\X0 = \L0^{-1}\X0\).
\end{proof}
\end{property}

\begin{corollary}
设多项式\(f(x)=\sum_k a_k x^k\).
若\(\lambda_0\)是方阵\(\A\)的特征值,则\(f(\L0)\)是\(f(\A)\)的特征值.
\end{corollary}

\begin{definition}
设\(\A=(a_{ij})_n \in M_n(K)\),\(\l \in K\).
把\(\l\E-\A\)称为“\(\A\)的\DefineConcept{特征矩阵}”.
把特征矩阵的行列式\[
	f(\l)
	= \abs{\l\E-\A}
	= \begin{vmatrix}
		\l-a_{11} & -a_{12} & \dots & -a_{1n} \\
		-a_{21} & \l-a_{22} & \dots & -a_{2n} \\
		\vdots & \vdots & & \vdots \\
		-a_{n1} & -a_{n2} & \dots & \l-a_{nn}
	\end{vmatrix}
\]称为“\(\A\)的\DefineConcept{特征多项式}(eigenpolynomial)”.
\end{definition}

\begin{property}
\(\x_0\)是\(\A\)属于\(\lambda_0\)的特征向量的充分必要条件是:
\(\x_0\)是齐次线性方程组\((\L0\E-\A)\x=\z\)的非零解.
\begin{proof}
\(\X0\neq\z
\implies
[\A\X0=\L0\X0
\iff \A\X0-\L0\X0=\z
\iff (\L0\E-\A)\X0=\z]\).
\end{proof}
\end{property}

\begin{property}
\(\lambda_0\)是\(\A\)的特征值的充分必要条件是:
\(\abs{\L0\E-\A}=0\),
即\(\lambda_0\)是特征多项式\(f(\l)=\abs{\l\E-\A}=0\)的根.
\begin{proof}
因为“\(\x_0\)是\(\A\)属于\(\lambda_0\)的特征向量”的充分必要条件是
“\(\x_0\)是齐次线性方程组\((\L0\E-\A)\x=\z\)的非零解”,
而后者成立的充分必要条件是“\(\rank(\L0\E-\A)<n\)”,
从而有等价命题\(\abs{\L0\E-\A}=0\).
\end{proof}
\end{property}

大数学家高斯在1799年证明了以下\DefineConcept{代数基本定理}:
\begin{lemma}[代数基本定理]
任何\(n\ (n\geq1)\)次多项式至少有一个复数根.
\end{lemma}

\begin{theorem}[代数基本定理']
任何\(n\ (n>0)\)次多项式有且仅有\(n\)个复根,其中规定\(m\)重根算\(m\)个根.
\end{theorem}
由此可知,任意\(n\)阶矩阵的特征多项式有且仅有\(n\)个复根.

\begin{theorem}
设\(\A \in M_n(\mathbb{C})\),
则\[
	\text{\(\lambda\)是\(\A\)的特征值}
	\iff
	\text{\(\lambda\)是\(\A^T\)的特征值}.
\]
\begin{proof}
因为\[
	\abs{\lambda\E-\A^T}
	= \abs{(\lambda\E-\A^T)^T}
	= \abs{(\lambda\E)^T-(\A^T)^T}
	= \abs{\lambda\E-\A},
\]
所以\(\A\)与\(\A^T\)具有相同的特征值.
\end{proof}
\end{theorem}

\begin{example}
设\(\A = \begin{bmatrix}
	2 & -1 & 2 \\
	5 & -3 & 3 \\
	-1 & 0 & -2
\end{bmatrix}\),
求\(\A\)的特征值与对应的特征向量.
\begin{solution}
\(\A\)的特征多项式\begin{align*}
	\abs{\l\E-\A}
	&= \begin{vmatrix}
		\l-2 & 1 & -2 \\
		-5 & \l+2 & -3 \\
		1 & 0 & \l+2
	\end{vmatrix} \\
	&= \l^3 + 3\l^2 + 3\l + 1
	= (\l+1)^3,
\end{align*}
特征值为\(\L{1}=-1\)(三重).

当\(\l=-1\)时,解方程组\((-\E-\A)\x = \z\),对系数矩阵施行初等行变换,\[
	-\E-\A = \begin{bmatrix}
		-3 & 1 & -2 \\
		-5 & 2 & -3 \\
		1 & 0 & 1
	\end{bmatrix} \to \begin{bmatrix}
		1 & 0 & 1 \\
		5 & 2 & -3 \\
		-3 & 1 & -2
	\end{bmatrix} \to \begin{bmatrix}
		1 & 0 & 1 \\
		0 & 1 & 1 \\
		0 & 0 & 0
	\end{bmatrix}.
\]
可知\(\rank(-\E-\A) = 2\),
方程组\((-\E-\A)\x = \z\)的解空间只有1个基向量.
令\(x_3 = -1\),得基础解系\[
	\X1 = (1,1,-1)^T,
\]
那么属于\(-1\)的全部特征向量为\(k (1,1,-1)^T\),
\(k\)是非零的任意常数.
\end{solution}
\end{example}

\begin{example}
设\(\A = \begin{bmatrix}
	-1 & 0 & 0 \\
	8 & 2 & 4 \\
	8 & 3 & 3
\end{bmatrix}\),
求\(\A\)的特征值与对应的特征向量.
\begin{solution}
\(\A\)的特征多项式\[
	\abs{\l\E-\A}
	= \begin{vmatrix}
		\l+1 & 0 & 0 \\
		-8 & \l-2 & -4 \\
		-8 & -3 & \l-3
	\end{vmatrix}
	= (\l+1)^2 (\l-6),
\]
故\(\A\)的特征值为\(\L{1}=-1\)(二重),\(\L{2}=6\).

当\(\l=-1\)时,解方程组\((-\E-\A)\x = \z\),\[
	-\E-\A = \begin{bmatrix}
		0 & 0 & 0 \\
		-8 & -3 & -4 \\
		-8 & -3 & -4
	\end{bmatrix} \to \begin{bmatrix}
		-8 & -3 & -4 \\
		0 & 0 & 0 \\
		0 & 0 & 0
	\end{bmatrix},
\]
可知\(\rank(-\E-\A) = 1\),
方程组\((-\E-\A)\x = \z\)的解空间有2个基向量.
分别令\(x_2 = 8, x_3 = 0\)和\(x_2 = 0, x_3 = 2\),得基础解系\[
	\X1 = \begin{bmatrix} -3 \\ 8 \\ 0 \end{bmatrix},
	\qquad
	\X2 = \begin{bmatrix} -1 \\ 0 \\ 2 \end{bmatrix},
\]
故属于\(-1\)的全部特征向量为\(k_1 \X1 + k_2 \X2\),
\(k_1,k_2\)是不全为零的任意常数.

当\(\l=6\)时,解方程组\((6\E-\A)\x = \z\),\[
	6\E-\A = \begin{bmatrix}
		7 & 0 & 0 \\
		-8 & 4 & -4 \\
		-8 & -3 & 3
	\end{bmatrix} \to \begin{bmatrix}
		1 & 0 & 0 \\
		0 & 1 & -1 \\
		0 & 0 & 0
	\end{bmatrix},
\]
可知\(\rank(6\E-\A) = 2\),方程组\((6\E-\A)\x = \z\)的解空间有1个基向量.
令\(x_3 = 1\),得\(x_1 = 0, x_2 = 1\),基础解系\[
	\X3 = (0,1,1)^T,
\]
故属于\(6\)的全部特征向量为\(k_3 \X3\),\(k_3\)是任意非零常数.
\end{solution}
\end{example}

从特征值、特征向量的性质可以看出,
矩阵\(\A\)的一个特征值对应若干个线性无关的特征向量;
但反之,一个特征向量只能属于一个特征值.
事实上,设\(\x_0\)为某个矩阵\(\A\)的特征向量,若有\(\L{1},\L{2}\)满足\[
	\A\X0=\L{1}\X0,
	\quad
	\A\X0=\L{2}\X0,
\]
则必有\(\L{1}\X0=\L{2}\X0\)或\((\L{1}-\L{2})\X0=\z\),
因为\(\X0\neq\z\),所以\(\L{1}-\L{2}=0\),\(\L{1}=\L{2}\).

前面已经知道,
矩阵\(\A\)的同一个特征值\(\lambda_0\)对应的特征向量的非零线性组合仍为\(\A\)的属于\(\lambda_0\)的特征向量.
那么,\(\A\)的不同特征值对应的特征向量的非零线性组合又如何呢?
\begin{example}
设\(\L{1}\)、\(\L{2}\)是矩阵\(\A\)的两个不同的特征值,
\(\X1\)、\(\X2\)分别是\(\L{1}\)、\(\L{2}\)对应的特征向量.
证明:\(\X1+\X2\)不是\(\A\)的特征向量.
\begin{proof}
\(\A\X1 = \L{1}\X1\),\(\A\X2 = \L{2}\X2\),
假设\(\A(\X1+\X2) = \L0(\X1+\X2)\),则\[
	\A\X1+\A\X2 =\L{1}\X1+\L{2}\X2 = \L0\X1+\L0\X2,
\]\[
	(\L0-\L{1})\X1+(\L0-\L{2})\X2 = \z,
\]
在上式左右两端同乘\(\A\)和\(\L{1}\)可得\[
	\left\{ \begin{array}{l}
		(\L0-\L{1})\A\X1+(\L0-\L{2})\A\X2 = (\L0-\L{1})\L{1}\X1 + (\L0-\L{2})\L{2}\X2 = \z, \\
		(\L0-\L{1})\L{1}\X1+(\L0-\L{2})\L{1}\X2 = \z,
	\end{array} \right.
\]\[
	(\L0-\L{2})(\L{2}-\L{1})\X2 = \z,
\]
因为\(\X2\neq\z\),所以\((\L0-\L{2})(\L{2}-\L{1})=0\);
又因为\(\L{2}\neq\L{1}\),所以\(\L0=\L{2}\).
同理有\[
	(\L0-\L{1})\L{2}\X1+(\L0-\L{2})\L{2}\X2 = \z
	\implies
	(\L0-\L{1})(\L{1}-\L{2})\X1 = \z,
\]
因为\(\X1\neq\z\),所以\(\L0=\L{1}\).

于是导出\(\L{1}=\L{2}\),与题设矛盾,说明\(\X1+\X2\)不是\(\A\)的特征向量.
\end{proof}
\end{example}

\begin{example}
设\(\lambda_0\)是矩阵\(\A\)的特征值,\(k\)是任意常数,则\(k\L0\)是矩阵\(k\A\)的特征值.
\begin{proof}
\(\A\X0=\L0\X0\),\((k\A)\X0=k(\A\X0)=k(\L0\X0)=(k\L0)\X0\).
\end{proof}
\end{example}

\begin{example}
证明:矩阵\(\A\)与其转置矩阵\(\A^T\)的特征值相同.
\begin{proof}
因为矩阵\(\A\)与\(\A^T\)的特征多项式相同,
即\(\abs{\l\E-\A} = \abs{(\l\E-\A)^T} = \abs{\l\E-\A^T}\),所以特征值相同.
\end{proof}
\end{example}

\begin{example}
证明:幂等矩阵的特征值只能为0或1.
\begin{proof}
设\(\A\)是幂等矩阵,即有\(\A^2=\A\).
设\(\A\X0=\L0\X0\ (\X0\neq\z)\),
则\[
	\A^2\X0=\A(\A\X0)=\A(\L0\X0)=\L0(\A\X0)=\L0(\L0\X0)=\L0^2\X0.
\]
因为\[
	\A^2=\A
	\iff
	\A^2-\A=\z
	\implies
	\A^2\X0-\A\X0=(\A^2-\A)\X0=\z\X0=\z,
\]
所以\[
	\L0^2\X0-\L0\X0=(\L0^2-\L0)\X0=\L0(\L0-1)\X0=\z,
\]
进一步有\(\L0(\L0-1)=0\),所以\(\L0=0\)或\(\L0=1\).
\end{proof}
\end{example}

\begin{example}
试讨论:在什么条件下,矩阵\(\A\)的任一特征向量总是\(\A^T\)的特征向量.
\begin{solution}
假设\(\A\)的特征向量都是\(\A^T\)的特征向量,
\(\A\x=\L{1}\x\),\(\A^T\x=\L{2}\x\),那么有\[
	(\A^T\x)^T=(\L{2}\x)^T
	\implies
	\x^T\A=\L{2}\x^T
	\implies
	(\x^T\A)\x=(\L{2}\x^T)\x,
\]\[
	\x^T(\A\x)=\x^T(\L{1}\x)=\L{1}\x^T\x=\L{2}\x^T\x
	\implies
	(\L{1}-\L{2})\x^T\x=0,
\]
因为\(\x\neq\z\),\(\x^T\x\neq0\),
所以\(\L{1}-\L{2}=0\),\(\L{1}=\L{2}\).
那么\[
	\A\x=\L{1}\x=\L{2}\x=\A^T\x,
\]
从而\((\A-\A^T)\x=\z\),\(\A=\A^T\).
也就是说,当且仅当\(\A\)是对称矩阵时,\(\A\)的特征向量都是\(\A^T\)的特征向量.
\end{solution}
\end{example}

\begin{example}
设\(n\)阶矩阵\(\A=(a_{ij})_n\)的特征多项式为\[
	f(\l) = \abs{\l\E-\A}
	= \l^n - a_1 \l^{n-1} + a_2 \l^{n-2} - \dotsb + (-1)^n a_n,
\]
证明:系数\(a_i\)是矩阵\(\A\)的\(i\)阶主子式之和(\(i=1,2,\dotsc,n\)).
特别地,\(a_1 = \tr\A\),\(a_n = \abs{\A}\).
\end{example}

\begin{example}
设\(n\)阶矩阵\(\A=(a_{ij})_n\)的特征值为\(\L{1},\L{2},\dotsc,\L{n}\),
即\(\A\)的特征多项式为\(\abs{\l\E-\A}=(\l-\L{1})(\l-\L{2})\dotsm(\l-\L{n})\).
证明:
\begin{enumerate}
	\item \(\abs{\A} = \L{1} \L{2} \dotsm \L{n}\);
	\item \(\L{1} + \L{2} + \dotsb + \L{n} = \tr\A\).
\end{enumerate}
\begin{proof}
比较特征多项式的两种形式:\[
	\abs{\l\E-\A}
	=\l^n-\l^{n-1}(a_{11}+a_{22}+ \dotsb +a_{nn})+ \dotsb +(-1)^n\abs{\A}
\]
和\begin{align*}
	\abs{\l\E-\A} &= (\l-\L{1})(\l-\L{2})\dotsm(\l-\L{n}) \\
	&= \l^n-\l^{n-1}(\L{1}+\L{2}+ \dotsb +\L{n}) + \dotsb + (-1)^n \L{1} \L{2} \dotsm \L{n},
\end{align*}
可得\[
	\L{1} + \L{2} + \dotsb + \L{n} = a_{11} + a_{22} + \dotsb + a_{nn} = \tr\A,
\]\[
	\abs{\A} = \L{1} \L{2} \dotsm \L{n}.
	\qedhere
\]
\end{proof}
\end{example}

\begin{example}\label{example:矩阵乘积的秩.两个向量的乘积的特征值和特征向量}
设\(\a,\b\)是\(n\)维非零列向量.
证明:求矩阵\(\a\b^T\)的特征值和特征向量.
\begin{proof}
显然有\((\a\b^T)\a = \a(\b^T\a)\),
那么根据定义可知\(\b^T\a\)就是矩阵\(\a\b^T\)的特征值,
\(\a\)是\(\a\b^T\)属于\(\b^T\a\)的特征向量.

又由\cref{example:行列式.两个向量的乘积矩阵的行列式} 可知\(\abs{\a\b^T} = 0\),
所以\(\abs{0\cdot\E-\a\b^T}=0\),也就是说\(0\)也是矩阵\(\a\b^T\)的特征值.
再由\cref{example:矩阵乘积的秩.两个向量的乘积的秩} 可知\(\rank(\a\b^T) = 1\),
于是根据\cref{theorem:线性方程组.齐次线性方程组的解向量个数} 可知
\((\a\b^T)\x=\vb0\)的解空间的维数为\(n-1\),
这就是说\(0\)是矩阵\(\a\b^T\)的\(n-1\)重特征值.
现在我们来求\(\a\b^T\)属于\(0\)的特征向量,
解方程组\((\a\b^T)\x=\vb0\),
左乘\(\a^T\)得\(\a^T\a\b^T\x=\vb0\),
由于\(\a^T\a\)是正实数,
消去便得\(\b^T\x=\vb0\),
这就是说\(\a\b^T\)属于\(0\)的特征向量是\(\b^T\x=\vb0\)的基础解系.
\end{proof}
\end{example}

\begin{example}
设\(\A \in M_{m \times n}(P),
\B \in M_{n \times m}(P)\),
且\(m \geq n\).
求证:\[
	\abs{\l\E_m-\A\B} = \l^{m-n} \abs{\l\E_n-\B\A}.
\]
\begin{proof}
当\(\l\neq0\)时,考虑下列分块矩阵:\[
	\begin{bmatrix}
		\l\E_m & \A \\
		\B & \E_n
	\end{bmatrix}.
\]
因为\(\l\E_m,\E_n\)都是可逆矩阵,
故由\hyperref[theorem:逆矩阵.行列式第一降阶定理]{降阶公式}可得\[
	\abs{\E_n} \cdot \abs{\l\E_m - \A (\E_n)^{-1} \B}
	= \abs{\l\E_m} \cdot \abs{\E_n - \B (\l\E_m)^{-1} \A},
\]
即有\(\abs{\l\E_m-\A\B} = \l^{m-n} \abs{\l\E_n-\B\A}\)成立.

当\(\l=0\)时,若\(m>n\),则\[
	\rank(\A\B) \leq \min\{\rank\A,\rank\B\} \leq \min\{m,n\} < m,
\]
故\(\abs{-\A\B}=0\),结论也成立;
若\(m = n\),则由\cref{theorem:行列式.矩阵乘积的行列式} 可知结论也成立.

事实上,\(\l=0\)的情形也可通过摄动法由\(\l\neq0\)的情形来得到.
\end{proof}
\end{example}
上例还有其他证法:
你可以将\(\A\)化为等价标准型来证明,
或先证\(\A\)非异的情形,再用摄动法进行讨论.

\subsection{求解特征值和特征向量的一般程序}
求解\(n\)阶矩阵\(\A = (a_{ij})_n\)的特征值和特征向量的一般程序:
\begin{enumerate}
	\item 计算特征多项式\(\abs{\l\E-\A}\);
	\item 求出\(\abs{\l\E-\A}=0\)的全部根,
	得\(\A\)的全部特征值\(\AutoTuple{\lambda}{n}\);
	\item 对于每个不同的特征值\(\L{j}\),
	求出齐次线性方程组\((\L{j} \E - \A)\x = \z\)的一个基础解系\(\AutoTuple{\x}{t}\),
	则\(\A\)的属于\(\L{j}\)的全部特征向量为\(k_1 \X1 + k_2 \X2 + \dotsb + k_t \X t\)
	(其中\(\AutoTuple{k}{t}\)是不全为零的任意常数).
\end{enumerate}

注1:根据高斯代数基本定理,任何\(n\ (n \geq 1)\)次多项式至少有一个复数根.
由此推出,任何\(n\ (n>0)\)次多项式有且仅由\(n\)个复根,其中规定\(m\)重根算\(m\)个根.

注2:\(n\)阶矩阵\(\A\)恰有\(n\)个特征值,但它不一定有\(n\)个线性无关的特征向量.

注3:如果\(\A\)是\(n\)阶实矩阵,则它有\(n\)个复特征值,
其中实特征值个数\(m\)满足\(0 \leq m \leq n\).

\begin{example}
设\(\A = \begin{bmatrix} 2 & -1 & 2 \\ 5 & -3 & 3 \\ -1 & 0 & -2 \end{bmatrix}\),
求\(\A\)的特征值与对应的特征向量.
\begin{solution}
\(\A\)的特征多项式\[
	\abs{\l\E-\A}
	= \begin{vmatrix} \l-2 & 1 & -2 \\ -5 & \l+3 & -3 \\ 1 & 0 & \l+2 \end{vmatrix}
	= (\l+1)^3,
\]
解\(\abs{\l\E-\A}=0\)得\(\l=-1\)(三重).

当\(\l=-1\)时,解方程组\((-\E-\A)\x=\z\),\[
	-\E-\A = \begin{bmatrix} -3 & 1 & -2 \\ -5 & 2 & -3 \\ 1 & 0 & 1 \end{bmatrix}
	\to \begin{bmatrix} 1 & 0 & 1 \\ 5 & 2 & -3 \\ -3 & 1 & -2 \end{bmatrix}
	\to \begin{bmatrix} 1 & 0 & 1 \\ 0 & 1 & 1 \\ 0 & 0 & 0 \end{bmatrix},
\]
\(\rank(-\E-\A)=2\),
令\(x_3=1\),
得基础解系\[
	\X1=\begin{bmatrix} -1 \\ -1 \\ 1 \end{bmatrix},
\]
属于\(-1\)的全部特征向量为\(k\X1\)(\(k\)为非零的任意常数).
\end{solution}
\end{example}

\begin{example}
设\(\A = \begin{bmatrix} -1 & 0 & 0 \\ 8 & 2 & 4 \\ 8 & 3 & 3 \end{bmatrix}\),
求\(\A\)的特征值与对应的特征向量.
\begin{solution}
\(\A\)的特征多项式\[
	\a = \begin{vmatrix}
		\l+1 & 0 & 0 \\
		-8 & \l-2 & -4 \\
		-8 & -3 & \l-3
	\end{vmatrix}
	= (\l+1)(\l^2-5\l-6)
	= (\l+1)^2(\l-6).
\]
令\(\a = 0\)可得\(\A\)的特征值为\(\L{1}=-1\)(二重),\(\L{2}=6\).

当\(\l=-1\)时,解方程组\((-\E-\A)\x=\z\),\[
	-\E-\A
	= \begin{bmatrix} 0 & 0 & 0 \\ -8 & -3 & -4 \\ -8 & -3 & -4 \end{bmatrix}
	\to \begin{bmatrix} -8 & -3 & -4 \\ 0 & 0 & 0 \\ 0 & 0 & 0 \end{bmatrix}.
\]
分别令\(\left\{ \begin{array}{l} x_2=8 \\ x_3=0 \end{array} \right.\)
和\(\left\{ \begin{array}{l} x_2=0 \\ x_3=2 \end{array} \right.\),
得基础解系\[
	\X1 = \begin{bmatrix} -3 \\ 8 \\ 0 \end{bmatrix},
	\quad
	\X2 = \begin{bmatrix} -1 \\ 0 \\ 2 \end{bmatrix},
\]
属于\(-1\)的全部特征向量为\(k_1\X1+k_2\X2\)(\(k_1,k_2\)为不全为零的任意常数);

当\(\l=6\)时,解方程组\((6\E-\A)\x=\z\),\[
	6\E-\A = \begin{bmatrix} 7 & 0 & 0 \\ -8 & 4 & -4 \\ -8 & -3 & 3 \end{bmatrix} \to \begin{bmatrix} 1 & 0 & 0 \\ 0 & 1 & -1 \\ 0 & 0 & 0 \end{bmatrix},
\]
令\(x_3=1\)得\(x_1=0\),\(x_2=1\),
基础解系\[
	\X3 = \begin{bmatrix} 0 \\ 1 \\ 1 \end{bmatrix},
\]
属于\(6\)的全部特征向量为\(k_3\X3\)(\(k_3\)为任意常数).
\end{solution}
\end{example}

\begin{example}
设矩阵\(\A = (a_{ij})_n \in \mathbb{C}^n\),
但\(a_{ij} \in \mathbb{R}\ (i,j=1,2,\dotsc,n)\).
证明:如果\(\L0\in\mathbb{C}\)是\(\A\)的一个特征值,
\(\x_0\)是\(\A\)属于\(\lambda_0\)的一个特征向量,
那么\(\complexconjugate{\L0}\)也是\(\A\)的一个特征值,
且\(\complexconjugate{\X0}\)是\(\A\)属于\(\complexconjugate{\L0}\)的一个特征向量.
\begin{proof}
在\(\A\X0=\L0\X0\)两边取共轭得
\(\complexconjugate{\A}\complexconjugate{\X0}
=\complexconjugate{\L0}\complexconjugate{\X0}\).
又因为\(\A=\complexconjugate{\A}\),
因此\(\A\complexconjugate{\X0}=\complexconjugate{\L0}\complexconjugate{\X0}\).
这就表明\(\complexconjugate{\L0}\)也是\(\A\)的一个特征值,
\(\complexconjugate{\X0}\)是\(\A\)的属于\(\complexconjugate{\L0}\)的一个特征向量.
\end{proof}
\end{example}

\begin{example}
求复数域上矩阵\[
	\A = \begin{bmatrix}
		4 & 7 & -3 \\
		-2 & -4 & 2 \\
		-4 & -10 & 4
	\end{bmatrix}
\]的全部特征值和特征向量.
\begin{solution}
\(\A\)的特征多项式为\[
	\abs{\l\E-\A}
	= \begin{vmatrix}
		\l-4 & -7 & 3 \\
		2 & \l+4 & -2 \\
		4 & 10 & \l-4
	\end{vmatrix}
	= \l^3 - 4\l^2 + 6\l - 4
	= (\l-2)(\l^2-2\l+2).
\]
令\(\abs{\l\E-\A}=0\)解得\(\l=2,1\pm\iu\).

当\(\l=2\)时,解方程组\((2\E-\A)\x=\z\),\[
	2\E-\A = \begin{bmatrix}
		-2 & -7 & 3 \\
		2 & 6 & -2 \\
		4 & 10 & -2
	\end{bmatrix} \to \begin{bmatrix}
		2 & 4 & 0 \\
		0 & 1 & -1 \\
		0 & 0 & 0
	\end{bmatrix}.
\]
令\(x_2=x_3=1\)得\(x_1=-2\),基础解系为\[
	\X1 = (-2,1,1)^T,
\]
属于\(2\)的全部特征向量为\(k_1\X1\ (k_1\in\mathbb{C}-\{0\})\).

当\(\l=1+\iu\)时,解方程组\([(1+\iu)\E-\A]\x=\z\),\[
	(1+\iu)\E-\A = \begin{bmatrix}
		-3+\iu & -7 & 3 \\
		2 & 5+\iu & -2 \\
		4 & 10 & -3+\iu
	\end{bmatrix}
	\to \def\arraystretch{1.5}\begin{bmatrix}
		1 & 0 & \frac{1}{2}-\iu \\
		0 & 1 & -\frac{1}{2}+\frac{1}{2}\iu \\
		0 & 0 & 0
	\end{bmatrix}.
\]
令\(x_3=-2\)得\(x_1=1-2\iu,x_2=-1+\iu\),
基础解系为\[
	\X2 = (1-2\iu,-1+\iu,-2)^T,
\]
属于\(1+\iu\)的全部特征向量为\(k_2\X2\ (k_2\in\mathbb{C}-\{0\})\).

当\(\l=1-\iu\)时,
\(\X2\)也是它的一个特征向量,
那么属于\(1-\iu\)的全部特征向量为\(k_3\X2\ (k_3\in\mathbb{C}-\{0\})\).
\end{solution}
\end{example}

\section{矩阵的相似}
\subsection{矩阵相似的概念}
\begin{definition}
设\(\A\)、\(\B\)是两个\(n\)阶矩阵.若存在可逆矩阵\(\P\),使得
\begin{equation}\label{equation:特征值与特征向量.相似矩阵的定义}
	\P^{-1}\A\P=\B
\end{equation}
则称\(\A\)与\(\B\)相似,记作\(\A\sim\B\).
\end{definition}

\subsection{矩阵相似的性质}
\begin{property}\label{theorem:特征值与特征向量.相似关系是等价关系}
矩阵之间的相似关系,是矩阵集合上的等价关系,因为它满足:
\begin{enumerate}
	\item 反身性,即对任意矩阵\(\A\),都有\(\A\sim\A\);
	\item 对称性,即若\(\A\sim\B\),则有\(\B\sim\A\);
	\item 传递性,即若\(\A\sim\B\)且\(\B\sim\C\),则\(\A\sim\C\).
\end{enumerate}
\begin{proof}
在\cref{equation:特征值与特征向量.相似矩阵的定义} 中,
令\(\A=\B\)、\(\P=\E\),得\(\E\A\E=\A\),
即有相似矩阵的反身性成立.

再在\cref{equation:特征值与特征向量.相似矩阵的定义} 中取\(\Q=\P^{-1}\),
得\(\A = \Q^{-1}(\P^{-1}\A\P)\Q = \Q^{-1}\B\Q\),即有相似矩阵的对称性成立.

设\(\P_1^{-1}\A\P_1=\B,
\P_2^{-1}\B\P_2=\C\),
于是\(\P_2^{-1}(\P_1^{-1}\A\P_1)\P_2=\C\).
取\(\Q=\P_1\P_2\),\(\Q\)是可逆矩阵,且\(\Q^{-1}\A\Q=\C\),所以\(\A\sim\C\),
即有相似矩阵的传递性成立.
\end{proof}
\end{property}

以下性质是两个矩阵相似的必要条件.
\begin{property}\label{theorem:特征值与特征向量.矩阵相似的必要条件1}
若\(\A\sim\B\),则\(\abs{\A}=\abs{\B}\).
\begin{proof}
因为\(\A\sim\B\),所以存在可逆矩阵\(\P\),
使得\(\P^{-1}\A\P=\B\),两端取行列式,得\[
	\abs{\B} = \abs{\P^{-1}\A\P}
	= \abs{\P^{-1}}\abs{\A}\abs{\P}
	= \abs{\P}^{-1}\abs{\A}\abs{\P}
	= \abs{\A}.
	\qedhere
\]
\end{proof}
\end{property}
我们还可以进一步推得如下结论:
若\(\A\sim\B\),则\(\A\)、\(\B\)同为可逆或不可逆.

\begin{property}\label{theorem:特征值与特征向量.矩阵相似的必要条件2}
若\(\A\sim\B\),则\(\A^m \sim \B^m\ (m\in\mathbb{N})\).
\begin{proof}
因为\(\A\sim\B\),所以存在可逆矩阵\(\P\),使得\(\P^{-1}\A\P=\B\),于是\[
	(\B)^m = (\P^{-1}\A\P)^m
	= (\P^{-1}\A\P)(\P^{-1}\A\P)\dotsb(\P^{-1}\A\P)
	= \P^{-1}\A^m\P.
	\qedhere
\]
\end{proof}
\end{property}
如果\(\A\)、\(\B\)可逆,那么上述结论可以扩展为\(\A^m\sim\B^m\ (m\in\mathbb{Z})\).

\begin{property}\label{theorem:特征值与特征向量.矩阵相似的必要条件3}
相似矩阵有相同的特征多项式,从而有相同的特征值.
\begin{proof}
因为\(\A\sim\B\),所以存在可逆矩阵\(\P\),使得\(\P^{-1}\A\P=\B\),于是\[
	\abs{\l\E-\B}
	=\abs{\P^{-1}(\l\E-\A)\P}
	=\abs{\P^{-1}}\abs{\l\E-\A}\abs{\P}
	=\abs{\l\E-\A}.
	\qedhere
\]
\end{proof}
\end{property}

应该注意到,\cref{theorem:特征值与特征向量.矩阵相似的必要条件3} 只是矩阵相似的必要不充分条件.
下面我们举出一条反例,不相似的两个矩阵有相同的特征值.
取\(\A=\begin{bmatrix} 2 & 1 \\ 0 & 2 \end{bmatrix}\)
和\(\B=\begin{bmatrix} 2 & 0 \\ 0 & 2 \end{bmatrix}\),
显然两者的特征值相同,即\(\l=2\)(二重).
但\(\A\)与\(\B\)不相似,
这是因为\(\B=2\E\)是数乘矩阵,可以和所有二阶矩阵交换,
那么对任意二阶可逆矩阵\(\P\)都有\(\P^{-1}\B\P=\B\P^{-1}\P=\B\),
即\(\B\)只能与自身相似,
\(\A\)与\(\B\)不相似.

\begin{property}\label{theorem:特征值与特征向量.矩阵相似的必要条件4}
相似矩阵有相同的迹,即\(\A\sim\B \implies \tr\A=\tr\B\).
\begin{proof}
设可逆矩阵\(\P\)满足\(\P^{-1}\A\P=\B\),
于是\begin{align*}
	\tr\B
	&= \tr(\P^{-1}\A\P) \\
	&= \tr(\P^{-1}(\A\P)) \\
	&= \tr((\A\P)\P^{-1})
		\tag{\cref{theorem:矩阵的迹.矩阵乘积交换次序不变迹}} \\
	&= \tr\A.
	\qedhere
\end{align*}
\end{proof}
% \begin{proof}
% 设\[
% 	\A = (a_{ij})_n
% 	= \begin{bmatrix}
% 		a_{11} & \dots & a_{1i} & a_{1j}  & \dots & a_{1n} \\
% 		\vdots & & \vdots & \vdots & & \vdots \\
% 		a_{i1} & \dots & a_{ii} & a_{ij}  & \dots & a_{in} \\
% 		\vdots & & \vdots & \vdots  & & \vdots \\
% 		a_{j1} & \dots & a_{ji} & a_{jj}  & \dots & a_{jn} \\
% 		\vdots & & \vdots & \vdots  & & \vdots \\
% 		a_{n1} & \dots & a_{ni} & a_{nj}  & \dots & a_{nn}
% 	\end{bmatrix}.
% \]

% 我们首先考察在成对的初等变换\(\P^{-1}\)、\(\P\)的作用下,
% 任意矩阵\(\A\)的迹\(\tr\A\)与\(\P^{-1} \A \P\)的迹\(\tr(\P^{-1} \A \P)\)相比,
% 会如何变化:
% \begin{enumerate}
% 	\item 令\(\P = \P(i,j)\),有\(\P^{-1} = \P\),
% 	那么\(\P^{-1} \A \P\)相当于
% 	“首先交换\(\A\)的\(i\)、\(j\)两行,然后交换所得矩阵的\(i\)、\(j\)两列”,
% 	即\[
% 		\P^{-1} \A \P
% 		= \begin{bmatrix}
% 			a_{11} & \dots & a_{1j} & a_{1i}  & \dots & a_{1n} \\
% 			\vdots & & \vdots & \vdots & & \vdots \\
% 			a_{j1} & \dots & a_{jj} & a_{ji}  & \dots & a_{jn} \\
% 			\vdots & & \vdots & \vdots  & & \vdots \\
% 			a_{i1} & \dots & a_{ij} & a_{ii}  & \dots & a_{in} \\
% 			\vdots & & \vdots & \vdots  & & \vdots \\
% 			a_{n1} & \dots & a_{nj} & a_{ni}  & \dots & a_{nn}
% 		\end{bmatrix}.
% 	\]
% 	可以看到经过变换\(\P(i,j)\)前后两个矩阵的主对角线的元素之和不变.

% 	\item 令\(\P = \P(i(c))\ (c\neq0)\),
% 	有\(\P^{-1} = \P(i(c^{-1}))\),
% 	那么\(\P^{-1} \A \P\)相当于
% 	“首先用\(1/c\)乘以\(\A\)的第\(i\)行,再用\(c\)乘以\(\A\)的第\(i\)列”,
% 	即\[
% 		\P^{-1} \A \P
% 		= \begin{bmatrix}
% 			a_{11} & \dots & c a_{1i} & a_{1j}  & \dots & a_{1n} \\
% 			\vdots & & \vdots & \vdots & & \vdots \\
% 			\frac{1}{c} a_{i1} & \dots & \left( c \cdot \frac{1}{c} \right) a_{ii} & \frac{1}{c} a_{ij}  & \dots & \frac{1}{c} a_{in} \\
% 			\vdots & & \vdots & \vdots  & & \vdots \\
% 			a_{j1} & \dots & c a_{ji} & a_{jj}  & \dots & a_{jn} \\
% 			\vdots & & \vdots & \vdots  & & \vdots \\
% 			a_{n1} & \dots & c a_{ni} & a_{nj}  & \dots & a_{nn}
% 		\end{bmatrix}.
% 	\]
% 	可以看到经过变换\(\P(i(c))\)前后两个矩阵的主对角线的元素之和也不变.

% 	\item 令\(\P = \P(i,j(k))\),
% 	有\(\P^{-1} = \P(i,j(-k))\),
% 	那么\(\P^{-1} \A \P\)相当于
% 	“首先将\(\A\)的第\(j\)行的\((-k)\)倍加到第\(i\)行,
% 	再将所得矩阵的第\(i\)列的\(k\)倍加到第\(j\)列”,
% 	即\[
% 		\P^{-1} \A \P
% 		= \scalebox{.8}{\(\begin{bmatrix}
% 			a_{11} & \dots & a_{1i} & a_{1j} + k a_{1i}  & \dots & a_{1n} \\
% 			\vdots & & \vdots & \vdots & & \vdots \\
% 			a_{i1} - k a_{j1} & \dots & a_{ii} - k a_{ji} & a_{ij} - k a_{jj} + k(a_{ii} - k a_{ji})  & \dots & a_{in} - k a_{jn} \\
% 			\vdots & & \vdots & \vdots  & & \vdots \\
% 			a_{j1} & \dots & a_{ji} & a_{jj} + k a_{ji}  & \dots & a_{jn} \\
% 			\vdots & & \vdots & \vdots  & & \vdots \\
% 			a_{n1} & \dots & a_{ni} & a_{nj} + k a_{ni}  & \dots & a_{nn}
% 		\end{bmatrix}\)}.
% 	\]
% 	可以看到经过变换\(\P(i,j(k))\)前后两个矩阵的主对角线的元素之和还是不变.
% \end{enumerate}
% 综上所述,成对的初等变换不改变矩阵的迹.
% 由\cref{theorem:逆矩阵.可逆矩阵与初等矩阵的关系} 可知,
% 任意可逆矩阵都可以分解为若干个初等矩阵的乘积,
% 那么根据上述讨论结果,对于任意可逆矩阵\(\P\),总有\[
% 	\tr(\P^{-1}\A\P) = \tr\A.
% 	\qedhere
% \]
% \end{proof}
\end{property}

\begin{example}
证明:单位矩阵只与它本身相似.
\begin{proof}
设\(\A \in M_n(K)\),
\(\E\)是\(M_n(K)\)的单位矩阵,
且\(\E \sim \A\).
因此,根据矩阵相似的定义,
存在可逆\(\P \in M_n(K)\),
使得\(\P^{-1} \E \P = \A\).
又因为单位矩阵可以与任意同阶矩阵交换,
所以\(\A
= \P^{-1} \E \P
= \P^{-1} \P \E
= \E \E
= \E\).
\end{proof}
\end{example}

\begin{example}
设矩阵\(\A\)可逆,\(\A^T\)是\(\A\)的转置.
证明:\(\A\A^T \sim \A^T\A\).
\begin{proof}
取\(\P=\A^{-1}\),
那么\(\P\A=\A\P=\E\),\(\A^T = (\P\A)\A^T = \P(\A\A^T)\),\[
	\P(\A\A^T)\P^{-1} = \A^T\P^{-1} = \A^T\A,
\]
故\(\A\A^T \sim \A^T\A\).
\end{proof}
\end{example}

\begin{example}
已知矩阵\(\A = \begin{bmatrix}
	2 & 0 & 0 \\
	0 & 0 & 1 \\
	0 & 1 & x
\end{bmatrix}\)与\(\B = \begin{bmatrix}
	2 & 0 & 0 \\
	0 & y & 0 \\
	0 & 0 & -1
\end{bmatrix}\)相似.
求\(x\)与\(y\).
\begin{solution}
因为\(\A\sim\B\),所以,由\cref{theorem:特征值与特征向量.矩阵相似的必要条件1},
\[
	\begin{vmatrix}
		2 & 0 & 0 \\
		0 & 0 & 1 \\
		0 & 1 & x
	\end{vmatrix}
	= -2 = -2y =
	\begin{vmatrix}
		2 & 0 & 0 \\
		0 & y & 0 \\
		0 & 0 & -1
	\end{vmatrix}
	\implies y = 1;
\]
又由\cref{theorem:特征值与特征向量.矩阵相似的必要条件4},
\[
	\tr\A = 2+x
	= 1+y = \tr\B
	\implies
	x = 0.
\]
\end{solution}
\end{example}

\section{矩阵的对角化}
\begin{theorem}[矩阵可对角化的充要条件]\label{theorem:矩阵可对角化的充分必要条件.定理1}
\(n\)阶矩阵\(\A\)相似于对角阵的充要条件为\(\A\)有\(n\)个线性无关的特征向量.
\begin{proof}
必要性.
\(\A\)相似于对角阵,即存在可逆矩阵\(\P\),使\[
	\P^{-1}\A\P=\V
	=\begin{bmatrix}
		\L{1} \\ & \L{2} \\ & & \ddots \\ & & &\L{n}
	\end{bmatrix}
	=\diag(\L{1},\L{2},\dotsc,\L{n}),
\]
用\(\P\)左乘上式两端,得\[
	\A\P=\P\V.
\]
将\(\P\)按列分块,则\(\P=(\AutoTuple{\x}{n})\),
由于\(\P\)可逆,所以\(\AutoTuple{\x}{n}\)线性无关,有\[
	\A(\AutoTuple{\x}{n})=(\AutoTuple{\x}{n})\V,
\]\[
	(\A\X1,\A\X2,\dotsc,\A\X{n})=(\X1\V,\X2\V,\dotsc,\X{n}\V),
\]
于是\[
	A\X i=\L{i}\X i,
	\quad i=1,2,\dotsc,n,
\]
即\(\AutoTuple{\x}{n}\)是\(\A\)分别对应于\(\AutoTuple{\lambda}{n}\)的\(n\)个线性无关的特征向量.

同理可证充分性.
\end{proof}
\end{theorem}

由上述定理的证明可知:
{\color{red}当\(\P^{-1}\A\P=\V\)时,
\(\V\)的\(n\)个主对角元是\(\A\)的\(n\)个特征值;
可逆矩阵\(\P\)的\(n\)个列向量\(\AutoTuple{\x}{n}\)是
\(\A\)分别属于\(\L{1},\L{2},\dotsc,\L{n}\)的线性无关特征向量.}

\begin{theorem}
矩阵\(\A\)的属于不同特征值的特征向量线性无关.
\begin{proof}
设\(n\)阶矩阵\(\A\)的\(m\)个不同的特征值\(\L{1},\L{2},\dotsc,\L{m}\)
对应的特征向量分别为\(\AutoTuple{\x}{m}\).

由上述所有特征向量构成的向量组,
记作\(X_m=\{\AutoTuple{\x}{m}\}\).

当\(m=1\)时,
由于\(\X1 \neq 0\),
故向量组\(X_1=\{\X1\}\)线性无关.

当\(m>1\)时,
假设\(m-1\)个不同特征值对应的特征向量\(X_{m-1}=\{\AutoTuple{\x}{m-1}\}\)线性无关.
对于\(m\)个不同特征值对应的特征向量组\(X_m\),
令\begin{gather}
	k_1\X1+k_2\X2+\dotsb+k_m\X{m}=\z,
	\tag1
\end{gather}
由于\(\A\X{j}=\L{j}\X{j}\),
用\(\A\)左乘(1)式两端,
得\begin{gather}
	k_1\L{1}\X1+k_2\L{2}\X2+\dotsb+k_{m-1}\L{m-1}\X{m-1}+k_m\L{m}\X{m}=\z.
	\tag2
\end{gather}
再用\(\L{m}\)数乘(1)式两端,得\begin{gather}
	\L{m}k_1\X1+\L{m}k_2\X2+\dotsb+\L{m}k_{m-1}\X{m-1}+\L{m}k_m\X{m}=\z,
	\tag3
\end{gather}
(2)、(3)两式相减,得\begin{gather}
	(\L{1}-\L{m})k_1\X1+(\L{2}-\L{m})k_2\X2+\dotsb+(\L{m-1}-\L{m})k_{m-1}\X{m-1}=\z.
	\tag4
\end{gather}
根据归纳假设,向量组\(X_{m-1}\)线性无关,
则\((\L{i}-\L{m})k_i=0\ (i=1,2,\dotsc,m-1)\).
由于\(\L{i}\neq\L{m}\ (i=1,2,\dotsc,m-1)\),
所以\(k_1=k_2=\dotsb=k_{m-1}=0\),得\(k_m\X{m}=\z\),
但特征向量\(\X{m}\neq\z\),则\(k_m=0\),从而向量组\(X_m\)线性无关.
\end{proof}
\end{theorem}

\begin{corollary}[矩阵可对角化的充分条件]\label{theorem:矩阵可对角化的充分条件.定理1}
若\(n\)阶矩阵\(\A\)有\(n\)个不同的特征值,则\(\A\)可对角化.
\end{corollary}

\begin{theorem}
设\(\L{1},\L{2},\dotsc,\L{m}\)是\(n\)阶矩阵\(\A\)的不同的特征值,
而\(\X{ij}\ (j=1,2,\dotsc)\)是\(\A\)属于\(\L{i}\ (i=1,2,\dotsc,m)\)的线性无关的特征向量,
即\[
	\A \X{ij} = \L{i} \X{ij},
	\quad i=1,2,\dotsc,m;j=1,2,\dotsc,
\]
则\[
	\X{11},\X{12},\dotsc,\X{21},\X{22},\dotsc,\X{m1},\X{m2},\dotsc
\]线性无关.
\end{theorem}

\begin{example}
设\[
	\A = \begin{bmatrix}
		1 & 0 & 0 \\
		-2 & 5 & -2 \\
		-2 & 4 & -1
	\end{bmatrix}.
\]
试问:\(\A\)能否对角化?
若能,则求出可逆矩阵\(\P\),使\(\P^{-1}\A\P\)为对角形矩阵.
\begin{solution}
\(\A\)的特征多项式为\[
	\abs{\l\E-\A} = \begin{bmatrix}
		\l-1 & 0 & 0 \\
		2 & \l-5 & 2 \\
		2 & -4 & \l+1
	\end{bmatrix}
	= (\l-1)^2 (\l-3),
\]
则\(\A\)的特征值为\(\L{1}=1\)(二重),\(\L{2}=3\).

当\(\L{1}=1\)时,解齐次线性方程组\((\E-\A)\x=\z\),\[
	\E-\A=\begin{bmatrix}
		0 & 0 & 0 \\
		2 & -4 & 2 \\
		2 & -4 & 2
	\end{bmatrix}
	\to \begin{bmatrix}
		1 & -2 & 1 \\
		0 & 0 & 0 \\
		0 & 0 & 0
	\end{bmatrix},
\]
基础解系为\(\X1 = \begin{bmatrix} 2 \\ 1 \\ 0 \end{bmatrix},
\X2 = \begin{bmatrix} -1 \\ 0 \\ 1 \end{bmatrix}\).

对于\(\L{2}=3\),解方程组\((3\E-\A)\x=\z\),\[
	3\E-\A=\begin{bmatrix}
		2 & 0 & 0 \\
		2 & -2 & 2 \\
		2 & -4 & 4
	\end{bmatrix} \to \begin{bmatrix}
		2 & 0 & 0 \\
		0 & -2 & 2 \\
		0 & 0 & 0
	\end{bmatrix},
\]
基础解系为\(\X3 = \begin{bmatrix} 0 \\ 1 \\ 1 \end{bmatrix}\).

特征向量\(\X1,\X2,\X3\)线性无关,所以\(\A\)可以对角化.
令\[
	\P = \begin{bmatrix} \X1 & \X2 & \X3 \end{bmatrix} = \begin{bmatrix}
		2 & -1 & 0 \\
		1 & 0 & 1 \\
		0 & 1 & 1
	\end{bmatrix},
	\quad\text{则有}\quad
	\P^{-1} \A \P = \begin{bmatrix} 1 \\ & 1 \\ && 3 \end{bmatrix}.
\]
\end{solution}
\end{example}

\begin{example}
设\(\A = \begin{bmatrix}
	2 & 0 & 0 \\
	0 & 2 & 0 \\
	0 & 1 & 2
\end{bmatrix}\),证明:\(\A\)不可对角化.
\begin{proof}
\(\A\)的特征多项式为\[
	\abs{\l\E-\A} = \begin{vmatrix}
		\l-2 & 0 & 0 \\
		0 & \l-2 & 0 \\
		0 & -1 & \l-2
	\end{vmatrix} = (\l-2)^2,
\]
令\(\abs{\l\E-\A} = 0\)解得特征值\(\L{1}=2\)(三重).
由于\(\rank(\L{1}\E-\A)=1\),
那么对应于唯一的特征值\(\L{1}=2\),
\(\A\)只有两个线性无关的特征向量,
因而不存在可逆矩阵\(\P\)使得\(\P^{-1}\A\P\)为对角形矩阵.
\end{proof}
\end{example}

从上述例子可以看出,当矩阵\(\A\)的某个特征值\(\L0\)为\(k\)重根式,
对应于\(\L0\)的线性无关的特征向量的个数可能等于\(k\),
也可能小于\(k\).这个规律对于一般的矩阵是成立的.

\begin{theorem}
设\(\L0\)为\(n\)阶矩阵\(\A\)的\(k\)重特征值,
则属于\(\L0\)的\(\A\)的线性无关的特征向量最多只有\(k\)个.
\end{theorem}

\begin{theorem}\label{theorem:矩阵可对角化的充分必要条件.定理2}
\(n\)阶矩阵\(\A\)可对角化的充要条件是:
对于\(\A\)的每个\(k_i\)重特征值\(\L{i}\),
\(\A\)有\(k_i\)个线性无关的特征向量.
\end{theorem}

\begin{corollary}\label{theorem:矩阵可对角化的充分必要条件.定理3}
\(n\)阶矩阵\(\A\)可对角化的充要条件是:
对于\(\A\)的每个\(k_i\)重特征值\(\L{i}\),
都有\(\rank(\L{i}\E-\A) = n-k_i\).
\end{corollary}

\begin{example}
设\(\A\)、\(\B\)是\(n\)阶矩阵,且\(\A\)可逆,证明:\(\A\B\)与\(\B\A\)相似.
\begin{proof}
因为\(\A\)可逆,则\[
	\B\A
	=\E\B\A
	=(\A^{-1}\A)\B\A
	=\A^{-1}(\A\B)\A,
\]
根据定义可得\(\A\B \sim \B\A\).
\end{proof}
\end{example}

\begin{example}
设\(\A\)为可逆矩阵且可对角化,证明:\(\A^{-1}\)也可对角化.
\begin{proof}
设存在可逆矩阵\(\P\)使得\begin{gather}
	\P^{-1}\A\P = \V,
	\tag1
\end{gather}
其中\(\V=\diag(\L{1},\L{2},\dotsc,\L{n})\),
\(n\)是矩阵\(\A\)的阶数,
\(\L{1},\L{2},\dotsc,\L{n}\)是矩阵\(\A\)的特征值.
显然有\(\abs{\V}
= \abs{\P^{-1}\A\P}
= \abs{\P^{-1}}\abs{\A}\abs{\P}
= (\abs{\P^{-1}}\abs{\P})\abs{\A}
= 1 \cdot \abs{\A}
= \abs{\A} \neq 0\),
即\(\V\)可逆.
在(1)式两端左乘\(\P\)得\(\P(\P^{-1}\A\P) = \P\V\)
即\begin{gather}
	\A\P = \P\V.
	\tag2
\end{gather}
在(2)式两端左乘\(\P^{-1}\A^{-1}\),
右乘\(\V^{-1}\)得\[
	(\P^{-1}\A^{-1})(\A\P)\V^{-1} = (\P^{-1}\A^{-1})(\P\V)\V^{-1},
\]
即\(\V^{-1} = \P^{-1}\A^{-1}\P\).
\end{proof}
\end{example}

\begin{example}
设\(m\)阶矩阵\(\A\)与\(n\)阶矩阵\(\B\)都可对角化,证明:\(m+n\)阶矩阵\[
	\begin{bmatrix} \A & \z \\ \z & \B \end{bmatrix}
\]可对角化.
\begin{proof}
设存在\(m\)阶可逆矩阵\(\P\)和\(n\)阶可逆矩阵\(\Q\)使得
\begin{align*}
	\P^{-1}\A\P &= \V_1 \\
	\Q^{-1}\B\Q &= \V_2
\end{align*}
则可构造矩阵使得\[
	\begin{bmatrix}
		\P^{-1} & \z \\
		\z & \Q^{-1}
	\end{bmatrix}
	\begin{bmatrix} \A & \z \\ \z & \B \end{bmatrix}
	\begin{bmatrix}
		\P & \z \\
		\z & \Q
	\end{bmatrix}
	= \begin{bmatrix}
		\V_1 & \z \\
		\z & \V_2
	\end{bmatrix}.
	\qedhere
\]
\end{proof}
\end{example}

\begin{example}
设\(\A\)是非零的幂零矩阵,
即\(\A\neq\z\),且存在自然数\(m\),使得\(\A^m=\z\).
证明:\(\A\)的特征值全为零,且\(\A\)不可以对角化.
\begin{proof}
设存在非零列向量\(\X0\),使得\(\A\X0=\L0\X0\)成立,则\[
\A^m\X0=\L0^m\X0.
\]
令\(\A^m=\z\),则解\(\A^m\X0=\z=\L0^m\X0\)得\(\L0=0\),可见\(\A\)的特征值全为零.

假设\(\A\)可以对角化,即存在可逆矩阵\(\P\)使得\[
	\P^{-1}\A\P = \diag(\L{1},\L{2},\dotsc,\L{n}) = \z.
\]
在等式两边同时左乘\(\P\),并右乘\(\P^{-1}\),得\[
	\A = \P(\P^{-1}\A\P)\P^{-1} = \P\z\P^{-1} = \z.
\]矛盾,故\(\A\)不可以对角化.
\end{proof}
\end{example}

\begin{example}
设\(\A \in M_n(K)\),
\(\rank\A=1\).
证明:\[
	\tr\A\neq0
	\iff
	\text{\(\A\)可相似对角化}.
\]
\begin{proof}
因为\(\rank\A=1
\iff
(\exists\a,\b \in K^n-\{\vb0\})[\A=\a\b^T]\),
所以根据\cref{example:矩阵乘积的秩.两个向量的乘积的特征值和特征向量},
\(\A\)的特征值为\(\tr\A\)和\(0\ (\text{$n-1$重})\).
又因为\(\rank(0\E-\A)=\rank\A=1\),
所以根据\cref{theorem:矩阵可对角化的充分必要条件.定理3} 可知,
\(\A\)可以对角化.
\end{proof}
\end{example}

\begin{example}
\def\J{\vb{J}_n}
形式为\[
	\J = \begin{bmatrix}
		\L0 & 0 & 0 & \dots & 0 & 0 \\
		1 & \L0 & 0 & \dots & 0 & 0 \\
		0 & 1 & \L0 & \dots & 0 & 0 \\
		\vdots & \vdots & \vdots & \ddots & \vdots & \vdots \\
		0 & 0 & 0 & \dots & \L0 & 0 \\
		0 & 0 & 0 & \dots & 1 & \L0
	\end{bmatrix}_n
\]的复数三角形阵称为\DefineConcept{若尔当块}.
证明:\(n>1\)阶若尔当块不可以对角化.
\begin{proof}
令\(\abs{\l\E-\J}=(\lambda-\L0)^n=0\),解得\(\l=\L0\)(\(n\)重),那么\[
	\L0\E-\J = \begin{bmatrix}
		0 \\
		-1 & 0 \\
		& -1 & 0 \\
		& & \ddots & \ddots \\
		& & & -1 & 0
	\end{bmatrix}_n,
\]
\(\rank(\L0\E-\J)=n-1 > 0\),
故当\(n>1\)时\(\J\)不可以对角化.
\end{proof}
\end{example}

\begin{definition}
由若干个若尔当块构成的准对角矩阵称为\DefineConcept{若尔当形矩阵}.
\end{definition}

\begin{theorem}
每个\(n\)阶复数矩阵不一定与对角阵相似,但必与一个若尔当形矩阵相似.
\end{theorem}

\section{正交矩阵}
在平面上取一个直角坐标系\(Oxy\),
设向量\(\a,\b\)的坐标分别是\((a_1,a_2),(b_1,b_2)\).
如果\(\a,\b\)都是单位向量,并且互相垂直,
那么它们的坐标满足:\[
	\begin{split}
		a_1^2+a_2^2=1, \qquad
		a_1b_1+a_2b_2=0, \\
		b_1a_1+b_2a_2=0, \qquad
		b_1^2+b_2^2=1,
	\end{split}
\]
这组等式可以写成一个矩阵等式:\[
	\begin{bmatrix}
		a_1 & a_2 \\
		b_1 & b_2
	\end{bmatrix}
	\begin{bmatrix}
		a_1 & b_1 \\
		a_2 & b_2
	\end{bmatrix}
	= \begin{bmatrix}
		1 & 0 \\
		0 & 1
	\end{bmatrix}.
\]
如果记\(\A=(\a^T,\b^T)\),
那么上式又可写为\[
	\A^T\A=\E.
\]
根据\(\a,\b\)的几何意义,
我们很自然地把矩阵\(\A\)称为“正交矩阵”.

这一节我们来研究正交矩阵的性质,尤其是它的行(列)向量组的特性.

\begin{definition}
在欧几里得空间中,如果
\begin{enumerate}
	\item 向量组\(A=\{\AutoTuple{\a}{m}\}\)不含零向量,即\(\z \notin A\);
	\item \(A\)中向量两两正交,即\(\a_i \cdot \a_j = 0\ (i \neq j)\),
\end{enumerate}
则称\(A\)为一个\DefineConcept{正交向量组},简称\DefineConcept{正交组}.
由单位向量构成的正交组叫做\DefineConcept{规范正交组}或\DefineConcept{标准正交组}.
称含有\(n\)个向量的规范正交组
\[
	\AutoTuple{\e}{n}
\]
为\(\mathbb{R}^n\)的一个\DefineConcept{规范正交基}%
或\DefineConcept{标准正交基}(orthonormal basis).
\end{definition}

\begin{definition}\label{definition:正交矩阵.正交矩阵的定义}
%@see: 《高等代数(第三版 上册)》(丘维声) P145. 定义1
设\(\Q \in M_n(\mathbb{R})\).
若\(\Q\)满足\begin{equation}\label{equation:正交矩阵.正交矩阵的定义式}
	\Q^T\Q = \E,
\end{equation}
则称“\(\Q\)是\DefineConcept{正交矩阵}”.
\end{definition}

在正交矩阵的定义式 \labelcref{equation:正交矩阵.正交矩阵的定义式} 等号两端分别取行列式,
利用\cref{theorem:行列式.矩阵乘积的行列式,theorem:行列式.性质1} 得
\begin{equation}\label{equation:正交矩阵.正交矩阵的行列式}
	\abs{\Q}^2
	=\abs{\Q^T}\abs{\Q\vphantom{^T}}
	=\abs{\Q^T\Q}
	=\abs{\E}
	=1,
\end{equation}
于是\(\abs{\Q}\neq0\).
根据\cref{theorem:逆矩阵.矩阵可逆的充要条件1} 我们知道,任一正交矩阵可逆.
再根据\cref{definition:可逆矩阵.可逆矩阵的定义}\[
	\Q^{-1}\Q=\E,
\]
将上式与\cref{equation:正交矩阵.正交矩阵的定义式} 比较,可知\begin{equation}
	\Q^T=\Q^{-1}.
\end{equation}
那么由\(\Q\Q^{-1}=\E\)便得\begin{equation}
	\Q\Q^T=\E.
\end{equation}

\begin{example}
由于单位矩阵\(\E\)满足\[
	\E^T=\E, \qquad
	\E^T \E = \E \E^T = \E,
\]
因此\(\E\)也是正交矩阵.
\end{example}

\begin{property}
设\(\Q\)是正交矩阵,则\(\abs{\Q} = \pm1\).
\begin{proof}
由\cref{equation:正交矩阵.正交矩阵的行列式}
得\(\abs{\Q} = \pm 1\).
\end{proof}
\end{property}

\begin{property}
若\(\A,\B\)都是\(n\)阶正交矩阵,则\(\A\B\)也是正交矩阵.
\end{property}

\begin{property}
若\(\A\)是正交矩阵,则\(\A^T\)和\(\A^{-1}\)也是正交矩阵.
\end{property}

\begin{theorem}
正交矩阵\(\Q\)的伴随矩阵\(\Q^*\)满足\[
	\Q^* = \abs{\Q} \Q^{-1}
	= \abs{\Q} \Q^T
	= \left\{ \begin{array}{rc}
		\Q^T, & \abs{\Q}>0, \\
		-\Q^T, & \abs{\Q}<0.
	\end{array} \right.
\]
\end{theorem}

\begin{example}
设\(\Q=(\AutoTuple{\a}{n})\)是\(n\)阶实矩阵,
则\(\Q\)是正交矩阵的充要条件是\(\AutoTuple{\a}{n}\)是\(\mathbb{R}^{n \times 1}\)的规范正交基.
\begin{proof}
在\(\Q\)是\(n\)阶实矩阵的前提下,\begin{align*}
	&\text{\(\Q\)是正交矩阵}
	\iff \Q^T\Q = \Q\Q^T = \E \\
	&\iff \E = \begin{bmatrix}
		\a_1^T \\ \a_2^T \\ \vdots \\ \a_n^T
	\end{bmatrix} (\AutoTuple{\a}{n})
	= \begin{bmatrix}
		\a_1^T \a_1 & \a_1^T \a_2 & \dots & \a_1^T \a_n \\
		\a_2^T \a_1 & \a_2^T \a_2 & \dots & \a_2^T \a_n \\
		\vdots & \vdots & & \vdots \\
		\a_n^T \a_1 & \a_n^T \a_2 & \dots & \a_n^T \a_n
	\end{bmatrix} \\
	&\iff \a_i^T \a_j = (\a_i,\a_j)
	= \left\{ \begin{array}{ll}
		1, & i=j, \\
		0, & i \neq j,
	\end{array} \right. i,j=1,2,\dotsc,n \\
	&\iff \text{\(\AutoTuple{\a}{n}\)是规范正交基}.
	\qedhere
\end{align*}
\end{proof}
\end{example}

可以看出,正交矩阵是由一系列初等矩阵\(\P(i,j)\)的乘积.

\begin{example}
设\(\A\)是正交矩阵.
证明:\(\A\)的特征值是模为1的复数.
\begin{proof}
设\(\L0\)是\(\A\)的任意一个特征值,
\(\X0=(\AutoTuple{c}{n})^T\neq\z\)是\(\A\)属于特征值\(\L0\)的特征向量,
即\begin{gather}
	\A\X0=\L0\X0,
	\tag1
\end{gather}
取共轭转置,得\begin{gather}
	\overline{\X0}^T \overline{\A}^T = \overline{\L0} \overline{\X0}^T,
	\tag2
\end{gather}
将(1)式两端分别右乘到(2)式两端,
得\begin{gather}
	\overline{\X0}^T \overline{\A}^T \A\X0 = \overline{\L0} \overline{\X0}^T \L0\X0,
	\tag3
\end{gather}
因为\(\A\)是正交矩阵,
\(\A\)是实矩阵,
\(\overline{\A}^T = \A^T\),
且\(\A^T\A=\A\A^T=\E\),
所以\[
	\overline{\X0}^T \X0 = \overline{\L0} \L0 \overline{\X0}^T \X0,
\]\[
	(\overline{\L0} \L0 - 1) \overline{\X0}^T \X0 = 0,
\]
又因为\(\overline{\X0}^T \X0
= \overline{c_1}c_1 + \overline{c_2}c_2 + \dotsb + \overline{c_n}{c_n} > 0\),
所以\[
	\overline{\L0} \L0 = 1,
\]
即\(\A\)的特征值\(\L0\)是模为1的复数.
\end{proof}
\end{example}

\section{正交规范化方法}
在许多实际问题中,我们需要构造正交矩阵,于是我们要设法求标准正交基.
以下讨论将线性无关向量组改造为规范正交组的施密特正交规范化方法.

\begin{theorem}
设\(\a_1,\a_2,\dotsc,\a_m\)是\(\mathbb{R}^n\)中的一个线性无关组,
令\begin{align*}
	\b_1 &= \a_1, \\
	\b_k &= \a_k - \sum\limits_{i=1}^{k-1}
		\frac{\vectorinnerproduct{\a_k}{\b_i}}{\vectorinnerproduct{\b_i}{\b_i}} \b_i,
	\quad k=2,3,\dotsc,m,
\end{align*}
再将之单位化(或规范化)得\[
	\g_k = \frac{1}{\abs{\b_k}}\b_k,
	\quad k=1,2,\dotsc,m,
\]
则\(\g_1,\g_2,\dotsc,\g_m\)是一个规范正交组,且满足\[
	\{\a_1,\a_2,\dotsc,\a_j\}
	\cong
	\{\g_1,\g_2,\dotsc,\g_j\},
	\quad j=1,2,\dotsc,m.
\]
\end{theorem}

\section{正规矩阵}
\begin{definition}
若矩阵\(\A\)满足\(\A\A^H = \A^H\A\),则称该矩阵为\DefineConcept{正规矩阵}.
\end{definition}

\begin{property}
实正交矩阵、实对称矩阵、实反对称矩阵都是实正规矩阵.
\end{property}

\section{实对称矩阵的对角化}
由上节讨论我们知道,\(n\)阶矩阵分成可对角化与不可对角化两类.
实际上,矩阵的相似概念与数域有关,矩阵能否对角化也与数域有关.
因为一般矩阵的特征值是复数,即使\(\A\)的元素都是实数,也可能没有实数特征值.
例如,\(\A = \begin{bmatrix} 0 & -1 \\ 1 & 0 \end{bmatrix}\)是一个二阶实矩阵,
由于\[
	\abs{\l\E-\A}
	= \begin{vmatrix}
		\l & 1 \\
		-1 & \l
	\end{vmatrix}
	= (\l+\iu)(\l-\iu),
\]
可以看出\(\A\)有两个特征值\(\pm\iu\),
其对应的特征向量\((\iu,1)^T\)和\((\iu,-1)^T\)是复向量,
因此\(\A\)在复数域上可对角化,
但不存在可逆实阵\(\P\)使得\(\P^{-1}\A\P\)为对角阵.

\begin{theorem}\label{theorem:特征值与特征向量.实对称矩阵1}
实对称矩阵的特征值都是实数.
\begin{proof}
设\(\A \in M_n(\mathbb{R})\)满足\(\A^T=\A\).
显然\(\A\)的特征多项式\(\abs{\l\E-\A}\)在复数范围内有\(n\)个根.
假设\(\L0\in\mathbb{C}\)是\(\A\)的任意一个特征值,
则存在\(n\)维复向量\(\X0=(\AutoTuple{c}{n})^T \neq \z\),使得
\begin{gather}
	\A\X0 = \L0 \X0, \tag1
\end{gather}
用\(\X0\)的共轭转置向量\(\overline{\X0}^T
=(\overline{c_1},\overline{c_2},\dotsc,\overline{c_n})\)
左乘(1)式两端,
得\begin{gather}
	\overline{\X0}^T \A \X0 = \l \overline{\X0}^T \X0, \tag2
\end{gather}
其中\(\A^T=\A=\overline{\A}\),
\(\overline{\X0}^T \X0
= \overline{c_1}c_1 + \overline{c_2}c_2 + \dotsb + \overline{c_n}{c_n} \in \mathbb{R}^+\).

又因为\(\overline{\X0}^T \A \X0 \in \mathbb{C}\)取转置不变,
且实对称矩阵满足\(\A^T = \A = \overline{\A}\),
所以\[
	\overline{\X0}^T \A \X0
	= (\overline{\X0}^T \A \X0)^T
	= \X0^T \A^T \overline{\X0}
	= \overline{\overline{\X0}^T} \overline{\A} \overline{\X0}
	= \overline{\overline{\X0}^T \A \X0},
\]
说明\(\overline{\X0}^T \A \X0 \in \mathbb{R}\),进而有\(\L0 \in \mathbb{R}\).
\end{proof}
\end{theorem}

\begin{theorem}\label{theorem:特征值与特征向量.实对称矩阵2}
实对称矩阵\(\A\)的不同特征值所对应的特征向量正交.
\begin{proof}
设\(\L{1}\neq\L{2}\)是\(\A\)的两个不同的特征值,
\(\X1\neq\z\),
\(\X2\neq\z\)分别是\(\A\)对应于\(\L{1}\)、\(\L{2}\)的特征向量,
则\(\X1\)、\(\X2\)都是实向量,
\begin{align*}
	\A\X1 &= \L{1}\X1, \tag1 \\
	\A\X2 &= \L{2}\X2. \tag2
\end{align*}
对(1)式左乘\(\X2^T\),得\begin{gather}
	\X2^T \A \X1 = \L{1} \X2^T \X1, \tag3
\end{gather}
对(2)式左乘\(\X1^T\),得\begin{gather}
	\X1^T \A \X2 = \L{2} \X1^T \X2, \tag4
\end{gather}
(3)式取转置,
得\(\L{1}(\X2^T \X1)^T = (\X2^T \A \X1)^T\),
又由\(\A=\A^T\),
得\begin{gather}
	\L{1} \X1^T \X2 = \X1^T \A^T \X2 = \X1^T \A \X2, \tag5
\end{gather}
(5)式减(4)式,得
\begin{gather}
	(\L{2}-\L{1})\X1^T\X2=0. \tag6
\end{gather}
因为\(\L{2} \neq \L{1}\),
所以\(\X1^T \X2 = 0\),
即\((\X1,\X2) = 0\),
也就是\(\X1\)与\(\X2\)正交.
\end{proof}
\end{theorem}

一般地,对于\(n\)阶实对称矩阵\(\A\),
属于\(\A\)的同一特征值的一组线性无关的特征向量不一定相互正交,
可用施密特正交化方法将其正交化,得到\(\A\)的属于该特征值的正交特征向量组.
由以上定理,\(\A\)的几个属于不同特征值的正交特征向量组仍构成正交组.
特别地,\(\A\)有\(n\)个正交的特征向量,\(\A\)相似于对角形矩阵.

\begin{theorem}\label{theorem:特征值与特征向量.实对称矩阵3}
若\(\A\)为\(n\)阶实对称矩阵,则一定存在正交矩阵\(\Q\),使得\(\V = \Q^{-1}\A\Q\)为对角形矩阵.
\begin{proof}
\def\M{\vb{M}}%
用数学归纳法.
当\(n=1\)时,矩阵\(\A\)是一个实数\(a_{11}\),定理成立.
假设当\(n=k-1\)时定理成立,下面证明当\(n=k\)时定理也成立.

由\cref{theorem:特征值与特征向量.实对称矩阵1},
\(\A\)的特征值全为实数.
假设\(\L1\)是\(\A\)的特征值,
并且相应地存在非零实向量\(\X1=(\AutoTuple{c}{n})^T\),
使得\[
	\A\X1=\L1\X1.
\]
不妨设\(c_1\neq0\),则\(n\)元向量组\[
	\X1,\X2=(0,1,\dotsc,0)^T,\dotsc,\X{n}=(0,0,\dotsc,1)^T
\]线性无关.
对向量组\(X=\{\AutoTuple{\x}{n}\}\)
用施密特正交规范化方法可得规范正交组\(Y=\{\AutoTuple{\y}{n}\}\),
则\(\P=(\AutoTuple{\y}{n})\)是正交矩阵,
其中\(\y_1=\abs{\X1}^{-1}\X1\)是\(\A\)的特征向量.

因为\(\mathbb{R}^n\)中任意\(n+1\)个向量线性相关,
故任意向量都可由\(Y\)线性表出,
即\begin{align*}
	\A\y_1 &= \L1\y_1 = \L1\y_1 + 0\y_2 + \dotsb + 0\y_n, \\
	\A\y_k &= b_{1k}\y_1 + b_{2k}\y_2 + \dotsb + b_{nk}\y_n \quad(k=2,3,\dotsc,n),
\end{align*}
由分块矩阵乘法,\[
	\A(\AutoTuple{\y}{n})
	= (\AutoTuple{\y}{n})
	\begin{bmatrix}
		\l_1 & b_{11} & \dots & b_{1n} \\
		0 & b_{22} & \dots & b_{2n} \\
		\vdots & \vdots & & \vdots \\
		0 & b_{n2} & \dots & b_{nn}
	\end{bmatrix},
\]
令\(\P=(\AutoTuple{\y}{n})\),
显然\(\P\)是正交阵,
而\(\P^{-1}\A\P=\P^T\A\P=\begin{bmatrix}
	\L1 & \a \\
	\z & \B
\end{bmatrix}\).

由\((\P^T\A\P)^T=\P^T\A^T\P=\P^T\A\P\)可知\(\begin{bmatrix}
	\L1 & \a \\
	\z & \B
\end{bmatrix}
= \begin{bmatrix}
	\L1 & \z \\
	\a^T & \B^T
\end{bmatrix}\),
于是\(\a=\z\),\(\B=\B^T\),也就是说\(\B\)是\(n-1\)阶实对称矩阵.
又由归纳假设,存在\(n-1\)阶正交阵\(\M\),使得\(\M^{-1}\B\M\)成为对角阵.

令\(\Q=\P\begin{bmatrix} 1 & \z \\ \z & \M \end{bmatrix}\),
因为\(\Q\Q^T=\Q^T\Q=\E\),所以\(\Q\)是正交矩阵,
而\begin{align*}
	\Q^{-1}\A\Q
	&=\begin{bmatrix}
		1 & \z \\
		\z & \M^{-1}
	\end{bmatrix}\P^{-1}\A\P\begin{bmatrix}
		1 & \z \\
		\z & \M
	\end{bmatrix}=\begin{bmatrix}
		1 & \z \\
		\z & \M^{-1}
	\end{bmatrix}\begin{bmatrix}
		\L1 & \z \\
		\z & \B
	\end{bmatrix}\begin{bmatrix}
		1 & \z \\
		\z & \M
	\end{bmatrix} \\
	&=\begin{bmatrix}
		\L1 & \z \\
		\z & \M^{-1}\B\M
	\end{bmatrix}
	=\diag(\AutoTuple{\lambda}{n}).
\end{align*}
由上可知当\(n=k\)时定理也成立.
\end{proof}
\end{theorem}

\cref{theorem:特征值与特征向量.实对称矩阵3} 表明,
实对称矩阵总可以相似对角化,
实对称矩阵总是\DefineConcept{正交相似}({orthogonally similar})于某个对角形矩阵.

\begin{corollary}
\(n\)阶实对称矩阵\(\A\)存在\(n\)个正交的单位特征向量.
\end{corollary}

\begin{remark}
\color{red}
对于实对称矩阵\(\A\),求正交矩阵\(\Q\),使得\(\Q^{-1}\A\Q\)为对角形矩阵的方法:
\begin{enumerate}
	\item 求出\(\A\)的全部不同的特征值\(\AutoTuple{\lambda}{m}\);
	\item 求出\((\L{i}\E-\A)\x=\z\)的基础解系,将其正交化,
	得到\(\A\)属于\(\L{i}\)的正交特征向量(\(i=1,2,\dotsc,m\)),
	共求出\(\A\)的\(n\)个正交特征向量;
	\item 将以上\(n\)个正交特征向量单位化,由所得向量作为列构成正交矩阵\(\Q\),则\[
		\Q^{-1}\A\Q = \Q^T \A \Q = \diag(\AutoTuple{\lambda}{n}).
	\]
\end{enumerate}
\end{remark}

\begin{example}
设\(\A\)为\(n\)阶实对称矩阵,满足\(\A^2=\E\),证明:存在正交矩阵\(\Q\),使得\[
	\Q^{-1}\A\Q=\begin{bmatrix} \E_r \\ & -\E_{n-r} \end{bmatrix}.
\]
\begin{proof}
因为\(\A\)为\(n\)阶实对称矩阵,
则\(\A\)有\(n\)个实特征值,
\(\A\)有\(n\)个正交的单位特征向量,
适当调整它们的顺序,可以构成正交矩阵\(\Q\),
满足\begin{gather}
	\Q^{-1}\A\Q=\diag(\AutoTuple{\lambda}{n}), \tag1
\end{gather}
其中,\(\L{i}>0\ (i=1,2,\dotsc,r),
\L{i}\leq0\ (i=r+1,r+2,\dotsc,n)\).
对(1)式两端分别平方,又由\(\A^2=\E\),得\[
	\Q^{-1}\A^2\Q
	= \Q^{-1}\E\Q
	= \E
	= \diag(\L{1}^2,\L{2}^2,\dotsc,\L{n}^2),
\]
于是\(\L{i}^2=1\ (i=1,2,\dotsc,n)\),
进而有\[
	\L{i}= \begin{cases}
		1, & i=1,2,\dotsc,r, \\
		-1, & i=r+1,r+2,\dotsc,n.
	\end{cases}
	\qedhere
\]
\end{proof}
\end{example}

\begin{example}
设\(\A\)是4阶实对称矩阵,且\(\A^2+\A=\z\).
若\(\rank\A=3\),求\(\A\)的特征值以及与\(\A\)相似的对角阵.
\begin{solution}
\(\A\)是实对称矩阵,根据\cref{theorem:特征值与特征向量.实对称矩阵3},
\(\A\)一定可对角化,不妨设\(\A\x=\l\x\ (\x\neq0)\),
那么\(\A^2\x=\l^2\x\),\((\A^2+\A)\x=(\l^2+\l)\x\).
因为\(\A^2+\A=\z\),所以\((\l^2+\l)\x=\z\),\(\l^2+\l=0\),
解得\(\A\)的特征值为\(\l=0,-1\).
又因为\(\rank\A=3\),所以\(\A\)具有3个非零特征值,
因此与\(\A\)相似的对角阵为\(\diag(-1,-1,-1,0)\).
\end{solution}
\end{example}

\begin{example}
设\(\A\)、\(\B\)是两个\(n\)阶正交矩阵.证明:
\begin{enumerate}
	\item \(\A\B\)是正交矩阵;
	\item \(\A^{-1}\)是正交矩阵;
\end{enumerate}
\begin{proof}
因为\(\A\)、\(\B\)是两个\(n\)阶正交矩阵,所以\(\A\A^T = \E\),\(\B\B^T = \E\).
\begin{enumerate}
	\item \((\A\B)(\A\B)^T
	= (\A\B)(\B^T\A^T)
	= \E\).

	\item \(\A^{-1}(\A^{-1})^T
	= \A^{-1} (\A^T)^{-1}
	= \A^{-1} (\A^{-1})^{-1}
	= \A^{-1} \A
	= \E\).

	\qedhere
\end{enumerate}
\end{proof}
\end{example}

\begin{example}
设\(\A\)是特征值仅为1与0的\(n\)阶实对称矩阵,证明:\(\A^2=\A\).
\begin{proof}
\def\M{\begin{bmatrix} \E_r \\ & \z_{n-r} \end{bmatrix}}%
因为\(\A\)是实对称矩阵,所以存在正交矩阵\(\Q\)使得\[
	\Q^{-1}\A\Q = \M,
\]
从而有\[
	\A = \Q\M\Q^{-1},
\]
进而有\[
	\A^2 = \Q\M\Q^{-1}\Q\M\Q^{-1} = \Q\M\Q^{-1} = \A.
	\qedhere
\]
\end{proof}
\end{example}

\begin{example}
设\(\A\)为\(n\)阶实对称矩阵,满足\(\A^2=\z\),证明:\(\A=\z\).
\begin{proof}
因为\(\A\)是实对称矩阵,所以存在正交矩阵\(\Q\)使得\[
	\Q^{-1}\A\Q = \diag(\AutoTuple{\lambda}{n}) = \V,
\]
从而有\(\A = \Q\V\Q^{-1}\),\(\A^2 = (\Q\V\Q^{-1})^2 = \Q\V^2\Q^{-1} = \z\),那么\[
	\V^2 = \diag(\L{1}^2,\L{2}^2,\dotsc,\L{n}^2) = \Q^{-1}\z\Q = \z,
\]\[
	\L{1}=\L{2}=\dotsb=\L{n} = 0,
\]
所以\(\A=\z\).
\end{proof}
\end{example}

\section{实反对称矩阵的对角化}
\begin{theorem}
%@see: 《线性代数》(张慎语、周厚隆) P113. 习题5.3 6.
实反对称矩阵的特征值为零或纯虚数.
\begin{proof}
设\(\A \in M_n(\mathbb{R})\)满足\(\A^T=-\A\).
又设\(\mathbb{C} \ni \L0 = a_0 + \iu b_0\ (a_0,b_0 \in \mathbb{R})\)是\(\A\)的任意一个特征值,
\(\mathbb{C}^{n \times 1} \ni \X0=(\AutoTuple{c}{n})^T \neq \z\)是
\(\A\)属于特征值\(\L0\)的特征向量,
即\begin{gather}
\A\X0 = \L0\X0, \tag1
\end{gather}
在(1)式两端左乘\(\overline{\X0}^T\),
得\begin{gather}
	\overline{\X0}^T \A \X0
	= \L0\ \overline{\X0}^T \X0, \tag2
\end{gather}
取共轭转置,得\begin{gather}
\overline{\X0}^T \A^T \X0
= \overline{\L0}\ \overline{\X0}^T \X0, \tag3
\end{gather}
由于\(\A\)是实反对称矩阵,即\(\A^T = -\A\),所以\begin{gather}
	\overline{\X0}^T \A \X0
	= -\overline{\L0}\ \overline{\X0}^T \X0, \tag4
\end{gather}
其中,\(\overline{\X0}^T \X0
= \overline{c_1}c_1 + \overline{c_2}c_2 + \dotsb + \overline{c_n}{c_n} > 0\).
由(2)式与(4)式,得\[
	\L0\ \overline{\X0}^T \X0
	= -\overline{\L0}\ \overline{\X0}^T \X0,
\]\[
	\L0 = -\overline{\L0},
\]\[
	a_0 + \iu b_0 = -(a_0 - \iu b_0) = -a_0 + \iu b_0,
\]\[
	\Re \L0 = a_0 = 0.
\]
也就是说\(\L0\)要么为零要么为纯虚数.
\end{proof}
\end{theorem}

\section{矩阵的分解}
\subsection{LU分解}
\begingroup
\def\L{\vb{L}}%
\def\U{\vb{U}}%
\begin{theorem}
设\(\A = (a_{ij})_n \in M_n(\mathbb{R})\),
存在下三角阵\(\L = (l_{ij})_n\)和上三角阵\(\U = (u_{ij})_n\),
使得\(\A = \L \U\),
其中\(l_{ii} = 1\ (i=1,2,\dotsc,n),
l_{ij} = 0\ (i<j),
u_{ij} = 0\ (i>j)\).
\end{theorem}

举例来说,令\[
	\A = \begin{bmatrix}
		a_{11} & a_{12} \\
		a_{21} & a_{22}
	\end{bmatrix}
	= \begin{bmatrix}
		1 & 0 \\
		l_{21} & 1
	\end{bmatrix}
	\begin{bmatrix}
		u_{11} & u_{12} \\
		0 & u_{22}
	\end{bmatrix}
	= \L \U,
\]
得\[
	\left.\begin{array}{r}
		1 \cdot u_{11} + 0 \cdot 0 = a_{11} \\
		1 \cdot u_{12} + 0 \cdot u_{22} = a_{12} \\
		l_{21} u_{11} + 1 \cdot 0 = a_{21} \\
		l_{21} u_{12} + 1 \cdot u_{22} = a_{22}
	\end{array}\right\}
	\implies
	\left\{\begin{array}{l}
		u_{11} = a_{11}, \\
		u_{12} = a_{12}, \\
		l_{21} = a_{21} / u_{11}, \\
		u_{22} = a_{22} - l_{21} u_{12}.
	\end{array}\right.
\]

又令\[
	\A = \begin{bmatrix}
		a_{11} & a_{12} & a_{13} \\
		a_{21} & a_{22} & a_{23} \\
		a_{31} & a_{32} & a_{33}
	\end{bmatrix}
	= \begin{bmatrix}
		1 & 0 & 0 \\
		l_{21} & 1 & 0 \\
		l_{31} & l_{32} & 1
	\end{bmatrix}
	\begin{bmatrix}
		u_{11} & u_{12} & u_{13} \\
		0 & u_{22} & u_{23} \\
		0 & 0 & u_{33}
	\end{bmatrix} = \L \U,
\]
得\[
	\left\{\begin{array}{l}
		u_{11} = a_{11}, \\
		u_{12} = a_{12}, \\
		u_{13} = a_{13}, \\
		l_{21} = a_{21} / u_{11}, \\
		l_{31} = a_{31} / u_{11}, \\
		u_{22} = a_{22} - l_{21} \cdot u_{12}, \\
		u_{23} = a_{23} - l_{21} \cdot u_{13}, \\
		l_{32} = (a_{32} - l_{31} \cdot u_{12}) / u_{22}, \\
		u_{33} = a_{33} - (l_{31} \cdot u_{13} + l_{32} \cdot u_{23}).
	\end{array}\right.
\]
\endgroup%LU分解

\subsection{谱分解}
\begin{theorem}
设\(\A \in M_n(\mathbb{R})\),
数\(\L{1},\L{2},\dotsc,\L{n}\)是\(\A\)的\(n\)个特征值,
且\[
	\L{1}\leq\L{2}\leq\dotsb\leq\L{n},
\]
而\(\X{1},\X{2},\dotsc,\X{n}\)是其对应的\(n\)个线性无关的特征向量,
则存在正交矩阵\(\Q\),使得\[
	\Q^{-1}\A\Q = \Q^T\A\Q = \diag(\L{1},\L{2},\dotsc,\L{n}).
\]
\end{theorem}

\subsection{奇异值分解}
\begin{theorem}
\def\U{\vb{U}}
\def\S{\vb{\Sigma}}
\def\V{\vb{V}}
\let\Q\V
\let\P\U
\def\p{\vb{u}}
\def\q{\vb{v}}
设矩阵\(\A \in M_{m \times n}(\mathbb{R})\),
则存在\(m\)阶正交矩阵\(\U\)、\(n\)阶正交矩阵\(\V\)和\(m \times n\)对角阵\(\S\),
使得\[
	\A = \U \S \V^T,
\]
其中\(\S = (\sigma_{ij})_{m \times n}\)的元素\(\sigma_{ij}\)满足\[
	\sigma_{ij} = \left\{ \begin{array}{cc}
	0, & i \neq j, \\
	s_i \geq 0, & i = j.
	\end{array} \right.
\]

这里,矩阵\(\S\)的对角元\(s_i\)称为\(\A\)的\DefineConcept{奇异值}(通常按\(s_i \geq s_{i+1}\)排列),
\(\U\)的列分块向量称为\(\A\)的\DefineConcept{左奇异向量},
\(\V\)的列分块向量称为\DefineConcept{右奇异向量}.
\begin{proof}
由于\(\A^T \A \in M_n(\mathbb{R})\),故可作谱分解,即存在正交矩阵\(\Q\),使得\[
	\Q^{-1}\A\Q = \Q^T\A\Q = \diag(\L{1},\L{2},\dotsc,\L{n}),
\]
其中\(\Q=(\AutoTuple{\q}{n})\)中的列分块向量\(\q_i\)是\(\A^T \A\)对应于特征值\(\L{i}\)的特征向量,
而\(\{\AutoTuple{\q}{n}\}\)构成\(\mathbb{R}^n\)的一组标准正交基.

注意到\(\A^T \A\)是半正定矩阵\footnote{当\(\A^T \A\)是可逆矩阵时,\(\A^T \A\)是正定矩阵.},
故其特征值\(\L{i}\geq0\).

考虑映射\(\A_{m \times n}\colon \mathbb{R}^n \to \mathbb{R}^m, \x \mapsto \A\x\),
设\(\rank\A = r\),
将\(\A\)作用到\(\mathbb{R}^n\)的标准正交基\(\{\AutoTuple{\q}{n}\}\)上,
利用维数公式,得\[
\dim(\ker \A) + \dim(\Im \A) = n,
\]
可知\(\A\q_1,\A\q_2,\dotsc,\A\q_n\)这\(n\)个向量中有\(r\)个向量构成\(\mathbb{R}^m\)的一组部分基,
而\(\A\q_{r+1} = \A\q_{r+2} = \dotsb = \A\q_n = 0\).

有\(\A^T \A \q_j = \L{j} \q_j\),又有\[
	\q_i \cdot \q_j = \q_i^T \q_j
	= \left\{ \begin{array}{lc}
		1, & i=j, \\
		0, & i \neq j.
	\end{array} \right.
\]
所以,当\(i \neq j\)时,\[
	(\A\q_i)\cdot(\A\q_j) = \q_i^T \A^T \A \q_j = \L{j} \q_i^T \q_j = 0;
\]
而当\(i = j\)时,\[
	\abs{\A\q_i}^2 = (\A\q_i)\cdot(\A\q_j) = \L{i} \q_i^T \q_i = \L{i}.
\]
也就是说,向量组\(\{\A\q_1,\A\q_2,\dotsc,\A\q_r\}\)是两两正交的.
单位化该向量组,又记\[
	\p_i = \frac{\A\q_i}{\abs{\A\q_i}}
	= \frac{\A\q_i}{\sqrt{\L{i}}}
	\quad(i=1,2,\dotsc,r),
\]
于是\(\A\q_i = s_i \p_i\),其中\(s_i = \sqrt{\L{i}}\).

将\(\p_1,\p_2,\dotsc,\p_r\)扩充成\(\mathbb{R}^m\)的标准正交基
\(\{\p_1,\p_2,\dotsc,\p_r,\p_{r+1},\dotsc,\p_m\}\),
在这组基下,有\[
	\A\Q = \A(\AutoTuple{\q}{n}) = \begin{bmatrix}
		s_1 \p_1 \\
		& \ddots \\
		& & s_r \p_r \\
		& & & 0 \\
		& & & & \ddots \\
		& & & & & 0
	\end{bmatrix}
	= \P \S,
\]
其中\(\P = (\p_1,\p_2,\dotsc,\p_m)\),
\(\S = \diag(s_1,\dotsc,s_r,0,\dotsc,0)\),
将上式两边右乘\(\Q^{-1}\),即得\(\A = \P\S\Q^T\).
\end{proof}
\end{theorem}

\begin{example}
\def\U{\vb{U}}
\def\S{\vb{\Sigma}}
\def\V{\vb{V}}
\def\M#1{\mu_{#1}}
对矩阵\(\A = \begin{bmatrix} 0 & 1 \\ 1 & 1 \\ 1 & 0 \end{bmatrix}\)进行奇异值分解.
\begin{solution}
经计算\[
	\A^T \A = \begin{bmatrix} 2 & 1 \\ 1 & 2 \end{bmatrix},
\]
其特征值是\(\L{1} = 3\)和\(\L{2} = 1\).
\(\A^T \A\)属于特征值\(\L{1}\)的特征向量为
\(\vb{v}_1 = \begin{bmatrix} 1/\sqrt{2} \\ 1/\sqrt{2} \end{bmatrix}\);
\(\A^T \A\)属于特征值\(\L{2}\)的特征向量为
\(\vb{v}_2 = \begin{bmatrix} -1/\sqrt{2} \\ 1/\sqrt{2} \end{bmatrix}\).

同时有\[
	\A \A^T = \begin{bmatrix} 1 & 1 & 0 \\ 1 & 2 & 1 \\ 0 & 1 & 1 \end{bmatrix},
\]
其特征值是\(\M{1} = 3\)、\(\M{2} = 1\)、\(\M{3} = 0\).
\(\A \A^T\)属于特征值\(\M{1}\)的特征向量为
\(\vb{u}_1 = \begin{bmatrix} 1/\sqrt{6} \\ 2/\sqrt{6} \\ 1/\sqrt{6} \end{bmatrix}\);
\(\A \A^T\)属于特征值\(\M{2}\)的特征向量为
\(\vb{u}_2 = \begin{bmatrix} 1/\sqrt{2} \\ 0 \\ -1/\sqrt{2} \end{bmatrix}\);
\(\A \A^T\)属于特征值\(\M{3}\)的特征向量为
\(\vb{u}_3 = \begin{bmatrix} 1/\sqrt{3} \\ -1/\sqrt{3} \\ 1/\sqrt{3} \end{bmatrix}\).

再根据\(s_i = \sqrt{\L{i}}\)求得奇异值\(s_1 = \sqrt{3}\)和\(s_2 = 1\).

于是\[
	\U = (\vb{u}_1,\vb{u}_2,\vb{u}_3)
	= \begin{bmatrix}
		1/\sqrt{6} & 1/\sqrt{2} & 1/\sqrt{3} \\
		2/\sqrt{6} & 0 & -1/\sqrt{3} \\
		1/\sqrt{6} & -1/\sqrt{2} & 1/\sqrt{3}
	\end{bmatrix},
\]\[
	\V = (\vb{v}_1,\vb{v}_2,\vb{v}_3)
	= \begin{bmatrix}
		1/\sqrt{2} & -1/\sqrt{2} \\
		1/\sqrt{2} & 1/\sqrt{2}
	\end{bmatrix},
\]\[
	\S = \begin{bmatrix}
		\sqrt{3} & 0 \\
		0 & 1 \\
		0 & 0
	\end{bmatrix}.
\]
\end{solution}
\end{example}

\subsection{极分解}
\begin{theorem}
\def\S{\vb{S}}
\def\M{\vb{\Omega}}
任意实方阵\(\A\)可表为\[
	\A = \S\M = \M_1 \S_1,
\]
其中\(\S\)和\(\S_1\)为半正定实对称方阵,
\(\M\)与\(\M_1\)为实正交方阵,
而且\(\S\)和\(\S_1\)都是唯一的.
\begin{proof}
当\(\A\)可逆时,
\(\A^T \A\)是正定阵,
存在正定阵\(\S_1\),
使得\(\A^T \A = \S_1^2\),
于是\(\A = \A \S_1^{-1} \S_1\),
注意到\((\A \S_1^{-1})^T (\A \S_1^{-1}) = (\S_1^{-1})^T \A^T \A \S_1^{-1} = \E\),
即\(\A \S_1^{-1}\)正交,
那么只需要令\(\M_1 = \A \S_1^{-1}\)即有\(\A = \M_1 \S_1\).

当\(\A\)不可逆时,可以运用正交相似标准型;
也可以运用扰动法,
即令\(\S_1(t) = \S_1 + t\E\),
则当\(t\)充分大时,
\(\S_1(t)\)可逆.
\end{proof}
\end{theorem}


\chapter{二次型}
为了研究几何问题(特别是平面二次曲线、空间二次曲面的方程的化简)和物理问题,
我们抽象出“二次型”的概念,
利用代数方法对其进行研究.
\section{二次型的基本概念}
我们首先研究平面解析几何中以坐标原点为中心的二次曲线的方程:
\begin{center}
\def\arraystretch{1.5}
\begin{tabular}{cl}
圆 & \(x^2+y^2=r^2\) \\
椭圆 & \(\frac{x^2}{a^2}+\frac{y^2}{b^2}=1\) \\
双曲线 & \(\frac{x^2}{a^2}-\frac{y^2}{b^2}=1\) \\
\end{tabular}
\end{center}

可以看出,它们都具有\[
a x^2 + 2b xy + c y^2 = d
\]的形式.
在研究二次曲线时,如果得到的方程不是标准方程,我们通常希望通过旋转、平移等几何变换将其化为标准方程,进而判别曲线的形状和几何性质.

\subsection{二次型的基本概念}
\begin{definition}
称系数\(a_{ij}\ (1 \leq i \leq j \leq n)\)属于数域\(P\)的\(n\)个变量的二次齐次多项式\begin{equation}\label{equation:二次型.二次型}
f(\AutoTuple{x}{n})
= \sum\limits_{i=1}^n \sum\limits_{j=1}^n a_{ij} x_i x_j
\quad(a_{ji}=a_{ij})
\end{equation}为数域\(P\)上的一个\(n\)元\DefineConcept{二次型}(quadratic form)\footnote{%
本章不作特别声明时,“二次型”均指实二次型.}.
\end{definition}

数域对于一个二次齐次多项式是否成为二次型是决定性的.
多项式\[
f(x_1,x_2,x_3) = x_1^2 + 4 x_1 x_2 + 3 x_2^2 + 5 x_2 x_3 - x_3^2
\]和\[
g(x_1,x_2,x_3) = x_1^2 + 2\sqrt{2} x_1 x_2 + 2 x_1 x_3 + 2 x_2^2 + 4\sqrt{3} x_2 x_3
\]都是实数域上的二次型;
但在有理数域上,只有\(f\)是二次型,\(g\)不是二次型.

前面提到我们希望将一般方程化为标准方程,现在我们就要定义何种形式的方程应该被称为标准方程.
再次观察平面二次曲线的标准方程可以发现,标准方程的等号左边应该是二次齐次多项式(即若干个变量的平方和),等号右边则应该是任意(非零)常数.

\begin{definition}
若\(n\)阶对称矩阵\(\A = (a_{ij})_n\)满足
\begin{equation}\label{equation:二次型.二次型的矩阵表示}
f(\AutoTuple{x}{n}) = \x^T\A\x,
\end{equation}其中\(\x = (\AutoTuple{x}{n})^T\),
则称\cref{equation:二次型.二次型的矩阵表示} 为二次型\(f(\AutoTuple{x}{n})\)的\DefineConcept{矩阵表示},
对称矩阵\(\A\)为\(f\)的\DefineConcept{矩阵},
\(\A\)的秩\(\rank\A\)为\(f\)的\DefineConcept{秩}.
\end{definition}

显然,对于任一\(n\)阶矩阵\(\B\),\(\x^T\B\x\)必定是一个二次型.
需要注意的是,矩阵\(\B\)不必是对称矩阵,但“二次型\(\x^T\B\x\)的矩阵”必定是一个对称矩阵.

\begin{property}
二次型和它的矩阵是相互唯一确定的.
\begin{proof}
对于二次型\(f(\AutoTuple{x}{n})\),设非零\(n\)阶对称矩阵\(\A\)和\(\B\)都是\(f\)的矩阵,即\[
\x^T\A\x
=\x^T\B\x
=f(\AutoTuple{x}{n}),
\]则二次型\(\x^T\A\x\)与\(\x^T\B\x\)中\(x_i x_j\)的系数\(2 a_{ij}\)与\(2 b_{ij}\)(\(1 \leq i < j \leq n\))必相等,\(x_i^2\)的系数\(a_{ii}\)与\(b_{ii}\)(\(i=1,2,\dotsc,n\))必相等,故\(\A=\B\).
\end{proof}
\end{property}

\begin{example}
将\(f(x_1,x_2,x_3) = x_1^2 + 4 x_1 x_2 + 3 x_2^2 + 5 x_2 x_3 - x_3^2\)写成矩阵形式.
\begin{solution}
\(f(x_1,x_2,x_3) = \begin{bmatrix}
x_1 & x_2 & x_3
\end{bmatrix} \begin{bmatrix}
1 & 2 & 0 \\
2 & 3 & \frac{5}{2} \\
0 & \frac{5}{2} & -1
\end{bmatrix} \begin{bmatrix}
x_1 \\ x_2 \\ x_3
\end{bmatrix}\).
\end{solution}
\end{example}

\begin{example}
写出二次型\(\begin{bmatrix}
x_1 & x_2 & x_3
\end{bmatrix} \begin{bmatrix}
2 & -3 & 1 \\
1 & 0 & 1 \\
2 & 11 & 3
\end{bmatrix} \begin{bmatrix}
x_1 \\ x_2 \\ x_3
\end{bmatrix}\)的矩阵.
\begin{solution}
注意到矩阵\(\A = \begin{bmatrix}
2 & -3 & 1 \\
1 & 0 & 1 \\
2 & 11 & 3
\end{bmatrix}\)不是对称矩阵,二次型\(\x^T\A\x\)的矩阵应为\[
\B = \frac{\A+\A^T}{2} = \begin{bmatrix}
2 & -1 & \frac{3}{2} \\
-1 & 0 & 6 \\
\frac{3}{2} & 6 & 3
\end{bmatrix}.
\]
\end{solution}
\end{example}

\subsection{线性替换}
\begin{definition}
因为平面二次曲线方程通过旋转变换化为标准方程,
实际上是用新变量的一次式代替原来的变量.
同样地,使用这种基本的方法来化简一般的\(n\)元二次型,
作如下的变量替换:\[
	\left\{ \begin{array}{l}
		x_1 = c_{11}y_1 + c_{12}y_2 + \dotsb + c_{1n}y_n \\
		x_2 = c_{21}y_1 + c_{22}y_2 + \dotsb + c_{2n}y_n \\
		\hdotsfor{1} \\
		x_n = c_{n1}y_1 + c_{n2}y_2 + \dotsb + c_{nn}y_n
	\end{array} \right.
\]
写成矩阵形式\[
	\begin{bmatrix}
		x_1 \\ x_2 \\ \vdots \\ x_n
	\end{bmatrix}
	= \begin{bmatrix}
		c_{11} & c_{12} & \dots & c_{1n} \\
		c_{21} & c_{22} & \dots & c_{2n} \\
		\vdots & \vdots & & \vdots \\
		c_{n1} & c_{n2} & \dots & c_{nn}
	\end{bmatrix}
	\begin{bmatrix}
		y_1 \\ y_2 \\ \vdots \\ y_n
	\end{bmatrix}
	\quad\text{或}\quad
	\x=\C\y.
\]

上述变量之间的替换称为\DefineConcept{线性替换}.

当矩阵\(\C=(c_{ij})_n\)可逆时,
称为\DefineConcept{可逆线性替换},
或\DefineConcept{满秩线性替换},
或\DefineConcept{非退化线性替换}.
\end{definition}

\begin{theorem}
对二次型\(f(\AutoTuple{x}{n})=\x^T\A\x\)(\(\A=\A^T\))作可逆线性替换\(\x=\C\y\),则\(f\)化为新变量的二次型\(g(\AutoTuple{y}{n})=\y^T\B\y\),其中\(\B=\C^T\A\C\)为\(g\)的矩阵.
\begin{proof}
\(f(\AutoTuple{x}{n}) = \x^T\A\x%
\xlongequal{\x=\C\y} (\C\y)^T\A(\C\y)%
= \y^T (\C^T\A\C) \y\),
令\(\B = \C^T\A\C\),由于\(\B^T = (\C^T\A\C)^T = \C^T\A\C = \B\),以及\(\C\)可逆,所以\(\B\)是对称矩阵.
\(f\)是二次型,它的矩阵\(\A\neq\z\),\(\B\cong\A\),故\(\B\neq\z\).
于是,\(g(\AutoTuple{y}{n})=\y^T\B\y\)是二次型,对称矩阵\(\B\)是\(g\)的矩阵.
\end{proof}
\end{theorem}

\subsection{矩阵合同的概念}
\begin{definition}
设\(\A\)和\(\B\)是两个\(n\)阶矩阵,
若存在可逆矩阵\(\C\),
使得\(\B=\C^T\A\C\),
则称“\(\A\)与\(\B\) \DefineConcept{合同}(congruent)”%
或“\(\B\)是\(\A\)的\DefineConcept{合同矩阵}”,
记为\(\A\simeq\B\).
\end{definition}

\begin{property}
矩阵的合同关系是等价关系,即具备下列三条性质:
\begin{enumerate}
\item 反身性,\(\A\simeq\A\);
\item 对称性,\(\A\simeq\B \implies \B\simeq\A\);
\item 传递性,\(\A\simeq\B \land \B\simeq\C \implies \A\simeq\C\).
\end{enumerate}
\end{property}

\begin{property}
合同矩阵与原矩阵等价,即\(\A\simeq\B \implies \A\cong\B\).
\begin{proof}
由\hyperref[definition:逆矩阵.矩阵等价]{矩阵等价的定义}显然有.
\end{proof}
\end{property}

\begin{property}
合同矩阵的秩与原矩阵相等,即\(\A\simeq\B \implies \rank\A=\rank\B\).
\begin{proof}
由\cref{theorem:线性方程组.初等变换不变秩} 立得.
\end{proof}
\end{property}

\begin{property}
对称矩阵的合同矩阵也是对称的,即\(\A^T = \A, \A\simeq\B \implies \B^T = \B\).
\begin{proof}
设可逆矩阵\(\C\)满足\(\C^T\A\C=\B\),那么%
\(\B^T = (\C^T\A\C)^T = \C^T\A^T\C = \C^T\A\C = \B\).
\end{proof}
\end{property}

\begin{property}
矩阵相似是矩阵合同的充分不必要条件,即\[
\A\sim\B
\implies
\A\cong\B.
\]
\end{property}

\begin{example}\label{example:二次型.实对称矩阵相似必合同}
实对称矩阵\(\A\)、\(\B\)相似,证明:\(\A\)与\(\B\)合同.
\begin{proof}
因为\(\A\)、\(\B\)都是实对称矩阵,\(\A^T=\A\),\(\B^T=\B\),且存在正交矩阵\(\Q_1,\Q_2\)使得\[
\Q_1^{-1} \A \Q_1 = \Q_1^T \A \Q_1 = \V_1,
\]\[
\Q_2^{-1} \B \Q_2 = \Q_2^T \B \Q_2 = \V_2,
\]其中\(\V_1,\V_2\)是对角阵.
又因为\(\A\sim\B\),所以\(\A\)与\(\B\)有相同的特征多项式、特征值,即\(\V_1=\V_2\),或\[
\Q_1^{-1} \A \Q_1 = \Q_2^{-1} \B \Q_2,
\]\[
(\Q_2 \Q_1^{-1}) \A (\Q_1 \Q_2^{-1}) = \B.
\]令\(\P = \Q_1 \Q_2^{-1}\),\(\P^T = (\Q_1 \Q_2^{-1})^T = (\Q_2^{-1})^T \Q_1^T = \Q_2 \Q_1^{-1} = (\Q_1 \Q_2^{-1})^{-1} = \P^{-1}\),那么\[
\P^T \A \P = \B,
\]也就是说\(\A\)与\(\B\)合同.
\end{proof}
\end{example}

\begin{example}
尽管在\cref{example:二次型.实对称矩阵相似必合同} 中我们看到“两个相似的实对称矩阵必定合同”,但两个合同的实对称矩阵未必相似,我们立即可以拿出一个反例:
\def\diagx(#1){\begin{bmatrix} #1 \\ & #1 \\ && #1 \end{bmatrix}}%
因为\[
\diagx(3^{-1/2})^T \diagx(3) \diagx(3^{-1/2}) = \diagx(1),
\]即\(\A = \diag(3,3,3) \cong \E\),但\(\tr\A = 9 \neq \tr\E = 3\),也就是说两个矩阵并不相似.
\end{example}

\begin{example}\label{example:二次型.合同矩阵符号相同}
设\(\A\)、\(\B\)、\(\C\)都是可逆矩阵,且满足\(\C^T\A\C=\B\),
则\(\abs{\A}\)与\(\abs{\B}\)的符号相同.
\begin{proof}
因为\(\B=\C^T\A\C\),
所以\(\A\B=\A\C^T\A\C\),
进而有\[
\abs{\A}\abs{\B}
=\abs{\A\B}
=\abs{\A\C^T\A\C}
=\abs{\A}\abs{\C^T}\abs{\A}\abs{\C}
=\abs{\A}^2\abs{\C}^2.
\]
又因为\(\abs{\A} \neq 0\),\(\abs{\A}^2 > 0\),\(\abs{\C} \neq 0\),\(\abs{\C}^2 > 0\),所以\(\abs{\A}\abs{\B} > 0\),即\(\abs{\A}\)与\(\abs{\B}\)同号.
\end{proof}
\end{example}
由\cref{example:二次型.合同矩阵符号相同} 可知,若矩阵\(\A,\B\)合同,则\(\sgn\abs{\A}=\sgn\abs{\B}\).

\begin{example}
设\(\AutoTuple{i}{n}\)是\(1,2,\dotsc,n\)的一个排列,
\[
\A=\diag(\AutoTuple{d}{n}),
\qquad
\B=\diag(d_{i_1},d_{i_2},\dotsc,d_{i_n}).
\]
证明:矩阵\(\A,\B\)合同且相似.
\begin{proof}
显然\(\diag(d_{i_1},d_{i_2},\dotsc,d_{i_n})\)可由\(\diag(\AutoTuple{d}{n})\)同时左乘和右乘若干个初等矩阵\[
\P(i,j) \quad(1 \leq i < j \leq n)
\]得到,又因为\(\P(i,j)^T = \P(i,j)^{-1} = \P(i,j)\),所以只要令这些初等矩阵的乘积为\(\P\),就有\[
\P^{-1} \A \P = \B,
\qquad
\P^T \A \P = \B.
\]也就是说\(\A\simeq\B\land\A\sim\B\).
\end{proof}
\end{example}

根据\cref{theorem:特征值与特征向量.实对称矩阵3},任一实对称矩阵都与某个对角阵合同且相似,实际上,我们可以把这个结论推广到复数域上的对称矩阵.
\begin{theorem}
任一对称矩阵都合同于某个对角矩阵.
\end{theorem}

\section{二次型化为标准型的三种方法}
二次型的基本问题是研究如何通过非退化的线性替换将一个热词性化简为平方和的形式,从而讨论其性质.
\begin{definition}
如果二次型\(f(\AutoTuple{x}{n})=\x^T\A\x\)可以经过非退化的线性替换\(\x=\C\vb{Y}\)化简为\[
d_1 y_1^2 + d_2 y_2^2 + \dotsb + d_n y_n^2
\]的形式,则称上式为二次型\(f\)的\DefineConcept{标准型}或\DefineConcept{规范型}(standard form).
\end{definition}
将二次型化为标准型的问题可以归结为对称矩阵合同于对角阵的问题.
下面介绍三种化二次型为标准型的方法.

\subsection{正交变换法}
根据\cref{theorem:特征值与特征向量.实对称矩阵3},对于任何一个实对称矩阵\(\A\),存在正交矩阵\(\Q\),使得\[
\Q^{-1}\A\Q = \Q^T\A\Q = \diag(\AutoTuple{\lambda}{n}),
\]即实对称矩阵\(\A\)与对角阵\(\diag(\AutoTuple{\lambda}{n})\)合同且相似.

\begin{theorem}
对于任一\(n\)元实二次型\(f(\AutoTuple{x}{n})=\x^T\A\x\ (\A=\A^T)\),都存在正交矩阵\(\Q\),由\(\Q\)构成的线性替换\(\x=\Q\vb{Y}\)(称为\DefineConcept{正交变换},{\rm orthogonal operator})将\(f\)化为标准型\[
f(\AutoTuple{x}{n})
\xlongequal{\x=\Q\vb{Y}}
\lambda_1y_1^2+\lambda_2y_2^2+ \dotsb +\lambda_ny_n^2
\]其中\(\lambda_1,\lambda_2,\dotsc,\lambda_n\)是\(\A\)的全部特征值.
\end{theorem}

\begin{corollary}
正交变换的特点之一是保持向量的内积不变,也就是保持向量的长度不变,保持图形的形状不变.
\begin{proof}
设\(\Q\)为正交矩阵.
令\(\x_1=\Q\vb{Y}_1\),
\(\x_2=\Q\vb{Y}_2\),
必有\begin{align*}
\vectorinnerproduct{\x_1}{\x_2}
&=\vectorinnerproduct{(\Q\vb{Y}_1)}{(\Q\vb{Y}_2)}
=(\Q\vb{Y}_1)^T (\Q\vb{Y}_2)
=\vb{Y}_1^T \Q^T \Q \vb{Y}_2 \\
&=\vb{Y}_1^T \E \vb{Y}_2
=\vb{Y}_1^T \vb{Y}_2
=\vectorinnerproduct{\vb{Y}_1}{\vb{Y}_2},
\end{align*}
又令\(\x_1=\x_2=\x\),
\(\vb{Y}_1=\vb{Y}_2=\vb{Y}\),
则\(\abs{\x}^2=\abs{\vb{Y}}^2\),
\(\abs{\x}=\abs{\vb{Y}}\).
\end{proof}
\end{corollary}

\begingroup
\color{red}
用正交变换法将二次型化为标准型的步骤如下:
\begin{enumerate}
\item 首先根据\(f\)的表达式写出\(f\)的矩阵\(\A\);
\item 写出\(f\)矩阵的特征多项式\(\abs{\lambda\E-\A}=0\),求解\(\A\)的特征值\(\AutoTuple{\lambda}{n}\)以及对应的特征向量\(\AutoTuple{\x}{n}\);
\item 运用施密特规范化方法将\(\A\)的特征向量正交单位化为\(\AutoTuple{\g}{n}\),然后写成正交矩阵\(\Q=(\AutoTuple{\g}{n})\);
\item 计算得出标准型的矩阵\(\B=\Q^T\A\Q=\Q^{-1}\A\Q=\diag(\AutoTuple{\lambda}{n})\).
\end{enumerate}
\endgroup

\begin{example}
设\(f(x_1,x_2,x_3) = -x_1^2-x_2^2-7x_3^2-4x_1x_2+8x_1x_3+8x_2x_3\).
利用正交变换将\(f\)化为标准型,并写出所用的正交变换.
\begin{solution}
\def\z{\vb{Z}}%
\(f\)的矩阵为\[
\A = \begin{bmatrix}
-1 & -2 & 4 \\
-2 & -1 & 4 \\
4 & 4 & -7
\end{bmatrix}.
\]令\[
\abs{\l\E-\A} = \begin{bmatrix}
\l+1 & 2 & -4 \\
2 & \l+1 & -4 \\
-4 & -4 & \l+7
\end{bmatrix} = (\l-1)^2 (\l+11) = 0,
\]解得特征值\(\L1=1\)(二重),\(\L2=-11\).

当\(\l=1\)时,解方程\((\E-\A)\x=\z\),\[
\E-\A = \begin{bmatrix}
2 & 2 & -4 \\
2 & 2 & -4 \\
-4 & -4 & 8
\end{bmatrix} \to \begin{bmatrix}
2 & 2 & -4 \\
0 & 0 & 0 \\
0 & 0 & 0
\end{bmatrix},
\]基础解系为\[
\X1 = \begin{bmatrix} -1 \\ 1 \\ 0 \end{bmatrix}, \qquad
\X2 = \begin{bmatrix} 2 \\ 0 \\ 1 \end{bmatrix}.
\]利用施密特方法将其正交化,得\[
\y_1=\X1,\qquad
\y_2=\X2-\frac{\X2\cdot\y_1}{\y_1\cdot\y_1}\y_1=\begin{bmatrix} 1 \\ 1 \\ 1 \end{bmatrix};
\]再将其单位化,得\[
\z_1=\frac{1}{\sqrt{2}} \begin{bmatrix} -1 \\ 1 \\ 0 \end{bmatrix}, \qquad
\z_2=\frac{1}{\sqrt{3}} \begin{bmatrix} 1 \\ 1 \\ 1 \end{bmatrix}.
\]

当\(\l=-11\)时,解方程\((-11\E-\A)\x=\z\),\[
-11\E-\A = \begin{bmatrix}
-10 & 2 & -4 \\
2 & -10 & -4 \\
-4 & -4 & -4
\end{bmatrix} \to \begin{bmatrix}
1 & 1 & 1 \\
0 & 2 & 1 \\
0 & 0 & 0
\end{bmatrix},
\]基础解系为\[
\X3 = \begin{bmatrix} 1 \\ 1 \\ -2 \end{bmatrix},
\]单位化得\[
\z_3 = \frac{1}{\sqrt{6}} \begin{bmatrix} 1 \\ 1 \\ -2 \end{bmatrix}.
\]

令\[
\Q = (\z_1,\z_2,\z_3) = \begin{bmatrix}
-\frac{1}{\sqrt{2}} & \frac{1}{\sqrt{3}} & \frac{1}{\sqrt{6}} \\
\frac{1}{\sqrt{2}} & \frac{1}{\sqrt{3}} & \frac{1}{\sqrt{6}} \\
0 & \frac{1}{\sqrt{3}} & -\frac{2}{\sqrt{6}} \\
\end{bmatrix},
\]\(\Q\)是正交矩阵,满足\(\Q^T\A\Q=\Q^{-1}\A\Q=\diag(1,1,-11)\).
作正交变换\(\x=\Q\y\),于是\(f\)化为标准型\(y_1^2+y_2^2-11y_3^2\).
\end{solution}
\end{example}

\begin{example}
设实二次型\[
f(\AutoTuple{x}{n})=\x^T\A\x
\]的矩阵\(\A\)的特征值为\(\AutoTuple{\lambda}{n}\),\(c=\max\{\AutoTuple{\lambda}{n}\}\).
证明:对于任意\(n\)维实向量\(\x\),都有\[
f(\AutoTuple{x}{n}) \leq c \x^T\x.
\]
\begin{proof}
因为\(\A\)是实对称矩阵,即存在正交矩阵\(\Q\)使得\(\Q^T\A\Q = \diag(\AutoTuple{\lambda}{n})\),作正交变换\(\x=\Q\y\),则\(f\)化为标准型\begin{align*}
f(\AutoTuple{x}{n}) &= \L{1} y_1^2 + \L{2} y_2^2 + \dotsb + \L{n} y_n^2 \\
&\leq c y_1^2 + c y_2^2 + \dotsb + c y_n^2
= c \y^T \y = c \x^T \x.
\qedhere
\end{align*}
\end{proof}
\end{example}

\subsection{拉格朗日配方法}
用正交变换能够化实二次型为标准型,这种方法是根据实对称矩阵的性质,求出二次型矩阵的特征值和规范正交的特征向量,条件要求较强.
当研究一般数域\(P\)上的二次型(包括实二次型)的标准型时,可以用\DefineConcept{拉格朗日配方法}.
这种方法不用解矩阵特征值问题,只需反复利用以下两个初等代数公式\[
a^2+2ab+b^2=(a+b)^2 ,\quad a^2-b^2=(a+b)(a-b)
\]就能将二次型化为标准型.

\begin{example}
用配方法化二次型\(f(x_1,x_2,x_3)
= x_1^2 + 2 x_1 x_2 + 2 x_2^2 - 3 x_2 x_3\)为标准型,并求出所用的可逆线性替换.
\begin{solution}
首先有
\begin{align*}
f(x_1,x_2,x_3)
&= x_1^2 + 2 x_1 x_2 + 2 x_2^2 - 3 x_2 x_3 \\
&= x_1^2 + 2 x_1 x_2 + x_2^2 + x_2^2 - 3 x_2 x_3 \\
&= (x_1 + x_2)^2 + x_2^2 - 3 x_2 x_3 + \frac{9}{4} x_3^2 - \frac{9}{4} x_3^2 \\
&= (x_1 + x_2)^2 + \left( x_2 - \frac{3}{2} x_3 \right)^2 - \frac{9}{4} x_3^2,
\end{align*}
令\[
\left\{ \def\arraystretch{1.5} \begin{array}{*7r}
y_1 &= &x_1 &+&x_2 \\
y_2 &= & & & x_2 & -& \frac{3}{2} x_3 \\
y_3 &= & & & & & x_3
\end{array} \right.
\eqno(1)
\]则\[
\left\{ \def\arraystretch{1.5} \begin{array}{*7r}
x_1 &= &y_1 &-&y_2 &-&\frac{3}{2} y_3 \\
x_2 &= & & & y_2 & +& \frac{3}{2} y_3 \\
x_3 &= & & & & & y_3
\end{array} \right.
\eqno(2)
\]
(2)是可逆线性替换,使\(f(x_1,x_2,x_3) = y_1^2 + y_2^2 - \frac{9}{4} y_3^2\).
\end{solution}
\end{example}

\begin{theorem}
对于任一\(n\)元二次型\(f(\AutoTuple{x}{n})=\x^T\A\x\ (\A=\A^T)\),都存在非退化的线性替换\(\x=\C\vb{Y}\),使之成为\[
f(\AutoTuple{x}{n})=d_1 y_1^2 + d_2 y_2^2 + \dotsb + d_n y_n^2.
\]
\begin{proof}
应用数学归纳法.
当\(n=1\)时,\(f(x_1) = a_{11} x_1^2 = d_1 y_1^2\),结论成立.

假设当\(n=k-1\ (k\geq2)\)时结论成立;那么当\(n=k\)时,我们分以下两种情况进行讨论:
\begin{enumerate}
\item 若\(f\)的平方项系数不全为零,不妨设\(a_{11}\neq0\),
从而有\begin{align*}
	f(\AutoTuple{x}{n})
	&= a_{11} x_1^2 + 2 x_1 \sum_{j=2}^n a_{1j} x_j
		+ \sum_{i=2}^n \sum_{j=2}^n a_{ij} x_i x_j \\
	&= a_{11} \left[
		x_1 + \frac{1}{a_{11}} \sum_{j=2}^n a_{1j} x_j
	\right]^2
	- \frac{1}{a_{11}} \left[
		\sum_{j=2}^n a_{1j} x_j
	\right]^2
	+ \sum_{i=2}^n \sum_{j=2}^n a_{ij} x_i x_j,
\end{align*}
令\[
	y_1 = x_1 + \frac{1}{a_{11}} \sum_{j=2}^n a_{1j} x_j, \qquad
	y_2 = x_2, \qquad
	\dotsc, \qquad
	y_n = x_n,
\]
则\[
	f(\AutoTuple{x}{n}) = a_{11} y_1^2 + g(\AutoTuple{y}[2]{n}),
\]
其中\(g(\AutoTuple{y}[2]{n})
= - \frac{1}{a_{11}} \left[
	\sum_{j=2}^n a_{1j} x_j
\right]^2
+ \sum_{i=2}^n \sum_{j=2}^n a_{ij} x_i x_j\)%
是一个\(n-1\)元二次型或零多项式.
由归纳假设,存在\(n-1\)阶可逆矩阵\(\Q\),
得到非退化线性变换\((\AutoTuple{y}[2]{n})^T = \Q (\AutoTuple{z}[2]{n})^T\),
使得\[
	g(\AutoTuple{y}[2]{n})
	= d_2 z_2^2 + d_3 z_3^2 + \dotsb + d_n z_n^2.
\]
记\[
\P = \begin{bmatrix}
1 & -a_{11}^{-1} a_{12} & -a_{11}^{-1} a_{13} & \dots & -a_{11}^{-1} a_{1n} \\
& 1 & 0 & \dots & 0 \\
& & 1 & \dots & 0 \\
& & & \ddots & \vdots \\
& & & & 1
\end{bmatrix},
\]则非退化线性替换\[
\begin{bmatrix}
x_1 \\ x_2 \\ \vdots \\ x_n
\end{bmatrix}
= \P \begin{bmatrix}
y_1 \\ y_2 \\ \vdots \\ y_n
\end{bmatrix}
= \P \begin{bmatrix} 1 & \z \\ \z & \Q \end{bmatrix} \begin{bmatrix}
z_1 \\ z_2 \\ \vdots \\ z_n
\end{bmatrix}
\]使得\(f(\AutoTuple{x}{n}) = a_{11} z_1^2 + d_2 z_2^2 + d_3 z_3^2 + \dotsb + d_n z_n^2\).

\item 若\(f\)不含平方项,则必存在\(1 \leq i < j \leq n\)使得\(a_{ij}\neq0\);
于是\[
\left\{ \begin{array}{l}
x_i = y_i + y_j \\
x_j = y_i - y_j \\
x_k = y_k\ (1 \leq k \leq n \land k \neq i \land k \neq j)
\end{array} \right.
\]是非退化的线性替换,使得\[
f(\AutoTuple{x}{n})
= g(\AutoTuple{y}{n})
= a_{ij} y_i^2 - a_{ij} y_j^2 + h(\AutoTuple{y}{n}),
\]其中\(h(\AutoTuple{y}{n})\)是不含平方项的二次型或零多项式,故\(g(\AutoTuple{y}{n})\)含有平方项,这归结为第一种情形,从而可以化为标准型.
\qedhere
\end{enumerate}
\end{proof}
\end{theorem}

\begin{corollary}
任意\(n\)阶对称矩阵\(\A\)都与对角形矩阵合同.
\end{corollary}

\subsection{初等变换法}
\begin{definition}
设\(\P\)为初等矩阵.
对于任意矩阵\(\B\),称变换\(\B \to \P^T\B\P\)为对\(\B\)作一次\DefineConcept{合同变换}.
\end{definition}

\begin{theorem}
因为对于任一对称阵\(\A\),存在可逆矩阵\(\C\),使得\[
\C^T\A\C=\diag(\AutoTuple{d}{n}).
\]
又因为存在初等矩阵\(\P_1,\P_2,\dotsc,\P_m\),使得\[
\C=\P_1\P_2\dotsm\P_m,
\]那么\[
\C^T\A\C = (\P_m^T\P_{m-1}^T\dotsm\P_1^T)\A(\P_1\P_2\dotsm\P_m)=\diag(\AutoTuple{d}{n}).
\]

因为初等矩阵有三类:
\begin{enumerate}
\item \(\P(i,j)^T\B\P(i,j)=\P(i,j)\B\P(i,j)\),相当于交换\(\B\)的\(i,j\)两行,再交换\(\P(i,j)\B\)的\(i,j\)两列.
\item \(\P(i(c))^T\B\P(i(c))=\P(i(c))\B\P(i(c))\ (c \neq 0)\),相当于用\(c\)乘\(\B\)的\(i\)行,再用\(c\)乘\(\P(i(c))\B\)的\(i\)列.
\item \(\P(i,j(k))^T\B\P(i,j(k))=\P(j,i(k))\B\P(i,j(k))\),相当于将\(\B\)的\(i\)行的\(k\)倍加到\(j\)行,再将\(\P(j,i(k))\B\)的\(i\)列的\(k\)倍加到\(j\)列.
\end{enumerate}
可见,对称矩阵\(\A\)可以经过一系列合同变换化为对角形矩阵.
则\[
\begin{bmatrix} \A \\ \E \end{bmatrix}
\to
\begin{bmatrix} \C^T & \z \\ \z & \E \end{bmatrix}
\begin{bmatrix} \A \\ \E \end{bmatrix}
\C = \begin{bmatrix} \C^T\A\C \\ \C \end{bmatrix},
\]\[
(\A,\E)
\to
\C^T (\A,\E) \begin{bmatrix}
\C & \z \\
\z & \E
\end{bmatrix}
= (\C^T\A\C,\C^T).
\]其中\(\C=\P_1\P_2\dotsm\P_m\),
即对\(\A\)作一系列合同变换化为对角阵\(\C^T\A\C\),只对\(\E\)进行列变换,将\(\E\)变成\(\C\);或者只对\(\E\)作其中的行变换,则将\(\E\)变为\(\C^T\).
\end{theorem}

\section{实二次型的分类}
\subsection{实二次型的分类标准}
\begin{definition}\label{definition:实二次型的分类.实二次型的分类}
设有\(n\)元实二次型\(f(\x) = \x^T\A\x\).
\begin{enumerate}
	\item 如果\[
		(\forall\x\in\mathbb{R}^n-\{\z\})
		[f(\x) > 0],
	\]
	则称“\(f\)是\DefineConcept{正定的}(positive definite)”;
	把\(\A\)称为\DefineConcept{正定矩阵}(positive definite matrix),
	记为\(\A\succ\z\).

	\item 如果\[
		(\forall\x\in\mathbb{R}^n-\{\z\})
		[f(\x) \geq 0],
	\]
	则称“\(f\)是\DefineConcept{半正定的}(positive semi-definite)”;
	把\(\A\)称为\DefineConcept{半正定矩阵}(positive semi-definite matrix),
	记为\(\A\succeq\z\).

	\item 如果\[
		(\forall\x\in\mathbb{R}^n-\{\z\})
		[f(\x) < 0],
	\]
	则称“\(f\)是\DefineConcept{负定的}(negative definite)”,
	把\(\A\)称为\DefineConcept{负定矩阵}(negative definite matrix),
	记为\(\A\prec\z\).

	\item 如果\[
		(\forall\x\in\mathbb{R}^n-\{\z\})
		[f(\x) \leq 0],
	\]
	则称“\(f\)是\DefineConcept{半负定的}(negative semi-definite)”;
	把\(\A\)称为\DefineConcept{半负定矩阵}(negative semi-definite matrix),
	记为\(\A\preceq\z\).

	\item 否则,称\(f\)是\DefineConcept{不定的}(indefinite).
\end{enumerate}
\end{definition}

\subsection{惯性定理}
\begin{theorem}\label{theorem:二次型.惯性定理}
\(n\)元实二次型\(f(\x) = \x^T\A\x\)经过任意满秩线性变换化为标准型,
所得的标准型的正平方项的项数\(p\)及负平方项的项数\(q\)都是唯一确定的.
\begin{proof}
\def\z{\vb{z}}%
设实二次型的秩为\(r\).
假设\(f\)经过两个不同的可逆线性替换\(\x=\C\y,\x=\D\z\)分别化为标准型\[
	f \xlongequal{\x=\C\y}
	c_1 y_1^2 + c_2 y_2^2 + \dotsb + c_p y_p^2 - c_{p+1} y_{p+1}^2 - \dotsb - c_r y_r^2,
	\eqno{(1)}
\]\[
	f \xlongequal{\x=\D\z}
	d_1 z_1^2 + d_2 z_2^2 + \dotsb + d_q z_q^2 - d_{q+1} z_{q+1}^2 - \dotsb - d_r z_r^2,
	\eqno{(2)}
\]
其中\(c_i,d_i>0\ (i=1,2,\dotsc,r)\).

用反证法.
设\(p > q\),由\(\x = \C\y = \D\z\),\(\D\)可逆,得\(\z = \D^{-1} \C \y\).
\def\H{\vb{H}}%
\def\zexpr#1{h_{#1 1} y_1 + h_{#1 2} y_2 + \dotsb + h_{#1 n} y_n}%
记\(\H = (h_{ij})_n = \B^{-1} \C\),
则\(\z = \H\y\),
即\[
	z_i = \zexpr{i}
	\quad(i=1,2,\dotsc,n).
\]
于是\[
	\begin{aligned}
		&\hspace{-40pt}
		c_1 y_1^2 + c_2 y_2^2 + \dotsb
			+ c_p y_p^2 - c_{p+1} y_{p+1}^2 - \dotsb - c_r y_r^2 \\
		&\hspace{-20pt}= d_1 (\zexpr{1})^2 + d_2 (\zexpr{2})^2 \\
		&+ \dotsb + d_q (\zexpr{q})^2 \\
		&- d_{q+1} (\zexpr{q+1})^2 - \dotsb \\
		&- d_r (\zexpr{r})^2.
	\end{aligned}
	\eqno{(3)}
\]
由此可以构造齐次线性方程组\[
	\begin{cases}
		\zexpr{1} = 0, \\
		\hdotsfor{1} \\
		\zexpr{q} = 0, \\
		y_{p+1} = 0, \\
		\hdotsfor{1} \\
		y_n = 0.
	\end{cases}
	\eqno{(4)}
\]
这个方程组中有\(n\)个未知量,\(q+n-p < n\)个方程,
于是它有非零解\((\AutoTuple{y}{p},0,\dotsc,0)^T\),
代入(3)式两端,得到左边大于零,右边小于等于零,矛盾,因此\(p \leq q\).
同理又有\(q \leq p\),于是\(p = q\).
\end{proof}
\end{theorem}
\cref{theorem:二次型.惯性定理}
又称为“惯性定理(Inertial Theorem)”.

\begin{corollary}
任意\(n\)元实二次型\(f(\x) = \x^T\A\x\),
总可经过满秩线性变换,化为以下形式:\[
	y_1^2+y_2^2+ \dotsb +y_p^2
	-y_{p+1}^2-\dotsb-y_r^2.
\]
我们将其称为“\(f(\x)\)的\DefineConcept{规范型}({\rm normal form})”,
且规范型是唯一的.
\begin{proof}
\def\z{\vb{z}}%
根据惯性定理,\(f\)经过可逆线性替换化为标准型:\[
	f \xlongequal{\x=\D\z}
	d_1 z_1^2 + d_2 z_2^2 + \dotsb + d_q z_q^2 - d_{q+1} z_{q+1}^2 - \dotsb - d_r z_r^2,
\]
其中\(d_i>0\ (i=1,2,\dotsc,r)\).
令\[
	\C = \D \diag(d_1^{-1/2},\dotsc,d_r^{-1/2},1,\dotsc,1),
\]
则\(\x = \D\z = \C\y\)是可逆线性替换,
使得\[
	f \xlongequal{\x=\D\z} y_1^2 + y_2^2 + \dotsb + y_q^2 - y_{q+1}^2 - \dotsb - y_r^2.
\]
二次型的规范型的唯一性可以由惯性定理得到.
\end{proof}
\end{corollary}

\begin{definition}\label{definition:二次型.惯性系数的定义}
在秩为\(r\)的实二次型\(f(\x)\)所化成的标准型(或规范型)中,
\begin{enumerate}
	\item 正平方项的项数\(p\)
	称为“\(f\)的\DefineConcept{正惯性指数}(positive index of inertia)”;
	\item 负平方项的项数\(q=r-p\)
	称为“\(f\)的\DefineConcept{负惯性指数}(minus index of inertia)”;
	\item 正、负惯性指数之差\(d=p-q\)
	称为“\(f\)的\DefineConcept{符号差}(signature)”.
\end{enumerate}
\end{definition}
由\hyperref[theorem:二次型.惯性定理]{惯性定理}%
和\cref{definition:二次型.惯性系数的定义} 可知:
可逆线性替换不改变二次型的正、负惯性指数.
因此我们可以根据二次型的正、负惯性指数确定二次型的类型.

\begin{theorem}
设\(\A\)和\(\B\)是同阶实对称矩阵.
这两个矩阵合同的充要条件是两者的秩、正负惯性指数均相等,
即\[
	\A\simeq\B
	\iff
	\rank\A=\rank\B \land p_{\A}=p_{\B} \land q_{\A}=q_{\B}.
\]
\end{theorem}

\subsection{正定矩阵的等价条件}
\begin{theorem}
设\(\A\)为\(n\)阶实对称矩阵,
\(f(\AutoTuple{x}{n}) = \x^T\A\x\),
则下列命题相互等价:
\begin{enumerate}
	\item \(\A\)为正定矩阵;
	\item \(\A\)的特征值全是正实数;
	\item \(f(\AutoTuple{x}{n})\)的正惯性指数\(p=n\);
	\item \(\A \cong \E\)(即,存在可逆实阵\(\C\),使得\(\C^T\A\C=\E\));
	\item 存在可逆实阵\(\P\),使得\(\A=\P^T\P\).
\end{enumerate}
\end{theorem}

\begin{corollary}
正定矩阵的行列式大于零.
\begin{proof}
因为\(\A\)正定,
所以存在可逆实阵\(\P\),使得\(\A=\P^T\P\),
则\[
	\abs{\A}
	=\abs{\P^T\P}
	=\abs{\P^T} \abs{\P\vphantom{\P^T}}
	=\abs{\P}^2>0.
	\qedhere
\]
\end{proof}
\end{corollary}

\begin{theorem}
\(n\)元实二次型\(f(\AutoTuple{x}{n}) = \x^T\A\x\ (\A=\A^T)\)正定的充要条件是:
\(\A\)的各阶顺序主子式均大于零.
\begin{proof}
必要性.
对于任意不全为零的\(n\)个实数\(c_1,c_2,\dotsc,c_k,0,\dotsc,0\),
总有\[
	f(c_1,c_2,\dotsc,c_k,0,\dotsc,0)
	= \sum_{i=1}^k \sum_{j=1}^k c_i c_j a_{ij} > 0,
\]
从而\(k\)元实二次型\(f_k(x_1,x_2,\dotsc,x_k)
=\sum_{i=1}^k
\sum_{j=1}^k
x_i x_j a_{ij}\)正定,
而\(f_k\)的矩阵为\(\A_k = (a_{ij})_k\),
那么\(\abs{\A_k} > 0\ (k=1,2,\dotsc,n)\).

充分性.当\(n=1\)时,\(a_{11} > 0\),\(f_1(x_1) = a_{11} x_1^2\)正定.
设\(n=k-1\)时结论成立,
当\(n=k\)时,
将\(\A\)分块得\(\A = \begin{bmatrix}
	\A_{k-1} & \a \\
	\a^T & a_{nn}
\end{bmatrix}\),
其中\(\A_{k-1}\)为各阶顺序主子式都大于零的\(k-1\)阶实对称矩阵.
由归纳假设,\(\A_{k-1}\)正定,故存在\(k-1\)阶可逆矩阵\(\Q\),
使得\(\A_{k-1} = \Q^T \Q\),\(\A_{k-1}\)可逆,
\(\A_{k-1}^{-1} = \Q^{-1}(\Q^{-1})^T\)是对称矩阵.
令\(\P = \begin{bmatrix}
	\Q^{-1} & -\A_{k-1}^{-1} \a \\
	\z & 1
\end{bmatrix}\),
则\(\P\)可逆,
于是\begin{align*}
	\P^T \A \P &= \begin{bmatrix}
		(\Q^{-1})^T & \z \\
		-\a^T \A_{k-1}^{-1} & 1
	\end{bmatrix}
	\begin{bmatrix}
		\Q^T \Q & \a \\
		\a^T & a_{nn}
	\end{bmatrix}
	\begin{bmatrix}
		\Q^{-1} & -\A_{k-1}^{-1} \a \\
		\z & 1
	\end{bmatrix} \\
	&= \begin{bmatrix}
		\Q & (\Q^{-1})^T \a \\
		\z & b
	\end{bmatrix}
	\begin{bmatrix}
		\Q^{-1} & -\A_{k-1}^{-1} \a \\
		\z & 1
	\end{bmatrix}
	= \begin{bmatrix}
		\E_{k-1} & \z \\
		\z & b
	\end{bmatrix} = \B,
\end{align*}
其中\(b=a_{nn}-\a^T \A_{k-1}^{-1} \a\).
由于\(\A\)与\(\B\)合同,\(\abs{\A} > 0\),
得\(\abs{\B} = b > 0\),
作可逆线性替换\(\x = \P\y\),则\[
	f \xlongequal{\x=\Q\y} y_1^2 + y_2^2 + \dotsb + y_{n-1}^2 + b y_n^2,
\]
故\(f\)的正惯性指数为\(n\),\(f\)正定.
\end{proof}
\end{theorem}

\begin{corollary}
\(n\)元实二次型\(f(\AutoTuple{x}{n}) = \x^T\A\x\ (\A=\A^T)\)负定的充要条件是:\[
	(-1)^k D_k
	= (-1)^k \begin{vmatrix}
		a_{11} & a_{12} & \dots & a_{1k} \\
		a_{21} & a_{22} & \dots & a_{2k} \\
		\vdots & \vdots & & \vdots \\
		a_{k1} & a_{k2} & \dots & a_{kk}
	\end{vmatrix} > 0,
	\quad k=1,2,\dotsc,n.
\]
\end{corollary}

\begin{proposition}
设\(\A \in M_{s \times n}(\mathbb{R})\),
则\[
	\text{\(\A^T\A\)正定}
	\iff
	\rank\A = n.
\]
\begin{proof}
显然矩阵\(\A^T\A\)是\(n\)阶实对称矩阵,
于是\begin{align*}
	&\text{矩阵\(\A^T\A\)正定} \\
	&\iff (\forall\x\in\mathbb{R}^n-\{\vb0\})[\x^T (\A^T \A) \x > 0]
		\tag{\hyperref[definition:实二次型的分类.实二次型的分类]{正定矩阵的定义}} \\
	&\iff (\forall\x\in\mathbb{R}^n-\{\vb0\})[\norm{\A\x} > 0] \\
	&\iff (\forall\x\in\mathbb{R}^n-\{\vb0\})[\A\x\neq\vb0] \\
	&\iff \text{\(\A\x=\vb0\)只有零解} \\
	&\iff \rank\A = n,
		\tag{\cref{theorem:向量空间.有解的非齐次线性方程组的解的个数定理}}
\end{align*}
也就是说,“矩阵\(\A^T\A\)正定”的充分必要条件是“\(\A\)是列满秩矩阵”.
\end{proof}
\end{proposition}

\begin{example}
设\(\A\)为实对称矩阵,证明:当实数\(t\)充分大时,\(t\E+\A\)是正定矩阵.
\begin{proof}
因为\(\A\)为实对称矩阵,所以存在正交矩阵\(\Q\),使得\[
\Q^T\A\Q = \diag(\AutoTuple{\lambda}{n}).
\]又因为\[
\Q^T(t\E+\A)\Q
= \diag(t+\L{1},t+\L{2},\dotsc,t+\L{n}),
\]所以当\(t+\L{1},t+\L{2},\dotsc,t+\L{n}\)都大于零时,\(t\E+\A\)正定.
\end{proof}
\end{example}

\begin{example}
设\(\A\)、\(\B\)是同阶正定矩阵,证明:\(\A+\B\)、\(\A^{-1}\)、\(\A^*\)是正定矩阵.
\begin{proof}
根据正定矩阵的定义,因为\(\A\)是正定矩阵,任取非零列向量\(\x\),都有\[
\x^T\A\x > 0;
\]同样地,有\(\x^T\B\x > 0\).

又根据矩阵的乘法分配律,有\[
\x^T(\A+\B)\x = \x^T\A\x + \x^T\B\x > 0
\]成立,即\(\A+\B\)是正定矩阵.

因为\(\A\)是正定矩阵,存在可逆实阵\(\P\)使得\(\P^T\A\P=\E\),所以\[
\A = (\P^T)^{-1}\E\P^{-1} = (\P^T)^{-1}\P^{-1}
\implies
\A^{-1} = \P\P^T,
\]说明\(\A^{-1}\)是正定矩阵.

由逆矩阵的定义,\(\A^{-1}=\frac{1}{\abs{\A}}\A^*\),那么\(\A^*=\abs{\A}\A^{-1}\),\(\abs{\A}>0\),显然\(\A^*\)也是正定矩阵.
\end{proof}
\end{example}

\begin{example}
设\(\A\)是正定矩阵,试证:存在正定矩阵\(\B\),使得\(\A=\B^2\).
\begin{proof}
设\(\A\)是\(n\)阶正定矩阵,那么存在正交矩阵\(\P\)满足\[
\P^T\A\P = \diag(\AutoTuple{\lambda}{n}) = \V,
\]其中\(\AutoTuple{\lambda}{n} \in \mathbb{R}^+\).
又设矩阵\(\B\)满足\(\B^2=\A\),那么\[
\P^T\A\P = \P^T\B^2\P = \V
\iff
\B^2 = \P\V\P^T,
\]
只需要令\(\B = \P \diag(\sqrt{\L{1}},\sqrt{\L{2}},\dotsc,\sqrt{\L{n}}) \P^T\)即可.
\end{proof}
\end{example}

\section{二次型的应用}
下面利用矩阵的运算及二次型理论讨论平面二次曲线、空间二次曲面的分类问题.


\chapter{线性空间与线性变换}
\section{线性空间}
\subsection{线性空间的概念与性质}
\begin{definition}
设\(V\)是一个非空集合,\(P\)是一个数域,定义两种运算:
\begin{enumerate}
\item {\bf 加法}:\(\forall \a,\b \in V\),\(V\)中有唯一的元素\(\vb{\delta}\)与之对应,称为\(\a\)与\(\b\)之和,记作\(\a+\b\);
\item {\bf 数乘}:\(\forall k \in P\),\(\forall \a \in V\),\(V\)中有唯一的元素与之对应,记作\(k\a\).
\end{enumerate}

如果\emph{加法}与\emph{数乘}满足以下八条公理,即\(\forall \a,\b,\g \in V\),\(\forall k,l \in P\),有

\begin{center}
\begin{minipage}{.8\textwidth}
\begin{axiom}
\(\a+\b=\b+\a\).
\end{axiom}
\begin{axiom}
\((\a+\b)+\g=\a+(\b+\g)\).
\end{axiom}
\begin{axiom}
\(\exists \z \in V (\forall\a \in V \implies \a+\z=\a)\),\(\z\)称为\DefineConcept{零元素}.
\end{axiom}
\begin{axiom}
\(\forall\a \in V, \exists \vb{\eta} \in V (\a+\vb{\eta}=\z)\),\(\vb{\eta}\)称为\(\a\)的\DefineConcept{负元素},记作\(\vb{\eta} = -\a\).
\end{axiom}
\begin{axiom}
\(1\a=\a\).
\end{axiom}
\begin{axiom}
\(k(l\a)=(kl)\a\).
\end{axiom}
\begin{axiom}
\(k(\a+\b)=k\a+k\b\).
\end{axiom}
\end{minipage}
\end{center}

则称\(V\)为数域\(P\)上的\DefineConcept{线性空间}(linear space).
\(V\)中的元素称为\DefineConcept{向量}(vector)\nolinebreak.

特别地,当\(P = \mathbb{R}\)时,称\(V\)为\DefineConcept{实线性空间};
当\(P = \mathbb{C}\)时,称\(V\)为\DefineConcept{复线性空间}.
\end{definition}

\begin{example}
下面列举一些常见的线性空间.
实线性空间与复线性空间是代数结构完全不同的两个线性空间.
集合\(\mathbb{R}^{n \times 1}\)关于向量的加法、实数与向量的数乘构成实线性空间.
集合\(\mathbb{R}^{s \times n}\)关于矩阵的加法、实数与矩阵的数乘构成实线性空间.
全体实系数多项式关于多项式的加法、实数与多项式的乘法(数乘)构成实线性空间.
\end{example}

\begin{example}
设\(\A \in P^{s \times n}\),
齐次线性方程组\(\A\x=\z\)的解集\(W\)关于向量加法及数乘构成数域\(P\)上的线性空间,
所以\(W\)又称为\(\A\x=\z\)的\DefineConcept{解空间}.
特别地,当\(\rank\A=n\)时,\(\A\x=\z\)只有零解,
解空间\(W\)只有零向量.
只含零向量的空间称为\DefineConcept{零空间},记为\(\{\z\}\).
\end{example}

\begin{definition}
设\(V\)是数域\(P\)上的线性空间,\(W \subseteq V\)是一个非空集合.
如果\(W\)关于\(V\)中的加法及数乘运算也构成数域\(P\)上的线性空间,则称\(W\)是\(V\)的一个\DefineConcept{子空间}(subspace).
\end{definition}

\begin{theorem}\label{theorem:线性空间.子空间的判定}
设\(W\)是线性空间\(V\)的非空子集.
如果\(W\)关于\(V\)的加法与数乘运算封闭,即\[
\forall \a,\b \in W,
\forall k \in P
\bigl(
\a+\b, k \a \in W
\bigr),
\]则称\(W\)是\(V\)的子空间.
\end{theorem}

\begin{example}
设\(V\)是数域\(P\)上的线性空间.
在线性空间\(V\)中取定\(s\)个向量\[
\AutoTuple{\a}{s}
\]组成向量组\(A\).证明:集合\[
W = \Set{ k_1 \a_1 + k_2 \a_2 + \dotsb + k_s \a_s \given k_i \in P, i=1,2,\dotsc,s }
\]是\(V\)的子空间.
\begin{proof}
首先\(W\)是\(V\)的非空子集.其次\(\forall \a,\b \in W\)有\[
\a = k_1 \a_1 + k_2 \a_2 + \dotsb + k_s \a_s,
\qquad
\b = p_1 \a_1 + p_2 \a_2 + \dotsb + p_s \a_s,
\]故\[
\a+\b = (k_1+p_1)\a_1 + (k_2+p_2)\a_2 + \dotsb + (k_s+p_s)\a_s \in W;
\]同理可证\(\forall \a \in W, \forall k \in P\)有\(k\a \in W\).
由\cref{theorem:线性空间.子空间的判定},\(W\)是\(V\)的子空间.
\end{proof}
集合\(W\)称为\(A\)的\DefineConcept{生成空间}(spanning space),记作\(L(\AutoTuple{\a}{s})\).
\end{example}

\subsection{线性空间的基与维数}
\begin{definition}
\def\B{\mathcal{B}}%
设\(V\)是数域\(P\)上的线性空间,如果\begin{enumerate}
\item \(\e_1,\e_2,\dotsc,\e_n \in V\);
\item 向量组\(\B = \{ \e_1,\e_2,\dotsc,\e_n \}\)线性无关;
\item 在\(V\)中任取一个向量\(\a\),\(\a\)总可由向量组\(\B\)线性表出,即\[
\a = k_1 \e_1 + k_2 \e_2 + \dotsb + k_n \e_n,
\]
\end{enumerate}
则称\(\B\)是\(V\)的一个\DefineConcept{基}(basis).
称系数\(\AutoTuple{k}{n}\)为\(\a\)在基\(\B\)下的\DefineConcept{坐标}(coordinate).
称整数\(n\)为\(V\)的\DefineConcept{维数},记作\(\dim V = n\).
\end{definition}

\begin{definition}
\def\B{\mathcal{B}}%
\def\Ba{\B_{\alpha}}%
\def\Bb{\B_{\beta}}%
设\[
\Ba = \{ \AutoTuple{\a}{n} \}
\quad\text{和}\quad
\Bb = \{ \AutoTuple{\b}{n} \}
\]是\(V^n\)的两组基.

显然,对基\(\Bb\)中的每个向量\(\b_1\),可以求出其在基\(\Ba\)下的坐标:\[
\b_i = \Ba \P_i \quad(i=1,2,\dotsc,n),
\]其中\(\P_i = (p_{i1},p_{i2},\dotsc,p_{in})^T \in F^n\ (i=1,2,\dotsc,n)\).

若矩阵\(\P = (p_{ij})_n = (\P_1,\P_2,\dotsc,\P_n)\)满足\[
(\AutoTuple{\b}{n}) = (\AutoTuple{\a}{n}) \P,
\]则称矩阵\(\P\)是基\(\Ba\)到基\(\Bb\)的\DefineConcept{过渡矩阵}(或\DefineConcept{变换矩阵}).
\end{definition}

\begin{example}
设\(\a_1,\a_2,\a_3\)是\(\mathbb{R}^3\)的一组基,求:基\(\a_1,\frac{1}{2}\a_2,\frac{1}{3}\a_3\)到基\(\a_1+\a_2,\a_2+\a_3,\a_3+\a_1\)的过渡矩阵.
\begin{solution}
设所求过渡矩阵为\(\P\),则根据定义有\[
\begin{bmatrix}
\a_1 & \frac{1}{2}\a_2 & \frac{1}{3}\a_3
\end{bmatrix} \P
= \begin{bmatrix}
\a_1+\a_2 & \a_2+\a_3 & \a_3+\a_1
\end{bmatrix},
\]即\[
\begin{bmatrix}
\a_1 & \a_2 & \a_3
\end{bmatrix} \begin{bmatrix}
1 \\
& \frac{1}{2} \\
&& \frac{1}{3}
\end{bmatrix} \P
= \begin{bmatrix}
\a_1 & \a_2 & \a_3
\end{bmatrix} \begin{bmatrix}
1 & 0 & 1 \\
1 & 1 & 0 \\
0 & 1 & 1
\end{bmatrix},
\]所以\[
\P = \begin{bmatrix}
1 \\
& \frac{1}{2} \\
&& \frac{1}{3}
\end{bmatrix}^{-1} \begin{bmatrix}
1 & 0 & 1 \\
1 & 1 & 0 \\
0 & 1 & 1
\end{bmatrix}
= \begin{bmatrix}
1 \\
& 2 \\
&& 3
\end{bmatrix} \begin{bmatrix}
1 & 0 & 1 \\
1 & 1 & 0 \\
0 & 1 & 1
\end{bmatrix}
= \begin{bmatrix}
1 & 0 & 1 \\
2 & 2 & 0 \\
0 & 3 & 3
\end{bmatrix}.
\]
\end{solution}
\end{example}


\chapter{多项式环}
\section{多项式}
\begin{definition}\label{definition:多项式.多项式的定义}
设\(K\)是一个数域,
\(x\)是一个不属于\(K\)的符号.
任意给定一个非负整数\(n\),
在\(K\)中任意取定\(\AutoTuple{a}[0]{n}\),
如果表达式\begin{equation}\label[polynomial]{equation:多项式.多项式}
	a_n x^n + a_{n-1} x^{n-1} + \dotsb + a_1 x + a_0
\end{equation}
满足\begin{enumerate}
	\item 两个这种形式的表达式相等当且仅当它们除了系数为零的项以外含有完全相同的项,
	即\begin{align*}
		&a_n x^n + a_{n-1} x^{n-1} + \dotsb + a_1 x + a_0
		= b_n y^n + b_{n-1} y^{n-1} + \dotsb + b_1 y + b_0 \\
		&\iff
		(\forall i\in\Set{0,1,\dotsc,n})[a_i = b_i = 0 \lor a_i x^i = b_i y^i].
	\end{align*}
	\item 允许从表达式中删去系数为零的项,也允许向表达式中添进系数为零的项,
\end{enumerate}
那么称之为“数域\(K\)上的一个一元\DefineConcept{多项式}(polynomial)”,
把\(x\)称为\DefineConcept{不定元}.
\end{definition}

系数全为零的多项式称为\DefineConcept{零多项式}.
在\cref{equation:多项式.多项式} 中,
把\(a_i x^i\)称为“\(i\)次\DefineConcept{项}”,
把\(a_0\)称为\DefineConcept{零次项}或\DefineConcept{常数项}.

从\cref{definition:多项式.多项式的定义} 知道,
数域\(K\)上两个一元多项式相等当且仅当它们的同次项的系数都相等.

设\(f(x)\)表示\cref{equation:多项式.多项式}.
如果\(a_n\neq0\),
则称“\(a_n x^n\)是多项式\(f(x)\)的\DefineConcept{首项}”;
并称“\(f(x)\)的\DefineConcept{次数}是\(n\)”,
记作\(\deg f\).

零多项式的次数定义为\(-\infty\),并规定:\begin{gather*}
	(-\infty)+(-\infty)=-\infty, \\
	(-\infty)+n=-\infty, \\
	-\infty<n,
\end{gather*}
其中\(n\)是任意非负整数.

零次多项式总是一个常数\(a\),它满足\(a \in K \land a \neq 0\).

我们把数域\(K\)上的所有一元多项式组成的集合记作\(K[x]\).
我们可以在\(K[x]\)中规定“加法”与“乘法”运算.
设\[
	f(x) = \sum\limits_{i=0}^n a_i x^i, \qquad
	g(x) = \sum\limits_{i=0}^n b_i x^i,
\]
如果\(n \ge m\),那么\begin{gather}
	f(x) + g(x) \defeq \sum\limits_{i=0}^n (a_i+b_i) x^i, \\
	f(x) \cdot g(x) \defeq \sum\limits_{s=0}^{n+m} \left( \sum\limits_{i+j=s} a_i b_j \right) x^s,
\end{gather}
我们把\(f(x)+g(x)\)称为“\(f(x)\)与\(g(x)\)的\DefineConcept{和}”,
把\(f(x) \cdot g(x)\)称为“\(f(x)\)与\(g(x)\)的\DefineConcept{积}”.

容易验证上面所定义的多项式的加法与乘法满足下列运算法则:
\begin{enumerate}
	\item 加法交换律,即\[
		(\forall f,g \in K[x])[f+g=g+f].
	\]

	\item 加法结合律,即\[
		(\forall f,g,h \in K[x])[(f+g)+h=f+(g+h)].
	\]

	\item 零多项式\(0\)是加法单位元,即\[
		(\forall f \in K[x])[0+f=f+0=f].
	\]

	\item \(K[x]\)具有负元.

	设\(f(x)=\sum\limits_{i=0}^n a_i x^i\),
	定义\(-f(x)=\sum\limits_{i=0}^n (-a_i) x^i\),则\[
		f+(-f)=0.
	\]
	称\(-f\)为\(f\)的\DefineConcept{负元}.

	\item 乘法交换律,即\[
		(\forall f,g \in K[x])[fg=gf].
	\]

	\item 乘法结合律,即\[
		(\forall f,g,h \in K[x])[(fg)h=f(gh)].
	\]

	\item 零次多项式\(1\)是乘法单位元,即\[
		1f=f1=f.
	\]

	\item 乘法对加法的分配律,即\[
		(\forall f,g,h \in K[x])[f(g+h)=fg+fh],
	\]\[
		(\forall f,g,h \in K[x])[(g+h)f=gf+hf].
	\]

	\item 乘法消去律,即\[
		fg=fh \land f\neq0 \implies g=h.
	\]
\end{enumerate}

多项式的减法定义如下:\begin{equation}
	f-g \defeq f+(-g).
\end{equation}

\begin{proposition}
%@see: 《高等代数(第三版 下册)》(丘维声) P3. 命题1
设\(f,g \in K[x]\),则\begin{gather}
	\deg(f \pm g) \leq \max\{\deg f, \deg g\}, \\
	\deg(fg) = \deg f + \deg g.
\end{gather}
\end{proposition}

\(K[x]\)中所有零次多项式添上零多项式组成的集合\(S\),
对于多项式的减法与乘法封闭,
因此\(S\)是\(K[x]\)的一个子环.
显然\(K[x]\)的单位元\(1\)属于\(S\),
从而\(1\)也是\(S\)的单位元.
我们可以建立数域\(K\)到\(S\)的一个映射\(\sigma\):
让非零数\(a\)对应于零次多项式\(a\),
让数\(0\)对应到零次多项式,
可以证明\(\sigma\)是一个同构映射.

给定\(\A\in M_n(K)\),
形如\[
	a_m \A^m + a_{m-1} \A^{m-1} + \dotsb + a_1 \A + a_0 \E
\]的表达式称为“数域\(K\)上矩阵\(\A\)的多项式”,
其中\(m\)是非负整数,
\(\E\)是\(n\)阶单位矩阵,
\(a_i \in K\ (i=0,1,\dotsc,m)\).
把数域\(K\)上矩阵\(\A\)的所有多项式组成的集合记作\(K[\A]\),即\[
	K[\A]
	\defeq
	\Set{
		a_m \A^m + a_{m-1} \A^{m-1} + \dotsb + a_1 \A + a_0 \E
		\given
		m \in \mathbb{N}
		\land
		a_i \in K\ (i=0,1,\dotsc,m)
	}.
\]

设\(f(\A)=\sum\limits_{i=0}^m a_i \A^i\),
\(g(\A)=\sum\limits_{i=0}^n b_i \A^i\),
\(m \geq n\).
从矩阵的运算法则可以得到
\begin{gather}
	f(\A)-g(\A)
	= \sum\limits_{i=0}^m (a_i - b_i) \A^i, \\
	f(\A) g(\A)
	= \sum\limits_{s=0}^{m+n} \left( \sum\limits_{i+j=s} a_i b_j \right) \A^s.
	\label{equation:多项式.矩阵多项式.乘法}
\end{gather}
因此\(K[\A]\)是\(M_n(K)\)的一个子环.
从\cref{equation:多项式.矩阵多项式.乘法} 容易看出\[
	f(\A) g(\A) = g(\A) f(\A);
\]
又由于\(\E \in K[\A]\),
所以\(K[\A]\)是有单位元的交换环.

\(K[\A]\)中所有数量矩阵组成的集合\(W\),
对于矩阵的减法与乘法封闭,
因此\(W\)是\(K[\A]\)的一个子环.
显然\(\E \in W\).
我们可以建立数域\(K\)到\(W\)的一个映射\(\tau\colon a \mapsto a \E\).
显然\(\tau\)是同构映射.

\section{整除性,带余除法}
从一元多项式环的通用性质看到,
我们应当尽可能多地得到\(K[x]\)中有关加法和乘法的等式,
为此需要研究一元多项式环\(K[x]\)的结构.
从本节开始我们将主要研究\(K[x]\)的结构,其中\(K\)是任一数域.

观察\(K[x]\)中两个多项式\(f(x)\)与\(g(x)\)之间有什么关系:\[
	f(x)=x^2-1, \qquad
	g(x)=x-1.
\]
显然,\[
	f(x)=(x+1) g(x).
\]
由此我们抽象出“整除”的概念.

\begin{definition}
%@see: 《高等代数(第三版 下册)》(丘维声) P10. 定义1
设\(f,g \in K[x]\).
如果存在\(h \in K[x]\),使得\[
	f(x) = h(x) g(x),
\]
则称“\(g(x)\) \DefineConcept{整除} \(f(x)\)”,
记作\(g(x) \vert f(x)\),
又称“\(g(x)\)是\(f(x)\)的\DefineConcept{因式}”
“\(f(x)\)是\(g(x)\)的\DefineConcept{倍式}”;
否则称“\(g(x)\)不能整除\(f(x)\)”.
\end{definition}

容易看出下列事实:
\begin{enumerate}
	\item \(0 \vert f(x) \iff f(x) = 0\).
	\item \((\forall f \in K[x])[f(x) \vert 0]\).
	\item \((\forall b \in K - \{0\})(\forall f \in K[x])[b \vert f(x)]\).
\end{enumerate}

\begin{definition}
%@see: 《高等代数(第三版 下册)》(丘维声) P10. 定义2
在\(K[x]\)中,如果\(f(x) \vert g(x)\)且\(g(x) \vert f(x)\),
则称“\(f(x)\)与\(g(x)\) \DefineConcept{相伴}”,
记作\(f(x) \sim g(x)\).
\end{definition}

\begin{proposition}
%@see: 《高等代数(第三版 下册)》(丘维声) P10. 命题1
在\(K[x]\)中,\(f(x) \sim g(x)\)当且仅当存在\(c \in K-\{0\}\),使得\[
	f(x) = c g(x).
\]
\end{proposition}

\begin{proposition}
%@see: 《高等代数(第三版 下册)》(丘维声) P10. 命题2
在\(K[x]\)中,如果\(g(x) \vert f_i(x)\ (i=1,2,\dotsc,s)\),
则对于任意\(u_i \in K[x]\ (i=1,2,\dotsc,s)\),有\[
	g(x) \vert (u_1(x) f_1(x) + u_2(x) f_2(x) + \dotsb + u_s(x) f_s(x)).
\]
\end{proposition}

\begin{theorem}
%@see: 《高等代数(第三版 下册)》(丘维声) P11. 定理3
对于\(K[x]\)中任意两个多项式\(f(x)\)与\(g(x)\),其中\(g(x)\neq0\),
则在\(K[x]\)中存在唯一的一对多项式\(h(x),r(x)\),使得\[
	f(x)=h(x) g(x) + r(x)
	\land
	\deg r(x) < \deg g(x).
\]
\end{theorem}

\section{最大公因式}
\subsection{最大公因式}
从上一节知道,数域\(K\)上的一元多项式环\(K[x]\)具有带余除法,这是\(K[x]\)的一个重要性质.
这一节我们要由此出发推导出\(K[x]\)的另一个重要性质:
\(K[x]\)中任何两个多项式都有最大公因式,
并且\(f(x)\)与\(g(x)\)的最大公因式可以表成\(f(x)\)与\(g(x)\)的倍式和.

\begin{definition}
%@see: 《高等代数(第三版 下册)》(丘维声) P15
在\(K[x]\)中,如果\(c(x)\)既是\(f(x)\)的因式,又是\(g(x)\)的因式,
则称“\(c(x)\)是\(f(x)\)与\(g(x)\)的一个\DefineConcept{公因式}”.
\end{definition}

\begin{definition}
%@see: 《高等代数(第三版 下册)》(丘维声) P15 定义1
设\(d(x)\)是\(K[x]\)中多项式\(f(x)\)与\(g(x)\)的一个公因式.
如果对于\(f(x)\)与\(g(x)\)的任一公因式\(c(x)\),都有\(c(x) \mid d(x)\),
则称“\(d(x)\)是\(f(x)\)与\(g(x)\)的一个\DefineConcept{最大公因式}”.
\end{definition}

对于任意多项式\(f(x)\),由于\(f(x) \mid f(x)\)且\(f(x) \mid 0\),
所以\(f(x)\)是\(f(x)\)与\(0\)的一个公因式.
又由于\(f(x)\)与\(0\)的任一公因式\(c(x)\)总可整除\(f(x)\),
因此\(f(x)\)是\(f(x)\)与\(0\)的一个最大公因式.
特别地,\(0\)是\(0\)与\(0\)的最大公因式.

现在我们想要知道,对于\(K[x]\)中任意两个多项式,是否存在它们的最大公因式?
如果存在,我们又该如何找出它们的最大公因式?
对于给定的两个多项式\(f(x)\)与\(g(x)\),它们的最大公因式是否唯一?
这些就是本节要讨论的问题.

我们先指出几个简单而有用的结论.
\begin{proposition}\label{theorem:多项式.最大公因式.命题1}
%@see: 《高等代数(第三版 下册)》(丘维声) P15 命题1
设\(f,g,p,q \in K[x]\).
如果\[
	\Set{ h(x) \given \text{\(h(x)\)是\(f(x)\)与\(g(x)\)的公因式} }
	= \Set{ r(x) \given \text{\(r(x)\)是\(p(x)\)与\(q(x)\)的公因式} },
\]
那么\[
	\Set{ h(x) \given \text{\(h(x)\)是\(f(x)\)与\(g(x)\)的最大公因式} }
	= \Set{ r(x) \given \text{\(r(x)\)是\(p(x)\)与\(q(x)\)的最大公因式} }.
\]
\begin{proof}
设\(d(x)\)是\(f(x)\)与\(g(x)\)的一个最大公因式,
则\(d(x)\)是\(p(x)\)与\(q(x)\)的一个公因式.
任取\(p(x)\)与\(q(x)\)的一个公因式\(\phi(x)\),
则\(\phi(x)\)也是\(f(x)\)与\(g(x)\)的一个公因式,
从而\(\phi(x) \mid d(x)\).
所以\(d(x)\)是\(p(x)\)与\(q(x)\)的一个最大公因式.
同理,\(p(x)\)与\(q(x)\)的任一最大公因式也是\(f(x)\)与\(g(x)\)的最大公因式.
\end{proof}
\end{proposition}

\begin{corollary}\label{theorem:多项式.最大公因式.推论2}
%@see: 《高等代数(第三版 下册)》(丘维声) P15 推论2
设\(f,g \in K[x]\),\(a,b \in K-\{0\}\),
则\[
	\Set{ h(x) \given \text{\(h(x)\)是\(f(x)\)与\(g(x)\)的最大公因式} }
	= \Set{ h(x) \given \text{\(h(x)\)是\(a f(x)\)与\(b g(x)\)的最大公因式} }.
\]
\begin{proof}
显然\(f(x)\)与\(g(x)\)的任一公因式是\(a f(x)\)与\(b g(x)\)的公因式.
对于\(a f(x)\)与\(b g(x)\)的任一公因式\(c(x)\),
有\(c(x) \mid a f(x)\).
又由于\(a\neq0\),
因此\(a f(x) \mid f(x)\),
从而\(c(x) \mid f(x)\).
同理\(c(x) \mid g(x)\).
因此\(c(x)\)也是\(f(x)\)与\(g(x)\)的公因式.
于是由\cref{theorem:多项式.最大公因式.命题1} 立即得出结论.
\end{proof}
\end{corollary}

\begin{lemma}\label{theorem:多项式.最大公因式.引理1}
%@see: 《高等代数(第三版 下册)》(丘维声) P15 引理1
在\(K[x]\)中,如果多项式\(f,g,h,r\)满足\[
	f(x) = h(x) g(x) + r(x)
\]成立,
则\[
	\Set{ u(x) \given \text{\(u(x)\)是\(f(x)\)与\(g(x)\)的最大公因式} }
	= \Set{ u(x) \given \text{\(u(x)\)是\(g(x)\)与\(r(x)\)的最大公因式} }.
\]
\begin{proof}
设\(d(x)\)是\(f(x)\)与\(g(x)\)的一个公因式,
则\(d(x) \mid f(x)\)且\(d(x) \mid g(x)\).
因为由\(f(x) = h(x) g(x) + r(x)\)得\[
	r(x) = f(x) - h(x) g(x),
\]
所以\(d(x) \mid r(x)\),
也就是说\(d(x)\)是\(g(x)\)与\(r(x)\)的一个公因式.
现在任取\(g(x)\)与\(r(x)\)的一个公因式\(c(x)\),
由\(f(x) = h(x) g(x) + r(x)\)得\(c(x) \mid f(x)\),
也就是说\(c(x)\)是\(f(x)\)与\(g(x)\)的一个公因式.
由\cref{theorem:多项式.最大公因式.命题1} 立即得出所要求的结论.
\end{proof}
\end{lemma}

\begin{theorem}\label{theorem:多项式.辗转相除法}
%@see: 《高等代数(第三版 下册)》(丘维声) P16 定理3
对于\(K[x]\)中任意两个多项式\(f(x)\)与\(g(x)\),
存在它们的一个最大公因式\(d(x)\),
并且\(d(x)\)可以表示成\(f(x)\)与\(g(x)\)的倍式和,即存在\(u,v \in K[x]\),使得\[
	d(x) = u(x) f(x) + v(x) g(x).
\]
\begin{proof}
假设\(g(x)=0\),
则\(f(x)\)就是\(f(x)\)与\(g(x)\)的一个最大公因式,
并且\[
	f(x) = 1 \cdot f(x) + 1 \cdot 0.
\]

现在设\(g(x)\neq0\).
根据带余除法,
存在\(h_1(x),r_1(x) \in K[x]\),
使得\[
	f(x) = h_1(x) g(x) + r_1(x), \qquad
	\deg r_1(x) < \deg g(x).
\]
如果\(r_1(x)\neq0\),
则用\(r_1(x)\)去除\(g(x)\),
存在\(h_2(x),r_2(x) \in K[x]\),
使得\[
	g(x) = h_2(x) r_1(x) + r_2(x), \qquad
	\deg r_2(x) < \deg r_1(x).
\]
又如果\(r_2\neq0\),
则用\(r_2(x)\)去除\(r_1(x)\),
存在\(h_3(x),r_3(x) \in K[x]\),
使得\[
	r_1(x) = h_3(x) r_2(x) + r_3(x), \qquad
	\deg r_3(x) < \deg r_2(x).
\]
如此辗转相除下去,
显然,所得余式的次数不断降低,
因此在有限次之后,必然有余式为零,
即\[\begin{array}{ll}
	r_2(x) = h_4(x) r_3(x) + r_4(x), \qquad
		&\deg r_4(x) < \deg r_3(x), \\
	\hdotsfor{2}, \\
	r_{i-2}(x) = h_i(x) r_{i-1}(x) + r_i(x), \qquad
		&\deg r_i(x) < \deg r_{i-1}(x), \\
	\hdotsfor{2}, \\
	r_{s-3}(x) = h_{s-1}(x) r_{s-2}(x) + r_{s-1}(x), \qquad
		&\deg r_{s-1}(x) < \deg r_{s-2}(x), \\
	r_{s-2}(x) = h_s(x) r_{s-1}(x) + r_s(x), \qquad
		&\deg r_s(x) < \deg r_{s-1}(x), \\
	r_{s-1}(x) = h_{s+1}(x) r_s(x) + 0,
\end{array}\]
其中\(h_i(x),r_i(x) \in K[x]\).
由于\(r_s(x)\)是\(r_s(x)\)与\(0\)的一个最大公因式,
因此根据\cref{theorem:多项式.最大公因式.引理1},
从上述等式的最后一个式子得出:
\(r_s(x)\)是\(r_{s-1}(x)\)与\(r_s(x)\)的一个最大公因式.
于是\(r_s(x)\)是\(r_{s-2}(x)\)与\(r_{s-1}(x)\)的一个最大公因式,
从而\(r_s(x)\)是\(r_{s-3}(x)\)与\(r_{s-2}(x)\)的一个最大公因式,
依次递推,
\(r_s(x)\)是\(f(x)\)与\(g(x)\)的一个最大公因式.
这就证明了:
在对\(f(x)\)与\(g(x)\)作辗转相除时,
最后一个不等于零的余式是\(f(x)\)与\(g(x)\)的一个最大公因式.
对上述等式中倒数第二个式子得\[
	r_s(x) = r_{s-2}(x) - h_s(x) r_{s-1}(x),
\]
再由倒数第三个式子得\[
	r_{s-1}(x) = r_{s-3}(x) - h_{s-1}(x) r_{s-2}(x),
\]
合并以上两式得\[
	r_s(x) = [1 + h_s(x) h_{s-1}(x)] r_{s-2}(x) - h_s(x) r_{s-3}(x).
\]
同理用更上面的等式逐个地消去\(r_{s-2}(x),r_{s-3}(x),\dotsc,r_1(x)\),
可得\[
	r_s(x) = u(x) f(x) + v(x) g(x),
\]
其中\(u(x),v(x) \in K[x]\).
\end{proof}
\end{theorem}

\cref{theorem:多项式.辗转相除法} 给出了求两个多项式的最大公因式的方法 --- “辗转相除法”.

我们想要知道,
任意给定\(K[x]\)中的两个多项式\(f(x)\)与\(g(x)\),
它们的最大公因式是否唯一?
设\(d_1(x),d_2(x)\)都是\(f(x)\)与\(g(x)\)的最大公因式,
根据定义得\(d_1(x) \mid d_2(x)\)且\(d_2(x) \mid d_1(x)\).
因此\(d_1(x)\)与\(d_2(x)\)相伴,即\(d_1(x)\)与\(d_2(x)\)仅相差一个非零数因子.
这说明:两个多项式的最大公因式在相伴的意义下是唯一确定的.
容易看出,两个不全为零的多项式的最大公因式一定是非零多项式,
在这个情形,我们约定,用\[
	(f(x), g(x))
\]表示首项系数是\(1\)的那个最大公因式.

应该注意到,
在\cref{theorem:多项式.辗转相除法} 的证明过程中,
我们证明了\(r_s(x)\)是\(f(x)\)与\(g(x)\)的一个最大公因式,
并且有\(r_s(x) = u(x) f(x) + v(x) g(x)\).
对于\(f(x)\)与\(g(x)\)的任一最大公因式\(d(x)\),
由于\(d(x)\)与\(r_s(x)\)相伴,
因此\(d(x) = c r_s(x)\),
其中\(c\)是\(K\)中某个非零数.
于是有\(d(x) = c u(x) f(x) + c v(x) g(x)\).
这表明\(d(x)\)也可以表示成\(f(x)\)与\(g(x)\)的倍式和.

由\cref{theorem:多项式.最大公因式.推论2} 得出,
当\(f(x),g(x)\)不全为零时,
对于\(a,b \in K-\{0\}\),
有\[
	(f(x),g(x))
	= (a f(x),b g(x)).
\]

\begin{example}
设\(f(x)=x^3+x^2-7x+2,
g(x)=3x^2-5x-2\),
求\((f(x),g(x))\),
并且把它表示成\(f(x)\)与\(g(x)\)的倍式和.
\begin{solution}
根据上面的结论,在作辗转相除时,
可以用适当的非零数去乘被除式或者除式,简化计算.
\[
	\def\arraystretch{1.5}
	\begin{array}{r|*3r|*4r|l}
		3x+1 & 3x^2 & -5x & -2 & 3x^3 & +3x^2 & -21x & +6 & x+\frac83 \\
		& 3x^2 & -6x && 3x^3 & -5x^2 & -2x & \\ \cline{2-8}
		&& x & -2 && 8x^2 & -19x & +6 \\
		&& x & -2 && 8x^2 & -\frac{40}3x & -\frac{16}3 \\ \cline{2-8}
		&&& 0 &&& -\frac{17}3x & +\frac{34}3 \\
		&&& &&& x & -2 \\
	\end{array}
\]
因为最后一个不等于零的余式是\(r_1(x) = -\frac{17}3x + \frac{34}3\),
所以\[
	(f(x),g(x)) = x-2.
\]
把上述辗转相除过程写出来就是\begin{align*}
	3 f(x) = \left(x+\frac83\right) g(x) + r_1(x), \\
	g(x) = (3x+1) \left[-\frac3{17} r_1(x)\right] + 0.
\end{align*}
于是\begin{align*}
	(f(x),g(x))
	&= -\frac3{17} r_1(x) \\
	&= -\frac3{17} \left[3 f(x) - \left(x+\frac83\right) g(x)\right] \\
	&= -\frac9{17} f(x) + \frac1{17} (3x+8) g(x).
\end{align*}
\end{solution}
\end{example}

\subsection{互素}
现在我们来研究两个多项式的最大公因式是零次多项式的情形.

\begin{definition}\label{definition:多项式.互素}
%@see: 《高等代数(第三版 下册)》(丘维声) P18 定义2
设\(f(x),g(x)\)是\(K[x]\)中的两个多项式.
如果\((f(x),g(x))=1\),
则称“\(f(x)\)与\(g(x)\) \DefineConcept{互素}”.
\end{definition}

从\cref{definition:多项式.互素} 立即得出,
两个多项式互素当且仅当它们的公因式都是零次多项式,
这是因为它们的任一公因式\(c(x) \mid 1\),
所以\(\deg c(x) = 0\).

下面我们给出两个多项式互素的一个充分必要条件.
\begin{theorem}\label{theorem:多项式.两个多项式互素的充分必要条件}
%@see: 《高等代数(第三版 下册)》(丘维声) P18 定理4
设\(f,g \in K[x]\).
\(f(x)\)与\(g(x)\)互素的充分必要条件是:
存在\(u,v \in K[x]\),使得\[
	u(x) f(x) + v(x) g(x) = 1.
\]
\begin{proof}
必要性.
由\cref{theorem:多项式.辗转相除法} 立即可得.

充分性.
假设\(u(x) f(x) + v(x) g(x) = 1\)成立.
因为\((f(x),g(x)) \mid f(x)\)且\((f(x),g(x)) \mid g(x)\),
所以\((f(x),g(x)) \mid 1\),
于是\((f(x),g(x)) = 1\).
\end{proof}
\end{theorem}

利用\cref{theorem:多项式.两个多项式互素的充分必要条件} 可以证明关于互素的多项式的一些重要性质.

\begin{property}
%@see: 《高等代数(第三版 下册)》(丘维声) P19 性质1
在\(K[x]\)中,如果\[
	f(x) \mid g(x) h(x)
	\quad\text{且}\quad
	(f(x),g(x))=1,
\]
则\[
	f(x) \mid h(x).
\]
\begin{proof}
当\(h(x)=0\)时,
有\(f(x) \mid h(x)\).

当\(h(x)\neq0\)时,
因为\((f(x),g(x))=1\),
所以,存在\(u(x),v(x) \in K[x]\),
使得\[
	u(x) f(x) + v(x) g(x) = 1.
\]
等式两边同乘\(h(x)\),
得\[
	u(x) f(x) h(x) + v(x) g(x) h(x) = h(x).
\]
因为\(f(x) \mid g(x) h(x)\),
所以用\(f(x)\)整除上式左端,
就有\(f(x) \mid h(x)\).
\end{proof}
\end{property}

\begin{property}
%@see: 《高等代数(第三版 下册)》(丘维声) P19 性质2
在\(K[x]\)中,如果\[
	f(x) \mid h(x)
	\quad\text{且}\quad
	g(x) \mid h(x)
	\quad\text{且}\quad
	(f(x),g(x))=1,
\]
则\[
	f(x) g(x) \mid h(x).
\]
\begin{proof}
因为\(f(x) \mid h(x)\),
所以存在\(p(x) \in K[x]\),
使得\(h(x) = p(x) f(x)\).
因为\(g(x) \mid h(x)\),
所以\(g(x) \mid p(x) f(x)\).
因为\((g(x),f(x))=1\),
所以\(g(x) \mid p(x)\).
因此存在\(q(x) \in K[x]\),
使得\(p(x) = q(x) g(x)\).
于是\(h(x) = q(x) g(x) f(x)\),
那么\(f(x) g(x) \mid h(x)\).
\end{proof}
\end{property}

\begin{property}\label{theorem:多项式.互素.性质3}
%@see: 《高等代数(第三版 下册)》(丘维声) P19 性质3
在\(K[x]\)中,如果\[
	(f(x),h(x))=1
	\quad\text{且}\quad
	(g(x),h(x))=1,
\]
则\[
	(f(x) g(x),h(x))=1.
\]
\begin{proof}
因为\((f(x),g(x))=1\),
\((g(x),h(x))=1\),
所以存在\(u_1(x),u_2(x),v_1(x),v_2(x) \in K[x]\),
使得\begin{gather*}
	u_1(x) f(x) + v_1(x) h(x) = 1, \\
	u_2(x) g(x) + v_2(x) h(x) = 1.
\end{gather*}
将上面两个等式相乘,
得\[
	u_1(x) u_2(x) f(x) g(x)
	+ [
		u_1(x) f(x) v_2(x)
		+ v_1(x) u_2(x) g(x)
		+ v_1(x) v_2(x) h(x)
	] h(x)
	= 1.
\]
根据\cref{theorem:多项式.两个多项式互素的充分必要条件}
得\((f(x) g(x),h(x))=1\).
\end{proof}
\end{property}

\subsection{最大公因式、互素的概念推广}
最大公因式和互素的概念可以推广到\(n>2\)个多项式的情形.
\begin{definition}
%@see: 《高等代数(第三版 下册)》(丘维声) P20 定义3
在\(K[x]\)中,
如果多项式\(c(x)\)能整除多项式\(f_i(x)\ (i=1,2,\dotsc,n)\)的每一个,
那么把\(c(x)\)称为这\(n\)个多项式的一个\DefineConcept{公因式}.
\end{definition}

\begin{definition}
%@see: 《高等代数(第三版 下册)》(丘维声) P20 定义3
在\(K[x]\)中,
设多项式\(d(x)\)是\(f_i(x)\ (i=1,2,\dotsc,n)\)的一个公因式.
如果\(f_i(x)\ (i=1,2,\dotsc,n)\)的每一个公因式都能整除\(d(x)\),
那么把\(d(x)\)称为这\(n\)个多项式的一个\DefineConcept{最大公因式}.
\end{definition}

用数学归纳法可以证明,
在\(K[x]\)中,
任意\(n\geq2\)个多项式
\(f_1(x),\dotsc,f_n(x)\)的最大公因式存在,
并且如果\(d_1(x)\)是\(f_1(x),\dotsc,f_{n-1}(x)\)的一个最大公因式,
则\(d_1(x)\)与\(f_n(x)\)的最大公因式就是\(f_1(x),\dotsc,f_{n-1}(x),f_n(x)\)的最大公因式.
因此我们依然可以逐次使用辗转相除法求出\(n\)个多项式的一个最大公因式.

从定义可知,
\(n\)个多项式\(f_1(x),\dotsc,f_n(x)\)的最大公因式在相伴的意义下是唯一的.
对于\(n\)个不全为零的多项式\(f_1(x),\dotsc,f_n(x)\),
我们约定使用\[
	(f_1(x),\dotsc,f_n(x))
\]表示首项系数是\(1\)的那个最大公因式.
于是我们断言\[
%@see: 《高等代数(第三版 下册)》(丘维声) P20 公式(6)
	(f_1(x),\dotsc,f_n(x))
	= ((f_1(x),\dotsc,f_{n-1}(x)),f_n(x)).
\]
从上式出发,根据\cref{theorem:多项式.辗转相除法},
存在\(u_1(x),\dotsc,u_n(x) \in K[x]\),
使得\[
%@see: 《高等代数(第三版 下册)》(丘维声) P20 公式(7)
	u_1(x) f_1(x) + \dotsb + u_n(x) f_n(x)
	= (f_1(x),\dotsc,f_n(x)).
\]

\begin{definition}
%@see: 《高等代数(第三版 下册)》(丘维声) P20 定义4
如果\(K[x]\)中\(n\geq2\)个多项式\(f_1(x),\dotsc,f_n(x)\)满足\[
	(f_1(x),\dotsc,f_n(x)) = 1,
\]
那么称“\(f_1(x),\dotsc,f_n(x)\)~\DefineConcept{互素}”.
\end{definition}

与\cref{theorem:多项式.两个多项式互素的充分必要条件} 一样,
我们可以证明:
在\(K[x]\)中,
\(n\)个多项式\(f_1(x),\dotsc,f_n(x)\)
互素的充分必要条件是
存在\(K[x]\)中多项式\(u_1(x),\dotsc,u_n(x)\)
使得\[
%@see: 《高等代数(第三版 下册)》(丘维声) P20 公式(8)
	u_1(x) f_1(x) + \dotsb + u_n(x) f_n(x) = 1.
\]
但要注意点,\(n>2\)个多项式互素时,
它们不一定两两互素.
例如,多项式\[
	f_1(x) = x+1, \qquad
	f_2(x) = x^2+3x+2, \qquad
	f_3(x) = x-1
\]满足\[
	(f_1(x),f_2(x))=x+1, \qquad
	(f_1(x),f_2(x),f_3(x))=1,
\]
也就是说\(f_1(x),f_2(x),f_3(x)\)互素,
但是\(f_1(x),f_2(x)\)不互素.

我们还要指出一点,
设\(K\)与\(F\)都是数域,
并且\(K \subseteq F\).
设\(f(x),g(x) \in K[x]\),
则我们也可以把\(f(x)\)与\(g(x)\)看成是\(F[x]\)中的多项式.
注意\(f(x)\)与\(g(x)\)在\(K[x]\)中的公因式
和它们在\(F[x]\)中的公因式不一定相同.
例如,设\[
	f(x) = x^2+1, \qquad
	g(x) = x^3+x^2+x+1,
\]
则\(f(x)\)与\(g(x)\)在\(\mathbb{R}[x]\)中没有一次公因式,
但是它们在\(\mathbb{C}[x]\)中有一次公因式\(x+\iu\)与\(x-\iu\).
容易看出它们在\(\mathbb{R}[x]\)中的最大公因式是\(x^2+1\),
在\(\mathbb{C}[x]\)中的最大公因式也是\(x^2+1\).
一般地,我们有如下结论.

\begin{proposition}
%@see: 《高等代数(第三版 下册)》(丘维声) P20 命题5
设\(F,K\)都是数域,且\(F \supseteq K\),
则对于\(K[x]\)中任意两个多项式\(f(x)\)与\(g(x)\),
它们在\(K[x]\)中的首项系数为\(1\)的最大公因式
与它们在\(F[x]\)中的首项系数为\(1\)的最大公因式相同.
也就是说,当数域扩大时,\(f(x)\)与\(g(x)\)的首项系数为\(1\)的最大公因式不改变.
\begin{proof}
若\(f(x)=g(x)=0\),
则\(f(x)\)与\(g(x)\)在\(K[x]\)中的最大公因式是零多项式,
在\(F[x]\)中的最大公因式也是零多项式.
下面设\(f(x)\)与\(g(x)\)不全为零.
设\(d_1(x)\)是\(f(x)\)与\(g(x)\)在\(K[x]\)中的首项系数为\(1\)的最大公因式,
设\(d_2(x)\)是\(f(x)\)与\(g(x)\)在\(F[x]\)中的首项系数为\(1\)的最大公因式.
在\(K[x]\)中对\(f(x)\)与\(g(x)\)作辗转相除法,
设最后一个不等于零的余式是\(r_s(x)\),
其首项系数为\(c\),
则\(d_1(x) = \frac1c r_s(x)\);
由于每一步带余除法也可看成是在\(F[x]\)中进行的(根据带余除法的唯一性),
因此\(r_s(x)\)也是\(f(x)\)与\(g(x)\)在\(F[x]\)中的一个最大公因式,
从而\[
	d_2(x) = \frac1c r_s(x)
	= d_1(x).
	\qedhere
\]
\end{proof}
\end{proposition}

\begin{corollary}
%@see: 《高等代数(第三版 下册)》(丘维声) P21 推论6
设\(F,K\)都是数域,且\(F \supseteq K\),
\(f,g \in K[x]\),
则\(f(x)\)与\(g(x)\)在\(K[x]\)中互素的充分必要条件是:
\(f(x)\)在\(g(x)\)在\(F[x]\)中互素.
也就是说,互素性不随数域的扩大而改变.
\begin{proof}
容易看出\begin{align*}
	&\text{$f(x)$与$g(x)$在$K[x]$中互素} \\
	&\iff \text{在$K[x]$中,$(f(x),g(x))=1$} \\
	&\iff \text{在$F[x]$中,$(f(x),g(x))=1$} \\
	&\iff \text{$f(x)$与$g(x)$在$F[x]$中互素}.
	\qedhere
\end{align*}
\end{proof}
\end{corollary}

\begin{example}
%@see: 《高等代数(第三版 下册)》(丘维声) P21 习题7.3 2.
证明:在\(K[x]\)中,
如果\(d(x)\)既是\(f(x)\)与\(g(x)\)的倍式和,
又是\(f(x)\)与\(g(x)\)的一个公因式,
则\(d(x)\)是\(f(x)\)与\(g(x)\)的一个最大公因式.
\begin{proof}
设\(c(x)\)是\(f(x)\)与\(g(x)\)的一个公因式,
则\(c(x) \mid f(x)\)且\(c(x) \mid g(x)\).
又设\[
	d(x) = u(x) f(x) + v(x) g(x),
\]
其中\(u(x),v(x) \in K[x]\).
那么由\cref{theorem:多项式.整除的线性性}
可知\(c(x) \mid d(x)\).
由定义可知\(d(x)\)是\(f(x)\)与\(g(x)\)的一个最大公因式.
\end{proof}
\end{example}

\begin{example}
%@see: 《高等代数(第三版 下册)》(丘维声) P21 习题7.3 4.
证明:在\(K[x]\)中,
如果\(f(x),g(x)\)不全为零,
则\[
	\left(
		\frac{f(x)}{(f(x),g(x))},
		\frac{g(x)}{(f(x),g(x))}
	\right)=1.
\]
\begin{proof}
设\(f(x) = u(x) (f(x),g(x)),
g(x) = v(x) (f(x),g(x))\).
由\cref{theorem:多项式.辗转相除法} 可知\[
	(f(x),g(x)) = p(x) f(x) + q(x) g(x),
\]
其中\(p(x),q(x) \in K[x]\).
于是\begin{align*}
	(f(x),g(x))
	&= p(x) u(x) (f(x),g(x)) + q(x) v(x) (f(x),g(x)) \\
	&= [p(x) u(x) + q(x) v(x)] (f(x),g(x)),
\end{align*}
消去\((f(x),g(x))\)得\[
	p(x) u(x) + q(x) v(x) = 1.
\]
由\cref{theorem:多项式.两个多项式互素的充分必要条件} 可知
\(u(x)\)与\(v(x)\)互素,
所以\[
	\left(
		\frac{f(x)}{(f(x),g(x))},
		\frac{g(x)}{(f(x),g(x))}
	\right)
	= (u(x),v(x))
	= 1.
	\qedhere
\]
\end{proof}
\end{example}

\begin{example}
%@see: 《高等代数(第三版 下册)》(丘维声) P21 习题7.3 5.
证明:在\(K[x]\)中,
如果\(f(x),g(x)\)不全为零,
并且\[
	u(x) f(x) + v(x) g(x) = (f(x),g(x)),
\]
则\((u(x),v(x))=1\).
\begin{proof}
设\(f(x) = p(x) (f(x),g(x)),
g(x) = q(x) (f(x),g(x))\),
其中\(p(x),q(x) \in K[x]\).
那么\begin{align*}
	u(x) f(x) + v(x) g(x)
	&= u(x) p(x) (f(x),g(x))
	+ v(x) q(x) (f(x),g(x)) \\
	&= [u(x) p(x) + v(x) q(x)] (f(x),g(x)).
\end{align*}
根据题设有\(u(x) p(x) + v(x) q(x) = 1\),
于是\((u(x),v(x)) = 1\).
\end{proof}
\end{example}

\begin{example}
%@see: 《高等代数(第三版 下册)》(丘维声) P22 习题7.3 6.
证明:在\(K[x]\)中,
如果\((f,g)=1\),
那么\((fg,f+g)=1\).
%TODO proof
\begin{proof}
% 首先证\((f,f+g)=1\).
设\((f,g)=1\),
由\cref{theorem:多项式.两个多项式互素的充分必要条件}
可知\(uf+vg=1\),
其中\(u,v \in K[x]\).
于是\[
	(u-v)f+v(f+g)=1;
\]
再次利用\cref{theorem:多项式.两个多项式互素的充分必要条件}
便知\((f,f+g)=1\).
同理有\[
	(v-u)g+u(f+g)=1,
\]
即\((g,f+g)=1\).
由\cref{theorem:多项式.互素.性质3}
可知\((fg,f+g)=1\).
\end{proof}
\end{example}

\begin{example}
%@see: 《高等代数(第三版 下册)》(丘维声) P22 习题7.3 7.
设\(f(x),g(x) \in K[x]\),
并且\(a,b,c,d \in K\)满足\(ad-bc\neq0\).
证明:\((af+bg,cf+dg)=(f,g)\).
%TODO proof
\end{example}

\begin{example}
%@see: 《高等代数(第三版 下册)》(丘维声) P22 习题7.3 8.
证明:在\(K[x]\)中,如果\((f(x),g(x))=1\),
则对任意正整数\(m\),
有\((f(x^m),g(x^m))=1\).
%TODO proof
\end{example}

\begin{example}
%@see: 《高等代数(第三版 下册)》(丘维声) P22 习题7.3 9.
证明:\(K[x]\)中两个非零多项式\(f(x)\)与\(g(x)\)不互素的充分必要条件是
存在两个非零多项式\(u(x),v(x)\)
使得\begin{gather*}
	u(x) f(x) = v(x) g(x), \\
	\deg u(x) < \deg g(x), \\
	\deg v(x) < \deg f(x).
\end{gather*}
%TODO proof
\end{example}

\section{最小公倍式}
在\(K[x]\)中,如果\(c(x)\)既是\(f(x)\)的倍式,又是\(g(x)\)的倍式,
则称“\(c(x)\)是\(f(x)\)与\(g(x)\)的一个\DefineConcept{公倍式}”.

\begin{definition}
%@see: 《高等代数(第三版 下册)》(丘维声) P22 习题7.3 10.
设\(f(x),g(x),m(x) \in K[x]\).
如果\(f(x) \mid m(x),
g(x) \mid m(x)\)
且\(f(x)\)与\(g(x)\)的任一公倍式都是\(m(x)\)的倍式,
则称“\(m(x)\)是\(f(x)\)与\(g(x)\)的一个\DefineConcept{最小公倍式}”.
\end{definition}

我们约定,用\[
	[f(x), g(x)]
\]表示首项系数是\(1\)的那个最小公倍式.

\begin{example}
%@see: 《高等代数(第三版 下册)》(丘维声) P22 习题7.3 10.(1)
证明:\(K[x]\)中任意两个多项式都有最小公倍式,
并且在相伴的意义下是唯一的.
%TODO proof
\end{example}

\begin{example}
%@see: 《高等代数(第三版 下册)》(丘维声) P22 习题7.3 10.(2)
证明:如果\(f(x),g(x)\)的首项系数都是\(1\),
则\[
	[f(x),g(x)]
	= \frac{f(x) g(x)}{(f(x),g(x))}.
\]
%TODO proof
\end{example}

\section{不可约多项式,唯一因式分解定理}
我们已经知道,数域\(K\)上的一元多项式环\(K[x]\)具有带余除法.
由此推导出,\(K[x]\)中任意两个多项式都有最大公因式.
现在我们利用这些结论来研究\(K[x]\)的结构.
与整数环\(\mathbb{Z}\)类比:
每一个正整数都能表示成有限多个素数的乘积.
我们不禁发问:\(K[x]\)中每一个多项式是否能表示成有限多个具有类似“素数”那样的性质的多项式的乘积?
联系我们对素数的定义,
对于一个大于\(1\)的正整数\(p\),如果它的正因子只有\(1\)和\(p\),那么称其为素数.
我们可以给出如下概念:
\begin{definition}
%@see: 《高等代数(第三版 下册)》(丘维声) P24. 定义1
\(K[x]\)中一个次数大于零的多项式\(p(x)\),
如果它在\(K[x]\)中的因式只有零次多项式和\(p(x)\)的相伴元,
则称“\(p(x)\)是数域\(K\)上的一个\DefineConcept{不可约多项式}”;
否则称“\(p(x)\)是数域\(K\)上的一个\DefineConcept{可约多项式}”.
\end{definition}

\begin{property}%\label{theorem:多项式.不可约多项式.性质1}
%@see: 《高等代数(第三版 下册)》(丘维声) P24. 性质1
\(K[x]\)中不可约多项式\(p(x)\)与任一多项式\(f(x)\)的关系只有两种可能:
\begin{enumerate}
	\item 要么\(p(x) \mid f(x)\).
	\item 要么\(p(x)\)与\(f(x)\)互素.
\end{enumerate}
\end{property}

\begin{property}\label{theorem:多项式.不可约多项式.性质2}
%@see: 《高等代数(第三版 下册)》(丘维声) P25. 性质2
在\(K[x]\)中,如果\(p(x)\)不可约,且\(p(x) \mid f(x) g(x)\),
则\(p(x) \mid f(x)\)或\(p(x) \mid g(x)\).
\end{property}

利用数学归纳法,\cref{theorem:多项式.不可约多项式.性质2} 可以推广为:
在\(K[x]\)中,如果\(p(x)\)不可约,
且\[
	p(x) \mid f_1(x) f_2(x) \dotsm f_s(x),
\]
则对于某个\(j \in \Set{1,2,\dotsc,s}\),有\(p(x) \mid f_j(x)\).

\begin{property}\label{theorem:多项式.不可约多项式.性质3}
%@see: 《高等代数(第三版 下册)》(丘维声) P25. 性质3
\(K[x]\)中,\(p(x)\)不可约,当且仅当\(p(x)\)不能分解成两个次数较\(p(x)\)的次数低的多项式的乘积.
\end{property}

从\cref{theorem:多项式.不可约多项式.性质3} 立即得出,
\(K[x]\)中的每一个\(1\)次多项式一定是不可约多项式.

\begin{theorem}\label{theorem:多项式.唯一因式分解定理}
%@see: 《高等代数(第三版 下册)》(丘维声) P25. 定理1
\(K[x]\)中每一个次数大于零的多项式\(f(x)\)都能唯一地分解成数域\(K\)上有限多个不可约多项式的乘积.
\end{theorem}

我们把\cref{theorem:多项式.唯一因式分解定理} 称为\DefineConcept{唯一因式分解定理}.
从中可以看出,\(f(x)\)的任一不可约因式一定与\(f(x)\)的分解式中的某一个不可约因式相伴.
因此,\(f(x)\)的分解式给出了它在相伴意义下的全部不可约因式.

\(K[x]\)中的唯一因式分解定理在理论上非常重要,
但是至今仍没有一个统一的方法来做因式分解,
也就是没有统一的方法求出一个次数大于零的多项式的所有不可约因式.

在多项式\(f(x)\)的分解式中,可以把每一个不可约因式的首项系数提出来,
使它们称为首项系数为\(1\)的多项式,再把相同的不可约因式的乘积写成乘幂的形式,
于是\(f(x)\)的分解式成为
\begin{equation}\label{equation:多项式.标准分解式}
	f(x) = c p_1^{r_1}(x) p_2^{r_2}(x) \dotsm p_m^{r_m}(x),
\end{equation}
其中\(c\)是\(f(x)\)的首项系数,
\(p_1(x),p_2(x),\dotsc,p_m(x)\)是不同的首项系数为\(1\)的不可约多项式,
\(\AutoTuple{r}{m}\)是正整数.
我们把分解式 \labelcref{equation:多项式.标准分解式}
称为“\(f(x)\)的\DefineConcept{标准分解式}”.

从理论研究的角度,如果已知两个多项式\(f(x)\)与\(g(x)\)的标准分解式\[
	f(x) = a p_1^{k_1}(x) \dotsm p_l^{k_l}(x) p_{l+1}^{k_{l+1}}(x) \dotsm p_m^{k_m}(x),
\]\[
	g(x) = b p_1^{t_1}(x) \dotsm p_l^{t_l}(x) q_{l+1}^{t_{l+1}}(x) \dotsm q_s^{t_s}(x),
\]
则\(f(x)\)与\(g(x)\)的最大公因式为\[
	(f(x),g(x))
	= p_1^{\min\{k_1,t_1\}}(x) \dotsm p_l^{\min\{k_l,t_l\}}(x).
\]

由于把多项式分解成不可约因式的乘积没有统一的方法,
因此上述最最大公因式的方法不能代替辗转相除法.

\section{重因式}
上一节我们已证明\(K[x]\)中每一个次数大于零的多项式\(f(x)\)能唯一地分解成
数域\(K\)上有限多个不可约多项式的乘积.
如果\(f(x)\)的分解式中每一个不可约因式只出现\(1\)次,
这种情形是特别重要的情形.
这一节我们要给出识别这种情形的一个统一的方法.

\begin{definition}
%@see: 《高等代数(第三版 下册)》(丘维声) P29. 定义1
设\(f \in K[x]\).
如果不可约多项式\(p(x)\)满足\[
	p^k(x) \mid f(x)
	\quad\text{且}\quad
	p^{k+1}(x) \nmid f(x),
\]
那么把\(p(x)\)称为“\(f(x)\)的\(k\)重因式”.

如果\(k=0\),则\(p(x) \nmid f(x)\),因此\(p(x)\)不是\(f(x)\)的因式.
如果\(k=1\),则把\(p(x)\)称为\(f(x)\)的\DefineConcept{单因式}.
如果\(k>1\),则把\(p(x)\)称为\(f(x)\)的\DefineConcept{重因式}.
\end{definition}

显然,如果\(f(x)\)的标准分解式为\[
	f(x) = c p_1^{r_1}(x) p_2^{r_2}(x) \dotsm p_m^{r_m}(x),
\]
则\(p_i^{r_i}(x)\ (i=1,2,\dotsc,m)\)是\(f(x)\)的\(r_i\)重因式.
指数\(r_i = 1\)的那些不可约因式是单因式,
指数\(r_i > 1\)的那些不可约因式是重因式.
因此,\(f(x)\)的分解式中每一个不可约因式只出现\(1\)的情形也就是\(f(x)\)没有重因式的情形.
如何判别一个多项式有没有重因式呢?
由于没有一般的方法来求一个多项式的标准分解式,
因此我们必须寻找别的方法来判断一个多项式有没有重因式.

我们先来看一个简单例子,以便从中受到启发.

设\(f(x) = (x+1)^3 \in \mathbb{R}[x]\),
这时\(f(x)\)有重因式.
如果我们把\(f(x)\)看成数学分析中讨论的多项式函数,
那么对\(f(x)\)可以求导数,得\(f'(x) = 3(x+1)^2\).
于是\((f(x),f'(x)) = (x+1)^2\).
从这个例子受到启发,
有可能运用导数概念以及最大公因式的求法来讨论一个多项式有没有重因式的问题.
由于我们现在讲的多项式是任意数域\(K\)上一个不定元的多项式,
而数学分析中的多项式函数是实变量\(x\)的函数,
其导数概念涉及极限概念,
因此我们不能直接引用数学分析中多项式函数的导数概念,
我们必须给任意数域\(K\)上一元多项式的导数下个定义,
当然这个定义是从数学分析中多项式函数的导数公式得到启发的.

\begin{definition}\label{definition:多项式.导数}
%@see: 《高等代数(第三版 下册)》(丘维声) P30. 定义2
对于\(K[x]\)中的多项式\[
	f(x) = a_n x^n + a_{n-1} x^{n-1} + \dotsb + a_1 x + a_0,
\]
我们把\(K[x]\)中的多项式\[
	n a_n x^{n-1} + (n-1) a_{n-1} x^{n-2} + \dotsb + a_1
\]
叫做“\(f(x)\)的\DefineConcept{一阶导数}”,记作\(f'(x)\).
我们还把\(f'(x)\)的一阶导数称为“\(f(x)\)的\DefineConcept{二阶导数}”,记作\(f''(x)\);
把\(f''(x)\)的一阶导数称为“\(f(x)\)的\DefineConcept{三阶导数}”,记作\(f'''(x)\);
把\(f'''(x)\)的一阶导数称为“\(f(x)\)的\DefineConcept{四阶导数}”,记作\(f^{(4)}(x)\);
以此类推.
\end{definition}

从\cref{definition:多项式.导数} 立即得出,
一个\(n\)次多项式的导数是一个\(n-1\)次多项式,
它的\(n\)阶导数是\(K\)中一个非零数,
它的\(n+1\)阶导数等于零.
零多项式的导数是零多项式.

根据\cref{definition:多项式.导数},可以验证得到\(K[x]\)中多项式的导数的基本公式:\begin{gather}
	[f(x)+g(x)]' = f'(x) + g'(x), \\
	[c f(x)]' = c f'(x), \quad c \in K, \\
	[f(x) g(x)]' = f'(x) g(x) + f(x) g'(x), \\
	[f^m(x)]' = m f^{m-1}(x) f'(x).
\end{gather}

让我们回头再看一遍之前举的简单例子,
不可约多项式\(x+1\)是\(f(x) = (x+1)^3\)的\(3\)重因式.
由于按\cref{definition:多项式.导数} 和上述公式可得出,
\(f'(x) = 3(x+1)^2\),
因此\(x+1\)是\(f'(x)\)的\(2\)重因式.
我们从这个例子得出的结论具有一般性.

\begin{theorem}
%@see: 《高等代数(第三版 下册)》(丘维声) P30. 定理1
设\(K\)是数域,在\(K[x]\)中,
如果不可约多项式\(p(x)\)是\(f(x)\)的一个\(k\ (k\geq1)\)重因式,
则\(p(x)\)是\(f(x)\)的导数\(f'(x)\)的一个\(k-1\)重因式.
特别地,多项式\(f(x)\)的单因式不是\(f(x)\)的导数\(f'(x)\)的因式.
\end{theorem}

\begin{corollary}
%@see: 《高等代数(第三版 下册)》(丘维声) P31. 推论2
设\(K\)是数域,在\(K[x]\)中,不可约多项式\(p(x)\)是\(f(x)\)的重因式的充要条件是:
\(p(x)\)是\(f(x)\)与\(f'(x)\)的公因式.
\end{corollary}

\section{多项式的根}
从唯一因式分解定理知道,
\(K[x]\)中每一个次数大于零的多项式都能唯一地分解成数域\(K\)上有限多个不可约多项式的乘积.
由此看出,不可约多项式之于\(K[x]\)正如砖块之于城市,
这促使我们取搞清楚\(K[x]\)中不可约多项式有哪些.
我们已经知道,\(K[x]\)中每一个一次多项式都是不可约的.
于是需要进一步研究的是,
\(K[x]\)中有没有次数大于\(1\)的不可约多项式?
显然,在\(K[x]\)中,如果\(p(x)\)是次数大于\(1\)的不可约多项式,则\(p(x)\)没有一次因式.
从这点受到启发,首先需要研究\(K[x]\)中一个多项式\(f(x)\)有一次因式的充分必要条件.
为此,我们需要用一次多项式去除\(f(x)\),观察它的余式.

\begin{theorem}[余数定理]\label{theorem:多项式.余数定理}
%@see: 《高等代数(第三版 下册)》(丘维声) P33 定理1
在\(K[x]\)中,用\(x-a\)去除\(f(x)\)所得的余式是\(f(a)\).
\begin{proof}
作带余除法,得\[
	f(x) = h(x) (x-a) + r(x), \qquad
	\deg r(x) < \deg(x-a)=1.
\]
可见\(r(x)\)要么是零多项式,要么是零次多项式.
不妨设\(r(x)=r \in K\).
于是上式成为\[
	f(x) = h(x) (x-a) + r, \qquad
	r \in K.
\]
在上式中,\(x\)用\(a\)代入,得\(f(a) = r\).
因此用\(x-a\)去除\(f(x)\)所得的余式是\(f(a)\).
\end{proof}
\end{theorem}

\begin{corollary}\label{theorem:多项式.余数定理的推论}
%@see: 《高等代数(第三版 下册)》(丘维声) P34 推论2
在\(K[x]\)中,\(x-a\)整除\(f(x)\)当且仅当\(f(a)=0\).
\begin{proof}
由\cref{theorem:多项式.带余除法.推论}
和\cref{theorem:多项式.余数定理}
立即可得.
\end{proof}
\end{corollary}

从\cref{theorem:多项式.余数定理的推论} 受到启发,引出多项式的根的概念.

\begin{definition}\label{theorem:多项式.根的定义}
%@see: 《高等代数(第三版 下册)》(丘维声) P34 定义1
设\(K\)是一个数域,
\(R\)是一个有单位元的交换环,
且\(R\)可看成是\(K\)的一个扩环.
对于\(K[x]\)中一个多项式\(f(x)\),
如果\(R\)中有一个元素\(c\)使得\(f(c)=0\),
则称“\(c\)是\(f(x)\)在\(R\)中的一个\DefineConcept{根}”.
\end{definition}

多项式在复数域中的根称为\DefineConcept{复根}.
实系数多项式在实数域中的根称为\DefineConcept{实根}.
有理系数多项式在有理数域中的根称为\DefineConcept{有理根}.

从\cref{theorem:多项式.根的定义} 和\cref{theorem:多项式.余数定理的推论} 立即得到下述重要结论:
\begin{theorem}\label{theorem:多项式.贝祖定理}
%@see: 《高等代数(第三版 下册)》(丘维声) P34 定理3
在\(K[x]\)中,\(x-a\)整除\(f(x)\)当且仅当\(a\)是\(f(x)\)在\(K\)中的一个根.
\end{theorem}

我们把\cref{theorem:多项式.贝祖定理} 称为\DefineConcept{贝祖定理}.
从\cref{theorem:多项式.贝祖定理} 看出,
\(K[x]\)中的多项式\(f(x)\)有一次因式的充分必要条件是\(f(x)\)在\(K\)中有根.

利用根与一次因式的关系,
对于K[x]中的多项式在K中的根,我们可以定义“重根”的概念:
如果\(x-a\)是\(f(x)\)的\(k\)重因式,
那么我们把\(a \in K\)称为“\(f(x) \in K[x]\)的一个\(k\)重根”.
当\(k=1\)时,\(a\)称为\DefineConcept{单根};
当\(k>1\)时,\(a\)称为\DefineConcept{重根}.

另外,再次利用根与一次因式的关系,
我们还可以得到\(K[x]\)中的多项式在\(f(x)\)在\(K\)中的根的数目的一个上界:
\begin{theorem}\label{theorem:多项式.根的数目的上界}
%@see: 《高等代数(第三版 下册)》(丘维声) P34 定理4
\(K[x]\)中的\(n\ (n\geq0)\)次多项式在\(K\)中至多有\(n\)个根(重根按重数计算).
\end{theorem}

从\cref{theorem:多项式.根的数目的上界} 可以得到一个重要推论:
\begin{corollary}
%@see: 《高等代数(第三版 下册)》(丘维声) P34 推论5
设\(K[x]\)中两个多项式\(f(x)\)与\(g(x)\)的次数都不超过\(n\).
如果\(K\)中有\(n+1\)个不同元素\(\AutoTuple{a}{n+1}\),
使得\(f(a_i)=g(a_i)\ (i=1,2,\dotsc,n,n+1)\),
则\(f(x)=g(x)\).
\end{corollary}

为了研究多项式的根,我们需要多项式函数的概念,并且研究多项式函数与多项式之间的关系.

设\(f \in K[x]\).
对于\(K\)中每一个元素\(a\),
\(x\)用\(a\)代入得\(f(a) \in K\).
于是\(K[x]\)中的一个多项式\(f(x)\)确定了\(K\)到\(K\)的一个映射\[
    f\colon K \to K, a \mapsto f(a).
\]
这种由\(K[x]\)中的多项式确定的\(K\)上的函数称为\(K\)上的\DefineConcept{一元多项式函数}.

我们已经知道,\(K[x]\)中的每一个多项式都确定一个\(K\)上的医院多项式函数.
现在要问:给定\(K[x]\)中两个不相等的多项式\(f(x)\)与\(g(x)\),
它们确定的\(K\)上的医院多项式函数\(f\)和\(g\)是否不相等?
\begin{theorem}\label{theorem:多项式.多项式函数是否相等取决于多项式是否相等}
%@see: 《高等代数(第三版 下册)》(丘维声) P35 定理6
数域\(K\)上的两个多项式\(f(x)\)与\(g(x)\)如果不相等,
则它们确定的\(K\)上的一元多项式函数\(f\)与\(g\)也不相等.
\end{theorem}

我们把数域\(K\)上的所有一元多项式函数组成的集合也记作\(K[x]\).
让一元多项式环\(K[x]\)中的多项式\(f(x)\)对应到它确定的\(K\)上的函数\(f\),
这时从一元多项式环\(K[x]\)到一元多项式函数族\(K[x]\)的一个映射,这里记为\(\sigma\).
显然\(\sigma\)是满射;
根据\cref{theorem:多项式.多项式函数是否相等取决于多项式是否相等},\(\sigma\)又是单射;
因此\(\sigma\)是双射,是一个从一元多项式环\(K[x]\)到一元多项式函数族\(K[x]\)的同构.

\section{贝祖定理}
\begin{theorem}
设\(a,b\in\mathbb{Z}\).
则\(a\)与\(b\)互素的充要条件是:
存在\(u,v\in\mathbb{Z}\),使得\[
	u a + v b = 1.
\]
\end{theorem}

\begin{corollary}
设\(f(x)\)和\(g(x)\)是\(\mathbb{P}[x]\)中两个不全为0的多项式,
则\(f(x)\)与\(g(x)\)互素(即\(f(x)\)与\(g(x)\)在\(\mathbb{C}\)上没有公共根)的充要条件是:
存在\(u(x),v(x)\in\mathbb{P}[x]\),使得\[
	u(x) f(x) + v(x) g(x) = 1.
\]
\end{corollary}

\begin{example}
设矩阵\(\A\)满足\(\A^3+\E=2\A\),其中\(\E\)是单位矩阵,
证明:\(2\A^2+\A-\E\)可逆.
\begin{proof}
令\(f(x)=x^3-2x+1\),\(g(x)=2x^2+x-1\),
因式分解可得\[
	f(x) = (x-1)(x^2+x-1),
\]\[
	g(x) = (2x-1)(x+1).
\]
显然\(f(x)\)与\(g(x)\)在\(\mathbb{C}\)上没有公共根,互素.
故根据贝祖定理,存在\(u(x),v(x)\in\mathbb{P}[x]\),使得\[
	u(x) \cdot (x^3-2x+1) + v(x) \cdot (2x^2+x-1) = 1,
\]
代入矩阵\(\A\),并注意到\(\A^3-2\A+\E=\z\),得到\[
	v(\A) \cdot (2\A^2+\A-\E) = \E,
\]
也就是说,矩阵\(2\A^2+\A-\E\)可逆,
其逆矩阵为\(v(\A)\),
而\(v(\A)\)可以通过辗转相除法得到.
\end{proof}
\end{example}

\begin{example}
设\(\A\)是数域\(\mathbb{P}\)上的\(n\)阶方阵.
证明:若\(\A^2=\E\),则\[
	\rank(\A+\E)+\rank(\A-\E)=n.
\]
\begin{proof}
由于\(x+1\)与\(x-1\)互素,存在\(u(x),v(x)\in\mathbb{P}[x]\),使得\[
	u(x) \cdot (x+1) + v(x) \cdot (x-1) = 1.
\]
代入矩阵\(\A\),得\[
	u(\A) (\A+\E) + v(\A) (\A-\E) = \E.
\]

考虑\(2n\)阶方阵
\begin{align*}
	\begin{bmatrix}
		\A+\E & \z \\
		\z & \A-\E
	\end{bmatrix}
	&\xlongrightarrow{\text{(2列)}+=u(\A)\times\text{(1列)}}
	\begin{bmatrix}
		\A+\E & u(\A) (\A+\E) \\
		\z & \A-\E
	\end{bmatrix} \\
	&\xlongrightarrow{\text{(1行)}+=v(\A)\times\text{(2行)}}
	\begin{bmatrix}
		\A+\E & \E \\
		\z & \A-\E
	\end{bmatrix} \\
	&\to
	\begin{bmatrix}
		\z & \E \\
		\A^2-\E & \z
	\end{bmatrix}
	= \begin{bmatrix}
		\z & \E \\
		\z & \z
	\end{bmatrix}.
\end{align*}
于是\(\rank(\A+\E)+\rank(\A-\E)=n\).
\end{proof}
\end{example}

\endgroup
