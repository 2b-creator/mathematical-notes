\chapter*{写在前面的话}
我在记笔记的时候参考了
恩德尔顿 编写的《Elements of Set Theory》(ISBN 0-12-238440-7)、《A Mathematical Introduction to Logic (2nd Edition)》(ISBN 0-12-238452-0),
潘承洞 编写的《初等数论》(ISBN 978-7-301-21612-5),
希尔伯特 编写的《几何基础》,
张慎语、周厚隆 编写的《线性代数》(ISBN 978-7-04-010545-2),
丘维声 编写的
《解析几何(第三版)》(ISBN 978-7-301-25921-4)、
《高等代数(第三版\ 上册)》(ISBN 978-7-04-041880-4)、
《高等代数(第三版\ 上册)》(ISBN 978-7-04-042235-1)、
《近世代数》(ISBN 978-7-301-25580-3),
同济大学数学系 编写的《高等数学(第六版\ 上册)》(ISBN 978-7-04-020549-7)和《高等数学(第六版\ 下册)》(ISBN 978-7-04-021277-8),
李逸 编写的《基本分析讲义:第一卷(单变量理论)》、《基本分析讲义:第二卷(多变量理论)》,
辛钦 编写的《数学分析八讲》(ISBN 978-7-115-39747-8),
熊金城 编写的《点集拓扑讲义》(ISBN 978-7-04-032237-8),
黄大奎、舒慕曾 编写的《数学物理方法》(ISBN 978-7-04-010326-7),
龚昇 编写的《简明复分析(第2版)》(ISBN 978-7-312-02169-5),
陈鸿建、赵永红、翁洋 编写的《概率论与数理统计》(ISBN 978-7-04-024894-4),
施利亚耶夫 编写的《概率论》,
茆诗松 编写的《概率论与数理统计(第二版)》、《概率论与数理统计(第四版)》(ISBN 978-7-5037-9345-5),
薛留根 编写的《概率论解题方法与技巧》(ISBN 7-118-01414-1).

在此对先生们表达由衷的感激!
