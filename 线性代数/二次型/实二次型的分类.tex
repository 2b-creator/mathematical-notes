\section{实二次型的分类}
\subsection{实二次型的分类标准}
\begin{definition}
设有\(n\)元实二次型\(f(\x) = \x^T\A\x\).
\begin{enumerate}
	\item 如果\[
		(\forall\x\in\mathbb{R}^n-\{\z\})
		[f(\x) > 0],
	\]
	则称“\(f\)是\DefineConcept{正定的}(positive definite)”;
	把\(\A\)称为\DefineConcept{正定矩阵}(positive definite matrix),
	记为\(\A\succ\z\).

	\item 如果\[
		(\forall\x\in\mathbb{R}^n-\{\z\})
		[f(\x) \geq 0],
	\]
	则称“\(f\)是\DefineConcept{半正定的}(positive semi-definite)”;
	把\(\A\)称为\DefineConcept{半正定矩阵}(positive semi-definite matrix),
	记为\(\A\succeq\z\).

	\item 如果\[
		(\forall\x\in\mathbb{R}^n-\{\z\})
		[f(\x) < 0],
	\]
	则称“\(f\)是\DefineConcept{负定的}(negative definite)”,
	把\(\A\)称为\DefineConcept{负定矩阵}(negative definite matrix),
	记为\(\A\prec\z\).

	\item 如果\[
		(\forall\x\in\mathbb{R}^n-\{\z\})
		[f(\x) \leq 0],
	\]
	则称“\(f\)是\DefineConcept{半负定的}(negative semi-definite)”;
	把\(\A\)称为\DefineConcept{半负定矩阵}(negative semi-definite matrix),
	记为\(\A\preceq\z\).

	\item 否则,称\(f\)是\DefineConcept{不定的}(indefinite).
\end{enumerate}
\end{definition}

\subsection{惯性定理}
\begin{theorem}\label{theorem:二次型.惯性定理}
\(n\)元实二次型\(f(\x) = \x^T\A\x\)经过任意满秩线性变换化为标准型,
所得的标准型的正平方项的项数\(p\)及负平方项的项数\(q\)都是唯一确定的.
\begin{proof}
\def\z{\mat{z}}%
设实二次型的秩为\(r\).
假设\(f\)经过两个不同的可逆线性替换\(\x=\C\y,\x=\D\z\)分别化为标准型\[
	f \xlongequal{\x=\C\y}
	c_1 y_1^2 + c_2 y_2^2 + \dotsb + c_p y_p^2 - c_{p+1} y_{p+1}^2 - \dotsb - c_r y_r^2,
	\eqno{(1)}
\]\[
	f \xlongequal{\x=\D\z}
	d_1 z_1^2 + d_2 z_2^2 + \dotsb + d_q z_q^2 - d_{q+1} z_{q+1}^2 - \dotsb - d_r z_r^2,
	\eqno{(2)}
\]
其中\(c_i,d_i>0\ (i=1,2,\dotsc,r)\).

用反证法.
设\(p > q\),由\(\x = \C\y = \D\z\),\(\D\)可逆,得\(\z = \D^{-1} \C \y\).
\def\H{\mat{H}}%
\def\zexpr#1{h_{#1 1} y_1 + h_{#1 2} y_2 + \dotsb + h_{#1 n} y_n}%
记\(\H = (h_{ij})_n = \B^{-1} \C\),
则\(\z = \H\y\),
即\[
	z_i = \zexpr{i}
	\quad(i=1,2,\dotsc,n).
\]
于是\[
	\begin{aligned}
		&\hspace{-40pt}
		c_1 y_1^2 + c_2 y_2^2 + \dotsb
			+ c_p y_p^2 - c_{p+1} y_{p+1}^2 - \dotsb - c_r y_r^2 \\
		&\hspace{-20pt}= d_1 (\zexpr{1})^2 + d_2 (\zexpr{2})^2 \\
		&+ \dotsb + d_q (\zexpr{q})^2 \\
		&- d_{q+1} (\zexpr{q+1})^2 - \dotsb \\
		&- d_r (\zexpr{r})^2.
	\end{aligned}
	\eqno{(3)}
\]
由此可以构造齐次线性方程组\[
	\begin{cases}
		\zexpr{1} = 0, \\
		\hdotsfor{1} \\
		\zexpr{q} = 0, \\
		y_{p+1} = 0, \\
		\hdotsfor{1} \\
		y_n = 0.
	\end{cases}
	\eqno{(4)}
\]
这个方程组中有\(n\)个未知量,\(q+n-p < n\)个方程,
于是它有非零解\((\AutoTuple{y}{p},0,\dotsc,0)^T\),
代入(3)式两端,得到左边大于零,右边小于等于零,矛盾,因此\(p \leq q\).
同理又有\(q \leq p\),于是\(p = q\).
\end{proof}
\end{theorem}
\cref{theorem:二次型.惯性定理}
又称为“惯性定理(Inertial Theorem)”.

\begin{corollary}
任意\(n\)元实二次型\(f(\AutoTuple{x}{n}) = \x^T\A\x\),总可经过满秩线性变换化为以下形式的标准型\[
f(\AutoTuple{x}{n})
=y_1^2+y_2^2+ \dotsb +y_p^2
-y_{p+1}^2-\dotsb-y_r^2
\]称为\(f(\AutoTuple{x}{n})\)的\DefineConcept{规范型}({\rm normal form}),且规范型是唯一的.
\begin{proof}
根据惯性定理,\(f\)经过可逆线性替换化为标准型:\[
f \xlongequal{\x=\D\z} d_1 z_1^2 + d_2 z_2^2 + \dotsb + d_q z_q^2 - d_{q+1} z_{q+1}^2 - \dotsb - d_r z_r^2,
\]其中\(d_i>0\ (i=1,2,\dotsc,r)\).
令\[
\C = \D \diag(d_1^{-1/2},\dotsc,d_r^{-1/2},1,\dotsc,1),
\]则\(\x = \D\z = \C\y\)是可逆线性替换,使得\[
f \xlongequal{\x=\D\z} y_1^2 + y_2^2 + \dotsb + y_q^2 - y_{q+1}^2 - \dotsb - y_r^2.
\]二次型的规范型的唯一性可以由惯性定理得到.
\end{proof}
\end{corollary}

\begin{definition}
在秩为\(r\)的实二次型\(f(\AutoTuple{x}{n})\)所化成的标准型(或规范型)中,
\begin{enumerate}
	\item 正平方项的项数\(p\)
	称为“\(f\)的\DefineConcept{正惯性指数}(positive index of inertia)”;
	\item 负平方项的项数\(q=r-p\)
	称为“\(f\)的\DefineConcept{负惯性指数}(minus index of inertia)”;
	\item 正、负惯性指数之差\(d=p-q=2p-r\)
	称为“\(f\)的\DefineConcept{符号差}(signature)”.
\end{enumerate}
\end{definition}
由惯性定理和以上定义可知:
可逆线性替换不改变二次型的正、负惯性指数.
从而可由二次型的正、负惯性指数确定二次型的类型.

\begin{theorem}
设\(\A\)和\(\B\)是同阶实对称矩阵.
这两个矩阵合同的充要条件是两者的秩、正负惯性指数均相等,
即\[
	\A\simeq\B
	\iff
	\rank\A=\rank\B \land p_{\A}=p_{\B} \land q_{\A}=q_{\B}.
\]
\end{theorem}

\subsection{正定矩阵的等价条件}
\begin{theorem}
设\(\A\)为\(n\)阶实对称矩阵,
\(f(\AutoTuple{x}{n}) = \x^T\A\x\),
则下列命题相互等价:
\begin{enumerate}
	\item \(\A\)为正定矩阵;
	\item \(\A\)的特征值全是正实数;
	\item \(f(\AutoTuple{x}{n})\)的正惯性指数\(p=n\);
	\item \(\A \cong \E\)(即,存在可逆实阵\(\C\),使得\(\C^T\A\C=\E\));
	\item 存在可逆实阵\(\P\),使得\(\A=\P^T\P\).
\end{enumerate}
\end{theorem}

\begin{corollary}
正定矩阵的行列式大于零.
\begin{proof}
因为\(\A\)正定,所以存在可逆实阵\(\P\),
使得\(\A=\P^T\P\),
则\(\abs{\A}=\abs{\P^T\P}=\abs{\P^T}\abs{\P}=\abs{\P}^2>0\).
\end{proof}
\end{corollary}

\begin{theorem}
\(n\)元实二次型\(f(\AutoTuple{x}{n}) = \x^T\A\x\ (\A=\A^T)\)正定的充要条件是:
\(\A\)的各阶顺序主子式均大于零.
\begin{proof}
必要性.对于任意不全为零的\(n\)个实数\(c_1,c_2,\dotsc,c_k,0,\dotsc,0\),总有\[
f(c_1,c_2,\dotsc,c_k,0,\dotsc,0) = \sum\limits_{i=1}^k \sum\limits_{j=1}^k c_i c_j a_{ij} > 0,
\]
从而\(k\)元实二次型\(f_k(x_1,x_2,\dotsc,x_k)
= \sum\limits_{i=1}^k \sum\limits_{j=1}^k x_i x_j a_{ij}\)正定,
而\(f_k\)的矩阵为\(\A_k = (a_{ij})_k\),
那么\(\abs{\A_k} > 0\ (k=1,2,\dotsc,n)\).

充分性.当\(n=1\)时,\(a_{11} > 0\),\(f_1(x_1) = a_{11} x_1^2\)正定.
设\(n=k-1\)时结论成立,当\(n=k\)时,将\(\A\)分块得\(\A = \begin{bmatrix}
\A_{k-1} & \a \\
\a^T & a_{nn}
\end{bmatrix}\),其中\(\A_{k-1}\)为各阶顺序主子式都大于零的\(k-1\)阶实对称矩阵.
由归纳假设,\(\A_{k-1}\)正定,故存在\(k-1\)阶可逆矩阵\(\Q\),
使得\(\A_{k-1} = \Q^T \Q\),\(\A_{k-1}\)可逆,
\(\A_{k-1}^{-1} = \Q^{-1}(\Q^{-1})^T\)是对称矩阵.
令\(\P = \begin{bmatrix}
\Q^{-1} & -\A_{k-1}^{-1} \a \\
\z & 1
\end{bmatrix}\),
则\(\P\)可逆,
于是\begin{align*}
	\P^T \A \P &= \begin{bmatrix}
		(\Q^{-1})^T & \z \\
		-\a^T \A_{k-1}^{-1} & 1
	\end{bmatrix}
	\begin{bmatrix}
		\Q^T \Q & \a \\
		\a^T & a_{nn}
	\end{bmatrix}
	\begin{bmatrix}
		\Q^{-1} & -\A_{k-1}^{-1} \a \\
		\z & 1
	\end{bmatrix} \\
	&= \begin{bmatrix}
		\Q & (\Q^{-1})^T \a \\
		\z & b
	\end{bmatrix}
	\begin{bmatrix}
		\Q^{-1} & -\A_{k-1}^{-1} \a \\
		\z & 1
	\end{bmatrix}
	= \begin{bmatrix}
		\E_{k-1} & \z \\
		\z & b
	\end{bmatrix} = \B,
\end{align*}
其中\(b=a_{nn}-\a^T \A_{k-1}^{-1} \a\).
由于\(\A\)与\(\B\)合同,\(\abs{\A} > 0\),
得\(\abs{\B} = b > 0\),
作可逆线性替换\(\x = \P\y\),则\[
	f \xlongequal{\x=\Q\y} y_1^2 + y_2^2 + \dotsb + y_{n-1}^2 + b y_n^2,
\]
故\(f\)的正惯性指数为\(n\),\(f\)正定.
\end{proof}
\end{theorem}

\begin{corollary}
\(n\)元实二次型\(f(\AutoTuple{x}{n}) = \x^T\A\x\ (\A=\A^T)\)负定的充要条件是:\[
(-1)^k D_k = (-1)^k \begin{vmatrix}
a_{11} & a_{12} & \dots & a_{1k} \\
a_{21} & a_{22} & \dots & a_{2k} \\
\vdots & \vdots & \ddots & \vdots \\
a_{k1} & a_{k2} & \dots & a_{kk}
\end{vmatrix} > 0,
\quad k=1,2,\dotsc,n.
\]
\end{corollary}

\begin{example}
设\(\A\)为实对称矩阵,证明:当实数\(t\)充分大时,\(t\E+\A\)是正定矩阵.
\begin{proof}
因为\(\A\)为实对称矩阵,所以存在正交矩阵\(\Q\),使得\[
\Q^T\A\Q = \diag(\AutoTuple{\lambda}{n}).
\]又因为\[
\Q^T(t\E+\A)\Q
= \diag(t+\L{1},t+\L{2},\dotsc,t+\L{n}),
\]所以当\(t+\L{1},t+\L{2},\dotsc,t+\L{n}\)都大于零时,\(t\E+\A\)正定.
\end{proof}
\end{example}

\begin{example}
设\(\A\)、\(\B\)是同阶正定矩阵,证明:\(\A+\B\)、\(\A^{-1}\)、\(\A^*\)是正定矩阵.
\begin{proof}
根据正定矩阵的定义,因为\(\A\)是正定矩阵,任取非零列向量\(\x\),都有\[
\x^T\A\x > 0;
\]同样地,有\(\x^T\B\x > 0\).

又根据矩阵的乘法分配律,有\[
\x^T(\A+\B)\x = \x^T\A\x + \x^T\B\x > 0
\]成立,即\(\A+\B\)是正定矩阵.

因为\(\A\)是正定矩阵,存在可逆实阵\(\P\)使得\(\P^T\A\P=\E\),所以\[
\A = (\P^T)^{-1}\E\P^{-1} = (\P^T)^{-1}\P^{-1}
\implies
\A^{-1} = \P\P^T,
\]说明\(\A^{-1}\)是正定矩阵.

由逆矩阵的定义,\(\A^{-1}=\frac{1}{\abs{\A}}\A^*\),那么\(\A^*=\abs{\A}\A^{-1}\),\(\abs{\A}>0\),显然\(\A^*\)也是正定矩阵.
\end{proof}
\end{example}

\begin{example}
设\(\A\)是正定矩阵,试证:存在正定矩阵\(\B\),使得\(\A=\B^2\).
\begin{proof}
设\(\A\)是\(n\)阶正定矩阵,那么存在正交矩阵\(\P\)满足\[
\P^T\A\P = \diag(\AutoTuple{\lambda}{n}) = \V,
\]其中\(\AutoTuple{\lambda}{n} \in \mathbb{R}^+\).
又设矩阵\(\B\)满足\(\B^2=\A\),那么\[
\P^T\A\P = \P^T\B^2\P = \V
\iff
\B^2 = \P\V\P^T,
\]
只需要令\(\B = \P \diag(\sqrt{\L{1}},\sqrt{\L{2}},\dotsc,\sqrt{\L{n}}) \P^T\)即可.
\end{proof}
\end{example}
