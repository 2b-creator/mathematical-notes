\section{二次型的基本概念}
我们首先研究平面解析几何中以坐标原点为中心的二次曲线的方程:
\begin{center}
\def\arraystretch{1.5}
\begin{tabular}{cl}
圆 & \(x^2+y^2=r^2\) \\
椭圆 & \(\frac{x^2}{a^2}+\frac{y^2}{b^2}=1\) \\
双曲线 & \(\frac{x^2}{a^2}-\frac{y^2}{b^2}=1\) \\
\end{tabular}
\end{center}

可以看出,它们都具有\[
a x^2 + 2b xy + c y^2 = d
\]的形式.
在研究二次曲线时,如果得到的方程不是标准方程,我们通常希望通过旋转、平移等几何变换将其化为标准方程,进而判别曲线的形状和几何性质.

\subsection{二次型的基本概念}
\begin{definition}
称系数\(a_{ij}\ (1 \leq i \leq j \leq n)\)属于数域\(P\)的\(n\)个变量的二次齐次多项式\begin{equation}\label{equation:二次型.二次型}
f(\AutoTuple{x}{n})
= \sum\limits_{i=1}^n \sum\limits_{j=1}^n a_{ij} x_i x_j
\quad(a_{ji}=a_{ij})
\end{equation}为数域\(P\)上的一个\(n\)元\DefineConcept{二次型}(quadratic form)\footnote{%
本章不作特别声明时,“二次型”均指实二次型.}.
\end{definition}

数域对于一个二次齐次多项式是否成为二次型是决定性的.
多项式\[
f(x_1,x_2,x_3) = x_1^2 + 4 x_1 x_2 + 3 x_2^2 + 5 x_2 x_3 - x_3^2
\]和\[
g(x_1,x_2,x_3) = x_1^2 + 2\sqrt{2} x_1 x_2 + 2 x_1 x_3 + 2 x_2^2 + 4\sqrt{3} x_2 x_3
\]都是实数域上的二次型;
但在有理数域上,只有\(f\)是二次型,\(g\)不是二次型.

前面提到我们希望将一般方程化为标准方程,现在我们就要定义何种形式的方程应该被称为标准方程.
再次观察平面二次曲线的标准方程可以发现,标准方程的等号左边应该是二次齐次多项式(即若干个变量的平方和),等号右边则应该是任意(非零)常数.

\begin{definition}
若\(n\)阶对称矩阵\(\A = (a_{ij})_n\)满足
\begin{equation}\label{equation:二次型.二次型的矩阵表示}
f(\AutoTuple{x}{n}) = \x^T\A\x,
\end{equation}其中\(\x = (\AutoTuple{x}{n})^T\),
则称\cref{equation:二次型.二次型的矩阵表示} 为二次型\(f(\AutoTuple{x}{n})\)的\DefineConcept{矩阵表示},
对称矩阵\(\A\)为\(f\)的\DefineConcept{矩阵},
\(\A\)的秩\(\rank\A\)为\(f\)的\DefineConcept{秩}.
\end{definition}

显然,对于任一\(n\)阶矩阵\(\B\),\(\x^T\B\x\)必定是一个二次型.
需要注意的是,矩阵\(\B\)不必是对称矩阵,但“二次型\(\x^T\B\x\)的矩阵”必定是一个对称矩阵.

\begin{property}
二次型和它的矩阵是相互唯一确定的.
\begin{proof}
对于二次型\(f(\AutoTuple{x}{n})\),设非零\(n\)阶对称矩阵\(\A\)和\(\B\)都是\(f\)的矩阵,即\[
\x^T\A\x
=\x^T\B\x
=f(\AutoTuple{x}{n}),
\]则二次型\(\x^T\A\x\)与\(\x^T\B\x\)中\(x_i x_j\)的系数\(2 a_{ij}\)与\(2 b_{ij}\ (1 \leq i < j \leq n)\)必相等,\(x_i^2\)的系数\(a_{ii}\)与\(b_{ii}\ (i=1,2,\dotsc,n)\)必相等,故\(\A=\B\).
\end{proof}
\end{property}

\begin{example}
将\(f(x_1,x_2,x_3) = x_1^2 + 4 x_1 x_2 + 3 x_2^2 + 5 x_2 x_3 - x_3^2\)写成矩阵形式.
\begin{solution}
\(f(x_1,x_2,x_3) = \begin{bmatrix}
x_1 & x_2 & x_3
\end{bmatrix} \begin{bmatrix}
1 & 2 & 0 \\
2 & 3 & \frac{5}{2} \\
0 & \frac{5}{2} & -1
\end{bmatrix} \begin{bmatrix}
x_1 \\ x_2 \\ x_3
\end{bmatrix}\).
\end{solution}
\end{example}

\begin{example}
写出二次型\(\begin{bmatrix}
x_1 & x_2 & x_3
\end{bmatrix} \begin{bmatrix}
2 & -3 & 1 \\
1 & 0 & 1 \\
2 & 11 & 3
\end{bmatrix} \begin{bmatrix}
x_1 \\ x_2 \\ x_3
\end{bmatrix}\)的矩阵.
\begin{solution}
注意到矩阵\(\A = \begin{bmatrix}
2 & -3 & 1 \\
1 & 0 & 1 \\
2 & 11 & 3
\end{bmatrix}\)不是对称矩阵,二次型\(\x^T\A\x\)的矩阵应为\[
\B = \frac{\A+\A^T}{2} = \begin{bmatrix}
2 & -1 & \frac{3}{2} \\
-1 & 0 & 6 \\
\frac{3}{2} & 6 & 3
\end{bmatrix}.
\]
\end{solution}
\end{example}

\subsection{线性替换}
\begin{definition}
因为平面二次曲线方程通过旋转变换化为标准方程,
实际上是用新变量的一次式代替原来的变量.
同样地,使用这种基本的方法来化简一般的\(n\)元二次型,
作如下的变量替换:\[
	\left\{ \begin{array}{l}
		x_1 = c_{11}y_1 + c_{12}y_2 + \dotsb + c_{1n}y_n \\
		x_2 = c_{21}y_1 + c_{22}y_2 + \dotsb + c_{2n}y_n \\
		\hdotsfor{1} \\
		x_n = c_{n1}y_1 + c_{n2}y_2 + \dotsb + c_{nn}y_n
	\end{array} \right.
\]
写成矩阵形式\[
	\begin{bmatrix}
		x_1 \\ x_2 \\ \vdots \\ x_n
	\end{bmatrix}
	= \begin{bmatrix}
		c_{11} & c_{12} & \dots & c_{1n} \\
		c_{21} & c_{22} & \dots & c_{2n} \\
		\vdots & \vdots & & \vdots \\
		c_{n1} & c_{n2} & \dots & c_{nn}
	\end{bmatrix}
	\begin{bmatrix}
		y_1 \\ y_2 \\ \vdots \\ y_n
	\end{bmatrix}
	\quad\text{或}\quad
	\x=\C\y.
\]

上述变量之间的替换称为\DefineConcept{线性替换}.

当矩阵\(\C=(c_{ij})_n\)可逆时,
称为\DefineConcept{可逆线性替换},
或\DefineConcept{满秩线性替换},
或\DefineConcept{非退化线性替换}.
\end{definition}

\begin{theorem}
对二次型\(f(\AutoTuple{x}{n})=\x^T\A\x\ (\A=\A^T)\)作可逆线性替换\(\x=\C\y\),则\(f\)化为新变量的二次型\(g(\AutoTuple{y}{n})=\y^T\B\y\),其中\(\B=\C^T\A\C\)为\(g\)的矩阵.
\begin{proof}
\(f(\AutoTuple{x}{n}) = \x^T\A\x%
\xlongequal{\x=\C\y} (\C\y)^T\A(\C\y)%
= \y^T (\C^T\A\C) \y\),
令\(\B = \C^T\A\C\),由于\(\B^T = (\C^T\A\C)^T = \C^T\A\C = \B\),以及\(\C\)可逆,所以\(\B\)是对称矩阵.
\(f\)是二次型,它的矩阵\(\A\neq\z\),\(\B\cong\A\),故\(\B\neq\z\).
于是,\(g(\AutoTuple{y}{n})=\y^T\B\y\)是二次型,对称矩阵\(\B\)是\(g\)的矩阵.
\end{proof}
\end{theorem}

\subsection{矩阵合同的概念}
\begin{definition}
设\(\A\)和\(\B\)是两个\(n\)阶矩阵,
若存在可逆矩阵\(\C\),
使得\(\B=\C^T\A\C\),
则称“\(\A\)与\(\B\) \DefineConcept{合同}(congruent)”%
或“\(\B\)是\(\A\)的\DefineConcept{合同矩阵}”,
记为\(\A\simeq\B\).
\end{definition}

\begin{property}
矩阵的合同关系是等价关系,即具备下列三条性质:
\begin{enumerate}
\item 反身性,\(\A\simeq\A\);
\item 对称性,\(\A\simeq\B \implies \B\simeq\A\);
\item 传递性,\(\A\simeq\B \land \B\simeq\C \implies \A\simeq\C\).
\end{enumerate}
\end{property}

\begin{property}
合同矩阵与原矩阵等价,即\(\A\simeq\B \implies \A\cong\B\).
\begin{proof}
由\hyperref[definition:逆矩阵.矩阵等价]{矩阵等价的定义}显然有.
\end{proof}
\end{property}

\begin{property}
合同矩阵的秩与原矩阵相等,即\(\A\simeq\B \implies \rank\A=\rank\B\).
\begin{proof}
由\cref{theorem:线性方程组.初等变换不变秩} 立得.
\end{proof}
\end{property}

\begin{property}
对称矩阵的合同矩阵也是对称的,即\(\A^T = \A, \A\simeq\B \implies \B^T = \B\).
\begin{proof}
设可逆矩阵\(\C\)满足\(\C^T\A\C=\B\),那么%
\(\B^T = (\C^T\A\C)^T = \C^T\A^T\C = \C^T\A\C = \B\).
\end{proof}
\end{property}

\begin{property}
矩阵相似是矩阵合同的充分不必要条件,即\[
\A\sim\B
\implies
\A\cong\B.
\]
\end{property}

\begin{example}\label{example:二次型.实对称矩阵相似必合同}
实对称矩阵\(\A\)、\(\B\)相似,证明:\(\A\)与\(\B\)合同.
\begin{proof}
因为\(\A\)、\(\B\)都是实对称矩阵,\(\A^T=\A\),\(\B^T=\B\),且存在正交矩阵\(\Q_1,\Q_2\)使得\[
\Q_1^{-1} \A \Q_1 = \Q_1^T \A \Q_1 = \V_1,
\]\[
\Q_2^{-1} \B \Q_2 = \Q_2^T \B \Q_2 = \V_2,
\]其中\(\V_1,\V_2\)是对角阵.
又因为\(\A\sim\B\),所以\(\A\)与\(\B\)有相同的特征多项式、特征值,即\(\V_1=\V_2\),或\[
\Q_1^{-1} \A \Q_1 = \Q_2^{-1} \B \Q_2,
\]\[
(\Q_2 \Q_1^{-1}) \A (\Q_1 \Q_2^{-1}) = \B.
\]令\(\P = \Q_1 \Q_2^{-1}\),\(\P^T = (\Q_1 \Q_2^{-1})^T = (\Q_2^{-1})^T \Q_1^T = \Q_2 \Q_1^{-1} = (\Q_1 \Q_2^{-1})^{-1} = \P^{-1}\),那么\[
\P^T \A \P = \B,
\]也就是说\(\A\)与\(\B\)合同.
\end{proof}
\end{example}

\begin{example}
尽管在\cref{example:二次型.实对称矩阵相似必合同} 中我们看到“两个相似的实对称矩阵必定合同”,但两个合同的实对称矩阵未必相似,我们立即可以拿出一个反例:
\def\diagx(#1){\begin{bmatrix} #1 \\ & #1 \\ && #1 \end{bmatrix}}%
因为\[
\diagx(3^{-1/2})^T \diagx(3) \diagx(3^{-1/2}) = \diagx(1),
\]即\(\A = \diag(3,3,3) \cong \E\),但\(\tr\A = 9 \neq \tr\E = 3\),也就是说两个矩阵并不相似.
\end{example}

\begin{example}\label{example:二次型.合同矩阵符号相同}
设\(\A\)、\(\B\)、\(\C\)都是可逆矩阵,且满足\(\C^T\A\C=\B\),
则\(\abs{\A}\)与\(\abs{\B}\)的符号相同.
\begin{proof}
因为\(\B=\C^T\A\C\),
所以\(\A\B=\A\C^T\A\C\),
进而有\[
\abs{\A}\abs{\B}
=\abs{\A\B}
=\abs{\A\C^T\A\C}
=\abs{\A}\abs{\C^T}\abs{\A}\abs{\C}
=\abs{\A}^2\abs{\C}^2.
\]
又因为\(\abs{\A} \neq 0\),\(\abs{\A}^2 > 0\),\(\abs{\C} \neq 0\),\(\abs{\C}^2 > 0\),所以\(\abs{\A}\abs{\B} > 0\),即\(\abs{\A}\)与\(\abs{\B}\)同号.
\end{proof}
\end{example}
由\cref{example:二次型.合同矩阵符号相同} 可知,若矩阵\(\A,\B\)合同,则\(\sgn\abs{\A}=\sgn\abs{\B}\).

\begin{example}
设\(\AutoTuple{i}{n}\)是\(1,2,\dotsc,n\)的一个排列,
\[
\A=\diag(\AutoTuple{d}{n}),
\qquad
\B=\diag(d_{i_1},d_{i_2},\dotsc,d_{i_n}).
\]
证明:矩阵\(\A,\B\)合同且相似.
\begin{proof}
显然\(\diag(d_{i_1},d_{i_2},\dotsc,d_{i_n})\)可由\(\diag(\AutoTuple{d}{n})\)同时左乘和右乘若干个初等矩阵\[
\P(i,j) \quad(1 \leq i < j \leq n)
\]得到,又因为\(\P(i,j)^T = \P(i,j)^{-1} = \P(i,j)\),所以只要令这些初等矩阵的乘积为\(\P\),就有\[
\P^{-1} \A \P = \B,
\qquad
\P^T \A \P = \B.
\]也就是说\(\A\simeq\B\land\A\sim\B\).
\end{proof}
\end{example}

根据\cref{theorem:特征值与特征向量.实对称矩阵3},任一实对称矩阵都与某个对角阵合同且相似,实际上,我们可以把这个结论推广到复数域上的对称矩阵.
\begin{theorem}
任一对称矩阵都合同于某个对角矩阵.
\end{theorem}
