\section{初等矩阵}
\subsection{初等变换}
\begin{definition}
对矩阵施行以下变换,称为矩阵的\DefineConcept{初等行变换}(elementary row operation):
\begin{enumerate}
	\item 互换两行的位置;
	\item 用一非零数\(c\)乘以某行;
	\item 将某行的\(k\)倍加到另一行.
\end{enumerate}
类似地,可以定义矩阵的\DefineConcept{初等列变换}(elementary column operation):
\begin{enumerate}
	\item 互换两列的位置;
	\item 用一非零数\(c\)乘以某列;
	\item 将某列的\(k\)倍加到另一列.
\end{enumerate}
矩阵的初等行变换、初等列变换统称为矩阵的\DefineConcept{初等变换}(elementary operation).
\end{definition}

%我们约定:
%矩阵\(\A\)经过一次初等行变换\(\sigma_1\)化为矩阵\(\B\)的过程
%可以表示为在连接矩阵\(\A\)和\(\B\)的箭头上方标记\(\sigma_1\),即\[
%	\A \xlongrightarrow{\sigma_1} \B;
%\]而矩阵\(\A\)经过一次初等列变换\(\sigma_2\)化为矩阵\(\B\)的过程
%可以表示为在连接矩阵\(\A\)和\(\B\)的箭头下方标记\(\sigma_2\),即\[
%	\A \xlongrightarrow[\sigma_2]{} \B.
%\]

\subsection{初等矩阵的概念}
\begin{definition}\label{definition:逆矩阵.矩阵等价}
若矩阵\(\A\)可以经过一系列初等变换化为矩阵\(\B\),
则称“\(\A\)与\(\B\) \DefineConcept{等价}(equivalent)”,
或“\(\A\)与\(\B\) \DefineConcept{相抵}”,
记作\(\A\cong\B\).
\end{definition}

\begin{definition}
由\(n\)阶单位矩阵\(\E\)经过\emph{一次}初等变换所得矩阵称为\(n\)阶\DefineConcept{初等矩阵}(elementary matrix).
\end{definition}

对应于矩阵的三类初等变换,有三种类型的初等矩阵:
\begin{enumerate}
	\item 互换\(\E\)的\(i\),\(j\)两行(列)所得的矩阵\[
		\P(i,j) = \begin{bmatrix}
			\E_{i-1} & & & \\
			& 0 & & 1 & \\
			& & \E_{j-i-1} & & \\
			& 1 & & 0 & \\
			& & & & \E_{n-j}
		\end{bmatrix}_n;
	\]
	\item 用非零数\(c\)乘以\(\E\)的第\(i\)行(列)所得的矩阵\[
		\P(i(c)) = \begin{bmatrix}
			\E_{i-1} & & \\
			& c & \\
			& & \E_{n-i}
		\end{bmatrix}_n;
	\]
	\item 把\(\E\)的第\(j\)行(第\(i\)列)的\(k\)倍加到第\(i\)行(第\(j\)列)所得的矩阵\[
		\P(i,j(k)) = \begin{bmatrix}
			\E_{i-1} & & & \\
			& 1 & & k & \\
			& & \E_{j-i-1} & & \\
			& 0 & & 1 & \\
			& & & & \E_{n-j}
		\end{bmatrix}_n.
	\]
\end{enumerate}

\subsection{初等矩阵的性质}
\begin{property}\label{theorem:逆矩阵.初等矩阵的性质1}
初等矩阵具有以下性质:
\begin{enumerate}
	\item \(\abs{\P(i,j)}=-1\);
	\item \(\abs{\P(i(c))}=c\);
	\item \(\abs{\P(i,j(k))}=1\);
	\item \(\P(i,j)^T=\P(i,j)\);
	\item \(\P(i(c))^T=\P(i(c))\);
	\item \(\P(i,j(k))^T=\P(j,i(k))\);
	\item \(\P(i,j)^{-1}=\P(i,j)\);
	\item \(\P(i(c))^{-1}=\P(i(c^{-1}))\);
	\item \(\P(i,j(k))^{-1}=\P(i,j(-k))\).
\end{enumerate}
\end{property}

\begin{property}\label{theorem:逆矩阵.初等矩阵的性质2}
对\(n \times t\)矩阵\(\A\)施行一次初等行变换,相当于用一个相应的\(n\)阶初等矩阵左乘\(\A\);
对\(\A\)施行一次初等列变换,相当于用一个相应的\(t\)阶初等矩阵右乘\(\A\).
\begin{proof}
用\(n\)阶矩阵\(\P(i,j)\)左乘\(\A\),将矩阵\(\A\)作相应分块,有\[
	\P(i,j) \A = \begin{bmatrix}
		\E_{i-1} \\
		& 0 & & 1 \\
		& & \E_{j-i-1} \\
		& 1 & & 0 \\
		& & & & \E_{n-j}
	\end{bmatrix}
	\begin{bmatrix}
		\A_1 \\ \a_i \\ \A_2 \\ \a_j \\ \A_3
	\end{bmatrix}
	= \begin{bmatrix}
		\A_1 \\ \a_j \\ \A_2 \\ \a_i \\ \A_3
	\end{bmatrix},
\]
即\(\A\)交换\(i\)、\(j\)两行.

用\(n\)阶矩阵\(\P(i(c))\)左乘\(\A\),将矩阵\(\A\)作相应分块,有\[
	\P(i(c)) \A = \begin{bmatrix}
		\E_{i-1} \\
		& c \\
		& & \E_{n-i}
	\end{bmatrix}
	\begin{bmatrix}
		\A_1 \\ \a_i \\ \A_2
	\end{bmatrix}
	= \begin{bmatrix}
		\A_1 \\ c \a_i \\ a_2
	\end{bmatrix},
\]
即用一非零数\(c\)乘以第\(i\)行.

用\(n\)阶矩阵\(\P(i,j(k))\ (i < j)\)左乘\(\A\),将矩阵\(\A\)作相应分块,有\[
	\P(i,j(k)) \A = \begin{bmatrix}
		\E_{i-1} \\
		& 1 & & k \\
		& & \E_{j-i-1} \\
		& 0 & & 1 \\
		& & & & \E_{n-j}
	\end{bmatrix}
	\begin{bmatrix}
		\A_1 \\ \a_i \\ \A_2 \\ \a_j \\ \A_3
	\end{bmatrix}
	= \begin{bmatrix}
		\A_1 \\ \a_i + k \a_j \\ \A_2 \\ \a_j \\ \A_3
	\end{bmatrix},
\]
即把\(\A\)第\(j\)行的\(k\)倍加到第\(i\)行.
\end{proof}
\end{property}
