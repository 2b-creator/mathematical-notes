\section{舒尔定理}
\begin{theorem}\label{theorem:逆矩阵.舒尔定理}
设\(\A\)是\(m\)阶可逆矩阵,
\(\B,\C,\D\)分别是\(m \times p, n \times m, n \times p\)矩阵,
则有\begin{gather}
	\begin{bmatrix}
		\E_m & \z \\
		-\C\A^{-1} & \E_n
	\end{bmatrix}
	\begin{bmatrix}
		\A & \B \\
		\C & \D
	\end{bmatrix}
	= \begin{bmatrix}
		\A & \B \\
		\z & \D - \C \A^{-1} \B
	\end{bmatrix},
	\\
	\begin{bmatrix}
		\A & \B \\
		\C & \D
	\end{bmatrix}
	\begin{bmatrix}
		\E_m & -\A^{-1} \B \\
		\z & \E_p
	\end{bmatrix}
	= \begin{bmatrix}
		\A & \z \\
		\C & \D - \C \A^{-1} \B
	\end{bmatrix},
	\\
	\begin{bmatrix}
		\E_m & \z \\
		-\C \A^{-1} & \E_n
	\end{bmatrix}
	\begin{bmatrix}
		\A & \B \\
		\C & \D
	\end{bmatrix}
	\begin{bmatrix}
		\E_m & -\A^{-1} \B \\
		\z & \E_p
	\end{bmatrix}
	= \begin{bmatrix}
		\A & \z \\
		\z & \D - \C \A^{-1} \B
	\end{bmatrix},
\end{gather}
\rm
其中\(\D - \C \A^{-1} \B\)
称为矩阵\(\begin{bmatrix}
	\A & \B \\
	\C & \D
\end{bmatrix}\)
关于\(\A\)的\DefineConcept{舒尔补}(Schur complement).
\end{theorem}
我们把\cref{theorem:逆矩阵.舒尔定理} 称为“舒尔定理”.

可以看到只要\(\A\)可逆,
就能通过初等分块矩阵直接对分块矩阵进行分块对角化,
而在操作过程中就把一些原本不为0的分块矩阵变成了零矩阵,
这个过程可以形象地称为“矩阵打洞”,
即让矩阵出现尽可能多的0.

利用\hyperref[theorem:逆矩阵.舒尔定理]{舒尔定理}可以证明下述行列式降阶定理:
\begin{theorem}[行列式第一降阶定理]\label{theorem:逆矩阵.行列式第一降阶定理}
\def\M{\vb{M}}
设\(\M = \begin{bmatrix}
	\A & \B \\
	\C & \D
\end{bmatrix}\)是方阵.
\begin{enumerate}
	\item 若\(\A\)可逆,则\[
		\abs{\M} = \abs{\A} \abs{\D - \C \A^{-1} \B};
	\]

	\item 若\(\D\)可逆,则\[
		\abs{\M} = \abs{\D} \abs{\A - \B \D^{-1} \C};
	\]

	\item 若\(\A,\D\)均可逆,则\[
		\abs{\A} \abs{\D - \C \A^{-1} \B}
		= \abs{\D} \abs{\A - \B \D^{-1} \C}.
	\]
\end{enumerate}
\end{theorem}

\begin{example}\label{example:逆矩阵.行列式降阶定理的重要应用1}
设\(\A \in M_{s \times n}(K),
\B \in M_{n \times s}(K)\).
证明:\[
	\begin{vmatrix}
		\E_n & \B \\
		\A & \E_s
	\end{vmatrix} = \abs{\E_s - \A\B}.
\]
\begin{proof}
由于\(\E_n,\E_s\)都是单位矩阵,必可逆,
那么根据\cref{theorem:逆矩阵.行列式第一降阶定理},有\[
	\begin{vmatrix}
		\E_n & \B \\
		\A & \E_s
	\end{vmatrix}
	= \abs{\E_n} \abs{\E_s - \A(\E_n)^{-1}\B}
	= \abs{\E_s - \A\B}.
	\qedhere
\]
\end{proof}
\end{example}

\begin{example}
\def\M{\vb{M}}
求解行列式\(\det \M\),其中\[
	\M = \begin{bmatrix}
		1+a_1 b_1 & a_1 b_2 & \dots & a_1 b_n \\
		a_2 b_1 & 1+a_2 b_2 & \dots & a_2 b_n \\
		\vdots & \vdots & & \vdots \\
		a_n b_1 & a_n b_2 & \dots & 1+a_n b_n
	\end{bmatrix}.
\]
\begin{solution}
记\(\a = (\AutoTuple{a}{n})^T,
\b = (\AutoTuple{b}{n})\),
则\(\M = \E_n + \a\b\).

注意到\(\M = \E_n + \a\b\)形似某个矩阵的舒尔补,因此考虑下面的矩阵:\[
	\A = \begin{bmatrix}
		1 & -\b \\
		\a & \E_n
	\end{bmatrix}.
\]由于\[
	\begin{bmatrix}
		1 & \z \\
		-\a & \E_n
	\end{bmatrix} \A
	= \begin{bmatrix}
		1 & \z \\
		-\a & \E_n
	\end{bmatrix}
	\begin{bmatrix}
		1 & -\b \\
		\a & \E_n
	\end{bmatrix}
	= \begin{bmatrix}
		1 & -\b \\
		\z & \M
	\end{bmatrix},
\]
且\[
	\begin{bmatrix}
		1 & \b \\
		\z & \E_n
	\end{bmatrix} \A
	= \begin{bmatrix}
		1 & \b \\
		\z & \E_n
	\end{bmatrix}
	\begin{bmatrix}
		1 & -\b \\
		\a & \E_n
	\end{bmatrix}
	= \begin{bmatrix}
		1+\a\b & \z \\
		\a & \E_n
	\end{bmatrix},
\]
所以\[
	\abs{\A}
	= \abs{\M}
	= \abs{1+\a\b}
	= 1+\a\b
	= 1 + \sum\limits_{k=1}^n a_k b_k.
\]
\end{solution}
\end{example}
