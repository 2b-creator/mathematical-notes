\chapter{线性空间与线性变换}
\section{线性空间}
\subsection{线性空间的概念与性质}
\begin{definition}
设\(V\)是一个非空集合,\(P\)是一个数域,定义两种运算:
\begin{enumerate}
\item {\bf 加法}:\(\forall \a,\b \in V\),\(V\)中有唯一的元素\(\mat{\delta}\)与之对应,称为\(\a\)与\(\b\)之和,记作\(\a+\b\);
\item {\bf 数乘}:\(\forall k \in P\),\(\forall \a \in V\),\(V\)中有唯一的元素与之对应,记作\(k\a\).
\end{enumerate}

如果\emph{加法}与\emph{数乘}满足以下八条公理,即\(\forall \a,\b,\g \in V\),\(\forall k,l \in P\),有

\begin{center}
\begin{minipage}{.8\textwidth}
\begin{axiom}
\(\a+\b=\b+\a\).
\end{axiom}
\begin{axiom}
\((\a+\b)+\g=\a+(\b+\g)\).
\end{axiom}
\begin{axiom}
\(\exists \z \in V (\forall\a \in V \implies \a+\z=\a)\),\(\z\)称为\textbf{零元素}.
\end{axiom}
\begin{axiom}
\(\forall\a \in V, \exists \mat{\eta} \in V (\a+\mat{\eta}=\z)\),\(\mat{\eta}\)称为\(\a\)的\textbf{负元素},记作\(\mat{\eta} = -\a\).
\end{axiom}
\begin{axiom}
\(1\a=\a\).
\end{axiom}
\begin{axiom}
\(k(l\a)=(kl)\a\).
\end{axiom}
\begin{axiom}
\(k(\a+\b)=k\a+k\b\).
\end{axiom}
\end{minipage}
\end{center}

则称\(V\)为数域\(P\)上的\textbf{线性空间}(linear space).
\(V\)中的元素称为\textbf{向量}(vector)\nolinebreak.

特别地,当\(P = \mathbb{R}\)时,称\(V\)为\textbf{实线性空间};
当\(P = \mathbb{C}\)时,称\(V\)为\textbf{复线性空间}.
\end{definition}

\begin{example}
下面列举一些常见的线性空间.
实线性空间与复线性空间是代数结构完全不同的两个线性空间.
集合\(\mathbb{R}^{n \times 1}\)关于向量的加法、实数与向量的数乘构成实线性空间.
集合\(\mathbb{R}^{s \times n}\)关于矩阵的加法、实数与矩阵的数乘构成实线性空间.
全体实系数多项式关于多项式的加法、实数与多项式的乘法(数乘)构成实线性空间.
\end{example}

\begin{example}
设\(\A \in P^{s \times n}\),齐次线性方程组\(\A\x=\z\)的解集\(W\)关于向量加法及数乘构成数域\(P\)上的线性空间,所以\(W\)又称为\(\A\x=\z\)的\textbf{解空间}.特别地,当\(r_{\A}=n\)时,\(\A\x=\z\)只有零解,解空间\(W\)只有零向量.只含零向量的空间称为\textbf{零空间},记为\(\{\z\}\).
\end{example}

\begin{definition}
设\(V\)是数域\(P\)上的线性空间,\(W \subseteq V\)是一个非空集合.
如果\(W\)关于\(V\)中的加法及数乘运算也构成数域\(P\)上的线性空间,则称\(W\)是\(V\)的一个\DefineConcept{子空间}(subspace).
\end{definition}

\begin{theorem}\label{theorem:线性空间.子空间的判定}
设\(W\)是线性空间\(V\)的非空子集.
如果\(W\)关于\(V\)的加法与数乘运算封闭,即\[
\forall \a,\b \in W,
\forall k \in P
\bigl(
\a+\b, k \a \in W
\bigr),
\]则称\(W\)是\(V\)的子空间.
\end{theorem}

\begin{example}
设\(V\)是数域\(P\)上的线性空间.
在线性空间\(V\)中取定\(s\)个向量\[
\v{\a}{s}
\]组成向量组\(A\).证明:集合\[
W = \Set{ k_1 \a_1 + k_2 \a_2 + \dotsb + k_s \a_s \given k_i \in P, i=1,2,\dotsc,s }
\]是\(V\)的子空间.
\begin{proof}
首先\(W\)是\(V\)的非空子集.其次\(\forall \a,\b \in W\)有\[
\a = k_1 \a_1 + k_2 \a_2 + \dotsb + k_s \a_s,
\qquad
\b = p_1 \a_1 + p_2 \a_2 + \dotsb + p_s \a_s,
\]故\[
\a+\b = (k_1+p_1)\a_1 + (k_2+p_2)\a_2 + \dotsb + (k_s+p_s)\a_s \in W;
\]同理可证\(\forall \a \in W, \forall k \in P\)有\(k\a \in W\).
由\cref{theorem:线性空间.子空间的判定},\(W\)是\(V\)的子空间.
\end{proof}
集合\(W\)称为\(A\)的\DefineConcept{生成空间}(spanning space),记作\(L(\v{\a}{s})\).
\end{example}

\subsection{线性空间的基与维数}
\begin{definition}
\def\B{\mathcal{B}}%
设\(V\)是数域\(P\)上的线性空间,如果\begin{enumerate}
\item \(\e_1,\e_2,\dotsc,\e_n \in V\);
\item 向量组\(\B = \{ \e_1,\e_2,\dotsc,\e_n \}\)线性无关;
\item 在\(V\)中任取一个向量\(\a\),\(\a\)总可由向量组\(\B\)线性表出,即\[
\a = k_1 \e_1 + k_2 \e_2 + \dotsb + k_n \e_n,
\]
\end{enumerate}
则称\(\B\)是\(V\)的一个\DefineConcept{基}(basis).
称系数\(\v{k}{n}\)为\(\a\)在基\(\B\)下的\DefineConcept{坐标}(coordinate).
称整数\(n\)为\(V\)的\DefineConcept{维数},记作\(\dim V = n\).
\end{definition}

\begin{definition}
\def\B{\mathcal{B}}%
\def\Ba{\B_{\alpha}}%
\def\Bb{\B_{\beta}}%
设\[
\Ba = \{ \v{\a}{n} \}
\quad\text{和}\quad
\Bb = \{ \v{\b}{n} \}
\]是\(V^n\)的两组基.

显然,对基\(\Bb\)中的每个向量\(\b_1\),可以求出其在基\(\Ba\)下的坐标:\[
\b_i = \Ba \P_i \quad(i=1,2,\dotsc,n),
\]其中\(\P_i = (p_{i1},p_{i2},\dotsc,p_{in})^T \in F^n\ (i=1,2,\dotsc,n)\).

若矩阵\(\P = (p_{ij})_n = (\P_1,\P_2,\dotsc,\P_n)\)满足\[
(\v{\b}{n}) = (\v{\a}{n}) \P,
\]则称矩阵\(\P\)是基\(\Ba\)到基\(\Bb\)的\DefineConcept{过渡矩阵}(或\DefineConcept{变换矩阵}).
\end{definition}

\begin{example}
设\(\a_1,\a_2,\a_3\)是\(\mathbb{R}^3\)的一组基,求:基\(\a_1,\frac{1}{2}\a_2,\frac{1}{3}\a_3\)到基\(\a_1+\a_2,\a_2+\a_3,\a_3+\a_1\)的过渡矩阵.
\begin{solution}
设所求过渡矩阵为\(\P\),则根据定义有\[
\begin{bmatrix}
\a_1 & \frac{1}{2}\a_2 & \frac{1}{3}\a_3
\end{bmatrix} \P
= \begin{bmatrix}
\a_1+\a_2 & \a_2+\a_3 & \a_3+\a_1
\end{bmatrix},
\]即\[
\begin{bmatrix}
\a_1 & \a_2 & \a_3
\end{bmatrix} \begin{bmatrix}
1 \\
& \frac{1}{2} \\
&& \frac{1}{3}
\end{bmatrix} \P
= \begin{bmatrix}
\a_1 & \a_2 & \a_3
\end{bmatrix} \begin{bmatrix}
1 & 0 & 1 \\
1 & 1 & 0 \\
0 & 1 & 1
\end{bmatrix},
\]所以\[
\P = \begin{bmatrix}
1 \\
& \frac{1}{2} \\
&& \frac{1}{3}
\end{bmatrix}^{-1} \begin{bmatrix}
1 & 0 & 1 \\
1 & 1 & 0 \\
0 & 1 & 1
\end{bmatrix}
= \begin{bmatrix}
1 \\
& 2 \\
&& 3
\end{bmatrix} \begin{bmatrix}
1 & 0 & 1 \\
1 & 1 & 0 \\
0 & 1 & 1
\end{bmatrix}
= \begin{bmatrix}
1 & 0 & 1 \\
2 & 2 & 0 \\
0 & 3 & 3
\end{bmatrix}.
\]
\end{solution}
\end{example}
