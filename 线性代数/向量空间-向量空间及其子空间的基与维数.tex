\section{向量空间及其子空间的基与维数}
\begin{definition}
设\(U\)是\(K^n\)的一个子空间.
如果\(\v{\a}{r} \in U\),并且满足\begin{enumerate}
	\item \(\v{\a}{r}\)线性无关,%
	\item \(U\)中的每一个向量都可以由\(\v{\a}{r}\)线性表出,%
\end{enumerate}
那么称\(\v{\a}{r}\)是\(U\)的一个\DefineConcept{基}.
\end{definition}
由于基\(\v{\a}{r}\)线性无关,因此如果\(\a\)可以由\(\v{\a}{r}\)线性表出,那么表法唯一.

\begin{example}
试证:任意\(n\)维向量\(\a = (\v{k}{n})\)是向量组\(\e_1 = (1,0,\dotsc,0)\),\(\e_2 = (0,1,\dotsc,0)\),\(\e_n = (0,0,\dotsc,1)\)的一个线性组合.
\begin{proof}
由向量的线性运算规律易知,\[
(\v{k}{n})
= k_1 (1,0,\dotsc,0)
+ k_2 (0,1,\dotsc,0)
+ \dotsb
+ k_n (0,0,\dotsc,1),
\]即\[
\a = \sum\limits_{i=1}^n k_i \e_i.
\qedhere
\]
\end{proof}
\end{example}

\begin{property}
基本向量组\(\v{\e}{n}\)是\(K^n\)的一个基,称之为\(K^n\)的\DefineConcept{标准基}.
\end{property}

\begin{theorem}\label{theorem:线性方程组.向量空间1}
\(K^n\)的任一非零子空间\(U\)都有一个基.
\begin{proof}
取\(U\)中一个非零向量\(\a_1\),向量组\(\a_1\)是线性无关的.
若\(\opair{\a_1} \neq U\),则存在\(\a_2 \in U\),使得\(\a_2 \notin \opair{\a_1}\).
于是\(\a_2\)不能由\(\a_1\)线性表出,因此\(\a_1,\a_2\)线性无关.
若\(\opair{\a_1,\a_2} \neq U\),则存在\(\a_3 \in U\),使得\(\a_3 \notin \opair{\a_1,\a_2}\),从而\(\a_1,\a_2,\a_3\)线性无关.
以此类推,由于\(K^n\)的任一线性无关的向量组的向量个数不超过\(n\),因此到某一步必定终止,且有\(\opair{\v{\a}{s}} = U\),于是\(\v{\a}{s}\)是\(U\)的一个基.
\end{proof}
\end{theorem}
\cref{theorem:线性方程组.向量空间1} 的证明过程也表明,子空间\(U\)的任一线性无关的向量组都可以扩充成\(U\)的一个基.

\begin{theorem}\label{theorem:线性方程组.向量空间2}
\(K^n\)的非零子空间\(U\)的任意两个基所含向量的个数相等.
\begin{proof}
等价的线性无关的向量组含有相同个数的向量.
\end{proof}
\end{theorem}

\begin{definition}
\(K^n\)的非零子空间\(U\)的一个基所含向量的个数称为\(U\)的\DefineConcept{维数},记作\(\dim_K U\),或简记为\(\dim U\).

特别地,规定零空间的维数为0.
\end{definition}

\begin{property}
\(\dim K^n = n\).
\begin{proof}
基本向量组\(\v{\e}{n}\)是\(K^n\)的一个基.
\end{proof}
\end{property}

设\(\v{\a}{r}\)是\(K^n\)的子空间\(U\)的一个基,则\(U\)的每一个向量\(\a\)都可以由\(\v{\a}{r}\)唯一地线性表出:\[
\a = x_1 \a_1 + x_2 \a_2 + \dotsb + x_r \a_r.
\]把元组\(\opair{x_1,x_2,\dotsc,x_r}\)称为\(\a\)在基\(\v{\a}{r}\)下的\DefineConcept{坐标}.

\begin{example}
设\(\dim U = r\),证明:\(U\)中任意\(r+1\)个向量都线性相关.
\end{example}

\begin{example}
设\(\dim U = r\),证明:\(U\)中任意\(r\)个线性无关的向量都是\(U\)的一个基.
\end{example}

\begin{example}
设\(\dim U = r\),\(\v{\a}{r} \in U\).
证明:如果\(U\)中每一个向量都可以由\(\v{\a}{r}\)线性表出,那么\(\v{\a}{r}\)是\(U\)的一个基.
\end{example}

\begin{example}
设\(U\)和\(W\)都是\(K^n\)的非零子空间.
证明:如果\(U \subseteq W\),那么\(\dim U \leqslant \dim W\).
\end{example}

\begin{example}
设\(U\)和\(W\)都是\(K^n\)的非零子空间,且\(U \subseteq W\).
证明:如果\(\dim U = \dim W\),那么\(U = W\).
\end{example}

\begin{theorem}
向量组\(\v{\a}{s}\)的一个极大无关组是这个向量组生成的子空间\(\opair{\v{\a}{s}}\)的一个基,从而\begin{equation}\label{equation:线性方程组.子空间的维数与向量组的秩的联系}
\dim\opair{\v{\a}{s}} = \rank\{\v{\a}{s}\}.
\end{equation}
\end{theorem}
这里要注意区分“子空间的维数\(\dim\opair{\v{\a}{s}}\)”
和“向量组的秩\(\rank\{\v{\a}{s}\}\)”这两个概念:
维数是对子空间而言,秩是对向量组而言;
在子空间\(\opair{\v{\a}{s}}\)这个集合中有无穷多个向量,
而向量组\(\{\v{\a}{s}\}\)这个集合中只有有限的\(s\)个向量.

数域\(K\)上任一\(s \times n\)矩阵\(\A\)的列向量组\(\v{\a}{n}\)%
生成的子空间称为\(\A\)的\DefineConcept{列空间};
\(\A\)的行向量组\(\v{\g}{s}\)生成的子空间称为\(\A\)的\DefineConcept{行空间}.
易知,\(\A\)的列(行)空间的维数等于\(\A\)的列(行)向量组的秩.
