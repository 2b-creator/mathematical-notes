\section{非齐次线性方程组的解集的结构}
\begin{definition}
与非齐次线性方程组\(\A\x=\b\)的系数矩阵相同的齐次线性方程组\(\A\x=\z\)称为前者的\DefineConcept{导出组}.
\end{definition}

\begin{theorem}\label{theorem:线性方程组.非齐次线性方程组有解的条件及解的结构}
已知数域\(K\),矩阵\(\A \in P^{s \times n}\).

对于线性方程组\(\A\x=\b\),其增广矩阵记为\(\widetilde{\A} = (\A,\b)\),则
\begin{enumerate}
\item \(\A\x=\b\ \text{有解} \iff \rank\A=\rank\widetilde{\A}\);

\item \(\rank\A=\rank\widetilde{\A}=n\)时,非齐次线性方程组有唯一解;

\item \(\rank\A=\rank\widetilde{\A}<n\)时,非齐次线性方程组有无穷多解.
设\[
\x_1,\x_2,\dotsc,\x_{n-r}
\]是导出组\(\A\x=\z\)的一个基础解系,那么通解为\[
\x_0+k_1\x_1+k_2\x_2+ \dotsb +k_{n-r}\x_{n-r},
\quad
k_1,k_2,\dotsc,k_{n-r} \in P,
\]其中\(\x_0\)是非齐次线性方程组的一个解,称为\DefineConcept{特解}.
\end{enumerate}
\end{theorem}

一般地,对\(\A\x=\b\)的增广矩阵\[
\widetilde{\A}=(\A,\b)
\]施行一系列初等行变换化为阶梯形矩阵\[
\widetilde{\B}=(\B,\g),
\]则\(\A\x=\b\)与\(\B\x=\g\)同解,\(\A\x=\z\)与\(\B\x=\z\)同解.

于是\(\A\x=\b\)有解的充要条件是\(\rank\B=\rank\widetilde{\B}\).

有解时,求出\(\B\x=\g\)的一个特解和\(\B\x=\z\)的一个基础解系,就可得出\(\A\x=\b\)的通解.

\begin{example}
求线性方程组\[
\left\{ \begin{array}{*{9}{r}}
x_1 &-& 2 x_2 &-& x_3 &+& 2 x_4 &=& 4 \\
2 x_1 &-& 2 x_2 &-& 3 x_3 && &=& 2 \\
4 x_1 &-& 2 x_2 &-& 7 x_3 &-& 4 x_4 &=& -2
\end{array} \right.
\]的通解.
\begin{solution}
写出增广矩阵\(\widetilde{\A}\),并作初等行变换化简\begin{align*}
\widetilde{\A}
&= \begin{bmatrix}
1 & -2 & -1 & 2 & 4 \\
2 & -2 & -3 & 0 & 2 \\
4 & -2 & -7 & -4 & -2
\end{bmatrix}
\xlongrightarrow{\begin{array}{c}
	-2\times\text{(1行)}+\text{(2行)} \\
	-4\times\text{(1行)}+\text{(3行)}
\end{array}}
\begin{bmatrix}
1 & -2 & -1 & 2 & 4 \\
0 & 2 & -1 & -4 & -6 \\
0 & 6 & -3 & -12 & -18
\end{bmatrix} \\
&\to \begin{bmatrix}
1 & -2 & -1 & 2 & 4 \\
0 & 2 & -1 & -4 & -6 \\
0 & 0 & 0 & 0 & 0
\end{bmatrix}
\to \begin{bmatrix}
1 & 0 & -2 & -2 & -2 \\
0 & 2 & -2 & -4 & -6 \\
0 & 0 & 0 & 0 & 0
\end{bmatrix}
= \widetilde{\B}
= (\B,\g),
\end{align*}
故\(\rank\A = \rank\widetilde{\A} = 2\),于是原方程组有解.

解同解方程组\[
\left\{ \begin{array}{*{9}{c}}
x_1 && &-& 2 x_3 &-& 2 x_4 &=& -2 \\
&& 2 x_2 &-& x_3 &-& 4 x_4 &=& -6
\end{array} \right..
\]令\(x_3 = 0\),\(x_4 = 0\),解得\(x_1 = -2\),\(x_2 = -3\),得特解\[
\x_0 = (-2,-3,0,0)^T.
\]

又因为导出组的基础解系含\(4 - \rank\A = 2\)个向量.
将\(x_3,x_4\)的两组值\((2,0),(0,1)\)分别代入\[
\left\{ \begin{array}{*{9}{c}}
x_1 && &-& 2 x_3 &-& 2 x_4 &=& 0 \\
&& 2 x_2 &-& x_3 &-& 4 x_4 &=& 0
\end{array} \right.
\]得基础解系\(\x_1 = (4,1,2,0)^T\),\(\x_2 = (2,2,0,1)^T\).

于是原方程的通解为\[
\x = \x_0 + k_1 \x_1 + k_2 \x_2
= \begin{bmatrix} -2 \\ -3 \\ 0 \\ 0 \end{bmatrix}
+ k_1 \begin{bmatrix} 4 \\ 1 \\ 2 \\ 0 \end{bmatrix}
+ k_2 \begin{bmatrix} 2 \\ 2 \\ 0 \\ 1 \end{bmatrix},
\]其中\(k_1,k_2\)是任意常数.
\end{solution}
\end{example}

\begin{example}
写出线性方程组\[
\left\{ \begin{array}{l}
x_1 - x_2 = a_1 \\
x_2 - x_3 = a_2 \\
x_3 - x_4 = a_3 \\
x_4 - x_1 = a_4
\end{array} \right.
\]有解的充要条件,并求解.
\begin{solution}
对增广矩阵\(\widetilde{\A}\)作初等行变换化简\begin{align*}
\widetilde{\A}
&= \begin{bmatrix}
1 & -1 & 0 & 0 & a_1 \\
0 & 1 & -1 & 0 & a_2 \\
0 & 0 & 1 & -1 & a_3 \\
-1 & 0 & 0 & 0 & a_4
\end{bmatrix} \to \begin{bmatrix}
1 & -1 & 0 & 0 & a_1 \\
0 & 1 & -1 & 0 & a_2 \\
0 & 0 & 1 & -1 & a_3 \\
0 & 0 & 0 & 0 & \sum\limits_{i=1}^4 a_i
\end{bmatrix} \\
&\to \begin{bmatrix}
1 & 0 & 0 & -1 & a_1 + a_2 + a_3 \\
0 & 1 & 0 & -1 & a_2 + a_3 \\
0 & 0 & 1 & -1 & a_3 \\
0 & 0 & 0 & 0 & \sum\limits_{i=1}^4 a_i
\end{bmatrix}.
\end{align*}
可见\(\rank\A = 3\).
方程组有解的充要条件是\(\rank\widetilde{\A} = \rank\A = 3\),那么充要条件就是\(\sum\limits_{i=1}^4 a_i = 0\)

当方程组有解时,通解为\[
\begin{bmatrix}
a_1 + a_2 + a_3 \\ a_2 + a_3 \\ a_3 \\ 0
\end{bmatrix} + k \begin{bmatrix}
1 \\ 1 \\ 1 \\ 1
\end{bmatrix},
\]其中\(k\)为任意常数.
\end{solution}
\end{example}

至此,我们讨论了线性方程组的解的存在性、解的性质、解的结构及求解方法,建立起了线性方程组的完整理论.
解线性方程组是线性代数的基本问题之一,现代科学技术方面用到的数学问题也有很多要归结到解线性方程组.

\begin{example}
设\(\X0\)是非齐次线性方程组\(\A\x=\b\)的一个解,\(\AutoTuple{\x}{n-r}\)是其导出组\(\A\x=\z\)的一个基础解系.证明:\(\X0,\AutoTuple{\x}{n-r}\)线性无关.
\begin{proof}
因为\(\AutoTuple{\x}{n-r}\)是其导出组\(\A\x=\z\)的一个基础解系,根据基础解系的定义,显然有\(\AutoTuple{\x}{n-r}\)线性无关.
假设\(\X0,\AutoTuple{\x}{n-r}\)线性相关,那么\(\X0\)可由\(\AutoTuple{\x}{n-r}\)线性表出,即存在数\(k_1,k_2,\dotsc,k_{n-r}\)使得\[
\X0 = k_1 \X1 + k_2 \X2 + \dotsb + k_{n-r} \X{n-r},
\]进而有\begin{align*}
&\A\X0 = \A(k_1 \X1 + k_2 \X2 + \dotsb + k_{n-r} \X{n-r}) \\
&= k_1 \A\X1 + k_2 \A\X2 + \dotsb + k_{n-r} \A\X{n-r}
= \z + \z + \dotsb + \z = \z,
\end{align*}即\(\X0\)是\(\A\x=\z\)的一个解,这与\(\X0\)是\(\A\x=\b\neq\z\)的一个解矛盾,所以\(\X0,\AutoTuple{\x}{n-r}\)线性无关.
\end{proof}
\end{example}

\begin{example}
\def\wA{\widetilde{\A}}
设线性方程组\(\A\x=\b\)的增广矩阵\(\wA = (\A,\b)\)是一个\(n\)阶可逆矩阵,证明:方程组无解.
\begin{proof}
因为\(\wA\)是\(n\)阶方阵,所以\(\A\)是\(n \times (n-1)\)矩阵,从而\(\rank\A \leq \min\{n-1,n\} = n-1\).
又因为\(\wA\)可逆,所以\(\rank\wA = n\).
因为\(\rank\wA = n > n-1 \geq \rank\A\),所以方程组\(\A\x=\b\)无解.
\end{proof}
\end{example}
