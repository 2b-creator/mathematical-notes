\section{向量组的秩}
\subsection{向量组的等价关系}
\begin{definition}
在\(K^n\)中,如果向量组\[
	A=\{\AutoTuple{\a}{s}\}
\]的每个向量都可由向量组\[
	B=\{\AutoTuple{\b}{t}\}
\]线性表出,
则称“\(A\)可由\(B\) \DefineConcept{线性表出}”.

如果\(A\)与\(B\)可以相互线性表出,
则称\(A\)与\(B\) \DefineConcept{等价},
记作\(A \cong B\).
\end{definition}

\begin{theorem}
部分组可由全组线性表出.
\begin{proof}
设数域\(K\)上一个向量组\(A=\{\AutoTuple{\a}{s}\}\),%
从中任取\(t\ (t \leqslant s)\)个向量构成向量组\[
A'=\{ \a_1, \a_2, \dotsc, \a_t \}.
\]欲证部分组可由全组线性表出,即证\(\forall \a_j \in A'\),\(\exists \AutoTuple{k}{j},\dotsc,k_s \in P\),使得\[
\a_j = k_1 \a_1 + k_2 \a_2 + \dotsb + k_j \a_j + \dotsb + k_s \a_s.
\]
显然只需要使\(k_j = 1\),使除了\(k_j\)以外的数都等于0,则上式恒成立.
\end{proof}
\end{theorem}

\begin{theorem}
\(\alpha=\{\AutoTuple{\a}{s}\}\ (s>1)\)%
线性相关的充要条件是:
\(\alpha\)可由某个部分组%
\[\a_1,\dotsc,\a_{i-1},\a_{i+1},\dotsc,\a_s\]
线性表出.
\end{theorem}

\begin{theorem}
设向量组\(A=\{\AutoTuple{\a}{s}\}\)可由%
\(B=\{\AutoTuple{\b}{t}\}\)线性表出;
如果\(s>t\),则\(A\)线性相关.
\begin{proof}
欲证\(A\)线性相关,须找到不全为零的\(s\)个数\(\AutoTuple{k}{s}\)使得\[
	k_1 \a_1 + k_2 \a_2 + \dotsb + k_s \a_s = \z.
\]
因为向量组\(A\)可由\(B\)线性表出,即有\[
	\left\{ \begin{array}{l}
		\a_1 = c_{11} \b_1 + c_{21} \b_2 + \dotsb + c_{t1} \b_t, \\
		\a_2 = c_{12} \b_1 + c_{22} \b_2 + \dotsb + c_{t2} \b_t, \\
		\hdotsfor{1} \\
		\a_s = c_{1s} \b_1 + c_{2s} \b_2 + \dotsb + c_{ts} \b_t.
	\end{array} \right.
\]代入可得\[
	\sum\limits_{j=1}^s k_j \a_j
	=\sum\limits_{j=1}^s k_j \sum\limits_{i=1}^t c_{ij} \b_i
	=\sum\limits_{j=1}^s \sum\limits_{i=1}^t k_j c_{ij} \b_i
	=\sum\limits_{i=1}^t \b_i \sum\limits_{j=1}^s k_j c_{ij}
	=\z.
\]
如此只需证存在不全为零的\(s\)个数\(\AutoTuple{k}{s}\)
使得对于任意\(i=1,2,\dotsc,t\)都有\[
	\sum\limits_{j=1}^s k_j c_{ij} = 0.
\]
而关于\(k_i\ (i=1,2,\dotsc,s)\)的齐次线性方程组
\[
	\left\{ \begin{array}{l}
		c_{11} k_1 + c_{12} k_2 + \dotsb + c_{1s} k_s = 0, \\
		c_{21} k_1 + c_{22} k_2 + \dotsb + c_{2s} k_s = 0, \\
		\hdotsfor{1} \\
		c_{t1} k_1 + c_{t2} k_2 + \dotsb + c_{ts} k_s = 0.
	\end{array} \right.
\]中方程数\(t\)小于未知量个数\(s\),必有非零解.
\end{proof}
\end{theorem}

\begin{corollary}
任意\(n+1\)个\(n\)维向量线性相关.
换言之,向量个数大于维数的向量组线性相关.
\begin{proof}
\(P^n\)中任意\(n+1\)个\(n\)维向量\(\alpha = \{ \v{\a}{n+1} \}\)
可由基本向量组\(\v{\e}{n}\)线性表出,
这两个向量组中的向量个数满足\(n+1 > n\),向量组\(\alpha\)线性相关.
\end{proof}
\end{corollary}

\begin{corollary}
若线性无关向量组\[
\alpha=\{\v{\a}{s}\}
\]可由向量组\[
\beta=\{\b_1,\b_2,\dotsc,\b_t\}
\]线性表出,则\(s \leqslant t\).
\begin{proof}
假设\(s > t\),因为向量组\(\alpha\)可由\(\beta\)线性表出,所以向量组\(\alpha\)线性相关,矛盾,故\(s \leqslant t\).
\end{proof}
\end{corollary}

\begin{corollary}
两个等价的线性无关向量组含有相同的向量个数.
\begin{proof}
设\(A=\{\AutoTuple{\a}{s}\}\)
与\(B=\{\AutoTuple{\b}{t}\}\)
都线性无关,且\(A \cong B\).
因为\(A \cong B\),%
所以\(A\)可由\(B\)线性表出,%
从而\(s \leqslant t\);
同理可得\(t \leqslant s\);
综上所述,\(s = t\).
\end{proof}
\end{corollary}

\begin{example}
在数域\(K\)上,满足\[
\abs{a_{ii}} > \sum\limits_{\substack{1 \leqslant j \leqslant n \\ j \neq i}} \abs{a_{ij}}
\quad (i=1,2,\dotsc,n)
\]的\(n\)阶矩阵\(\A = (a_{ij})_n\)称为\DefineConcept{主对角占优矩阵}.
证明:\(\A\)的列向量组\(\v{\a}{n}\)的秩等于\(n\).
\begin{proof}
假设\(\v{\a}{n}\)线性相关,则在\(K\)中有一组不全为0的数\(\v{k}{n}\),使得\[
k_1 \a_1 + k_2 \a_2 + \dotsb + k_n \a_n = \z.
\]不妨设\(\abs{k_l} = \max\{\abs{k_1},\abs{k_2},\dotsc,\abs{k_n}\}\neq0\).
由\[
k_1 a_{l1} + k_2 a_{l2} + \dotsb + k_l a_{ll} + \dotsb + k_n a_{ln} = 0,
\]可得\[
a_{ll} = -\frac{1}{k_l} (k_1 a_{l1} + \dotsb + k_{l-1} a_{l,l-1} + k_{l+1} a_{l,l+1} + \dotsb + k_n a_{ln})
= - \sum\limits_{\substack{1 \leqslant j \leqslant n \\ j \neq l}} \frac{k_j}{k_l} a_{lj},
\]\[
\abs{a_{ll}} \leqslant \sum\limits_{\substack{1 \leqslant j \leqslant n \\ j \neq l}} \frac{\abs{k_j}}{\abs{k_l}} \abs{a_{lj}}
\leqslant \sum\limits_{\substack{1 \leqslant j \leqslant n \\ j \neq l}} \abs{a_{lj}}.
\]这与已知条件矛盾!
因此\(\v{\a}{n}\)线性无关,\(\rank\{\v{\a}{n}\} = n\).
\end{proof}
\end{example}

\subsection{极大线性无关组的概念}
\begin{definition}
在\(K^n\)中,设\(B\)是\(A\)的一个部分组.
如果\begin{enumerate}
	\item \(B\)线性无关,
	\item \(A\)可由\(B\)线性表出,
\end{enumerate}
则称“\(B\)是\(A\)的一个\DefineConcept{极大线性无关组}(maximally linearly independent subset)”.
%@see: https://mathworld.wolfram.com/MaximallyLinearlyIndependent.html
\end{definition}

\begin{property}
向量组与其极大无关组等价.
\begin{proof}
因为作为部分组,极大无关组可由全组线性表出,
同时根据极大无关组的定义全组可由极大无关组线性表出,
则根据向量组等价的定义可知原向量组与极大无关组等价.
\end{proof}
\end{property}

\begin{corollary}
向量组的任何两个极大无关组等价,且包含相同个数的向量.
\begin{proof}
设\(A=\{\AutoTuple{\a}{r}\}\)
与\(B=\{\AutoTuple{\b}{t}\}\)
是某个向量组的两个极大无关组.
因为向量组的任意向量可由其极大无关组线性表出,%
所以\(A\)可由\(B\)线性表出,%
\(B\)也可由\(A\)线性表出,%
即\(A \cong B\),%
进而这两个等价的线性无关向量组含有相同的向量个数.
\end{proof}
\end{corollary}

\begin{theorem}
在\(P^n\)中,任意向量组的极大无关组的向量个数不大于\(n\)个.
\begin{proof}
根据定义,任意向量组的极大无关组是线性无关的,而向量个数大于维数的向量组总是线性相关,故任意向量组的极大无关组的向量个数总是不大于其维数\(n\)的.
\end{proof}
\end{theorem}

\begin{example}
求向量组\(\{\a\}\)的极大无关组.
\begin{solution}
显然有\[
\Powerset\{\a\} = \{ \emptyset, \{\a\} \},
\]即\(\{\a\}\)的部分组只有\(\emptyset\)和\(\{\a\}\),从而它的极大无关组也只能是这两者中的一个.

当\(\a=\z\)时,\(\{\a\}\)线性相关,不能满足极大无关组的定义,故\(\{\a\}\)的极大无关组是\(\emptyset\).

当\(\a\neq\z\)时,\(\{\a\}\)线性无关,所以\(\{\a\}\)的极大无关组是它本身.
\end{solution}
\end{example}

\subsection{向量组的秩}
\begin{definition}
向量组的极大无关组所含向量的个数,称为向量组的\DefineConcept{秩}(rank).
记\[
A = \{\v{\a}{s}\}
\]的秩为\(\rank A\)或\(r\{\v{\a}{s}\}\).
\end{definition}

\begin{property}
空集\(\emptyset\)的秩为零,即\(\rank\emptyset = 0\).
\end{property}

\begin{property}
零向量组的秩为零,即\(r\{\z,\z,\dotsc,\z\}=0\).
\begin{proof}
因为向量组本质是向量的集合,所以\[
\{\z,\z,\dotsc,\z\} = \{\z\},
\]而由上例知道\(\{\z\}\)的极大无关组是\(\emptyset\),故\[
r\{\z,\z,\dotsc,\z\}
= r\{\z\}
= \abs{\emptyset}
= 0.
\qedhere
\]
\end{proof}
\end{property}

\begin{corollary}
设向量组\(A=\{\v{\a}{s}\}\).
如果\(\rank{A}<s\),则向量组\(A\)线性相关;
如果\(\rank{A}=s\),则向量组\(A\)线性无关.
\end{corollary}

\begin{corollary}
设向量组
\(A=\{\AutoTuple{\a}{s}\}\)
可由向量组
\(B=\{\AutoTuple{\b}{t}\}\)
线性表出,%
则\(\rank A \leqslant \rank B\).
\begin{proof}
设\(\rank A = r\),\(\rank B = u\).因为\(A\)可由\(B\)线性表出,即\[
\a_k = \sum\limits_{i=1}^t l_{ki} \b_i,
\quad k=1,2,\dotsc,s.
\]
设\(A'=\{\AutoTuple{\a}{r}\}\)
和\(B'=\{\AutoTuple{\b}{u}\}\)
分别是\(A\)和\(B\)的极大无关组,%
则\(B\)可由\(B'\)线性表出,即\[
\b_i = \sum\limits_{j=1}^u b_{ij} \b_j,
\quad i=1,2,\dotsc,t;
\]所以有\[
\a_k = \sum\limits_{i=1}^t l_{ki} \sum\limits_{j=1}^u b_{ij} \b_j
= \sum\limits_{i=1}^t \sum\limits_{j=1}^u l_{ki} b_{ij} \b_j
= \sum\limits_{j=1}^u \b_j \sum\limits_{i=1}^t l_{ki} b_{ij},
\quad k=1,2,\dotsc,s.
\]

特别地,\(A'\)可由\(B'\)线性表出,%
则有\(r \leqslant u\),即\(\rank A \leqslant \rank B\).
\end{proof}
\end{corollary}

\begin{corollary}
等价向量组的秩相等.秩相等的向量组却不一定等价.
\begin{proof}
设向量组\(A\)与\(B\)等价,则\(A\)可由\(B\)线性表出,那么\(\rank A \leqslant \rank B\);同理可得\(\rank A \geqslant \rank B\),所以\(\rank A = \rank B\).

设\(A=\{ (0,1) \}\),\(B=\{ (1,0) \}\),虽然\(\rank A = \rank B = 1\),但\(A\)与\(B\)不等价.
\end{proof}
\end{corollary}

\begin{example}
证明:在秩为\(r\)的向量组中,任意\(r+1\)个向量必线性相关.
\begin{proof}
设向量组\(\AutoTuple{\a}{s}\)的秩为\(r\).
假设部分组\(\AutoTuple{\a}{r+1}\)线性无关,%
那么\[
\rank\{\AutoTuple{\a}{r+1}\} = r+1.
\]
因为部分组总可由全组线性表出,所以部分组的秩总是小于或等于全组的秩,即\[
r+1 = \rank\{\AutoTuple{\a}{r+1}\} \leqslant \rank\{\AutoTuple{\a}{s}\} = r,
\]矛盾,所以部分组\(\AutoTuple{\a}{r+1}\)一定线性相关.
\end{proof}
\end{example}

\begin{example}
设向量组\(\AutoTuple{\a}{s}\)的秩为\(r\).
如果\(\AutoTuple{\a}{r}\)线性无关,证明:
\(\AutoTuple{\a}{r}\)
是\(\AutoTuple{\a}{s}\)的一个极大无关组.
\begin{proof}
设\[
A=\{\AutoTuple{\a}{s}\},
\qquad
B=\{\AutoTuple{\a}{r}\}.
\]要证\(B\)是\(A\)的一个极大无关组,须证\(A\)的任意向量可由\(B\)线性表出.

\begin{enumerate}
\item 显然地,\(\a_i\ (i=1,2,\dotsc,r)\)可由\(B\)线性表出.

\item 根据上例,在秩为\(r\)的向量组中,任意\(r+1\)个向量必线性相关,那么向量组\[
A_i = \{\AutoTuple{\a}{r},\a_i\}\quad(i=r+1,\dotsc,s)
\]必线性相关.

又因为\(B\)线性无关,所以\(\a_i\ (i=r+1,\dotsc,s)\)可由\(B\)线性表出.
\end{enumerate}

综上所述,\(A\)的任意向量可由\(B\)线性表出,且\(B\)线性无关,根据极大无关组的定义,\(B\)是\(A\)的一个极大无关组.
\end{proof}
\end{example}

\begin{example}
向量组
\(\AutoTuple{\a}{r+1}\)
与部分组
\(\AutoTuple{\a}{r}\)
的秩相等.
证明:\(\a_{r+1}\)可由
\(\AutoTuple{\a}{r}\)
线性表出.
\begin{proof}
记\(A=\{\AutoTuple{\a}{r+1}\}\),
\(B=\{\AutoTuple{\a}{r}\}\).
设\(B\)的极大无关组为
\[
B'=\{\a_1,\a_2,\dotsc,\a_t\},
\quad 0 \leqslant t \leqslant r.
\]
由题意有
\(\rank A = \rank B = \rank B' = \abs{B'} = t\).

由上例可知,因为\(\rank A = t\),%
而\(B'\)线性无关,%
所以\(B'\)是\(A\)的一个极大无关组.
那么向量组\(A\)中的向量\(\a_{r+1}\)可以由极大无关组\(B'\)线性表出.
又由于\(B'\)是\(B\)的部分组,故\(B'\)可由\(B\)线性表出.
总而言之,\(A\)可由\(B\)线性表出.
\end{proof}
\end{example}

\subsection{极大无关组的求解}
\begin{theorem}
设矩阵\[
\A=(\AutoTuple{\a}{m})
\]经一系列初等行变换化为矩阵\[
\B=(\AutoTuple{\b}{m}),
\]则\(\a_{j1},\a_{j2},\dotsc,\a_{jk}\)为\(\A\)的列极大无关组的充要条件是:
\(\b_{j1},\b_{j2},\dotsc,\b_{jk}\)为\(\B\)的列极大无关组.
\begin{proof}
矩阵\((\a_{j1},\a_{j2},\dotsc,\a_{jk},\a_l)\)经相同的初等行变换化为
\[
(\b_{j1},\b_{j2},\dotsc,\b_{jk},\b_l) \quad(l=1,2,\dotsc,m).
\]
考虑以下四个向量形式的线性方程组
\begin{gather}
x_1 \a_{j1} + x_2 \a_{j2} + \dotsb + x_k \a_{jk} = \z, \tag1 \\
x_1 \b_{j1} + x_2 \b_{j2} + \dotsb + x_k \b_{jk} = \z, \tag2 \\
y_1 \a_{j1} + y_2 \a_{j2} + \dotsb + y_k \a_{jk} = \a_l, \tag3 \\
y_1 \b_{j1} + y_2 \b_{j2} + \dotsb + y_k \b_{jk} = \b_l, \tag4
\end{gather}
其中(1)与(2)同解,(3)与(4)同解.

必要性.
当\(\a_{j1},\a_{j2},\dotsc,\a_{jk}\)是\(\A\)的列极大无关组时,%
(1)仅有零解,(3)有解.于是(2)仅有零解,(4)有解,%
从而\(\b_{j1},\b_{j2},\dotsc,\b_{jk}\)线性无关,%
\(\b_l\ (l=1,2,\dotsc,m)\)可由其线性表出;
由极大无关组定义,%
\(\b_{j1},\b_{j2},\dotsc,\b_{jk}\)是\(\B\)的列极大无关组.

同理可证充分性.
\end{proof}
\end{theorem}

\begin{example}
求列向量组\[
\a_1 = \begin{bmatrix} -1 \\ 1 \\ 0 \\ 0 \end{bmatrix},
\a_2 = \begin{bmatrix} -1 \\ 2 \\ -1 \\ 1 \end{bmatrix},
\a_3 = \begin{bmatrix} 0 \\ -1 \\ 1 \\ -1 \end{bmatrix},
\a_4 = \begin{bmatrix} 1 \\ -1 \\ 2 \\ 3 \end{bmatrix},
\a_5 = \begin{bmatrix} 2 \\ -6 \\ 4 \\ 1 \end{bmatrix}
\]的秩与一个极大无关组.
\begin{solution}
对矩阵\(\A = (\AutoTuple{\a}{5})\)作初等行变换化为阶梯形矩阵:
\begin{align*}
\A &= \begin{bmatrix}
-1 & -1 & 0 & 1 & 2 \\
1 & 2 & -1 & -3 & -6 \\
0 & -1 & 1 & 2 & 4 \\
0 & 1 & -1 & 3 & 1 \\
\end{bmatrix}
\xlongrightarrow{\begin{array}{c}
(2\text{行}) \addeq 1 \times (1\text{行}) \\
(4\text{行}) \addeq (3\text{行})
\end{array}}
\begin{bmatrix}
-1 & -1 & 0 & 1 & 2 \\
0 & 1 & -1 & -2 & -4 \\
0 & -1 & 1 & 2 & 4 \\
0 & 0 & 0 & 5 & 5 \\
\end{bmatrix} \\
&\xlongrightarrow{\begin{array}{c}
(3\text{行}) \addeq (2\text{行}) \\
(4\text{行}) \diveq 5
\end{array}}
\begin{bmatrix}
-1 & -1 & 0 & 1 & 2 \\
0 & 1 & -1 & -2 & -4 \\
0 & 0 & 0 & 0 & 0 \\
0 & 0 & 0 & 1 & 1 \\
\end{bmatrix} \\
&\xlongrightarrow{\begin{array}{c} \text{交换(3行)与(4行)} \end{array}}
\begin{bmatrix}
-1 & -1 & 0 & 1 & 2 \\
0 & 1 & -1 & -2 & -4 \\
0 & 0 & 0 & 1 & 1 \\
0 & 0 & 0 & 0 & 0 \\
\end{bmatrix} = \B.
\end{align*}
若按列分块有\(\B = (\AutoTuple{\b}{5})\).
阶梯形矩阵\(\B\)有3行不为零,故\[
\rank\{\AutoTuple{\a}{5}\}=3.
\]又因为\(\B\)的非零首元分别位于1、2、4列,%
则\(\b_1,\b_2,\b_4\)是\(\B\)的一个列极大无关组,%
相应地,\(\a_1,\a_2,\a_4\)是\(\A\)的一个列极大无关组,%
即\(\{\AutoTuple{\a}{5}\}\)的极大无关组.
\end{solution}
\end{example}
