\section{向量空间及其子空间的基与维数}
\begin{definition}
%@see: 《高等代数(第三版 上册)》(丘维声) P77. 定义1
设\(U\)是\(K^n\)的一个子空间.
如果\(A=\{\AutoTuple{\a}{r}\}\subseteq U\)满足\begin{enumerate}
	\item \(A\)线性无关,
	\item \(U\)中的每一个向量都可以由\(A\)线性表出,
\end{enumerate}
那么称\(A\)是\(U\)的一个\DefineConcept{基}.
\end{definition}

在\(K^n\)中,基本向量组\(\AutoTuple{\e}{n}\)线性无关,
并且根据\cref{theorem:向量空间.任一向量可由基本向量组唯一线性表出},
每一个向量\(\a=(\AutoTuple{a}{n})^T\)可由基本向量组线性表出,
于是基本向量组是\(K^n\)的一个基,
称之为\(K^n\)的\DefineConcept{标准基}.

\begin{theorem}\label{theorem:线性方程组.向量空间1}
%@see: 《高等代数(第三版 上册)》(丘维声) P77. 定理1
\(K^n\)的任一非零子空间\(U\)都有一个基.
\begin{proof}
因为\(U\neq\{\z\}\),
所以\(U\)中至少有一个非零向量\(\a_1\).
由\cref{example:线性方程组.单向量组线性相关的充要条件} 可知,
向量组\(\{\a_1\}\)是线性无关的.
若\(\opair{\a_1} \neq U\),
则\((\exists \a_2 \in U)[\a_2 \notin \opair{\a_1}]\).
于是\(\a_2\)不能由\(\a_1\)线性表出,
由\cref{theorem:向量空间.增加一个向量对线性相关性的影响2},
\(\{\a_1,\a_2\}\)线性无关.
若\(\opair{\a_1,\a_2} \neq U\),
则\((\exists \a_3 \in U)[\a_3 \notin \opair{\a_1,\a_2}]\).
同理\(\{\a_1,\a_2,\a_3\}\)线性无关.
以此类推,
根据\cref{theorem:向量空间.线性无关向量组的基数不大于可以线性表出它的任意向量组的基数},
由于\(K^n\)的任一线性无关向量组所含向量个数不超过\(n\),
因此上述过程不能无限进行下去,到某一步必定终止.
即将我们得到了\(U\)中一个线性无关向量组\(\{\AutoTuple{\a}{s}\}\)以后,
有\(\opair{\AutoTuple{\a}{s}} = U\),
则\(\{\AutoTuple{\a}{s}\}\)就是\(U\)的一个基.
\end{proof}
\end{theorem}
\cref{theorem:线性方程组.向量空间1} 的证明过程也表明,
从子空间\(U\)的一个非零向量出发,可以扩充成\(U\)的一个基.

\begin{theorem}\label{theorem:线性方程组.向量空间2}
\(K^n\)的非零子空间\(U\)的任意两个基所含向量的个数相等.
\begin{proof}
等价的线性无关的向量组含有相同个数的向量.
\end{proof}
\end{theorem}

\begin{definition}
%@see: 《高等代数(第三版 上册)》(丘维声) P77. 定义2
设\(U\)是\(K^n\)的一个非零子空间.
\(U\)的一个基所含向量的个数称为“\(U\)的\DefineConcept{维数}(dimension)”,
记作\(\dim_K U\),或简记为\(\dim U\).

规定:零子空间的维数等于零,即\(\dim\{\z\}=0\).
\end{definition}

由于基本向量组\(\AutoTuple{\e}{n}\)是\(K^n\)的一个基,
所以\(\dim K^n = n\).
这就是为什么我们把\(K^n\)称为“\(n\)维向量空间”.

在几何空间中,
任意三个不共面的向量是它的一个基,
因此几何空间是三维的空间.
对于过原点的一个平面,它上面不共线的两个向量是它的一个基,
因此这个平面是二维的子空间.
对于过原点的一条直线,
它的一个方向向量是它的一个基,
因此这条直线是一维的子空间.

为了判断线性方程组有没有解,为了研究解集的结构,
我们就必须研究维数更高的向量空间.
我们会发现,对于子空间的结构,基和维数都起到了决定性作用.

\begin{theorem}
%@see: 《高等代数(第三版 上册)》(丘维声) P78. 命题3
设\(U\)是\(K^n\)的一个非零子空间,
\(A=\{\AutoTuple{\a}{r}\}\)是\(U\)的一个基,
那么\(U\)中每一个向量\(\a\)可以由\(A\)线性表出,
并且表出方式是唯一的.
%TODO
\end{theorem}

设\(\AutoTuple{\a}{r}\)是\(K^n\)的子空间\(U\)的一个基,
则\(U\)的每一个向量\(\a\)都可以由\(\AutoTuple{\a}{r}\)唯一地线性表出:\[
	\a = x_1 \a_1 + x_2 \a_2 + \dotsb + x_r \a_r.
\]
把元组\(\opair{x_1,x_2,\dotsc,x_r}\)%
称为“\(\a\)在基\(\AutoTuple{\a}{r}\)下的\DefineConcept{坐标}”.

\begin{theorem}\label{theorem:向量空间.子空间维数.命题4}
%@see: 《高等代数(第三版 上册)》(丘维声) P78. 命题4
设\(U\)是\(K^n\)的\(r\)维子空间,
那么\(U\)中任意\(r+1\)个向量都线性相关.
\begin{proof}
在\(U\)中任取\(r+1\)个向量\(B=\{\AutoTuple{\b}{r+1}\}\).
设\(A=\{\AutoTuple{\a}{r}\}\)是\(U\)的一个基,
则\(B\)可以由\(A\)线性表出.
由于\(r+1>r\),
因此根据\cref{theorem:向量空间.可由比自己基数小的向量组线性表出的向量组线性相关},
\(B\)线性相关.
\end{proof}
\end{theorem}

\begin{theorem}
%@see: 《高等代数(第三版 上册)》(丘维声) P78. 命题5
设\(U\)是\(K^n\)的\(r\)维子空间,
则\(U\)中任意\(r\)个线性无关的向量都是\(U\)的一个基.
\begin{proof}
设\(A=\{\AutoTuple{\a}{r}\}\)是\(U\)中线性无关的向量组.
任意取定\(\b\in U\).
根据\cref{theorem:向量空间.子空间维数.命题4},
向量组\(\{\AutoTuple{\a}{r},\b\}\)必线性相关.
那么由\cref{theorem:向量空间.增加一个向量对线性相关性的影响1},
\(\b\)可以由\(A\)线性表出.
因此\(A\)是\(U\)的一个基.
\end{proof}
\end{theorem}

\begin{example}
设\(\dim U = r\),\(\AutoTuple{\a}{r} \in U\).
证明:如果\(U\)中每一个向量都可以由\(\AutoTuple{\a}{r}\)线性表出,
那么\(\AutoTuple{\a}{r}\)是\(U\)的一个基.
%TODO
\end{example}

\begin{example}
设\(U\)和\(W\)都是\(K^n\)的非零子空间.
证明:如果\(U \subseteq W\),那么\(\dim U \leqslant \dim W\).
%TODO
\end{example}

\begin{example}
设\(U\)和\(W\)都是\(K^n\)的非零子空间,且\(U \subseteq W\).
证明:如果\(\dim U = \dim W\),那么\(U = W\).
%TODO
\end{example}

\begin{theorem}
向量组\(\AutoTuple{\a}{s}\)的一个极大线性无关组是%
这个向量组生成的子空间\(\opair{\AutoTuple{\a}{s}}\)的一个基,
从而
\begin{equation}\label{equation:线性方程组.子空间的维数与向量组的秩的联系}
	\dim\opair{\AutoTuple{\a}{s}} = \rank\{\AutoTuple{\a}{s}\}.
\end{equation}
%TODO
\end{theorem}
这里要注意区分“子空间的维数\(\dim\opair{\AutoTuple{\a}{s}}\)”
和“向量组的秩\(\rank\{\AutoTuple{\a}{s}\}\)”这两个概念:
维数是对子空间而言,秩是对向量组而言;
在子空间\(\opair{\AutoTuple{\a}{s}}\)这个集合中有无穷多个向量,
而向量组\(\{\AutoTuple{\a}{s}\}\)这个集合中只有有限的\(s\)个向量.

数域\(K\)上任一\(s \times n\)矩阵\(\A\)的列向量组\(\AutoTuple{\a}{n}\)%
生成的子空间称为\(\A\)的\DefineConcept{列空间};
\(\A\)的行向量组\(\AutoTuple{\g}{s}\)生成的子空间称为\(\A\)的\DefineConcept{行空间}.
易知,\(\A\)的列(行)空间的维数等于\(\A\)的列(行)向量组的秩.
