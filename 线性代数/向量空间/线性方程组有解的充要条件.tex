\section{线性方程组有解的充要条件}
现在我们可以来回答直接根据线性方程组的系数和常数项判断方程组有没有解,有多少解的问题.

\begin{theorem}\label{theorem:向量空间.线性方程组有解判别定理}
%@see: 《高等代数(第三版 上册)》(丘维声) P87. 定理1
线性方程组\[
	x_1 \a_1 + x_2 \a_2 + \dotsb + x_n \a_n = \b
\]有解的充要条件是:
它的系数矩阵与增广矩阵由相同的秩.
\begin{proof}
记系数矩阵\(\A=(\AutoTuple{\a}{n})\),
增广矩阵\(\wA=(\A,\b)\),
那么
\begin{align*}
	&\hspace{-20pt}
	\text{线性方程组\(x_1 \a_1 + x_2 \a_2 + \dotsb + x_n \a_n = \b\)有解} \\
	&\iff \b\in\opair{\AutoTuple{\a}{n}} \\
	&\iff \opair{\AutoTuple{\a}{n},\b}\subseteq\opair{\AutoTuple{\a}{n}} \\
	&\iff \opair{\AutoTuple{\a}{n},\b}=\opair{\AutoTuple{\a}{n}} \\
	&\iff \dim\opair{\AutoTuple{\a}{n},\b}=\dim\opair{\AutoTuple{\a}{n}} \\
	&\iff \rank\A=\rank\wA.
	\qedhere
\end{align*}
\end{proof}
\end{theorem}

从\cref{theorem:向量空间.线性方程组有解判别定理} 看出,
判断线性方程组有没有解,只要去比较它的系数矩阵与增广矩阵的秩是否相等.
这种判别方法有几种优越之处:
首先,求矩阵的秩有多种方法,不一定要把系数矩阵和增广矩阵化成阶梯形矩阵.
其次,有时不用求出系数矩阵的秩和增广矩阵的秩,也能比较它们的秩是否相等.
由于系数矩阵\(\A\)是增广矩阵\(\wA\)的子矩阵,总有\(\rank\A\leq\rank\wA\),
那么只要能够证明\(\rank\wA\leq\rank\A\),就能得出\(\rank\A=\rank\wA\).

现在我们想知道,当线性方程组\(x_1 \a_1 + x_2 \a_2 + \dotsb + x_n \a_n = \b\)有解时,
能不能用系数矩阵的秩去判别它有唯一解,还是有无穷多个解?

\begin{theorem}\label{theorem:向量空间.线性方程组的解的个数定理}
%@see: 《高等代数(第三版 上册)》(丘维声) P88. 定理2
设线性方程组\[
	x_1 \a_1 + x_2 \a_2 + \dotsb + x_n \a_n = \b
\]有解,\(\A\)是它的系数矩阵.
那么\begin{enumerate}
	\item 如果\(\rank\A=n\),则这个方程组有唯一解.
	\item 如果\(\rank\A<n\),则这个方程组有无穷多个解.
\end{enumerate}
\end{theorem}

把\cref{theorem:向量空间.线性方程组的解的个数定理} 应用到齐次线性方程组上,便得出以下推论.
\begin{corollary}\label{theorem:线性方程组.齐次线性方程组有非零解的充要条件}
%@see: 《高等代数(第三版 上册)》(丘维声) P88. 推论3
%@see: 《线性代数》(张慎语、周厚隆) P80. 定理8
齐次线性方程组有非零解的充要条件是:它的系数矩阵的秩小于未知量的数目.
\end{corollary}
\cref{theorem:线性方程组.齐次线性方程组有非零解的充要条件}
的逆否命题“齐次线性方程组\(\A\x=\z\)只有零解的充要条件为\(\rank\A=n\)”也成立.
