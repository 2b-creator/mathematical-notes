\section{矩阵的秩}
\subsection{矩阵的行秩与列秩}
为了求解向量组的秩,我们可以把向量组看成矩阵的行向量组或列向量组,
利用矩阵的性质,得出这个矩阵的行向量组、列向量组的秩,
最后得到所求向量组的秩.

\begin{definition}\label{definition:线性方程组.行秩与列秩的定义}
%@see: 《线性代数》(张慎语、周厚隆) P76. 定义11(1)
设\(\A\)是向量.
\begin{enumerate}
	\item \(\A\)的行向量组生成的子空间称为
	“\(\A\)的\DefineConcept{行空间}(row space)”,记为\(\Span_R\A\).
	\item \(\A\)的列向量组生成的子空间称为
	“\(\A\)的\DefineConcept{列空间}(column space)”,记为\(\Span_C\A\).
	\item \(\A\)的行向量组的秩称为
	“\(\A\)的\DefineConcept{行秩}(row rank)”,记为\(\rank_R\A\).
	\item \(\A\)的列向量组的秩称为
	“\(\A\)的\DefineConcept{列秩}(column rank)”,记为\(\rank_C\A\).
\end{enumerate}
%@see: https://mathworld.wolfram.com/RowSpace.html
%@see: https://mathworld.wolfram.com/ColumnSpace.html
\end{definition}

矩阵的列秩等于它的列空间的维数,它的行秩等于它的行空间的维数,即\[
	\rank_C\A=\dim(\Span_C\A), \qquad
	\rank_R\A=\dim(\Span_R\A).
\]

\begin{theorem}
设\(\A\)是矩阵,那么\begin{gather}
	\rank_R\A=\rank_C\A^T, \\
	\rank_C\A=\rank_R\A^T.
\end{gather}
\begin{proof}
显然\(\A\)的行向量组就是\(\A^T\)的列向量组,
而\(\A\)的列向量组就是\(\A^T\)的行向量组.
\end{proof}
\end{theorem}

\begin{definition}
设矩阵\(\A = (a_{ij})_{s \times n}\).
\begin{enumerate}
	\item 若\(\rank_R\A = s\),
	则称“\(\A\)为\DefineConcept{行满秩矩阵}(row full rank matrix)”.
	\item 若\(\rank_C\A = n\),
	则称“\(\A\)为\DefineConcept{列满秩矩阵}(column full rank matrix)”.
\end{enumerate}
\end{definition}

现在我们来研究一个问题:矩阵的列秩与行秩之间有什么联系?

\subsection{矩阵的秩}
\begin{definition}\label{definition:线性方程组.矩阵的秩的定义}
%@see: 《线性代数》(张慎语、周厚隆) P76. 定义11(2)
\(\A\)的非零子式的最高阶数称为
“矩阵\(\A\)的\DefineConcept{秩}(rank)”,
记为\(\rank\A\).
%@see: https://mathworld.wolfram.com/MatrixRank.html
\end{definition}

\begin{property}\label{theorem:线性方程组.矩阵的秩的性质1}
零矩阵的秩为零,即\(\rank\z = 0\).
\end{property}

\begin{property}\label{theorem:线性方程组.矩阵的秩的性质2}
设\(\A\)是\(s \times n\)矩阵,则\(0 \leq \rank\A \leq \min\{s,n\}\).
\end{property}

\begin{theorem}
设\(\A\in M_{s \times n}(K)\).
如果\(\A\)有一个\(k\)阶非零子式,那么\(\rank\A\geqslant k\).
\begin{proof}
用反证法.
假设\(\rank\A<k\),
也就是说\(\A\)的非零子式的最高阶数\(r\)小于\(k\),
但是根据前提条件,\(\A\)有一个\(k\)阶非零子式,\(k>r\),
这就和\hyperref[definition:线性方程组.矩阵的秩的定义]{矩阵的秩的定义}矛盾!
因此\(\rank\A\geqslant k\).
\end{proof}
\end{theorem}

\begin{example}
求矩阵\(\A = \begin{bmatrix} 1 & 7 \\ 2 & 6 \\ -3 & 1 \end{bmatrix}\)的秩、行秩与列秩.
\begin{solution}
因为至少存在一个\(\A\)的二阶子式不为零,即\[
	\begin{vmatrix} 1 & 7 \\ 2 & 6 \end{vmatrix} = -8 \neq 0,
\]
且\(\A\)没有三阶子式,所以\(\rank\A = 2\).

写出\(\A\)的行向量组\[
	(1,7), \qquad
	(2,6), \qquad
	(-3,1),
\]
分别转置后,列成一个矩阵(显然就是\(\A^T\)),
作初等行变换化成阶梯形矩阵\[
	\A^T = \begin{bmatrix}
		1 & 2 & -3 \\
		7 & 6 & 1
	\end{bmatrix} \to \begin{bmatrix}
		1 & 0 & 17/2 \\
		0 & 4 & -11
	\end{bmatrix} = \B_1.
\]阶梯形矩阵\(\B_1\)有2行不为零,故矩阵\(\A\)的行秩为2.

写出\(\A\)的列向量组\[
	(1,2,-3)^T,
	(7,6,1)^T,
\]列成一个矩阵(显然就是\(\A\)本身),
作初等行变换化成阶梯形矩阵\[
	\A = \begin{bmatrix} 1 & 7 \\ 2 & 6 \\ -3 & 1 \end{bmatrix}
	\to \begin{bmatrix} 1 & 0 \\ 0 & 1 \\ 0 & 0 \end{bmatrix} = \B_2.
\]
阶梯形矩阵\(\B_2\)有2行不为零,故矩阵\(\A\)的列秩为2.
\end{solution}
\end{example}

\subsection{矩阵的秩、行秩、列秩的关系}
\begin{lemma}\label{theorem:向量空间.矩阵的秩与行秩和列秩的关系.引理}
%@see: 《线性代数》(张慎语、周厚隆) P77. 引理
设矩阵\(\A = (a_{ij})_{s \times n}\)的列秩等于\(\A\)的列数\(n\),
则\(\A\)行秩、秩都等于\(n\).
\begin{proof}
对\(\A\)分别进行列分块与行分块:\[
	\A = (\AutoTuple{\a}{n})
	= (\AutoTuple{\A}{s}[,][T])^T.
\]
根据\cref{theorem:向量空间.秩与线性相关性的关系},
由于\(\rank_C\A=n\),
所以\(\A\)的列向量组线性无关,
也就是说齐次线性方程组\[
	x_1 \a_1 + x_2 \a_2 + \dotsb + x_n \a_n = \z
\]只有零解,或者说\(\A\x=\z\)只有零解.
进而根据\cref{theorem:线性方程组.方程个数少于未知量个数的齐次线性方程组必有非零解},
在\(\A\x=\z\)中,方程个数\(s\)不少于未知量个数\(n\),即\(s \geq n\).

设\(\rank_R\A=t\).
根据\cref{theorem:向量空间.子空间维数.命题4},
因为\(\A\)的行向量的维数是\(n\),
所以\(t \leq n\).

设\(\A\)的行极大线性无关组是
\(\A_{i_1},\A_{i_2},\dotsc,\A_{i_t}\),
其中\(1 \leq i_1 < i_2 < \dotsb < i_t \leq s\).

假设\(t < n\),
则\(\A\)可通过初等行变换化为\[
	\begin{bmatrix}
		\A_{i_1} \\ \A_{i_2} \\ \vdots \\ \A_{i_t} \\ \z_{(s-t) \times n}
	\end{bmatrix},
\]
于是,\(\A\x=\z\)表示的齐次线性方程组中,
非零方程的个数\(t\)小于未知量个数\(n\),
那么根据\cref{theorem:线性方程组.方程个数少于未知量个数的齐次线性方程组必有非零解},
\(\A\x=\z\)有非零解,
这与我们前面得到的结论“\(\A\x=\z\)只有零解”矛盾,
所以假设\(t < n\)不成立,
因此\(t = n\).
这就是说\[
	\rank_R\A=\rank_C\A=n.
\]

既然\(t=n\),
那么\(\A\)的行极大线性无关组恰好可以构成一个方阵,
而这个方阵的行列式为\[
	d=\begin{vmatrix} \A_{i_1} \\ \A_{i_2} \\ \vdots \\ \A_{i_n} \end{vmatrix}.
\]
根据\cref{theorem:线性方程组.克拉默法则},
因为\(\A\x=\z\)只有零解作为它的唯一解,
所以\(d\neq0\).
于是\(d\)是\(\A\)的一个\(n\)阶非零子式.
\(\A\)是一个\(s \times n\)矩阵,
不可能有阶数大于\(n\)的子式,
因此\(\rank\A = n\).
\end{proof}
\end{lemma}

\begin{theorem}\label{theorem:向量空间.矩阵的秩与行秩和列秩的关系.定理}
%@see: 《线性代数》(张慎语、周厚隆) P77. 定理5
矩阵的行秩、列秩、秩都相等.
\begin{proof}
如果矩阵\(\A = \z\),
则\(\rank\A=\rank_R\A=\rank_C\A=0\),
结论成立.

当\(\A\neq\z\)时,
设\(\rank\A=r\),
则根据\hyperref[definition:线性方程组.矩阵的秩的定义]{矩阵的秩的定义},
\(\A\)的所有\(t\ (t > r)\)阶子式全为零;
且\(\A\)有一个\(r\)阶子式不为零,
从而该\(r\)阶子式的列向量组线性无关,
它们的延伸组也线性无关,
\(\A\)有\(r\)列线性无关,
于是\(\rank_C\A=p \geq r\);
由\cref{theorem:向量空间.矩阵的秩与行秩和列秩的关系.引理},
\(\A\)的列极大线性无关组构成的矩阵有一非零\(p\)阶子式,
也是\(\A\)的子式,
故\(p \leq r\);
综上得\(p = r\).
因此\(\rank\A=\rank_C\A\).
同样地,将这一结论用于\(\A^T\),
得\(\rank\A^T=\rank_C\A^T=\rank_R\A\).
\end{proof}
\end{theorem}

\begin{theorem}
%@see: 《高等代数(第三版 上册)》(丘维声) P83. 推论6
设\(\A\)是矩阵,那么\(\rank\A = \rank\A^T\).
\begin{proof}
\(\rank\A=\rank_C\A=\rank_R\A^T=\rank\A^T\).
\end{proof}
\end{theorem}

\begin{theorem}\label{theorem:线性方程组.初等变换不变秩}
%@see: 《高等代数(第三版 上册)》(丘维声) P81. 定理2
%@see: 《高等代数(第三版 上册)》(丘维声) P82. 定理3
%@see: 《高等代数(第三版 上册)》(丘维声) P83. 推论7
初等变换不改变矩阵的秩.
\begin{proof}
由\cref{theorem:向量空间.利用初等行变换求取列极大线性无关组的依据},
初等行变换不改变矩阵的列秩;
同理,初等列变换不改变矩阵的行秩;
再由\cref{theorem:向量空间.矩阵的秩与行秩和列秩的关系.定理},
初等变换不改变矩阵的秩.
\end{proof}
\end{theorem}

\subsection{满秩矩阵}
\begin{definition}
%@see: 《线性代数》(张慎语、周厚隆) P76.
设矩阵\(\A = (a_{ij})_n\).
若\(\rank\A = n\),
则称“\(\A\)是\DefineConcept{满秩矩阵}(full rank matrix)”,
或“\(\A\)是\DefineConcept{非退化矩阵}(non-degenerate matrix)”.
\end{definition}

%应该注意到,“可逆矩阵”“非奇异矩阵”“满秩矩阵”“非退化矩阵”是从不同侧重点对同一类矩阵的四种称谓.

\begin{theorem}\label{theorem:向量空间.满秩方阵的行列式非零}
%@see: 《高等代数(第三版 上册)》(丘维声) P84. 推论9
设\(\A\in M_n(K)\).
\(\A\)是满秩矩阵的充要条件是\(\abs{\A}\neq0\).
\begin{proof}
\(\rank\A=n
\iff \text{\(A\)的非零子式的最高阶数为\(n\)}
\iff \abs{\A}\neq0\).
\end{proof}
\end{theorem}

\begin{corollary}
%@see: 《高等代数(第三版 上册)》(丘维声) P84. 推论10
设\(\A\in M_{s \times n}(K)\)且\(\rank\A=r\).
那么,\(\A\)的\(r\)阶非零子式
所在的列构成\(\A\)的列向量组的一个极大线性无关组,
所在的行构成\(\A\)的行向量组的一个极大线性无关组.
\end{corollary}
