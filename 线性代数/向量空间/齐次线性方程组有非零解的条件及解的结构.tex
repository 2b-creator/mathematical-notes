\section{齐次线性方程组的解集的结构}
\begin{theorem}\label{theorem:线性方程组.齐次线性方程组的解的线性组合也是解}
齐次线性方程组\(\A\x=\z\)的解的任意线性组合也是解.
\begin{proof}
设\(\X1\)与\(\X2\)是齐次线性方程组\(\A\x=\z\)的任意两个解,
即\(\A\X1=\z\),\(\A\X2=\z\).
又设\(k\)是任意常数,那么有\[
	\A (\X1 + \X2) = \A \X1 + \A \X2 = \z + \z = \z,
\]\[
	\A (k \X1) = k (\A \X1) = k \z = \z,
\]
所以\(\X1 + \X2\)与\(k \X1\)都是\(\A \x = \z\)的解.
\end{proof}
\end{theorem}

\begin{definition}\label{definition:线性方程组.齐次线性方程组的解空间}
齐次线性方程组的解集\(V\)关于向量的加法、数乘构成一个线性空间,
称这个线性空间为“齐次线性方程组的\DefineConcept{解空间}(space of solution)”.
\end{definition}

\begin{definition}\label{definition:线性方程组.齐次线性方程组的基础解系}
已知数域\(K\)上的齐次线性方程组\(\A\x=\z\).
如果向量组\[
	X=\Set{\AutoTuple{\x}{t}}
\]满足
\begin{enumerate}
	\item \(X\)的每个向量都是\(\A\x=\z\)的解;
	\item \(X\)线性无关;
	\item \(\A\x=\z\)的任意一个解都可由向量组\(X\)线性表出,
\end{enumerate}
那么称向量组\(X\)为“齐次线性方程组\(\A\x=\z\)的\DefineConcept{基础解系}(basic set of solutions)”.

\def\tongjie{ k_1\x_1+k_2\x_2+ \dotsb +k_t\x_t }
相应地,有齐次线性方程组\(\A\x=\z\)的解空间为\[
	S = \Set{ \tongjie \given \AutoTuple{k}{t} \in P }.
\]
我们把\(\tongjie\)称为“齐次线性方程组\(\A\x=\z\)的\DefineConcept{通解}(general solution)”.
上述基础解系\(X\)又称为“齐次线性方程组\(\A\x=\z\)的解空间的\DefineConcept{基}”.
\end{definition}

\begin{theorem}\label{theorem:线性方程组.齐次线性方程组的解向量个数}
设\(\A\)是\(s \times n\)矩阵,\(\rank\A = r < n\),则齐次线性方程组\(\A\x=\z\)存在基础解系,且基础解系含\(n-r\)个解向量.
\begin{proof}
设\(\A\)经一系列初等行变换化为阶梯形矩阵\(\B\),则\(\rank\B = r\),即\(\B\)的前\(r\)行向量不为零.
不失一般性,设\(\B\)的第\(i\)行非零首元为\(b_{ii}\ (i=1,2,\dotsc,r)\),\[
\A \to \B = \begin{bmatrix}
\B_1 & \B_2 \\
\z & \z
\end{bmatrix},
\]其中\[
\B_1 = \begin{bmatrix}
b_{11} & b_{12} & \dots & b_{1r} \\
& b_{22} & \dots & b_{2r} \\
& & \ddots & \vdots \\
& & & b_{rr}
\end{bmatrix},
\qquad
\B_2 = \begin{bmatrix}
b_{1,r+1} & \dots & b_{1n} \\
b_{2,r+1} & \dots & b_{2n} \\
\vdots & & \vdots \\
b_{r,r+1} & \dots & b_{rn}
\end{bmatrix}.
\]

记\[
\x = (x_1,x_2,\dotsc,x_r,x_{r+1},\dotsc,x_n)^T,
\]将自由未知量\(x_{r+1},x_{r+2},\dotsc,x_n\)的一组值\((1,0,\dotsc,0)\)代入\[
\B \x = \z,
\]去掉\(0 = 0\)的等式,移项得线性方程组\begin{gather}
\begin{bmatrix}
b_{11} & b_{12} & \dots & b_{1r} \\
& b_{22} & \dots & b_{2r} \\
& & \ddots & \vdots \\
& & & b_{rr}
\end{bmatrix}
\begin{bmatrix}
x_1 \\ x_2 \\ \vdots \\ x_r
\end{bmatrix}
=
\begin{bmatrix}
-b_{1,r+1} \\
-b_{2,r+1} \\
\vdots \\
-b_{r,r+1}
\end{bmatrix}
\tag1
\end{gather}系数行列式\(D = b_{11} b_{22} \dotsm b_{rr} \neq 0\).

由克拉默法则,(1)式有唯一解,于是得\(\A\x=\z\)的一个解\[
\X1 = (c_{11},c_{21},\dotsc,c_{r1},1,0,\dotsc,0)^T.
\]

同理,分别将\(x_{r+1},x_{r+2},\dotsc,x_n\)的值\((0,1,\dotsc,0),\dotsc,(0,0,\dotsc,1)\)代入\[
\B \x = \z,
\]求出\(\A\x=\z\)的相应的解\[
\begin{array}{rcl}
\X2 &=& (c_{12},c_{22},\dotsc,c_{r2},0,1,\dotsc,0)^T, \\
&\vdots& \\
\X{n-r} &=& (c_{1,n-r},c_{2,n-r},\dotsc,c_{r,n-r},0,0,\dotsc,1)^T.
\end{array}
\]

易见,\begin{enumerate}
\item \(\AutoTuple{\x}{n-r}\)是\(\A\x=\z\)的解;

\item 考虑向量方程\(k_1 \X1 + k_2 \X2 + \dotsb + k_{n-r} \X{n-r} = \z\),即\[
(l_1,l_2,\dotsc,l_r,k_1,k_2,\dotsc,k_{n-r})^T
= (0,0,\dotsc,0,0,\dotsc,0)^T,
\]有\[
k_1 = k_2 = \dotsb = k_{n-r} = 0,
\]即\(\AutoTuple{\x}{n-r}\)线性无关;

\item 设\(\x = (c_1,c_2,\dotsc,c_r,k_1,k_2,\dotsc,k_{n-r})^T\)是方程组\(\A\x=\z\)的任意一个解,则\[
\x - (k_1 \X1 + k_2 \X2 + \dotsb + k_{n-r} \X{n-r})
= (d_1,d_2,\dotsc,d_r,0,0,\dotsc,0)^T
\]是齐次方程组的解,代入\(\B\x=\z\),去掉\(0 = 0\)的等式,得\[
\begin{bmatrix}
b_{11} & b_{12} & \dots & b_{1r} \\
& b_{22} & \dots & b_{2r} \\
& & \ddots & \vdots \\
& & & b_{rr}
\end{bmatrix}
\begin{bmatrix}
d_1 \\ d_2 \\ \vdots \\ d_r
\end{bmatrix} = \begin{bmatrix}
0 \\ 0 \\ \vdots \\ 0
\end{bmatrix}.
\]因为系数行列式\(\abs{\B_1} \neq 0\),所以\(d_1 = d_2 = \dotsb = d_r = 0\).
于是\[
\x - (k_1 \X1 + k_2 \X2 + \dotsb + k_{n-r} \X{n-r}) = \z,
\]或\[
\x = k_1 \X1 + k_2 \X2 + \dotsb + k_{n-r} \X{n-r}.
\]
\end{enumerate}

综上所述,\(\AutoTuple{\x}{n-r}\)是\(\A\x=\z\)的一个基础解系,含有\(n-r\)个解向量.
\end{proof}
\end{theorem}

\begin{corollary}
设齐次线性方程组\(\A\x=\z\)的系数矩阵\(\A\)是\(s \times n\)矩阵,若\(\rank\A = r < n\),则\begin{enumerate}
\item \(\A\x=\z\)的每个基础解系都含有\(n-r\)个解向量;
\item \(\A\x=\z\)的任意\(n-r+1\)个解向量线性相关;
\item \(\A\x=\z\)的任意\(n-r\)个线性无关的解都是它的一个基础解系.
\end{enumerate}
\end{corollary}

\begin{example}
求齐次线性方程组\[
\left\{ \begin{array}{*{11}{r}}
x_1 &-& 2 x_2 &-& x_3 &+& 2 x_4 &+& 4 x_5 &=& 0 \\
2 x_1 &-& 2 x_2 &-& 3 x_3 && &+& 2 x_5 &=& 0 \\
4 x_1 &-& 2 x_2 &-& 7 x_3 &-& 4 x_4 &-& 2 x_5 &=& 0
\end{array} \right.
\]的通解.
\begin{solution}
写出系数矩阵\(\A\),并作初等行变换化简\begin{align*}
\A &= \begin{bmatrix}
1 & -2 & -1 & 2 & 4 \\
2 & -2 & -3 & 0 & 2 \\
4 & -2 & -7 & -4 & -2
\end{bmatrix} \\
&\xlongrightarrow{\begin{array}{c}
	-2\times\text{(1行)}+\text{(2行)} \\
	-4\times\text{(1行)}+\text{(3行)}
\end{array}} \begin{bmatrix}
1 & -2 & -1 & 2 & 4 \\
0 & 2 & -1 & -4 & -6 \\
0 & 6 & -3 & -12 & -18
\end{bmatrix} \\
&\xlongrightarrow{\begin{array}{c}
	-3\times\text{(2行)}+\text{(3行)} \\
	1\times\text{(2行)}+\text{(1行)}
\end{array}} \begin{bmatrix}
1 & 0 & -2 & -2 & -2 \\
0 & 2 & -1 & -4 & -6 \\
0 & 0 & 0 & 0 & 0
\end{bmatrix} = \B,
\end{align*}
因为\(\rank\A=\rank\B=2\),所以基础解系含\(5-2=3\)个向量.
分别将\(x_3,x_4,x_5\)的3组值\((2,0,0),(0,1,0),(0,0,1)\)代入\(\B\x=\z\),得基础解系:\[
\X1 = (4,1,2,0,0)^T, \quad
\X2 = (2,2,0,1,0)^T, \quad
\X3 = (2,3,0,0,1)^T.
\]

原方程组的通解为\(k_1 \X1 + k_2 \X2 + k_3 \X3\),其中\(k_1,k_2,k_3\)为任意常数.
\end{solution}
\end{example}

\cref{theorem:线性方程组.齐次线性方程组的解向量个数} 揭示矩阵\(\A\)的秩与方程\(\A\x=\z\)的基础解系所含向量个数的关系.
它不仅对\(\A\x=\z\)的求解有重要意义,而且可以用来解决矩阵的秩的一些问题.
\begin{example}
设\(\A\)为\(s \times n\)矩阵,\(\B\)为\(n \times m\)矩阵,\(\A\B=\z\).证明:\(\rank\A + \rank\B \leq n\).
\begin{proof}
设矩阵\(\B\)与\(\A\B=\z\)右端的零矩阵列分块矩阵分别为\[
\B=(\AutoTuple{\b}{m}),
\quad
\z=(\z,\dotsc,\z)
\]那么\[
\A(\AutoTuple{\b}{m})
= (\A\b_1,\A\b_2,\dotsc,\A\b_m)
= (\z,\z,\dotsc,\z)
\]或\[
\A\b_j = \z \quad(j=1,2,\dotsc,m)
\]即\(\beta=\{\AutoTuple{\b}{m}\}\)是齐次线性方程组\(\A\x=\z\)解向量组.

若\(\rank\A=n\),则\(\A\x=\z\)只有零解,\(\B=\z\),\(\rank\B=0=n-\rank\A\).

若\(\rank\A<n\),则\(X=\{\AutoTuple{\x}{n-r}\}\)是\(\A\x=\z\)的一个基础解系,
则\(\beta\)可由\(X\)线性表出,\(\rank\beta \leq \rank X\).
而\(\rank\beta=\rank\B\),\(\rank X=n-\rank\A\).

综上所述,\(\rank\A+\rank\B \leq n\)成立.
\end{proof}
\end{example}

\begin{example}
设\(\A\)为\(n\ (n>1)\)阶方阵,\(\A^*\)是\(\A\)的伴随矩阵.
证明:\[
\rank\A^* = \left\{ \begin{array}{cl}
n, & \rank\A=n, \\
1, & \rank\A=n-1, \\
0, & \rank\A<n-1.
\end{array} \right.
\]
\begin{proof}
我们根据矩阵\(\A\)的秩的取值分类讨论.
\begin{enumerate}
\item 当\(\rank\A = n\)时,\(\A\)满秩,%
那么由\cref{theorem:向量空间.满秩方阵的行列式非零} 有\(\abs{\A} \neq 0\);%
根据恒等式 \labelcref{equation:行列式.伴随矩阵.恒等式1} 有\[
\A \A^* = \A^* \A = \abs{\A} \E;
\]
根据\cref{theorem:行列式.矩阵乘积的行列式} 有\[
\abs{\A \A^*} = \abs{\A} \abs{\A^*};
\]
根据\cref{theorem:行列式.性质2.推论2} 有\[
\abs{\abs{\A} \E} = \abs{\A}^n \abs{\E} = \abs{\A}^n;
\]
于是\[
\abs{\A^*}
= \frac{\abs{\A \A^*}}{\abs{\A}}
= \frac{\abs{\abs{\A} \E}}{\abs{\A}}
= \frac{\abs{\A}^n}{\abs{\A}}
= \abs{\A}^{n-1} \neq 0,
\]
进而有\[
\rank\A^* = n.
\]

\item 当\(\rank\A = n-1 < n\)时,%
\(\abs{\A} = 0\),\(\A \A^* = \z\),据上例,\[
\rank\A + \rank\A^* \leq n
\implies
\rank\A^*
\leq n - \rank\A
= n - (n-1)
= 1.
\]
又因为\(n > 1 \implies \rank\A = n-1 > 0\),%
所以\(\rank\A = 1\).

\item 当\(\rank\A < n-1\)时,%
\(\A\)的所有\(n-1\)阶子式全为零,%
\(\A\)的任意元素代数余子式为零,%
\(\A^* = \z\),\(\rank\A^* = 0\).
\end{enumerate}
\end{proof}
\end{example}

\begin{example}
设\(n\)阶矩阵\(\A\)满足\(\A^2 - 3 \A - 10 \E = \z\),证明:\[
\rank(\A - 5\E) + \rank(\A + 2\E) = n.
\]
\begin{proof}
因为\(\A\)满足\(\A^2 - 3 \A - 10 \E = (\A - 5\E)(\A + 2\E) = \z\),所以\[
\rank(\A - 5\E) + \rank(\A + 2\E) \leq n.
\]又因为\begin{align*}
&\rank(\A - 5\E) + \rank(\A + 2\E) \\
&\geq \max\{
	\rank[(\A - 5\E) + (\A + 2\E)],
	\rank[(5\E - \A) + (\A + 2\E)]
	\} \\
&= \max\{ \rank(2\A - 3\E),\rank(7\E) \}
= n,
\end{align*}所以\(\rank(\A - 5\E) + \rank(\A + 2\E) = n\).
\end{proof}
\end{example}

\begin{example}
已知向量\(\a_1 = \begin{bmatrix}
1 \\ 2 \\ 3
\end{bmatrix},
\a_2 = \begin{bmatrix}
2 \\ 1 \\ 1
\end{bmatrix},
\b_1 = \begin{bmatrix}
2 \\ 5 \\ 9
\end{bmatrix},
\b_2 = \begin{bmatrix}
1 \\ 0 \\ 1
\end{bmatrix}\).
若向量\(\g\)既可由\(\a_1,\a_2\)线性表出,也可由\(\b_1,\b_2\)线性表出,求\(\g\).
\begin{solution}
设\(\g = x_1 \a_1 + x_2 \a_2 = x_3 \b_1 + x_4 \b_2\),则\(x_1 \a_1 + x_2 \a_2 - x_3 \b_1 - x_4 \b_2 = \z\).
又由于\[
(\a_1,\a_2,-\b_1,-\b_2) = \begin{bmatrix}
1 & 2 & -2 & -1 \\
2 & 1 & -5 & 0 \\
3 & 1 & -9 & -1
\end{bmatrix} \to \begin{bmatrix}
1 & 0 & 0 & 3 \\
0 & 1 & 0 & -1 \\
0 & 0 & 1 & 1
\end{bmatrix},
\]解得\((x_1,x_2,x_3,x_4)^T = c (-3,1,-1,1)^T\ (c\in\mathbb{R})\).
因此\[
\g = c (x_3 \b_1 + x_4 \b_2)
= c \begin{bmatrix}
-1 \\ -5 \\ -8
\end{bmatrix}
= k \begin{bmatrix}
1 \\ 5 \\ 8
\end{bmatrix}
\quad(k\in\mathbb{R}).
\]
\end{solution}
\end{example}
