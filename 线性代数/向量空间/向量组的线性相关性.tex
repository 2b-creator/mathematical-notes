\section{向量组的线性相关性}
在上一节,我们把线性方程组有没有解的问题归结为:
常数项列向量\(\b\)能否由系数矩阵的列向量组\(\AutoTuple{\a}{s}\)线性表出.
那么,如何研究\(K^n\)中一个向量能不能由一个向量组线性表出呢?

\subsection{线性相关性的概念}
我们首先回顾\cref{theorem:解析几何.两向量共线的充分必要条件1,%
theorem:解析几何.三向量共面的充分必要条件1},
以及\cref{theorem:解析几何.两向量不共线的充分必要条件1,%
theorem:解析几何.三向量不共面的充分必要条件1}.

受此启发,我们提出以下两个概念.
\begin{definition}\label{definition:线性方程组.线性相关与线性无关的定义}
%@see: 《线性代数》(张慎语、周厚隆) P68. 定义6
设\(A=\Set{\AutoTuple{\a}{s}}\)是\(n\)维向量空间\(K^n\)中的一个向量组.

如果\(K\)中存在不全为零的数\(\AutoTuple{k}{s}\),使得\[
	k_1 \a_1 + k_2 \a_2 + \dotsb + k_s \a_s = \z,
\]
则称“向量组\(A\) \DefineConcept{线性相关}(linearly dependent)”;
否则,称“向量组\(A\) \DefineConcept{线性无关}(linearly independent)”.
%@see: https://mathworld.wolfram.com/LinearlyIndependent.html
\end{definition}

显然,从\cref{definition:线性方程组.线性相关与线性无关的定义} 立即可得
\[
	\text{向量组\(A\)线性无关}
	\iff
	[k_1 \a_1 + k_2 \a_2 + \dotsb + k_s \a_s = \z
	\implies
	(\AutoTuple{k}{s}) = \z].
\]

特别地,我们规定:
\begin{axiom}
空集\(\emptyset\)线性无关.
\end{axiom}

\subsection{线性相关性的判定条件}
根据线性相关、线性无关的定义和解析几何的结论,
在几何空间中,共线的两个向量是线性相关的,
共面的三个向量是线性相关的,
不共面的三个向量是线性无关的,
不共线的两个向量是线性无关的.

下面我们再来看几个例子.
\begin{example}\label{example:线性方程组.含有零向量的向量组线性相关}
%@see: 《线性代数》(张慎语、周厚隆) P68. 例1
向量空间\(K^n\)中的零向量可以由任意向量组\(\AutoTuple{\b}{t}\)线性表出,
这是因为恒等式\[
	0\b_1+0\b_2+\dotsb+0\b_t=\z.
\]
进一步,
含有零向量\(\z\)的向量组\[
	\Set{\z,\AutoTuple{\a}{s}}
\]总是线性相关的,
这是因为\[
	1 \z + 0 \a_1 + 0 \a_2 + \dotsb + 0 \a_s = \z.
\]
\end{example}

\begin{example}\label{example:线性方程组.基本向量组线性无关}
%@see: 《线性代数》(张慎语、周厚隆) P68. 例2
\(K^n\)的基本向量组\(\AutoTuple{\e}{n}\)线性无关.
\begin{proof}
令\(k_1 \e_1 + k_2 \e_2 + \dotsb + k_n \e_n = \z\),即\[
	k_1 (1,0,\dotsc,0)^T + k_2 (0,1,\dotsc,0)^T + \dotsb k_n (0,0,\dotsc,1)^T = \z.
\]
进一步,有\[
	(\AutoTuple{k}{n})^T = (0,\dotsc,0)^T,
\]
于是\(k_1 = k_2 = \dotsb = k_n = 0\),
因此\(\e_1,\e_2,\dotsc,\e_n\)线性无关.
\end{proof}
\end{example}

\begin{proposition}\label{theorem:线性方程组.单向量组线性相关的充分必要条件}
设\(\a \in K^n\),
则\[
	\begin{split}
		\text{向量组\(\{\a\}\)线性相关}
		\iff
		\a=\vb0, \\
		\text{向量组\(\{\a\}\)线性无关}
		\iff
		\a\neq\vb0.
	\end{split}
\]
\begin{proof}
必要性.
设\(\{\a\}\)线性相关,存在数\(k \neq 0\)使得\(k\a = \vb0\),可得\(\a = \vb0\).

充分性.
设\(\a = \vb0\),则\(1\a = \vb0\),而数\(1 \neq 0\),故\(\{\a\}\)线性相关.
\end{proof}
\end{proposition}

\begin{theorem}\label{theorem:线性方程组.向量组线性相关的充分必要条件1}
设向量组\(A=\{\AutoTuple{\a}{s}\}\ (s>1)\),
则\[
	\begin{split}
		\text{\(\A\)线性相关}
		&\iff
		\text{\(A\)中至少有一个向量可由其余\(s-1\)个向量线性表出} \\
		&\iff
		(\exists \a\in A)[\a \in \Span(A-\{\a\})], \\
		\text{\(A\)线性无关}
		&\iff
		\text{\(A\)中每一个向量都不能由其余向量线性表出} \\
		&\iff
		(\forall \a\in\A)[\a \notin \Span(\A-\{\a\})].
	\end{split}
\]
\begin{proof}
必要性.
\(A\)线性相关,则存在不全为零的数\(\AutoTuple{k}{s}\),使得\[
	k_1 \a_1 + k_2 \a_2 + \dotsb + k_s \a_s = \z.
\]
设\(k_i\neq0\ (1 \leq i \leq s)\),于是\[
	\a_i = -\frac{1}{k_i} (
		k_1 \a_1 + k_2 \a_2 + \dotsb
		+ k_{i-1} \a_{i-1} + k_{i+1} \a_{i+1}
		+ \dotsb + k_s \a_s
	),
\]
即\(\a_i\)可由其余\(s-1\)个向量线性表出.

充分性.
若\(\a_j \in A\)可由其余\(s-1\)个向量线性表出,即\[
	\a_j = l_1 \a_1 + \dotsb + l_{j-1} \a_{j-1} + l_{j+1} \a_{j+1} + \dotsb + l_s \a_s,
\]
移项得\[
	l_1 \a_1 + \dotsb
	+ l_{j-1} \a_{j-1} + (-1) \a_j + l_{j+1} \a_{j+1}
	+ \dotsb + l_s \a_s = \z,
\]
上式等号左边的系数中至少有一个数\(-1\neq0\),
因此\(A\)线性相关.
\end{proof}
\end{theorem}


\begin{theorem}\label{theorem:向量空间.增加一个向量对线性相关性的影响1}
%@see: 《线性代数》(张慎语、周厚隆) P68. 例4
%@see: 《高等代数(第三版 上册)》(丘维声) P69. 命题1
设向量组\(A=\{\AutoTuple{\a}{s}\}\)线性无关.
则“向量\(\b\)可由\(A\)线性表出”的充分必要条件是“\(B=\{\AutoTuple{\a}{s},\b\}\)线性相关”.
\begin{proof}
先证必要性.
由\cref{theorem:线性方程组.向量组线性相关的充分必要条件1} 可知,
“\(\b\)可由\(A\)线性表出”显然蕴含“\(B\)线性相关”.

再证充分性.
由于向量组\(B\)线性相关,
则存在不全为零的数\(\AutoTuple{k}{s},k\)使得\[
	k_1 \a_1 + k_2 \a_2 + \dotsb + k_s \a_s + k \b = \z.
\]
假设\(k = 0\),
则\(\AutoTuple{k}{s}\)不全为零,
且有\(k_1 \a_1 + k_2 \a_2 + \dotsb + k_s \a_s = \z\),
即\(A\)线性相关,
与题设矛盾,说明\(k \neq 0\).
于是\[
	\b = -\frac{1}{k} (k_1 \a_1 + k_2 \a_2 + \dotsb + k_s \a_s).
	\qedhere
\]
\end{proof}
\end{theorem}

从\cref{theorem:向量空间.增加一个向量对线性相关性的影响1} 立即得到如下推论.
\begin{corollary}\label{theorem:向量空间.增加一个向量对线性相关性的影响2}
%@see: 《高等代数(第三版 上册)》(丘维声) P69. 推论2
设向量组\(A=\{\AutoTuple{\a}{s}\}\)线性无关.
那么“\(\b\)不能由\(A\)线性表出”的充分必要条件是“\(B=\{\AutoTuple{\a}{s},\b\}\)线性无关”.
\end{corollary}

\begin{theorem}\label{theorem:线性方程组.部分组线性相关则全组线性相关}
若向量组\(A=\{\AutoTuple{\a}{s}\}\)的一个部分组线性相关,则\(A\)线性相关.
\begin{proof}
假设\(A\)的部分组\(B=\{\AutoTuple{\a}{t}\}\ (t \leq s)\)线性相关,
即存在不全为零的数\(\AutoTuple{k}{t}\)使得\[
	k_1 \a_1 + k_2 \a_2 + \dotsb + k_t \a_t = \z;
\]
从而有\[
	k_1 \a_1 + k_2 \a_2 + \dotsb + k_t \a_t + 0 \a_{t+1} + \dotsb + 0 \a_s = \z;
\]
由于上式等号左边的系数\(\AutoTuple{k}{t},0,\dotsc,0\)不全为零,
因此向量组\(A\)线性相关.
\end{proof}
\end{theorem}

由\cref{theorem:线性方程组.部分组线性相关则全组线性相关} 立即得到:
\begin{corollary}\label{theorem:线性方程组.全组线性无关则任一部分组线性无关}
如果向量组\(A\)线性无关,
那么\(A\)的任意一个部分组也线性无关.
\end{corollary}

给定\(n\)维向量组\(\AutoTuple{\a}{s}\),
为其中的每个向量都添上\(m\)个分量,
所添分量的位置对于每个向量都一样,
把得到的\(n+m\)维向量组\(\AutoTuple{\b}{s}\)称为%
“\(\AutoTuple{\a}{s}\)的\DefineConcept{延伸组}”;
反过来,把\(\AutoTuple{\a}{s}\)称为%
“\(\AutoTuple{\b}{s}\)的\DefineConcept{缩短组}”.

如果向量组线性无关,那么它的延伸组也线性无关.
如果向量组线性相关,那么它的缩短组也线性相关.

\begin{theorem}
\(n\)个\(n\)维列向量\(\AutoTuple{\a}{n}\)线性相关的充分必要条件是:\[
	\det(\AutoTuple{\a}{n})=0.
\]
\end{theorem}

\begin{corollary}
\(n\)个\(n\)维列向量\(\AutoTuple{\a}{n}\)线性无关的充分必要条件是:\[
	\det(\AutoTuple{\a}{n})\neq0.
\]
\end{corollary}

需要注意的是,当\(s \neq n\)时,
\(s\)个\(n\)维向量\(\AutoTuple{\a}{s}\)不能构成行列式,
只能用其他方法判断其线性相关性.



\begin{theorem}[替换定理]
设向量组\(\AutoTuple{\a}{s}\)线性无关,
\(\b=b_1\a_1+\dotsb+b_s\a_s\).
如果\(b_j\neq0\),
那么用\(\b\)替换\(\a_j\)以后得到的向量组
\(\AutoTuple{\a}{j-1},\b,\AutoTuple{\a}[j+1]{s}\)
也线性无关.
\end{theorem}

\begin{example}
设\(\A\)是3阶矩阵,\(\a_1,\a_2,\a_3\)为3维列向量组,
若\(\A\a_1,\A\a_2,\A\a_3\)线性无关,
证明:\(\a_1,\a_2,\a_3\)线性无关,且\(\A\)为可逆矩阵.
\begin{proof}
因为\(\A\a_1,\A\a_2,\A\a_3\)线性无关,所以齐次线性方程组\[
	x_1 \A\a_1 + x_2 \A\a_2 + x_3 \A\a_3
	= (\A\a_1,\A\a_2,\A\a_3) (x_1,x_2,x_3)^T
	= \z
\]
只有零解(即\(x_1 = x_2 = x_3 = 0\)),根据克拉默法则,有\[
	\det(\A\a_1,\A\a_2,\A\a_3) \neq 0.
\]

因为\(\det(\A\a_1,\A\a_2,\A\a_3) = \abs{\A} \cdot \det(\a_1,\a_2,\a_3) \neq 0\),
所以\(\abs{\A} \neq 0\)(即\(\A\)是可逆矩阵),
且\(\det(\a_1,\a_2,\a_3) \neq 0\)%
(即齐次线性方程组\(x_1 \a_1 + x_2 \a_2 + x_3 \a_3 = \z\)只有零解,
向量组\(\a_1,\a_2,\a_3\)线性无关).
\end{proof}
\end{example}

%\begin{example}
%证明:\(\mathbb{R}^n\)中的任意正交组线性无关.
%\begin{proof}
%设\(A=\{\AutoTuple{\a}{m}\}\)是\(\mathbb{R}^n\)的一个正交组,令\[
%	k_1 \a_1 + k_2 \a_2 + \dotsb + k_m \a_m = \z,
%\]
%两端分别与\(\a_1\)作内积,
%即\[
%	\vectorinnerproduct{(k_1 \a_1 + k_2 \a_2 + \dotsb + k_m \a_m)}{\a_1}
%	= \vectorinnerproduct{\z}{\a_1};
%\]
%由内积性质,\[
%	k_1 (\vectorinnerproduct{\a_1}{\a_1})
%	+ k_2 (\vectorinnerproduct{\a_2}{\a_1})
%	+ \dotsb
%	+ k_m (\vectorinnerproduct{\a_m}{\a_1})
%	= \vectorinnerproduct{\z}{\a_1},
%\]
%其中\(\vectorinnerproduct{\z}{\a_1} = 0\),
%\(\vectorinnerproduct{\a_j}{\a_1} = 0\ (j=2,3,\dotsc,m)\),
%故\(k_1 \vectorinnerproduct{\a_1}{\a_1} = 0\),
%而\(\vectorinnerproduct{\a_1}{\a_1} > 0\),
%所以\(k_1=0\).
%同理可得\(k_2=k_3=\dotsb=k_m=0\),从而\(A\)线性无关.
%\end{proof}
%\end{example}


最后我们对本节内容作一个小结.
\(K^n\)中线性相关的向量组与线性无关的向量组的本质区别可以从以下几个方面刻画.
\begin{enumerate}
	\item 从线性组合的角度看.
	\begin{enumerate}
		\item \(\begin{aligned}[t]
			&\text{向量组\(\AutoTuple{\a}{s}\ (s\geq1)\)线性相关} \\
			&\iff
			\text{它们有系数不全为零的线性组合等于零向量}.
		\end{aligned}\)
		\item \(\begin{aligned}[t]
			&\text{向量组\(\AutoTuple{\a}{s}\ (s\geq1)\)线性无关} \\
			&\iff
			\text{它们只有系数全为零的线性组合才会等于零向量}.
		\end{aligned}\)
	\end{enumerate}
	\item 从线性表出的角度看.
	\begin{enumerate}
		\item \(\begin{aligned}[t]
			&\text{向量组\(\AutoTuple{\a}{s}\ (s\geq2)\)线性相关} \\
			&\iff
			\text{其中至少有一个向量可以由其余向量线性表出}.
		\end{aligned}\)
		\item \(\begin{aligned}[t]
			&\text{向量组\(\AutoTuple{\a}{s}\ (s\geq2)\)线性无关} \\
			&\iff
			\text{其中每一个向量都不能由其余向量线性表出}.
		\end{aligned}\)
	\end{enumerate}
	\item 从齐次线性方程的角度看.
	\begin{enumerate}
		\item \(\begin{aligned}[t]
			&\text{列向量组\(\AutoTuple{\a}{s}\ (s\geq1)\)线性相关} \\
			&\iff
			\text{有\(K\)中不全为零的数\(\AutoTuple{k}{s}\)使得\(k_1\a_1+\dotsb+k_s\a_s=\z\)} \\
			&\iff
			\text{齐次线性方程组\(x_1\a_1+\dotsb+x_s\a_s=\z\)有非零解}.
		\end{aligned}\)
		\item \(\begin{aligned}[t]
			&\text{列向量组\(\AutoTuple{\a}{s}\ (s\geq1)\)线性无关} \\
			&\iff
			\text{齐次线性方程组\(x_1\a_1+\dotsb+x_s\a_s=\z\)只有零解}.
		\end{aligned}\)
	\end{enumerate}
	\item 从行列式的角度看.
	\begin{enumerate}
		\item \(\begin{aligned}[t]
			&\text{\(n\)个\(n\)维列向量\(\AutoTuple{\a}{n}\)线性相关} \\
			&\iff
			\text{以\(\AutoTuple{\a}{n}\)为列向量组的矩阵的行列式等于零}. \\
			&\text{\(n\)个\(n\)维行向量\(\AutoTuple{\a}{n}[,][T]\)线性相关} \\
			&\iff
			\text{以\(\AutoTuple{\a}{n}\)为行向量组的矩阵的行列式等于零}.
		\end{aligned}\)
		\item \(\begin{aligned}[t]
			&\text{\(n\)个\(n\)维列向量\(\AutoTuple{\a}{n}\)线性无关} \\
			&\iff
			\text{以\(\AutoTuple{\a}{n}\)为列向量组的矩阵的行列式不等于零}. \\
			&\text{\(n\)个\(n\)维行向量\(\AutoTuple{\a}{n}[,][T]\)线性无关} \\
			&\iff
			\text{以\(\AutoTuple{\a}{n}\)为行向量组的矩阵的行列式不等于零}.
		\end{aligned}\)
	\end{enumerate}
\end{enumerate}
