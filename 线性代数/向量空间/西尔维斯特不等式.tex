\section{西尔维斯特不等式}
\begin{theorem}
设\(\A \in M_{s \times n}(K),
\B \in M_{n \times t}(K)\).
证明:\begin{equation}\label{equation:线性方程组.西尔维斯特不等式}
	\rank\A + \rank\B - n \leq \rank(\A\B).
\end{equation}
\begin{proof}
由\cref{equation:矩阵的秩.分块矩阵的秩的等式1},\[
	\rank\begin{bmatrix}
		\E_n & \z \\
		\z & \A\B
	\end{bmatrix}
	= n + \rank(\A\B).
	\eqno(1)
\]
又因为\[
	\begin{bmatrix}
		\B & \E_n \\
		\z & \A
	\end{bmatrix}
	= \begin{bmatrix}
		\E_n & \z \\
		\A & \E_s
	\end{bmatrix}
	\begin{bmatrix}
		\E_n & \z \\
		\z & \A\B
	\end{bmatrix}
	\begin{bmatrix}
		\E_n & -\B \\
		\z & \E_t
	\end{bmatrix}
	\begin{bmatrix}
		\z & \E_s \\
		-\E_t & \z
	\end{bmatrix},
\]
而\[
	\begin{bmatrix}
		\E_n & \z \\
		\A & \E_s
	\end{bmatrix}, \qquad
	\begin{bmatrix}
		\E_n & -\B \\
		\z & \E_t
	\end{bmatrix},
	\quad\text{和}\quad
	\begin{bmatrix}
		\z & \E_s \\
		-\E_t & \z
	\end{bmatrix}
\]这三个矩阵都是满秩矩阵,
所以\[
	\rank\begin{bmatrix}
		\E_n & \z \\
		\z & \A\B
	\end{bmatrix}
	= \rank\begin{bmatrix}
		\B & \E_n \\
		\z & \A
	\end{bmatrix}.
	\eqno(2)
\]
再由\cref{equation:矩阵的秩.分块矩阵的秩的不等式} 有\[
	\rank\begin{bmatrix}
		\B & \E_n \\
		\z & \A
	\end{bmatrix}
	\geq \rank\A+\rank\B.
	\eqno(3)
\]
因此,\(\rank\A + \rank\B \leq n + \rank(\A\B)\).
%@see: https://math.stackexchange.com/a/2414197/591741
%@see: http://www.m-hikari.com/imf-password2009/33-36-2009/luIMF33-36-2009.pdf
\end{proof}
\end{theorem}

我们把\cref{equation:线性方程组.西尔维斯特不等式} 称为“西尔维斯特不等式”.

进一步,如果有\(\A\B=\z\),那么\[
	\rank\A + \rank\B \leq n.
\]

我们还可以将\hyperref[equation:线性方程组.西尔维斯特不等式]{西尔维斯特不等式}进行如下的推广.
\begin{theorem}
\begin{equation}\label{equation:线性方程组.弗罗贝尼乌斯不等式}
	\rank(\A\B\C) \geq \rank(\A\B) + \rank(\B\C) - \rank\B.
\end{equation}
\begin{proof}
利用初等变换,有\[
	\begin{bmatrix}
		\B & \z \\
		\z & \A\B\C
	\end{bmatrix}
	\to \begin{bmatrix}
		\B & \z \\
		\A\B & \A\B\C
	\end{bmatrix}
	\to \begin{bmatrix}
		\B & -\B\C \\
		\A\B & \z
	\end{bmatrix}
	\to \begin{bmatrix}
		\B\C & \B \\
		\z & \A\B
	\end{bmatrix},
\]
于是\[
	\rank\B + \rank(\A\B\C)
	= \rank\begin{bmatrix}
		\B & \z \\
		\z & \A\B\C
	\end{bmatrix}
	= \rank\begin{bmatrix}
		\B\C & \B \\
		\z & \A\B
	\end{bmatrix}
	\geq \rank(\A\B) + \rank(\B\C).
	\qedhere
\]
\end{proof}
\end{theorem}

我们把\cref{equation:线性方程组.弗罗贝尼乌斯不等式} 称为“弗罗贝尼乌斯不等式”.

\begin{example}
%@see: 《高等代数(第三版 上册)》(丘维声) P143. 习题4.5 4.
证明:如果\(n\)阶矩阵\(\A\)是\DefineConcept{对合矩阵},
即满足\(\A^2=\E\),
则\[
	\rank(\E+\A)+\rank(\E-\A)=n.
\]
\begin{proof}
因为利用初等变换可以得到\begin{align*}
	&\hspace{-20pt}
	\begin{bmatrix}
		\E+\A & \z \\
		\z & \E-\A
	\end{bmatrix}
	\to \begin{bmatrix}
		\E+\A & \z \\
		\A(\E+\A) & \E-\A
	\end{bmatrix}
	= \begin{bmatrix}
		\E+\A & \z \\
		\A+\E & \E-\A
	\end{bmatrix} \\
	&\to \begin{bmatrix}
		\E+\A & \E-\A \\
		\z & \z
	\end{bmatrix}
	\to \begin{bmatrix}
		\E+\A & 2\E \\
		\z & \z
	\end{bmatrix}
	\to \begin{bmatrix}
		\E+\A & \E \\
		\z & \z
	\end{bmatrix}
	\to \begin{bmatrix}
		\A & \E \\
		\z & \z
	\end{bmatrix},
\end{align*}
所以\[
	\rank(\E+\A)+\rank(\E-\A)
	=\rank\begin{bmatrix}
		\E+\A & \z \\
		\z & \E-\A
	\end{bmatrix}
	= \rank\begin{bmatrix}
		\A & \E \\
		\z & \z
	\end{bmatrix}
	= n.
	\qedhere
\]
\end{proof}
\end{example}

\begin{example}
%@see: 《高等代数(第三版 上册)》(丘维声) P143. 习题4.5 5.
证明:如果\(n\)阶矩阵\(\A\)是\DefineConcept{幂等矩阵}(idempotent matrix),
即满足\(\A^2=\A\),
则\[
	\rank\A+\rank(\E-\A)=n.
\]
\begin{proof}
由于\[
	\A^2=\A
	\iff
	\A^2-\A=\z
	\iff
	\rank(\A^2-\A)=0.
\]
又因为利用初等变换可以得到\[
	\begin{bmatrix}
		\A & \z \\
		\z & \E_n-\A
	\end{bmatrix}
	\to \begin{bmatrix}
		\A & \z \\
		\A & \E_n-\A
	\end{bmatrix}
	\to \begin{bmatrix}
		\A & \A \\
		\A & \E_n
	\end{bmatrix}
	\to \begin{bmatrix}
		\A-\A^2 & \z \\
		\A & \E_n
	\end{bmatrix}
	\to \begin{bmatrix}
		\A-\A^2 & \z \\
		\z & \E_n
	\end{bmatrix},
\]
所以\(\rank\A+\rank(\E_n-\A)
=\rank(\A-\A^2)+n
=n\).
\end{proof}
\end{example}

\begin{example}
设\(\A \in M_{m \times n}(K)\).
证明:\[
	\rank(\E_m-\A\A^T)-\rank(\E_n-\A^T\A)=m-n.
\]
\begin{proof}
因为利用初等变换可以得到\begin{align*}
	\begin{bmatrix}
		\E_m-\A\A^T & \z \\
		\z & \E_n
	\end{bmatrix}
	&\to \begin{bmatrix}
		\E_m-\A\A^T & \A \\
		\z & \E_n
	\end{bmatrix}
	\to \begin{bmatrix}
		\E_m & \A \\
		\A^T & \E_n
	\end{bmatrix} \\
	&\to \begin{bmatrix}
		\E_m & \z \\
		\A^T & \E_n-\A^T\A
	\end{bmatrix}
	\to \begin{bmatrix}
		\E_m & \z \\
		\z & \E_n-\A^T\A
	\end{bmatrix},
\end{align*}
所以\[
	\rank(\E_m-\A\A^T)+n=m+\rank(\E_n-\A^T\A),
\]
移项便得\(\rank(\E_m-\A\A^T)-\rank(\E_n-\A^T\A)=m-n\).
\end{proof}
\end{example}

\begin{example}
设\(\A,\B,\C \in M_n(K)\),
\(\rank\C=n\),
\(\A(\B\A+\C)=\z\).
证明:\[
	\rank(\B\A+\C)=n-\rank\A.
\]
\begin{proof}
因为\(\A(\B\A+\C)=\z\),
所以根据\cref{equation:线性方程组.西尔维斯特不等式} 有\[
	\rank(\B\A+\C)+\rank\A \leq n.
\]
又利用初等变换可以得到\[
	\begin{bmatrix}
		\B\A+\C & \z \\
		\z & \A
	\end{bmatrix}
	\to \begin{bmatrix}
		\B\A+\C & \B\A \\
		\z & \A
	\end{bmatrix}
	\to \begin{bmatrix}
		\C & \B\A \\
		-\A & \A
	\end{bmatrix},
\]
于是\[
	\rank(\B\C+\C)
	+\rank\A
	= \rank\begin{bmatrix}
		\B\A+\C & \z \\
		\z & \A
	\end{bmatrix}
	= \rank\begin{bmatrix}
		\C & \B\A \\
		-\A & \A
	\end{bmatrix}
	\geq \rank\C = n.
	\qedhere
\]
\end{proof}
\end{example}
