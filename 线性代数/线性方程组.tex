\chapter{线性方程组}
\section{线性方程组}
\subsection{线性方程组的概念}
我们把含有\(n\)个未知量\(\AutoTuple{x}{n}\)的一次方程组
\begin{equation}\label[equation-system]{equation:线性方程组.线性方程组的代数形式}
	\left\{ \begin{array}{l}
		a_{11} x_1 + a_{12} x_2 + \dotsb + a_{1n} x_n = b_1, \\
		a_{21} x_1 + a_{22} x_2 + \dotsb + a_{2n} x_n = b_2, \\
		\hdotsfor{1} \\
		a_{s1} x_1 + a_{s2} x_2 + \dotsb + a_{sn} x_n = b_s
	\end{array} \right.
\end{equation}
称为“\(n\)元\DefineConcept{线性方程组}(\emph{linear system of equations} in \(n\) variables)”.
这里,\(s\)为方程的个数\footnote{%
在\cref{equation:线性方程组.线性方程组的代数形式} 中,%
方程的数目\(s\)可以等于未知数的数目\(n\),%
也可以不相等(即\(s<n\)或\(s>n\)).
};%
我们把数
\[
	a_{ij}
	\quad(i=1,2,\dotsc,s; j=1,2,\dotsc,n)
\]
称为“第\(i\)个方程中\(x_j\)的\DefineConcept{系数}(coefficient)”;
把数
\[
	b_i
	\quad(i=1,2,\dotsc,s)
\]
叫做“第\(i\)个方程的\DefineConcept{常数项}(constant term)”.

\subsection{线性方程组的表示}
\begin{definition}
将\cref{equation:线性方程组.线性方程组的代数形式} 的系数按原位置构成的\(s \times n\)矩阵\[
	\A = \begin{bmatrix}
		a_{11} & a_{12} & \dots & a_{1n} \\
		a_{21} & a_{22} & \dots & a_{2n} \\
		\vdots & \vdots & & \vdots \\
		a_{n1} & a_{n2} & \dots & a_{nn}
	\end{bmatrix}
\]叫做\DefineConcept{系数矩阵}(coefficient matrix).

特别地,如果\(s = n\)(即系数矩阵\(\A\)是一个方阵),%
则系数矩阵的行列式\(\abs{\A}\)叫做\DefineConcept{系数行列式}.
\end{definition}

为使表述简明,常用向量、矩阵表示线性方程组.
若记\[
	\x=\begin{bmatrix}
		x_1 \\ x_2 \\ \vdots \\ x_n
	\end{bmatrix},
	\quad
	\b=\begin{bmatrix}
		b_1 \\ b_2 \\ \vdots \\ b_n
	\end{bmatrix},
	\quad
	\a_j=\begin{bmatrix}
		a_{1j} \\ a_{2j} \\ \vdots \\ a_{sj}
	\end{bmatrix},
	\quad
	j=1,2,\dotsc,n.
\]
\(\A\)的列分块阵为\(\A = (\a_1,\a_2,\dotsc,\a_n)\),%
则\cref{equation:线性方程组.线性方程组的代数形式} 有以下两种等价表示:
\begin{enumerate}
	\item 矩阵形式\[
		\A \x = \b.
	\]
	\item 向量形式\[
		x_1 \a_1 + x_2 \a_2 + \dotsb + x_n \a_n = \b.
	\]
\end{enumerate}

\begin{definition}
称常数项全为零(即\(\b=\z\))的线性方程组为\DefineConcept{齐次线性方程组}.
称常数项中有非零数(即\(\b\neq\z\))的线性方程组为\DefineConcept{非齐次线性方程组}.
\end{definition}

\begin{definition}
如果存在\(n\)个数\(\AutoTuple{c}{n}\)满足\cref{equation:线性方程组.线性方程组的代数形式},即\[
a_{i1} c_1 + a_{12} c_2 + \dotsb + a_{in} c_n \equiv b_i
\quad(i=1,2,\dotsc,s),
\]
则称“\cref{equation:线性方程组.线性方程组的代数形式} 有解”,%
或“\cref{equation:线性方程组.线性方程组的代数形式} 是相容的”;
否则,称“\cref{equation:线性方程组.线性方程组的代数形式} 无解”,%
或“\cref{equation:线性方程组.线性方程组的代数形式} 是不相容的”.

称这\(n\)个数\(\AutoTuple{c}{n}\)构成的列向量\((\AutoTuple{c}{n})^T\)为%
\cref{equation:线性方程组.线性方程组的代数形式} 的一个\DefineConcept{解}(solution)%
或\DefineConcept{解向量}(solution vector).
\cref{equation:线性方程组.线性方程组的代数形式} 的解的全体构成的集合,%
称为“\cref{equation:线性方程组.线性方程组的代数形式} 的\DefineConcept{解集}”.
\end{definition}

\begin{definition}
元素全为零的解向量称为\DefineConcept{零解}.
元素不全为零的解向量称为\DefineConcept{非零解}.
\end{definition}

\begin{theorem}
任意一个齐次线性方程组都有零解.
\end{theorem}

\begin{definition}
解集相等的两个线性方程组称为\DefineConcept{同解方程组}.
\end{definition}

需要注意的是,线性方程组的解集与数域有关.

\begin{theorem}
\(n\)元线性方程组的解的情况有且只有三种可能:
无解、有唯一解、有无穷多个解.
\end{theorem}

\subsection{克拉默法则}
\begin{theorem}[克拉默法则]\label{theorem:线性方程组.克拉默法则}
%@see: https://mathworld.wolfram.com/CramersRule.html
给定一个未知量个数与方程个数相同的线性方程组\[
	\left\{ \begin{array}{l}
		a_{11}x_1 + a_{12}x_2 + \dotsb + a_{1n}x_n = b_1, \\
		a_{21}x_1 + a_{22}x_2 + \dotsb + a_{2n}x_n = b_2, \\
		\hdotsfor{1} \\
		a_{n1}x_1 + a_{n2}x_2 + \dotsb + a_{nn}x_n = b_n.
	\end{array} \right.
	\eqno(1)
\]
如果它的系数行列式满足\[
	d=\abs{\A}=\begin{vmatrix}
	a_{11} & a_{12} & \dots & a_{1n} \\
	a_{21} & a_{22} & \dots & a_{2n} \\
	\vdots & \vdots & \ddots & \vdots \\
	a_{n1} & a_{n2} & \dots & a_{nn}
	\end{vmatrix} \neq 0,
\]
则线性方程组(1)存在唯一解:\[
	\x_0 = \left( \frac{d_1}{d},\frac{d_2}{d},\dotsc,\frac{d_n}{d} \right)^T,
\]
其中\[
	d_k = \begin{vmatrix}
		a_{11} & \dots & a_{1\ k-1} & b_1 & a_{1\ k+1} & \dots & a_{1n} \\
		a_{21} & \dots & a_{2\ k-1} & b_2 & a_{2\ k+1} & \dots & a_{2n} \\
		\vdots & & \vdots & \vdots & \vdots & & \vdots \\
		a_{n1} & \dots & a_{n\ k-1} & b_n & a_{n\ k+1} & \dots & a_{nn}
	\end{vmatrix},
	\quad k=1,2,\dotsc,n.
\]
\begin{proof}
上述原线性方程组可以改写为矩阵形式\(\A \x = \b\).
因为\(d = \abs{\A} \neq 0\),故\(\A\)可逆,那么线性方程组有唯一解\[
	\X0 = \A^{-1} \b = \frac{1}{d} \A^* \b,
\]其中\(\A^*\)是\(\A\)的伴随矩阵.
设\(A_{ij}\)表示\(\A\)中元素\(a_{ij}\ (i,j=1,2,\dotsc,n)\)的代数余子式,
则行列式\(d_k\)可按第\(k\)列展开,得\[
	d_k = b_1 A_{1k} + b_2 A_{2k} + \dotsb + b_n A_{nk},
	\quad k=1,2,\dotsc,n,
\]
则\[
	\X0 = \frac{1}{d} \A^* \b
	= \frac{1}{d} \begin{bmatrix}
		A_{11} & A_{21} & \dots & A_{n1} \\
		A_{12} & A_{22} & \dots & A_{n2} \\
		\vdots & \vdots & \ddots & \vdots \\
		A_{1n} & A_{2n} & \dots & A_{nn}
	\end{bmatrix} \begin{bmatrix} b_1 \\ b_2 \\ \vdots \\ b_n \end{bmatrix}
	= \frac{1}{d} \begin{bmatrix} d_1 \\ d_2 \\ \vdots \\ d_n \end{bmatrix}.
	\qedhere
\]
\end{proof}
\end{theorem}

\subsection{消元法}
\begin{definition}
在数域\(P\)上,设矩阵\(\A \in P^{s \times n}\),向量\(\x \in P^n\),\(\b \in P^s\).由\(n\)元线性方程组\[
\A \x = \b
\]的系数和常数项按原位置构成的\(s \times (n+1)\)矩阵\((\A,\b)\),称为该\(n\)元线性方程组的\DefineConcept{增广矩阵}(augmented matrix),记作\(\widetilde{\A}\).
\end{definition}

\begin{theorem}
对线性方程组\(\A \x = \b\)的增广矩阵\(\widetilde{\A} = (\A,\b)\)作初等行变换,变为\(\widetilde{\C} = (\C,\g)\),则相应的线性方程组\(\C \x = \g\)与原线性方程组同解.
\begin{proof}
显然存在可逆矩阵\(\P\)使得\(\widetilde{\A} \to \widetilde{\C} = \P \widetilde{\A} = (\P\A,\P\b)\),\(\C = \P\A\),\(\g = \P\b\).
如果\(\X0\)是原线性方程组的解,即\(\A \X0 = \b\),用\(\P\)左乘等式两端得\(\C \X0 = \g\);反之,若\(\X0\)满足\(\C \X0 = \g\),用\(\P^{-1}\)左乘等式两端得\(\A \X0 = \b\),故两方程组同解.
\end{proof}
\end{theorem}
这就是消元法解线性方程组的理论根据.
具体化简\(\widetilde{\A}\)时,可用一系列初等行变换将其变成一个较为简单的“阶梯形矩阵”(或更简单的“若尔当阶梯形矩阵”).

\begin{definition}
称如下形式的\(s \times n\)矩阵\[
\A = \begin{bmatrix}
0 & \dots & a_{1 j_1} & \dots & a_{1 j_2} & \dots & a_{1 j_r} & \dots & a_{1n} \\
0 & \dots & 0 & \dots & a_{2 j_2} & \dots & a_{2 j_r} & \dots & a_{2n} \\
\vdots & & \vdots & & \vdots & & \vdots & & \vdots \\
0 & \dots & 0 & \dots & 0 & \dots & a_{r j_r} & \dots & a_{rn} \\
0 & \dots & 0 & \dots & 0 & \dots & 0 & \dots & 0 \\
\vdots & & \vdots & & \vdots & & \vdots & & \vdots \\
0 & \dots & 0 & \dots & 0 & \dots & 0 & \dots & 0 \\
\end{bmatrix}
\]为\DefineConcept{阶梯形矩阵}(ladder matrix),%
其中\(a_{1 j_1},a_{2 j_2},\dotsc,a_{r j_r}\)均不为零,%
\(j_1 < j_2 < \dotsb < j_r\),%
\(\A\)的后\(s-r\)行全为零.

元素全为\(0\)的行,称为\DefineConcept{零行}.
元素不全为\(0\)的行,称为\DefineConcept{非零行}.

在非零行中,从左数起,第一个不为\(0\)的元素\(a_{i j_i}\ (i=1,2,\dotsc,r)\),称为\DefineConcept{主元}或\DefineConcept{非零首元}.

以主元为系数的未知量\(x_{j_1},x_{j_2},\dotsc,x_{j_r}\),称为\DefineConcept{主变量};
不以主元为系数的未知量,称为\DefineConcept{自由未知量}.
\end{definition}

\begin{definition}
若阶梯形矩阵\(\A\)的非零行的非零首元全为\(1\),%
它们所在列的其余元素全为零,%
则称\(\A\)为\DefineConcept{若尔当阶梯形矩阵}或\DefineConcept{行约化矩阵}.
\end{definition}
% 在Mathematica中可以用RowReduce对矩阵进行初等行变换化为若尔当阶梯形矩阵.

\begin{lemma}
任何一个非零矩阵都可经初等行变换化为阶梯形矩阵.
\begin{proof}
设\(\A_{s \times n} \neq \z\),则\(\A\)经0次或1次交换两行的变换化为\(\B\),即\[
\A \to \B = \begin{bmatrix}
0 & \dots & 0 & b_{1 j_1} & \dots & b_{1n} \\
0 & \dots & 0 & b_{2 j_1} & \dots & b_{2n} \\
\vdots & & \vdots & \vdots & & \vdots \\
0 & \dots & 0 & b_{s j_1} & \dots & b_{sn}
\end{bmatrix},
\]其中\(b_{1 j_1} \neq 0\).分别将\(\B\)的第一行的\(-b_{i j_1}/b_{1 j_1}\)倍加到第\(i\)(\(i=2,3,\dotsc,s\))行,则\[
\B \to \C = \begin{bmatrix}
\z & b_{1 j_1} & \B_1 \\
\z & \z & \C_1
\end{bmatrix},
\]其中\(\C_1\)是\((s-1)\times(n-j_1)\)矩阵,再对\(\C\)的后面\(s-1\)行作类似的初等行变换化简.因为矩阵行数有限,这样下去,最后总可化为阶梯形矩阵.
\end{proof}
\end{lemma}

\begin{corollary}\label{theorem:线性方程组.非零矩阵可经初等行变换化为若尔当阶梯形矩阵}
任何一个非零矩阵都可经初等行变换化为若尔当阶梯形矩阵.
\end{corollary}

\begin{theorem}
设\(\A\)为\(n\)阶方阵,则齐次线性方程组\(\A\x = \z\)有非零解的充要条件是:\(\abs{\A} = 0\).
\begin{proof}
必要性.
给定\(\X0 \neq \z\)满足\(\A \X0 = \z\).
假设\(\abs{\A} \neq 0\),%
由克拉默法则可知方程组有唯一解\(\X0 = \A^{-1} \z = \z\),%
矛盾,%
故\(\abs{\A} = 0\).

充分性.
用数学归纳法证明.
给定\(\abs{\A} = 0\).
当\(n=1\)时,\(\A = \z_{1 \times 1} = 0\),%
\(0 \cdot x_1 = 0\)有非零解;
假设当\(n=k-1\geqslant1\)时,%
结论成立;
那么当\(n=k\)时,%
设\(\A\)经初等行变换\(\P\)化为阶梯形矩阵\[
\B = \begin{bmatrix}
b & \g \\
\z & \C \\
\end{bmatrix} = \P \A,
\]
其中\(\C\)是\(n-1\)阶方阵,\(\P\)是\(n\)阶可逆矩阵.
取行列式得
\[
\abs{\B} = \abs{\P} \abs{\A} = 0 = b \abs{\C}.
\]
解同解方程组\(\B \x = \z\).
若\(b = 0\)则\((1,0,\dotsc,0)^T\)是一个非零解;
若\(b \neq 0\),则\(\abs{\C} = 0\),由归纳假设,齐次线性方程组\[
\C \begin{bmatrix} x_2 \\ x_3 \\ \vdots \\ x_n \end{bmatrix} = \z
\]有非零解\((k_2,k_3,\dotsc,k_n)^T\),代入\(\B \x = \z\)的第一个方程,因为\(x_1\)的系数\(b \neq 0\),可解出\(x_1\).于是\((\AutoTuple{k}{n})^T\)是\(\A \x = \z\)的一个非零解.
\end{proof}
\end{theorem}


\begin{corollary}
方程个数少于未知量个数的齐次线性方程组必有非零解.
\begin{proof}
设数域\(P\)上的线性方程组\(\A \x = \z\),%
它的系数矩阵\(\A \in M_{s \times n}(P)\ (s<n)\).
在原方程组的后面添加\(n-s\)个\(0=0\)的方程,解不变,新方程组的系数矩阵为:
\[
\B_{n} = \begin{bmatrix} \A_{s \times n} \\ \z_{(n-s) \times n} \end{bmatrix},
\]
由于\(s < n\),有\(\abs{\B} = 0\),故\(\B \x = \z\)有非零解,从而\(\A \x = \z\)有非零解.
\end{proof}
\end{corollary}
注:上述推论的逆命题不成立,即方程组有非零解,不能得出未知量个数与方程个数的关系.
