\chapter{向量、矩阵及其基本运算}
线性代数的核心内容是研究有限维线性空间的结构和线性空间的线性变换.由于数域\(P\)上的任意一个\(n\)维线性空间\(V\)都与\(n\)维向量空间\(P^n\)同构,\(V\)上的全体线性变换构成的集合\(L(V,V)\)与数域\(P\)上的全体\(n \times n\)矩阵构成的集合\(P^{n \times n}\)同构,因此,在本书中线性代数主要研究矩阵理论、\(n\)维向量的线性关系、线性方程组、行列式、二次型、矩阵的特征值与特征向量、内积等内容.

\section{向量}
\subsection{向量的概念}
\begin{definition}
由\(n\)个数构成的有序数组,若排成一行记作\[
\a = (\alpha_1, \alpha_2, \dotsc, \alpha_n),
\]则称为\(n\)维\DefineConcept{行向量};
若排成一列记作\[
\a = \begin{bmatrix}
a_1 \\ a_2 \\ \vdots \\ a_n
\end{bmatrix},
\]则称作\(n\)维\DefineConcept{列向量}.
并称数\(a_i\)为\(\a\)的第\(i\)个\DefineConcept{分量}(\(i=1,2,\dotsc,n\)).

\(n\)维行向量和\(n\)维列向量统称为\(n\)维\DefineConcept{向量}(vector),常用小写黑体字母表示.
\end{definition}

\subsection{向量的关系与运算}
\begin{definition}
对于\(n\)维向量\(\a = (\v{a}{n})\)和\(\b = (\v{b}{n})\).
如果这两个向量对应的分量全部相等,
即\[
	a_i = b_i,
	\quad
	i=1,2,\dotsc,n;
\]
则称“\(\a\)与\(\b\) \DefineConcept{相等}”,记作\[
\a = \b.
\]
\end{definition}

\begin{definition}
对于\(n\)维向量\(\a = (\v{a}{n})\)和\(\b = (\v{b}{n})\).
\begin{enumerate}
\item {\bf 加法}:称向量\((a_1+b_1,a_2+b_2,\dotsc,a_n+b_n)\)为\(\a\)与\(\b\)的\DefineConcept{和},记作\[
\a+\b=(a_1+b_1,a_2+b_2,\dotsc,a_n+b_n)
\]
\item {\bf 数量乘法}:设\(k\)为数,称向量\((k a_1, k a_2, \dotsc, k a_n)\)为\(k\)与\(\a\)的\DefineConcept{数乘},记作\[
k\a = (k a_1, k a_2, \dotsc, k a_n)
\]
\item 分量全为零的向量\((0,0,\dotsc,0)\)称为\DefineConcept{零向量},记作\(\z\).
\item 称\((-a_1,-a_2,\dotsc,-a_n)\)为\(\a\)的\DefineConcept{负向量},记作\(-\a\).
\end{enumerate}

向量的加法、数乘统称为向量的\DefineConcept{线性运算}.
\end{definition}

\begin{theorem}
由上述定义可知,向量的线性运算满足下面八条运算规律:
\begin{enumerate}
\item \(\a + \b = \b + \a\)
\item \((\a + \b) + \g = \a + (\b + \g)\)
\item \(\a + \z = \a\)
\item \(\a + (-\a) = \z\)
\item \(1 \a = \a\)
\item \(k(l \a) = (kl) \a\)
\item \(k(\a + \b) = k\a + k\b\)
\item \((k+l)\a = k\a + l\a\)
\end{enumerate}
\end{theorem}

\begin{property}
向量的运算还满足以下性质:
\begin{enumerate}
\item \(0 \a = \z\)
\item \((-1) \a = -\a\)
\item \(k \z = \z\)
\item \(k \a = \z \implies k = 0 \lor \a = \z\)
\end{enumerate}
\end{property}

\begin{definition}
对于\(n\)维向量\(\a = (\v{a}{n})\)和\(\b = (\v{b}{n})\).
称向量\(\a\)与向量\(\b\)的负向量\(-\b\)的和为向量\(\a\)与向量\(\b\)的\DefineConcept{差},即\[
\a - \b = \a + (-\b).
\]
\end{definition}

\begin{definition}
两个\(n\)维复向量\(\a=(\v{a}{n})^T\),\(\b=(\v{b}{n})^T\)的\DefineConcept{内积}(inner product)定义为复数\[
a_1b_1 + a_2b_2 + \dotsb + a_nb_n
\]记作\(\vectorinnerproduct{\a}{\b}\).
\end{definition}

\begin{definition}
若向量\(\a\)与\(\b\)满足\(\vectorinnerproduct{\a}{\b}=0\),则称\(\a\)与\(\b\)正交(orthogonal),记作\(\a\perp\b\).
\end{definition}

\begin{property}
向量内积具有以下性质:
\begin{enumerate}
\item \(\vectorinnerproduct{\a}{\b} = \vectorinnerproduct{\b}{\a}\);
\item \(\vectorinnerproduct{(\a+\b)}{\g} = \vectorinnerproduct{\a}{\g} + \vectorinnerproduct{\b}{\g}\);
\item \(\vectorinnerproduct{(k\a)}{\b} = k (\vectorinnerproduct{\a}{\b})\)(\(k\in\mathbb{R}\));
\item \(\a\neq\z \iff \vectorinnerproduct{\a}{\a} > 0\);\(\a=\z \iff \vectorinnerproduct{\a}{\a} = 0\);
\item \(\vectorinnerproduct{\z}{\a} = 0\);
\end{enumerate}
\end{property}

\subsection{向量的长度(模、范数)与单位向量}
\begin{definition}
设\(n\)维向量\(\a = (\v{a}{n})\).
定义向量的\DefineConcept{长度}为\[
\sqrt{\vectorinnerproduct{\a}{\a}} = \sqrt{a_1^2+a_2^2+\dotsb+a_n^2}.
\]同样地可以定义\(n\)维列向量的长度.
2维向量、3维向量的长度常被称作向量的\DefineConcept{模}(module),记作\(\abs{\a}\).
高维(\(n > 3\))向量的长度常被称作向量的\DefineConcept{范数}(norm),记作\(\norm{\a}\).
\end{definition}

\begin{property}
显然有向量的长度为非负实数,即\(\abs{\a}\geqslant0\).
\end{property}

\begin{definition}
长度为1的向量被称为\DefineConcept{单位向量}.
\end{definition}

\begin{definition}
\def\f{\frac{1}{\abs{\a}}}
设\(\a\)满足\(\abs{\a}>0\).
用\(\f\)数乘\(\a\),
称为“将\(\a\) \DefineConcept{单位化}”,
得单位向量\(\f\a\).
\end{definition}

尽管我们通常出于几何(特别是欧式几何)的考量,像上面一样将向量\(\a\)的模(或范数)定义为\(\sqrt{\vectorinnerproduct{\a}{\a}}\),不过我们还可以定义其他形式的模(或范数).
观察上面的模(或范数)的定义,我们可以发现,向量的模(或范数)实际上是满足以下3条性质的映射\(f: P^n \to P, \mat{x} \mapsto m\):
\begin{enumerate}
\item {\bf 非负性},即\(\forall \mat{x} \in P^n : f(\mat{x}) \geqslant 0\);
\item {\bf 齐次性},即\(\forall \mat{x} \in P^n, \forall c \in P : f(c \mat{x}) = \abs{c} f(\mat{x})\);
\item {\bf 三角不等式},即\(\forall \mat{x},\mat{y} \in P^n : f(\mat{x}+\mat{y}) \leqslant f(\mat{x}) + f(\mat{y})\).
\end{enumerate}

据此我们可以定义p范数如下:
\begin{definition}\label{definition:向量与矩阵.p范数}
形如\[
f: \mathbb{R}^n \to \mathbb{R},
\mat{x} = \opair{\v{x}{n}}
\mapsto
\sqrt[p]{\abs{x_1}^p + \abs{x_2}^p + \dotsb + \abs{x_n}^p}
\]的这一类映射\(f(p)\ (p\geqslant1)\),称为\DefineConcept{p范数},记作\(\norm{\mat{x}}_p\).
\end{definition}

易证\[
\begin{array}{lll}
\norm{\mat{x}}_1 &=& \abs{x_1} + \abs{x_2} + \dotsb + \abs{x_n}, \\
\norm{\mat{x}}_2 &=& \sqrt{x_1^2 + x_2^2 + \dotsb + x_n^2}, \\
\norm{\mat{x}}_\infty &=& \max\{\abs{x_1},\abs{x_2},\dotsc,\abs{x_n}\}.
\end{array}
\]

\section{矩阵的概念与运算}
\subsection{矩阵的基本概念}
\begin{definition}[矩阵]
数域\(P\)中\(s \times n\)个数排成的\(s\)行\(n\)列的矩形表,加上方括号:\[
	\begin{bmatrix}
		a_{11} & a_{12} & \dots & a_{1n} \\
		a_{21} & a_{22} & \dots & a_{2n} \\
		\vdots & \vdots & \ddots & \vdots \\
		a_{s1} & a_{s2} & \dots & a_{sn}
	\end{bmatrix}
\]称为数域\(P\)上的\(s \times n\)矩阵(matrix),
通常用一个大写黑体字母如\(\A\)或\(\A_{s \times n}\)表示,
其中\(s\)称为矩阵的\DefineConcept{行数},
\(n\)称为矩阵的\DefineConcept{列数}.
有时矩阵也记作\((a_{ij})_{s \times n}\),
其中\(a_{ij}\)(\(i=1,2,\dotsc,s\);
\(j=1,2,\dotsc,n\))称为矩阵\(\A\)的第\(i\)行第\(j\)列\DefineConcept{元素},
或\(\opair{i,j}\) \DefineConcept{元素}(entry).
“\(\A\)是数域\(P\)上的\(s \times n\)矩阵”%
可以记作\(\A \in M_{s \times n}(P)\)或\(\A \in P^{s \times n}\).

当数域\(P\)上的矩阵\(\A_{s \times n}\)的行数等于列数(即\(s=n\))时,称\[
\A = \begin{bmatrix}
a_{11} & a_{12} & \dots & a_{1n} \\
a_{21} & a_{22} & \dots & a_{2n} \\
\vdots & \vdots & \ddots & \vdots \\
a_{n1} & a_{n2} & \dots & a_{nn}
\end{bmatrix}
\]为\(n\)阶矩阵或\(n\)阶方阵,记作\(\A \in M_n(P)\).称\(a_{11},a_{22},\dotsc,a_{nn}\)为\(\A\)的\DefineConcept{主对角线}上的元素.

如果两个矩阵行数相同且列数相同,则称两者\DefineConcept{同型}.

特别地,可以视\(n\)维行向量为\(1 \times n\)矩阵,\(n\)维列向量为\(n \times 1\)矩阵.
\end{definition}

\subsection{矩阵的线性运算}
\begin{definition}
设\(\A=(a_{ij})_{s \times n}\)和\(\B=(b_{ij})_{s \times n}\)是(数域\(P\)上)两个\(s \times n\)(同型)矩阵.
如果它们对应的元素分别相等,
即\[
	a_{ij} = b_{ij},
	\quad
	i=1,2,\dotsc,s;
	j=1,2,\dotsc,n;
\]
则称“\(\A\)与\(\B\) \DefineConcept{相等}”,
记作\(\A=\B\).
\end{definition}

\begin{definition}
设\(\A=(a_{ij})_{s \times n}\)和\(\B=(b_{ij})_{s \times n}\)是(数域\(P\)上)两个\(s \times n\)(同型)矩阵.
\begin{enumerate}
\item 称矩阵\[
(a_{ij} + b_{ij})_{s \times n} = \begin{bmatrix}
a_{11}+b_{11} & a_{12}+b_{12} & \dots & a_{1n}+b_{1n} \\
a_{21}+b_{21} & a_{22}+b_{22} & \dots & a_{2n}+b_{2n} \\
\vdots & \vdots & \ddots & \vdots \\
a_{s1}+b_{s1} & a_{s2}+b_{s2} & \dots & a_{sn}+b_{sn}
\end{bmatrix}
\]为\(\A\)与\(\B\)的\DefineConcept{和}(sum),记作\(\A+\B\).
\item 设\(k\)为(数域\(P\)中的)数,称矩阵\[
(ka_{ij})_{s \times n} = \begin{bmatrix}
ka_{11} & ka_{12} & \dots & ka_{1n} \\
ka_{21} & ka_{22} & \dots & ka_{2n} \\
\vdots & \vdots & \ddots & \vdots \\
ka_{s1} & ka_{s2} & \dots & ka_{sn}
\end{bmatrix}
\]为数\(k\)与矩阵\(\A\)的\DefineConcept{数乘},记作\(k\A\).
\item 称元素全为零的矩阵为\DefineConcept{零矩阵}(zero matrix),记作\(\z\).
\item 称矩阵\[
(-a_{ij})_{s \times n}=\begin{bmatrix}
-a_{11} & -a_{12} & \dots & -a_{1n} \\
-a_{21} & -a_{22} & \dots & -a_{2n} \\
\vdots & \vdots & \ddots & \vdots \\
-a_{s1} & -a_{s2} & \dots & -a_{sn}
\end{bmatrix}
\]为\(\A\)的\DefineConcept{负矩阵},记作\(-\A\).
\end{enumerate}
\end{definition}

\begin{theorem}
由上述定义可知,矩阵的线性运算满足下面八条运算规律:
\begin{enumerate}
\item \(\A + \B = \B + \A\)
\item \((\A + \B) + \C = \A + (\B + \C)\)
\item \(\A + \z = \A\)
\item \(\A + (-\A) = \z\)
\item \(1 \A = \A\)
\item \(k(l \A) = (kl) \A\)
\item \(k(\A + \B) = k\A + k\B\)
\item \((k+l)\A = k\A + l\A\)
\end{enumerate}
\end{theorem}

\begin{property}
矩阵的运算还满足以下性质:
\begin{enumerate}
\item \(0\A = \z\)
\item \((-1)\A = -\A\)
\item \(k\z = \z\)
\item \(k\A = \z \implies k = 0 \lor \A = \z\)
\end{enumerate}
\end{property}

\subsection{矩阵的乘法}
\begin{definition}
设\(\A = (a_{ij})_{s \times n}\),\(\B = (b_{ij})_{n \times m}\),\(\C = (c_{ij})_{s \times m}\),如果满足\[
c_{ij} = \sum\limits_{k=1}^n {a_{ik} b_{kj}},
\]其中\(i=1,2,\dotsc,s\),\(j=1,2,\dotsc,m\),则称矩阵\(\C\)是\(\A\)与\(\B\)的\DefineConcept{乘积},记作\(\C = \A \B\).
\end{definition}

\begin{example}
设\(\A = (\v{a}{n})^T\),\(\B = (\v{b}{n})\),计算\(\A\B\)和\(\B\A\).
\begin{solution}
\[
\A\B = \begin{bmatrix}
a_1 b_1 & a_1 b_2 & \dots & a_1 b_n \\
a_2 b_1 & a_2 b_2 & \dots & a_2 b_n \\
\vdots & \vdots & \ddots & \vdots \\
a_n b_1 & a_n b_2 & \dots & a_n b_n \\
\end{bmatrix},
\]\[
\B\A = \sum\limits_{i=1}^n{b_i a_i}.
\]
\end{solution}
\end{example}

\begin{definition}
一般地,矩阵乘法不满足交换律.
但如果同阶矩阵\(\A = (a_{ij})_n\)和\(\B = (b_{ij})_n\)的乘积满足交换律,
即\[
	\A \B = \B \A,
\]则称“\(\A\)与\(\B\) \DefineConcept{可交换}”.
\end{definition}

\begin{example}
举例说明非零矩阵的乘积可能是零矩阵.
\begin{solution}
矩阵\[
\A = \begin{bmatrix}
0 & 0 & 0 \\
a_{21} & 0 & 0 \\
a_{31} & a_{32} & 0 \\
\end{bmatrix}
\]和\[
\B = \begin{bmatrix}
b_{11} & b_{12} & b_{13} \\
0 & b_{22} & b_{23} \\
0 & 0 & b_{33} \\
\end{bmatrix}
\]都不是零矩阵,但他们的乘积是零矩阵.
\end{solution}
\end{example}

\begin{example}
举例说明矩阵方程\(\A \B = \A \C\)与\(\A = \z \lor \B = \C\)不等价,即消去律不成立.
\end{example}

\begin{theorem}
矩阵乘法满足结合律,%
即对\(\A = (a_{ij})_{s \times n}\),%
\(\B = (b_{ij})_{n \times m}\),%
\(\C = (c_{ij})_{m \times p}\),%
总有\((\A\B)\C = \A(\B\C)\).
\end{theorem}

\begin{definition}
若对角矩阵\(\A=(a_{ij})_n\)总有\(a_{ii} = 1\)(\(i=1,2,\dotsc,n\))成立,则称\(\A\)为\DefineConcept{单位矩阵},记作\(\E\)或\(\E_n\)或\(\mat{I}\).
\end{definition}

\begin{property}
矩阵的乘法满足以下性质:
\begin{enumerate}
\item \(k(\A\B) = (k\A)\B = \A(k\B)\)(\(k \in P\))
\item 左分配律\(\A(\B+\C) = \A\B + \A\C\)
\item 右分配律\((\A+\B)\C = \A\C + \B\C\)
\item \(\z_{q \times s} \A_{s \times n} = \z_{q \times n}\),\(\A_{s \times n} \z_{n \times p} = \z_{s \times p}\)
\item \(\E_s \A_{s \times n} = \A\),\(\A_{s \times n} \E_n = \A\)
\end{enumerate}
\end{property}

\begin{theorem}
单位矩阵\(\E_n\)与任意\(n\)阶方阵可交换.
\end{theorem}

\begin{definition}
设\(\A\)为\(n\)阶方阵,由乘法结合律,可定义\(\A\)的乘幂.规定
\begin{align*}
\A^0 &= \E \\
\A^k &= \underbrace{\A\A\dotsm\A}_{k\text{个}}
\end{align*}
\end{definition}

\begin{theorem}
指数律\(\A^k\A^l = \A^{k+l}\),\((\A^k)^l = \A^{kl}\)(\(k,l \in \mathbb{N}\))成立.
\end{theorem}


对于同阶方阵\(\A\)、\(\B\),恒有\((\A\B)^k = \A(\B\A)^{k-1}\B\)(\(k \geqslant 1\)).
注意,当同阶方阵\(\A\)、\(\B\)不可交换时,通常有\((\A\B)^k \neq \A^k\B^k\).

\begin{example}
设\(\A,\B,\C \in M_n(P)\),则\[
(\A + \B + \C)^2 = \A^2 + \B^2 + \C^2 + \A\B + \B\A + \A\C + \C\A + \B\C + \C\B.
\]
\end{example}

\begin{theorem}
方阵\(\A\)与\(\A^k\)可交换,与\(f(\A) = \sum\limits_{i=0}^n a_i \A^i\)可交换.
\end{theorem}

\begin{theorem}
如果\(g(x)\)和\(h(x)\)是两个多项式,设\(l(x) = g(x) + h(x)\),\(m(x) = g(x) h(x)\),则\[
l(\A) = g(\A) + h(\A),
\quad
m(\A) = g(\A) h(\A)
\]
\end{theorem}

\begin{example}
设\(\A,\B,\x \in M_n(P)\).证明:若\(\A\x=\x\B\),则对任意多项式\[
f(x) = a_0 + a_1 x + a_2 x^2 + \dotsb + a_k x^k,
\quad
a_0,a_1,a_2,\dotsc,a_k \in P,
\]总有\[
f(\A) \x = \x f(\B).
\]
\begin{proof}
因为\[
\A\x = \x\B,
\]所以\[
\A^2 \x = \A(\A\x) = \A(\x\B),
\]\[
\x \B^2 = (\x\B)\B = (\A\x)\B.
\]以此类推,可证\[
\A^n \x = \x \B^n,
\quad n=1,2,\dotsc.
\]

因为\[
f(\A) = a_0 \E + a_1 \A + a_2 \A^2 + \dotsb + a_k \A^k,
\]根据左分配律,有\begin{gather}
f(\A) \x = a_0 \x + a_1 \A\x + a_2 \A^2 \x + \dotsb + a_k \A^k \x. \tag1
\end{gather}同理,根据右分配律,有\begin{gather}
\x f(\B) = a_0 \x + a_1 \x\B + a_2 \x \B^2 + \dotsb + a_k \x \B^k. \tag2
\end{gather}
因为式1与式2中各项逐项相等,故\(f(\A) \x = \x f(\B)\).
\end{proof}
\end{example}

\subsection{矩阵的转置}
\begin{definition}[矩阵的转置]
将矩阵\(\A=(a_{ij})_{s \times n}\)的行、列互换得到矩阵\(\B=(b_{ij})_{n \times s}\),其中\(b_{ij} = a_{ji}\),那么称矩阵\(\B\)为\(\A\)的\DefineConcept{转置矩阵}(简称\DefineConcept{转置},transpose),记为\(\A^T\)(或\(\A'\)).
\end{definition}
矩阵的转置运算可以看作集合\(P^{s \times n}\)到集合\(P^{n \times s}\)的一个映射.

\begin{definition}[矩阵的共轭转置]
设矩阵\(\A \in M_{s \times n}(\mathbb{C})\).
将\(\A\)转置后,再对各元素取共轭,
称为“\(\A\)的\DefineConcept{共轭转置矩阵}(conjugate transpose)”,
记作\(\A^H\),即\[
    \A^H \defeq \overline{\A^T} \equiv \overline{\A}^T.
\]
\end{definition}

\begin{property}
矩阵的转置满足以下性质:
\begin{enumerate}
\item \((\A^T)^T = \A\);
\item \((\A+\B)^T = \A^T + \B^T\);
\item \((k\A)^T = k\A^T \quad(k \in P)\).
\end{enumerate}
\end{property}

\begin{theorem}\label{theorem:矩阵.矩阵乘积的转置}
对于矩阵\(\A \in P^{s \times n}\)和矩阵\(\B \in P^{n \times m}\),总有\[
(\A\B)^T = \B^T \A^T.
\]
\begin{proof}
\begin{align*}
(\A \B)^T_{\opair{i,j}}
&= (\A \B)_{\opair{j,i}}
= \A_{\opair{j,*}} \B_{\opair{*,i}}
= \sum\limits_{k=1}^n a_{jk} b_{ki}, \\
(\B^T \A^T)_{\opair{i,j}}
&= \B^T_{\opair{i,*}} \A^T_{\opair{*,j}}
= (\B_{\opair{*,i}})^T (\A_{\opair{j,*}})^T
= \sum\limits_{k=1}^n b_{ki} a_{jk}
= \sum\limits_{k=1}^n a_{jk} b_{ki}.
\qedhere
\end{align*}
\end{proof}
\end{theorem}

\begin{corollary}
\((\A_1 \A_2 \dotsb \A_n)^T = \A_n^T \dots \A_2^T \A_1^T\).
\end{corollary}

\begin{example}
设\(\A=\diag(\v{a}{n})\),\(\B=\diag(\v{b}{n})\),试证:\[
\A \B = \diag(a_1 b_1,a_2 b_2,\dotsc,a_n b_n).
\]
\end{example}

\begin{example}
设二阶矩阵\(\A=\begin{bmatrix} 1 & \lambda \\ 0 & 1 \end{bmatrix}\),试证\(\A^k=\begin{bmatrix} 1 & k\lambda \\ 0 & 1 \end{bmatrix}\).
\end{example}

\begin{example}
设\[
\A = \begin{bmatrix}
\cos t & \sin t \\
-\sin t & \cos t
\end{bmatrix}.
\]令\[
\B = \begin{bmatrix}
\cos t & 0 \\
0 & \cos t
\end{bmatrix}, \qquad
\C = \begin{bmatrix}
0 & \sin t \\
-\sin t & 0
\end{bmatrix},
\]则\(\A=\B+\C\).
因为\[
\B\C = \begin{bmatrix}
\cos t & 0 \\
0 & \cos t
\end{bmatrix} \begin{bmatrix}
0 & \sin t \\
-\sin t & 0
\end{bmatrix} = \begin{bmatrix}
0 & \cos t \sin t \\
-\cos t \sin t & 0
\end{bmatrix},
\]\[
\C\B = \begin{bmatrix}
0 & \sin t \\
-\sin t & 0
\end{bmatrix} \begin{bmatrix}
\cos t & 0 \\
0 & \cos t
\end{bmatrix} = \begin{bmatrix}
0 & \cos t \sin t \\
-\cos t \sin t & 0
\end{bmatrix},
\]所以\(\B\C=\C\B\),\(\B\)与\(\C\)可交换.
由牛顿二项式定理有,\[
\A^n=(\B+\C)^n
=\sum\limits_{k=0}^n C_n^k \B^{n-k} \C^k.
\]
\end{example}

\subsection{矩阵的分块}
\begin{definition}
设\(\A\)是\(s \times n\)矩阵,用一些水平线与垂直线(通常不画出来)将\(\A\)分成若干块小矩阵,每一块称为\(\A\)的\DefineConcept{子块},以子块为元素的矩阵叫\DefineConcept{分块阵}.

将\(\A\)按行分块,\[
\A=\begin{bmatrix} \A_1 \\ \A_2 \\ \vdots \\ \A_s \end{bmatrix},
\]其中\(\A_i=(a_{i1},a_{i2},\dotsc,a_{in})\)为\(\A\)的第\(i\)(\(i=1,2,\dotsc,s\))行(向量).

将\(\A\)按列分块,\[
\A=\begin{bmatrix} \a_1 & \a_2 & \dots & \a_n \end{bmatrix},
\]其中\(\a_j=(a_{1j},a_{2j},\dotsc,a_{sj})^T\)为\(\A\)的第\(j\)(\(j=1,2,\dotsc,n\))列(向量).

\(\A\)的分块矩阵的一般形式为\[
\begin{matrix}
& \begin{matrix} n_1 & n_2 & \dots & n_r \end{matrix} \\
\begin{matrix} s_1 \\ s_2 \\ \vdots \\ s_t \end{matrix} & \begin{bmatrix}
\A_{11} & \A_{12} & \dots & \A_{1r} \\
\A_{21} & \A_{22} & \dots & \A_{2r} \\
\vdots & \vdots & \ddots & \vdots \\
\A_{t1} & \A_{t2} & \dots & \A_{tr}
\end{bmatrix}
\end{matrix}
\]
\end{definition}

\begin{theorem}
分块阵的运算服从以下规律:
\begin{enumerate}
\item {\bf 分块阵的加法}

设\(\A,\B \in P^{s \times n}\),
若将\(\A\)、\(\B\)按同样的规则分块为
\[
    \A=(\A_{ij})_{t \times r}, \qquad
    \B=(\B_{ij})_{t \times r},
\]
其中\(\A_{ij}\)、\(\B_{ij}\)都是\(s_i \times n_j\)矩阵
(\(i=1,2,\dotsc,t;\;j=1,2,\dotsc,r\)),
则\[
    \A+\B=(\A_{ij}+\B_{ij})_{t \times r}.
\]

\item {\bf 分块阵的数乘}
\[
    k\A=(k\A_{ij})_{t \times r}.
\]

\item {\bf 分块阵的转置}
设\(\A=(\A_{ij})_{t \times r}\),则\[
    \A^T=(\A_{ji}^T)_{r \times t}.
\]
这就是说,在转置分块阵时,要将每个子块转置.

\item {\bf 分块阵的乘法}
设\(\A \in P^{s \times n}\),\(\B \in P^{n \times m}\),
若将\(\A\)、\(\B\)分别分块为
\[
    \A=(\A_{ij})_{t \times r}, \qquad
    \B=(\B_{jk})_{r \times p},
\]
且\(\A\)的列的分块法与\(\B\)的行的分块法一致,即
\begin{align*}
    \A = \begin{matrix}
        & \begin{matrix} n_1 & n_2 & \dots & n_r \end{matrix} \\
        \begin{matrix} s_1 \\ s_2 \\ \vdots \\ s_t \end{matrix} & \begin{bmatrix}
        \A_{11} & \A_{12} & \dots & \A_{1r} \\
        \A_{21} & \A_{22} & \dots & \A_{2r} \\
        \vdots & \vdots & \ddots & \vdots \\
        \A_{t1} & \A_{t2} & \dots & \A_{tr}
        \end{bmatrix}
    \end{matrix}
    ,\qquad
    \B = \begin{matrix}
        & \begin{matrix} m_1 & m_2 & \dots & m_p \end{matrix} \\
        \begin{matrix} n_1 \\ n_2 \\ \vdots \\ n_r \end{matrix} & \begin{bmatrix}
        \B_{11} & \B_{12} & \dots & \B_{1p} \\
        \B_{21} & \B_{22} & \dots & \B_{2p} \\
        \vdots & \vdots & \ddots & \vdots \\
        \B_{r1} & \B_{r2} & \dots & \B_{rp}
        \end{bmatrix},
    \end{matrix}
\end{align*}
则
\begin{align*}
    \A\B = \begin{matrix}
        & \begin{matrix} m_1 & m_2 & \dots & m_p \end{matrix} \\
        \begin{matrix} s_1 \\ s_2 \\ \vdots \\ s_t \end{matrix} & \begin{bmatrix}
        \C_{11} & \C_{12} & \dots & \C_{1p} \\
        \C_{21} & \C_{22} & \dots & \C_{2p} \\
        \vdots & \vdots & \ddots & \vdots \\
        \C_{t1} & \C_{t2} & \dots & \C_{tp}
        \end{bmatrix}
    \end{matrix}.
\end{align*}
其中\(\C_{ij}=\sum\limits_{k=1}^r \A_{ik} \B_{kj}\ (i=1,2,\dotsc,t;\;j=1,2,\dotsc,p)\).
\end{enumerate}
\end{theorem}

\begin{example}
设\(\A\)、\(\B\)都是\(n\)阶上三角阵,证明:\(\A\B\)是上三角阵.
\begin{proof}
利用数学归纳法.当\(n=1\)时,\(\A=a\),\(\B=b\),\(\A\B = ab\),结论成立.
假设\(n=k\)时上三角阵的乘积是上三角阵.当\(n=k+1\)时,对矩阵\(\A\)和\(b\)分块如下:\[
\A = \begin{bmatrix}
a_{11} & \A_2 \\
\z & \A_4
\end{bmatrix},
\qquad
\B = \begin{bmatrix}
b_{11} & \B_2 \\
\z & \B_4
\end{bmatrix},
\]其中\(\A_4\)和\(\B_4\)都是\(k\)阶上三角阵,由归纳假设,\(\A_4 \B_4\)是\(k\)阶上三角阵,则\[
\A\B = \begin{bmatrix}
a_{11} b_{11} & a_{11} \B_2 + \A_2 \B_4 \\
\z & \A_4 \B_4
\end{bmatrix},
\]即\(\A\B\)是\(k+1\)阶上三角阵.
\end{proof}
\end{example}

\section{特殊矩阵}
\subsection{三角矩阵}
\begin{definition}
设方阵\(\A=(a_{ij})_n\).
如果\(\A\)满足当\(i>j\)时必有\(a_{ij} = 0\)成立,称之为\DefineConcept{上三角形矩阵}(或\DefineConcept{上三角阵}).
如果\(\A\)满足当\(i<j\)时必有\(a_{ij} = 0\)成立,称之为\DefineConcept{下三角形矩阵}(或\DefineConcept{下三角阵}).
\end{definition}

\subsection{对角矩阵}
\begin{definition}
若方阵\(\A=(a_{ij})_n\)满足:
\begin{enumerate}
\item 当\(i \neq j\)时,必有\(a_{ij} = 0\)成立;
\item 存在\(i\)使得\(a_{ii} \neq 0\).
\end{enumerate}
则称\(\A\)为\DefineConcept{对角矩阵},记作\(\diag(a_{11},a_{22},\dotsc,a_{nn})\).
\end{definition}

\subsection{对称矩阵,厄米矩阵}
\begin{definition}
若矩阵\(\A \in M_n(\mathbb{C})\)满足\[
    \A^T = \A,
\]
则称\(\A\)为\DefineConcept{对称矩阵}(symmetric matrix).
\end{definition}

\begin{definition}
如果矩阵\(\A \in M_n(\mathbb{C})\)满足\[
    \A^H = \A,
\]
那么把\(\A\)称为\DefineConcept{厄米矩阵}(Hermitian matrix).
\end{definition}


\begin{example}
设矩阵\(\A = (a_{ij})_n \in \mathbb{R}^{n \times n}\),\(\A \neq \z\),试证:\(\A \A^T\)为对称矩阵,且\(\A \A^T \neq \z\).
\begin{proof}
因为\((\A \A^T)^T = (\A^T)^T \A^T = \A \A^T\),所以\(\A \A^T\)是对称矩阵.设\[
\B = (b_{ij})_n = \A \A^T = \begin{bmatrix}
a_{11} & a_{12} & a_{13} \\
a_{21} & a_{22} & a_{23} \\
a_{31} & a_{32} & a_{33} \\
\end{bmatrix} \begin{bmatrix}
a_{11} & a_{21} & a_{31} \\
a_{12} & a_{22} & a_{32} \\
a_{13} & a_{23} & a_{33} \\
\end{bmatrix},
\]则\[
b_{ij} = a_{i1} a_{j1} + a_{i2} a_{j2} + a_{i3} a_{j3},
\quad i,j=1,2,3.
\]特别地,\(\B\)主对角线上的元素\(b_{ii}\)(\(i=1,2,3\))是实数的平方和,即\[
b_{ii} = a_{i1}^2 + a_{i2}^2 + a_{i3}^2 \geqslant 0,
\quad i=1,2,3.
\]因为\(\A \neq \z\),所以存在\(a_{kl} \neq 0\),从而有\(b_{kk} > 0\),故\(\B = \A \A^T \neq \z\).
\end{proof}
\end{example}

\begin{example}
设\(\A\)和\(\B\)是同阶对称矩阵,试证:\(\A\B\)是对称矩阵的充要条件是\(\A\B = \B\A\).
\begin{proof}
因为\(\A\)和\(\B\)都是对称矩阵,所以\(\A^T = \A\),\(\B^T = \B\).
又因为\(\A\B = \B\A\),\((\A\B)^T = \B^T \A^T = \B\A = \A\B\),即\(\A\B\)是对称矩阵.
\end{proof}
\end{example}

\subsection{反对称矩阵}
\begin{definition}
若方阵\(\A\)满足条件\(\A^T = -\A\),则称\(\A\)为\DefineConcept{反对称矩阵}(antisymmetric matrix)或\DefineConcept{斜对称矩阵}.
\end{definition}

\begin{property}
反对称矩阵主对角线上的元素全为零.
\end{property}

\begin{example}
零矩阵\(\z\)是唯一一个既是实对称矩阵又是实反对称矩阵的矩阵.
\begin{proof}
\(\A^T = \A = -\A \implies 2\A = \A+\A = \z \implies \A = \z\).
\end{proof}
\end{example}

\begin{example}
设\(\A\)是一个方阵,证明:\(\A+\A^T\)为对称矩阵,\(\A-\A^T\)为反对称矩阵.
\begin{proof}
因为\((\A+\A^T)^T = \A^T+\A\),而\((\A-\A^T)^T = \A^T - \A = -(\A-\A^T)\),所以\(\A+\A^T\)为对称矩阵,\(\A-\A^T\)为反对称矩阵.
显然有\(\A = \frac{\A + \A^T}{2} + \frac{\A - \A^T}{2}\).
\end{proof}
\end{example}

\begin{example}
设\(\A\)是3阶实对称矩阵,\(\B\)是3阶实反对称矩阵,\(\A^2 = \B^2\),试证:\(\A = \B = \z\).
\begin{proof}
设\(\A = (a_{ij})_n\),\(\B = (b_{ij})_n\).
因为\(\A = \A^T\),\(\A^2 = \A^T \A\),所以\(\A^2\)的\(\opair{i,j}\)元素为\(a_{1i} a_{1j} + a_{2i} a_{2j} + \dotsb + a_{ni} a_{nj}\).
因为\(\B = -\B^T\),\(\B^2 = -\B^T \B\),所以\(\B^2\)的\(\opair{i,j}\)元素为\(-(b_{1i} b_{1j} + b_{2i} b_{2j} + \dotsb + b_{ni} b_{nj})\).
因为\(\A^2 = \B^2\),所以\[
a_{1i} a_{1j} + a_{2i} a_{2j} + \dotsb + a_{ni} a_{nj}
= -(b_{1i} b_{1j} + b_{2i} b_{2j} + \dotsb + b_{ni} b_{nj}).
\]

当\(i=j\)时,上式变为\(
a_{1i}^2 + a_{2i}^2 + \dotsb + a_{ni}^2
= -(b_{1i}^2 + b_{2i}^2 + \dotsb + b_{ni}^2)
\),又由\(a_{ij},b_{ij} \in \mathbb{R}\)可知\(a_{1i}^2 + a_{2i}^2 + \dotsb + a_{ni}^2 \geqslant 0\),\(-(b_{1i}^2 + b_{2i}^2 + \dotsb + b_{ni}^2) \leqslant 0\),所以\[
a_{1i}^2 + a_{2i}^2 + \dotsb + a_{ni}^2
= -(b_{1i}^2 + b_{2i}^2 + \dotsb + b_{ni}^2) = 0,
\]进而有\[
a_{1i} = a_{2i} = \dotsb = a_{ni} = b_{1i} = b_{2i} = \dotsb = b_{ni} = 0.
\qedhere
\]
\end{proof}
\end{example}

\subsection{幂零矩阵}
\begin{definition}
设矩阵\(\A \in M_n(K)\).
若\(\exists m\in\mathbb{N}^+\),
使得\(\A^m = \z\),
则称“\(\A\)是\DefineConcept{幂零矩阵}”;
称使得\(\A^m = \z\)成立的最小正整数\[
    \min\Set{ m\in\mathbb{N}^+ \given \A^m = \z }
\]为“\(\A\)的\DefineConcept{幂零指数}”.
\end{definition}
