\chapter{矩阵的逆}
\section{可逆矩阵}
\begin{definition}
对于\(n\)阶方阵\(\A\),如果存在\(n\)阶方阵\(\B\),使得\[
\A\B=\B\A=\E
\]则称\(\A\)为\textbf{可逆矩阵},或称\(\A\)\textbf{可逆}(invertible),并称\(\B\)为\(\A\)的\textbf{逆矩阵}(inverse matrix),记作\(\A^{-1}\),即\(\A^{-1} = \B\).
\end{definition}

\begin{definition}
对于\(n\)阶方阵\(\A\),%
若\(\abs{\A}=0\),称\(\A\)为\textbf{奇异矩阵};%
若\(\abs{\A} \neq 0\),称\(\A\)为\textbf{非奇异矩阵}.
\end{definition}

\begin{definition}
\def\Ainv{\A^{-1}}
若\(\A\)是可逆矩阵,那么规定\begin{equation}
\A^{-k} = (\Ainv)^k
= \underbrace{\Ainv\Ainv\dotsm\Ainv}_{k\text{个}}.
\end{equation}
\end{definition}

\begin{theorem}\label{theorem:逆矩阵.矩阵可逆的充要条件1}
设\(\A\)是\(n\)阶方阵,则\(\A\)可逆的充要条件是\(\abs{\A} \neq 0\).
\begin{proof}
必要性.
假设矩阵\(\A\)可逆,那么存在\(n\)阶方阵\(\B\),使得\(\A\B=\E\),于是\(\abs{\A\B}=\abs{\E}\).
而根据\cref{theorem:行列式.矩阵乘积的行列式} ,\(\abs{\A\B}=\abs{\A}\abs{\B}=1\),可知\(\abs{\A}\neq0\).

充分性.
设\(\abs{\A}\neq0\),令\(\B=\frac{1}{\abs{\A}} \A^*\),其中\(\A^*\)是\(\A\)的伴随矩阵.
根据\cref{equation:行列式.伴随矩阵.恒等式1} ,可知\(\A\B = \B\A = \E\),故由定义可知\(\A\)可逆,且\[
\A^{-1} = \B = \frac{1}{\abs{\A}} \A^*.
\qedhere
\]
\end{proof}
\end{theorem}

\begin{property}\label{theorem:逆矩阵.逆矩阵的唯一性}
设\(\A\)是可逆矩阵,则它的逆矩阵唯一,且有\(\A^{-1}=\frac{1}{\abs{\A}}\A^*\).
\begin{proof}
在\cref{theorem:逆矩阵.矩阵可逆的充要条件1} 的证明过程中我们看到矩阵\(\B=\frac{1}{\abs{\A}} \A^*\)是可逆矩阵\(\A\)的一个逆矩阵,即\(\A\B=\E\).
设矩阵\(\C\)也是\(\A\)的逆矩阵,即\(\C\A=\E\),于是\[
\C=\C\E=\C(\A\B)=(\C\A)\B=\E\B=\B.
\]由此可知,任意可逆矩阵的逆矩阵是唯一的.
\end{proof}
\end{property}

\begin{property}\label{theorem:逆矩阵.逆矩阵的对称性}
设\(\A\)、\(\B\)均为\(n\)阶矩阵,满足\(\A\B=\E\),则\(\A\)、\(\B\)互为逆矩阵.
\end{property}

\begin{property}\label{theorem:逆矩阵.逆矩阵的行列式}
设\(\A\)可逆,则\(\abs{\A^{-1}}=\abs{\A}^{-1}\).
\end{property}

\begin{property}\label{theorem:逆矩阵.逆矩阵的逆}
设\(\A\)可逆,则\(\A^{-1}\)可逆,且\((\A^{-1})^{-1}=\A\).
\end{property}

\begin{property}\label{theorem:逆矩阵.矩阵乘积的逆1}
设\(\A\)、\(\B\)都是\(n\)阶可逆矩阵,则\(\A\B\)可逆,且\begin{equation}
(\A \B)^{-1} = \B^{-1} \A^{-1}.
\end{equation}
\end{property}

\cref{theorem:逆矩阵.矩阵乘积的逆1} 可以推广到有限个\(n\)阶可逆矩阵乘积的情形.
\begin{property}\label{theorem:逆矩阵.矩阵乘积的逆2}
设\(\v{\A}{n}\)都是\(n\)阶可逆矩阵,则\(\A_1 \A_2 \dotsm \A_{n-1} \A_n\)可逆,且\begin{equation}
(\A_1 \A_2 \dotsm \A_{n-1} \A_n)^{-1}
= \A_n^{-1} \A_{n-1}^{-1} \dotsm \A_2^{-1} \A_1^{-1}.
\end{equation}
\end{property}

\begin{property}\label{theorem:逆矩阵.数与矩阵乘积的逆}
设\(\A\)可逆,数\(k\neq0\),则\(k\A\)可逆,且\((k\A)^{-1}=k^{-1}\A^{-1}\).
\begin{proof}
假设矩阵\(\A\)是\(n\)阶可逆矩阵.
由\cref{theorem:行列式.性质2.推论2} ,\(\abs{k\A} = k^n\abs{\A}\).
因为\(\A\)可逆,\(\abs{\A}\neq0\),加之\(k\neq0\),于是\(\abs{k\A}\neq0\),即\(k\A\)可逆.\[
(k\A)^{-1} = \frac{1}{\abs{k\A}} (k\A)^*
\]
\end{proof}
\end{property}

\begin{property}\label{theorem:逆矩阵.转置矩阵的逆与逆矩阵的转置}
设\(\A\)可逆,则\(\A^T\)可逆,\(\left(\A^T\right)^{-1}=\left(\A^{-1}\right)^T\).
\begin{proof}
由\cref{theorem:行列式.性质1} ,\(\abs{\A^T}=\abs{\A}\neq0\),于是\(\A^T\)可逆.
由\cref{theorem:矩阵.矩阵乘积的转置} ,\((\A \A^{-1})^T = (\A^{-1})^T \A^T\).
既然\(\A \A^{-1} = \E, \E^T = \E\),于是\((\A^{-1})^T \A^T = \E\),那么由\cref{theorem:逆矩阵.逆矩阵的对称性} 即得\(\left(\A^T\right)^{-1}=\left(\A^{-1}\right)^T\).
\end{proof}
\end{property}

\begin{example}
若\(\A\)、\(\B\)可交换,证明:\(\A^{-1}\)与\(\B\)可交换.
\begin{proof}
因为\(\A\B = \B\A\),在等式两边同时左乘\(\A^{-1}\),得\[
\B = (\A^{-1}\A)\B = \A^{-1}(\A\B) = \A^{-1}(\B\A);
\]再在等式两边右乘\(\A^{-1}\),得\[
\B\A^{-1} = (\A^{-1}\B\A)\A^{-1} = \A^{-1}\B(\A\A^{-1}) = \A^{-1}\B.
\qedhere
\]
\end{proof}
\end{example}

\begin{example}
设\(\A=\diag(\lambda_1,\lambda_2,\dotsc,\lambda_n)\),证明:\(\A^{-1}=\diag(\lambda_1^{-1},\lambda_2^{-1},\dotsc,\lambda_n^{-1})\).
\end{example}

\begin{example}
设\(\A=\begin{bmatrix}
& & & & \lambda_1 \\
& & & \lambda_2 \\
& & \iddots \\
& \lambda_{n-1} \\
\lambda_n
\end{bmatrix}\).
证明:矩阵\[
\B=\begin{bmatrix}
& & & & \lambda_n^{-1} \\
& & & \lambda_{n-1}^{-1} \\
& & \iddots \\
& \lambda_2^{-1} \\
\lambda_1^{-1}
\end{bmatrix}
\]是\(\A\)的逆矩阵.
\begin{proof}
因为\(\A \B = \E\),所以\(\A\)与\(\B\)互为逆矩阵.
\end{proof}
\end{example}

\begin{example}
设\(\A\)可逆.
证明:\(\A\)的伴随矩阵\(\A^*\)可逆,且\[
(\A^*)^{-1}=\frac{1}{\abs{\A}}\A=(\A^{-1})^*.
\]
\begin{proof}
因为\(\A^{-1}=\frac{1}{\abs{\A}} \A^*\),所以\[
(\A^{-1})^{-1} = \left( \frac{1}{\abs{\A}} \A^* \right)^{-1},
\]\[
\A = \abs{\A} (\A^*)^{-1}.
\]又因为\(\A\)可逆,\(\abs{\A}\neq0\),\[
(\A^*)^{-1} = \frac{1}{\abs{\A}} \A.
\]另外,\[
(\A^{-1})^* = \abs{\A^{-1}} (\A^{-1})^{-1}
= \frac{1}{\abs{\A}} \A,
\]所以\[
(\A^*)^{-1}=\frac{1}{\abs{\A}}\A=(\A^{-1})^*.
\qedhere
\]
\end{proof}
\end{example}

\begin{example}
设\(\A\)、\(\B\)都是\(n\)阶可逆矩阵,\(\C\)是\(n\)阶矩阵.
证明:矩阵\[
\mat{M} = \begin{bmatrix}
\A & \C \\
\z & \B
\end{bmatrix}
\]可逆,且\[
\mat{M}^{-1} = \begin{bmatrix}
\A^{-1} & -\A^{-1} \C \B^{-1} \\
\z & \B^{-1}
\end{bmatrix}.
\]
\begin{proof}
因为\(\A\)、\(\B\)为可逆矩阵,\(\abs{\A} \neq 0\),\(\abs{\B} \neq 0\).
所以\(\abs{\mat{M}}=\abs{\A}\abs{\B} \neq 0\),即\(\mat{M}\)可逆.

令\(\mat{M}\mat{X}=\E\),即\[
\begin{bmatrix}
\A & \C \\
\z & \B
\end{bmatrix}\begin{bmatrix}
\mat{X}_1 & \mat{X}_2 \\
\mat{X}_3 & \mat{X}_4
\end{bmatrix} = \begin{bmatrix}
\E & \z \\
\z & \E
\end{bmatrix}
\]则\[
\begin{bmatrix}
\A\mat{X}_1+\C\mat{X}_3 & \A\mat{X}_2+\C\mat{X}_4 \\
\B\mat{X}_3 & \B\mat{X}_4
\end{bmatrix} = \begin{bmatrix}
\E & \z \\
\z & \E
\end{bmatrix}
\]进而有\[
\left\{ \begin{array}{l}
\A\mat{X}_1+\C\mat{X}_3 = \E \\
\A\mat{X}_2+\C\mat{X}_4 = \z \\
\B\mat{X}_3 = \z \\
\B\mat{X}_4 = \E
\end{array} \right.
\]
由第4式可得\(\mat{X}_4 = \B^{-1}\).
代入第2式得\(\A\mat{X}_2=-\C\B^{-1}\),%
\(\mat{X}_2=-\A^{-1}\C\B^{-1}\).
用\(\B^{-1}\)左乘第3式左右两端,\(\B^{-1}\B\mat{X}_3=\mat{X}_3=\z\).
则第1式化为\(\A\mat{X}_1=\E\),显然\(\mat{X}_1=\A^{-1}\),所以\[
\mat{M}^{-1} = \mat{X} = \begin{bmatrix}
\A^{-1} & -\A^{-1}\C\B^{-1} \\
\z & \B^{-1}
\end{bmatrix}.
\qedhere
\]
\end{proof}
\end{example}

\begin{example}
设矩阵\(\P,\Q\)都是可逆矩阵,求矩阵\(\begin{bmatrix}
\z & \P \\ \Q & \z
\end{bmatrix}\)的逆矩阵.
\begin{solution}
假设\[
\begin{bmatrix}
\z & \P \\ \Q & \z
\end{bmatrix} \begin{bmatrix}
\A & \B \\ \C & \D
\end{bmatrix} = \begin{bmatrix}
\P \C & \P \D \\
\Q \A & \Q \B
\end{bmatrix} = \begin{bmatrix}
\E & \z \\ \z & \E
\end{bmatrix},
\]那么可以写出矩阵方程组如下:\[
\left\{ \begin{array}{l}
\P \C = \E, \\
\P \D = \z, \\
\Q \A = \z, \\
\Q \B = \E,
\end{array} \right.
\]解得\[
\A = \z, \qquad
\B = \Q^{-1}, \qquad
\C = \P^{-1}, \qquad
\D = \z;
\]也就是说,矩阵\(\begin{bmatrix}
\z & \P \\ \Q & \z
\end{bmatrix}\)的逆矩阵是\(\begin{bmatrix}
\z & \Q^{-1} \\ \P^{-1} & \z
\end{bmatrix}\).
\end{solution}
\end{example}

\section{初等矩阵}
\subsection{初等变换}
\begin{definition}
对矩阵施行以下变换,称为矩阵的\textbf{初等行变换}(elementary row operation):
\begin{enumerate}
\item 互换两行的位置;
\item 用一非零数\(c\)乘以某行;
\item 将某行的\(k\)倍加到另一行.
\end{enumerate}
类似地,可以定义矩阵的\textbf{初等列变换}(elementary column operation).
矩阵的初等行变换、初等列变换统称为矩阵的\textbf{初等变换}(elementary operation).

有的书上约定:矩阵\(\A\)经过一次初等行变换\(\sigma_1\)化为矩阵\(\B\)的过程可以表示为在连接矩阵\(\A\)和\(\B\)的箭头上方标记\(\sigma_1\),即\[
\A \xlongrightarrow{\sigma_1} \B;
\]而矩阵\(\A\)经过一次初等列变换\(\sigma_2\)化为矩阵\(\B\)的过程可以表示为在连接矩阵\(\A\)和\(\B\)的箭头下方标记\(\sigma_2\),即\[
\A \xlongrightarrow[\sigma_2]{} \B.
\]

当初等变换的效果是互换两行(列)时,记\[
i \rightleftharpoons j.
\]

当初等变换的效果是用一非零数\(c\)乘以第\(i\)行(列)时,记\[
(i) \muleq c
\quad\text{或}\quad
(i) \diveq \frac{1}{c}.
\]

当初等行变换的效果是将第\(j\)行的\(k\)倍加到第\(i\)行时,记\[
(i) \addeq k \times (j)
\quad\text{或}\quad
(i) \subeq -k \times (j).
\]

当初等列变换的效果是将第\(i\)列的\(k\)倍加到第\(j\)列时,记\[
(j) \addeq k \times (i)
\quad\text{或}\quad
(j) \subeq -k \times (i).
\]
\end{definition}

\subsection{初等矩阵的概念}
\begin{definition}\label{definition:逆矩阵.矩阵等价}
若矩阵\(\A\)可以经过一系列初等变换化为矩阵\(\B\),则称矩阵\(\A\)与\(\B\)\textbf{等价}\footnote{在丘维声教授所著的《高等代数》上又将这种关系称作“相抵”.}(equivalent),记作\(\A\cong\B\).
\end{definition}

\begin{definition}
由\(n\)阶单位矩阵\(\E\)经过\emph{一次}初等变换所得矩阵称为\(n\)阶\textbf{初等矩阵}(elementary matrix).
\end{definition}
对应于矩阵的三类初等变换,有三种类型的初等矩阵:
\begin{enumerate}
\item 互换\(\E\)的\(i\),\(j\)两行(列)所得的矩阵\[
\P(i,j) = \begin{bmatrix}
\E_{i-1} & & & \\
& 0 & & 1 & \\
& & \E_{j-i-1} & & \\
& 1 & & 0 & \\
& & & & \E_{n-j}
\end{bmatrix}_n;
\]
\item 用非零数\(c\)乘以\(\E\)的第\(i\)行(列)所得的矩阵\[
\P(i(c)) = \begin{bmatrix}
\E_{i-1} & & \\
& c & \\
& & \E_{n-i}
\end{bmatrix}_n;
\]
\item 把\(\E\)的第\(j\)行(第\(i\)列)的\(k\)倍加到第\(i\)行(第\(j\)列)所得的矩阵\[
\P(i,j(k)) = \begin{bmatrix}
\E_{i-1} & & & \\
& 1 & & k & \\
& & \E_{j-i-1} & & \\
& 0 & & 1 & \\
& & & & \E_{n-j}
\end{bmatrix}_n.
\]
\end{enumerate}

\subsection{初等矩阵的性质}
\begin{property}
初等矩阵具有以下性质:
\begin{enumerate}
\item \(\abs{\P(i,j)}=-1\);
\item \(\abs{\P(i(c))}=c\);
\item \(\abs{\P(i,j(k))}=1\);
\item \(\P(i,j)^T=\P(i,j)\);
\item \(\P(i(c))^T=\P(i(c))\);
\item \(\P(i,j(k))^T=\P(j,i(k))\);
\item \(\P(i,j)^{-1}=\P(i,j)\);
\item \(\P(i(c))^{-1}=\P(i(c^{-1}))\);
\item \(\P(i,j(k))^{-1}=\P(i,j(-k))\).
\end{enumerate}
\end{property}

\begin{property}
对\(n \times t\)矩阵\(\A\)施行一次初等行变换,相当于用一个相应的\(n\)阶初等矩阵左乘\(\A\);
对\(\A\)施行一次初等列变换,相当于用一个相应的\(t\)阶初等矩阵右乘\(\A\).
\begin{proof}
用\(n\)阶矩阵\(\P(i,j)\)左乘\(\A\),将矩阵\(\A\)作相应分块,有\[
\P(i,j) \A = \begin{bmatrix}
\E_{i-1} \\
& 0 & & 1 \\
& & \E_{j-i-1} \\
& 1 & & 0 \\
& & & & \E_{n-j}
\end{bmatrix} \begin{bmatrix}
\A_1 \\ \a_i \\ \A_2 \\ \a_j \\ \A_3
\end{bmatrix} = \begin{bmatrix}
\A_1 \\ \a_j \\ \A_2 \\ \a_i \\ \A_3
\end{bmatrix},
\]即\(\A\)交换\(i\)、\(j\)两行.

用\(n\)阶矩阵\(\P(i(c))\)左乘\(\A\),将矩阵\(\A\)作相应分块,有\[
\P(i(c)) \A = \begin{bmatrix}
\E_{i-1} \\
& c \\
& & \E_{n-i}
\end{bmatrix} \begin{bmatrix}
\A_1 \\ \a_i \\ \A_2
\end{bmatrix} = \begin{bmatrix}
\A_1 \\ c \a_i \\ a_2
\end{bmatrix},
\]即用一非零数\(c\)乘以第\(i\)行.

用\(n\)阶矩阵\(\P(i,j(k))\)(\(i < j\))左乘\(\A\),将矩阵\(\A\)作相应分块,有\[
\P(i,j(k)) \A = \begin{bmatrix}
\E_{i-1} \\
& 1 & & k \\
& & \E_{j-i-1} \\
& 0 & & 1 \\
& & & & \E_{n-j}
\end{bmatrix} \begin{bmatrix}
\A_1 \\ \a_i \\ \A_2 \\ \a_j \\ \A_3
\end{bmatrix} = \begin{bmatrix}
\A_1 \\ \a_i + k \a_j \\ \A_2 \\ \a_j \\ \A_3
\end{bmatrix},
\]即把\(\A\)第\(j\)行的\(k\)倍加到第\(i\)行.
\end{proof}
\end{property}

\begin{property}
初等矩阵可逆.
\end{property}

\begin{theorem}
设\(\A=(a_{ij})_n\),则\(\A\)可逆的充要条件是:\(\A\)可经一系列初等行变换化为单位矩阵\(\E_n\),即\(\A \cong \E_n\).
\begin{proof}
\def\Ps{\P_t \P_{t-1} \dotsm \P_2 \P_1}
存在与\(t\)次初等行变换对应的\(t\)个初等矩阵\(\P_t,\P_{t-1},\dotsc,\P_2,\P_1\),使\[
\A \to \E_n = \Ps \A,
\]则\(\A\)可逆且\(\A^{-1} = \Ps\).

对矩阵\((\A,\E_n)\)作以上初等行变换,则\begin{align*}
(\A,\E_n) \to &\Ps(\A,\E_n) = \A^{-1}(\A,\E_n) \\
&= (\A^{-1}\A,\A^{-1}\E_n) = (\E_n,\A^{-1}).
\qedhere
\end{align*}
\end{proof}
\end{theorem}

\begin{corollary}
如果方阵\(\A\)经\(t\)次初等行变换为\(\E_n\),%
那么同样的初等行变换会将\(\E_n\)变为\(\A^{-1}\).
\end{corollary}

\subsection{应用初等变换求解逆矩阵}
\begin{corollary}
可逆矩阵\(\A\)可以表示成若干个初等矩阵的乘积.
\end{corollary}

\begin{corollary}
\(n\)阶方阵\(\A\)可逆的充要条件是:\(\A\)可经过一系列初等列变换变为\(\E_n\),%
且同样的初等列变换将\(\begin{bmatrix}\A\\\E_n\end{bmatrix}\)变为
\(\begin{bmatrix}\E_n\\\A^{-1}\end{bmatrix}\).
\end{corollary}

\begin{theorem}
设\(\A\)与\(\B\)都是\(s \times n\)矩阵,则\(\A\)与\(\B\)等价的充要条件是:存在\(s\)阶可逆矩阵\(\P\)与\(n\)阶可逆矩阵\(\Q\),使得\(\B=\P\A\Q\).
\end{theorem}

\begin{example}
初等矩阵的逆:
\begin{enumerate}
\item \([\P(i,j)]^{-1}=\P(i,j)\)
\item \([\P(i(c))]^{-1}=\P(i(c^{-1}))\)
\item \([\P(i,j(k))]^{-1}=\P(i,j(-k))\)
\end{enumerate}
\end{example}

\begin{example}
设\(\A\)和\(\B\)都是可逆矩阵,证明:\(\A\)与\(\B\)等价.
\end{example}

\section{广义初等变换}
\subsection{广义初等变换的概念}
\begin{definition}
\def\originalmatrix{%
	\begin{bmatrix}%
	\A & \B \\
	\C & \D
	\end{bmatrix}%
}%
\def\r#1{{\color{red}#1}}%
分块矩阵有以下三种\textbf{广义初等行变换}:
\def\originalmatrixTail{%
	\originalmatrix \begin{matrix} m \\ n \end{matrix}%
}%
\begin{enumerate}
\item 交换两行,\[
\originalmatrixTail \to \begin{bmatrix}
\C & \D \\
\A & \B
\end{bmatrix} = \r{\begin{bmatrix}
\z & \E_n \\
\E_m & \z
\end{bmatrix}} \originalmatrix
\]

\item 用一个可逆矩阵\(\P_m\)左乘某一行,\[
\originalmatrixTail \to \begin{bmatrix}
\P\A & \P\B \\
\C & \D
\end{bmatrix} = \r{\begin{bmatrix}
\P & \z \\
\z & \E_n
\end{bmatrix}} \originalmatrix
\]

\item 用一个矩阵\(\Q_{n \times m}\)左乘某一行后加到另一行,\[
\originalmatrixTail \to \begin{bmatrix}
\A & \B \\
\C+\Q\A & \D+\Q\B
\end{bmatrix} = \r{\begin{bmatrix}
\E_m & \z \\
\Q & \E_n
\end{bmatrix}} \originalmatrix
\]
\end{enumerate}

类似地,有\textbf{广义初等列变换}:
\def\originalmatrixHead{%
	\overset{\begin{matrix} s & t \end{matrix}}{ \originalmatrix }%
}%
\begin{enumerate}
\item 交换两列,\[
\originalmatrixHead \to \begin{bmatrix}
\B & \A \\
\D & \C
\end{bmatrix} = \originalmatrix \r{\begin{bmatrix}
\z & \E_s \\
\E_t & \z
\end{bmatrix}}
\]

\item 用一个可逆矩阵\(\P_t\)右乘某一列,\[
\originalmatrixHead \to \begin{bmatrix}
\A & \B \P \\
\C & \D \P
\end{bmatrix} = \originalmatrix \r{\begin{bmatrix}
\E_s & \z \\
\z & \P
\end{bmatrix}}
\]

\item 用一个矩阵\(\Q_{t \times s}\)右乘某一列后加到另一列,\[
\originalmatrixHead \to \begin{bmatrix}
\A + \B \Q & \B \\
\C + \D \Q & \D
\end{bmatrix} = \originalmatrix \r{\begin{bmatrix}
\E_s & \z \\
\Q & \E_t
\end{bmatrix}}
\]
\end{enumerate}

广义初等行变换与广义初等列变换统称为\textbf{广义初等变换}.
\end{definition}

类比于初等矩阵,我们定义分块初等矩阵如下:
\begin{definition}
将\(n\)阶单位矩阵\(\E\)分为\(m\)块后,进行\emph{一次}广义初等变换所得的矩阵称为\textbf{分块初等矩阵}.
\end{definition}

\begin{property}
分块初等矩阵都是可逆矩阵.
\end{property}

\subsection{Schur定理}
\begin{theorem}[Schur定理]\label{theorem:逆矩阵.Schur定理}
设\(\A\)是\(m\)阶可逆矩阵,\(\B,\C,\D\)分别是\(m \times p, n \times m, n \times p\)矩阵,则有\begin{gather}
\begin{bmatrix}
\E_m & \z \\
-\C\A^{-1} & \E_n
\end{bmatrix} \begin{bmatrix}
\A & \B \\
\C & \D
\end{bmatrix} = \begin{bmatrix}
\A & \B \\
\z & \D - \C \A^{-1} \B
\end{bmatrix},
\\
\begin{bmatrix}
\A & \B \\
\C & \D
\end{bmatrix} \begin{bmatrix}
\E_m & -\A^{-1} \B \\
\z & \E_p
\end{bmatrix} = \begin{bmatrix}
\A & \z \\
\C & \D - \C \A^{-1} \B
\end{bmatrix},
\\
\begin{bmatrix}
\E_m & \z \\
-\C \A^{-1} & \E_n
\end{bmatrix} \begin{bmatrix}
\A & \B \\
\C & \D
\end{bmatrix} \begin{bmatrix}
\E_m & -\A^{-1} \B \\
\z & \E_p
\end{bmatrix} = \begin{bmatrix}
\A & \z \\
\z & \D - \C \A^{-1} \B
\end{bmatrix},
\end{gather}
其中\(\D - \C \A^{-1} \B\)称为矩阵\(\begin{bmatrix} \A & \B \\ \C & \D \end{bmatrix}\)关于\(\A\)的\DefineConcept{Schur补}.
\end{theorem}
可以看到只要\(\A\)可逆,就能通过初等分块矩阵直接对分块矩阵进行分块对角化,而在操作过程中就把一些原本不为0的分块矩阵变成了零矩阵,这个过程可以形象地称为“矩阵打洞”,即让矩阵出现尽可能多的0.

利用\hyperref[theorem:逆矩阵.Schur定理]{Schur定理}可以证明下述行列式降阶定理:
\begin{theorem}[行列式第一降阶定理]\label{theorem:逆矩阵.行列式第一降阶定理}
\def\M{\mat{M}}
设\(\M = \begin{bmatrix}
\A & \B \\
\C & \D
\end{bmatrix}\)是方阵.\begin{enumerate}
\item 若\(\A\)可逆,则\[
\abs{\M} = \abs{\A} \abs{\D - \C \A^{-1} \B};
\]

\item 若\(\D\)可逆,则\[
\abs{\M} = \abs{\D} \abs{\A - \B \D^{-1} \C};
\]

\item 若\(\A,\D\)均可逆,则\[
\abs{\A} \abs{\D - \C \A^{-1} \B}
= \abs{\D} \abs{\A - \B \D^{-1} \C}.
\]
\end{enumerate}
\end{theorem}

\begin{example}
\def\M{\mat{M}}
求解行列式\(\det \M\),其中\[
\M = \begin{bmatrix}
1+a_1 b_1 & a_1 b_2 & \dots & a_1 b_n \\
a_2 b_1 & 1+a_2 b_2 & \dots & a_2 b_n \\
\vdots & \vdots & & \vdots \\
a_n b_1 & a_n b_2 & \dots & 1+a_n b_n
\end{bmatrix}.
\]
\begin{solution}
记\(\a = (\v{a}{n})^T, \b = (\v{b}{n})\),则\(\M = \E_n + \a\b\).
注意到\(\M = \E_n + \a\b\)形似某个矩阵的Schur补,因此考虑下面的矩阵:\[
\A = \begin{bmatrix}
1 & -\b \\
\a & \E_n
\end{bmatrix}.
\]由于\[
\begin{bmatrix}
1 & \z \\
-\a & \E_n
\end{bmatrix} \A
= \begin{bmatrix}
1 & \z \\
-\a & \E_n
\end{bmatrix} \begin{bmatrix}
1 & -\b \\
\a & \E_n
\end{bmatrix} = \begin{bmatrix}
1 & -\b \\
\z & \M
\end{bmatrix},
\]且\[
\begin{bmatrix}
1 & \b \\
\z & \E_n
\end{bmatrix} \A
= \begin{bmatrix}
1 & \b \\
\z & \E_n
\end{bmatrix} \begin{bmatrix}
1 & -\b \\
\a & \E_n
\end{bmatrix} = \begin{bmatrix}
1+\a\b & \z \\
\a & \E_n
\end{bmatrix},
\]所以\[
\abs{\A} = \abs{\M} = \abs{1+\a\b}
= 1+\a\b = 1 + \sum\limits_{k=1}^n a_k b_k.
\]
\end{solution}
\end{example}
