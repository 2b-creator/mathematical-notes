\section{实数域上的不可约多项式}
这一节我们要找出实数域上的所有不可约多项式.
由于每一个复数都可以表示成\(a+b\iu\)的形式,
其中\(a,b\)都是实数,
因此我们可以利用复数域上多项式的信息来研究实数域上的不可约多项式.

\begin{theorem}
%@see: 《高等代数(第三版 下册)》(丘维声) P40 定理1
设\(f(x)\)是实系数多项式,
如果\(c\)是\(f(x)\)的一个复根,
则\(c\)的共轭复数\(\overline{c}\)也是\(f(x)\)的一个复根.
\begin{proof}
设\(f(x)=a_n x^n+a_{n-1} x^{n-1}+\dotsb+a_1 x+a_0\),
其中\(a_i\in\mathbb{R}\ (i=0,1,\dotsc,n)\).
因为\(c\)是\(f(x)\)的复根,
所以\[
	f(c)=a_n c^n+a_{n-1} c^{n-1}+\dotsb+a_1 c+a_0=0.
\]
在上式两边取共轭,得\[
	a_n \overline{c}^n+a_{n-1} \overline{c}^{n-1}+\dotsb+a_1 \overline{c}+a_0=0,
\]
因此\(\overline{c}\)是\(f(x)\)的一个复根.
\end{proof}
\end{theorem}
