\section{多项式}
\begin{definition}
设\(K\)是一个数域,
\(x\)是一个不属于\(K\)的符号.
任意给定一个非负整数\(n\),
在\(K\)中任意取定\(\AutoTuple{a}[0]{n}\),
如果表达式\begin{equation}
	a_n x^n + a_{n-1} x^{n-1} + \dotsb + a_1 x + a_0
\end{equation}
满足\begin{enumerate}
	\item \(\begin{aligned}[t]
		&a_n x^n + a_{n-1} x^{n-1} + \dotsb + a_1 x + a_0
		= b_n y^n + b_{n-1} y^{n-1} + \dotsb + b_1 y + b_0 \\
		&\iff
		(\forall i\in\Set{0,1,\dotsc,n})[a_i = b_i = 0 \lor a_i x^i = b_i y^i].
	\end{aligned}\)
	\item 允许从表达式中删去系数为零的项,也允许向表达式中添进系数为零的项,
\end{enumerate}
那么称之为“数域\(K\)上的一个一元\DefineConcept{多项式}”,
把\(x\)称为\DefineConcept{不定元}.
\end{definition}

\section{贝祖定理}
\begin{theorem}
设\(a,b\in\mathbb{Z}\),则\(a\)与\(b\)互素的充要条件是:
存在\(u,v\in\mathbb{Z}\),使得\[
	u a + v b = 1.
\]
\end{theorem}

\begin{corollary}
设\(f(x)\)和\(g(x)\)是\(\mathbb{P}[x]\)中两个不全为0的多项式,
则\(f(x)\)与\(g(x)\)互素(即\(f(x)\)与\(g(x)\)在\(\mathbb{C}\)上没有公共根)的充要条件是:
存在\(u(x),v(x)\in\mathbb{P}[x]\),使得\[
	u(x) f(x) + v(x) g(x) = 1.
\]
\end{corollary}

\begin{example}
设矩阵\(\A\)满足\(\A^3+\E=2\A\),其中\(\E\)是单位矩阵,
证明:\(2\A^2+\A-\E\)可逆.
\begin{proof}
令\(f(x)=x^3-2x+1\),\(g(x)=2x^2+x-1\),
因式分解可得\[
	f(x) = (x-1)(x^2+x-1),
\]\[
	g(x) = (2x-1)(x+1).
\]
显然\(f(x)\)与\(g(x)\)在\(\mathbb{C}\)上没有公共根,互素.
故根据贝祖定理,存在\(u(x),v(x)\in\mathbb{P}[x]\),使得\[
u(x) \cdot (x^3-2x+1) + v(x) \cdot (2x^2+x-1) = 1,
\]代入矩阵\(\A\),并注意到\(\A^3-2\A+\E=\z\),得到\[
v(\A) \cdot (2\A^2+\A-\E) = \E,
\]也就是说,矩阵\(2\A^2+\A-\E\)可逆,其逆矩阵为\(v(\A)\),而\(v(\A)\)可以通过辗转相除法得到.
\end{proof}
\end{example}

\begin{example}
设\(\A\)是数域\(\mathbb{P}\)上的\(n\)阶方阵,证明:若\(\A^2=\E\),则\[
\rank(\A+\E)+\rank(\A-\E)=n.
\]
\begin{proof}
由于\(x+1\)与\(x-1\)互素,存在\(u(x),v(x)\in\mathbb{P}[x]\),使得\[
u(x) \cdot (x+1) + v(x) \cdot (x-1) = 1.
\]代入矩阵\(\A\),得\[
u(\A) (\A+\E) + v(\A) (\A-\E) = \E.
\]

考虑\(2n\)阶方阵\begin{align*}
\begin{bmatrix}
\A+\E & \z \\
\z & \A-\E
\end{bmatrix}
&\xlongrightarrow{\text{(2列)}+=u(\A)\times\text{(1列)}} \begin{bmatrix}
\A+\E & u(\A) (\A+\E) \\
\z & \A-\E
\end{bmatrix} \\
&\xlongrightarrow{\text{(1行)}+=v(\A)\times\text{(2行)}} \begin{bmatrix}
\A+\E & \E \\
\z & \A-\E
\end{bmatrix} \\
&\to \begin{bmatrix}
\z & \E \\
\A^2-\E & \z
\end{bmatrix} = \begin{bmatrix}
\z & \E \\
\z & \z
\end{bmatrix}.
\end{align*}
于是\(\rank(\A+\E)+\rank(\A-\E)=n\).
\end{proof}
\end{example}
