\section{多项式的根}\label{section:多项式.多项式的根}
从唯一因式分解定理知道,
\(K[x]\)中每一个次数大于零的多项式都能唯一地分解成数域\(K\)上有限多个不可约多项式的乘积.
由此看出,不可约多项式之于\(K[x]\)正如砖块之于城市,
这促使我们取搞清楚\(K[x]\)中不可约多项式有哪些.
我们已经知道,\(K[x]\)中每一个一次多项式都是不可约的.
于是需要进一步研究的是,
\(K[x]\)中有没有次数大于\(1\)的不可约多项式?
显然,在\(K[x]\)中,如果\(p(x)\)是次数大于\(1\)的不可约多项式,则\(p(x)\)没有一次因式.
从这点受到启发,首先需要研究\(K[x]\)中一个多项式\(f(x)\)有一次因式的充分必要条件.
为此,我们需要用一次多项式去除\(f(x)\),观察它的余式.

\begin{theorem}[余数定理]\label{theorem:多项式.余数定理}
%@see: 《高等代数(第三版 下册)》(丘维声) P33 定理1
在\(K[x]\)中,用\(x-a\)去除\(f(x)\)所得的余式是\(f(a)\).
\begin{proof}
作带余除法,得\[
	f(x) = h(x) (x-a) + r(x), \qquad
	\deg r(x) < \deg(x-a)=1.
\]
可见\(r(x)\)要么是零多项式,要么是零次多项式.
不妨设\(r(x)=r \in K\).
于是上式成为\[
	f(x) = h(x) (x-a) + r, \qquad
	r \in K.
\]
在上式中,\(x\)用\(a\)代入,得\(f(a) = r\).
因此用\(x-a\)去除\(f(x)\)所得的余式是\(f(a)\).
\end{proof}
\end{theorem}

\begin{corollary}\label{theorem:多项式.余数定理的推论}
%@see: 《高等代数(第三版 下册)》(丘维声) P34 推论2
在\(K[x]\)中,\(x-a\)整除\(f(x)\)当且仅当\(f(a)=0\).
\begin{proof}
由\cref{theorem:多项式.带余除法.推论}
和\cref{theorem:多项式.余数定理}
立即可得.
\end{proof}
\end{corollary}

从\cref{theorem:多项式.余数定理的推论} 受到启发,引出多项式的根的概念.

\begin{definition}\label{theorem:多项式.根的定义}
%@see: 《高等代数(第三版 下册)》(丘维声) P34 定义1
设\(K\)是一个数域,
\(R\)是一个有单位元的交换环,
且\(R\)可看成是\(K\)的一个扩环.
对于\(K[x]\)中一个多项式\(f(x)\),
如果\(R\)中有一个元素\(c\)使得\(f(c)=0\),
则称“\(c\)是\(f(x)\)在\(R\)中的一个\DefineConcept{根}”.
\end{definition}

多项式在复数域中的根称为\DefineConcept{复根}.
实系数多项式在实数域中的根称为\DefineConcept{实根}.
有理系数多项式在有理数域中的根称为\DefineConcept{有理根}.

从\cref{theorem:多项式.根的定义} 和\cref{theorem:多项式.余数定理的推论} 立即得到下述重要结论:
\begin{theorem}[贝祖定理]\label{theorem:多项式.贝祖定理}
%@see: 《高等代数(第三版 下册)》(丘维声) P34 定理3
在\(K[x]\)中,\(x-a\)整除\(f(x)\)当且仅当\(a\)是\(f(x)\)在\(K\)中的一个根.
\end{theorem}

从\cref{theorem:多项式.贝祖定理} 看出,
\(K[x]\)中的多项式\(f(x)\)有一次因式的充分必要条件是\(f(x)\)在\(K\)中有根.

利用根与一次因式的关系,
对于K[x]中的多项式在K中的根,我们可以定义“重根”的概念:
如果\(x-a\)是\(f(x)\)的\(k\)重因式,
那么我们把\(a \in K\)称为“\(f(x) \in K[x]\)的一个\(k\)重根”.
当\(k=1\)时,\(a\)称为\DefineConcept{单根};
当\(k>1\)时,\(a\)称为\DefineConcept{重根}.

另外,再次利用根与一次因式的关系,
我们还可以得到\(K[x]\)中的多项式在\(f(x)\)在\(K\)中的根的数目的一个上界:
\begin{theorem}\label{theorem:多项式.根的数目的上界}
%@see: 《高等代数(第三版 下册)》(丘维声) P34 定理4
\(K[x]\)中的\(n\ (n\geq0)\)次多项式在\(K\)中至多有\(n\)个根(重根按重数计算).
\begin{proof}
零次多项式没有根,因此结论成立.

设多项式\(f(x)\)的次数为\(\deg f(x)=n>0\).
把\(f(x)\)分解成不可约多项式的乘积.
根据\cref{theorem:多项式.贝祖定理} 以及根的重数的定义容易看出,
\(f(x)\)在\(K\)中的根的数目(重根按重数计算)
等于分解式中一次因式的数目(重因式按重数计算),
这个数目当然不超过\(f(x)\)的次数\(n\).
\end{proof}
\end{theorem}

从\cref{theorem:多项式.根的数目的上界} 可以得到一个重要推论:
\begin{corollary}\label{theorem:多项式.根的数目的上界.推论}
%@see: 《高等代数(第三版 下册)》(丘维声) P34 推论5
设\(K[x]\)中两个多项式\(f(x)\)与\(g(x)\)的次数都不超过\(n\).
如果\(K\)中有\(n+1\)个不同元素\(\AutoTuple{a}{n+1}\),
使得\(f(a_i)=g(a_i)\ (i=1,2,\dotsc,n,n+1)\),
则\(f(x)=g(x)\).
\begin{proof}
设\(h(x)=f(x)-g(x)\),
假设\(h(x)\neq0\),
则\[
	0 \leq \deg h \leq \max\{\deg f,\deg g\} \leq n.
\]
因为\[
	h(a_i) = f(a_i) - g(a_i) = 0,
	\quad i=1,2,\dotsc,n+1,
\]
所以\(h(x)\)在\(K\)中至少有\(n+1\)个不同的根.
这与\cref{theorem:多项式.根的数目的上界} 矛盾.
因此\(h(x)=0\).
于是\(f(x)=g(x)\).
\end{proof}
\end{corollary}

为了研究多项式的根,我们需要多项式函数的概念,并且研究多项式函数与多项式之间的关系.

设\(f \in K[x]\).
对于\(K\)中每一个元素\(a\),
\(x\)用\(a\)代入得\(f(a) \in K\).
于是\(K[x]\)中的一个多项式\(f(x)\)确定了\(K\)到\(K\)的一个映射\[
    f\colon K \to K, a \mapsto f(a).
\]
这种由\(K[x]\)中的多项式确定的\(K\)上的函数称为\(K\)上的\DefineConcept{一元多项式函数}.

我们已经知道,\(K[x]\)中的每一个多项式都确定一个\(K\)上的一元多项式函数.
现在要问:给定\(K[x]\)中两个不相等的多项式\(f(x)\)与\(g(x)\),
它们确定的\(K\)上的一元多项式函数\(f\)和\(g\)是否不相等?
\begin{theorem}\label{theorem:多项式.多项式函数是否相等取决于多项式是否相等}
%@see: 《高等代数(第三版 下册)》(丘维声) P35 定理6
数域\(K\)上的两个多项式\(f(x)\)与\(g(x)\)如果不相等,
则它们确定的\(K\)上的一元多项式函数\(f\)与\(g\)也不相等.
\begin{proof}
证逆否命题.
设\(K\)上的一元多项式函数\(f\)与\(g\)相等,
则对\(\forall a \in K\),有\(f(a)=g(a)\).
由于\(K\)是数域,它有无穷多个元素,
于是根据\cref{theorem:多项式.根的数目的上界.推论} 得,
\(f(x)=g(x)\),
即多项式\(f(x)\)与\(g(x)\)相等.
\end{proof}
\end{theorem}

%@see: 《高等代数(第三版 下册)》(丘维声) P36 定理7
我们把数域\(K\)上的所有一元多项式函数组成的集合也记作\(K[x]\).
让一元多项式环\(K[x]\)中的多项式\(f(x)\)对应到它确定的\(K\)上的函数\(f\),
这时从一元多项式环\(K[x]\)到一元多项式函数族\(K[x]\)的一个映射,这里记为\(\sigma\).
显然\(\sigma\)是满射;
根据\cref{theorem:多项式.多项式函数是否相等取决于多项式是否相等},\(\sigma\)又是单射;
因此\(\sigma\)是双射,是一个从一元多项式环\(K[x]\)到一元多项式函数族\(K[x]\)的同构.
