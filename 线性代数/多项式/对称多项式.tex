\section{对称多项式}
观察下述三元多项式\(f(x_1,x_2,x_3)\)有什么特点?
\[
	f(x_1,x_2,x_3)
	=x_1^3+x_2^3+x_3^3
	+x_1^2x_2
	+x_1^2x_3
	+x_2^2x_3
	+x_1x_2^2
	+x_1x_3^2
	+x_2x_3^2.
\]
直观上看,
\(x_1,x_2,x_3\)在\(f(x_1,x_2,x_3)\)中的地位是对称的,
即同时有\(x_1^3,x_2^3,x_3^3\)这三项,
且同时有\(x_1^2x_2,
x_1^2x_3,
x_2^2x_3,
x_1x_2^2,
x_1x_3^2,
x_2x_3^2\)这六项.
由此受到启发,
我们来研究具有这种性质的\(n\)元多项式\(f(x_1,\dotsc,x_n)\):
若\(f(x_1,\dotsc,x_n)\)含有一项\(a x_1^{i_1} \dotsm x_n^{i_n}\),
则它也含有一项\(a x_{j_1}^{i_1} \dotsm x_{j_n}^{i_n}\),
其中\(j_1 \dotso j_n\)是任意一个\(n\)元排列.

于是我们抽象出下述概念.
\begin{definition}
%@see: 《高等代数(第三版 下册)》(丘维声) P57 定义1
设\(f(x_1,\dotsc,x_n)\)是数域\(K\)上的一个\(n\)元多项式.
如果对于任意一个\(n\)元排列\(j_1 \dotso j_n\)都有\[
	f(x_{j_1},\dotsc,x_{j_n})
	=f(x_1,\dotsc,x_n),
\]
则称“\(f(x_1,\dotsc,x_n)\)是数域\(K\)上的一个\(n\)元\DefineConcept{对称多项式}”.
\end{definition}

定义表明,
在数域\(K\)上的\(n\)元多项式环\(K[x_1,\dotsc,x_n]\)中,
对于\(f(x_1,\dotsc,x_n)\),
如果任给一个\(n\)元排列\(j_1 \dotso j_n\),
不定元\(x_1,\dotsc,x_n\)用\(x_{j_1},\dotsc,x_{j_n}\)代入,
都有\(f(x_{j_1},\dotsc,x_{j_n})=f(x_1,\dotsc,x_n)\),
那么\(n\)元多项式\(f(x_1,\dotsc,x_n)\)是一个对称多项式.

容易看出,零多项式和零次多项式都是对称多项式.

在\(K[x_1,\dotsc,x_n]\)中,
我们来构造含有项\(x_1\)且项数最少的对称多项式.
由定义可知,
\(x_1+\dotsb+x_n\)就是\(n\)元对称多项式,
把它记作\(\sigma_1(x_1,\dotsc,x_n)\),
即\[
	\sigma_1(x_1,\dotsc,x_n)
	=x_1+\dotsb+x_n.
\]
我们来构造含有项\(x_1x_2\)且项数最少的对称多项式.
令\begin{align*}
	\sigma_2(x_1,\dotsc,x_n)
	&=\begin{array}[t]{l}
		x_1x_2+x_1x_3+\dotsb+x_1x_n \\
		+x_2x_3+\dotsb+x_2x_n
		+\dotsb
		+x_{n-1}x_n
	\end{array} \\
	&=\sum_{1\leq i<j\leq n} x_i x_j,
\end{align*}
则\(\sigma_2(x_1,\dotsc,x_n)\)是\(n\)元对称多项式.
同理,对于\(\forall k\in\{2,\dotsc,n-1\}\),
我们来构造含有项\(x_1 \dotsm x_k\),
且项数最少得对称多项式.
令\[
	\sigma_k(x_1,\dotsc,x_n)
	=\sum_{1\leq j_1<\dotsb<j_k\leq n}
	x_{j_1} \dotsm x_{j_k},
\]
则\(\sigma_k(x_1,\dotsc,x_n)\)是\(n\)元对称多项式.
最后,根据定义有,\[
	\sigma_n(x_1,\dotsc,x_n)
	=x_1 \dotsm x_n
\]是\(n\)元对称多项式.

我们把上述\(n\)个\(n\)元对称多项式
\(\sigma_i(x_1,\dotsc,x_n)\ (i=1,\dotsc,n)\)
统称为\(n\)元\DefineConcept{初等对称多项式}.

下面我们把数域\(K\)上所有\(n\)元对称多项式组成的集合记为\(W\).
我们想要知道\(W\)的结构是怎样的.

\begin{proposition}
%@see: 《高等代数(第三版 下册)》(丘维声) P58 命题1
\(W\)是\(K[x_1,\dotsc,x_n]\)的一个子环.
\end{proposition}

\begin{proposition}
%@see: 《高等代数(第三版 下册)》(丘维声) P59 命题2
设\(f_1,\dotsc,f_n \in W\),
则对\(K[x_1,\dotsc,x_n]\)中任意一个多项式\[
	g(x_1,\dotsc,x_n)
	=\sum_{i_1,\dotsc,i_n}
	b_{i_1 \dotso i_n}
	x_1^{i_1} \dotsm x_n^{i_n},
\]
有\[
	g(f_1,\dotsc,f_n)
	=\sum_{i_1,\dotsc,i_n}
	b_{i_1 \dotso i_n}
	f_1^{i_1} \dotsm f_n^{i_n}
	\in W.
\]
\end{proposition}
