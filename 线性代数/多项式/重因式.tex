\section{重因式}
上一节我们已证明\(K[x]\)中每一个次数大于零的多项式\(f(x)\)能唯一地分解成
数域\(K\)上有限多个不可约多项式的乘积.
如果\(f(x)\)的分解式中每一个不可约因式只出现\(1\)次,
这种情形是特别重要的情形.
这一节我们要给出识别这种情形的一个统一的方法.

\begin{definition}
%@see: 《高等代数(第三版 下册)》(丘维声) P29. 定义1
设\(f \in K[x]\).
如果不可约多项式\(p(x)\)满足\[
	p^k(x) \mid f(x)
	\quad\text{且}\quad
	p^{k+1}(x) \nmid f(x),
\]
那么把\(p(x)\)称为“\(f(x)\)的\(k\)重因式”.

如果\(k=0\),则\(p(x) \nmid f(x)\),因此\(p(x)\)不是\(f(x)\)的因式.
如果\(k=1\),则把\(p(x)\)称为\(f(x)\)的\DefineConcept{单因式}.
如果\(k>1\),则把\(p(x)\)称为\(f(x)\)的\DefineConcept{重因式}.
\end{definition}
