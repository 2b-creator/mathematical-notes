\section{整除性,带余除法}
从一元多项式环的通用性质看到,
我们应当尽可能多地得到\(K[x]\)中有关加法和乘法的等式,
为此需要研究一元多项式环\(K[x]\)的结构.
从本节开始我们将主要研究\(K[x]\)的结构,其中\(K\)是任一数域.

观察\(K[x]\)中两个多项式\(f(x)\)与\(g(x)\)之间有什么关系:\[
	f(x)=x^2-1, \qquad
	g(x)=x-1.
\]
显然,\[
	f(x)=(x+1) g(x).
\]
由此我们抽象出“整除”的概念.

\begin{definition}
%@see: 《高等代数(第三版 下册)》(丘维声) P10. 定义1
设\(f,g \in K[x]\).
如果存在\(h \in K[x]\),使得\[
	f(x) = h(x) g(x),
\]
则称“\(g(x)\) \DefineConcept{整除} \(f(x)\)”,
记作\(g(x) \vert f(x)\),
又称“\(g(x)\)是\(f(x)\)的\DefineConcept{因式}”
“\(f(x)\)是\(g(x)\)的\DefineConcept{倍式}”;
否则称“\(g(x)\)不能整除\(f(x)\)”.
\end{definition}

容易看出下列事实:
\begin{enumerate}
	\item \(0 \vert f(x) \iff f(x) = 0\).
	\item \((\forall f \in K[x])[f(x) \vert 0]\).
	\item \((\forall b \in K - \{0\})(\forall f \in K[x])[b \vert f(x)]\).
\end{enumerate}

\begin{definition}
%@see: 《高等代数(第三版 下册)》(丘维声) P10. 定义2
在\(K[x]\)中,如果\(f(x) \vert g(x)\)且\(g(x) \vert f(x)\),
则称“\(f(x)\)与\(g(x)\) \DefineConcept{相伴}”,
记作\(f(x) \sim g(x)\).
\end{definition}

\begin{proposition}
%@see: 《高等代数(第三版 下册)》(丘维声) P10. 命题1
在\(K[x]\)中,\(f(x) \sim g(x)\)当且仅当存在\(c \in K-\{0\}\),使得\[
	f(x) = c g(x).
\]
\end{proposition}

\begin{proposition}
%@see: 《高等代数(第三版 下册)》(丘维声) P10. 命题2
在\(K[x]\)中,如果\(g(x) \vert f_i(x)\ (i=1,2,\dotsc,s)\),
则对于任意\(u_i \in K[x]\ (i=1,2,\dotsc,s)\),有\[
	g(x) \vert (u_1(x) f_1(x) + u_2(x) f_2(x) + \dotsb + u_s(x) f_s(x)).
\]
\end{proposition}

\begin{theorem}
%@see: 《高等代数(第三版 下册)》(丘维声) P11. 定理3
对于\(K[x]\)中任意两个多项式\(f(x)\)与\(g(x)\),其中\(g(x)\neq0\),
则在\(K[x]\)中存在唯一的一对多项式\(h(x),r(x)\),使得\[
	f(x)=h(x) g(x) + r(x)
	\land
	\deg r(x) < \deg g(x).
\]
\end{theorem}
