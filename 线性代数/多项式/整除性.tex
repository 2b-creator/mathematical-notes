\section{整除性,带余除法}
从一元多项式环的通用性质看到,
我们应当尽可能多地得到\(K[x]\)中有关加法和乘法的等式,
为此需要研究一元多项式环\(K[x]\)的结构.
从本节开始我们将主要研究\(K[x]\)的结构,其中\(K\)是任一数域.

观察\(K[x]\)中两个多项式\(f(x)\)与\(g(x)\)之间有什么关系:\[
	f(x)=x^2-1, \qquad
	g(x)=x-1.
\]
显然,\[
	f(x)=(x+1) g(x).
\]
由此我们抽象出“整除”的概念.

\begin{definition}
%@see: 《高等代数(第三版 下册)》(丘维声) P10 定义1
设\(f,g \in K[x]\).
如果存在\(h \in K[x]\),使得\[
	f(x) = h(x) g(x),
\]
则称“\(g(x)\) \DefineConcept{整除} \(f(x)\)”,
记作\(g(x) \mid f(x)\),
又称“\(g(x)\)是\(f(x)\)的\DefineConcept{因式}”
“\(f(x)\)是\(g(x)\)的\DefineConcept{倍式}”;
否则称“\(g(x)\)不能整除\(f(x)\)”,
记作\(g(x) \nmid f(x)\).
\end{definition}

容易看出下列事实:
\begin{enumerate}
	\item \(0 \mid f(x) \iff f(x) = 0\).
	\item \((\forall f \in K[x])[f(x) \mid 0]\).
	\item \((\forall b \in K - \{0\})(\forall f \in K[x])[b \mid f(x)]\).
\end{enumerate}

\begin{example}
证明:整除关系具有传递性,即在\(K[x]\)中,\[
	f(x) \mid g(x) \land g(x) \mid h(x)
	\implies
	f(x) \mid h(x).
\]
%TODO
\end{example}

\begin{definition}
%@see: 《高等代数(第三版 下册)》(丘维声) P10 定义2
在\(K[x]\)中,如果\(f(x) \mid g(x)\)且\(g(x) \mid f(x)\),
则称“\(f(x)\)与\(g(x)\) \DefineConcept{相伴}”,
记作\(f(x) \sim g(x)\).
\end{definition}

\begin{proposition}
%@see: 《高等代数(第三版 下册)》(丘维声) P10 命题1
在\(K[x]\)中,\(f(x) \sim g(x)\)当且仅当存在\(c \in K-\{0\}\),使得\[
	f(x) = c g(x).
\]
\begin{proof}
充分性.
假设\(f(x)=c g(x)\),其中\(c \in K-\{0\}\).
显然有\(g(x) \mid f(x)\).
又因为\(g(x)=\frac1c f(x)\),
所以\(f(x) \mid g(x)\).
因此\(f(x) \sim g(x)\).

必要性.
假设\(f(x) \sim g(x)\).
由定义有\(f(x) \mid g(x)\)和\(g(x) \mid f(x)\).
于是存在\(h_1(x),h_2(x) \in K[x]\),
使得\[
	g(x) = h_1(x) f(x), \qquad
	f(x) = h_2(x) g(x).
\]
于是\[
	f(x) = h_2(x) h_1(x) f(x).
\]
如果\(f(x)=0\),则\(g(x)=0\).
下面假设\(f(x)\neq0\).
运用消去律,
由上式可得\[
	1 = h_2(x) h_1(x).
\]
继而可得\[
	\deg h_2(x) + \deg h_1(x) = 0.
\]
因此\(\deg h_1(x) = \deg h_2(x) = 0\),
从而\(h_2(x)\)等于\(K\)中某个非零常数\(c\),
于是\(f(x) = c g(x)\).
\end{proof}
\end{proposition}

\begin{proposition}
%@see: 《高等代数(第三版 下册)》(丘维声) P10 命题2
在\(K[x]\)中,如果\(g(x) \mid f_i(x)\ (i=1,2,\dotsc,s)\),
则对于任意\(u_i \in K[x]\ (i=1,2,\dotsc,s)\),有\[
	g(x) \mid (u_1(x) f_1(x) + u_2(x) f_2(x) + \dotsb + u_s(x) f_s(x)).
\]
\end{proposition}

\begin{theorem}\label{theorem:多项式.带余除法}
%@see: 《高等代数(第三版 下册)》(丘维声) P11 定理3
对于\(K[x]\)中任意两个多项式\(f(x)\)与\(g(x)\),其中\(g(x)\neq0\),
则在\(K[x]\)中存在唯一的一对多项式\(h(x),r(x)\),使得\[
	f(x)=h(x) g(x) + r(x)
	\land
	\deg r(x) < \deg g(x).
\]
\end{theorem}

\cref{theorem:多项式.带余除法} 中的\(h(x)\)称为“\(g(x)\)除\(f(x)\)的\DefineConcept{商式}”,
\(r(x)\)称为“\(g(x)\)除\(f(x)\)的\DefineConcept{余式}”.

\begin{corollary}
%@see: 《高等代数(第三版 下册)》(丘维声) P12 推论4
设\(f,g \in K[x]\),且\(g(x) \neq 0\),
则\(g(x) \mid f(x)\)当且仅当\(g(x)\)除\(f(x)\)的余式为零.
\end{corollary}

利用带余除法可以证明:
对于\(K[x]\)中的多项式\(f(x),g(x)\),
如果在\(K[x]\)中,\(g(x)\)不能整除\(f(x)\),
那么把数域\(K\)扩大成数域\(F\)后,
在\(F[x]\)中,\(g(x)\)仍然不能整除\(f(x)\).

\begin{proposition}\label{theorem:多项式.整除性不随数域的扩大而改变}
%@see: 《高等代数(第三版 下册)》(丘维声) P12 命题5
设\(F,K\)都是数域,且\(F \supseteq K\).
如果\(f,g \in K[x]\),那么\[
	\text{在\(K[x]\)中成立\(g(x) \mid f(x)\)}
	\iff
	\text{在\(F[x]\)中成立\(g(x) \mid f(x)\)}.
\]
\end{proposition}

\cref{theorem:多项式.整除性不随数域的扩大而改变} 表明,整除性不随数域的扩大而改变.

\begin{example}
设\(f(x) = 2x^3+5x^2+5\),
\(g(x) = x^2+2x-1\),
求用\(g(x)\)除\(f(x)\)的商式与余式.
\begin{solution}
我们可以参考整数除法的竖式,作出如下计算:
\[
	\begin{array}{r|*4r|l}
		x^2+2x-1 &
		2x^3 & +3x^2 & & +5
		& 2x-1 \\
		& 2x^3 & +4x^2 & -2x & \\ \cline{2-5}
		& & -x^2 & +2x & +5 \\
		& & -x^2 & -2x & +1 \\ \cline{3-5}
		& & & 4x & +4
	\end{array}
\]
因此\[
	2x^3+3x^2+5=(2x-1)(x^2+2x-1)+(4x+4),
\]
即\(g(x)\)除\(f(x)\)的商式是\(2x-1\),余式是\(4x+4\).
\end{solution}
\end{example}
