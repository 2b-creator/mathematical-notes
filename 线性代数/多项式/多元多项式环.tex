\section{多元多项式环}
\subsection{多元多项式}
\begin{definition}
%@see: 《高等代数(第三版 下册)》(丘维声) P50 定义1
设\(K\)是一个数域,
用不属于\(K\)的\(n\)个符号\(\AutoTuple{x}{n}\)作表达式\[
	\sum_{\AutoTuple{i}{n}}
	a_{i_1 \dotsm i_n}
	x_1^{i_1} \dotsm x_n^{i_n},
\]
其中\(a_{i_1 \dotsm i_n} \in K\),
\(\AutoTuple{i}{n}\)是非负整数,
上式中的每一项称为一个\DefineConcept{单项式},
上式称为\DefineConcept{数域\(K\)上的\(n\)元多项式}.
如果它具有下述性质:
只有有限多个单项式的系数不为零,
并且两个这种形式的表达式相等当且仅当它们除去系数为零的单项式外含有完全相同的单项式,
而系数为零的单项式允许任意删去或添入.
这时,符号\(\AutoTuple{x}{n}\)称为\(n\)个\DefineConcept{无关不定元}.

在数域\(K\)上的\(n\)元多项式中,
如果两个单项式的幂指数都对应相等,
则称这两个单项式为\DefineConcept{同类项}.
我们约定\(n\)元多项式中的单项式都是不同类的,
即要把同类项合并成一项.

如果数域\(K\)上一个\(n\)元多项式的所有系数全为零,
则称它为\DefineConcept{零多项式},记为\(0\).

我们把\(i_1+\dotsb+i_n\)
称为“单项式\(a_{i_1 \dotsm i_n}
x_1^{i_1} \dotsm x_n^{i_n}\)的\DefineConcept{次数}”.

一个\(n\)元多项式\(f(x_1,\dotsc,x_n)\)的系数不为零的单项式的次数的最大值,
称为“\(f(x_1,\dotsc,x_n)\)的\DefineConcept{次数}”.

零多项式的全次数规定为\(-\infty\).
\end{definition}

\begin{example}
\(5x_1^4+3x_1^3x_2+2x_1x_2x_3^2+x_2^3+x_2x_3\)
是3元4次多项式,
其中单项式\(5x_1^4,3x_1^3x_2,2x_1x_2x_3^2\)的次数都是4.
\end{example}

数域\(K\)上所有\(n\)元多项式组成的集合,
记作\(K[x_1,\dotsc,x_n]\).

在\(K[x_1,\dotsc,x_n]\)中定义加法与乘法如下:
\begin{gather}
	\begin{split}
		&\hspace{-20pt}
		\sum_{\AutoTuple{i}{n}}
		a_{i_1 \dotsm i_n}
		x_1^{i_1} \dotsm x_n^{i_n}
		+
		\sum_{\AutoTuple{i}{n}}
		b_{i_1 \dotsm i_n}
		x_1^{i_1} \dotsm x_n^{i_n} \\
		&\defeq
		\sum_{\AutoTuple{i}{n}}
		(a_{i_1 \dotsm i_n} + b_{i_1 \dotsm i_n})
		x_1^{i_1} \dotsm x_n^{i_n},
	\end{split} \\
	\begin{split}
		&\hspace{-20pt}
		\left(
		\sum_{\AutoTuple{i}{n}}
		a_{i_1 \dotsm i_n}
		x_1^{i_1} \dotsm x_n^{i_n}
		\right) \left(
		\sum_{\AutoTuple{j}{n}}
		b_{j_1 \dotsm j_n}
		x_1^{j_1} \dotsm x_n^{j_n}
		\right) \\
		&\defeq
		\sum_{\AutoTuple{s}{n}}
		c_{s_1 \dotsm s_n}
		x_1^{s_1} \dotsm x_n^{s_n},
	\end{split}
\end{gather}
其中\begin{equation}
	c_{s_1 \dotsm s_n}
	= \sum_{i_1+j_1=s_1}
	\sum_{i_2+j_2=s_2}
	\dotso
	\sum_{i_n+j_n=s_n}
	a_{i_1 \dotsm i_n}
	b_{j_1 \dotsm j_n}.
\end{equation}
不难验证\(K[x_1,\dotsc,x_n]\)对于如上定义的加法与乘法成为一个环.
它的零元是零多项式.
它有单位元,即零次多项式\(1\).
它是交换环.
我们把这个环称为\DefineConcept{数域\(K\)上的\(n\)元多项式环}.

显然有\begin{equation}
	\deg(f+g)
	\leq
	\max\{\deg f,\deg g\}.
\end{equation}

先来对\(n\)元多项式\(f(x_1,\dotsc,x_n)\)的各项规定一个排列顺序,
从而给出首项的概念.

每一类单项式\(a_{i_1 \dotsm i_n} x_1^{i_1} \dotsm x_n^{i_n}\)
对应一个\(n\)元有序非负整数组\((i_1,\dotsc,i_n)\),
这个对应是双射.
为了给出各类单项式之间的一个排列顺序的方法,
就只需要对\(n\)元有序非负整数组定义一个先后顺序.

对于两个\(n\)元有序非负整数组
\((i_1,\dotsc,i_n)\)
和\((j_1,\dotsc,j_n)\),
如果\[
	i_1=j_1,
	i_2=j_2,
	\dotsc,
	i_{s-1}=j_{s-1},
	i_s>j_s
	\quad(1\leq s\leq n)
\]
则称“\((i_1,\dotsc,i_n)\)~\DefineConcept{先于}~\((j_1,\dotsc,j_n)\)”,
记作\((i_1,\dotsc,i_n)>(j_1,\dotsc,j_n)\).

由上述定义立即看出,
对于任意两个\(n\)元有序非负整数组
\((i_1,\dotsc,i_n)\)
和\((j_1,\dotsc,j_n)\),
关系\begin{gather*}
	(i_1,\dotsc,i_n)>(j_1,\dotsc,j_n), \\
	(i_1,\dotsc,i_n)=(j_1,\dotsc,j_n), \\
	(j_1,\dotsc,j_n)>(i_1,\dotsc,i_n)
\end{gather*}
中有且仅有一个成立.

这里关系“\(>\)”具有传递性,
即,
如果\((i_1,\dotsc,i_n)>(j_1,\dotsc,j_n)\)
且\((j_1,\dotsc,j_n)>(k_1,\dotsc,k_n)\),
那么\((i_1,\dotsc,i_n)>(k_1,\dotsc,k_n)\).
这是因为\(i_l-k_l=(i_l-j_l)+(j_l-k_l)\).

\begin{example}
由\((4,2,3,3)>(4,2,2,4)\)和\((4,2,2,4)>(4,1,4,3)\)
可得\((4,2,3,3)>(4,1,4,3)\).
\end{example}

这样我们的确给出了\(n\)元有序非负整数组之间的一个顺序.
相应地,\(n\)元各类单项式之间也有了一个先后顺序.
这种排列顺序的方法是模仿字典中单词的排列原则给出的,
因而称之为\DefineConcept{字典排列法}.

\begin{example}
多项式\(2x_1^4x_2x_3+x_1x_2^5x_3+6x_1^3\)
按字典排列法写出来就是
\(2x_1^4x_2x_3+6x_1^3+x_1x_2^5x_3\).
\end{example}

按字典排列法写出来的第一个系数不为零的单项式称为\(n\)元多项式的\DefineConcept{首项}.

\begin{example}
多项式\(2x_1^4x_2x_3+x_1x_2^5x_3+6x_1^3\)的首项
是\(2x_1^4x_2x_3\).
要注意,首项不一定具有最大的次数.
多项式\(2x_1^4x_2x_3+x_1x_2^5x_3+6x_1^3\)的次数是7,
而它的首项的次数是6.
\end{example}

\begin{theorem}\label{theorem:多项式.多元多项式环.两个非零多项式的乘积的首项等于它们的首项的乘积}
%@see: 《高等代数(第三版 下册)》(丘维声) P52 定理1
在\(K[x_1,\dotsc,x_n]\)中两个非零多项式的乘积的首项等于它们的首项的乘积.
\begin{proof}
设\(f(x_1,\dotsc,x_n),g(x_1,\dotsc,x_n)\)是\(K[x_1,\dotsc,x_n]\)中两个非零多项式.
设\(f(x_1,\dotsc,x_n)\)的首项是\(a x_1^{p_1} x_2^{p_2} \dotsm x_n^{p_n}\ (a\neq0)\),
\(g(x_1,\dotsc,x_n)\)的首项是\(b x_1^{q_1} x_2^{q_2} \dotsm x_n^{q_n}\ (b\neq0)\).
为了证明\(fg\)的首项是
\(ab x_1^{p_1+q_1} x_2^{p_2+q_2} \dotsm x_n^{p_n+q_n}\),
只要证明\[
	(p_1+q_1,p_2+q_2,\dotsc,p_n+q_n)
\]先于\(fg\)中其他单项式的幂指数组就行了.
\(fg\)的其他单项式的幂指数组只有三种可能情形:\[
	(p_1+j_1,p_2+j_2,\dotsc,p_n+j_n)
\]或\[
	(i_1+q_1,i_2+q_2,\dotsc,i_n+q_n)
\]或\[
	(i_1+j_1,i_2+j_2,\dotsc,i_n+j_n),
\]
其中\[
	(p_1,p_2,\dotsc,p_n)
	>
	(i_1,i_2,\dotsc,i_n), \qquad
	(q_1,q_2,\dotsc,q_n)
	>
	(j_1,j_2,\dotsc,j_n).
\]
显然有\begin{gather*}
	(p_1+q_1,p_2+q_2,\dotsc,p_n+q_n)
	>
	(i_1+q_1,i_2+q_2,\dotsc,i_n+q_n), \\
	(i_1+q_1,i_2+q_2,\dotsc,i_n+q_n)
	>
	(i_1+j_1,i_2+j_2,\dotsc,i_n+j_n),
\end{gather*}
于是由传递性得\[
	(p_1+q_1,p_2+q_2,\dotsc,p_n+q_n)
	>
	(i_1+j_1,i_2+j_2,\dotsc,i_n+j_n).
\]
这就证明了
\(ab x_1^{p_1+q_1} x_2^{p_2+q_2} \dotsm x_n^{p_n+q_n}\)
不可能与\(fg\)中其他的单项式相消,
而且它先于\(fg\)中其他的单项式,
它就是\(fg\)的首项.
\end{proof}
\end{theorem}

从\cref{theorem:多项式.多元多项式环.两个非零多项式的乘积的首项等于它们的首项的乘积}
可以推得以下三个命题.

\begin{proposition}
%@see: 《高等代数(第三版 下册)》(丘维声) P52 定理1
在\(K[x_1,\dotsc,x_n]\)中两个非零多项式的乘积仍是非零多项式.
\end{proposition}

\begin{proposition}
%@see: 《高等代数(第三版 下册)》(丘维声) P52 定理1
\(K[x_1,\dotsc,x_n]\)是无零因子环.
\end{proposition}

\begin{proposition}
%@see: 《高等代数(第三版 下册)》(丘维声) P52 定理1
在\(K[x_1,\dotsc,x_n]\)中,消去律成立.
\end{proposition}

\begin{corollary}
%@see: 《高等代数(第三版 下册)》(丘维声) P52 推论2
在\(K[x_1,\dotsc,x_n]\)中,
如果\(f_i\neq0\ (i=1,2,\dotsc,m)\),
则\(f_1 f_2 \dotsm f_m\)的首项等于它们的首项的乘积.
\begin{proof}
对\cref{theorem:多项式.多元多项式环.两个非零多项式的乘积的首项等于它们的首项的乘积}
运用数学归纳法可以证得.
\end{proof}
\end{corollary}

\subsection{齐次多项式}
\begin{definition}
%@see: 《高等代数(第三版 下册)》(丘维声) P53 定义2
设\(g(x_1,\dotsc,x_n)\)是数域\(K\)上的\(n\)元多项式.
如果\(g(x_1,\dotsc,x_n)\)的每个系数不为零的单项式都是\(m\)次的,
则称其为~\DefineConcept{\(m\)次齐次多项式}.
\end{definition}

\begin{example}
\(2x_1^4+3x_1^2x_2x_3+x_1x_2x_3^2\)是一个4次齐次多项式.
\end{example}

\begin{proposition}
\(K[x_1,\dotsc,x_n]\)中任意两个齐次多项式的乘积仍是齐次多项式,
它的次数等于这两个多项式的次数的和.
\end{proposition}

对于任何一个\(n\)元多项式\(f(x_1,\dotsc,x_n)\),
如果把\(f\)中所有次数相同的单项式并在一起,
则\(f\)可以唯一地表示成\[
	f(x_1,\dotsc,x_n)
	=\sum_{i=0}^m
	f_i(x_1,\dotsc,x_n),
\]
其中\(m=\deg f\),
\(f_i(x_1,\dotsc,x_n)\)是\(i\)次齐次多项式,
它称为“\(f(x_1,\dotsc,x_n)\)的~\DefineConcept{\(i\)次齐次成分}”.

\begin{theorem}
%@see: 《高等代数(第三版 下册)》(丘维声) P53 定理3
设\(f(x_1,\dotsc,x_n),g(x_1,\dotsc,x_n) \in K[x_1,\dotsc,x_n]\),
则\begin{equation}
	\deg(fg)=\deg f+\deg g.
\end{equation}
\begin{proof}
若\(f,g\)中有一个是零多项式,
则\(\deg(fg)=\deg f+\deg g\)成立.

现在设\(f\neq0,g\neq0,\deg f=m,\deg g=s\),
则\[
	f=f_0+f_1+\dotsb+f_m, \qquad
	g=g_0+g_1+\dotsb+g_s,
\]
其中\(f_i\)是\(f\)的\(i\)次齐次成分,
\(g_j\)是\(g\)的\(j\)次齐次成分,
\(f_m\neq0\),
\(g_s\neq0\).
我们有\[
	fg
	=(f_0 g_0+\dotsb+f_0 g_s)
	+(f_1 g_0+\dotsb+f_1 g_s)
	+\dotsb
	+(f_m g_0+\dotsb+f_m g_s),
\]
其中\(f_i g_j\)是\(i+j\)次齐次多项式.
因为\(f_m\neq0,g_s\neq0\),
所以\(f_m g_s\neq0\).
于是\(f_m g_s\)是\(m+s\)次齐次多项式.
从而有\(\deg(fg)=m+s=\deg f+\deg g\).
\end{proof}
\end{theorem}
