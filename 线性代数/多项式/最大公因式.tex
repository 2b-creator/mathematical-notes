\section{最大公因式}
从上一节知道,数域\(K\)上的一元多项式环\(K[x]\)具有带余除法,这是\(K[x]\)的一个重要性质.
这一节我们要由此出发推导出\(K[x]\)的另一个重要性质:
\(K[x]\)中任何两个多项式都有最大公因式,
并且\(f(x)\)与\(g(x)\)的最大公因式可以表成\(f(x)\)与\(g(x)\)的倍式和.

在\(K[x]\)中,如果\(c(x)\)既是\(f(x)\)的因式,又是\(g(x)\)的因式,
则称“\(c(x)\)是\(f(x)\)与\(g(x)\)的一个\DefineConcept{公因式}”.

\begin{definition}
%@see: 《高等代数(第三版 下册)》(丘维声) P15 定义1
设\(d(x)\)是\(K[x]\)中多项式\(f(x)\)与\(g(x)\)的一个公因式.
如果对于\(f(x)\)与\(g(x)\)的任一公因式\(c(x)\),都有\(c(x) \mid d(x)\),
则称“\(d(x)\)是\(f(x)\)与\(g(x)\)的一个\DefineConcept{最大公因式}”.
\end{definition}

对于任意多项式\(f(x)\),由于\(f(x) \mid f(x)\)且\(f(x) \mid 0\),
所以\(f(x)\)是\(f(x)\)与\(0\)的一个公因式.
又由于\(f(x)\)与\(0\)的任一公因式\(c(x)\)总可整除\(f(x)\),
因此\(f(x)\)是\(f(x)\)与\(0\)的一个最大公因式.
特别地,\(0\)是\(0\)与\(0\)的最大公因式.

现在我们想要知道,对于\(K[x]\)中任意两个多项式,是否存在它们的最大公因式?
如果存在,我们又该如何找出它们的最大公因式?
对于给定的两个多项式\(f(x)\)与\(g(x)\),它们的最大公因式是否唯一?
这些就是本节要讨论的问题.

我们先指出几个简单而有用的结论.
\begin{proposition}\label{theorem:多项式.最大公因式.命题1}
%@see: 《高等代数(第三版 下册)》(丘维声) P15 命题1
设\(f,g,p,q \in K[x]\).
如果\[
	\Set{ h(x) \given \text{\(h(x)\)是\(f(x)\)与\(g(x)\)的公因式} }
	= \Set{ r(x) \given \text{\(r(x)\)是\(p(x)\)与\(q(x)\)的公因式} },
\]
那么\[
	\Set{ h(x) \given \text{\(h(x)\)是\(f(x)\)与\(g(x)\)的最大公因式} }
	= \Set{ r(x) \given \text{\(r(x)\)是\(p(x)\)与\(q(x)\)的最大公因式} }.
\]
\begin{proof}
设\(d(x)\)是\(f(x)\)与\(g(x)\)的一个最大公因式,
则\(d(x)\)是\(p(x)\)与\(q(x)\)的一个公因式.
任取\(p(x)\)与\(q(x)\)的一个公因式\(\phi(x)\),
则\(\phi(x)\)也是\(f(x)\)与\(g(x)\)的一个公因式,
从而\(\phi(x) \mid d(x)\).
所以\(d(x)\)是\(p(x)\)与\(q(x)\)的一个最大公因式.
同理,\(p(x)\)与\(q(x)\)的任一最大公因式也是\(f(x)\)与\(g(x)\)的最大公因式.
\end{proof}
\end{proposition}

\begin{corollary}\label{theorem:多项式.最大公因式.推论2}
%@see: 《高等代数(第三版 下册)》(丘维声) P15 推论2
设\(f,g \in K[x]\),\(a,b \in K-\{0\}\),
则\[
	\Set{ h(x) \given \text{\(h(x)\)是\(f(x)\)与\(g(x)\)的最大公因式} }
	= \Set{ h(x) \given \text{\(h(x)\)是\(a f(x)\)与\(b g(x)\)的最大公因式} }.
\]
\begin{proof}
显然\(f(x)\)与\(g(x)\)的任一公因式是\(a f(x)\)与\(b g(x)\)的公因式.
对于\(a f(x)\)与\(b g(x)\)的任一公因式\(c(x)\),
有\(c(x) \mid a f(x)\).
又由于\(a\neq0\),
因此\(a f(x) \mid f(x)\),
从而\(c(x) \mid f(x)\).
同理\(c(x) \mid g(x)\).
因此\(c(x)\)也是\(f(x)\)与\(g(x)\)的公因式.
于是由\cref{theorem:多项式.最大公因式.命题1} 立即得出结论.
\end{proof}
\end{corollary}

\begin{lemma}\label{theorem:多项式.最大公因式.引理1}
%@see: 《高等代数(第三版 下册)》(丘维声) P15 引理1
在\(K[x]\)中,如果有等式\[
	f(x) = h(x) g(x) + r(x)
\]成立,
则\[
	\Set{ h(x) \given \text{\(h(x)\)是\(f(x)\)与\(g(x)\)的最大公因式} }
	= \Set{ h(x) \given \text{\(h(x)\)是\(g(x)\)与\(r(x)\)的最大公因式} }.
\]
\begin{proof}
设\(d(x)\)是\(f(x)\)与\(g(x)\)的一个公因式,
则\(d(x) \mid f(x)\)且\(d(x) \mid g(x)\).
因为由\(f(x) = h(x) g(x) + r(x)\)得\[
	r(x) = f(x) - h(x) g(x),
\]
所以\(d(x) \mid r(x)\),
也就是说\(d(x)\)是\(g(x)\)与\(r(x)\)的一个公因式.
现在任取\(g(x)\)与\(r(x)\)的一个公因式\(c(x)\),
由\(f(x) = h(x) g(x) + r(x)\)得\(c(x) \mid f(x)\),
也就是说\(c(x)\)是\(f(x)\)与\(g(x)\)的一个公因式.
由\cref{theorem:多项式.最大公因式.命题1} 立即得出所要求的结论.
\end{proof}
\end{lemma}

\begin{theorem}\label{theorem:多项式.辗转相除法}
%@see: 《高等代数(第三版 下册)》(丘维声) P16 定理3
对于\(K[x]\)中任意两个多项式\(f(x)\)与\(g(x)\),
存在它们的一个最大公因式\(d(x)\),
并且\(d(x)\)可以表示成\(f(x)\)与\(g(x)\)的倍式和,即存在\(u,v \in K[x]\),使得\[
	d(x) = u(x) f(x) + v(x) g(x).
\]
\begin{proof}
假设\(g(x)=0\),
则\(f(x)\)就是\(f(x)\)与\(g(x)\)的一个最大公因式,
并且\[
	f(x) = 1 \cdot f(x) + 1 \cdot 0.
\]

现在设\(g(x)\neq0\).
根据带余除法,
存在\(h_1(x),r_1(x) \in K[x]\),
使得\[
	f(x) = h_1(x) g(x) + r_1(x), \qquad
	\deg r_1(x) < \deg g(x).
\]
如果\(r_1(x)\neq0\),
则用\(r_1(x)\)去除\(g(x)\),
存在\(h_2(x),r_2(x) \in K[x]\),
使得\[
	g(x) = h_2(x) r_1(x) + r_2(x), \qquad
	\deg r_2(x) < \deg r_1(x).
\]
又如果\(r_2\neq0\),
则用\(r_2(x)\)去除\(r_1(x)\),
存在\(h_3(x),r_3(x) \in K[x]\),
使得\[
	r_1(x) = h_3(x) r_2(x) + r_3(x), \qquad
	\deg r_3(x) < \deg r_2(x).
\]
如此辗转相除下去,
显然,所得余式的次数不断降低,
因此在有限次之后,必然有余式为零,
即\begin{gather*}
	r_2(x) = h_4(x) r_3(x) + r_4(x), \qquad \deg r_4(x) < \deg r_3(x), \\
	\hdotsfor1
	r_{i-2}(x) = h_i(x) r_{i-1}(x) + r_i(x), \qquad \deg r_i(x) < \deg r_{i-1}(x), \\
	\hdotsfor1
	r_{s-3}(x) = h_{s-1}(x) r_{s-2}(x) + r_{s-1}(x), \qquad \deg r_{s-1}(x) < \deg r_{s-2}(x), \\
	r_{s-2}(x) = h_s(x) r_{s-1}(x) + r_s(x), \qquad \deg r_s(x) < \deg r_{s-1}(x), \\
	r_{s-1}(x) = h_{s+1}(x) r_s(x) + 0,
\end{gather*}
其中\(h_i(x),r_i(x) \in K[x]\).
由于\(r_s(x)\)是\(r_s(x)\)与\(0\)的一个最大公因式,
因此根据\cref{theorem:多项式.最大公因式.引理1},
从上述等式的最后一个式子得出:
\(r_s(x)\)是\(r_{s-1}(x)\)与\(r_s(x)\)的一个最大公因式.
于是\(r_s(x)\)是\(r_{s-2}(x)\)与\(r_{s-1}(x)\)的一个最大公因式,
从而\(r_s(x)\)是\(r_{s-3}(x)\)与\(r_{s-2}(x)\)的一个最大公因式,
依次递推,
\(r_s(x)\)是\(f(x)\)与\(g(x)\)的一个最大公因式.
这就证明了:
在对\(f(x)\)与\(g(x)\)作辗转相除时,
最后一个不等于零的余式是\(f(x)\)与\(g(x)\)的一个最大公因式.
对上述等式中倒数第二个式子得\[
	r_s(x) = r_{s-2}(x) - h_s(x) r_{s-1}(x),
\]
再由倒数第三个式子得\[
	r_{s-1}(x) = r_{s-3}(x) - h_{s-1}(x) r_{s-2}(x),
\]
合并以上两式得\[
	r_s(x) = [1 + h_s(x) h_{s-1}(x)] r_{s-2}(x) - h_s(x) r_{s-3}(x).
\]
同理用更上面的等式逐个地消去\(r_{s-2}(x),r_{s-3}(x),\dotsc,r_1(x)\),
可得\[
	r_s(x) = u(x) f(x) + v(x) g(x),
\]
其中\(u(x),v(x) \in K[x]\).
\end{proof}
\end{theorem}

\cref{theorem:多项式.辗转相除法} 给出了求两个多项式的最大公因式的方法 --- “辗转相除法”.

我们想要知道,
任意给定\(K[x]\)中的两个多项式\(f(x)\)与\(g(x)\),
它们的最大公因式是否唯一?
设\(d_1(x),d_2(x)\)都是\(f(x)\)与\(g(x)\)的最大公因式,
根据定义得\(d_1(x) \mid d_2(x)\)且\(d_2(x) \mid d_1(x)\).
因此\(d_1(x)\)与\(d_2(x)\)相伴,即\(d_1(x)\)与\(d_2(x)\)仅相差一个非零数因子.
这说明:两个多项式的最大公因式在相伴的意义下是唯一确定的.
容易看出,两个不全为零的多项式的最大公因式一定是非零多项式,
在这个情形,我们约定,用\[
	(f(x), g(x))
\]表示首项系数是\(1\)的那个最大公因式.

应该注意到,
在\cref{theorem:多项式.辗转相除法} 的证明过程中,
我们证明了\(r_s(x)\)是\(f(x)\)与\(g(x)\)的一个最大公因式,
并且有\(r_s(x) = u(x) f(x) + v(x) g(x)\).
对于\(f(x)\)与\(g(x)\)的任一最大公因式\(d(x)\),
由于\(d(x)\)与\(r_s(x)\)相伴,
因此\(d(x) = c r_s(x)\),
其中\(c\)是\(K\)中某个非零数.
于是有\(d(x) = c u(x) f(x) + c v(x) g(x)\).
这表明\(d(x)\)也可以表示成\(f(x)\)与\(g(x)\)的倍式和.

由\cref{theorem:多项式.最大公因式.推论2} 得出,
当\(f(x),g(x)\)不全为零时,
对于\(a,b \in K-\{0\}\),
有\[
	(f(x),g(x))
	= (a f(x),b g(x)).
\]

\begin{example}
设\(f(x)=x^3+x^2-7x+2,
g(x)=3x^2-5x-2\),
求\((f(x),g(x))\),
并且把它表示成\(f(x)\)与\(g(x)\)的倍式和.
\begin{solution}
根据上面的结论,在作辗转相除时,
可以用适当的非零数去乘被除式或者除式,简化计算.
\[
	\begin{array}{r|*3r|*4r|l}
		3x+1 & 3x^2 & -5x & -2 & 3x^3 & +3x^2 & -21x & +6 & x+\frac83
		& 3x^2 & -6x && 3x^3 & -5x^2 & -2x \\ \cline{2-8}
		&& x & -2 && 8x^2 & -19x & +6 \\
		&& x & -2 && 8x^2 & -\frac{40}3x & -\frac{16}3 \\ \cline{2-8}
		&&& 0 &&& -\frac{17}3x & +\frac{34}3 \\
		&&& &&& x & -2 \\
	\end{array}
\]
因为最后一个不等于零的余式是\(r_1(x) = -\frac{17}3x + \frac{34}3\),
所以\[
	(f(x),g(x)) = x-2.
\]
把上述辗转相除过程写出来就是\begin{align*}
	3 f(x) = \left(x+\frac83\right) g(x) + r_1(x), \\
	g(x) = (3x+1) \left[-\frac3{17} r_1(x)] + 0.
\end{align*}
于是\begin{align*}
	(f(x),g(x))
	&= -\frac3{17} r_1(x) \\
	&= -\frac3{17} \left[3 f(x) - \left(x+\frac83\right) g(x)\right] \\
	&= -\frac9{17} f(x) + \frac1{17} (3x+8) g(x).
\end{align*}
\end{solution}
\end{example}

现在我们来研究两个多项式的最大公因式是零次多项式的情形.

\begin{definition}\label{definition:多项式.互素}
%@see: 《高等代数(第三版 下册)》(丘维声) P18 定义2
设\(f(x),g(x)\)是\(K[x]\)中的两个多项式.
如果\((f(x),g(x))=1\),
则称“\(f(x)\)与\(g(x)\) \DefineConcept{互素}”.
\end{definition}

从\cref{definition:多项式.互素} 立即得出,
两个多项式互素当且仅当它们的公因式都是零次多项式,
这时因为它们的任一公因式\(c(x) \mid 1\),
所以\(\deg c(x) = 0\).

下面我们给出两个多项式互素的一个充分必要条件.
\begin{theorem}\label{theorem:多项式.两个多项式互素的充分必要条件}
%@see: 《高等代数(第三版 下册)》(丘维声) P18 定理4
设\(f,g \in K[x]\).
\(f(x)\)与\(g(x)\)互素的充分必要条件是:
存在\(u,v \in K[x]\),使得\[
	u(x) f(x) + v(x) g(x) = 1.
\]
\begin{proof}
必要性.
由\cref{theorem:多项式.辗转相除法} 立即可得.

充分性.
假设\(u(x) f(x) + v(x) g(x) = 1\)成立.
因为\((f(x),g(x)) \mid f(x)\)且\((f(x),g(x)) \mid g(x)\),
所以\((f(x),g(x)) \mid 1\),
于是\((f(x),g(x)) = 1\).
\end{proof}
\end{theorem}

利用\cref{theorem:多项式.两个多项式互素的充分必要条件} 可以证明关于互素的多项式的一些重要性质.

\begin{property}
%@see: 《高等代数(第三版 下册)》(丘维声) P19 性质1
在\(K[x]\)中,如果\[
	f(x) \mid g(x) h(x)
	\quad\text{且}\quad
	(f(x),g(x))=1,
\]
则\[
	f(x) \mid h(x).
\]
\begin{proof}
当\(h(x)=0\)时,
有\(f(x) \mid h(x)\).

当\(h(x)\neq0\)时,
因为\((f(x),g(x))=1\),
所以,存在\(u(x),v(x) \in K[x]\),
使得\[
	u(x) f(x) + v(x) g(x) = 1.
\]
等式两边同乘\(h(x)\),
得\[
	u(x) f(x) h(x) + v(x) g(x) h(x) = h(x).
\]
因为\(f(x) \mid g(x) h(x)\),
所以用\(f(x)\)整除上式左端,
就有\(f(x) \mid h(x)\).
\end{proof}
\end{property}

\begin{property}
%@see: 《高等代数(第三版 下册)》(丘维声) P19 性质2
在\(K[x]\)中,如果\[
	f(x) \mid h(x)
	\quad\text{且}\quad
	g(x) \mid h(x)
	\quad\text{且}\quad
	(f(x),g(x))=1,
\]
则\[
	f(x) g(x) \mid h(x).
\]
\begin{proof}
因为\(f(x) \mid h(x)\),
所以存在\(p(x) \in K[x]\),
使得\(h(x) = p(x) f(x)\).
因为\(g(x) \mid h(x)\),
所以\(g(x) \mid p(x) f(x)).
因为\((g(x),f(x))=1\),
所以\(g(x) \mid p(x)\).
因此存在\(q(x) \in K[x]\),
使得\(p(x) = q(x) g(x)).
于是\(h(x) = q(x) g(x) f(x)\),
那么\(f(x) g(x) \mid h(x)\).
\end{proof}
\end{property}

\begin{property}
%@see: 《高等代数(第三版 下册)》(丘维声) P19 性质3
在\(K[x]\)中,如果\[
	(f(x),h(x))=1
	\quad\text{且}\quad
	(g(x),h(x))=1,
\]
则\[
	(f(x) g(x),h(x))=1.
\]
\begin{proof}
因为\((f(x),g(x))=1\),
\((g(x),h(x))=1\),
所以存在\(u_1(x),u_2(x),v_1(x),v_2(x) \in K[x]\),
使得\begin{gather*}
	u_1(x) f(x) + v_1(x) h(x) = 1, \\
	u_2(x) g(x) + v_2(x) h(x) = 1.
\end{gather*}
将上面两个等式相乘,
得\[
	u_1(x) u_2(x) f(x) g(x)
	+ [
		u_1(x) f(x) v_2(x)
		+ v_1(x) u_2(x) g(x)
		+ v_1(x) v_2(x) h(x)
	] h(x)
	= 1.
\]
根据\cref{theorem:多项式.两个多项式互素的充分必要条件}
得\((f(x) g(x),h(x))=1\).
\end{proof}
\end{property}

最大公因式和互素的概念可以推广到\(n>2\)个多项式的情形.
\begin{definition}
%@see: 《高等代数(第三版 下册)》(丘维声) P20 定义3
在\(K[x]\)中,
如果多项式\(c(x)\)能整除多项式\(f_i(x)\ (i=1,2,\dotsc,n)\)的每一个,
那么把\(c(x)\)称为这\(n\)个多项式的一个\DefineConcept{公因式}.
\end{definition}

\begin{definition}
%@see: 《高等代数(第三版 下册)》(丘维声) P20 定义3
在\(K[x]\)中,
设多项式\(d(x)\)是\(f_i(x)\ (i=1,2,\dotsc,n)\)的一个公因式.
如果\(f_i(x)\ (i=1,2,\dotsc,n)\)的每一个公因式都能整除\(d(x)\),
那么把\(d(x)\)称为这\(n\)个多项式的一个\DefineConcept{最大公因式}.
\end{definition}

用数学归纳法可以证明,
在\(K[x]\)中,
任意\(n\geq2\)个多项式
\(f_1(x),\dotsc,f_n(x)\)的最大公因式存在,
并且如果\(d_1(x)\)是\(f_1(x),\dotsc,f_{n-1}(x)\)的一个最大公因式,
则\(d_1(x)\)与\(f_n(x)\)的最大公因式就是\(f_1(x),\dotsc,f_{n-1}(x),f_n(x)\)的最大公因式.
因此我们依然可以逐次使用辗转相除法求出\(n\)个多项式的一个最大公因式.

从定义可知,
\(n\)个多项式\(f_1(x),\dotsc,f_n(x)\)的最大公因式在相伴的意义下是唯一的.
对于\(n\)个不全为零的多项式\(f_1(x),\dotsc,f_n(x)\),
我们约定使用\[
	(f_1(x),\dotsc,f_n(x))
\]表示首项系数是\(1\)的那个最大公因式.
于是我们断言\[
%@see: 《高等代数(第三版 下册)》(丘维声) P20 公式(6)
	(f_1(x),\dotsc,f_n(x))
	= ((f_1(x),\dotsc,f_{n-1}(x)),f_n(x)).
\]
从上式出发,根据\cref{theorem:多项式.辗转相除法},
存在\(u_1(x),\dotsc,u_n(x) \in K[x]\),
使得\[
%@see: 《高等代数(第三版 下册)》(丘维声) P20 公式(7)
	u_1(x) f_1(x) + \dotsb + u_n(x) f_n(x)
	= (f_1(x),\dotsc,f_n(x)).
\]

\begin{definition}
%@see: 《高等代数(第三版 下册)》(丘维声) P20 定义4
如果\(K[x]\)中\(n\geq2\)个多项式\(f_1(x),\dotsc,f_n(x)\)满足\[
	(f_1(x),\dotsc,f_n(x)) = 1,
\]
那么称“\(f_1(x),\dotsc,f_n(x)\)~\DefineConcept{互素}”.
\end{definition}

与\cref{theorem:多项式.两个多项式互素的充分必要条件} 一样,
我们可以证明:
在\(K[x]\)中,
\(n\)个多项式\(f_1(x),\dotsc,f_n(x)\)
互素的充分必要条件是
存在\(K[x]\)中多项式\(u_1(x),\dotsc,u_n(x)\)
使得\[
%@see: 《高等代数(第三版 下册)》(丘维声) P20 公式(8)
	u_1(x) f_1(x) + \dotsb + u_n(x) f_n(x) = 1.
\]
但要注意点,\(n>2\)个多项式互素时,
它们不一定两两互素.
例如,多项式\[
	f_1(x) = x+1, \qquad
	f_2(x) = x^2+3x+2, \qquad
	f_3(x) = x-1
\]满足\[
	(f_1(x),f_2(x))=x+1, \qquad
	(f_1(x),f_2(x),f_3(x))=1,
\]
也就是说\(f_1(x),f_2(x),f_3(x)\)互素,
但是\(f_1(x),f_2(x)\)不互素.

我们还要指出一点,
设\(K\)与\(F\)都是数域,
并且\(K \subseteq F\).
设\(f(x),g(x) \in K[x]\),
则我们也可以把\(f(x)\)与\(g(x)\)看成是\(F[x]\)中的多项式.
注意\(f(x)\)与\(g(x)\)在\(K[x]\)中的公因式
和它们在\(F[x]\)中的公因式不一定相同.
例如,设\[
	f(x) = x^2+1, \qquad
	g(x) = x^3+x^2+x+1,
\]
则\(f(x)\)与\(g(x)\)在\(\mathbb{R}[x]\)中没有一次公因式,
但是它们在\(\mathbb{C}[x]\)中有一次公因式\(x+\iu\)与\(x-\iu\).
容易看出它们在\(\mathbb{R}[x]\)中的最大公因式是\(x^2+1\),
在\(\mathbb{C}[x]\)中的最大公因式也是\(x^2+1\).
一般地,我们有如下结论.

\begin{proposition}
%@see: 《高等代数(第三版 下册)》(丘维声) P20 命题5
设\(F,K\)都是数域,且\(F \supseteq K\),
则对于\(K[x]\)中任意两个多项式\(f(x)\)与\(g(x)\),
它们在\(K[x]\)中的首项系数为\(1\)的最大公因式
与它们在\(F[x]\)中的首项系数为\(1\)的最大公因式相同.
也就是说,当数域扩大时,\(f(x)\)与\(g(x)\)的首项系数为\(1\)的最大公因式不改变.
\end{proposition}

\begin{corollary}
%@see: 《高等代数(第三版 下册)》(丘维声) P21 推论6
设\(F,K\)都是数域,且\(F \supseteq K\),
\(f,g \in K[x]\),
则\(f(x)\)与\(g(x)\)在\(K[x]\)中互素的充分必要条件是:
\(f(x)\)在\(g(x)\)在\(F[x]\)中互素.
也就是说,互素性不随数域的扩大而改变.
\end{corollary}
