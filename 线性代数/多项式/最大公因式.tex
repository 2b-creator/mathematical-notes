\section{最大公因式}
从上一节知道,数域\(K\)上的一元多项式环\(K[x]\)具有带余除法,这是\(K[x]\)的一个重要性质.
这一节我们要由此出发推导出\(K[x]\)的另一个重要性质:
\(K[x]\)中任何两个多项式都有最大公因式,
并且\(f(x)\)与\(g(x)\)的最大公因式可以表成\(f(x)\)与\(g(x)\)的倍式和.

在\(K[x]\)中,如果\(c(x)\)既是\(f(x)\)的因式,又是\(g(x)\)的因式,
则称“\(c(x)\)是\(f(x)\)与\(g(x)\)的一个\DefineConcept{公因式}”.

\begin{definition}
%@see: 《高等代数(第三版 下册)》(丘维声) P15. 定义1
设\(d(x)\)是\(K[x]\)中多项式\(f(x)\)与\(g(x)\)的一个公因式.
如果对于\(f(x)\)与\(g(x)\)的任一公因式\(c(x)\),都有\(c(x) \vert d(x)\),
则称“\(d(x)\)是\(f(x)\)与\(g(x)\)的一个\DefineConcept{最大公因式}”.
\end{definition}

对于任意多项式\(f(x)\),由于\(f(x) \vert f(x)\)且\(f(x) \vert 0\),
所以\(f(x)\)是\(f(x)\)与\(0\)的一个公因式.
又由于\(f(x)\)与\(0\)的任一公因式\(c(x)\)总可整除\(f(x)\),
因此\(f(x)\)是\(f(x)\)与\(0\)的一个最大公因式.
特别地,\(0\)是\(0\)与\(0\)的最大公因式.

现在我们想要知道,对于\(K[x]\)中任意两个多项式,是否存在它们的最大公因式?
如果存在,我们又该如何找出它们的最大公因式?
对于给定的两个多项式\(f(x)\)与\(g(x)\),它们的最大公因式是否唯一?
这些就是本节要讨论的问题.

\begin{proposition}
%@see: 《高等代数(第三版 下册)》(丘维声) P15. 命题1
设\(f,g,p,q \in K[x]\).
如果\[
	\Set{ h(x) \given \text{\(h(x)\)是\(f(x)\)与\(g(x)\)的公因式} }
	= \Set{ r(x) \given \text{\(r(x)\)是\(p(x)\)与\(q(x)\)的公因式} },
\]
那么\[
	\Set{ h(x) \given \text{\(h(x)\)是\(f(x)\)与\(g(x)\)的最大公因式} }
	= \Set{ r(x) \given \text{\(r(x)\)是\(p(x)\)与\(q(x)\)的最大公因式} }.
\]
\end{proposition}

\begin{corollary}
%@see: 《高等代数(第三版 下册)》(丘维声) P15. 推论2
设\(f,g \in K[x]\),\(a,b \in K-\{0\}\),
则\[
	\Set{ h(x) \given \text{\(h(x)\)是\(f(x)\)与\(g(x)\)的最大公因式} }
	= \Set{ h(x) \given \text{\(h(x)\)是\(a f(x)\)与\(b g(x)\)的最大公因式} }.
\]
\end{corollary}

\begin{lemma}
%@see: 《高等代数(第三版 下册)》(丘维声) P15. 引理1
在\(K[x]\)中,如果有等式\[
	f(x) = h(x) g(x) + r(x)
\]成立,
则\[
	\Set{ h(x) \given \text{\(h(x)\)是\(f(x)\)与\(g(x)\)的最大公因式} }
	= \Set{ h(x) \given \text{\(h(x)\)是\(g(x)\)与\(r(x)\)的最大公因式} }.
\]
\end{lemma}

\begin{theorem}\label{theorem:多项式.辗转相除法}
%@see: 《高等代数(第三版 下册)》(丘维声) P16. 定理3
对于\(K[x]\)中任意两个多项式\(f(x)\)与\(g(x)\),
存在它们的一个最大公因式\(d(x)\),
并且\(d(x)\)可以表示成\(f(x)\)与\(g(x)\)的倍式和,即存在\(u,v \in K[x]\),使得\[
	d(x) = u(x) f(x) + v(x) g(x).
\]
\end{theorem}

\cref{theorem:多项式.辗转相除法} 给出了求两个多项式的最大公因式的方法 --- “辗转相除法”.

设\(d_1(x),d_2(x)\)都是\(f(x)\)与\(g(x)\)的最大公因式,
根据定义得\(d_1(x) \vert d_2(x)\)且\(d_2(x) \vert d_1(x)\).
因此\(d_1(x)\)与\(d_2(x)\)相伴,即\(d_1(x)\)与\(d_2(x)\)仅相差一个非零数因子.
这说明:两个多项式的最大公因式在相伴的意义下是唯一确定的.
容易看出,两个不全为零的多项式的最大公因式一定是非零多项式,
在这个情形,我们约定,用\[
	(f(x), g(x))
\]表示首相系数是\(1\)的那个最大公因式.

\begin{proposition}
%@see: 《高等代数(第三版 下册)》(丘维声) P17.
设\(f,g \in K[x]\)且不全为零,
则对于\(a,b \in K-\{0\}\),有\[
	(f(x), g(x)) = (a f(x), b g(x)).
\]
\end{proposition}

\begin{definition}\label{definition:多项式.互素}
%@see: 《高等代数(第三版 下册)》(丘维声) P18. 定义2
设\(f(x),g(x)\)是\(K[x]\)中的两个多项式.
如果\((f(x),g(x))=1\),
则称“\(f(x)\)与\(g(x)\) \DefineConcept{互素}”.
\end{definition}

从\cref{definition:多项式.互素} 立即得出,
两个多项式互素当且仅当它们的公因式都是零次多项式,
这时因为它们的任一公因式\(c(x) \vert 1\),
所以\(\deg c(x) = 0\).

下面我们给出两个多项式互素的一个充要条件.
\begin{theorem}\label{theorem:多项式.两个多项式互素的充要条件}
%@see: 《高等代数(第三版 下册)》(丘维声) P18. 定理4
设\(f,g \in K[x]\).
\(f(x)\)与\(g(x)\)互素的充要条件是:
存在\(u,v \in K[x]\),使得\[
	u(x) f(x) + v(x) g(x) = 1.
\]
\end{theorem}

利用\cref{theorem:多项式.两个多项式互素的充要条件} 可以证明关于互素的多项式的一些重要性质.

\begin{property}
%@see: 《高等代数(第三版 下册)》(丘维声) P19. 性质1
在\(K[x]\)中,如果\[
	f(x) \vert g(x) h(x)
	\quad\text{且}\quad
	(f(x),g(x))=1,
\]
则\[
	f(x) \vert h(x).
\]
\end{property}

\begin{property}
%@see: 《高等代数(第三版 下册)》(丘维声) P19. 性质2
在\(K[x]\)中,如果\[
	f(x) \vert h(x)
	\quad\text{且}\quad
	g(x) \vert h(x)
	\quad\text{且}\quad
	(f(x),g(x))=1,
\]
则\[
	f(x) g(x) \vert h(x).
\]
\end{property}

\begin{property}
%@see: 《高等代数(第三版 下册)》(丘维声) P19. 性质3
在\(K[x]\)中,如果\[
	(f(x),h(x))=1
	\quad\text{且}\quad
	(g(x),h(x))=1,
\]
则\[
	(f(x) g(x),h(x))=1.
\]
\end{property}

我们还要指出一点,给定两个多项式\(f(x),g(x)\),
它们在不同数域中的公因式不一定相同.
例如,设\[
	f(x) = x^2+1, \qquad
	g(x) = x^3+x^2+x+1,
\]
则\(f(x)\)与\(g(x)\)在\(\mathbb{R}[x]\)中没有一次公因式,
但是它们在\(\mathbb{C}[x]\)中有一次公因式\(x+\iu\)与\(x-\iu\).
容易看出它们在\(\mathbb{R}[x]\)中的最大公因式是\(x^2+1\),
在\(\mathbb{C}[x]\)中的最大公因式也是\(x^2+1\).
一般地,我们有如下结论.

\begin{proposition}
%@see: 《高等代数(第三版 下册)》(丘维声) P20. 命题5
设\(F,K\)都是数域,且\(F \supseteq K\),
则对于\(K[x]\)中任意两个多项式\(f(x)\)与\(g(x)\),
它们在\(K[x]\)中的首项系数为\(1\)的最大公因式
与它们在\(F[x]\)中的首项系数为\(1\)的最大公因式相同.
也就是说,当数域扩大时,\(f(x)\)与\(g(x)\)的首项系数为\(1\)的最大公因式不改变.
\end{proposition}

\begin{corollary}
%@see: 《高等代数(第三版 下册)》(丘维声) P21. 推论6
设\(F,K\)都是数域,且\(F \supseteq K\),
\(f,g \in K[x]\),
则\(f(x)\)与\(g(x)\)在\(K[x]\)中互素的充要条件是:
\(f(x)\)在\(g(x)\)在\(F[x]\)中互素.
也就是说,互素性不随数域的扩大而改变.
\end{corollary}
