\section{最大公因式}
从上一节知道,数域\(K\)上的一元多项式环\(K[x]\)具有带余除法,这是\(K[x]\)的一个重要性质.
这一节我们要由此出发推导出\(K[x]\)的另一个重要性质:
\(K[x]\)中任何两个多项式都有最大公因式,
并且\(f(x)\)与\(g(x)\)的最大公因式可以表成\(f(x)\)与\(g(x)\)的倍式和.

在\(K[x]\)中,如果\(c(x)\)既是\(f(x)\)的因式,又是\(g(x)\)的因式,
则称“\(c(x)\)是\(f(x)\)与\(g(x)\)的一个\DefineConcept{公因式}”.

\begin{definition}
%@see: 《高等代数(第三版 下册)》(丘维声) P15. 定义1
设\(d(x)\)是\(K[x]\)中多项式\(f(x)\)与\(g(x)\)的一个公因式.
如果对于\(f(x)\)与\(g(x)\)的任一公因式\(c(x)\),都有\(c(x) \vert d(x)\),
则称“\(d(x)\)是\(f(x)\)与\(g(x)\)的一个\DefineConcept{最大公因式}”.
\end{definition}

对于任意多项式\(f(x)\),由于\(f(x) \vert f(x)\)且\(f(x) \vert 0\),
所以\(f(x)\)是\(f(x)\)与\(0\)的一个公因式.
又由于\(f(x)\)与\(0\)的任一公因式\(c(x)\)总可整除\(f(x)\),
因此\(f(x)\)是\(f(x)\)与\(0\)的一个最大公因式.
特别地,\(0\)是\(0\)与\(0\)的最大公因式.

现在我们想要知道,对于\(K[x]\)中任意两个多项式,是否存在它们的最大公因式?
如果存在,我们又该如何找出它们的最大公因式?
对于给定的两个多项式\(f(x)\)与\(g(x)\),它们的最大公因式是否唯一?
这些就是本节要讨论的问题.

\begin{proposition}
%@see: 《高等代数(第三版 下册)》(丘维声) P15. 命题1
设\(f,g,p,q \in K[x]\).
如果\[
	\Set{ h(x) \given \text{\(h(x)\)是\(f(x)\)与\(g(x)\)的公因式} }
	= \Set{ r(x) \given \text{\(r(x)\)是\(p(x)\)与\(q(x)\)的公因式} },
\]
那么\[
	\Set{ h(x) \given \text{\(h(x)\)是\(f(x)\)与\(g(x)\)的最大公因式} }
	= \Set{ r(x) \given \text{\(r(x)\)是\(p(x)\)与\(q(x)\)的最大公因式} }.
\]
\end{proposition}

\begin{corollary}
%@see: 《高等代数(第三版 下册)》(丘维声) P15. 推论2
设\(f,g \in K[x]\),\(a,b \in K-\{0\}\),
则\[
	\Set{ h(x) \given \text{\(h(x)\)是\(f(x)\)与\(g(x)\)的最大公因式} }
	= \Set{ h(x) \given \text{\(h(x)\)是\(a f(x)\)与\(b g(x)\)的最大公因式} }.
\]
\end{corollary}

\begin{lemma}
%@see: 《高等代数(第三版 下册)》(丘维声) P15. 引理1
在\(K[x]\)中,如果有等式\[
	f(x) = h(x) g(x) + r(x)
\]成立,
则\[
	\Set{ h(x) \given \text{\(h(x)\)是\(f(x)\)与\(g(x)\)的最大公因式} }
	= \Set{ h(x) \given \text{\(h(x)\)是\(g(x)\)与\(r(x)\)的最大公因式} }.
\]
\end{lemma}

\begin{theorem}\label{theorem:多项式.辗转相除法}
%@see: 《高等代数(第三版 下册)》(丘维声) P16. 定理3
对于\(K[x]\)中任意两个多项式\(f(x)\)与\(g(x)\),
存在它们的一个最大公因式\(d(x)\),
并且\(d(x)\)可以表示成\(f(x)\)与\(g(x)\)的倍式和,即存在\(u,v \in K[x]\),使得\[
	d(x) = u(x) f(x) + v(x) g(x).
\]
\end{theorem}

\cref{theorem:多项式.辗转相除法} 给出了求两个多项式的最大公因式的方法 --- “辗转相除法”.
