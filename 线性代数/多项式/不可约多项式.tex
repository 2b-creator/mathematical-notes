\section{不可约多项式,唯一因式分解定理}
我们已经知道,数域\(K\)上的一元多项式环\(K[x]\)具有带余除法.
由此推导出,\(K[x]\)中任意两个多项式都有最大公因式.
现在我们利用这些结论来研究\(K[x]\)的结构.
与整数环\(\mathbb{Z}\)类比:
每一个正整数都能表示成有限多个素数的乘积.
我们不禁发问:\(K[x]\)中每一个多项式是否能表示成有限多个具有类似“素数”那样的性质的多项式的乘积?
联系我们对素数的定义,
对于一个大于\(1\)的正整数\(p\),如果它的正因子只有\(1\)和\(p\),那么称其为素数.
我们可以给出如下概念:
\begin{definition}
%@see: 《高等代数(第三版 下册)》(丘维声) P24. 定义1
\(K[x]\)中一个次数大于零的多项式\(p(x)\),
如果它在\(K[x]\)中的因式只有零次多项式和\(p(x)\)的相伴元,
则称“\(p(x)\)是数域\(K\)上的一个\DefineConcept{不可约多项式}”;
否则称“\(p(x)\)是数域\(K\)上的一个\DefineConcept{可约多项式}”.
\end{definition}

\begin{property}%\label{theorem:多项式.不可约多项式.性质1}
%@see: 《高等代数(第三版 下册)》(丘维声) P24. 性质1
\(K[x]\)中不可约多项式\(p(x)\)与任一多项式\(f(x)\)的关系只有两种可能:
\begin{enumerate}
	\item 要么\(p(x) \vert f(x)\).
	\item 要么\(p(x)\)与\(f(x)\)互素.
\end{enumerate}
\end{property}

\begin{property}\label{theorem:多项式.不可约多项式.性质2}
%@see: 《高等代数(第三版 下册)》(丘维声) P25. 性质2
在\(K[x]\)中,如果\(p(x)\)不可约,且\(p(x) \vert f(x) g(x)\),
则\(p(x) \vert f(x)\)或\(p(x) \vert g(x)\).
\end{property}

利用数学归纳法,\cref{theorem:多项式.不可约多项式.性质2} 可以推广为:
在\(K[x]\)中,如果\(p(x)\)不可约,
且\[
	p(x) \vert f_1(x) f_2(x) \dotsm f_s(x),
\]
则对于某个\(j \in \Set{1,2,\dotsc,s}\),有\(p(x) \vert f_j(x)\).

\begin{property}\label{theorem:多项式.不可约多项式.性质3}
%@see: 《高等代数(第三版 下册)》(丘维声) P25. 性质3
\(K[x]\)中,\(p(x)\)不可约,当且仅当\(p(x)\)不能分解成两个次数较\(p(x)\)的次数低的多项式的乘积.
\end{property}

从\cref{theorem:多项式.不可约多项式.性质3} 立即得出,
\(K[x]\)中的每一个\(1\)次多项式一定是不可约多项式.
