\section{不可约多项式,唯一因式分解定理}
我们已经知道,数域\(K\)上的一元多项式环\(K[x]\)具有带余除法.
由此推导出,\(K[x]\)中任意两个多项式都有最大公因式.
现在我们利用这些结论来研究\(K[x]\)的结构.
与整数环\(\mathbb{Z}\)类比:
每一个正整数都能表示成有限多个素数的乘积.
我们不禁发问:\(K[x]\)中每一个多项式是否能表示成有限多个具有类似“素数”那样的性质的多项式的乘积?
联系我们对素数的定义,
对于一个大于\(1\)的正整数\(p\),如果它的正因子只有\(1\)和\(p\),那么称其为素数.
我们可以给出如下概念:
\begin{definition}
%@see: 《高等代数(第三版 下册)》(丘维声) P24. 定义1
\(K[x]\)中一个次数大于零的多项式\(p(x)\),
如果它在\(K[x]\)中的因式只有零次多项式和\(p(x)\)的相伴元,
则称“\(p(x)\)是数域\(K\)上的一个\DefineConcept{不可约多项式}”;
否则称“\(p(x)\)是数域\(K\)上的一个\DefineConcept{可约多项式}”.
\end{definition}
