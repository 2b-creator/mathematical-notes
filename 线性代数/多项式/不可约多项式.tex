\section{不可约多项式,唯一因式分解定理}
我们已经知道,数域\(K\)上的一元多项式环\(K[x]\)具有带余除法.
由此推导出,\(K[x]\)中任意两个多项式都有最大公因式.
现在我们利用这些结论来研究\(K[x]\)的结构.
与整数环\(\mathbb{Z}\)类比:
每一个正整数都能表示成有限多个素数的乘积.
我们不禁发问:\(K[x]\)中每一个多项式是否能表示成有限多个具有类似“素数”那样的性质的多项式的乘积?
联系我们对素数的定义,
对于一个大于\(1\)的正整数\(p\),如果它的正因子只有\(1\)和\(p\),那么称其为素数.
我们可以给出如下概念:
\begin{definition}
%@see: 《高等代数(第三版 下册)》(丘维声) P24 定义1
\(K[x]\)中一个次数大于零的多项式\(p(x)\),
如果它在\(K[x]\)中的因式只有零次多项式和\(p(x)\)的相伴元,
则称“\(p(x)\)是数域\(K\)上的一个\DefineConcept{不可约多项式}(irreducible polynomial)”;
否则称“\(p(x)\)是数域\(K\)上的一个\DefineConcept{可约多项式}(reducible polynomial)”.
%@see: https://mathworld.wolfram.com/IrreduciblePolynomial.html
\end{definition}

\begin{property}\label{theorem:多项式.不可约多项式.性质1}
%@see: 《高等代数(第三版 下册)》(丘维声) P24 性质1
\(K[x]\)中不可约多项式\(p(x)\)与任一多项式\(f(x)\)的关系只有两种可能:
\begin{enumerate}
	\item 要么\(p(x) \mid f(x)\).
	\item 要么\(p(x)\)与\(f(x)\)互素.
\end{enumerate}
\begin{proof}
由于\((p(x),f(x))\)是\(p(x)\)的因式,
而\(p(x)\)不可约,
因此\((p(x),f(x))\)是零次多项式,
即\((p(x),f(x)) \sim p(x)\).
从而\((p(x),f(x))=1\),
即\(p(x) \mid (p(x),f(x))\).
利用整除的传递性得出\(p(x) \mid f(x)\).
\end{proof}
\end{property}

\begin{property}\label{theorem:多项式.不可约多项式.性质2}
%@see: 《高等代数(第三版 下册)》(丘维声) P25 性质2
在\(K[x]\)中,如果\(p(x)\)不可约,且\(p(x) \mid f(x) g(x)\),
则\(p(x) \mid f(x)\)或\(p(x) \mid g(x)\).
\begin{proof}
如果\(p(x) \mid f(x)\),
则结论成立.
下面设\(p(x) \nmid f(x)\).
由于\(p(x)\)不可约,
因此根据\cref{theorem:多项式.不可约多项式.性质1}
得\((p(x),f(x))=1\).
于是从\(p(x) \mid f(x) g(x)\)
得出\(p(x) \mid g(x)\).
\end{proof}
\end{property}

利用数学归纳法,\cref{theorem:多项式.不可约多项式.性质2} 可以推广为:
在\(K[x]\)中,如果\(p(x)\)不可约,
且\[
	p(x) \mid f_1(x) f_2(x) \dotsm f_s(x),
\]
则对于某个\(j \in \Set{1,2,\dotsc,s}\),有\(p(x) \mid f_j(x)\).

\begin{property}\label{theorem:多项式.不可约多项式.性质3}
%@see: 《高等代数(第三版 下册)》(丘维声) P25 性质3
\(K[x]\)中,\(p(x)\)不可约,当且仅当\(p(x)\)不能分解成两个次数较\(p(x)\)的次数低的多项式的乘积.
\begin{proof}
必要性.
如果\(p(x)\)不可约,
则\(p(x)\)的因式只有零次多项式和\(p(x)\)的相伴元,
因此\(p(x)\)不能分解成两个次数较\(p(x)\)的次数低的多项式的乘积.

充分性.
用反证法.
假设\(p(x)\)可约,
则\(p(x)\)有因式\(g(x)\)使得\(0<\deg g(x)<\deg p(x)\).
从而存在\(h(x) \in K[x]\)
使得\[
	p(x) = h(x) g(x),
\]
于是\[
	\deg p(x) = \deg h(x) + \deg g(x).
\]
由此推出\(0<\deg h(x)<\deg p(x)\),
这与已知条件矛盾!
\end{proof}
\end{property}

从\cref{theorem:多项式.不可约多项式.性质3} 立即得出,
\(K[x]\)中的每一个\(1\)次多项式一定是不可约多项式.

\begin{theorem}[唯一因式分解定理]\label{theorem:多项式.唯一因式分解定理}
%@see: 《高等代数(第三版 下册)》(丘维声) P25 定理1
\(K[x]\)中每一个次数大于零的多项式\(f(x)\)都能唯一地分解成数域\(K\)上有限多个不可约多项式的乘积.
\begin{proof}
先证分解式\[
	f(x) = p_1(x) p_2(x) \dotsm p_s(x)
\]的存在性.
利用数学归纳法.
因为一次多项式都是不可约的,
所以\(n=1\)时,存在性成立.
假设对于次数小于\(n\)的多项式,存在性成立.
现在来看\(n\)次多项式\(f(x)\).
如果\(f(x)\)是不可约多项式,
则存在性显然成立.
如果\(f(x)\)是可约多项式,
则有\(f(x)=f_1(x) f_2(x)\),
其中\(f_1(x) \in K[x]\),
并且\(\deg f_1(x) < \deg f(x)\),
\(\deg f_2(x) < \deg f(x)\).
由归纳假设,\(f_1(x)\)与\(f_2(x)\)都可以分解成数域\(K\)上有限多个不可约多项式的乘积,
那么只要把\(f_1(x)\)与\(f_2(x)\)的分解式合起来就可以得到\(f(x)\)的一个分解式.
于是由归纳原理,存在性普遍成立.

现在证唯一性.
假设\(f(x)\)有两个分解式:\[
	f(x) = p_1(x) p_2(x) \dotsm p_s(x),
	\quad\text{和}\quad
	f(x) = q_1(x) q_2(x) \dotsm q_t(x),
\]
其中\(p_i(x),q_j(x)\ (i=1,\dotsc,s;j=1,\dotsc,t)\)都是数域\(K\)上的不可约多项式.
我们对第一个分解式中不可约因式个数\(s\)作归纳法.
当\(s=1\)时,
\(f(x) = p_1(x)\),
则\(f(x)\)是不可约多项式.
由于不可约多项式的因式只有它的相伴元和零次多项式,
所以由\(q_1(x) \mid f(x)\)可得\(q_1(x) \sim f(x)\),
从而\(f(x) = c q_1(x)\ (c \in K-\{0\})\).
因此\(t=1\)且\(p_1(x) \sim q_1(x)\).
假设当第一个分解式的不可约因式的个数为\(s-1\)时,唯一性成立.
现在来看第一个分解式的不可约因式的个数为\(s\)的情形.
由于两个分解式相等,我们有\(p_1(x) \mid q_1(x) \dotsm q_t(x)\),
因此\(p_1(x)\)必能整除其中的一个.
不妨设\(p_1(x) \mid q_1(x)\).
因为\(q_1(x)\)也是不可约多项式,
所以\(p_1(x) \sim q_1(x)\),
即\[
	p_1(x) = c_1 q_1(x), \qquad
	c_1 \in K-\{0\},
\]
将上式代入分解式,并在等号两边消去\(q_1(x)\),可得\[
	p_2(x) \dotsm p_s(x) = c_1^{-1} q_2(x) \dotsm q_t(x).
\]
由归纳假设有\(s-1=t-1\),即\(s=t\),
并且我们只要适当排列因式的次序就能发现\[
	p_2(x) \sim c_1^{-1} q_2(x),
	p_3(x) \sim q_3(x),
	\dotsc,
	p_s(x) \sim q_s(x).
\]
也就是说\(p_i(x) \sim q_i(x)\ (i=1,\dotsc,s)\).
因此由归纳原理,唯一性普遍成立.
\end{proof}
\end{theorem}

从\hyperref[theorem:多项式.唯一因式分解定理]{唯一因式分解定理}可以看出,
\(f(x)\)的任一不可约因式一定与\(f(x)\)的分解式中的某一个不可约因式相伴.
因此,\(f(x)\)的分解式给出了它在相伴意义下的全部不可约因式.

\(K[x]\)中的唯一因式分解定理在理论上非常重要,
但是至今仍没有一个统一的方法来做因式分解,
也就是没有统一的方法求出一个次数大于零的多项式的所有不可约因式.

在多项式\(f(x)\)的分解式中,可以把每一个不可约因式的首项系数提出来,
使它们称为首项系数为\(1\)的多项式,再把相同的不可约因式的乘积写成乘幂的形式,
于是\(f(x)\)的分解式成为
\begin{equation}\label{equation:多项式.标准分解式}
	f(x) = c p_1^{r_1}(x) p_2^{r_2}(x) \dotsm p_m^{r_m}(x),
\end{equation}
其中\(c\)是\(f(x)\)的首项系数,
\(p_1(x),p_2(x),\dotsc,p_m(x)\)是不同的首项系数为\(1\)的不可约多项式,
\(\AutoTuple{r}{m}\)是正整数.
我们把分解式 \labelcref{equation:多项式.标准分解式}
称为“\(f(x)\)的\DefineConcept{标准分解式}”.

从理论研究的角度,如果已知两个多项式\(f(x)\)与\(g(x)\)的标准分解式\[
	f(x) = a p_1^{k_1}(x) \dotsm p_l^{k_l}(x) p_{l+1}^{k_{l+1}}(x) \dotsm p_m^{k_m}(x),
\]\[
	g(x) = b p_1^{t_1}(x) \dotsm p_l^{t_l}(x) q_{l+1}^{t_{l+1}}(x) \dotsm q_s^{t_s}(x),
\]
则\(f(x)\)与\(g(x)\)的最大公因式为\[
	(f(x),g(x))
	= p_1^{\min\{k_1,t_1\}}(x) \dotsm p_l^{\min\{k_l,t_l\}}(x).
\]

由于把多项式分解成不可约因式的乘积没有统一的方法,
因此上述最最大公因式的方法不能代替辗转相除法.

\begin{example}
%@see: 《高等代数(第三版 下册)》(丘维声) P27 例1
证明:\(x^2-2\)在有理数域上不可约.
\begin{proof}
若\(x^2-2\)在有理数域\(\mathbb{Q}\)上可约,
则它的标准分解式为\[
	x^2-2=(x+a)(x+b),
	\qquad a,b\in\mathbb{Q}.
\]
由于实数域\(\mathbb{R}\)是有理数域\(\mathbb{Q}\)的一个扩域,
所以上式也可以看成是\(x^2-2\)在实数域\(\mathbb{R}\)上的一个不可约因式分解.
另一方面我们知道\(x^2-2\)在\(\mathbb{R}\)上有如下的不可约因式分解:\[
	x^2-2=(x+\sqrt2)(x-\sqrt2).
\]
由\(\mathbb{R}[x]\)中的唯一因式分解定理得\[
	x+a=x+\sqrt2
	\quad\text{或}\quad
	x+a=x-\sqrt2.
\]
由此推出\(a=\sqrt2\)或\(a=-\sqrt2\),
这与\(a\in\mathbb{Q}\)矛盾!
因此\(x^2-2\)在\(\mathbb{Q}\)上不可约.
\end{proof}
\end{example}

\begin{example}
%@see: 《高等代数(第三版 下册)》(丘维声) P28 习题7.4 5.
证明:数域\(K\)上一个次数大于零的多项式\(f(x)\)
与\(K[x]\)中某一不可约多项式的正整数次幂相伴的充分必要条件是
对于任意\(g(x) \in K[x]\),
必有\((f(x),g(x))=1\),
或者存在一个正整数\(m\),
使得\(f(x) \mid g^m(x)\).
%TODO proof
\end{example}

\begin{example}
%@see: 《高等代数(第三版 下册)》(丘维声) P28 习题7.4 6.
证明:数域\(K\)上一个次数大于零的多项式\(f(x)\)
与\(K[x]\)中某一不可约多项式的正整数次幂相伴的充分必要条件是
对于任意\(g(x),h(x) \in K[x]\),
从\(f(x) \mid g(x) h(x)\)可以推出\(f(x) \mid g(x)\),
或者存在一个正整数\(m\),
使得\(f(x) \mid h^m(x)\).
%TODO proof
\end{example}

\begin{example}
%@see: 《高等代数(第三版 下册)》(丘维声) P28 习题7.4 7.
%@see: 《高等代数创新教材(下册)》(丘维声) P41 例2
在\(K[x]\)中,设\((f,g_2)=1\),证明:\((fg_1,g_2)=(g_1,g_2)\).
\begin{proof}
易见\((g_1,g_2)\)是\(fg_1\)与\(g_2\)的一个公因式.

若\(c(x) \mid f(x) g_1(x)\)
且\(c(x) \mid g_2(x)\),
由于\((f,g_2)=1\),
因此存在\(u(x),v(x) \in K[x]\),
使得\(u(x) f(x) + v(x) g_2(x) = 1\),
那么当\(g_1(x)\neq0\)时有\[
	u(x) f(x) g_1(x) + v(x) g_1(x) g_2(x) = g_1(x),
\]
因此\(c(x) \mid g_1(x)\),
从而\(c(x) \mid (g_1,g_2)\),
因此\((fg_1,g_2)=(g_1,g_2)\).

若\(g_1(x)=0\),
则\((fg_1,g_2)=(0,g_2)=(g_1,g_2)\).
\end{proof}
\end{example}

\begin{example}
%@see: 《高等代数创新教材(下册)》(丘维声) P42 习题7.4 4.
证明:在\(K[x]\)中,
对于不全为零的多项式\(f(x)\)与\(g(x)\),
以及任意正整数\(m\),
有\[
	(f^m(x),g^m(x))=(f(x),g(x))^m.
\]
%TODO proof
\end{example}
