\section{不可约多项式,唯一因式分解定理}
我们已经知道,数域\(K\)上的一元多项式环\(K[x]\)具有带余除法.
由此推导出,\(K[x]\)中任意两个多项式都有最大公因式.
现在我们利用这些结论来研究\(K[x]\)的结构.
与整数环\(\mathbb{Z}\)类比:
每一个正整数都能表示成有限多个素数的乘积.
我们不禁发问:\(K[x]\)中每一个多项式是否能表示成有限多个具有类似“素数”那样的性质的多项式的乘积?
联系我们对素数的定义,
对于一个大于\(1\)的正整数\(p\),如果它的正因子只有\(1\)和\(p\),那么称其为素数.
我们可以给出如下概念:
\begin{definition}
%@see: 《高等代数(第三版 下册)》(丘维声) P24 定义1
\(K[x]\)中一个次数大于零的多项式\(p(x)\),
如果它在\(K[x]\)中的因式只有零次多项式和\(p(x)\)的相伴元,
则称“\(p(x)\)是数域\(K\)上的一个\DefineConcept{不可约多项式}”;
否则称“\(p(x)\)是数域\(K\)上的一个\DefineConcept{可约多项式}”.
\end{definition}

\begin{property}%\label{theorem:多项式.不可约多项式.性质1}
%@see: 《高等代数(第三版 下册)》(丘维声) P24 性质1
\(K[x]\)中不可约多项式\(p(x)\)与任一多项式\(f(x)\)的关系只有两种可能:
\begin{enumerate}
	\item 要么\(p(x) \mid f(x)\).
	\item 要么\(p(x)\)与\(f(x)\)互素.
\end{enumerate}
\begin{proof}
由于\((p(x),f(x))\)是\(p(x)\)的因式,
而\(p(x)\)不可约,
因此\((p(x),f(x))\)是零次多项式,
即\((p(x),f(x)) \sim p(x)\).
从而\((p(x),f(x))=1\),
即\(p(x) \mid (p(x),f(x))\).
利用整除的传递性得出\(p(x) \mid f(x)\).
\end{proof}
\end{property}

\begin{property}\label{theorem:多项式.不可约多项式.性质2}
%@see: 《高等代数(第三版 下册)》(丘维声) P25 性质2
在\(K[x]\)中,如果\(p(x)\)不可约,且\(p(x) \mid f(x) g(x)\),
则\(p(x) \mid f(x)\)或\(p(x) \mid g(x)\).
\end{property}

利用数学归纳法,\cref{theorem:多项式.不可约多项式.性质2} 可以推广为:
在\(K[x]\)中,如果\(p(x)\)不可约,
且\[
	p(x) \mid f_1(x) f_2(x) \dotsm f_s(x),
\]
则对于某个\(j \in \Set{1,2,\dotsc,s}\),有\(p(x) \mid f_j(x)\).

\begin{property}\label{theorem:多项式.不可约多项式.性质3}
%@see: 《高等代数(第三版 下册)》(丘维声) P25 性质3
\(K[x]\)中,\(p(x)\)不可约,当且仅当\(p(x)\)不能分解成两个次数较\(p(x)\)的次数低的多项式的乘积.
\end{property}

从\cref{theorem:多项式.不可约多项式.性质3} 立即得出,
\(K[x]\)中的每一个\(1\)次多项式一定是不可约多项式.

\begin{theorem}\label{theorem:多项式.唯一因式分解定理}
%@see: 《高等代数(第三版 下册)》(丘维声) P25 定理1
\(K[x]\)中每一个次数大于零的多项式\(f(x)\)都能唯一地分解成数域\(K\)上有限多个不可约多项式的乘积.
\end{theorem}

我们把\cref{theorem:多项式.唯一因式分解定理} 称为\DefineConcept{唯一因式分解定理}.
从中可以看出,\(f(x)\)的任一不可约因式一定与\(f(x)\)的分解式中的某一个不可约因式相伴.
因此,\(f(x)\)的分解式给出了它在相伴意义下的全部不可约因式.

\(K[x]\)中的唯一因式分解定理在理论上非常重要,
但是至今仍没有一个统一的方法来做因式分解,
也就是没有统一的方法求出一个次数大于零的多项式的所有不可约因式.

在多项式\(f(x)\)的分解式中,可以把每一个不可约因式的首项系数提出来,
使它们称为首项系数为\(1\)的多项式,再把相同的不可约因式的乘积写成乘幂的形式,
于是\(f(x)\)的分解式成为
\begin{equation}\label{equation:多项式.标准分解式}
	f(x) = c p_1^{r_1}(x) p_2^{r_2}(x) \dotsm p_m^{r_m}(x),
\end{equation}
其中\(c\)是\(f(x)\)的首项系数,
\(p_1(x),p_2(x),\dotsc,p_m(x)\)是不同的首项系数为\(1\)的不可约多项式,
\(\AutoTuple{r}{m}\)是正整数.
我们把分解式 \labelcref{equation:多项式.标准分解式}
称为“\(f(x)\)的\DefineConcept{标准分解式}”.

从理论研究的角度,如果已知两个多项式\(f(x)\)与\(g(x)\)的标准分解式\[
	f(x) = a p_1^{k_1}(x) \dotsm p_l^{k_l}(x) p_{l+1}^{k_{l+1}}(x) \dotsm p_m^{k_m}(x),
\]\[
	g(x) = b p_1^{t_1}(x) \dotsm p_l^{t_l}(x) q_{l+1}^{t_{l+1}}(x) \dotsm q_s^{t_s}(x),
\]
则\(f(x)\)与\(g(x)\)的最大公因式为\[
	(f(x),g(x))
	= p_1^{\min\{k_1,t_1\}}(x) \dotsm p_l^{\min\{k_l,t_l\}}(x).
\]

由于把多项式分解成不可约因式的乘积没有统一的方法,
因此上述最最大公因式的方法不能代替辗转相除法.
