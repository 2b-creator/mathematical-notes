\section{有理数域上的不可约多项式}
这一节讨论有理数域上的不可约多项式有哪些,
如何判别一个有理系数多项式是否不可约.
这些问题的回答比复系数多项式和实系数多项式困难得多.

在\cref{section:多项式.多项式的根}的开头,
我们曾指出,
在\(K[x]\)中,
如果一个次数大于1的多项式\(p(x)\)不可约,
则\(p(x)\)没有一次因式,
从而\(p(x)\)在\(K\)中没有根,
这样就可以缩小讨论\(\mathbb{Q}[x]\)中不可约多项式的范围.
那么如何判别\(\mathbb{Q}[x]\)中次数大于1的多项式\(f(x)\)有没有有理根呢?
显然\(f(x)\)有有理根当且仅当\(f(x)\)在\(\mathbb{Q}[x]\)中的相伴元也有有理根.
因此很自然地选取\(f(x)\)在\(\mathbb{Q}[x]\)中的一个最简单的相伴元来研究.
例如,设\(f(x)=\frac12x^4+\frac13x^3-2x+1\),
则\(g(x)=3x^4+2x^3-12x+6\)就是\(f(x)\)的一个相伴元.
注意到\(g(x)\)是整系数多项式,
而它的各项系数的最大公因数只有\(\pm1\).
受此启发,我们可以给出如下概念.

\begin{definition}
%@see: 《高等代数(第三版 下册)》(丘维声) P42 定义1
一个非零的整系数多项式\[
	g(x)=b_n x^n+\dotsb+b_1 x+b_0,
\]
如果它的各项系数的最大公因数只有\(\pm1\),
则称“\(g(x)\)是\DefineConcept{本原的}”
或“\(g(x)\)是一个\DefineConcept{本原多项式}”.
\end{definition}

\begin{proposition}
任一非零的有理系数多项式都与一个本原多项式相伴.
\begin{proof}
只需求出有理系数多项式\[
	f(x)=a_n x^n+\dotsb+a_1 x+a_0
\]的各项系数的分母的最小公倍数\(m\),
提取公因数\(\frac1m\)得到\[
	f(x)=\frac1m(m a_n x^n+\dotsb+m a_1 x+m a_0);
\]
接着求出括号内多项式的各项系数的最大公因数\(c\),
就有\[
	f(x)=\frac{c}{m}(b_n x^n+\dotsb+b_1 x+b_0),
\]
其中\(b_i=\frac{m a_i}{c}\ (i=0,1,\dotsc,n)\),
\(g(x)=b_n x^n+\dotsb+b_1 x+b_0\)就是与\(f(x)\)相伴的本原多项式.
\end{proof}
\end{proposition}

我们不禁想要知道,
一个非零的有理系数多项式
可以与几个本原多项式相伴?

\begin{lemma}\label{theorem:多项式.有理数域上的不可约多项式.引理1}
%@see: 《高等代数(第三版 下册)》(丘维声) P42 引理1
两个本原多项式\(f(x)\)和\(g(x)\)在\(Q[x]\)中相伴
当且仅当\(f(x)=\pm g(x)\).
\begin{proof}
充分性是显然的.
下面证必要性.
设\(f(x),g(x)\)是相伴的本原多项式,
则存在\(r\in\mathbb{Q}-\{0\}\),
使得\(f(x)=r g(x)\).
设\[
	f(x)=\sum_{i=0}^n a_i x^i, \qquad
	g(x)=\sum_{i=0}^n b_i x^i,
\]
其中\(a_i,b_i\in\mathbb{Z}\ (i=0,1,\dotsc,n)\).
假设\(r\neq\pm1\),
不妨设\(r=\frac{q}{p}\),
其中\((p,q)=1\).
于是\(p,q\)两者中至少有一个不等于\(\pm1\).
不妨设\(p\neq\pm1\),
从而有\(p f(x)=q g(x)\).
比较各项系数可知\(p a_i=q b_i\ (i=0,1,\dotsc,n)\).
于是\(p \mid q b_i\).
因为\((p,q)=1\),
所以根据\cref{theorem:多项式.互素.性质1}
有\(p \mid b_i\),
这与“\(g(x)\)是本原多项式”矛盾.
因此\(r=\pm1\),
\(f(x)=\pm g(x)\).
\end{proof}
\end{lemma}

\cref{theorem:多项式.有理数域上的不可约多项式.引理1}
告诉我们,对于一个非零的有理系数多项式\(f(x)\),
与它在\(\mathbb{Q}[x]\)中相伴的本原多项式有且仅有两个,
它们相差一个正负号.
现在我们来研究本原多项式在\(\mathbb{Q}[x]\)中的不可约性问题.
为此首先介绍本原多项式的一个重要性质.

\begin{lemma}[高斯引理]\label{theorem:多项式.有理数域上的不可约多项式.引理2}
%@see: 《高等代数(第三版 下册)》(丘维声) P43 引理2
两个本原多项式的乘积还是本原多项式.
\begin{proof}
设\(
	f(x)=\sum_{i=0}^n a_i x^i,
	g(x)=\sum_{i=0}^n b_i x^i
\)是两个本原多项式,
又设\(
	h(x) = f(x) g(x) = \sum_{i=0}^{n+m} c_i x^i
\),
其中\(c_k=\sum_{i+j=k} a_i b_j\ (k=0,1,\dotsc,n+m)\).

假如\(h(x)\)不是本原多项式,
那么存在一个素数\(p\),
使得\(p\)是\(h(x)\)各项系数的公因式,
即\(p \mid c_k\ (k=0,1,\dotsc,n+m)\).
因为\(f(x)\)是本原的,
所以\(p\)不能同时整除\(f(x)\)的各项系数,
也就是说,存在\(k\ (0\leq k\leq n)\)满足\[
	p \mid a_0,
	p \mid a_1,
	\dotsc
	p \mid a_{k-1},
	p \nmid a_k.
	\eqno(1)
\]
同理,存在\(l\ (0\leq l\leq m)\)满足\[
	p \mid b_0,
	p \mid b_1,
	\dotsc
	p \mid b_{l-1},
	p \nmid b_l.
	\eqno(2)
\]

考虑\(h(x)\)的\(k+l\)次项的系数\[
	c_{k+l}
	= a_{k+l} b_0
	+ a_{k+l-1} b_1
	+ \dotsb
	+ a_1 b_{k+l-1}
	+ a_0 b_{k+l}.
\]
由(1)(2)两式可知\(p \nmid c_{k+l}\)
(注意从\(p \nmid a_k\)且\(p \nmid b_l\)
可以推出\(p \nmid a_k b_l\)),矛盾!
因此\(h(x)\)是本原多项式.
\end{proof}
\end{lemma}

\begin{theorem}\label{theorem:多项式.有理数域上的不可约多项式.本原多项式的分解}
%@see: 《高等代数(第三版 下册)》(丘维声) P43 定理1
一个次数大于零的本原多项式\(g(x)\)在\(\mathbb{Q}\)可约
当且仅当\(g(x)\)可以分解成两个次数都比\(g(x)\)的次数低的本原多项式的乘积.
\begin{proof}
充分性是显然的.
下面证必要性.

设本原多项式\(g(x)\)在\(\mathbb{Q}\)上可约,
则存在\(g_1(x),g_2(x)\in\mathbb{Q}[x]\),
使得\(g(x) = g_1(x) g_2(x)\),
其中\(\deg g_1(x) < \deg g(x)\),
\(\deg g_2(x) < \deg g(x)\).
设\(g_i(x)=r_i h_i(x)\ (i=1,2)\),
其中\(r_1,r_2\in\mathbb{Q}^*\),
而\(h_1(x),h_2(x)\)是本原多项式,
则\[
	g(x) = r_1 r_2 h_1(x) h_2(x).
\]
由于根据\cref{theorem:多项式.有理数域上的不可约多项式.引理2}
两个本原多项式的乘积\(h_1(x) h_2(x)\)也是本原多项式,
因此根据\cref{theorem:多项式.有理数域上的不可约多项式.引理1}
有\(r_1 r_2 = \pm1\).
从而\(g(x)=[\pm h_1(x)]\cdot h_2(x)\).
由于\(\deg(\pm h_1(x))
= \deg g_1(x)
< \deg g(x)\),
\(\deg h_2(x)
= \deg g_2(x)
< \deg g(x)\),
因此\(g(x)\)分解成了两个次数较低的本原多项式的乘积.
\end{proof}
\end{theorem}

\begin{corollary}
%@see: 《高等代数(第三版 下册)》(丘维声) P43 推论2
如果一个次数大于零的整系数多项式在\(\mathbb{Q}\)上可约,
则它可以分解成两个次数比它低的整系数多项式的乘积.
\begin{proof}
设\(f(x)\)是一个次数大于零的整系数多项式,
在\(\mathbb{Q}\)上可约,
则\(f(x)= r g(x)\),
其中\(r\in\mathbb{Z}^*\),
\(g(x)\)是本原多项式.
根据\cref{theorem:多项式.有理数域上的不可约多项式.本原多项式的分解}
可知\(g(x)\)可以分解为\(h_1(x),h_2(x)\)
这两个次数都比\(g(x)\)的次数低的本原多项式,
即\(g(x)=h_1(x) h_2(x)\),
从而\(f(x)=[r h_1(x)] h_2(x)\).
这表明\(f(x)\)分解成了两个次数较低的整系数多项式的乘积.
\end{proof}
\end{corollary}

\begin{theorem}\label{theorem:多项式.有理数域上的不可约多项式.高次本原多项式可唯一分解为不可约本原多项式的乘积}
%@see: 《高等代数(第三版 下册)》(丘维声) P44 定理3
每一个次数大于零的本原多项式\(g(x)\)
可以唯一地分解成\(\mathbb{Q}\)上不可约的本原多项式的乘积.
\begin{proof}
设\(g(x)\)是一个次数大于零的本原多项式.

先证可分解性.
对本原多项式的次数\(n\)作数学归纳法.
当\(n=1\)时,
显然有\(g(x)=g(x)\),
且\(g(x)\)不可约.
假设任一次数小于\(n\)的本原多项式
都可以分解成\(\mathbb{Q}\)上不可约的本原多项式的乘积.
现在来看\(n\)次本原多项式\(g(x)\).
如果\(g(x)\)在\(\mathbb{Q}\)上不可约,
则\(g(x)=g(x)\),
可分解性成立.
如果\(g(x)\)在\(\mathbb{Q}\)上可约,
那么根据\cref{theorem:多项式.有理数域上的不可约多项式.本原多项式的分解} 得,
\(g(x)=g_1(x) g_2(x)\),
其中\(g_1(x),g_2(x)\)是本原多项式,
且\(\deg g_1(x) < \deg g(x),
\deg g_2(x) < \deg g(x)\),
可分解性也成立.

再证唯一性.
根据\(\mathbb{Q}[x]\)的\hyperref[theorem:多项式.唯一因式分解定理]{唯一因式分解定理}得,
\(s=t\),
且只要适当排列因式次序
就有\(p_i(x) \sim q_i(x)\ (i=1,2,\dotsc,s)\).
由于\(p_i(x),q_i(x)\)都是本原多项式,
因此\(p_i(x)=\pm q_i(x)\ (i=1,2,\dotsc,s)\).
\end{proof}
\end{theorem}

\begin{theorem}
%@see: 《高等代数(第三版 下册)》(丘维声) P44 定理4
设\(f(x)=a_n x^n+a_{n-1} x^{n-1}+\dotsb+a_1 x+a_0\)
是一个次数\(n\)大于零的整系数多项式.
如果\(\frac{q}{p}\)是\(f(x)\)的一个有理根,
其中\(p,q\)是互素的整数,
那么\(p \mid a_n,
q \mid a_0\).
\begin{proof}
设\(f(x)=r f_1(x)\),
其中\(r\in\mathbb{Z}^*\),
\(f_1(x)\)是本原多项式.
又设\(\frac{q}{p}\)是\(f(x)\)的一个根,
其中\(p,q\)是互素的整数,
则\(\frac{q}{p}\)是\(f_1(x)\)的一个根.
于是在\(\mathbb{Q}[x]\)中,
有\(\left(x-\frac{q}{p}\right) \mid f_1(x)\),
从而\((px-q) \mid f_1(x)\).
由于\((p,q)=1\),
因此\(px-q\)是本原多项式.
根据\cref{theorem:多项式.有理数域上的不可约多项式.高次本原多项式可唯一分解为不可约本原多项式的乘积}
和\hyperref[theorem:多项式.有理数域上的不可约多项式.引理2]{高斯引理}得\[
	f_1(x)=(px-q) g(x),
\]
其中\(g(x)=b_{n-1} x^{n-1}+\dotsb+b_1 x+b_0\)是本原多项式.
于是\[
	f(x)=r(px-q) g(x).
\]
分别比较上式等号两边的首项系数与常数项,得\[
	a_n = r p b_{n-1}, \qquad
	a_0 = -r q b_0.
\]
因此\(p \mid a_n,
q \mid a_0\).
\end{proof}
\end{theorem}
