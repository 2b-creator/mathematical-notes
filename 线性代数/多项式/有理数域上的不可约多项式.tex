\section{有理数域上的不可约多项式}
这一节讨论有理数域上的不可约多项式有哪些,
如何判别一个有理系数多项式是否不可约.
这些问题的回答比复系数多项式和实系数多项式困难得多.

在\cref{section:多项式.多项式的根}的开头,
我们曾指出,
在\(K[x]\)中,
如果一个次数大于1的多项式\(p(x)\)不可约,
则\(p(x)\)没有一次因式,
从而\(p(x)\)在\(K\)中没有根,
这样就可以缩小讨论\(\mathbb{Q}[x]\)中不可约多项式的范围.
那么如何判别\(\mathbb{Q}[x]\)中次数大于1的多项式\(f(x)\)有没有有理根呢?
显然\(f(x)\)有有理根当且仅当\(f(x)\)在\(\mathbb{Q}[x]\)中的相伴元也有有理根.
因此很自然地选取\(f(x)\)在\(\mathbb{Q}[x]\)中的一个最简单的相伴元来研究.
例如,设\(f(x)=\frac12x^4+\frac13x^3-2x+1\),
则\(g(x)=3x^4+2x^3-12x+6\)就是\(f(x)\)的一个相伴元.
注意到\(g(x)\)是整系数多项式,
而它的各项系数的最大公因数只有\(\pm1\).
受此启发,我们可以给出如下概念.

\begin{definition}
%@see: 《高等代数(第三版 下册)》(丘维声) P42 定义1
一个非零的整系数多项式\[
	g(x)=b_n x^n+\dotsb+b_1 x+b_0,
\]
如果它的各项系数的最大公因数只有\(\pm1\),
则称“\(g(x)\)是一个\DefineConcept{本原多项式}”.
\end{definition}

\begin{proposition}
任一非零的有理系数多项式都与一个本原多项式相伴.
\begin{proof}
只需求出有理系数多项式\[
	f(x)=a_n x^n+\dotsb+a_1 x+a_0
\]的各项系数的分母的最小公倍数\(m\),
提取公因数\(\frac1m\)得到\[
	f(x)=\frac1m(m a_n x^n+\dotsb+m a_1 x+m a_0);
\]
接着求出括号内多项式的各项系数的最大公因数\(c\),
就有\[
	f(x)=\frac{c}{m}(b_n x^n+\dotsb+b_1 x+b_0),
\]
其中\(b_i=\frac{m a_i}{c}\ (i=0,1,\dotsc,n)\),
\(g(x)=b_n x^n+\dotsb+b_1 x+b_0\)就是与\(f(x)\)相伴的本原多项式.
\end{proof}
\end{proposition}
