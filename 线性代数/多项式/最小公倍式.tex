\section{最小公倍式}
\begin{definition}
在\(K[x]\)中,如果\(c(x)\)既是\(f(x)\)的倍式,又是\(g(x)\)的倍式,
则称“\(c(x)\)是\(f(x)\)与\(g(x)\)的一个\DefineConcept{公倍式}”.
\end{definition}

\begin{definition}
%@see: 《高等代数(第三版 下册)》(丘维声) P22 习题7.3 10.
设\(f(x),g(x) \in K[x]\),
\(m(x)\)是\(f(x)\)与\(g(x)\)的一个公倍式.
如果\(f(x)\)与\(g(x)\)的任一公倍式都是\(m(x)\)的倍式,
则称“\(m(x)\)是\(f(x)\)与\(g(x)\)的一个\DefineConcept{最小公倍式}”.
\end{definition}

我们约定,用\[
	[f(x), g(x)]
\]表示首项系数是\(1\)的那个最小公倍式.

\begin{example}
%@see: 《高等代数(第三版 下册)》(丘维声) P22 习题7.3 10.(1)
%@see: 《高等代数创新教材(下册)》(丘维声) P35 例10(1)
证明:\(K[x]\)中任意两个多项式都有最小公倍式,
并且在相伴的意义下是唯一的.
\begin{proof}
首先,零多项式的倍式只有零多项式,
从而任一多项式\(f\)与\(0\)的最小公倍式是\(0\).

下面设\(f,g\)都是数域\(K\)上的非零多项式.
令\(d=(f,g)\),
则存在\(u,v \in K[x]\)
使得\(f=ud,g=vd\).
又令\(m=uvd\),
显然\(f \mid m\)且\(g \mid m\),
也就是说\(m\)是\(f\)与\(g\)的一个公倍式.
设\(c \in K[x]\)是\(f\)与\(g\)的一个公倍式,
则存在\(p,q \in K[x]\)
使得\(c=pf,c=qg\).
于是\(pf=qg\),
从而\(pud=qvd\),
消去\(d\)便得\(pu=qv\),
可见\(u \mid qv\);
但是由\cref{example:最大公因式.最大公因式除多项式的商式互素}
有\((u,v)=1\);
因此根据\cref{theorem:多项式.互素.性质1}
必有\(u \mid q\),
从而存在\(h \in K[x]\)
使得\(q=hu\).
于是\(c=hug=hm\),
因此\(m \mid c\),
也就是说\(m\)是\(f\)与\(g\)的最小公倍式.

假设\(m_1,m_2\)都是\(f\)与\(g\)的最小公倍式,
则\(m_1 \mid m_2,
m_2 \mid m_1\),
因此\(m_1 \sim m_2\).
\end{proof}
\end{example}

\begin{example}
%@see: 《高等代数(第三版 下册)》(丘维声) P22 习题7.3 10.(2)
%@see: 《高等代数创新教材(下册)》(丘维声) P35 例10(2)
证明:如果\(f(x),g(x)\)的首项系数都是\(1\),
则\[
	[f(x),g(x)]
	= \frac{f(x) g(x)}{(f(x),g(x))}.
\]
%TODO proof
\end{example}
