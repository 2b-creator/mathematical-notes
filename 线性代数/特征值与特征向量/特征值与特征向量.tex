\section{矩阵的特征值与特征向量}
\subsection{特征值、特征向量的概念与性质}
\begin{definition}
%@see: 《高等代数(第三版 上册)》(丘维声) P172. 定义1
设矩阵\(\A \in M_n(K)\).
如果\[
	(\exists \lambda_0 \in K)
	(\exists \x_0 \in K^n - \{\vb0\})
	[\A\x_0 = \lambda_0\x_0],
\]
则称“\(\lambda_0\)是\(\A\)的一个\DefineConcept{特征值}(eigenvalue)”,
称“\(\x_0\)是\(\A\)的属于特征值\(\lambda_0\)的一个\DefineConcept{特征向量}(eigenvector)”.
\end{definition}

容易观察到,当特征值\(\lambda=0\)时,\(\l\)对应的特征向量都是齐次方程\(\A\x=\z\)的解.
当\(\abs{\A}\neq0\)时,这个方程只有零解,因此,一个矩阵有特征值\(\lambda=0\)说明它不满秩.

\begin{property}\label{theorem:矩阵的特征值与特征向量.性质1}
若\(\x_1\),\(\x_2\)是\(\A\)属于同一个特征值\(\lambda_0\)的特征向量,
且\(\x_1 + \x_2 \neq 0\),则\(\x_1 + \x_2\)也是\(\A\)的属于\(\lambda_0\)的特征向量.
\begin{proof}
\(\A(\x_1+\x_2)=\A\x_1+\A\x_2=\lambda_0\x_1+\lambda_0\x_2=\lambda_0(\x_1+\x_2)\).
\end{proof}
\end{property}

\begin{property}\label{theorem:矩阵的特征值与特征向量.性质2}
若\(\x_0\)是\(\A\)属于特征值\(\lambda_0\)的特征向量,\(k\)为任意非零常数,
则\(k\x_0\)也是\(\A\)的属于\(\lambda_0\)的特征向量.
\begin{proof}
因为\(k\x_0\neq\z\),\(\A(k\x_0)=k(\A\x_0)=k(\lambda_0\x_0)=\lambda_0(k\x_0)\),
所以\(k\x_0\)也是\(\A\)的属于\(\lambda_0\)的特征向量.
\end{proof}
\end{property}

\begin{remark}
由\cref{theorem:矩阵的特征值与特征向量.性质1,theorem:矩阵的特征值与特征向量.性质2} 可知,
\(\A\)的属于同一个特征值\(\lambda_0\)的
特征向量\(\AutoTuple{\x}{t}\)的非零线性组合
\(\sum_{i=1}^t{k_i \X i}\)
也是\(\A\)的属于\(\lambda_0\)的特征向量.
\end{remark}

\begin{example}
设\(\A \in M_n(\mathbb{R})\),
\(\A\)的各行元素之和均为\(\lambda_0\).
证明:\(\lambda_0\)是\(\A\)的特征值,
\(n\)维列向量\(\x_0=(1,\dotsc,1)^T\)是\(\A\)属于\(\lambda_0\)的特征向量.
\begin{proof}
记\[
	\A
	= \begin{bmatrix}
		a_{11} & a_{12} & \dots & a_{1n} \\
		a_{21} & a_{22} & \dots & a_{2n} \\
		\vdots & \vdots & & \vdots \\
		a_{n1} & a_{n2} & \dots & a_{nn}
	\end{bmatrix},
\]
那么\[
	\A \x_0
	= \begin{bmatrix}
		a_{11} & a_{12} & \dots & a_{1n} \\
		a_{21} & a_{22} & \dots & a_{2n} \\
		\vdots & \vdots & & \vdots \\
		a_{n1} & a_{n2} & \dots & a_{nn}
	\end{bmatrix}
	\begin{bmatrix}
		1 \\ 1 \\ \vdots \\ 1
	\end{bmatrix}
	= \begin{bmatrix}
		a_{11} + a_{12} + \dotsb + a_{1n} \\
		a_{21} + a_{22} + \dotsb + a_{2n} \\
		\vdots \\
		a_{n1} + a_{n2} + \dotsb + a_{nn}
	\end{bmatrix}
	= \begin{bmatrix}
		\lambda_0 \\ \lambda_0 \\ \vdots \\ \lambda_0
	\end{bmatrix}
	= \lambda_0
	\begin{bmatrix}
		1 \\ 1 \\ \vdots \\ 1
	\end{bmatrix},
\]
由此可见,\(\lambda_0\)是\(\A\)的特征值,
\(n\)维列向量\(\x_0=(1,\dotsc,1)^T\)是\(\A\)属于\(\lambda_0\)的特征向量.
\end{proof}
\end{example}

\begin{property}
设\(\lambda_0\)是\(\A\)的特征值.
当\(m\in\mathbb{N}\)时,\(\lambda_0^m\)是\(\A^m\)的特征值.
若\(\A\)可逆,则当\(m\in\mathbb{Z}\)时,\(\lambda_0^m\)是\(\A^m\)的特征值.
\begin{proof}
由定义,存在\(\x_0\neq\z\),
使得\(\A\x_0 = \lambda_0\x_0\),则\[
	\A^2\x_0 = \A(\A\x_0)
	=\A(\lambda_0\x_0)
	=\lambda_0(\A\x_0)
	=\lambda_0(\lambda_0\x_0)
	=\lambda_0^2\x_0.
\]
设\(\A^{m-1}\x_0 = \lambda_0^{m-1}\x_0\)成立,则\[
	\A^m\x_0 = \A(\A^{m-1}\x_0)
	= \A(\lambda_0^{m-1}\x_0)
	= \lambda_0^{m-1}(\A\x_0)
	= \lambda_0^{m-1}(\lambda_0\x_0)
	= \lambda_0^m\x_0.
\]

当\(\A\)可逆时,\(\lambda_0\neq0\),由\(\lambda_0\x_0 = \A\x_0\)可得\[
	\lambda_0(\A^{-1}\x_0)
	= \A^{-1}(\lambda_0\x_0)
	= \A^{-1}(\A\x_0)
	= (\A^{-1}\A)\x_0
	= \E\x_0
	= \x_0,
\]
从而有\(\A^{-1}\x_0 = \lambda_0^{-1}\x_0\).
\end{proof}
\end{property}

\begin{corollary}
设多项式\(f(x)=\sum_k a_k x^k\).
若\(\lambda_0\)是方阵\(\A\)的特征值,则\(f(\lambda_0)\)是\(f(\A)\)的特征值.
\end{corollary}

\begin{definition}
设\(\A=(a_{ij})_n \in M_n(K)\),\(\l \in K\).
把\(\lambda\E-\A\)称为“\(\A\)的\DefineConcept{特征矩阵}”.
把特征矩阵的行列式\[
	f(\l)
	= \abs{\lambda\E-\A}
	= \begin{vmatrix}
		\l-a_{11} & -a_{12} & \dots & -a_{1n} \\
		-a_{21} & \l-a_{22} & \dots & -a_{2n} \\
		\vdots & \vdots & & \vdots \\
		-a_{n1} & -a_{n2} & \dots & \l-a_{nn}
	\end{vmatrix}
\]称为“\(\A\)的\DefineConcept{特征多项式}(eigenpolynomial)”.
\end{definition}

\begin{theorem}
%@see: 《高等代数(第三版 上册)》(丘维声) P174. 定理1
设\(\A \in M_n(K)\),
则\begin{align*}
	\text{\(\lambda_0\)是\(\A\)的特征值}
	&\iff
	\abs{\lambda_0\E-\A}=0 \land \lambda_0 \in K \\
	&\iff
	\text{\(\lambda_0\)是特征多项式\(\abs{\lambda\E-\A}\)在\(K\)中的一个根},
\end{align*}
\[
	\text{\(\x_0\)是\(\A\)的属于\(\lambda_0\)的特征向量}
	\iff
	\text{\(\x_0\)是齐次线性方程组\((\lambda_0\E-\A)\x=\z\)的非零解}
\]
\end{theorem}

\begin{definition}
设\(\lambda\)是\(\A\)的一个特征值,
我们把齐次线性方程\((\lambda\E-\A)\x=\vb0\)的解空间
称为“\(\A\)的属于\(\lambda\)的\DefineConcept{特征子空间}”.
\end{definition}

\begin{proposition}
\(\A\)的属于\(\lambda\)的特征子空间的全体非零向量
恰是\(\A\)的属于\(\lambda\)的全体特征向量.
\end{proposition}

大数学家高斯在1799年证明了以下\DefineConcept{代数基本定理}:
\begin{lemma}[代数基本定理]
任何\(n\ (n\geq1)\)次多项式至少有一个复数根.
\end{lemma}

\begin{theorem}[代数基本定理']
任何\(n\ (n>0)\)次多项式有且仅有\(n\)个复根,其中规定\(m\)重根算\(m\)个根.
\end{theorem}
由此可知,任意\(n\)阶矩阵的特征多项式有且仅有\(n\)个复根.

\begin{theorem}
设\(\A \in M_n(\mathbb{C})\),
则\[
	\text{\(\lambda\)是\(\A\)的特征值}
	\iff
	\text{\(\lambda\)是\(\A^T\)的特征值}.
\]
\begin{proof}
因为\[
	\abs{\lambda\E-\A^T}
	= \abs{(\lambda\E-\A^T)^T}
	= \abs{(\lambda\E)^T-(\A^T)^T}
	= \abs{\lambda\E-\A},
\]
所以\(\A\)与\(\A^T\)具有相同的特征值.
\end{proof}
\end{theorem}

\begin{example}
设\(\A = \begin{bmatrix}
	2 & -1 & 2 \\
	5 & -3 & 3 \\
	-1 & 0 & -2
\end{bmatrix}\),
求\(\A\)的特征值与对应的特征向量.
\begin{solution}
\(\A\)的特征多项式\begin{align*}
	\abs{\lambda\E-\A}
	&= \begin{vmatrix}
		\l-2 & 1 & -2 \\
		-5 & \l+2 & -3 \\
		1 & 0 & \l+2
	\end{vmatrix} \\
	&= \l^3 + 3\l^2 + 3\l + 1
	= (\l+1)^3,
\end{align*}
特征值为\(\L{1}=-1\)(三重).

当\(\lambda=-1\)时,解方程组\((-\E-\A)\x = \z\),对系数矩阵施行初等行变换,\[
	-\E-\A = \begin{bmatrix}
		-3 & 1 & -2 \\
		-5 & 2 & -3 \\
		1 & 0 & 1
	\end{bmatrix} \to \begin{bmatrix}
		1 & 0 & 1 \\
		5 & 2 & -3 \\
		-3 & 1 & -2
	\end{bmatrix} \to \begin{bmatrix}
		1 & 0 & 1 \\
		0 & 1 & 1 \\
		0 & 0 & 0
	\end{bmatrix}.
\]
可知\(\rank(-\E-\A) = 2\),
方程组\((-\E-\A)\x = \z\)的解空间只有1个基向量.
令\(x_3 = -1\),得基础解系\[
	\x_1 = (1,1,-1)^T,
\]
那么属于\(-1\)的全部特征向量为\(k (1,1,-1)^T\),
\(k\)是非零的任意常数.
\end{solution}
\end{example}

\begin{example}
设\(\A = \begin{bmatrix}
	-1 & 0 & 0 \\
	8 & 2 & 4 \\
	8 & 3 & 3
\end{bmatrix}\),
求\(\A\)的特征值与对应的特征向量.
\begin{solution}
\(\A\)的特征多项式\[
	\abs{\lambda\E-\A}
	= \begin{vmatrix}
		\l+1 & 0 & 0 \\
		-8 & \l-2 & -4 \\
		-8 & -3 & \l-3
	\end{vmatrix}
	= (\l+1)^2 (\l-6),
\]
故\(\A\)的特征值为\(\L{1}=-1\)(二重),\(\L{2}=6\).

当\(\lambda=-1\)时,解方程组\((-\E-\A)\x = \z\),\[
	-\E-\A = \begin{bmatrix}
		0 & 0 & 0 \\
		-8 & -3 & -4 \\
		-8 & -3 & -4
	\end{bmatrix} \to \begin{bmatrix}
		-8 & -3 & -4 \\
		0 & 0 & 0 \\
		0 & 0 & 0
	\end{bmatrix},
\]
可知\(\rank(-\E-\A) = 1\),
方程组\((-\E-\A)\x = \z\)的解空间有2个基向量.
分别令\(x_2 = 8, x_3 = 0\)和\(x_2 = 0, x_3 = 2\),得基础解系\[
	\x_1 = \begin{bmatrix} -3 \\ 8 \\ 0 \end{bmatrix},
	\qquad
	\x_2 = \begin{bmatrix} -1 \\ 0 \\ 2 \end{bmatrix},
\]
故属于\(-1\)的全部特征向量为\(k_1 \x_1 + k_2 \x_2\),
\(k_1,k_2\)是不全为零的任意常数.

当\(\lambda=6\)时,解方程组\((6\E-\A)\x = \z\),\[
	6\E-\A = \begin{bmatrix}
		7 & 0 & 0 \\
		-8 & 4 & -4 \\
		-8 & -3 & 3
	\end{bmatrix} \to \begin{bmatrix}
		1 & 0 & 0 \\
		0 & 1 & -1 \\
		0 & 0 & 0
	\end{bmatrix},
\]
可知\(\rank(6\E-\A) = 2\),方程组\((6\E-\A)\x = \z\)的解空间有1个基向量.
令\(x_3 = 1\),得\(x_1 = 0, x_2 = 1\),基础解系\[
	\x_3 = (0,1,1)^T,
\]
故属于\(6\)的全部特征向量为\(k_3 \x_3\),\(k_3\)是任意非零常数.
\end{solution}
\end{example}

从特征值、特征向量的性质可以看出,
矩阵\(\A\)的一个特征值对应若干个线性无关的特征向量;
但反之,一个特征向量只能属于一个特征值.
事实上,设\(\x_0\)为某个矩阵\(\A\)的特征向量,若有\(\L{1},\L{2}\)满足\[
	\A\x_0=\L{1}\x_0,
	\quad
	\A\x_0=\L{2}\x_0,
\]
则必有\(\L{1}\x_0=\L{2}\x_0\)或\((\L{1}-\L{2})\x_0=\z\),
因为\(\x_0\neq\z\),所以\(\L{1}-\L{2}=0\),\(\L{1}=\L{2}\).

前面已经知道,
矩阵\(\A\)的同一个特征值\(\lambda_0\)对应的特征向量的非零线性组合仍为\(\A\)的属于\(\lambda_0\)的特征向量.
那么,\(\A\)的不同特征值对应的特征向量的非零线性组合又如何呢?
\begin{example}
设\(\L{1}\)、\(\L{2}\)是矩阵\(\A\)的两个不同的特征值,
\(\x_1\)、\(\x_2\)分别是\(\L{1}\)、\(\L{2}\)对应的特征向量.
证明:\(\x_1+\x_2\)不是\(\A\)的特征向量.
\begin{proof}
\(\A\x_1 = \L{1}\x_1\),\(\A\x_2 = \L{2}\x_2\),
假设\(\A(\x_1+\x_2) = \lambda_0(\x_1+\x_2)\),则\[
	\A\x_1+\A\x_2 =\L{1}\x_1+\L{2}\x_2 = \lambda_0\x_1+\lambda_0\x_2,
\]\[
	(\lambda_0-\L{1})\x_1+(\lambda_0-\L{2})\x_2 = \z,
\]
在上式左右两端同乘\(\A\)和\(\L{1}\)可得\[
	\left\{ \begin{array}{l}
		(\lambda_0-\L{1})\A\x_1+(\lambda_0-\L{2})\A\x_2 = (\lambda_0-\L{1})\L{1}\x_1 + (\lambda_0-\L{2})\L{2}\x_2 = \z, \\
		(\lambda_0-\L{1})\L{1}\x_1+(\lambda_0-\L{2})\L{1}\x_2 = \z,
	\end{array} \right.
\]\[
	(\lambda_0-\L{2})(\L{2}-\L{1})\x_2 = \z,
\]
因为\(\x_2\neq\z\),所以\((\lambda_0-\L{2})(\L{2}-\L{1})=0\);
又因为\(\L{2}\neq\L{1}\),所以\(\lambda_0=\L{2}\).
同理有\[
	(\lambda_0-\L{1})\L{2}\x_1+(\lambda_0-\L{2})\L{2}\x_2 = \z
	\implies
	(\lambda_0-\L{1})(\L{1}-\L{2})\x_1 = \z,
\]
因为\(\x_1\neq\z\),所以\(\lambda_0=\L{1}\).

于是导出\(\L{1}=\L{2}\),与题设矛盾,说明\(\x_1+\x_2\)不是\(\A\)的特征向量.
\end{proof}
\end{example}

\begin{example}
设\(\lambda_0\)是矩阵\(\A\)的特征值,\(k\)是任意常数,则\(k\lambda_0\)是矩阵\(k\A\)的特征值.
\begin{proof}
\(\A\x_0=\lambda_0\x_0\),\((k\A)\x_0=k(\A\x_0)=k(\lambda_0\x_0)=(k\lambda_0)\x_0\).
\end{proof}
\end{example}

\begin{example}
证明:矩阵\(\A\)与其转置矩阵\(\A^T\)的特征值相同.
\begin{proof}
因为矩阵\(\A\)与\(\A^T\)的特征多项式相同,
即\(\abs{\lambda\E-\A} = \abs{(\lambda\E-\A)^T} = \abs{\lambda\E-\A^T}\),所以特征值相同.
\end{proof}
\end{example}

\begin{example}
证明:幂等矩阵的特征值只能为0或1.
\begin{proof}
设\(\A\)是幂等矩阵,即有\(\A^2=\A\).
设\(\A\x_0=\lambda_0\x_0\ (\x_0\neq\z)\),
则\[
	\A^2\x_0=\A(\A\x_0)=\A(\lambda_0\x_0)=\lambda_0(\A\x_0)=\lambda_0(\lambda_0\x_0)=\lambda_0^2\x_0.
\]
因为\[
	\A^2=\A
	\iff
	\A^2-\A=\z
	\implies
	\A^2\x_0-\A\x_0=(\A^2-\A)\x_0=\z\x_0=\z,
\]
所以\[
	\lambda_0^2\x_0-\lambda_0\x_0=(\lambda_0^2-\lambda_0)\x_0=\lambda_0(\lambda_0-1)\x_0=\z,
\]
进一步有\(\lambda_0(\lambda_0-1)=0\),所以\(\lambda_0=0\)或\(\lambda_0=1\).
\end{proof}
\end{example}

\begin{example}
试讨论:在什么条件下,矩阵\(\A\)的任一特征向量总是\(\A^T\)的特征向量.
\begin{solution}
假设\(\A\)的特征向量都是\(\A^T\)的特征向量,
\(\A\x=\L{1}\x\),\(\A^T\x=\L{2}\x\),那么有\[
	(\A^T\x)^T=(\L{2}\x)^T
	\implies
	\x^T\A=\L{2}\x^T
	\implies
	(\x^T\A)\x=(\L{2}\x^T)\x,
\]\[
	\x^T(\A\x)=\x^T(\L{1}\x)=\L{1}\x^T\x=\L{2}\x^T\x
	\implies
	(\L{1}-\L{2})\x^T\x=0,
\]
因为\(\x\neq\z\),\(\x^T\x\neq0\),
所以\(\L{1}-\L{2}=0\),\(\L{1}=\L{2}\).
那么\[
	\A\x=\L{1}\x=\L{2}\x=\A^T\x,
\]
从而\((\A-\A^T)\x=\z\),\(\A=\A^T\).
也就是说,当且仅当\(\A\)是对称矩阵时,\(\A\)的特征向量都是\(\A^T\)的特征向量.
\end{solution}
\end{example}

\begin{example}
设\(n\)阶矩阵\(\A=(a_{ij})_n\)的特征多项式为\[
	f(\l) = \abs{\lambda\E-\A}
	= \l^n - a_1 \l^{n-1} + a_2 \l^{n-2} - \dotsb + (-1)^n a_n,
\]
证明:系数\(a_i\)是矩阵\(\A\)的\(i\)阶主子式之和(\(i=1,2,\dotsc,n\)).
特别地,\(a_1 = \tr\A\),\(a_n = \abs{\A}\).
\end{example}

\begin{example}
设\(n\)阶矩阵\(\A=(a_{ij})_n\)的特征值为\(\L{1},\L{2},\dotsc,\L{n}\),
即\(\A\)的特征多项式为\(\abs{\lambda\E-\A}=(\l-\L{1})(\l-\L{2})\dotsm(\l-\L{n})\).
证明:
\begin{enumerate}
	\item \(\abs{\A} = \L{1} \L{2} \dotsm \L{n}\);
	\item \(\L{1} + \L{2} + \dotsb + \L{n} = \tr\A\).
\end{enumerate}
\begin{proof}
比较特征多项式的两种形式:\[
	\abs{\lambda\E-\A}
	=\l^n-\l^{n-1}(a_{11}+a_{22}+ \dotsb +a_{nn})+ \dotsb +(-1)^n\abs{\A}
\]
和\begin{align*}
	\abs{\lambda\E-\A} &= (\l-\L{1})(\l-\L{2})\dotsm(\l-\L{n}) \\
	&= \l^n-\l^{n-1}(\L{1}+\L{2}+ \dotsb +\L{n}) + \dotsb + (-1)^n \L{1} \L{2} \dotsm \L{n},
\end{align*}
可得\[
	\L{1} + \L{2} + \dotsb + \L{n} = a_{11} + a_{22} + \dotsb + a_{nn} = \tr\A,
\]\[
	\abs{\A} = \L{1} \L{2} \dotsm \L{n}.
	\qedhere
\]
\end{proof}
\end{example}

\begin{example}\label{example:矩阵乘积的秩.两个向量的乘积的特征值和特征向量}
设\(\a,\b\)是\(n\)维非零列向量.
证明:求矩阵\(\a\b^T\)的特征值和特征向量.
\begin{proof}
显然有\((\a\b^T)\a = \a(\b^T\a)\),
那么根据定义可知\(\b^T\a\)就是矩阵\(\a\b^T\)的特征值,
\(\a\)是\(\a\b^T\)属于\(\b^T\a\)的特征向量.

又由\cref{example:行列式.两个向量的乘积矩阵的行列式} 可知\(\abs{\a\b^T} = 0\),
所以\(\abs{0\cdot\E-\a\b^T}=0\),也就是说\(0\)也是矩阵\(\a\b^T\)的特征值.
再由\cref{example:矩阵乘积的秩.两个向量的乘积的秩} 可知\(\rank(\a\b^T) = 1\),
于是根据\cref{theorem:线性方程组.齐次线性方程组的解向量个数} 可知
\((\a\b^T)\x=\vb0\)的解空间的维数为\(n-1\),
这就是说\(0\)是矩阵\(\a\b^T\)的\(n-1\)重特征值.
现在我们来求\(\a\b^T\)属于\(0\)的特征向量,
解方程组\((\a\b^T)\x=\vb0\),
左乘\(\a^T\)得\(\a^T\a\b^T\x=\vb0\),
由于\(\a^T\a\)是正实数,
消去便得\(\b^T\x=\vb0\),
这就是说\(\a\b^T\)属于\(0\)的特征向量是\(\b^T\x=\vb0\)的基础解系.
\end{proof}
\end{example}

\begin{example}
设\(\A \in M_{m \times n}(P),
\B \in M_{n \times m}(P)\),
且\(m \geq n\).
求证:\[
	\abs{\l\E_m-\A\B} = \l^{m-n} \abs{\l\E_n-\B\A}.
\]
\begin{proof}
当\(\l\neq0\)时,考虑下列分块矩阵:\[
	\begin{bmatrix}
		\l\E_m & \A \\
		\B & \E_n
	\end{bmatrix}.
\]
因为\(\l\E_m,\E_n\)都是可逆矩阵,
故由\hyperref[theorem:逆矩阵.行列式第一降阶定理]{降阶公式}可得\[
	\abs{\E_n} \cdot \abs{\l\E_m - \A (\E_n)^{-1} \B}
	= \abs{\l\E_m} \cdot \abs{\E_n - \B (\l\E_m)^{-1} \A},
\]
即有\(\abs{\l\E_m-\A\B} = \l^{m-n} \abs{\l\E_n-\B\A}\)成立.

当\(\lambda=0\)时,若\(m>n\),则\[
	\rank(\A\B) \leq \min\{\rank\A,\rank\B\} \leq \min\{m,n\} < m,
\]
故\(\abs{-\A\B}=0\),结论也成立;
若\(m = n\),则由\cref{theorem:行列式.矩阵乘积的行列式} 可知结论也成立.

事实上,\(\lambda=0\)的情形也可通过摄动法由\(\l\neq0\)的情形来得到.
\end{proof}
\end{example}
上例还有其他证法:
你可以将\(\A\)化为等价标准型来证明,
或先证\(\A\)非异的情形,再用摄动法进行讨论.

\subsection{求解特征值和特征向量的一般程序}
求解\(n\)阶矩阵\(\A = (a_{ij})_n\)的特征值和特征向量的一般程序:
\begin{enumerate}
	\item 计算特征多项式\(\abs{\lambda\E-\A}\);
	\item 求出\(\abs{\lambda\E-\A}=0\)的全部根,
	得\(\A\)的全部特征值\(\AutoTuple{\lambda}{n}\);
	\item 对于每个不同的特征值\(\L{j}\),
	求出齐次线性方程组\((\L{j} \E - \A)\x = \z\)的一个基础解系\(\AutoTuple{\x}{t}\),
	则\(\A\)的属于\(\L{j}\)的全部特征向量为\(k_1 \x_1 + k_2 \x_2 + \dotsb + k_t \X t\)
	(其中\(\AutoTuple{k}{t}\)是不全为零的任意常数).
\end{enumerate}

注1:根据高斯代数基本定理,任何\(n\ (n \geq 1)\)次多项式至少有一个复数根.
由此推出,任何\(n\ (n>0)\)次多项式有且仅由\(n\)个复根,其中规定\(m\)重根算\(m\)个根.

注2:\(n\)阶矩阵\(\A\)恰有\(n\)个特征值,但它不一定有\(n\)个线性无关的特征向量.

注3:如果\(\A\)是\(n\)阶实矩阵,则它有\(n\)个复特征值,
其中实特征值个数\(m\)满足\(0 \leq m \leq n\).

\begin{example}
设\(\A = \begin{bmatrix} 2 & -1 & 2 \\ 5 & -3 & 3 \\ -1 & 0 & -2 \end{bmatrix}\),
求\(\A\)的特征值与对应的特征向量.
\begin{solution}
\(\A\)的特征多项式\[
	\abs{\lambda\E-\A}
	= \begin{vmatrix} \l-2 & 1 & -2 \\ -5 & \l+3 & -3 \\ 1 & 0 & \l+2 \end{vmatrix}
	= (\l+1)^3,
\]
解\(\abs{\lambda\E-\A}=0\)得\(\lambda=-1\)(三重).

当\(\lambda=-1\)时,解方程组\((-\E-\A)\x=\z\),\[
	-\E-\A = \begin{bmatrix} -3 & 1 & -2 \\ -5 & 2 & -3 \\ 1 & 0 & 1 \end{bmatrix}
	\to \begin{bmatrix} 1 & 0 & 1 \\ 5 & 2 & -3 \\ -3 & 1 & -2 \end{bmatrix}
	\to \begin{bmatrix} 1 & 0 & 1 \\ 0 & 1 & 1 \\ 0 & 0 & 0 \end{bmatrix},
\]
\(\rank(-\E-\A)=2\),
令\(x_3=1\),
得基础解系\[
	\x_1=\begin{bmatrix} -1 \\ -1 \\ 1 \end{bmatrix},
\]
属于\(-1\)的全部特征向量为\(k\x_1\)(\(k\)为非零的任意常数).
\end{solution}
\end{example}

\begin{example}
设\(\A = \begin{bmatrix} -1 & 0 & 0 \\ 8 & 2 & 4 \\ 8 & 3 & 3 \end{bmatrix}\),
求\(\A\)的特征值与对应的特征向量.
\begin{solution}
\(\A\)的特征多项式\[
	\a = \begin{vmatrix}
		\l+1 & 0 & 0 \\
		-8 & \l-2 & -4 \\
		-8 & -3 & \l-3
	\end{vmatrix}
	= (\l+1)(\l^2-5\l-6)
	= (\l+1)^2(\l-6).
\]
令\(\a = 0\)可得\(\A\)的特征值为\(\L{1}=-1\)(二重),\(\L{2}=6\).

当\(\lambda=-1\)时,解方程组\((-\E-\A)\x=\z\),\[
	-\E-\A
	= \begin{bmatrix} 0 & 0 & 0 \\ -8 & -3 & -4 \\ -8 & -3 & -4 \end{bmatrix}
	\to \begin{bmatrix} -8 & -3 & -4 \\ 0 & 0 & 0 \\ 0 & 0 & 0 \end{bmatrix}.
\]
分别令\(\left\{ \begin{array}{l} x_2=8 \\ x_3=0 \end{array} \right.\)
和\(\left\{ \begin{array}{l} x_2=0 \\ x_3=2 \end{array} \right.\),
得基础解系\[
	\x_1 = \begin{bmatrix} -3 \\ 8 \\ 0 \end{bmatrix},
	\quad
	\x_2 = \begin{bmatrix} -1 \\ 0 \\ 2 \end{bmatrix},
\]
属于\(-1\)的全部特征向量为\(k_1\x_1+k_2\x_2\)(\(k_1,k_2\)为不全为零的任意常数);

当\(\lambda=6\)时,解方程组\((6\E-\A)\x=\z\),\[
	6\E-\A = \begin{bmatrix} 7 & 0 & 0 \\ -8 & 4 & -4 \\ -8 & -3 & 3 \end{bmatrix} \to \begin{bmatrix} 1 & 0 & 0 \\ 0 & 1 & -1 \\ 0 & 0 & 0 \end{bmatrix},
\]
令\(x_3=1\)得\(x_1=0\),\(x_2=1\),
基础解系\[
	\x_3 = \begin{bmatrix} 0 \\ 1 \\ 1 \end{bmatrix},
\]
属于\(6\)的全部特征向量为\(k_3\x_3\)(\(k_3\)为任意常数).
\end{solution}
\end{example}

\begin{example}
设矩阵\(\A = (a_{ij})_n \in \mathbb{C}^n\),
但\(a_{ij} \in \mathbb{R}\ (i,j=1,2,\dotsc,n)\).
证明:如果\(\lambda_0\in\mathbb{C}\)是\(\A\)的一个特征值,
\(\x_0\)是\(\A\)属于\(\lambda_0\)的一个特征向量,
那么\(\complexconjugate{\lambda_0}\)也是\(\A\)的一个特征值,
且\(\complexconjugate{\x_0}\)是\(\A\)属于\(\complexconjugate{\lambda_0}\)的一个特征向量.
\begin{proof}
在\(\A\x_0=\lambda_0\x_0\)两边取共轭得
\(\complexconjugate{\A}\complexconjugate{\x_0}
=\complexconjugate{\lambda_0}\complexconjugate{\x_0}\).
又因为\(\A=\complexconjugate{\A}\),
因此\(\A\complexconjugate{\x_0}=\complexconjugate{\lambda_0}\complexconjugate{\x_0}\).
这就表明\(\complexconjugate{\lambda_0}\)也是\(\A\)的一个特征值,
\(\complexconjugate{\x_0}\)是\(\A\)的属于\(\complexconjugate{\lambda_0}\)的一个特征向量.
\end{proof}
\end{example}

\begin{example}
求复数域上矩阵\[
	\A = \begin{bmatrix}
		4 & 7 & -3 \\
		-2 & -4 & 2 \\
		-4 & -10 & 4
	\end{bmatrix}
\]的全部特征值和特征向量.
\begin{solution}
\(\A\)的特征多项式为\[
	\abs{\lambda\E-\A}
	= \begin{vmatrix}
		\l-4 & -7 & 3 \\
		2 & \l+4 & -2 \\
		4 & 10 & \l-4
	\end{vmatrix}
	= \l^3 - 4\l^2 + 6\l - 4
	= (\l-2)(\l^2-2\l+2).
\]
令\(\abs{\lambda\E-\A}=0\)解得\(\lambda=2,1\pm\iu\).

当\(\lambda=2\)时,解方程组\((2\E-\A)\x=\z\),\[
	2\E-\A = \begin{bmatrix}
		-2 & -7 & 3 \\
		2 & 6 & -2 \\
		4 & 10 & -2
	\end{bmatrix} \to \begin{bmatrix}
		2 & 4 & 0 \\
		0 & 1 & -1 \\
		0 & 0 & 0
	\end{bmatrix}.
\]
令\(x_2=x_3=1\)得\(x_1=-2\),基础解系为\[
	\x_1 = (-2,1,1)^T,
\]
属于\(2\)的全部特征向量为\(k_1\x_1\ (k_1\in\mathbb{C}-\{0\})\).

当\(\lambda=1+\iu\)时,解方程组\([(1+\iu)\E-\A]\x=\z\),\[
	(1+\iu)\E-\A = \begin{bmatrix}
		-3+\iu & -7 & 3 \\
		2 & 5+\iu & -2 \\
		4 & 10 & -3+\iu
	\end{bmatrix}
	\to \def\arraystretch{1.5}\begin{bmatrix}
		1 & 0 & \frac{1}{2}-\iu \\
		0 & 1 & -\frac{1}{2}+\frac{1}{2}\iu \\
		0 & 0 & 0
	\end{bmatrix}.
\]
令\(x_3=-2\)得\(x_1=1-2\iu,x_2=-1+\iu\),
基础解系为\[
	\x_2 = (1-2\iu,-1+\iu,-2)^T,
\]
属于\(1+\iu\)的全部特征向量为\(k_2\x_2\ (k_2\in\mathbb{C}-\{0\})\).

当\(\lambda=1-\iu\)时,
\(\x_2\)也是它的一个特征向量,
那么属于\(1-\iu\)的全部特征向量为\(k_3\x_2\ (k_3\in\mathbb{C}-\{0\})\).
\end{solution}
\end{example}
