\section{实反对称矩阵的对角化}
\begin{theorem}
%@see: 《线性代数》(张慎语、周厚隆) P113 习题5.3 6.
实反对称矩阵的特征值为零或纯虚数.
\begin{proof}
设\(\A \in M_n(\mathbb{R})\)满足\(\A^T=-\A\).
又设\(\mathbb{C} \ni \L0 = a_0 + \iu b_0\ (a_0,b_0 \in \mathbb{R})\)是\(\A\)的任意一个特征值,
\(\mathbb{C}^{n \times 1} \ni \X0=(\AutoTuple{c}{n})^T \neq \z\)是
\(\A\)属于特征值\(\L0\)的特征向量,
即\begin{gather}
\A\X0 = \L0\X0, \tag1
\end{gather}
在(1)式两端左乘\(\overline{\X0}^T\),
得\begin{gather}
	\overline{\X0}^T \A \X0
	= \L0\ \overline{\X0}^T \X0, \tag2
\end{gather}
取共轭转置,得\begin{gather}
\overline{\X0}^T \A^T \X0
= \overline{\L0}\ \overline{\X0}^T \X0, \tag3
\end{gather}
由于\(\A\)是实反对称矩阵,即\(\A^T = -\A\),所以\begin{gather}
	\overline{\X0}^T \A \X0
	= -\overline{\L0}\ \overline{\X0}^T \X0, \tag4
\end{gather}
其中,\(\overline{\X0}^T \X0
= \overline{c_1}c_1 + \overline{c_2}c_2 + \dotsb + \overline{c_n}{c_n} > 0\).
由(2)式与(4)式,得\[
	\L0\ \overline{\X0}^T \X0
	= -\overline{\L0}\ \overline{\X0}^T \X0,
\]\[
	\L0 = -\overline{\L0},
\]\[
	a_0 + \iu b_0 = -(a_0 - \iu b_0) = -a_0 + \iu b_0,
\]\[
	\Re \L0 = a_0 = 0.
\]
也就是说\(\L0\)要么为零要么为纯虚数.
\end{proof}
\end{theorem}
