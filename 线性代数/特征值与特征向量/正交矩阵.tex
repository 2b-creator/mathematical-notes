\section{正交矩阵}
在平面上取一个直角坐标系\(Oxy\),
设向量\(\a,\b\)的坐标分别是\((a_1,a_2),(b_1,b_2)\).
如果\(\a,\b\)都是单位向量,并且互相垂直,
那么它们的坐标满足:\[
	\begin{split}
		a_1^2+a_2^2=1, \qquad
		a_1b_1+a_2b_2=0, \\
		b_1a_1+b_2a_2=0, \qquad
		b_1^2+b_2^2=1,
	\end{split}
\]
这组等式可以写成一个矩阵等式:\[
	\begin{bmatrix}
		a_1 & a_2 \\
		b_1 & b_2
	\end{bmatrix}
	\begin{bmatrix}
		a_1 & b_1 \\
		a_2 & b_2
	\end{bmatrix}
	= \begin{bmatrix}
		1 & 0 \\
		0 & 1
	\end{bmatrix}.
\]
如果记\(\A=(\a^T,\b^T)\),
那么上式又可写为\[
	\A^T\A=\E.
\]
根据\(\a,\b\)的几何意义,
我们很自然地把矩阵\(\A\)称为“正交矩阵”.

这一节我们来研究正交矩阵的性质,尤其是它的行(列)向量组的特性.

\begin{definition}
在欧几里得空间中,如果
\begin{enumerate}
	\item 向量组\(A=\{\AutoTuple{\a}{m}\}\)不含零向量,即\(\z \notin A\);
	\item \(A\)中向量两两正交,即\(\a_i \cdot \a_j = 0\ (i \neq j)\),
\end{enumerate}
则称\(A\)为一个\DefineConcept{正交向量组},简称\DefineConcept{正交组}.
由单位向量构成的正交组叫做\DefineConcept{规范正交组}或\DefineConcept{标准正交组}.
称含有\(n\)个向量的规范正交组
\[
	\AutoTuple{\e}{n}
\]
为\(\mathbb{R}^n\)的一个\DefineConcept{规范正交基}%
或\DefineConcept{标准正交基}(orthonormal basis).
\end{definition}

\begin{definition}\label{definition:正交矩阵.正交矩阵的定义}
%@see: 《高等代数(第三版 上册)》(丘维声) P145. 定义1
设\(\Q \in M_n(\mathbb{R})\).
若\(\Q\)满足\begin{equation}\label{equation:正交矩阵.正交矩阵的定义式}
	\Q^T\Q = \E,
\end{equation}
则称“\(\Q\)是\DefineConcept{正交矩阵}”.
\end{definition}

\begin{property}
若\(\A,\B\)都是\(n\)阶正交矩阵,
则\begin{enumerate}
	\item \(\A\)的行列式\(\abs{\A}\)要么等于\(1\),要么等于\(-1\),即
	\begin{equation}\label{equation:正交矩阵.正交矩阵的行列式}
		\abs{\A}\in\Set{-1,1}.
	\end{equation}

	\item \(\A\)可逆.

	\item \(A\)的转置\(\A^T\)以及它的逆\(\A^{-1}\)满足
	\begin{equation}\label{equation:正交矩阵.正交矩阵的转置等于正交矩阵的逆}
		\A^{-1}=\A^T.
	\end{equation}

	\item \(\A\B\)也是正交矩阵.
\end{enumerate}
\begin{proof}
在正交矩阵的定义式 \labelcref{equation:正交矩阵.正交矩阵的定义式} 等号两端分别取行列式,
利用\cref{theorem:行列式.矩阵乘积的行列式,theorem:行列式.性质1} 得\[
	\abs{\A}^2
	=\abs{\A^T}\abs{\A\vphantom{^T}}
	=\abs{\A^T\A}
	=\abs{\E}
	=1,
\]
于是\(\abs{\A}\neq0\),
\(\A\)是非奇异矩阵.
根据\cref{theorem:逆矩阵.矩阵可逆的充分必要条件1} 我们知道,任一正交矩阵可逆.
再根据\cref{definition:可逆矩阵.可逆矩阵的定义} 有\[
	\A^{-1}\A=\E,
\]
将上式与\cref{equation:正交矩阵.正交矩阵的定义式} 比较,
可知\(\A^{-1}=\A^T\).
那么由\(\A\A^{-1}=\E\)便得\[
	\A\A^T=\E.
\]
利用结合律,便得
\((\A\B)(\A\B)^T
= \A(\B\B^T)\A^T
= \A\E\A^T
= \E\).
\end{proof}
\end{property}

\begin{example}
由于单位矩阵\(\E\)满足\[
	\E^T=\E, \qquad
	\E^T \E = \E \E^T = \E,
\]
因此\(\E\)也是正交矩阵.
\end{example}

\begin{proposition}
正交矩阵\(\Q\)的伴随矩阵\(\Q^*\)满足\[
	\Q^*
	= \left\{ \begin{array}{rc}
		\Q^T, & \abs{\Q}>0, \\
		-\Q^T, & \abs{\Q}<0.
	\end{array} \right.
\]
\begin{proof}
由\cref{theorem:逆矩阵.逆矩阵的唯一性}
可知\(\Q^* = \abs{\Q} \Q^{-1}\).
再由\cref{equation:正交矩阵.正交矩阵的转置等于正交矩阵的逆}
可知\(\Q^* = \abs{\Q} \Q^T\).
最后由\cref{equation:正交矩阵.正交矩阵的行列式}
就有\(\Q^* = \pm\Q^T\).
\end{proof}
\end{proposition}

\begin{example}
设\(\Q=(\AutoTuple{\a}{n})\)是\(n\)阶实矩阵,
则\(\Q\)是正交矩阵的充分必要条件是\(\AutoTuple{\a}{n}\)是\(\mathbb{R}^{n \times 1}\)的规范正交基.
\begin{proof}
在\(\Q\)是\(n\)阶实矩阵的前提下,\begin{align*}
	&\text{\(\Q\)是正交矩阵}
	\iff \Q^T\Q = \Q\Q^T = \E \\
	&\iff \E = \begin{bmatrix}
		\a_1^T \\ \a_2^T \\ \vdots \\ \a_n^T
	\end{bmatrix} (\AutoTuple{\a}{n})
	= \begin{bmatrix}
		\a_1^T \a_1 & \a_1^T \a_2 & \dots & \a_1^T \a_n \\
		\a_2^T \a_1 & \a_2^T \a_2 & \dots & \a_2^T \a_n \\
		\vdots & \vdots & & \vdots \\
		\a_n^T \a_1 & \a_n^T \a_2 & \dots & \a_n^T \a_n
	\end{bmatrix} \\
	&\iff \a_i^T \a_j = (\a_i,\a_j)
	= \left\{ \begin{array}{ll}
		1, & i=j, \\
		0, & i \neq j,
	\end{array} \right. i,j=1,2,\dotsc,n \\
	&\iff \text{\(\AutoTuple{\a}{n}\)是规范正交基}.
	\qedhere
\end{align*}
\end{proof}
\end{example}

可以看出,正交矩阵是由一系列初等矩阵\(\P(i,j)\)的乘积.

\begin{example}
设\(\A\)是正交矩阵.
证明:\(\A\)的特征值是模为1的复数.
\begin{proof}
设\(\L0\)是\(\A\)的任意一个特征值,
\(\X0=(\AutoTuple{c}{n})^T\neq\z\)是\(\A\)属于特征值\(\L0\)的特征向量,
即\begin{gather}
	\A\X0=\L0\X0,
	\tag1
\end{gather}
取共轭转置,得\begin{gather}
	\overline{\X0}^T \overline{\A}^T = \overline{\L0} \overline{\X0}^T,
	\tag2
\end{gather}
将(1)式两端分别右乘到(2)式两端,
得\begin{gather}
	\overline{\X0}^T \overline{\A}^T \A\X0 = \overline{\L0} \overline{\X0}^T \L0\X0,
	\tag3
\end{gather}
因为\(\A\)是正交矩阵,
\(\A\)是实矩阵,
\(\overline{\A}^T = \A^T\),
且\(\A^T\A=\A\A^T=\E\),
所以\[
	\overline{\X0}^T \X0 = \overline{\L0} \L0 \overline{\X0}^T \X0,
\]\[
	(\overline{\L0} \L0 - 1) \overline{\X0}^T \X0 = 0,
\]
又因为\(\overline{\X0}^T \X0
= \overline{c_1}c_1 + \overline{c_2}c_2 + \dotsb + \overline{c_n}{c_n} > 0\),
所以\[
	\overline{\L0} \L0 = 1,
\]
即\(\A\)的特征值\(\L0\)是模为1的复数.
\end{proof}
\end{example}

\begin{example}
%@see: 《高等代数(第三版 上册)》(丘维声) P152. 习题4.6 4.
设\(\A\)是实数域上的\(n\)阶矩阵.
证明:如果\(\A\)具有\begin{enumerate}
	\item \(\A\)是正交矩阵,
	\item \(\A\)是对称矩阵,
	\item \(\A\)是对合矩阵,
\end{enumerate}
这三个性质中的任意两个性质,
则必有第三个性质.
%TODO
\end{example}

\begin{example}
%@see: 《高等代数(第三版 上册)》(丘维声) P152. 习题4.6 5.
证明:如果正交矩阵\(\A\)是上三角矩阵,
则\(\A\)一定是对角矩阵,
并且其主对角元是\(\pm1\).
%TODO
\end{example}
