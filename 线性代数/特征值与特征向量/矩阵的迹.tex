\section{矩阵的迹}
\begin{definition}
矩阵\(\A=(a_{ij})_{s \times n}\)
主对角线上元素之和称为\(\A\)的\DefineConcept{迹}(trace),
记作\(\tr\A\),即\[
	\tr\A \defeq \sum_{i=1}^m a_{ii},
\]
其中\(m = \min\{s,n\}\).
\end{definition}

\begin{property}\label{theorem:矩阵的迹.性质1}
已知矩阵\(\A,\B \in M_{s \times n}(K)\),则
\begin{enumerate}
	\item \(\tr(\A+\B) = \tr\A + \tr\B\);
	\item \(\tr(k \A) = k \tr\A\ (k \in P)\).
\end{enumerate}
\begin{proof}
设\(\A=(a_{ij})_{s \times n},
\B=(b_{ij})_{s \times n}\),
取\(m = \min\{s,n\}\),
那么\[
	\tr(\A+\B) = \sum_{i=1}^m (a_{ii}+b_{ii})
	= \sum_{i=1}^m a_{ii}
	+ \sum_{i=1}^m b_{ii}
	= \tr\A + \tr\B,
\]\[
	\tr(k \A) = \sum_{i=1}^m (k a_{ii})
	= k \sum_{i=1}^m a_{ii}
	= k \tr\A.
\]
\end{proof}
\end{property}
\cref{theorem:矩阵的迹.性质1} 说明:
矩阵的迹具有“线性性”.

\begin{property}\label{theorem:矩阵的迹.性质2}
已知矩阵\(\A \in M_{s \times n}(K)\),
则\(\tr\A = \tr(\A^T)\).
\end{property}

\begin{property}
设\(\A\)是可逆矩阵,
则\(\tr(\A^*) = \abs{\A} \cdot \tr(\A^{-1})\).
\end{property}

\begin{property}\label{theorem:矩阵的迹.矩阵乘积交换次序不变迹}
已知矩阵\(\A,\B \in M_n(K)\),
则\[
	\tr(\A\B) = \tr(\B\A).
\]
\end{property}

\begin{property}
已知矩阵\(\A \in M_{s \times n}(K)\),
则\(\tr(\A\A^T) = \tr(\A^T\A)\).
\end{property}

\begin{property}
已知矩阵\(\A,\B \in M_n(K)\),且\(\A,\B\)均为实对称矩阵,则\[
	\tr(\A\B)^2 \leq \tr(\A^2\B^2).
\]
\end{property}
