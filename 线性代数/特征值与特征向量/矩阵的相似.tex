\section{矩阵的相似}
\subsection{矩阵相似的概念}
\begin{definition}
设\(\A\)、\(\B\)是两个\(n\)阶矩阵.若存在可逆矩阵\(\P\),使得
\begin{equation}\label{equation:特征值与特征向量.相似矩阵的定义}
	\P^{-1}\A\P=\B
\end{equation}
则称\(\A\)与\(\B\)相似,记作\(\A\sim\B\).
\end{definition}

\subsection{矩阵相似的性质}
\begin{property}\label{theorem:特征值与特征向量.相似关系是等价关系}
矩阵之间的相似关系,是矩阵集合上的等价关系,因为它满足:
\begin{enumerate}
	\item 反身性,即对任意矩阵\(\A\),都有\(\A\sim\A\);
	\item 对称性,即若\(\A\sim\B\),则有\(\B\sim\A\);
	\item 传递性,即若\(\A\sim\B\)且\(\B\sim\C\),则\(\A\sim\C\).
\end{enumerate}
\begin{proof}
在\cref{equation:特征值与特征向量.相似矩阵的定义} 中,
令\(\A=\B\)、\(\P=\E\),得\(\E\A\E=\A\),
即有相似矩阵的反身性成立.

再在\cref{equation:特征值与特征向量.相似矩阵的定义} 中取\(\Q=\P^{-1}\),
得\(\A = \Q^{-1}(\P^{-1}\A\P)\Q = \Q^{-1}\B\Q\),即有相似矩阵的对称性成立.

设\(\P_1^{-1}\A\P_1=\B,
\P_2^{-1}\B\P_2=\C\),
于是\(\P_2^{-1}(\P_1^{-1}\A\P_1)\P_2=\C\).
取\(\Q=\P_1\P_2\),\(\Q\)是可逆矩阵,且\(\Q^{-1}\A\Q=\C\),所以\(\A\sim\C\),
即有相似矩阵的传递性成立.
\end{proof}
\end{property}

以下性质是两个矩阵相似的必要条件.
\begin{property}\label{theorem:特征值与特征向量.矩阵相似的必要条件1}
若\(\A\sim\B\),则\(\abs{\A}=\abs{\B}\).
\begin{proof}
因为\(\A\sim\B\),所以存在可逆矩阵\(\P\),
使得\(\P^{-1}\A\P=\B\),两端取行列式,得\[
	\abs{\B} = \abs{\P^{-1}\A\P}
	= \abs{\P^{-1}}\abs{\A}\abs{\P}
	= \abs{\P}^{-1}\abs{\A}\abs{\P}
	= \abs{\A}.
	\qedhere
\]
\end{proof}
\end{property}
我们还可以进一步推得如下结论:
若\(\A\sim\B\),则\(\A\)、\(\B\)同为可逆或不可逆.

\begin{property}\label{theorem:特征值与特征向量.矩阵相似的必要条件2}
若\(\A\sim\B\),则\(\A^m \sim \B^m\ (m\in\mathbb{N})\).
\begin{proof}
因为\(\A\sim\B\),所以存在可逆矩阵\(\P\),使得\(\P^{-1}\A\P=\B\),于是\[
	(\B)^m = (\P^{-1}\A\P)^m
	= (\P^{-1}\A\P)(\P^{-1}\A\P)\dotsb(\P^{-1}\A\P)
	= \P^{-1}\A^m\P.
	\qedhere
\]
\end{proof}
\end{property}
如果\(\A\)、\(\B\)可逆,那么上述结论可以扩展为\(\A^m\sim\B^m\ (m\in\mathbb{Z})\).

\begin{property}\label{theorem:特征值与特征向量.矩阵相似的必要条件3}
相似矩阵有相同的特征多项式,从而有相同的特征值.
\begin{proof}
因为\(\A\sim\B\),所以存在可逆矩阵\(\P\),使得\(\P^{-1}\A\P=\B\),于是\[
	\abs{\l\E-\B}
	=\abs{\P^{-1}(\l\E-\A)\P}
	=\abs{\P^{-1}}\abs{\l\E-\A}\abs{\P}
	=\abs{\l\E-\A}.
	\qedhere
\]
\end{proof}
\end{property}

应该注意到,\cref{theorem:特征值与特征向量.矩阵相似的必要条件3} 只是矩阵相似的必要不充分条件.
下面我们举出一条反例,不相似的两个矩阵有相同的特征值.
取\(\A=\begin{bmatrix} 2 & 1 \\ 0 & 2 \end{bmatrix}\)
和\(\B=\begin{bmatrix} 2 & 0 \\ 0 & 2 \end{bmatrix}\),
显然两者的特征值相同,即\(\l=2\)(二重).
但\(\A\)与\(\B\)不相似,
这是因为\(\B=2\E\)是数乘矩阵,可以和所有二阶矩阵交换,
那么对任意二阶可逆矩阵\(\P\)都有\(\P^{-1}\B\P=\B\P^{-1}\P=\B\),
即\(\B\)只能与自身相似,
\(\A\)与\(\B\)不相似.

\begin{property}\label{theorem:特征值与特征向量.矩阵相似的必要条件4}
相似矩阵有相同的迹,即\(\A\sim\B \implies \tr\A=\tr\B\).
\begin{proof}
设\[
	\A = (a_{ij})_n
	= \begin{bmatrix}
		a_{11} & \dots & a_{1i} & a_{1j}  & \dots & a_{1n} \\
		\vdots & & \vdots & \vdots & & \vdots \\
		a_{i1} & \dots & a_{ii} & a_{ij}  & \dots & a_{in} \\
		\vdots & & \vdots & \vdots  & & \vdots \\
		a_{j1} & \dots & a_{ji} & a_{jj}  & \dots & a_{jn} \\
		\vdots & & \vdots & \vdots  & & \vdots \\
		a_{n1} & \dots & a_{ni} & a_{nj}  & \dots & a_{nn}
	\end{bmatrix}.
\]

我们首先考察在成对的初等变换\(\P^{-1}\)、\(\P\)的作用下,
任意矩阵\(\A\)的迹\(\tr\A\)与\(\P^{-1} \A \P\)的迹\(\tr(\P^{-1} \A \P)\)相比,
会如何变化:
\begin{enumerate}
	\item 令\(\P = \P(i,j)\),有\(\P^{-1} = \P\),
	那么\(\P^{-1} \A \P\)相当于
	“首先交换\(\A\)的\(i\)、\(j\)两行,然后交换所得矩阵的\(i\)、\(j\)两列”,
	即\[
		\P^{-1} \A \P
		= \begin{bmatrix}
			a_{11} & \dots & a_{1j} & a_{1i}  & \dots & a_{1n} \\
			\vdots & & \vdots & \vdots & & \vdots \\
			a_{j1} & \dots & a_{jj} & a_{ji}  & \dots & a_{jn} \\
			\vdots & & \vdots & \vdots  & & \vdots \\
			a_{i1} & \dots & a_{ij} & a_{ii}  & \dots & a_{in} \\
			\vdots & & \vdots & \vdots  & & \vdots \\
			a_{n1} & \dots & a_{nj} & a_{ni}  & \dots & a_{nn}
		\end{bmatrix}.
	\]
	可以看到经过变换\(\P(i,j)\)前后两个矩阵的主对角线的元素之和不变.

	\item 令\(\P = \P(i(c))\ (c\neq0)\),
	有\(\P^{-1} = \P(i(c^{-1}))\),
	那么\(\P^{-1} \A \P\)相当于
	“首先用\(1/c\)乘以\(\A\)的第\(i\)行,再用\(c\)乘以\(\A\)的第\(i\)列”,
	即\[
		\P^{-1} \A \P
		= \begin{bmatrix}
			a_{11} & \dots & c a_{1i} & a_{1j}  & \dots & a_{1n} \\
			\vdots & & \vdots & \vdots & & \vdots \\
			\frac{1}{c} a_{i1} & \dots & \left( c \cdot \frac{1}{c} \right) a_{ii} & \frac{1}{c} a_{ij}  & \dots & \frac{1}{c} a_{in} \\
			\vdots & & \vdots & \vdots  & & \vdots \\
			a_{j1} & \dots & c a_{ji} & a_{jj}  & \dots & a_{jn} \\
			\vdots & & \vdots & \vdots  & & \vdots \\
			a_{n1} & \dots & c a_{ni} & a_{nj}  & \dots & a_{nn}
		\end{bmatrix}.
	\]
	可以看到经过变换\(\P(i(c))\)前后两个矩阵的主对角线的元素之和也不变.

	\item 令\(\P = \P(i,j(k))\),
	有\(\P^{-1} = \P(i,j(-k))\),
	那么\(\P^{-1} \A \P\)相当于
	“首先将\(\A\)的第\(j\)行的\((-k)\)倍加到第\(i\)行,
	再将所得矩阵的第\(i\)列的\(k\)倍加到第\(j\)列”,
	即\[
		\P^{-1} \A \P
		= \scalebox{.8}{\(\begin{bmatrix}
			a_{11} & \dots & a_{1i} & a_{1j} + k a_{1i}  & \dots & a_{1n} \\
			\vdots & & \vdots & \vdots & & \vdots \\
			a_{i1} - k a_{j1} & \dots & a_{ii} - k a_{ji} & a_{ij} - k a_{jj} + k(a_{ii} - k a_{ji})  & \dots & a_{in} - k a_{jn} \\
			\vdots & & \vdots & \vdots  & & \vdots \\
			a_{j1} & \dots & a_{ji} & a_{jj} + k a_{ji}  & \dots & a_{jn} \\
			\vdots & & \vdots & \vdots  & & \vdots \\
			a_{n1} & \dots & a_{ni} & a_{nj} + k a_{ni}  & \dots & a_{nn}
		\end{bmatrix}\)}.
	\]
	可以看到经过变换\(\P(i,j(k))\)前后两个矩阵的主对角线的元素之和还是不变.
\end{enumerate}
综上所述,成对的初等变换不改变矩阵的迹.
由\cref{theorem:逆矩阵.可逆矩阵与初等矩阵的关系} 可知,
任意可逆矩阵都可以分解为若干个初等矩阵的乘积,
那么根据上述讨论结果,对于任意可逆矩阵\(\P\),总有\[
	\tr(\P^{-1}\A\P) = \tr\A.
	\qedhere
\]
\end{proof}
\end{property}

\begin{example}
设矩阵\(\A\)可逆,\(\A^T\)是\(\A\)的转置.
证明:\(\A\A^T \sim \A^T\A\).
\begin{proof}
取\(\P=\A^{-1}\),
那么\(\P\A=\A\P=\E\),\(\A^T = (\P\A)\A^T = \P(\A\A^T)\),\[
	\P(\A\A^T)\P^{-1} = \A^T\P^{-1} = \A^T\A,
\]
故\(\A\A^T \sim \A^T\A\).
\end{proof}
\end{example}

\begin{example}
已知矩阵\(\A = \begin{bmatrix}
	2 & 0 & 0 \\
	0 & 0 & 1 \\
	0 & 1 & x
\end{bmatrix}\)与\(\B = \begin{bmatrix}
	2 & 0 & 0 \\
	0 & y & 0 \\
	0 & 0 & -1
\end{bmatrix}\)相似.
求\(x\)与\(y\).
\begin{solution}
因为\(\A\sim\B\),所以,由\cref{theorem:特征值与特征向量.矩阵相似的必要条件1},
\[
	\begin{vmatrix}
		2 & 0 & 0 \\
		0 & 0 & 1 \\
		0 & 1 & x
	\end{vmatrix}
	= -2 = -2y =
	\begin{vmatrix}
		2 & 0 & 0 \\
		0 & y & 0 \\
		0 & 0 & -1
	\end{vmatrix}
	\implies y = 1;
\]
又由\cref{theorem:特征值与特征向量.矩阵相似的必要条件4},
\[
	\tr\A = 2+x
	= 1+y = \tr\B
	\implies
	x = 0.
\]
\end{solution}
\end{example}
