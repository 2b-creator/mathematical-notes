\section{实对称矩阵的对角化}
由上节讨论我们知道,\(n\)阶矩阵分成可对角化与不可对角化两类.
实际上,矩阵的相似概念与数域有关,矩阵能否对角化也与数域有关.
因为一般矩阵的特征值是复数,即使\(\A\)的元素都是实数,也可能没有实数特征值.
例如,\(\A = \begin{bmatrix} 0 & -1 \\ 1 & 0 \end{bmatrix}\)是一个二阶实矩阵,
由于\[
	\abs{\l\E-\A}
	= \begin{vmatrix}
		\l & 1 \\
		-1 & \l
	\end{vmatrix}
	= (\l+\iu)(\l-\iu),
\]
可以看出\(\A\)有两个特征值\(\pm\iu\),
其对应的特征向量\((\iu,1)^T\)和\((\iu,-1)^T\)是复向量,
因此\(\A\)在复数域上可对角化,
但不存在可逆实阵\(\P\)使得\(\P^{-1}\A\P\)为对角阵.

\begin{theorem}\label{theorem:特征值与特征向量.实对称矩阵1}
实对称矩阵的特征值都是实数.
\begin{proof}
设\(\A \in M_n(\mathbb{R})\)满足\(\A^T=\A\).
显然\(\A\)的特征多项式\(\abs{\l\E-\A}\)在复数范围内有\(n\)个根.
假设\(\L0\in\mathbb{C}\)是\(\A\)的任意一个特征值,
则存在\(n\)维复向量\(\X0=(\AutoTuple{c}{n})^T \neq \z\),使得
\begin{gather}
	\A\X0 = \L0 \X0, \tag1
\end{gather}
用\(\X0\)的共轭转置向量\(\overline{\X0}^T
=(\overline{c_1},\overline{c_2},\dotsc,\overline{c_n})\)
左乘(1)式两端,
得\begin{gather}
	\overline{\X0}^T \A \X0 = \l \overline{\X0}^T \X0, \tag2
\end{gather}
其中\(\A^T=\A=\overline{\A}\),
\(\overline{\X0}^T \X0
= \overline{c_1}c_1 + \overline{c_2}c_2 + \dotsb + \overline{c_n}{c_n} \in \mathbb{R}^+\).

又因为\(\overline{\X0}^T \A \X0 \in \mathbb{C}\)取转置不变,
且实对称矩阵满足\(\A^T = \A = \overline{\A}\),
所以\[
	\overline{\X0}^T \A \X0
	= (\overline{\X0}^T \A \X0)^T
	= \X0^T \A^T \overline{\X0}
	= \overline{\overline{\X0}^T} \overline{\A} \overline{\X0}
	= \overline{\overline{\X0}^T \A \X0},
\]
说明\(\overline{\X0}^T \A \X0 \in \mathbb{R}\),进而有\(\L0 \in \mathbb{R}\).
\end{proof}
\end{theorem}

\begin{theorem}\label{theorem:特征值与特征向量.实对称矩阵2}
实对称矩阵\(\A\)的不同特征值所对应的特征向量正交.
\begin{proof}
设\(\L{1}\neq\L{2}\)是\(\A\)的两个不同的特征值,
\(\X1\neq\z\),
\(\X2\neq\z\)分别是\(\A\)对应于\(\L{1}\)、\(\L{2}\)的特征向量,
则\(\X1\)、\(\X2\)都是实向量,
\begin{align*}
	\A\X1 &= \L{1}\X1, \tag1 \\
	\A\X2 &= \L{2}\X2. \tag2
\end{align*}
对(1)式左乘\(\X2^T\),得\begin{gather}
	\X2^T \A \X1 = \L{1} \X2^T \X1, \tag3
\end{gather}
对(2)式左乘\(\X1^T\),得\begin{gather}
	\X1^T \A \X2 = \L{2} \X1^T \X2, \tag4
\end{gather}
(3)式取转置,
得\(\L{1}(\X2^T \X1)^T = (\X2^T \A \X1)^T\),
又由\(\A=\A^T\),
得\begin{gather}
	\L{1} \X1^T \X2 = \X1^T \A^T \X2 = \X1^T \A \X2, \tag5
\end{gather}
(5)式减(4)式,得
\begin{gather}
	(\L{2}-\L{1})\X1^T\X2=0. \tag6
\end{gather}
因为\(\L{2} \neq \L{1}\),
所以\(\X1^T \X2 = 0\),
即\((\X1,\X2) = 0\),
也就是\(\X1\)与\(\X2\)正交.
\end{proof}
\end{theorem}

一般地,对于\(n\)阶实对称矩阵\(\A\),
属于\(\A\)的同一特征值的一组线性无关的特征向量不一定相互正交,
可用施密特正交化方法将其正交化,得到\(\A\)的属于该特征值的正交特征向量组.
由以上定理,\(\A\)的几个属于不同特征值的正交特征向量组仍构成正交组.
特别地,\(\A\)有\(n\)个正交的特征向量,\(\A\)相似于对角形矩阵.

\begin{theorem}\label{theorem:特征值与特征向量.实对称矩阵3}
若\(\A\)为\(n\)阶实对称矩阵,则一定存在正交矩阵\(\Q\),使得\(\V = \Q^{-1}\A\Q\)为对角形矩阵.
\begin{proof}
\def\M{\vb{M}}%
用数学归纳法.
当\(n=1\)时,矩阵\(\A\)是一个实数\(a_{11}\),定理成立.
假设当\(n=k-1\)时定理成立,下面证明当\(n=k\)时定理也成立.

由\cref{theorem:特征值与特征向量.实对称矩阵1},
\(\A\)的特征值全为实数.
假设\(\L1\)是\(\A\)的特征值,
并且相应地存在非零实向量\(\X1=(\AutoTuple{c}{n})^T\),
使得\[
	\A\X1=\L1\X1.
\]
不妨设\(c_1\neq0\),则\(n\)元向量组\[
	\X1,\X2=(0,1,\dotsc,0)^T,\dotsc,\X{n}=(0,0,\dotsc,1)^T
\]线性无关.
对向量组\(X=\{\AutoTuple{\x}{n}\}\)
用施密特正交规范化方法可得规范正交组\(Y=\{\AutoTuple{\y}{n}\}\),
则\(\P=(\AutoTuple{\y}{n})\)是正交矩阵,
其中\(\y_1=\abs{\X1}^{-1}\X1\)是\(\A\)的特征向量.

因为\(\mathbb{R}^n\)中任意\(n+1\)个向量线性相关,
故任意向量都可由\(Y\)线性表出,
即\begin{align*}
	\A\y_1 &= \L1\y_1 = \L1\y_1 + 0\y_2 + \dotsb + 0\y_n, \\
	\A\y_k &= b_{1k}\y_1 + b_{2k}\y_2 + \dotsb + b_{nk}\y_n \quad(k=2,3,\dotsc,n),
\end{align*}
由分块矩阵乘法,\[
	\A(\AutoTuple{\y}{n})
	= (\AutoTuple{\y}{n})
	\begin{bmatrix}
		\l_1 & b_{11} & \dots & b_{1n} \\
		0 & b_{22} & \dots & b_{2n} \\
		\vdots & \vdots & & \vdots \\
		0 & b_{n2} & \dots & b_{nn}
	\end{bmatrix},
\]
令\(\P=(\AutoTuple{\y}{n})\),
显然\(\P\)是正交阵,
而\(\P^{-1}\A\P=\P^T\A\P=\begin{bmatrix}
	\L1 & \a \\
	\z & \B
\end{bmatrix}\).

由\((\P^T\A\P)^T=\P^T\A^T\P=\P^T\A\P\)可知\(\begin{bmatrix}
	\L1 & \a \\
	\z & \B
\end{bmatrix}
= \begin{bmatrix}
	\L1 & \z \\
	\a^T & \B^T
\end{bmatrix}\),
于是\(\a=\z\),\(\B=\B^T\),也就是说\(\B\)是\(n-1\)阶实对称矩阵.
又由归纳假设,存在\(n-1\)阶正交阵\(\M\),使得\(\M^{-1}\B\M\)成为对角阵.

令\(\Q=\P\begin{bmatrix} 1 & \z \\ \z & \M \end{bmatrix}\),
因为\(\Q\Q^T=\Q^T\Q=\E\),所以\(\Q\)是正交矩阵,
而\begin{align*}
	\Q^{-1}\A\Q
	&=\begin{bmatrix}
		1 & \z \\
		\z & \M^{-1}
	\end{bmatrix}\P^{-1}\A\P\begin{bmatrix}
		1 & \z \\
		\z & \M
	\end{bmatrix}=\begin{bmatrix}
		1 & \z \\
		\z & \M^{-1}
	\end{bmatrix}\begin{bmatrix}
		\L1 & \z \\
		\z & \B
	\end{bmatrix}\begin{bmatrix}
		1 & \z \\
		\z & \M
	\end{bmatrix} \\
	&=\begin{bmatrix}
		\L1 & \z \\
		\z & \M^{-1}\B\M
	\end{bmatrix}
	=\diag(\AutoTuple{\lambda}{n}).
\end{align*}
由上可知当\(n=k\)时定理也成立.
\end{proof}
\end{theorem}

\cref{theorem:特征值与特征向量.实对称矩阵3} 表明,
实对称矩阵总可以相似对角化,
实对称矩阵总是\DefineConcept{正交相似}({orthogonally similar})于某个对角形矩阵.

\begin{corollary}
\(n\)阶实对称矩阵\(\A\)存在\(n\)个正交的单位特征向量.
\end{corollary}

\begingroup
\color{red}
对于实对称矩阵\(\A\),求正交矩阵\(\Q\),使得\(\Q^{-1}\A\Q\)为对角形矩阵的方法:
\begin{enumerate}
	\item 求出\(\A\)的全部不同的特征值\(\AutoTuple{\lambda}{m}\);
	\item 求出\((\L{i}\E-\A)\x=\z\)的基础解系,将其正交化,
	得到\(\A\)属于\(\L{i}\)的正交特征向量(\(i=1,2,\dotsc,m\)),
	共求出\(\A\)的\(n\)个正交特征向量;
	\item 将以上\(n\)个正交特征向量单位化,由所得向量作为列构成正交矩阵\(\Q\),则\[
		\Q^{-1}\A\Q = \Q^T \A \Q = \diag(\AutoTuple{\lambda}{n}).
	\]
\end{enumerate}
\endgroup

\begin{example}
设\(\A\)为\(n\)阶实对称矩阵,满足\(\A^2=\E\),证明:存在正交矩阵\(\Q\),使得\[
	\Q^{-1}\A\Q=\begin{bmatrix} \E_r \\ & -\E_{n-r} \end{bmatrix}.
\]
\begin{proof}
因为\(\A\)为\(n\)阶实对称矩阵,
则\(\A\)有\(n\)个实特征值,
\(\A\)有\(n\)个正交的单位特征向量,
适当调整它们的顺序,可以构成正交矩阵\(\Q\),
满足\begin{gather}
	\Q^{-1}\A\Q=\diag(\AutoTuple{\lambda}{n}), \tag1
\end{gather}
其中,\(\L{i}>0\ (i=1,2,\dotsc,r),
\L{i}\leq0\ (i=r+1,r+2,\dotsc,n)\).
对(1)式两端分别平方,又由\(\A^2=\E\),得\[
	\Q^{-1}\A^2\Q
	= \Q^{-1}\E\Q
	= \E
	= \diag(\L{1}^2,\L{2}^2,\dotsc,\L{n}^2),
\]
于是\(\L{i}^2=1\ (i=1,2,\dotsc,n)\),
进而有\[
	\L{i}= \begin{cases}
		1, & i=1,2,\dotsc,r, \\
		-1, & i=r+1,r+2,\dotsc,n.
	\end{cases}
	\qedhere
\]
\end{proof}
\end{example}

\begin{example}
设\(\A\)是4阶实对称矩阵,且\(\A^2+\A=\z\).
若\(\rank\A=3\),求\(\A\)的特征值以及与\(\A\)相似的对角阵.
\begin{solution}
\(\A\)是实对称矩阵,根据\cref{theorem:特征值与特征向量.实对称矩阵3},
\(\A\)一定可对角化,不妨设\(\A\x=\l\x\ (\x\neq0)\),
那么\(\A^2\x=\l^2\x\),\((\A^2+\A)\x=(\l^2+\l)\x\).
因为\(\A^2+\A=\z\),所以\((\l^2+\l)\x=\z\),\(\l^2+\l=0\),
解得\(\A\)的特征值为\(\l=0,-1\).
又因为\(\rank\A=3\),所以\(\A\)具有3个非零特征值,
因此与\(\A\)相似的对角阵为\(\diag(-1,-1,-1,0)\).
\end{solution}
\end{example}

\begin{example}
设\(\A\)、\(\B\)是两个\(n\)阶正交矩阵.证明:
\begin{enumerate}
	\item \(\A\B\)是正交矩阵;
	\item \(\A^{-1}\)是正交矩阵;
\end{enumerate}
\begin{proof}
因为\(\A\)、\(\B\)是两个\(n\)阶正交矩阵,所以\(\A\A^T = \E\),\(\B\B^T = \E\).
\begin{enumerate}
	\item \((\A\B)(\A\B)^T
	= (\A\B)(\B^T\A^T)
	= \E\).

	\item \(\A^{-1}(\A^{-1})^T
	= \A^{-1} (\A^T)^{-1}
	= \A^{-1} (\A^{-1})^{-1}
	= \A^{-1} \A
	= \E\).

	\qedhere
\end{enumerate}
\end{proof}
\end{example}

\begin{example}
设\(\A\)是特征值仅为1与0的\(n\)阶实对称矩阵,证明:\(\A^2=\A\).
\begin{proof}
\def\M{\begin{bmatrix} \E_r \\ & \z_{n-r} \end{bmatrix}}%
因为\(\A\)是实对称矩阵,所以存在正交矩阵\(\Q\)使得\[
	\Q^{-1}\A\Q = \M,
\]
从而有\[
	\A = \Q\M\Q^{-1},
\]
进而有\[
	\A^2 = \Q\M\Q^{-1}\Q\M\Q^{-1} = \Q\M\Q^{-1} = \A.
	\qedhere
\]
\end{proof}
\end{example}

\begin{example}
设\(\A\)为\(n\)阶实对称矩阵,满足\(\A^2=\z\),证明:\(\A=\z\).
\begin{proof}
因为\(\A\)是实对称矩阵,所以存在正交矩阵\(\Q\)使得\[
	\Q^{-1}\A\Q = \diag(\AutoTuple{\lambda}{n}) = \V,
\]
从而有\(\A = \Q\V\Q^{-1}\),\(\A^2 = (\Q\V\Q^{-1})^2 = \Q\V^2\Q^{-1} = \z\),那么\[
	\V^2 = \diag(\L{1}^2,\L{2}^2,\dotsc,\L{n}^2) = \Q^{-1}\z\Q = \z,
\]\[
	\L{1}=\L{2}=\dotsb=\L{n} = 0,
\]
所以\(\A=\z\).
\end{proof}
\end{example}
