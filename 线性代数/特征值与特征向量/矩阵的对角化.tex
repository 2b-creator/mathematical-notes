\section{矩阵的对角化}
\begin{theorem}[矩阵可对角化的充要条件]
\(n\)阶矩阵\(\A\)相似于对角阵的充要条件为\(\A\)有\(n\)个线性无关的特征向量.
\begin{proof}
必要性.
\(\A\)相似于对角阵,即存在可逆矩阵\(\P\),使\[
	\P^{-1}\A\P=\V
	=\begin{bmatrix}
		\L{1} \\ & \L{2} \\ & & \ddots \\ & & &\L{n}
	\end{bmatrix}
	=\diag(\L{1},\L{2},\dotsc,\L{n}),
\]
用\(\P\)左乘上式两端,得\[
	\A\P=\P\V.
\]
将\(\P\)按列分块,则\(\P=(\AutoTuple{\x}{n})\),
由于\(\P\)可逆,所以\(\AutoTuple{\x}{n}\)线性无关,有\[
	\A(\AutoTuple{\x}{n})=(\AutoTuple{\x}{n})\V,
\]\[
	(\A\X1,\A\X2,\dotsc,\A\X{n})=(\X1\V,\X2\V,\dotsc,\X{n}\V),
\]
于是\[
	A\X i=\L{i}\X i,
	\quad i=1,2,\dotsc,n,
\]
即\(\AutoTuple{\x}{n}\)是\(\A\)分别对应于\(\AutoTuple{\lambda}{n}\)的\(n\)个线性无关的特征向量.

同理可证充分性.
\end{proof}
\end{theorem}

由上述定理的证明可知:
{\color{red}当\(\P^{-1}\A\P=\V\)时,
\(\V\)的\(n\)个主对角元是\(\A\)的\(n\)个特征值;
可逆矩阵\(\P\)的\(n\)个列向量\(\AutoTuple{\x}{n}\)是
\(\A\)分别属于\(\L{1},\L{2},\dotsc,\L{n}\)的线性无关特征向量.}

\begin{theorem}
矩阵\(\A\)的属于不同特征值的特征向量线性无关.
\begin{proof}
设\(n\)阶矩阵\(\A\)的\(m\)个不同的特征值\(\L{1},\L{2},\dotsc,\L{m}\)
对应的特征向量分别为\(\AutoTuple{\x}{m}\).

由上述所有特征向量构成的向量组,
记作\(X_m=\{\AutoTuple{\x}{m}\}\).

当\(m=1\)时,
由于\(\X1 \neq 0\),
故向量组\(X_1=\{\X1\}\)线性无关.

当\(m>1\)时,
假设\(m-1\)个不同特征值对应的特征向量\(X_{m-1}=\{\AutoTuple{\x}{m-1}\}\)线性无关.
对于\(m\)个不同特征值对应的特征向量组\(X_m\),
令\begin{gather}
	k_1\X1+k_2\X2+\dotsb+k_m\X{m}=\z,
	\tag1
\end{gather}
由于\(\A\X{j}=\L{j}\X{j}\),
用\(\A\)左乘(1)式两端,
得\begin{gather}
	k_1\L{1}\X1+k_2\L{2}\X2+\dotsb+k_{m-1}\L{m-1}\X{m-1}+k_m\L{m}\X{m}=\z.
	\tag2
\end{gather}
再用\(\L{m}\)数乘(1)式两端,得\begin{gather}
	\L{m}k_1\X1+\L{m}k_2\X2+\dotsb+\L{m}k_{m-1}\X{m-1}+\L{m}k_m\X{m}=\z,
	\tag3
\end{gather}
(2)、(3)两式相减,得\begin{gather}
	(\L{1}-\L{m})k_1\X1+(\L{2}-\L{m})k_2\X2+\dotsb+(\L{m-1}-\L{m})k_{m-1}\X{m-1}=\z.
	\tag4
\end{gather}
根据归纳假设,向量组\(X_{m-1}\)线性无关,
则\((\L{i}-\L{m})k_i=0\)(\(i=1,2,\dotsc,m-1\)).
由于\(\L{i}\neq\L{m}\)(\(i=1,2,\dotsc,m-1\)),
所以\(k_1=k_2=\dotsb=k_{m-1}=0\),得\(k_m\X{m}=\z\),
但特征向量\(\X{m}\neq\z\),则\(k_m=0\),从而向量组\(X_m\)线性无关.
\end{proof}
\end{theorem}

\begin{corollary}[矩阵可对角化的充分条件]
若\(n\)阶矩阵\(\A\)有\(n\)个不同的特征值,则\(\A\)可对角化.
\end{corollary}

\begin{theorem}
设\(\L{1},\L{2},\dotsc,\L{m}\)是\(n\)阶矩阵\(\A\)的不同的特征值,
而\(\X{ij}\ (j=1,2,\dotsc)\)是\(\A\)属于\(\L{i}\ (i=1,2,\dotsc,m)\)的线性无关的特征向量,
即\[
	\A \X{ij} = \L{i} \X{ij},
	\quad i=1,2,\dotsc,m;j=1,2,\dotsc,
\]
则\[
	\X{11},\X{12},\dotsc,\X{21},\X{22},\dotsc,\X{m1},\X{m2},\dotsc
\]线性无关.
\end{theorem}

\begin{example}
设\[
	\A = \begin{bmatrix}
		1 & 0 & 0 \\
		-2 & 5 & -2 \\
		-2 & 4 & -1
	\end{bmatrix}.
\]
试问:\(\A\)能否对角化?
若能,则求出可逆矩阵\(\P\),使\(\P^{-1}\A\P\)为对角形矩阵.
\begin{solution}
\(\A\)的特征多项式为\[
	\abs{\l\E-\A} = \begin{bmatrix}
		\l-1 & 0 & 0 \\
		2 & \l-5 & 2 \\
		2 & -4 & \l+1
	\end{bmatrix}
	= (\l-1)^2 (\l-3),
\]
则\(\A\)的特征值为\(\L{1}=1\)(二重),\(\L{2}=3\).

当\(\L{1}=1\)时,解齐次线性方程组\((\E-\A)\x=\z\),\[
	\E-\A=\begin{bmatrix}
		0 & 0 & 0 \\
		2 & -4 & 2 \\
		2 & -4 & 2
	\end{bmatrix}
	\to \begin{bmatrix}
		1 & -2 & 1 \\
		0 & 0 & 0 \\
		0 & 0 & 0
	\end{bmatrix},
\]
基础解系为\(\X1 = \begin{bmatrix} 2 \\ 1 \\ 0 \end{bmatrix},
\X2 = \begin{bmatrix} -1 \\ 0 \\ 1 \end{bmatrix}\).

对于\(\L{2}=3\),解方程组\((3\E-\A)\x=\z\),\[
	3\E-\A=\begin{bmatrix}
		2 & 0 & 0 \\
		2 & -2 & 2 \\
		2 & -4 & 4
	\end{bmatrix} \to \begin{bmatrix}
		2 & 0 & 0 \\
		0 & -2 & 2 \\
		0 & 0 & 0
	\end{bmatrix},
\]
基础解系为\(\X3 = \begin{bmatrix} 0 \\ 1 \\ 1 \end{bmatrix}\).

特征向量\(\X1,\X2,\X3\)线性无关,所以\(\A\)可以对角化.
令\[
	\P = \begin{bmatrix} \X1 & \X2 & \X3 \end{bmatrix} = \begin{bmatrix}
		2 & -1 & 0 \\
		1 & 0 & 1 \\
		0 & 1 & 1
	\end{bmatrix},
	\quad\text{则有}\quad
	\P^{-1} \A \P = \begin{bmatrix} 1 \\ & 1 \\ && 3 \end{bmatrix}.
\]
\end{solution}
\end{example}

\begin{example}
设\(\A = \begin{bmatrix}
	2 & 0 & 0 \\
	0 & 2 & 0 \\
	0 & 1 & 2
\end{bmatrix}\),证明:\(\A\)不可对角化.
\begin{proof}
\(\A\)的特征多项式为\[
	\abs{\l\E-\A} = \begin{vmatrix}
		\l-2 & 0 & 0 \\
		0 & \l-2 & 0 \\
		0 & -1 & \l-2
	\end{vmatrix} = (\l-2)^2,
\]
令\(\abs{\l\E-\A} = 0\)解得特征值\(\L{1}=2\)(三重).
由于\(\rank(\L{1}\E-\A)=1\),
那么对应于唯一的特征值\(\L{1}=2\),
\(\A\)只有两个线性无关的特征向量,
因而不存在可逆矩阵\(\P\)使得\(\P^{-1}\A\P\)为对角形矩阵.
\end{proof}
\end{example}

从上述例子可以看出,当矩阵\(\A\)的某个特征值\(\L0\)为\(k\)重根式,
对应于\(\L0\)的线性无关的特征向量的个数可能等于\(k\),
也可能小于\(k\).这个规律对于一般的矩阵是成立的.

\begin{theorem}
设\(\L0\)为\(n\)阶矩阵\(\A\)的\(k\)重特征值,
则属于\(\L0\)的\(\A\)的线性无关的特征向量最多只有\(k\)个.
\end{theorem}

\begin{theorem}
\(n\)阶矩阵\(\A\)可对角化的充要条件是:
对于\(\A\)的每个\(k_i\)重特征值\(\L{i}\),
\(\A\)有\(k_i\)个线性无关的特征向量.
\end{theorem}

\begin{corollary}
\(n\)阶矩阵\(\A\)可对角化的充要条件是:
对于\(\A\)的每个\(k_i\)重特征值\(\L{i}\),
都有\(\rank(\L{i}\E-\A) = n-k_i\).
\end{corollary}

\begin{example}
设\(\A\)、\(\B\)是\(n\)阶矩阵,且\(\A\)可逆,证明:\(\A\B\)与\(\B\A\)相似.
\begin{proof}
因为\(\A\)可逆,则\[
	\B\A
	=\E\B\A
	=(\A^{-1}\A)\B\A
	=\A^{-1}(\A\B)\A,
\]
根据定义可得\(\A\B \sim \B\A\).
\end{proof}
\end{example}

\begin{example}
设\(\A\)为可逆矩阵且可对角化,证明:\(\A^{-1}\)也可对角化.
\begin{proof}
设存在可逆矩阵\(\P\)使得\begin{gather}
	\P^{-1}\A\P = \V,
	\tag1
\end{gather}
其中\(\V=\diag(\L{1},\L{2},\dotsc,\L{n})\),
\(n\)是矩阵\(\A\)的阶数,
\(\L{1},\L{2},\dotsc,\L{n}\)是矩阵\(\A\)的特征值.
显然有\(\abs{\V}
= \abs{\P^{-1}\A\P}
= \abs{\P^{-1}}\abs{\A}\abs{\P}
= (\abs{\P^{-1}}\abs{\P})\abs{\A}
= 1 \cdot \abs{\A}
= \abs{\A} \neq 0\),
即\(\V\)可逆.
在(1)式两端左乘\(\P\)得\(\P(\P^{-1}\A\P) = \P\V\)
即\begin{gather}
	\A\P = \P\V.
	\tag2
\end{gather}
在(2)式两端左乘\(\P^{-1}\A^{-1}\),
右乘\(\V^{-1}\)得\[
	(\P^{-1}\A^{-1})(\A\P)\V^{-1} = (\P^{-1}\A^{-1})(\P\V)\V^{-1},
\]
即\(\V^{-1} = \P^{-1}\A^{-1}\P\).
\end{proof}
\end{example}

\begin{example}
设\(m\)阶矩阵\(\A\)与\(n\)阶矩阵\(\B\)都可对角化,证明:\(m+n\)阶矩阵\[
	\begin{bmatrix} \A & \z \\ \z & \B \end{bmatrix}
\]可对角化.
\begin{proof}
设存在\(m\)阶可逆矩阵\(\P\)和\(n\)阶可逆矩阵\(\Q\)使得
\begin{align*}
	\P^{-1}\A\P &= \V_1 \\
	\Q^{-1}\B\Q &= \V_2
\end{align*}
则可构造矩阵使得\[
	\begin{bmatrix}
		\P^{-1} & \z \\
		\z & \Q^{-1}
	\end{bmatrix}
	\begin{bmatrix} \A & \z \\ \z & \B \end{bmatrix}
	\begin{bmatrix}
		\P & \z \\
		\z & \Q
	\end{bmatrix}
	= \begin{bmatrix}
		\V_1 & \z \\
		\z & \V_2
	\end{bmatrix}.
	\qedhere
\]
\end{proof}
\end{example}

\begin{example}
设\(\A\)是非零的幂零矩阵,
即\(\A\neq\z\),且存在自然数\(m\),使得\(\A^m=\z\).
证明:\(\A\)的特征值全为零,且\(\A\)不可以对角化.
\begin{proof}
设存在非零列向量\(\X0\),使得\(\A\X0=\L0\X0\)成立,则\[
\A^m\X0=\L0^m\X0.
\]
令\(\A^m=\z\),则解\(\A^m\X0=\z=\L0^m\X0\)得\(\L0=0\),可见\(\A\)的特征值全为零.

假设\(\A\)可以对角化,即存在可逆矩阵\(\P\)使得\[
	\P^{-1}\A\P = \diag(\L{1},\L{2},\dotsc,\L{n}) = \z.
\]
在等式两边同时左乘\(\P\),并右乘\(\P^{-1}\),得\[
	\A = \P(\P^{-1}\A\P)\P^{-1} = \P\z\P^{-1} = \z.
\]矛盾,故\(\A\)不可以对角化.
\end{proof}
\end{example}

\begin{example}
\def\J{\mat{J}_n}
形式为\[
	\J = \begin{bmatrix}
		\L0 & 0 & 0 & \dots & 0 & 0 \\
		1 & \L0 & 0 & \dots & 0 & 0 \\
		0 & 1 & \L0 & \dots & 0 & 0 \\
		\vdots & \vdots & \vdots & \ddots & \vdots & \vdots \\
		0 & 0 & 0 & \dots & \L0 & 0 \\
		0 & 0 & 0 & \dots & 1 & \L0
	\end{bmatrix}_n
\]的复数三角形阵称为\DefineConcept{若尔当块}.
证明:\(n>1\)阶若尔当块不可以对角化.
\begin{proof}
令\(\abs{\l\E-\J}=(\lambda-\L0)^n=0\),解得\(\l=\L0\)(\(n\)重),那么\[
	\L0\E-\J = \begin{bmatrix}
		0 \\
		-1 & 0 \\
		& -1 & 0 \\
		& & \ddots & \ddots \\
		& & & -1 & 0
	\end{bmatrix}_n,
\]
\(\rank(\L0\E-\J)=n-1 > 0\),
故当\(n>1\)时\(\J\)不可以对角化.
\end{proof}
\end{example}

\begin{definition}
由若干个若尔当块构成的准对角矩阵称为\DefineConcept{若尔当形矩阵}.
\end{definition}

\begin{theorem}
每个\(n\)阶复数矩阵不一定与对角阵相似,但必与一个若尔当形矩阵相似.
\end{theorem}
