\chapter{行列式}
\section{排列与逆序数}
\begin{definition}
由\(n\)个自然数\(1,2,\dotsc,n\)组成的一个有序数组称为一个\(n\)阶\textbf{排列}.

设\(\v{k}{n}\)是一个\(n\)阶排列,当\(1 \leqslant i<j \leqslant n\)时,如果\(k_i>k_j\),则称\(k_i,k_j\)构成一个\textbf{逆序};如果\(k_i<k_j\),则称\(k_i,k_j\)构成一个\textbf{顺序}.此排列中的逆序的总数叫它的\textbf{逆序数},记作\(\tau(\v{k}{n})\).

逆序数为偶数的排列叫\textbf{偶排列},逆序数为奇数的排列叫\textbf{奇排列}.
\end{definition}

\begin{property}
\(n\)阶排列共有\(n!\)个.
\end{property}

\begin{definition}
排列\(1,2,\dotsc,n\)由小到大按自然顺序排列,叫做\(n\)阶\textbf{自然排列}.
\end{definition}

\begin{property}
自然排列中没有逆序,即\(\tau(1,2,\dotsc,n)=0\).
\end{property}

\begin{theorem}
计算一个排列的逆序数时,排列中的逆序不能重复计算,也不能漏掉.可按公式\(\tau(\v{k}{n})=m_1+m_2+\dotsb+m_n\)计算,其中\(m_i\)为排列中排在\(k_i\)后面比它小的数的个数.
\end{theorem}

\begin{property}
\(\tau(n,n-1,\dotsc,1)=\frac{n(n-1)}{2}\).
\end{property}

\begin{definition}
把排列中的两个数\(i\),\(j\)的位置互换,其余数字不动,得到另一个排列,称这样的一个变换为\textbf{对换}.
\end{definition}

\begin{theorem}
排列经一次对换后奇偶性改变.
\end{theorem}

\begin{theorem}
任意一个\(n\)阶排列\(\v{k}{n}\)都可经一系列对换变成自然顺序排列,且对换的次数\(s\)与\(\tau(\v{k}{n})\)同奇偶.
\end{theorem}

\begin{theorem}
在全部\(n\)阶排列中,奇偶排列各占一半.
\end{theorem}

\begin{example}
试证:\(\tau(\v{i}{n})+\tau(i_n,i_{n-1},\dotsc,i_1)=\frac{n(n-1)}{2}\).
\end{example}

\section{行列式}
\subsection{行列式的概念}
\begin{definition}
设
\[
\A = \begin{bmatrix}
a_{11} & a_{12} & \dots & a_{1n} \\
a_{21} & a_{22} & \dots & a_{2n} \\
\vdots & \vdots & \ddots & \vdots \\
a_{n1} & a_{n2} & \dots & a_{nn}
\end{bmatrix}
\]
是数域\(P\)上的一个\(n\)阶方阵.
从矩阵\(\A\)中取出不同行又不同列的\(n\)个元素作乘积
\[
(-1)^{\tau(\v{j}{n})}
a_{1 j_1} a_{2 j_2} \dotsm a_{n j_n},
\]
构成一项;%
我们可以像这样构造\(n!\)项,%
并且称这\(n!\)项之和\[
\sum\limits_{\v{j}{n}}
(-1)^{\tau(\v{j}{n})}
a_{1 j_1} a_{2 j_2} \dotsm a_{n j_n}
\]为“矩阵\(\A\)的行列式(determinant)”,%
记作\[
\begin{vmatrix}
a_{11} & a_{12} & \dots & a_{1n} \\
a_{21} & a_{22} & \dots & a_{2n} \\
\vdots & \vdots & & \vdots \\
a_{n1} & a_{n2} & \dots & a_{nn}
\end{vmatrix},
\]或\(\det\A\),或\(\abs{\A}\);
即
\begin{equation}\label{equation:行列式.行列式的定义式}
\begin{vmatrix}
a_{11} & a_{12} & \dots & a_{1n} \\
a_{21} & a_{22} & \dots & a_{2n} \\
\vdots & \vdots & & \vdots \\
a_{n1} & a_{n2} & \dots & a_{nn}
\end{vmatrix}
\defeq
\sum\limits_{\v{j}{n}}
(-1)^{\tau(\v{j}{n})}
a_{1 j_1} a_{2 j_2} \dotsm a_{n j_n}.
\end{equation}
这里,求和指标\(\v{j}{n}\)表示遍取所有\(n\)阶排列.

我们称\cref{equation:行列式.行列式的定义式}
为“行列式\(\abs{A}\)的\textbf{完全展开式}”.
\end{definition}

特别地,%
\begin{gather}
\begin{vmatrix}
a_{11} & a_{12} \\
a_{21} & a_{22}
\end{vmatrix}
= a_{11} a_{22} - a_{12} a_{21}, \\
\begin{vmatrix}
a_{11} & a_{12} & a_{13} \\
a_{21} & a_{22} & a_{23} \\
a_{31} & a_{32} & a_{33}
\end{vmatrix}
= \begin{array}[t]{l}
(a_{11} a_{22} a_{33} + a_{12} a_{23} a_{31} + a_{13} a_{21} a_{32} \\
\hspace{20pt}
- a_{13} a_{22} a_{31} - a_{12} a_{21} a_{33} - a_{11} a_{23} a_{32})
\end{array}
\end{gather}

我们还可以用数学归纳法证明以下两条公式:
\begin{gather}
\begin{vmatrix}
a_{11} & a_{12} & \dots & a_{1n} \\
& a_{22} & \dots & a_{2n} \\
& & \ddots & \vdots \\
& & & a_{nn}
\end{vmatrix}
= a_{11} a_{22} \dotsm a_{nn}, \\
\begin{vmatrix}
& & & & a_{1n} \\
& & & a_{2,n-1} & a_{2n} \\
& & & \vdots & \vdots \\
& a_{n-1,2} & \dots & a_{n-1,n-1} & a_{n-1,n} \\
a_{n1} & a_{n2} & \dots & a_{n,n-1} & a_{nn}
\end{vmatrix}
=(-1)^{\frac{1}{2}n(n-1)} a_{1n} a_{2,n-1} \dotsm a_{n-1,2} a_{n1}.
\end{gather}

\begin{lemma}
设\(\A=(a_{ij})_n\),而\(\v{i}{n}\)与\(\v{j}{n}\)是两个\(n\)阶排列,则\begin{gather}
(-1)^{\tau(\v{i}{n})+\tau(\v{j}{n})}
	a_{i_1j_1} a_{i_2j_2} \dotsm a_{i_nj_n} \tag1
\end{gather}是\(\abs{\A}\)的项.
\begin{proof}
由乘法交换律,(1)式可以经过\(s\)次互换两个因子的次序写成\[
(-1)^{\tau(\v{i}{n})+\tau(\v{j}{n})}
	a_{1 l_1} a_{2 l_2} \dotsm a_{n l_n},
\]其中,\(\v{l}{n}\)是一个\(n\)阶排列.

同时,行标排列\(\v{i}{n}\)与列标排列\(\v{j}{n}\)分别经过\(s\)次对换变到\(1,2,\dotsc,n\)与\(\v{l}{n}\),而它们的奇偶性都分别改变了\(s\)次,总共改变了\(2s\)次(偶数次),故\[
(-1)^{\tau(\v{i}{n})+\tau(\v{j}{n})}
= (-1)^{\tau(1,2,\dotsc,n)+\tau(\v{l}{n})}
= (-1)^{\tau(\v{l}{n})},
\]这说明(1)式是行列式\(\abs{\A}\)的项.
\end{proof}
\end{lemma}

\begin{corollary}
设\(\v{i}{n}\)是一个确定的\(n\)阶排列,则\[
\abs{\A} = \sum\limits_{\v{j}{n}}{
	(-1)^{\tau(\v{i}{n})+\tau(\v{j}{n})} a_{i_1 j_1} a_{i_2 j_2} \dotsm a_{i_n j_n}
}.
\]
\end{corollary}

行列式的项的因子顺序也可按列标的自然序排列.
\begin{corollary}
\(\abs{\A} =
\sum\limits_{\v{i}{n}}
(-1)^{\tau(\v{i}{n})} a_{i_1 1} a_{i_2 2} \dotsm a_{i_n n}\).
\end{corollary}
这条推论说明了行列式的行与列的地位是相同的(即\(\det\A = \det\A^T\)).

\begin{example}
若\(n\)阶行列式\(\det\A\)中为零的元多于\(n^2-n\)个,证明:\(\det\A=0\).
% TODO
\end{example}

\begin{example}
证明:如果\(n\ (n\geqslant2)\)阶矩阵\(\A\)的元素为\(1\)或\(-1\),则\(\abs{\A}\)必为偶数.
% TODO
\end{example}

\subsection{行列式的性质}
\begin{property}\label{theorem:行列式.性质1}
设\(\A=(a_{ij})_n\),则\(\det \A = \det \A^T\).
\end{property}
这就说明,行列互换,行列式的值不变.

\begin{property}\label{theorem:行列式.性质2}
\(\det (\a_1,\dotsc,k\a_j,\dotsc,\a_n)
=k \cdot \det (\a_1,\dotsc,\a_j,\dotsc,\a_n)\).
\end{property}
这就说明,行列式某列(行)各元素的公因子可以提到行列式外.

\begin{corollary}\label{theorem:行列式.性质2.推论1}
\(\det (\a_1,\dotsc,\z,\dotsc,\a_n) = 0\).
\end{corollary}
也就是说,如果行列式中某一列(行)元素全为零,则行列式等于零.

\begin{corollary}\label{theorem:行列式.性质2.推论2}
\(\det (k\A) = k^n \det \A\).
\end{corollary}

一般说来,\(\det(k\A) \neq k \det\A\).

\begin{property}\label{theorem:行列式.性质3}
\(\det (\a_1,\dotsc,\b + \g,\dotsc,\a_n)
= \det (\a_1,\dotsc,\b,\dotsc,\a_n)
+ \det (\a_1,\dotsc,\g,\dotsc,\a_n)\).
\begin{proof}
\begin{align*}
\det (\a_1,\dotsc,\b + \g,\dotsc,\a_n)
&= \sum\limits_{\v{i}{n}}{
 (-1)^{\tau(\v{i}{n})}
 a_{i_1 1} \dotsm (b_{i_j} + c_{i_j}) \dotsm a_{i_n n}
} \\
&= \sum\limits_{\v{i}{n}}{
 (-1)^{\tau(\v{i}{n})}
 a_{i_1 1} \dotsm b_{i_j} \dotsm a_{i_n n}
} \\
&\qquad+ \sum\limits_{\v{i}{n}}{
 (-1)^{\tau(\v{i}{n})}
 a_{i_1 1} \dotsm c_{i_j} \dotsm a_{i_n n}
} \\
&= \det (\a_1,\dotsc,\b,\dotsc,\a_n)
+ \det (\a_1,\dotsc,\g,\dotsc,\a_n).
\qedhere
\end{align*}
\end{proof}
\end{property}
注:一般地,\(\det(\a_1+\b_1,\a_2+\b_2,\dotsc,\a_n+\b_n)\)可以拆成\(2^n\)个行列式之和.

\begin{property}\label{theorem:行列式.性质4}
\(\det(\a_1,\dotsc,\a_s,\dotsc,\a_t,\dotsc,\a_n)
=-\det(\a_1,\dotsc,\a_t,\dotsc,\a_s,\dotsc,\a_n)\).
\end{property}
也就是说,交换两列(行),行列式变号.

\begin{property}\label{theorem:行列式.性质5}
\(\det(\a_1,\dotsc,k\a,\dotsc,l\a,\dotsc,\a_n) = 0\).
\end{property}
这就说明,行列式中若有两列(行)成比例,则行列式等于零.

\begin{property}\label{theorem:行列式.性质6}
\(\det(\a_1,\dotsc,\a_s,\dotsc,\a_t,\dotsc,\a_n)
= \det(\a_1,\dotsc,\a_s,\dotsc,\a_t + k\a_s,\dotsc,\a_n)\).
\end{property}
这说明,将一列的\(k\)倍加到另一列,行列式的值不变.

\begin{example}
设\(\A\)为奇数阶反对称矩阵,即\(\A^T = -\A\),则\(\abs{\A}=0\).
\begin{proof}
设\(\A = (a_{ij})_n\).因为\(\A^T = -\A\),根据行列式的性质,有\[
\abs{\A} = \abs{\A^T} = \abs{-\A} = (-1)^n \abs{\A} = -\abs{\A},
\]于是\(\abs{\A} = 0\).
\end{proof}
\end{example}

\begin{example}
计算\(n\)阶行列式\[
D_n = \begin{vmatrix}
k & \lambda & \lambda & \dots & \lambda \\
\lambda & k & \lambda & \dots & \lambda \\
\lambda & \lambda & k & \dots & \lambda \\
\vdots & \vdots & \vdots & & \vdots \\
\lambda & \lambda & \lambda & \dots & k
\end{vmatrix},
\quad k\neq\lambda.
\]
\begin{solution}
当\(n>1\)时,有\begin{align*}
D_n &= \begin{vmatrix}
k+(n-1)\lambda & \lambda & \lambda & \dots & \lambda \\
k+(n-1)\lambda & k & \lambda & \dots & \lambda \\
k+(n-1)\lambda & \lambda & k & \dots & \lambda \\
\vdots & \vdots & \vdots & & \vdots \\
k+(n-1)\lambda & \lambda & \lambda & \dots & k
\end{vmatrix} \\
&= [k+(n-1)\lambda] \begin{vmatrix}
1 & \lambda & \lambda & \dots & \lambda \\
1 & k & \lambda & \dots & \lambda \\
1 & \lambda & k & \dots & \lambda \\
\vdots & \vdots & \vdots & & \vdots \\
1 & \lambda & \lambda & \dots & k
\end{vmatrix} \\
&= [k+(n-1)\lambda] \begin{vmatrix}
1 & \lambda & \lambda & \dots & \lambda \\
0 & k-\lambda & 0 & \dots & 0 \\
0 & 0 & k-\lambda & \dots & 0 \\
\vdots & \vdots & \vdots & & \vdots \\
0 & 0 & 0 & \dots & k-\lambda
\end{vmatrix} \\
&= [k+(n-1)\lambda] (k-\lambda)^{n-1}.
\tag1
\end{align*}

当\(n=1\)时,\(D_1 = k\)符合(1)式.
\end{solution}
\end{example}

\begin{example}\label{example:行列式.两个向量的乘积矩阵的行列式}
设\(\a,\b\)是\(n\)维列向量.
求:\(\abs{\a\b^T}\).
\begin{solution}
根据\cref{theorem:行列式.性质2},有\begin{align*}
\abs{\a\b^T} = \begin{vmatrix}
\a_1 \b_1 & \a_1 \b_2 & \dots & \a_1 \b_n \\
\a_2 \b_1 & \a_2 \b_2 & \dots & \a_2 \b_n \\
\vdots & \vdots & & \vdots \\
\a_n \b_1 & \a_n \b_2 & \dots & \a_n \b_n
\end{vmatrix}
= \a_1 \a_2 \dotsm \a_n \cdot \begin{vmatrix}
\b_1 & \b_2 & \dots & \b_n \\
\b_1 & \b_2 & \dots & \b_n \\
\vdots & \vdots & & \vdots \\
\b_1 & \b_2 & \dots & \b_n
\end{vmatrix}.
\end{align*}而\[
\begin{vmatrix}
\b_1 & \b_2 & \dots & \b_n \\
\b_1 & \b_2 & \dots & \b_n \\
\vdots & \vdots & & \vdots \\
\b_1 & \b_2 & \dots & \b_n
\end{vmatrix}
\]各行成比例,根据\cref{theorem:行列式.性质5},那么该行列式等于0,可知\(\abs{\a\b^T} = 0\).
\end{solution}
\end{example}

\section{行列式按行(或列)展开及其计算}
\subsection{子式}
\begin{definition}
在矩阵\(\A=(a_{ij})_{s \times n}\)中,任取\(k\)行\(k\)列,%
位于这些行与列交叉处的\(k^2\)个元素按原顺序排成的\(k\)阶矩阵的行列式\[
\begin{vmatrix}
a_{i_1,j_1} & a_{i_1,j_2} & \dots & a_{i_1,j_k} \\
a_{i_2,j_1} & a_{i_2,j_2} & \dots & a_{i_2,j_k} \\
\vdots & \vdots & \ddots & \vdots \\
a_{i_k,j_1} & a_{i_k,j_2} & \dots & a_{i_k,j_k}
\end{vmatrix}, \quad \begin{array}{c}
1 \leqslant i_1 < i_2 < \dotsb < i_k \leqslant s; \\
1 \leqslant j_1 < j_2 < \dotsb < j_k \leqslant n
\end{array}
\]称为“\(\A\)的一个\(k\)阶\textbf{子式}(minor)”,记作\[
\MatrixMinor\A{
	\v{i}{k} \\
	\v{j}{k}
}.
\]
\end{definition}

\begin{property}
设矩阵\(\A = (a_{ij})_{s \times n}\).
如果存在\(r < \min\{s,n\}\),%
使得所有\(r\)阶子式都等于零,%
则对任意\(k > r\)有\(\A\)的所有\(k\)阶子式全为零.
\end{property}

\subsection{主子式}
\begin{definition}
设\(\A=(a_{ij})_n\),%
\(k\)阶子式\[
	\MatrixMinor\A{
		i_1,i_2,\dotsc,i_k \\
		i_1,i_2,\dotsc,i_k
	}
\]
称为“\(\A\)的\(k\)阶\textbf{主子式}”.
\end{definition}

\subsection{顺序主子式}
\begin{definition}
设\(\A=(a_{ij})_n\),%
\(\A\)位于左上角的子式\[
	D_k = \MatrixMinor\A{
		1,2,\dotsc,k \\
		1,2,\dotsc,k
	}
	= \begin{vmatrix}
		a_{11} & a_{12} & \dots & a_{1k} \\
		a_{21} & a_{22} & \dots & a_{2k} \\
		\vdots & \vdots & \ddots & \vdots \\
		a_{k1} & a_{k2} & \dots & a_{kk}
	\end{vmatrix}
	\quad(1 \leqslant k \leqslant n)
\]称为\(\A\)的\(k\)阶\textbf{顺序主子式}(ordinal principal minor).
\end{definition}

\subsection{余子式、代数余子式}
\begin{definition}
在\(n\)阶矩阵\(\A=(a_{ij})_n\)中,取定\(k\)行\(k\)列得到一个\(k\)阶子式\[
	\alpha = \MatrixMinor\A{
		\v{i}{k} \\
		\v{j}{k}
	}.
\]
划去这个\(k\)阶子式所在的行和列,
剩下的元素按原顺序排成的\((n-k)\)阶矩阵的行列式称为
“子式\(\alpha\)的\textbf{余子式}(cofactor)”,
在这个余子式前乘以\((-1)^{i_1+\dotsb+i_k+j_1+\dotsb+j_k}\)
得到“子式\(\alpha\)的\textbf{代数余子式}(algebraic cofactor)”.
令\[\begin{split}
\{\v{i'}{n-k}\} &= \{1,2,\dotsc,n\}-\{\v{i}{k}\}, \\
\{\v{j'}{n-k}\} &= \{1,2,\dotsc,n\}-\{\v{j}{k}\},
\end{split}\]
并且\(i'_1<i'_2<\dotsb<i'_{n-k},%
j'_1<j'_2<\dotsb<j'_{n-k}\),
则子式\(\alpha\)的余子式可以记作\[
	\MatrixMinor\A{
		\v{i'}{n-k} \\
		\v{j'}{n-k}
	}.
\]

特别地,在\(\A\)中,取定第\(i\)行第\(j\)列确定一个元素\(a_{ij}\).
划去这个元素所在的行和列,剩下的\((n-1)^2\)个元素按原顺序排成的\((n-1)\)阶行列式叫“元素\(a_{ij}\)的\textbf{余子式}”,记作\(M_{ij}\),即\[
M_{ij} = \begin{vmatrix}
a_{11} & \dots & a_{1\ j-1} & a_{1\ j+1} & \dots & a_{1n} \\
\vdots & & \vdots & \vdots & & \vdots \\
a_{i-1\ 1} & \dots & a_{i-1\ j-1} & a_{i-1\ j+1} & \dots & a_{i-1\ n} \\
a_{i+1\ 1} & \dots & a_{i+1\ j-1} & a_{i+1\ j+1} & \dots & a_{i+1\ n} \\
\vdots & & \vdots & \vdots & & \vdots \\
a_{n1} & \dots & a_{n\ j-1} & a_{n\ j+1} & \dots & a_{nn} \\
\end{vmatrix}_{n-1}.
\]又称\[
A_{ij}=(-1)^{i+j} M_{ij}
\]为“\(a_{ij}\)的\textbf{代数余子式}”.
\end{definition}

\begin{theorem}[拉普拉斯定理]\label{theorem:行列式.拉普拉斯定理}
%@see: 《高等代数创新教材(上册)》(丘维声) P68. 定理1(Laplace定理)
在\(n\)阶矩阵\(\A\)中,%
取定第\(\v{i}{k}\)行(\(i_1<i_2<\dotsb<i_k\)),%
则这\(k\)行元素形成的所有\(k\)阶子式与它们自己的代数余子式的乘积之和等于\(\abs{\A}\),%
即
\begin{equation}
	\abs{\A} =
	\sum\limits_{1 \leqslant j_1 < j_2 < \dotsb < j_k \leqslant n}
	\MatrixMinor\A{
		\v{i}{k} \\
		\v{j}{k}
	}
	\cdot(-1)^{i_1+\dotsb+i_k+j_1+\dotsb+j_k}\cdot
	\MatrixMinor\A{
		\v{i'}{n-k} \\
		\v{j'}{n-k}
	}.
\end{equation}
\end{theorem}

\begin{theorem}
设\(\A=(a_{ij})_n\),\(A_{ij}\)为\(a_{ij}\)的代数余子式(\(i,j=1,2,\dotsc,n\)),%
\begin{enumerate}
\item 行列式等于它的任一行的各元与其代数余子式乘积之和,即\[
\abs{\A}=\sum\limits_{j=1}^n a_{ij} A_{ij}
\quad(i=1,2,\dotsc,n);
\]

\item 行列式的任一行的各元与另一行对应元素的代数余子式乘积之和为零,即\[
\sum\limits_{j=1}^n a_{ij} A_{kj} = 0 \quad (i \neq k).
\]
\end{enumerate}
\begin{proof}
注意\[
\tau(i,1,2,\dotsc,i-1,i+1,\dotsc,n) = i-1,
\]\[
\tau(j,j_1,j_2,\dotsc,j_{i-1},j_{i+1},\dotsc,j_n) = j-1+\tau(j_1,j_2,\dotsc,j_{i-1},j_{i+1},\dotsc,j_n),
\]于是有\begin{align*}
\abs{\A}
&= \sum\limits_{j,j_1,j_2,\dotsc,j_{i-1},j_{i+1},\dotsc,j_n}
	(-1)^{\tau(i,1,2,\dotsc,i-1,i+1,\dotsc,n) + \tau(j,j_1,j_2,\dotsc,j_{i-1},j_{i+1},\dotsc,j_n)} a_{ij} \prod\limits_{\substack{k=1 \\ k \neq i}}^n a_{k j_k} \\
&= \sum\limits_{j=1}^n a_{ij} (-1)^{(i-1)+(j-1)}
	\sum\limits_{j_1,j_2,\dotsc,j_{i-1},j_{i+1},\dotsc,j_n}
		(-1)^{\tau(j_1,j_2,\dotsc,j_{i-1},j_{i+1},\dotsc,j_n)}
			\prod\limits_{\substack{k=1 \\ k \neq i}}^n a_{k j_k} \\
&= \sum\limits_{j=1}^n a_{ij} (-1)^{i+j} M_{ij}
= \sum\limits_{j=1}^n a_{ij} A_{ij}.
\qedhere
\end{align*}
\end{proof}
\end{theorem}

\subsection{伴随矩阵}
\begin{definition}\label{definition:伴随矩阵.伴随矩阵的定义}
设\(\A=(a_{ij})_n\),\(A_{ij}\)为\(a_{ij}\)的代数余子式(\(i,j=1,2,\dotsc,n\)),%
以\(A_{ij}\)作为第\(j\)行第\(i\)列元素(\(i,j=1,2,\dotsc,n\))构成的矩阵%
称为\(\A\)的\textbf{伴随矩阵}(adjoint matrix),记为\(\A^*\),即\[
\A^*
= [(A_{ij})_n]^T
= (A_{ji})_n
= \begin{bmatrix}
A_{11} & A_{21} & \dots & A_{n1} \\
A_{12} & A_{22} & \dots & A_{n2} \\
\vdots & \vdots & \ddots & \vdots \\
A_{1n} & A_{2n} & \dots & A_{nn}
\end{bmatrix}.
\]
\end{definition}

\begin{theorem}
设\(\A=(a_{ij})_n\)的伴随矩阵为\(\A^*\),则\begin{gather}
\A \A^* = \A^* \A = \abs{\A} \E, \label{equation:行列式.伴随矩阵.恒等式1} \\
(\A^*)^T = (\A^T)^*, \label{equation:行列式.伴随矩阵.恒等式2} \\
(\A\B)^* = \B^* \A^*. \label{equation:行列式.伴随矩阵.恒等式3}
\end{gather}
\end{theorem}

\begin{example}
设\(\A=(a_{ij})_n\)的伴随矩阵为\(\A^*\),求\((k\A)^*\).
\begin{solution}
设\(\A\)的元素\(a_{ij}\)的代数余子式是\(A_{ij}\),%
那么矩阵\(k\A = (b_{ij})_n\)的元素\(b_{ij} = k a_{ij}\)的代数余子式是\[
B_{ij}
= (-1)^{i+j} \begin{vmatrix}
k a_{11} & \dots & k a_{1,j-1} & k a_{1,j+1} & \dots & k a_{1n} \\
\vdots & & \vdots & \vdots & & \vdots \\
k a_{i-1,1} & \dots & k a_{i-1,j-1} & k a_{i-1,j+1} & \dots & k a_{i-1,n} \\
k a_{i+1,1} & \dots & k a_{i+1,j-1} & k a_{i+1,j+1} & \dots & k a_{i+1,n} \\
\vdots & & \vdots & \vdots & & \vdots \\
k a_{n1} & \dots & k a_{n,j-1} & k a_{n,j+1} & \dots & k a_{nn} \\
\end{vmatrix}
= k^{n-1} A_{ij}.
\]
因此,\(k\A\)的伴随矩阵是\(
(B_{ji})_n = k^{n-1} (A_{ji})_n
= k^{n-1} \A^*
\).
\end{solution}
\end{example}

\begin{example}
试证范德蒙德行列式:
\begin{equation}\label{equation:行列式.范德蒙德行列式}
V_n = \begin{vmatrix}
1 & 1 & 1 & \dots & 1 \\
x_1 & x_2 & x_3 & \dots & x_n \\
x_1^2 & x_2^2 & x_3^2 & \dots & x_n^2 \\
\vdots & \vdots & \vdots& & \vdots \\
x_1^{n-1} & x_2^{n-1} & x_3^{n-1} & \dots & x_n^{n-1}
\end{vmatrix}=\prod\limits_{1 \leqslant j < i \leqslant n}{(x_i-x_j)}.
\end{equation}
\begin{proof}
利用数学归纳法.
当\(n=2\)时,\(V_2 = \begin{vmatrix}
	1 & 1 \\ x_1 & x_2
\end{vmatrix} = x_2 - x_1\),结论成立.
假设\(n=k-1\)时,结论成立,%
那么当\(n=k\)时,在\(V_k\)中,依次将第\(k-1\)行的\(-x_1\)倍加到第\(k\)行,%
将第\(k-2\)行的\(-x_1\)倍加到第\(k-1\)行,%
以此类推,%
直至把第\(1\)行的\(-x_1\)倍加到第\(2\)行,得\[
V_k = \begin{vmatrix}
1 & 1 & 1 & \dots & 1 \\
0 & x_2 - x_1 & x_3 - x_1 & \dots & x_k - x_1 \\
0 & x_2(x_2 - x_1) & x_3(x_3 - x_1) & \dots & x_k(x_k - x_1) \\
\vdots & \vdots & \vdots & & \vdots \\
0 & x_2^{k-2}(x_2 - x_1) & x_3^{k-2}(x_3 - x_1) & \dots & x_k^{k-2}(x_k - x_1) \\
\end{vmatrix}_k,
\]按第一列展开得\begin{align*}
V_n &= 1 \times (-1)^{1+1} \times \begin{vmatrix}
x_2 - x_1 & x_3 - x_1 & \dots & x_k - x_1 \\
x_2(x_2 - x_1) & x_3(x_3 - x_1) & \dots & x_k(x_k - x_1) \\
\vdots & \vdots & & \vdots \\
x_2^{k-2}(x_2 - x_1) & x_3^{k-2}(x_3 - x_1) & \dots & x_k^{k-2}(x_k - x_1) \\
\end{vmatrix}_{k-1} \\
&= (x_2 - x_1)(x_3 - x_1)\dotsm(x_k - x_1) \begin{vmatrix}
1 & 1 & \dots & 1 \\
x_2 & x_3 & \dots & x_k \\
\vdots & \vdots & & \vdots \\
x_2^{k-2} & x_3^{k-2} & \dots & x_k^{k-2} \\
\end{vmatrix}_{k-1} \\
&= (x_2 - x_1)(x_3 - x_1)\dotsm(x_k - x_1) \prod\limits_{2 \leqslant j < i \leqslant k}{x_i - x_j} \\
&= \prod\limits_{1 \leqslant j < i \leqslant k}{x_i - x_j}.
\qedhere
\end{align*}
\end{proof}
\end{example}

\begin{example}
计算\(n\)阶三对角行列式\begin{equation}
D_n = \begin{vmatrix}
a+b & ab & \\
1 & a+b & ab & \\
 & 1 & a + b & \ddots & \\
 & & \ddots & \ddots & ab \\
 & & & 1 & a+b \\
\end{vmatrix}_n.
\end{equation}
\begin{proof}
将\(D_n\)按第一行展开,得\begin{align*}
D_n &= (a+b) D_{n-1} - ab \begin{vmatrix}
1 & ab \\
0 & a+b & ab \\
 & 1 & a+b & \ddots \\
 & & \ddots & \ddots & ab \\
 & & & 1 & a+b \\
\end{vmatrix}_{n-1} \\
&= (a+b) D_{n-1} - ab D_{n-2},
\end{align*}
把上式改写为\(D_n - a D_{n-1} = b(D_{n-1} - a D_{n-2})\),继续递推下去,得\begin{align*}
D_n - a D_{n-1} &= b(D_{n-1} - a D_{n-2}) = b^2(D_{n-2} - a D_{n-3}) \\
&= \dotsb = b^{n-2}(D_2 - a D_1) \\
&= b^{n-2} [(a^2 + ab + b^2) - a(a+b)] = b^n,
\end{align*}所以\[
D_n - a D_{n-1} = b^n,
\]又由\(a\)和\(b\)的对称性可得\[
D_n - b D_{n-1} = a^n.
\]

当\(a \neq b\)时,可解得\[
D_n = \frac{a^{n+1} - b^{n+1}}{a - b}
= a^n + a^{n-1} b + a^{n-2} b^2 + \dotsb + a b^{n-1} + b^n.
\]当\(a = b\)时,由\[
D_n - a D_{n-1} = a^n
\]可继续递推得\begin{align*}
D_n &= a D_{n-1} + a^n
= a(a D_{n-2} + a^{n-1}) + a^n
= a^2 D_{n-2} + 2 a^n \\
&= a^3 D_{n-3} + 3 a^n
= \dotsb
= (n+1) a^n.
\end{align*}
综上所述,对任意\(a,b\in\mathbb{R}\)都有\[
D_n = a^n + a^{n-1} b + a^{n-2} b^2 + \dotsb + a b^{n-1} + b^n.
\qedhere
\]
\end{proof}
\end{example}

\section{矩阵乘积的行列式}
\begin{lemma}
设\(\A\)为\(n\)阶非零方阵,\(\B\)为\(m\)阶非零方阵,\(\C\)为\(m \times n\)矩阵,\(\D\)为\(n \times m\)矩阵,则\begin{align}
\begin{vmatrix}
\A & \z \\
\C & \B
\end{vmatrix}
&= \begin{vmatrix}
\A & \D \\
\z & \B
\end{vmatrix}
= \abs{\A} \abs{\B}, \label{equation:行列式.广义三角阵的行列式1} \\
\begin{vmatrix}
\z & \A \\ \B & \C \end{vmatrix}
&= \begin{vmatrix} \D & \A \\ \B & \z \end{vmatrix}
= (-1)^{mn} \abs{\A} \abs{\B}. \label{equation:行列式.广义三角阵的行列式2}
\end{align}
\end{lemma}

\begin{theorem}[矩阵乘积的行列式定理]\label{theorem:行列式.矩阵乘积的行列式}
设\(\A\)、\(\B\)都是\(n\)阶矩阵,则\begin{equation}
\abs{\A\B} = \abs{\A} \abs{\B}.
\end{equation}
\end{theorem}

\begin{example}
设\(\A\)是\(n\)阶矩阵,且\(\A\A^T = \E\),\(\abs{\A}<0\).证明:\(\abs{\E+\A}=0\).
\begin{proof}
等式\(\A\A^T=\E\)的两端取行列式,\(\abs{\A} \abs{\A^T} = \abs{\E}\),由\(\abs{\A} = \abs{\A^T}\)与\(\abs{\A} < 0\),得\(\abs{\A}^2 = 1\),\(\abs{\A} = -1\).故\begin{align*}
\abs{\E+\A}
&= \abs{\A\A^T+\A}
= \abs{\A(\A^T+\E)}
= \abs{\A} \abs{\A^T+\E} \\
&= -\abs{(\A^T+\E)^T}
= -\abs{\A+\E},
\end{align*}所以\(\abs{\E+\A} = -\abs{\E+\A}\),\(\abs{\E+\A}=0\).
\end{proof}
\end{example}

\begin{example}
设\(\A^*\)是\(n\)阶矩阵\(\A\)的伴随矩阵,若\(\abs{\A} \neq 0\),证明:\(\abs{\A^*} \neq 0\)且\(\abs{\A^*}=\abs{\A}^{n-1}\).
\begin{proof}
因为\(\A \A^* = \abs{\A} \E\),%
所以\(\abs{\A \A^*} = \abs{\abs{\A} \E} = \abs{\A}^n \abs{\E} = \abs{\A}^n\).
又因为\(\abs{\A \A^*} = \abs{\A} \abs{\A^*}\),%
所以\(\abs{\A^*} = \abs{\A}^{n-1} \neq 0\).
\end{proof}
\end{example}

\begin{example}
设矩阵\(\A,\B\)满足\(\abs{\A}=3,\abs{\B}=2,\abs{\A^{-1}+\B}=2\),试求\(\abs{\A+\B^{-1}}\).
\begin{solution}
由于\(\abs{\E+\A\B} = \abs{\A(\A^{-1}+\B)} = \abs{\A} \abs{\A^{-1}+\B} = 6\),所以\[
\abs{\A+\B^{-1}}
= \frac{\abs{(\A+\B^{-1})\B}}{\abs{\B}}
= \frac{\abs{\A\B+\E}}{\abs{\B}}
= 3.
\]
\end{solution}
\end{example}

\begin{example}
用\(\abs{\A}^2 = \abs{\A} \abs{\A^T}\)的方法计算行列式\[
\abs{\A} = \begin{vmatrix}
a & b & c & d \\
-b & a & d & -c \\
-c & -d & a & b \\
-d & c & -b & a
\end{vmatrix}.
\]
\begin{solution}
因为\begin{align*}
\abs{\A}^2 &= \abs{\A} \abs{\A^T} = \abs{\A \A^T} \\
&= \abs{\begin{bmatrix}
a & b & c & d \\
-b & a & d & -c \\
-c & -d & a & b \\
-d & c & -b & a
\end{bmatrix} \begin{bmatrix}
a & -b & -c & -d \\
b & a & -d & c \\
c & d & a & -b \\
d & -c & b & a
\end{bmatrix}} \\
&= \abs{(a^2+b^2+c^2+d^2) \E}_4
= (a^2+b^2+c^2+d^2)^4,
\end{align*}所以\(\abs{\A} = \pm(a^2+b^2+c^2+d^2)^2\),再由\(\abs{\A}\)中含有项\(a^4\),得\[
\abs{\A} = (a^2+b^2+c^2+d^2)^2.
\]
\end{solution}
\end{example}

\begin{example}
计算:\[
D = \begin{vmatrix}
a & a & a & a \\
a & a & -a & -a \\
a & -a & a & -a \\
a & -a & -a & a
\end{vmatrix}.
\]
\begin{solution}
因为\[
\begin{bmatrix}
1 & 0 & 0 & 0 \\
0 & 0 & 1 & 0 \\
0 & 1 & 0 & 0 \\
0 & 0 & 0 & 1
\end{bmatrix} \begin{bmatrix}
a & a & a & a \\
a & a & -a & -a \\
a & -a & a & -a \\
a & -a & -a & a
\end{bmatrix}
= \begin{bmatrix}
1 & 0 & 0 & 0 \\
1 & 1 & 0 & 0 \\
1 & 0 & 1 & 0 \\
1 & 1 & 1 & 1
\end{bmatrix} \begin{bmatrix}
a & a & a & a \\
0 & -2 a & 0 & -2 a \\
0 & 0 & -2 a & -2 a \\
0 & 0 & 0 & 4 a
\end{bmatrix},
\]而\[
\begin{vmatrix}
1 & 0 & 0 & 0 \\
0 & 0 & 1 & 0 \\
0 & 1 & 0 & 0 \\
0 & 0 & 0 & 1
\end{vmatrix} = -1, \qquad
\begin{vmatrix}
1 & 0 & 0 & 0 \\
1 & 1 & 0 & 0 \\
1 & 0 & 1 & 0 \\
1 & 1 & 1 & 1
\end{vmatrix} = 1,
\]\[
\begin{vmatrix}
a & a & a & a \\
0 & -2 a & 0 & -2 a \\
0 & 0 & -2 a & -2 a \\
0 & 0 & 0 & 4 a
\end{vmatrix} = a\cdot(-2a)\cdot(-2a)\cdot(4a) = 16a^2,
\]所以\[
\begin{vmatrix}
a & a & a & a \\
a & a & -a & -a \\
a & -a & a & -a \\
a & -a & -a & a
\end{vmatrix} = -16a^2.
\]
\end{solution}
\end{example}

\begin{example}
设\(s_k = a_1^k + a_2^k + a_3^k + a_4^k\ (k=1,2,3,4,5,6)\).计算:\[
D = \begin{vmatrix}
4 & s_1 & s_2 & s_3 \\
s_1 & s_2 & s_3 & s_4 \\
s_2 & s_3 & s_4 & s_5 \\
s_3 & s_4 & s_5 & s_6
\end{vmatrix}.
\]
\begin{solution}
令矩阵\[
\A = \begin{bmatrix}
1 & 1 & 1 & 1 \\
a_1 & a_2 & a_3 & a_4 \\
a_1^2 & a_2^2 & a_3^2 & a_4^2 \\
a_1^3 & a_2^3 & a_3^3 & a_4^3
\end{bmatrix},
\]显然\[
D = \det(\A\A^T) = \abs{\A}^2.
\]而根据\cref{equation:行列式.范德蒙德行列式},\(\abs{\A} = \prod\limits_{1 \leqslant j < i \leqslant n} (a_i - a_j)\),故\(D = \prod\limits_{1 \leqslant j < i \leqslant n} (a_i - a_j)^2\).
\end{solution}
\end{example}

\section{总结:计算行列式的方法}
\begin{enumerate}
\item 利用初等变换,将行列式化为上三角形;
\item 拆成若干个行列式的和;
\item 按行(或列)展开;
\item 归纳法;
\item 递推关系法;
\item 加边法(即升阶法);
\item 降阶法;
\item 利用范德蒙德行列式等特殊行列式计算.
\end{enumerate}
