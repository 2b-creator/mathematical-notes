\section{特殊矩阵}
\subsection{三角矩阵}
\begin{definition}
设\(\A=(a_{ij})_n\).
\begin{enumerate}
	\item 如果\(\A\)满足\[
		a_{ij} = 0
		\quad(i>j),
	\]
	则称之为\DefineConcept{上三角形矩阵},
	简称\DefineConcept{上三角阵}.

	\item 如果\(\A\)满足\[
		a_{ij} = 0
		\quad(i<j),
	\]
	则称之为\DefineConcept{下三角形矩阵},
	简称\DefineConcept{下三角阵}.
\end{enumerate}
\end{definition}

\begin{example}
设\(\A,\B\)都是\(n\)阶上三角阵,证明:\(\A\B\)是上三角阵.
\begin{proof}
利用数学归纳法.
当\(n=1\)时,\(\A=a\),\(\B=b\),\(\A\B = ab\),结论成立.

假设\(n=k\)时上三角阵的乘积是上三角阵.
当\(n=k+1\)时,对矩阵\(\A\)和\(b\)分块如下:\[
	\A = \begin{bmatrix}
		a_{11} & \A_2 \\
		\z & \A_4
	\end{bmatrix},
	\qquad
	\B = \begin{bmatrix}
		b_{11} & \B_2 \\
		\z & \B_4
	\end{bmatrix},
\]
其中\(\A_4\)和\(\B_4\)都是\(k\)阶上三角阵,
由归纳假设,\(\A_4 \B_4\)是\(k\)阶上三角阵,
则\[
	\A\B = \begin{bmatrix}
		a_{11} b_{11} & a_{11} \B_2 + \A_2 \B_4 \\
		\z & \A_4 \B_4
	\end{bmatrix},
\]
即\(\A\B\)是\(k+1\)阶上三角阵.
\end{proof}
\end{example}

\subsection{对角矩阵}
\begin{definition}
若方阵\(\A=(a_{ij})_n\)满足:\[
	[i \neq j \implies a_{ij} = 0]
	\land
	(\exists i)[a_{ii}\neq0].
\]
则称\(\A\)为\DefineConcept{对角矩阵},
记作\(\diag(a_{11},a_{22},\dotsc,a_{nn})\).
\end{definition}

\subsection{对称矩阵,厄米矩阵}
\begin{definition}
若矩阵\(\A \in M_n(K)\)满足\[
    \A^T = \A,
\]
则称\(\A\)为\DefineConcept{对称矩阵}(symmetric matrix).
\end{definition}

\begin{definition}
如果矩阵\(\A \in M_n(K)\)满足\[
    \A^H = \A,
\]
那么把\(\A\)称为\DefineConcept{厄米矩阵}(Hermitian matrix).
\end{definition}

\begin{example}
设矩阵\(\A = (a_{ij})_n\).
试证:\(\A\A^T\)为对称矩阵.
\begin{proof}
因为\((\A \A^T)^T = (\A^T)^T \A^T = \A \A^T\),所以\(\A \A^T\)是对称矩阵.
\end{proof}
\end{example}

\begin{example}
设\(\A\)和\(\B\)是同阶对称矩阵.
试证:\(\A\B\)是对称矩阵的充要条件是\(\A\B = \B\A\).
\begin{proof}
因为\(\A\)和\(\B\)都是对称矩阵,所以\[
	\A^T = \A, \qquad
	\B^T = \B.
\]
又因为\(\A\B = \B\A\),\[
	(\A\B)^T = \B^T \A^T = \B\A = \A\B,
\]
所以\(\A\B\)是对称矩阵.
\end{proof}
\end{example}

\subsection{反对称矩阵}
\begin{definition}
若方阵\(\A\)满足条件\(\A^T = -\A\),
则称\(\A\)为\DefineConcept{反对称矩阵}(antisymmetric matrix)或\DefineConcept{斜对称矩阵}.
\end{definition}

\begin{property}
反对称矩阵主对角线上的元素全为零.
\end{property}

\begin{example}
零矩阵\(\z\)是唯一一个既是实对称矩阵又是实反对称矩阵的矩阵.
\begin{proof}
\(\A^T = \A = -\A \implies 2\A = \A+\A = \z \implies \A = \z\).
\end{proof}
\end{example}

\begin{example}
设\(\A\)是一个方阵,证明:\(\A+\A^T\)为对称矩阵,\(\A-\A^T\)为反对称矩阵.
\begin{proof}
因为\((\A+\A^T)^T = \A^T+\A\),而\((\A-\A^T)^T = \A^T - \A = -(\A-\A^T)\),所以\(\A+\A^T\)为对称矩阵,\(\A-\A^T\)为反对称矩阵.
显然有\(\A = \frac{\A + \A^T}{2} + \frac{\A - \A^T}{2}\).
\end{proof}
\end{example}

\begin{example}
设\(\A\)是3阶实对称矩阵,\(\B\)是3阶实反对称矩阵,\(\A^2 = \B^2\),试证:\(\A = \B = \z\).
\begin{proof}
设\(\A = (a_{ij})_n\),\(\B = (b_{ij})_n\).
因为\(\A = \A^T\),\(\A^2 = \A^T \A\),所以\(\A^2\)的\(\opair{i,j}\)元素为\(a_{1i} a_{1j} + a_{2i} a_{2j} + \dotsb + a_{ni} a_{nj}\).
因为\(\B = -\B^T\),\(\B^2 = -\B^T \B\),所以\(\B^2\)的\(\opair{i,j}\)元素为\(-(b_{1i} b_{1j} + b_{2i} b_{2j} + \dotsb + b_{ni} b_{nj})\).
因为\(\A^2 = \B^2\),所以\[
a_{1i} a_{1j} + a_{2i} a_{2j} + \dotsb + a_{ni} a_{nj}
= -(b_{1i} b_{1j} + b_{2i} b_{2j} + \dotsb + b_{ni} b_{nj}).
\]

当\(i=j\)时,上式变为\(
a_{1i}^2 + a_{2i}^2 + \dotsb + a_{ni}^2
= -(b_{1i}^2 + b_{2i}^2 + \dotsb + b_{ni}^2)
\),又由\(a_{ij},b_{ij} \in \mathbb{R}\)可知\(a_{1i}^2 + a_{2i}^2 + \dotsb + a_{ni}^2 \geq 0\),\(-(b_{1i}^2 + b_{2i}^2 + \dotsb + b_{ni}^2) \leq 0\),所以\[
a_{1i}^2 + a_{2i}^2 + \dotsb + a_{ni}^2
= -(b_{1i}^2 + b_{2i}^2 + \dotsb + b_{ni}^2) = 0,
\]进而有\[
a_{1i} = a_{2i} = \dotsb = a_{ni} = b_{1i} = b_{2i} = \dotsb = b_{ni} = 0.
\qedhere
\]
\end{proof}
\end{example}

\subsection{幂零矩阵}
\begin{definition}
设矩阵\(\A \in M_n(K)\).
若\(\exists m\in\mathbb{N}^+\),
使得\(\A^m = \z\),
则称“\(\A\)是\DefineConcept{幂零矩阵}”;
称使得\(\A^m = \z\)成立的最小正整数\[
    \min\Set{ m\in\mathbb{N}^+ \given \A^m = \z }
\]为“\(\A\)的\DefineConcept{幂零指数}”.
\end{definition}
