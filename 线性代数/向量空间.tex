\chapter{向量空间}
\section{向量空间}
为了直接用线性方程组的系数和常数项判断方程组有没有解,有多少解,
我们再前面给出了用系数行列式判断\(n\)个方程的\(n\)元线性方程组有唯一解的充要条件.
这一判定方法只适用于方程数目与未知量数目相等的线性方程组;
而且,当系数行列式等于零时,只能得出方程组无解或有无穷多解的结论,
没有办法区分什么时候无解,什么时候有无穷多解.
对于任意的线性方程组,有没有这样一种判定方法:
直接依据它的系数和常数项,给出它有没有解,有多少解呢?
为此我们需要探讨和建立线性方程组的进一步的理论.
这一理论还将使我们弄清楚线性方程组有无穷多个解时解的结构.

\subsection{向量空间}
设\(K\)是数域,\(n\)是任意给定的一个正整数.
令\[
	K^n \defeq \Set{ (\AutoTuple{a}{n}) \given a_i \in K\ (i=1,2,\dotsc,n) }.
\]

如果\[
	a_i=b_i
	\quad(i=1,2,\dotsc,n),
\]
则称“\(K^n\)中的两个元素\((\v{a}{n})\)与\((\v{b}{n})\)相等”.

在\(K^n\)中规定“加法”运算如下:
\begin{equation}\label{equation:向量空间.向量的加法.定义式}
	(\v{a}{n}) + (\v{b}{n})
	\defeq (a_1+b_1,a_2+b_2,\dotsc,a_n+b_n).
\end{equation}

在\(K\)的元素与\(K^n\)的元素之间规定“数量乘法”运算如下:
\begin{equation}\label{equation:向量空间.向量的数量乘法.定义式}
	k (\v{a}{n})
	\defeq (k a_1,k a_2,\dotsc,k a_n).
\end{equation}

容易验证,上述加法和数量乘法满足下述8条运算法则:
\begin{enumerate}
	\item 加法交换律,即\((\forall \a,\b \in K^n)[\a+\b=\b+\a]\).

	\item 加法结合律,即\((\forall \a,\b,\g \in K^n)[(\a+\b)+\g=\a+(\b+\g)]\).

	\item 记\(\z=(0,0,\dotsc,0)\),\[
		(\forall\a \in K^n)[\a+\z = \z+\a = \a].
	\]
	称\(\z\)为“\(K^n\)的\DefineConcept{零元}(zero element)”.

	\item \(\forall\a=(\AutoTuple{a}{n}) \in K^n\),令\[
		-\a \defeq (\AutoTuple{-a}{n}),
	\]
	则\(-\a \in K^n\)且\[
		\a+-\a
		= -\a+\a
		= \z;
	\]
	称\(-\a\)为“\(\a\)的\DefineConcept{负元}(negative element)”.

	\item \((\forall \a \in K^n)[1 \a=\a]\).

	\item \((\forall \a \in K^n)(\forall k,l \in K)[k (l \a)=(kl) \a]\).

	\item \((\forall \a \in K^n)(\forall k,l \in K)[(k+l) \a=k \a+l \a]\).

	\item \((\forall \a,\b \in K^n)(\forall k \in K)[k (\a+\b)=k \a+k \b]\).
\end{enumerate}

\begin{definition}
数域\(K\)上全体\(n\)元组组成的集合\(K^n\),
连同定义在它上面的加法运算和数量乘法运算,
及其满足的8条运算法则一起,
称为“数域\(K\)上的一个\(n\)维\DefineConcept{向量空间}(vector space)”.
\(K^n\)的元素称为“\(n\)维\DefineConcept{向量}(vector)”.

对于\(K^n\)中的任意一个向量\(\a=(\AutoTuple{a}{n})\),
称数\[
	a_i\quad(i=1,2,\dotsc,n)
\]为“\(\a\)的第\(i\)个\DefineConcept{分量}”.
\end{definition}

在\(n\)维向量空间\(K^n\)中,我们可以额外定义减法运算如下:
\begin{equation}\label{equation:向量空间.向量的减法.定义式}
	\a-\b \defeq \a+(-\b).
\end{equation}

在\(n\)维向量空间\(K^n\)中,容易验证下述4条性质:
\begin{property}
\((\forall\a \in K^n)[0\cdot\a=\z]\).
\end{property}

\begin{property}
\((\forall\a \in K^n)[(-1)\cdot\a=-\a]\).
\end{property}

\begin{property}
\((\forall k \in K)[k\z=\z]\).
\end{property}

\begin{property}
\(k\a=\z \implies k=0 \lor \a=\z\).
\end{property}

把\(n\)元组写成一行,得\[
	(\AutoTuple{a}{n})
	\quad\text{或}\quad
	\begin{bmatrix}
		a_1 & a_2 & \dots & a_n
	\end{bmatrix},
\]
称之为“\(n\)维\DefineConcept{行向量}(row vector)”.

把\(n\)元组写成一列,得\[
	\begin{bmatrix} a_1 \\ a_2 \\ \vdots \\ a_n \end{bmatrix},
\]
称之为“\(n\)维\DefineConcept{列向量}(column vector)”;
不过,我们有时候会为了方便排版,把列向量写成\[
	(a_1,a_2,\dotsc,a_n)^T.
\]

\(K^n\)可以看成是\(n\)维行向量组成的向量空间,也可以看成是\(n\)维列向量组成的向量空间.
两者并没有本质的区别,只是它们的元素的写法不同而已.


由有限个\(n\)维行向量构成的集合,称为“\(n\)维\DefineConcept{行向量组}”.
由有限个\(n\)维列向量构成的集合,称为“\(n\)维\DefineConcept{列向量组}”.
\(n\)维行向量组和\(n\)维列向量组统称\(n\)维\DefineConcept{向量组},是\(n\)维向量空间的子集.

称满足
\[
	e_{ij} = \left\{ \begin{array}{ll}
		1, & i=j, \\
		0, & i \neq j
	\end{array} \right.
\]
的向量组
\[
	\e_i = \begin{bmatrix}
		e_{1i} \\ e_{2i} \\ \vdots \\ e_{ni}
	\end{bmatrix}
	\quad(i=1,2,\dotsc,n)
\]为“\(K^n\)的\DefineConcept{基本向量组}”.

\subsection{线性组合,线性表出}
%@see: 《线性代数》(张慎语、周厚隆) P67 定义5
在\(K^n\)中,给定向量组\(\AutoTuple{\a}{s}\),
任给\(K\)中一组数\(\AutoTuple{k}{s}\),
我们把\[
	k_1 \a_1 + k_2 \a_2 + \dotsb + k_s \a_s
\]
称为“向量组\(\AutoTuple{\a}{s}\)的一个\DefineConcept{线性组合}(linear combination)”,
把\(\AutoTuple{k}{s}\)称为\DefineConcept{系数};

对于\(\b \in K^n\),
如果存在\(K\)中一组数\(\AutoTuple{c}{s}\),使得\[
	\b = c_1 \a_1 + c_2 \a_2 + \dotsb + c_s \a_s,
\]
则称“\(\b\)可以由\(\AutoTuple{\a}{s}\) \DefineConcept{线性表出}”.

现在,利用向量的加法运算和数量乘法运算,
我们可以把数域\(K\)上\(n\)元线性方程组 \labelcref{equation:线性方程组.线性方程组的代数形式}
写成
\begin{equation}\label{equation:线性方程组.线性方程组的向量形式}
	x_1 \a_1 + x_2 \a_2 + \dotsb + x_n \a_n = \b,
\end{equation}
其中\[
	\a_j=(a_{1j},a_{2j},\dotsc,a_{sj})^T,
	\quad
	j=1,2,\dotsc,n.
\]
于是,\begin{align*}
	&\text{数域\(K\)上线性方程组\(x_1 \a_1 + x_2 \a_2 + \dotsb + x_n \a_n = \b\)有解} \\
	&\iff \text{\(K\)中存在一组数\(\AutoTuple{c}{n}\),使得\(c_1 \a_1 + c_2 \a_2 + \dotsb + c_n \a_n = \b\)成立} \\
	&\iff \text{\(\b\)可以由\(\AutoTuple{\a}{n}\)线性表出}.
\end{align*}
这样我们把线性方程组有没有解的问题归结为:
常数项列向量\(\b\)能不能由系数矩阵的列向量组线性表出.
这个结论有两方面的意义:
一方面,为了从理论上研究线性方程组有没有解,
就需要去研究\(\b\)能否由\(\AutoTuple{\a}{n}\)线性表出;
另一方面,对于\(K^n\)中给定的向量组\(\AutoTuple{\a}{n}\),
以及给定的\(\b\),
为了判断\(\b\)能否由\(\AutoTuple{\a}{n}\)线性表出,
就可以去判断线性方程组\(x_1 \a_1 + x_2 \a_2 + \dotsb + x_n \a_n = \b\)是否有解.

\subsection{线性子空间}
在\(K^n\)中,从理论上如何判断任一向量\(\b\)能否由向量组\(\AutoTuple{\a}{n}\)线性表出?
从线性表出的定义知道,这需要考察\(\b\)是否等于\(\AutoTuple{\a}{n}\)的某一个线性组合.
为此,我们把\(\AutoTuple{\a}{n}\)的所有线性组合组成一个集合\(W\),即\[
	W \defeq \Set{ k_1 \a_1 + k_2 \a_2 + \dotsb + k_s \a_s \given k_i \in K, i=1,2,\dotsc,s }.
\]
如果我们能够把\(W\)的结构研究清楚,那么就比较容易判断\(\b\)是否属于\(W\),
也就是判断\(\b\)能否由\(\AutoTuple{\a}{n}\)线性表出.

现在我们来研究\(W\)的结构.
任取\(\a,\g\in W\),设\[
	\a=a_1\a_1+a_2\a_2+\dotsb+a_s\a_s, \qquad
	\g=b_1\a_1+b_2\a_2+\dotsb+b_s\a_s,
\]
则\begin{align*}
	\a+\g
	&=(a_1\a_1+a_2\a_2+\dotsb+a_s\a_s)+(b_1\a_1+b_2\a_2+\dotsb+b_s\a_s) \\
	&=(a_1+b_1)\a_1+(a_2+b_2)\a_2+\dotsb+(a_s+b_s)\a_s,
\end{align*}
从而\(\a+\g\in W\).

再任取\(k\in W\),则\begin{align*}
	k\a
	&=k(a_1\a_1+a_2\a_2+\dotsb+a_s\a_s) \\
	&=(ka_1)\a_1+(ka_2)\a_2+\dotsb+(ka_s)\a_s,
\end{align*}
从而\(k\a\in W\).

受此启发,我们引出如下概念.
\begin{definition}
\(K^n\)的一个非空子集\(U\)如果满足:
\begin{enumerate}
	\item \(U\)对\(K^n\)的加法封闭,即\[
		\a,\b \in U \implies \a+\b \in U;
	\]
	\item \(U\)对\(K^n\)的数量乘法封闭,即\[
		\a \in U, k \in K \implies k\a \in U;
	\]
\end{enumerate}
那么称\(U\)是“\(K^n\)的一个\DefineConcept{线性子空间}(linear subspace)”,
简称为\DefineConcept{子空间}(subspace).
\end{definition}
\(\{\z\}\)是\(K^n\)的一个子空间,称之为\DefineConcept{零子空间}(zero subspace).
\(K^n\)也是其自身的一个子空间.

从上面的讨论知道,在\(K^n\)中,
向量组\(\AutoTuple{\a}{s}\)的所有线性组合组成的集合\(W\)是\(K^n\)的一个子空间,
称它为“\(\AutoTuple{\a}{s}\)生成的子空间”,
记作\[
	\opair{\AutoTuple{\a}{s}}.
\]

于是,我们得出结论,以下三个命题等价:
\begin{enumerate}
	\item 数域\(K\)上的\(n\)元线性方程组\(x_1 \a_1 + x_2 \a_2 + \dotsb + x_n \a_n = \b\)有解.
	\item \(\b\)可以由\(\AutoTuple{\a}{n}\)线性表出.
	\item \(\b\in\opair{\AutoTuple{\a}{n}}\).
\end{enumerate}

\begin{example}
设\(1 \leqslant r < n\).
证明:集合\[
	U = \Set{ (\v{a}{r},0,\dotsc,0) \given a_i \in K, i=1,2,\dotsc,r }
\]是\(K^n\)的子空间.
\begin{proof}
在\(K\)中任取一个数\(k\),%
再在\(U\)中任取两个向量
\[
	\a = (\v{a}{r},0,\dotsc,0),
	\qquad
	\b = (\v{b}{r},0,\dotsc,0),
\]
有\[
	\a+\b = (a_1+b_1,a_2+b_2,\dotsc,a_r+b_r,0,\dotsc,0) \in U,
\]\[
	k \a = (k a_1,k a_2,\dotsc,k a_r,0,\dotsc,0) \in U,
\]
因此\(U\)是\(K^n\)的一个子空间.
\end{proof}
\end{example}

%\begin{theorem}
%\(K^n\)中任一向量都可由基本向量组唯一地线性表出.
%\begin{proof}
%对于任意一个向量\(\a=(\v{a}{n})^T\),%
%线性方程组\(x_1 \e_1 + x_2 \e_2 + \dotsb + x_n \e_n = \a\)的系数行列式为
%\[
%\begin{vmatrix}
%	1 & 0 & \dots & 0 \\
%	0 & 1 & \dots & 0 \\
%	\vdots & \vdots & & \vdots \\
%	0 & 0 & \dots & 1
%\end{vmatrix}
%= 1 \neq 0,
%\]
%那么,根据克拉默法则,这个线性方程组有唯一解,%
%于是\(K^n\)中任一向量\(\a\)都能由基本向量组线性表出,且表出方式唯一.
%事实上,由于
%\[
%a_1 \begin{bmatrix}
%1 \\ 0 \\ 0 \\ \vdots \\ 0
%\end{bmatrix}
%+ a_2 \begin{bmatrix}
%0 \\ 1 \\ 0 \\ \vdots \\ 0
%\end{bmatrix}
%+ \dotsb + a_n \begin{bmatrix}
%0 \\ 0 \\ 0 \\ \vdots \\ 1
%\end{bmatrix}
%= \begin{bmatrix}
%a_1 \\ a_2 \\ a_3 \\ \vdots \\ a_n
%\end{bmatrix},
%\]
%因此,用基本向量组标出向量\(\a\)的方式为
%\[
%\a = a_1 \e_1 + a_2 \e_2 + \dotsb + a_n \e_n.
%\qedhere
%\]
%\end{proof}
%\end{theorem}

\section{向量组的线性相关性}
在上一节,我们把线性方程组有没有解的问题归结为:
常数项列向量\(\b\)能否由系数矩阵的列向量组\(\AutoTuple{\a}{s}\)线性表出.
那么,如何研究\(K^n\)中一个向量能不能由一个向量组线性表出呢?

\subsection{线性相关性的概念}
我们首先回顾\cref{theorem:解析几何.两向量共线的充要条件1,%
theorem:解析几何.三向量共面的充要条件1},
以及\cref{theorem:解析几何.两向量不共线的充要条件1,%
theorem:解析几何.三向量不共面的充要条件1}.

受此启发,我们提出以下两个概念.
\begin{definition}\label{definition:线性方程组.线性相关与线性无关的定义}
%@see: 《线性代数》(张慎语、周厚隆) P68 定义6
设\(A=\Set{\AutoTuple{\a}{s}}\)是\(n\)维向量空间\(K^n\)中的一个向量组.

如果\(K\)中存在不全为零的数\(\AutoTuple{k}{s}\),使得\[
	k_1 \a_1 + k_2 \a_2 + \dotsb + k_s \a_s = \z,
\]
则称“向量组\(A\) \DefineConcept{线性相关}(linearly dependent)”;
否则,称“向量组\(A\) \DefineConcept{线性无关}(linearly independent)”.
\end{definition}

显然,从\cref{definition:线性方程组.线性相关与线性无关的定义} 立即可得
\begin{align*}
	&\hspace{-20pt}\text{向量组\(A\)线性无关} \\
	&\iff
	[k_1 \a_1 + k_2 \a_2 + \dotsb + k_s \a_s = \z
	\implies
	k_1 = k_2 = \dotsb = k_s = 0].
\end{align*}

特别地,我们规定:空集\(\emptyset\)线性无关.

\subsection{线性相关性的判定条件}
根据线性相关、线性无关的定义和解析几何的结论,
在几何空间中,共线的两个向量是线性相关的,
共面的三个向量是线性相关的,
不共面的三个向量是线性无关的,
不共线的两个向量是线性无关的.

下面我们再来看几个例子.
\begin{example}\label{example:线性方程组.含有零向量的向量组线性相关}
%@see: 《线性代数》(张慎语、周厚隆) P68 例1
向量空间\(K^n\)中的零向量可以由任意向量组\(\AutoTuple{\b}{t}\)线性表出,
这是因为恒等式\[
	0\b_1+0\b_2+\dotsb+0b_t=\z.
\]
进一步,
含有零向量\(\z\)的向量组\[
	\Set{\z,\AutoTuple{\a}{s}}
\]总是线性相关的,
这是因为\[
	1 \z + 0 \a_1 + 0 \a_2 + \dotsb + 0 \a_s = \z.
\]
\end{example}

\begin{example}\label{example:线性方程组.基本向量组线性无关}
%@see: 《线性代数》(张慎语、周厚隆) P68 例2
\(K^n\)的基本向量组\(\AutoTuple{\e}{n}\)线性无关.
\begin{proof}
令\(k_1 \e_1 + k_2 \e_2 + \dotsb + k_n \e_n = \z\),即\[
	k_1 (1,0,\dotsc,0)^T + k_2 (0,1,\dotsc,0)^T + \dotsb k_n (0,0,\dotsc,1)^T = \z.
\]
进一步,有\[
	(\AutoTuple{k}{n})^T = (0,\dotsc,0)^T,
\]
于是\(k_1 = k_2 = \dotsb = k_n = 0\),
因此\(\e_1,\e_2,\dotsc,\e_n\)线性无关.
\end{proof}
\end{example}

\begin{example}\label{example:线性方程组.单向量组线性相关的充要条件}
证明:一个向量\(\a\)组成的向量组\(\{\a\}\)线性相关的充要条件是\(\a=\z\).
\begin{proof}
必要性.
设\(\{\a\}\)线性相关,存在数\(k \neq 0\)使得\(k\a = \z\),可得\(\a = \z\).

充分性.
设\(\a = \z\),则\(1\a = \z\),而数\(1 \neq 0\),故\(\{\a\}\)线性相关.
\end{proof}
\end{example}
我们还可以给出逆否命题:\(\text{向量组\(\{\a\}\)线性无关} \iff \a\neq\z\).

\begin{theorem}\label{theorem:线性方程组.向量组线性相关的充要条件1}
向量组\(A=\{\AutoTuple{\a}{s}\}\ (s>1)\)线性相关的充要条件是:
\(A\)中至少有一个向量可由其余\(s-1\)个向量线性表出.
\begin{proof}
必要性.
\(A\)线性相关,则存在不全为零的数\(\AutoTuple{k}{s}\),使得\[
	k_1 \a_1 + k_2 \a_2 + \dotsb + k_s \a_s = \z.
\]
设\(k_i\neq0\ (1 \leqslant i \leqslant s)\),于是\[
	\a_i = -\frac{1}{k_i} (
		k_1 \a_1 + k_2 \a_2 + \dotsb
		+ k_{i-1} \a_{i-1} + k_{i+1} \a_{i+1}
		+ \dotsb + k_s \a_s
	),
\]
即\(\a_i\)可由其余\(s-1\)个向量线性表出.

充分性.
若\(\a_j \in A\)可由其余\(s-1\)个向量线性表出,即\[
	\a_j = l_1 \a_1 + \dotsb + l_{j-1} \a_{j-1} + l_{j+1} \a_{j+1} + \dotsb + l_s \a_s,
\]
移项得\[
	l_1 \a_1 + \dotsb
	+ l_{j-1} \a_{j-1} + (-1) \a_j + l_{j+1} \a_{j+1}
	+ \dotsb + l_s \a_s = \z,
\]
上式等号左边的系数中至少有一个数\(-1\neq0\),
因此\(A\)线性相关.
\end{proof}
\end{theorem}

根据\cref{theorem:线性方程组.向量组线性相关的充要条件1},我们立即有它的逆否命题成立:
\begin{corollary}
向量组\(A=\{\AutoTuple{\a}{s}\}\ (s>1)\)线性无关的充要条件是:
\(A\)中每一个向量都不能由其余向量线性表出.
\end{corollary}

\begin{example}
%@see: 《线性代数》(张慎语、周厚隆) P68 例4
设向量组\(A=\{\AutoTuple{\a}{s}\}\)线性无关,
\(B=\{\AutoTuple{\a}{s},\b\}\)线性相关.
证明:\(\b\)可由\(A\)线性表出.
\begin{proof}
由于向量组\(B\)线性相关,
则存在不全为零的数\(\AutoTuple{k}{s},k\)使得\[
	k_1 \a_1 + k_2 \a_2 + \dotsb + k_s \a_s + k \b = \z.
\]
假设\(k = 0\),
则\(\AutoTuple{k}{s}\)不全为零,
且有\(k_1 \a_1 + k_2 \a_2 + \dotsb + k_s \a_s = \z\),
即\(A\)线性相关,
与题设矛盾,说明\(k \neq 0\).
于是\[
	\b = -\frac{1}{k} (k_1 \a_1 + k_2 \a_2 + \dotsb + k_s \a_s).
	\qedhere
\]
\end{proof}
利用\cref{theorem:线性方程组.向量组线性相关的充要条件1}
可以证明,“\(\b\)可由\(A\)线性表出”蕴含“\(B\)线性相关”.
因此,“\(\b\)可由\(A\)线性表出”是“\(B\)线性相关”成立的充要条件.
\end{example}

\begin{theorem}\label{theorem:线性方程组.部分组线性相关则全组线性相关}
若向量组\(A=\{\AutoTuple{\a}{s}\}\)的一个部分组线性相关,则\(A\)线性相关.
\begin{proof}
假设\(A\)的部分组\(B=\{\AutoTuple{\a}{t}\}\ (t \leqslant s)\)线性相关,
即存在不全为零的数\(\AutoTuple{k}{t}\)使得\[
	k_1 \a_1 + k_2 \a_2 + \dotsb + k_t \a_t = \z;
\]
从而有\[
	k_1 \a_1 + k_2 \a_2 + \dotsb + k_t \a_t + 0 \a_{t+1} + \dotsb + 0 \a_s = \z;
\]
由于上式等号左边的系数\(\AutoTuple{k}{t},0,\dotsc,0\)不全为零,
因此向量组\(A\)线性相关.
\end{proof}
\end{theorem}

由\cref{theorem:线性方程组.部分组线性相关则全组线性相关} 立即得到:
\begin{corollary}\label{theorem:线性方程组.全组线性无关则任一部分组线性无关}
如果向量组\(A\)线性无关,
那么\(A\)的任意一个部分组也线性无关.
\end{corollary}

给定\(n\)维向量组\(\AutoTuple{\a}{s}\),
为其中的每个向量都添上\(m\)个分量,
所添分量的位置对于每个向量都一样,
把得到的\(n+m\)维向量组\(\AutoTuple{\b}{s}\)称为%
“\(\AutoTuple{\a}{s}\)的\DefineConcept{延伸组}”;
反过来,把\(\AutoTuple{\a}{s}\)称为%
“\(\AutoTuple{\b}{s}\)的\DefineConcept{缩短组}”.

如果向量组线性无关,那么它的延伸组也线性无关.
如果向量组线性相关,那么它的缩短组也线性相关.

\begin{theorem}
\(n\)个\(n\)维列向量\(\AutoTuple{\a}{n}\)线性相关的充要条件是:\[
	\det(\AutoTuple{\a}{n})=0.
\]
\end{theorem}

\begin{corollary}
\(n\)个\(n\)维列向量\(\AutoTuple{\a}{n}\)线性无关的充要条件是:\[
	\det(\AutoTuple{\a}{n})\neq0.
\]
\end{corollary}

需要注意的是,当\(s \neq n\)时,
\(s\)个\(n\)维向量\(\AutoTuple{\a}{s}\)不能构成行列式,
只能用其他方法判断其线性相关性.

\begin{theorem}[替换定理]
设向量组\(\AutoTuple{\a}{s}\)线性无关,
\(\b=b_1\a_1+\dotsb+b_s\a_s\).
如果\(b_j\neq0\),
那么用\(\b\)替换\(\a_j\)以后得到的向量组
\(\AutoTuple{\a}{j-1},\b,\AutoTuple{\a}[j+1]{s}\)
也线性无关.
\end{theorem}

\begin{example}
设\(\A\)是3阶矩阵,\(\a_1,\a_2,\a_3\)为3维列向量组,若\(\A\a_1,\A\a_2,\A\a_3\)线性无关,证明:\(\a_1,\a_2,\a_3\)线性无关,且\(\A\)为可逆矩阵.
\begin{proof}
因为\(\A\a_1,\A\a_2,\A\a_3\)线性无关,所以齐次线性方程组\[
x_1 \A\a_1 + x_2 \A\a_2 + x_3 \A\a_3
= (\A\a_1,\A\a_2,\A\a_3) (x_1,x_2,x_3)^T
= \z
\]只有零解(即\(x_1 = x_2 = x_3 = 0\)),根据克拉默法则,有\[
\det(\A\a_1,\A\a_2,\A\a_3) \neq 0.
\]

因为\(\det(\A\a_1,\A\a_2,\A\a_3) = \abs{\A} \cdot \det(\a_1,\a_2,\a_3) \neq 0\),所以\(\abs{\A} \neq 0\)(即\(\A\)是可逆矩阵),且\(\det(\a_1,\a_2,\a_3) \neq 0\)(即齐次线性方程组\(x_1 \a_1 + x_2 \a_2 + x_3 \a_3 = \z\)只有零解,向量组\(\a_1,\a_2,\a_3\)线性无关).
\end{proof}
\end{example}

\begin{example}
证明:\(\mathbb{R}^n\)中的任意正交组线性无关.
\begin{proof}
设\(A=\{\AutoTuple{\a}{m}\}\)是\(\mathbb{R}^n\)的一个正交组,令\[
k_1 \a_1 + k_2 \a_2 + \dotsb + k_m \a_m = \z,
\]两端分别与\(\a_1\)作内积,%
即\(\vectorinnerproduct{(k_1 \a_1 + k_2 \a_2 + \dotsb + k_m \a_m)}{\a_1} = \vectorinnerproduct{\z}{\a_1}\);
由内积性质,\[
k_1 (\vectorinnerproduct{\a_1}{\a_1})
+ k_2 (\vectorinnerproduct{\a_2}{\a_1})
+ \dotsb
+ k_m (\vectorinnerproduct{\a_m}{\a_1})
= \vectorinnerproduct{\z}{\a_1},
\]
其中\(\vectorinnerproduct{\z}{\a_1} = 0\),%
\(\vectorinnerproduct{\a_j}{\a_1} = 0\ (j=2,3,\dotsc,m)\),%
故\(k_1 \vectorinnerproduct{\a_1}{\a_1} = 0\),%
而\(\vectorinnerproduct{\a_1}{\a_1} > 0\),%
所以\(k_1=0\).
同理可得\(k_2=k_3=\dotsb=k_m=0\),从而\(A\)线性无关.
\end{proof}
\end{example}

\section{向量组的秩}
\subsection{向量组的等价关系}
\begin{definition}
在\(P^n\)中,如果向量组\[
	\alpha=\{\v{\a}{s}\}
\]的每个向量都可由向量组\[
	\beta=\{\b_1,\b_2,\dotsc,\b_t\}
\]线性表出,
则称“\(\alpha\)可由\(\beta\) \DefineConcept{线性表出}”.

如果\(\alpha\)与\(\beta\)可以相互线性表出,
则称\(\alpha\)与\(\beta\) \DefineConcept{等价},
记作\(\alpha \cong \beta\).
\end{definition}

\begin{theorem}
部分组可由全组线性表出.
\begin{proof}
设数域\(P\)上一个向量组\(A=\{\v{\a}{s}\}\),%
从中任取\(t\ (t \leqslant s)\)个向量构成向量组\[
A'=\{ \a_1, \a_2, \dotsc, \a_t \}.
\]欲证部分组可由全组线性表出,即证\(\forall \a_j \in A'\),\(\exists \v{k}{j},\dotsc,k_s \in P\),使得\[
\a_j = k_1 \a_1 + k_2 \a_2 + \dotsb + k_j \a_j + \dotsb + k_s \a_s.
\]
显然只需要使\(k_j = 1\),使除了\(k_j\)以外的数都等于0,则上式恒成立.
\end{proof}
\end{theorem}

\begin{theorem}
\(\alpha=\{\v{\a}{s}\}\ (s>1)\)%
线性相关的充要条件是:
\(\alpha\)可由某个部分组%
\[\a_1,\dotsc,\a_{i-1},\a_{i+1},\dotsc,\a_s\]
线性表出.
\end{theorem}

\begin{theorem}
设向量组\(A=\{ \AutoTuple{\a}{s} \}\)可由%
向量组\(B=\{ \AutoTuple{\b}{t} \}\)线性表出;
如果\(s>t\),则\(A\)线性相关.
\begin{proof}
欲证\(A\)线性相关,须找到不全为零的\(s\)个数\(\AutoTuple{k}{s}\)使得\[
k_1 \a_1 + k_2 \a_2 + \dotsb + k_s \a_s = \z.
\]
因为向量组\(A\)可由\(B\)线性表出,即有\[
\left\{ \begin{array}{l}
\a_1 = c_{11} \b_1 + c_{21} \b_2 + \dotsb + c_{t1} \b_t, \\
\a_2 = c_{12} \b_1 + c_{22} \b_2 + \dotsb + c_{t2} \b_t, \\
\hdotsfor{1} \\
\a_s = c_{1s} \b_1 + c_{2s} \b_2 + \dotsb + c_{ts} \b_t.
\end{array} \right.
\]代入可得\[
\sum\limits_{j=1}^s k_j \a_j
=\sum\limits_{j=1}^s k_j \sum\limits_{i=1}^t c_{ij} \b_i
=\sum\limits_{j=1}^s \sum\limits_{i=1}^t k_j c_{ij} \b_i
=\sum\limits_{i=1}^t \b_i \sum\limits_{j=1}^s k_j c_{ij}
=\z.
\]
如此只需证存在不全为零的\(s\)个数\(\AutoTuple{k}{s}\)
使得对于任意\(i=1,2,\dotsc,t\)都有\[
\sum\limits_{j=1}^s k_j c_{ij} = 0.
\]
而关于\(k_i\ (i=1,2,\dotsc,s)\)的齐次线性方程组
\[
\left\{ \begin{array}{l}
c_{11} k_1 + c_{12} k_2 + \dotsb + c_{1s} k_s = 0, \\
c_{21} k_1 + c_{22} k_2 + \dotsb + c_{2s} k_s = 0, \\
\hdotsfor{1} \\
c_{t1} k_1 + c_{t2} k_2 + \dotsb + c_{ts} k_s = 0.
\end{array} \right.
\]
中方程数\(t\)小于未知量个数\(s\),必有非零解.
\end{proof}
\end{theorem}

\begin{corollary}
任意\(n+1\)个\(n\)维向量线性相关.换言之,向量个数大于维数的向量组线性相关.
\begin{proof}
\(P^n\)中任意\(n+1\)个\(n\)维向量\(\alpha = \{ \v{\a}{n+1} \}\)
可由基本向量组\(\v{\e}{n}\)线性表出,
这两个向量组中的向量个数满足\(n+1 > n\),向量组\(\alpha\)线性相关.
\end{proof}
\end{corollary}

\begin{corollary}
若线性无关向量组\[
\alpha=\{\v{\a}{s}\}
\]可由向量组\[
\beta=\{\b_1,\b_2,\dotsc,\b_t\}
\]线性表出,则\(s \leqslant t\).
\begin{proof}
假设\(s > t\),因为向量组\(\alpha\)可由\(\beta\)线性表出,所以向量组\(\alpha\)线性相关,矛盾,故\(s \leqslant t\).
\end{proof}
\end{corollary}

\begin{corollary}
两个等价的线性无关向量组含有相同的向量个数.
\begin{proof}
设\(A=\{\AutoTuple{\a}{s}\}\)
与\(B=\{\AutoTuple{\b}{t}\}\)
都线性无关,且\(A \cong B\).
因为\(A \cong B\),%
所以\(A\)可由\(B\)线性表出,%
从而\(s \leqslant t\);
同理可得\(t \leqslant s\);
综上所述,\(s = t\).
\end{proof}
\end{corollary}

\begin{example}
在数域\(K\)上,满足\[
\abs{a_{ii}} > \sum\limits_{\substack{1 \leqslant j \leqslant n \\ j \neq i}} \abs{a_{ij}}
\quad (i=1,2,\dotsc,n)
\]的\(n\)阶矩阵\(\A = (a_{ij})_n\)称为\DefineConcept{主对角占优矩阵}.
证明:\(\A\)的列向量组\(\v{\a}{n}\)的秩等于\(n\).
\begin{proof}
假设\(\v{\a}{n}\)线性相关,则在\(K\)中有一组不全为0的数\(\v{k}{n}\),使得\[
k_1 \a_1 + k_2 \a_2 + \dotsb + k_n \a_n = \z.
\]不妨设\(\abs{k_l} = \max\{\abs{k_1},\abs{k_2},\dotsc,\abs{k_n}\}\neq0\).
由\[
k_1 a_{l1} + k_2 a_{l2} + \dotsb + k_l a_{ll} + \dotsb + k_n a_{ln} = 0,
\]可得\[
a_{ll} = -\frac{1}{k_l} (k_1 a_{l1} + \dotsb + k_{l-1} a_{l,l-1} + k_{l+1} a_{l,l+1} + \dotsb + k_n a_{ln})
= - \sum\limits_{\substack{1 \leqslant j \leqslant n \\ j \neq l}} \frac{k_j}{k_l} a_{lj},
\]\[
\abs{a_{ll}} \leqslant \sum\limits_{\substack{1 \leqslant j \leqslant n \\ j \neq l}} \frac{\abs{k_j}}{\abs{k_l}} \abs{a_{lj}}
\leqslant \sum\limits_{\substack{1 \leqslant j \leqslant n \\ j \neq l}} \abs{a_{lj}}.
\]这与已知条件矛盾!
因此\(\v{\a}{n}\)线性无关,\(\rank\{\v{\a}{n}\} = n\).
\end{proof}
\end{example}

\subsection{极大(线性)无关组的概念}
\begin{definition}
设\(B=\{\a_{i_1},\a_{i_2},\dotsc,\a_{i_r}\}\)是\(A=\{\v{\a}{s}\}\)的一个部分组.
如果\begin{enumerate}
\item \(B\)线性无关,%
\item \(A\)可由\(B\)线性表出,%
\end{enumerate}
则称\(B\)是\(A\)的一个\DefineConcept{极大(线性)无关组}(maximal independent set).
\end{definition}

\begin{property}
向量组与其极大无关组等价.
\begin{proof}
因为作为部分组,极大无关组可由全组线性表出,同时根据极大无关组的定义全组可由极大无关组线性表出,则根据向量组等价的定义可知原向量组与极大无关组等价.
\end{proof}
\end{property}

\begin{corollary}
向量组的任何两个极大无关组等价,且包含相同个数的向量.
\begin{proof}
设\(A=\{\AutoTuple{\a}{r}\}\)
与\(B=\{\AutoTuple{\b}{t}\}\)
是某个向量组的两个极大无关组.
因为向量组的任意向量可由其极大无关组线性表出,%
所以\(A\)可由\(B\)线性表出,%
\(B\)也可由\(A\)线性表出,%
即\(A \cong B\),%
进而这两个等价的线性无关向量组含有相同的向量个数.
\end{proof}
\end{corollary}

\begin{theorem}
在\(P^n\)中,任意向量组的极大无关组的向量个数不大于\(n\)个.
\begin{proof}
根据定义,任意向量组的极大无关组是线性无关的,而向量个数大于维数的向量组总是线性相关,故任意向量组的极大无关组的向量个数总是不大于其维数\(n\)的.
\end{proof}
\end{theorem}

\begin{example}
求向量组\(\{\a\}\)的极大无关组.
\begin{solution}
显然有\[
\Powerset\{\a\} = \{ \emptyset, \{\a\} \},
\]即\(\{\a\}\)的部分组只有\(\emptyset\)和\(\{\a\}\),从而它的极大无关组也只能是这两者中的一个.

当\(\a=\z\)时,\(\{\a\}\)线性相关,不能满足极大无关组的定义,故\(\{\a\}\)的极大无关组是\(\emptyset\).

当\(\a\neq\z\)时,\(\{\a\}\)线性无关,所以\(\{\a\}\)的极大无关组是它本身.
\end{solution}
\end{example}

\subsection{向量组的秩}
\begin{definition}
向量组的极大无关组所含向量的个数,称为向量组的\DefineConcept{秩}(rank).
记\[
A = \{\v{\a}{s}\}
\]的秩为\(\rank A\)或\(r\{\v{\a}{s}\}\).
\end{definition}

\begin{property}
空集\(\emptyset\)的秩为零,即\(\rank\emptyset = 0\).
\end{property}

\begin{property}
零向量组的秩为零,即\(r\{\z,\z,\dotsc,\z\}=0\).
\begin{proof}
因为向量组本质是向量的集合,所以\[
\{\z,\z,\dotsc,\z\} = \{\z\},
\]而由上例知道\(\{\z\}\)的极大无关组是\(\emptyset\),故\[
r\{\z,\z,\dotsc,\z\}
= r\{\z\}
= \abs{\emptyset}
= 0.
\qedhere
\]
\end{proof}
\end{property}

\begin{corollary}
设向量组\(A=\{\v{\a}{s}\}\).
如果\(\rank{A}<s\),则向量组\(A\)线性相关;
如果\(\rank{A}=s\),则向量组\(A\)线性无关.
\end{corollary}

\begin{corollary}
设向量组
\(A=\{\AutoTuple{\a}{s}\}\)
可由向量组
\(B=\{\AutoTuple{\b}{t}\}\)
线性表出,%
则\(\rank A \leqslant \rank B\).
\begin{proof}
设\(\rank A = r\),\(\rank B = u\).因为\(A\)可由\(B\)线性表出,即\[
\a_k = \sum\limits_{i=1}^t l_{ki} \b_i,
\quad k=1,2,\dotsc,s.
\]
设\(A'=\{\AutoTuple{\a}{r}\}\)
和\(B'=\{\AutoTuple{\b}{u}\}\)
分别是\(A\)和\(B\)的极大无关组,%
则\(B\)可由\(B'\)线性表出,即\[
\b_i = \sum\limits_{j=1}^u b_{ij} \b_j,
\quad i=1,2,\dotsc,t;
\]所以有\[
\a_k = \sum\limits_{i=1}^t l_{ki} \sum\limits_{j=1}^u b_{ij} \b_j
= \sum\limits_{i=1}^t \sum\limits_{j=1}^u l_{ki} b_{ij} \b_j
= \sum\limits_{j=1}^u \b_j \sum\limits_{i=1}^t l_{ki} b_{ij},
\quad k=1,2,\dotsc,s.
\]

特别地,\(A'\)可由\(B'\)线性表出,%
则有\(r \leqslant u\),即\(\rank A \leqslant \rank B\).
\end{proof}
\end{corollary}

\begin{corollary}
等价向量组的秩相等.秩相等的向量组却不一定等价.
\begin{proof}
设向量组\(A\)与\(B\)等价,则\(A\)可由\(B\)线性表出,那么\(\rank A \leqslant \rank B\);同理可得\(\rank A \geqslant \rank B\),所以\(\rank A = \rank B\).

设\(A=\{ (0,1) \}\),\(B=\{ (1,0) \}\),虽然\(\rank A = \rank B = 1\),但\(A\)与\(B\)不等价.
\end{proof}
\end{corollary}

\begin{example}
证明:在秩为\(r\)的向量组中,任意\(r+1\)个向量必线性相关.
\begin{proof}
设向量组\(\AutoTuple{\a}{s}\)的秩为\(r\).
假设部分组\(\AutoTuple{\a}{r+1}\)线性无关,%
那么\[
\rank\{\AutoTuple{\a}{r+1}\} = r+1.
\]
因为部分组总可由全组线性表出,所以部分组的秩总是小于或等于全组的秩,即\[
r+1 = \rank\{\AutoTuple{\a}{r+1}\} \leqslant \rank\{\AutoTuple{\a}{s}\} = r,
\]矛盾,所以部分组\(\AutoTuple{\a}{r+1}\)一定线性相关.
\end{proof}
\end{example}

\begin{example}
设向量组\(\AutoTuple{\a}{s}\)的秩为\(r\).
如果\(\AutoTuple{\a}{r}\)线性无关,证明:
\(\AutoTuple{\a}{r}\)
是\(\AutoTuple{\a}{s}\)的一个极大无关组.
\begin{proof}
设\[
A=\{\AutoTuple{\a}{s}\},
\qquad
B=\{\AutoTuple{\a}{r}\}.
\]要证\(B\)是\(A\)的一个极大无关组,须证\(A\)的任意向量可由\(B\)线性表出.

\begin{enumerate}
\item 显然地,\(\a_i\ (i=1,2,\dotsc,r)\)可由\(B\)线性表出.

\item 根据上例,在秩为\(r\)的向量组中,任意\(r+1\)个向量必线性相关,那么向量组\[
A_i = \{\AutoTuple{\a}{r},\a_i\}\quad(i=r+1,\dotsc,s)
\]必线性相关.

又因为\(B\)线性无关,所以\(\a_i\ (i=r+1,\dotsc,s)\)可由\(B\)线性表出.
\end{enumerate}

综上所述,\(A\)的任意向量可由\(B\)线性表出,且\(B\)线性无关,根据极大无关组的定义,\(B\)是\(A\)的一个极大无关组.
\end{proof}
\end{example}

\begin{example}
向量组
\(\AutoTuple{\a}{r+1}\)
与部分组
\(\AutoTuple{\a}{r}\)
的秩相等.
证明:\(\a_{r+1}\)可由
\(\AutoTuple{\a}{r}\)
线性表出.
\begin{proof}
记\(A=\{\AutoTuple{\a}{r+1}\}\),
\(B=\{\AutoTuple{\a}{r}\}\).
设\(B\)的极大无关组为
\[
B'=\{\a_1,\a_2,\dotsc,\a_t\},
\quad 0 \leqslant t \leqslant r.
\]
由题意有
\(\rank A = \rank B = \rank B' = \abs{B'} = t\).

由上例可知,因为\(\rank A = t\),%
而\(B'\)线性无关,%
所以\(B'\)是\(A\)的一个极大无关组.
那么向量组\(A\)中的向量\(\a_{r+1}\)可以由极大无关组\(B'\)线性表出.
又由于\(B'\)是\(B\)的部分组,故\(B'\)可由\(B\)线性表出.
总而言之,\(A\)可由\(B\)线性表出.
\end{proof}
\end{example}

\subsection{极大无关组的求解}
\begin{theorem}
设矩阵\[
\A=(\AutoTuple{\a}{m})
\]经一系列初等行变换化为矩阵\[
\B=(\AutoTuple{\b}{m}),
\]则\(\a_{j1},\a_{j2},\dotsc,\a_{jk}\)为\(\A\)的列极大无关组的充要条件是:
\(\b_{j1},\b_{j2},\dotsc,\b_{jk}\)为\(\B\)的列极大无关组.
\begin{proof}
矩阵\((\a_{j1},\a_{j2},\dotsc,\a_{jk},\a_l)\)经相同的初等行变换化为
\[
(\b_{j1},\b_{j2},\dotsc,\b_{jk},\b_l) \quad(l=1,2,\dotsc,m).
\]
考虑以下四个向量形式的线性方程组
\begin{gather}
x_1 \a_{j1} + x_2 \a_{j2} + \dotsb + x_k \a_{jk} = \z, \tag1 \\
x_1 \b_{j1} + x_2 \b_{j2} + \dotsb + x_k \b_{jk} = \z, \tag2 \\
y_1 \a_{j1} + y_2 \a_{j2} + \dotsb + y_k \a_{jk} = \a_l, \tag3 \\
y_1 \b_{j1} + y_2 \b_{j2} + \dotsb + y_k \b_{jk} = \b_l, \tag4
\end{gather}
其中(1)与(2)同解,(3)与(4)同解.

必要性.
当\(\a_{j1},\a_{j2},\dotsc,\a_{jk}\)是\(\A\)的列极大无关组时,%
(1)仅有零解,(3)有解.于是(2)仅有零解,(4)有解,%
从而\(\b_{j1},\b_{j2},\dotsc,\b_{jk}\)线性无关,%
\(\b_l\ (l=1,2,\dotsc,m)\)可由其线性表出;
由极大无关组定义,%
\(\b_{j1},\b_{j2},\dotsc,\b_{jk}\)是\(\B\)的列极大无关组.

同理可证充分性.
\end{proof}
\end{theorem}

\begin{example}
求列向量组\[
\a_1 = \begin{bmatrix} -1 \\ 1 \\ 0 \\ 0 \end{bmatrix},
\a_2 = \begin{bmatrix} -1 \\ 2 \\ -1 \\ 1 \end{bmatrix},
\a_3 = \begin{bmatrix} 0 \\ -1 \\ 1 \\ -1 \end{bmatrix},
\a_4 = \begin{bmatrix} 1 \\ -1 \\ 2 \\ 3 \end{bmatrix},
\a_5 = \begin{bmatrix} 2 \\ -6 \\ 4 \\ 1 \end{bmatrix}
\]的秩与一个极大无关组.
\begin{solution}
对矩阵\(\A = (\AutoTuple{\a}{5})\)作初等行变换化为阶梯形矩阵:
\begin{align*}
\A &= \begin{bmatrix}
-1 & -1 & 0 & 1 & 2 \\
1 & 2 & -1 & -3 & -6 \\
0 & -1 & 1 & 2 & 4 \\
0 & 1 & -1 & 3 & 1 \\
\end{bmatrix}
\xlongrightarrow{\begin{array}{c}
(2\text{行}) \addeq 1 \times (1\text{行}) \\
(4\text{行}) \addeq (3\text{行})
\end{array}}
\begin{bmatrix}
-1 & -1 & 0 & 1 & 2 \\
0 & 1 & -1 & -2 & -4 \\
0 & -1 & 1 & 2 & 4 \\
0 & 0 & 0 & 5 & 5 \\
\end{bmatrix} \\
&\xlongrightarrow{\begin{array}{c}
(3\text{行}) \addeq (2\text{行}) \\
(4\text{行}) \diveq 5
\end{array}}
\begin{bmatrix}
-1 & -1 & 0 & 1 & 2 \\
0 & 1 & -1 & -2 & -4 \\
0 & 0 & 0 & 0 & 0 \\
0 & 0 & 0 & 1 & 1 \\
\end{bmatrix} \\
&\xlongrightarrow{\begin{array}{c} \text{交换(3行)与(4行)} \end{array}}
\begin{bmatrix}
-1 & -1 & 0 & 1 & 2 \\
0 & 1 & -1 & -2 & -4 \\
0 & 0 & 0 & 1 & 1 \\
0 & 0 & 0 & 0 & 0 \\
\end{bmatrix} = \B.
\end{align*}
若按列分块有\(\B = (\AutoTuple{\b}{5})\).
阶梯形矩阵\(\B\)有3行不为零,故\[
\rank\{\AutoTuple{\a}{5}\}=3.
\]又因为\(\B\)的非零首元分别位于1、2、4列,%
则\(\b_1,\b_2,\b_4\)是\(\B\)的一个列极大无关组,%
相应地,\(\a_1,\a_2,\a_4\)是\(\A\)的一个列极大无关组,%
即\(\{\AutoTuple{\a}{5}\}\)的极大无关组.
\end{solution}
\end{example}

\section{向量空间及其子空间的基与维数}
\begin{definition}
设\(U\)是\(K^n\)的一个子空间.
如果\(\v{\a}{r} \in U\),并且满足\begin{enumerate}
\item \(\v{\a}{r}\)线性无关,%
\item \(U\)中的每一个向量都可以由\(\v{\a}{r}\)线性表出,%
\end{enumerate}
那么称\(\v{\a}{r}\)是\(U\)的一个\DefineConcept{基}.
\end{definition}
由于基\(\v{\a}{r}\)线性无关,因此如果\(\a\)可以由\(\v{\a}{r}\)线性表出,那么表法唯一.

\begin{example}
试证:任意\(n\)维向量\(\a = (\v{k}{n})\)是向量组\(\e_1 = (1,0,\dotsc,0)\),\(\e_2 = (0,1,\dotsc,0)\),\(\e_n = (0,0,\dotsc,1)\)的一个线性组合.
\begin{proof}
由向量的线性运算规律易知,\[
(\v{k}{n})
= k_1 (1,0,\dotsc,0)
+ k_2 (0,1,\dotsc,0)
+ \dotsb
+ k_n (0,0,\dotsc,1),
\]即\[
\a = \sum\limits_{i=1}^n k_i \e_i.
\qedhere
\]
\end{proof}
\end{example}

\begin{property}
基本向量组\(\v{\e}{n}\)是\(K^n\)的一个基,称之为\(K^n\)的\DefineConcept{标准基}.
\end{property}

\begin{theorem}\label{theorem:线性方程组.向量空间1}
\(K^n\)的任一非零子空间\(U\)都有一个基.
\begin{proof}
取\(U\)中一个非零向量\(\a_1\),向量组\(\a_1\)是线性无关的.
若\(\opair{\a_1} \neq U\),则存在\(\a_2 \in U\),使得\(\a_2 \notin \opair{\a_1}\).
于是\(\a_2\)不能由\(\a_1\)线性表出,因此\(\a_1,\a_2\)线性无关.
若\(\opair{\a_1,\a_2} \neq U\),则存在\(\a_3 \in U\),使得\(\a_3 \notin \opair{\a_1,\a_2}\),从而\(\a_1,\a_2,\a_3\)线性无关.
以此类推,由于\(K^n\)的任一线性无关的向量组的向量个数不超过\(n\),因此到某一步必定终止,且有\(\opair{\v{\a}{s}} = U\),于是\(\v{\a}{s}\)是\(U\)的一个基.
\end{proof}
\end{theorem}
\cref{theorem:线性方程组.向量空间1} 的证明过程也表明,子空间\(U\)的任一线性无关的向量组都可以扩充成\(U\)的一个基.

\begin{theorem}\label{theorem:线性方程组.向量空间2}
\(K^n\)的非零子空间\(U\)的任意两个基所含向量的个数相等.
\begin{proof}
等价的线性无关的向量组含有相同个数的向量.
\end{proof}
\end{theorem}

\begin{definition}
\(K^n\)的非零子空间\(U\)的一个基所含向量的个数称为\(U\)的\DefineConcept{维数},记作\(\dim_K U\),或简记为\(\dim U\).

特别地,规定零空间的维数为0.
\end{definition}

\begin{property}
\(\dim K^n = n\).
\begin{proof}
基本向量组\(\v{\e}{n}\)是\(K^n\)的一个基.
\end{proof}
\end{property}

设\(\v{\a}{r}\)是\(K^n\)的子空间\(U\)的一个基,则\(U\)的每一个向量\(\a\)都可以由\(\v{\a}{r}\)唯一地线性表出:\[
\a = x_1 \a_1 + x_2 \a_2 + \dotsb + x_r \a_r.
\]把元组\(\opair{x_1,x_2,\dotsc,x_r}\)称为\(\a\)在基\(\v{\a}{r}\)下的\DefineConcept{坐标}.

\begin{example}
设\(\dim U = r\),证明:\(U\)中任意\(r+1\)个向量都线性相关.
\end{example}

\begin{example}
设\(\dim U = r\),证明:\(U\)中任意\(r\)个线性无关的向量都是\(U\)的一个基.
\end{example}

\begin{example}
设\(\dim U = r\),\(\v{\a}{r} \in U\).
证明:如果\(U\)中每一个向量都可以由\(\v{\a}{r}\)线性表出,那么\(\v{\a}{r}\)是\(U\)的一个基.
\end{example}

\begin{example}
设\(U\)和\(W\)都是\(K^n\)的非零子空间.
证明:如果\(U \subseteq W\),那么\(\dim U \leqslant \dim W\).
\end{example}

\begin{example}
设\(U\)和\(W\)都是\(K^n\)的非零子空间,且\(U \subseteq W\).
证明:如果\(\dim U = \dim W\),那么\(U = W\).
\end{example}

\begin{theorem}
向量组\(\v{\a}{s}\)的一个极大无关组是这个向量组生成的子空间\(\opair{\v{\a}{s}}\)的一个基,从而\begin{equation}\label{equation:线性方程组.子空间的维数与向量组的秩的联系}
\dim\opair{\v{\a}{s}} = \rank\{\v{\a}{s}\}.
\end{equation}
\end{theorem}
这里要注意区分“子空间的维数\(\dim\opair{\v{\a}{s}}\)”
和“向量组的秩\(\rank\{\v{\a}{s}\}\)”这两个概念:
维数是对子空间而言,秩是对向量组而言;
在子空间\(\opair{\v{\a}{s}}\)这个集合中有无穷多个向量,
而向量组\(\{\v{\a}{s}\}\)这个集合中只有有限的\(s\)个向量.

数域\(K\)上任一\(s \times n\)矩阵\(\A\)的列向量组\(\v{\a}{n}\)%
生成的子空间称为\(\A\)的\DefineConcept{列空间};
\(\A\)的行向量组\(\v{\g}{s}\)生成的子空间称为\(\A\)的\DefineConcept{行空间}.
易知,\(\A\)的列(行)空间的维数等于\(\A\)的列(行)向量组的秩.

\section{矩阵的秩}
\subsection{矩阵的秩的概念与性质}
\begin{definition}\label{definition:线性方程组.行秩与列秩的定义}
矩阵的\(\A\)的行向量组、列向量组的秩分别称为%
\(\A\)的\DefineConcept{行秩}(row rank)、\DefineConcept{列秩}(column rank).
\end{definition}

\begin{definition}\label{definition:线性方程组.矩阵的秩的定义}
\(\A\)的非零子式的最高阶数\(r\)称为
“矩阵\(\A\)的\DefineConcept{秩}(rank)\footnote{
这种定义方式最早是由西尔维斯特于1851年给出的.}”,
记为\(\rank\A\)、\(R(\A)\)或\(r_{\A}\).
\end{definition}

\begin{property}\label{theorem:线性方程组.矩阵的秩的性质1}
零矩阵的秩为零,即\(\rank\z = 0\).
\end{property}

\begin{property}\label{theorem:线性方程组.矩阵的秩的性质2}
设\(\A\)是\(s \times n\)矩阵,则\[
0 \leqslant \rank\A = \rank\A^T \leqslant \min\{s,n\}.
\]
\end{property}

\begin{property}\label{theorem:线性方程组.矩阵的秩的性质3}
设\(s \times n\)矩阵\(\A\)的秩为\(\rank\A = r < \min\{s,n\}\),那么对于\(\forall m > r\),总有\(m\)阶子式全为零,即\[
\A\begin{pmatrix}
\v{i}{m} \\
\v{j}{m}
\end{pmatrix} = 0,
\quad\begin{array}{c}
1 \leqslant i_1 < i_2 < \dotsb < i_m \leqslant s; \\
1 \leqslant j_1 < j_2 < \dotsb < j_m \leqslant n.
\end{array}
\]
\end{property}

\begin{definition}
设矩阵\(\A = (a_{ij})_{s \times n}\).
\begin{enumerate}
	\item 若\(\rank\A = s\),
	则称“\(\A\)为\DefineConcept{行满秩矩阵}(row full rank matrix)”.
	\item 若\(\rank\A = n\),
	则称“\(\A\)为\DefineConcept{列满秩矩阵}(column full rank matrix)”.
	\item 若\(\rank\A = s = n\),
	则称“\(\A\)是\DefineConcept{满秩矩阵}(full rank matrix)”,
	或“\(\A\)是\DefineConcept{非退化矩阵}(non-degenerate matrix)”.
\end{enumerate}
\end{definition}

应该注意到,“可逆矩阵”“非奇异矩阵”“满秩矩阵”“非退化矩阵”是从不同侧重点对同一类矩阵的四种称谓.

\begin{property}\label{theorem:线性方程组.矩阵的秩的性质4}
矩阵\(\A\)是满秩矩阵的充要条件是\(\abs{\A}\neq0\).
\end{property}

\begin{example}
求矩阵\(\A = \begin{bmatrix} 1 & 7 \\ 2 & 6 \\ -3 & 1 \end{bmatrix}\)的秩、行秩与列秩.
\begin{solution}
因为至少存在一个\(\A\)的二阶子式不为零,即\[
\begin{vmatrix} 1 & 7 \\ 2 & 6 \end{vmatrix} = -8 \neq 0,
\]且\(\A\)没有三阶子式,所以\(\rank\A = 2\).

写出\(\A\)的行向量组\[
(1,7), \qquad
(2,6), \qquad
(-3,1),
\]分别转置后,列成一个矩阵(显然就是\(\A^T\)),作初等行变换化成阶梯形矩阵\[
\A^T = \begin{bmatrix}
1 & 2 & -3 \\
7 & 6 & 1
\end{bmatrix} \to \begin{bmatrix}
1 & 0 & 17/2 \\
0 & 4 & -11
\end{bmatrix} = \B_1.
\]阶梯形矩阵\(\B_1\)有2行不为零,故矩阵\(\A\)的行秩为2.

写出\(\A\)的列向量组\[
(1,2,-3)^T,\,(7,6,1)^T,
\]列成一个矩阵(显然就是\(\A\)本身),作初等行变换化成阶梯形矩阵\[
\A = \begin{bmatrix} 1 & 7 \\ 2 & 6 \\ -3 & 1 \end{bmatrix}
\to \begin{bmatrix} 1 & 0 \\ 0 & 1 \\ 0 & 0 \end{bmatrix} = \B_2.
\]阶梯形矩阵\(\B_2\)有2行不为零,故矩阵\(\A\)的列秩为2.
\end{solution}
\end{example}

\begin{example}
设\(\a_1=(1,2,-1,0)^T,\a_2=(1,1,0,2)^T,\a_3=(2,1,1,a)^T\).
若\(\dim\opair{\v{\a}{3}}=2\),求\(a\).
\begin{solution}
除了利用\cref{equation:线性方程组.子空间的维数与向量组的秩的联系} 在将矩阵\[
\A = (\a_1,\a_2,\a_3) = \begin{bmatrix}
1 & 1 & 2 \\
2 & 1 & 1 \\
-1 & 0 & 1 \\
0 & 2 & a
\end{bmatrix}
\]化为阶梯形矩阵以后,根据\(\rank\{\v{\a}{3}\}=2\)求出\(a\)的值这种方法以外,%
我们还可以利用本节\hyperref[definition:线性方程组.矩阵的秩的定义]{矩阵的秩的定义},%
得出\(\rank\A=\dim\opair{\v{\a}{3}}=2\)这一结论,从而根据\cref{theorem:线性方程组.矩阵的秩的性质3} 可知\(\A\)中任意3阶子式全都为零.
对于\(\A\)这么一个\(4\times3\)矩阵,任意去掉不含\(a\)的一行(不妨去掉第一行)得到一个行列式必为零:\[
\begin{vmatrix}
2 & 1 & 1 \\
-1 & 0 & 1 \\
0 & 2 & a
\end{vmatrix}
= a - 6 = 0,
\]解得\(a = 6\).
\end{solution}
\end{example}

\subsection{矩阵的秩、行秩、列秩的关系}
\begin{lemma}
设矩阵\(\A = (a_{ij})_{s \times n}\)的列秩等于\(\A\)的列数\(n\),则\(\A\)行秩、秩都等于\(n\).
\begin{proof}
对\(\A\)分别进行列分块与行分块:\[
\A = (\AutoTuple{\a}{n})
= \begin{bmatrix} \A_1 \\ \A_2 \\ \vdots \\ \A_s \end{bmatrix}.
\]由于\(\A\)的列秩等于\(n\),\(\A\)的列向量组线性无关,故齐次线性方程组\[
x_1 \a_1 + x_2 \a_2 + \dotsb + x_n \a_n = \z
\quad\text{或}\quad
\A\x=\z
\]只有零解;
进而有方程个数不小于未知量个数,即\(s \geqslant n\).

因为\(\A\)的行向量的维数是\(n\),所以\(\A\)的行秩\(t \leqslant n\).
设\(\A\)的行极大无关组是\(\A_{i_1},\A_{i_2},\dotsc,\A_{i_t}\).
假设\(t < n\),则\(\A\)可通过初等行变换化为\[
\begin{bmatrix}
\A_{i_1} \\ \A_{i_2} \\ \vdots \\ \A_{i_t} \\ \z_{(s-t) \times n}
\end{bmatrix},
\]于是,\(\A\x=\z\)表示的齐次线性方程组中,%
非零方程的个数\(t\)小于未知量个数\(n\),%
故\(\A\x=\z\)有非零解,%
与\(\A\x=\z\)只有零解矛盾,%
所以假设\(t < n\)不成立,%
可得\(t = n\),\(\A\)的行秩等于\(n\).
又因为\(\A\)的行极大无关组构成\(\A\)的\(n\)阶子式\[
\begin{vmatrix} \A_{i_1} \\ \A_{i_2} \\ \vdots \\ \A_{i_n} \end{vmatrix} \neq 0,
\]
\(\A\)没有阶数大于\(n\)的子式,故秩\(\A = n\).
\end{proof}
\end{lemma}

\begin{theorem}
矩阵的行秩、列秩、秩都相等.
\begin{proof}
如果矩阵\(\A = \z\),%
则结论成立.

当\(\A \neq \z\)时,%
设\(r_{\A} = r\),%
则\(\A\)的所有\(t\ (t > r)\)阶子式全为零,%
且\(\A\)有一个\(r\)阶子式不为零,%
该子式的列向量线性无关,%
添加分量得\(\A\)有\(r\)列线性无关,%
于是\(\A\)的列秩\(p \geqslant r\);
由引理,\(\A\)的极大无关列构成的矩阵有一非零\(p\)阶子式,%
也是\(\A\)的子式,%
故\(p \leqslant r\),%
得\(p = r\).

因此\(r_{\A}\)等于\(\A\)的列秩.

同样地,将这一结论用于\(\A^T\),%
得\(r_{\A^T}\)等于\(\A^T\)的列秩,%
也就等于\(\A\)的行秩.
\end{proof}
\end{theorem}

\begin{theorem}\label{theorem:线性方程组.初等变换不变秩}
初等变换不改变矩阵的秩,即\[
\A\cong\B \implies \rank\A=\rank\B.
\]
\begin{proof}
因为初等行变换不改变矩阵的列秩;同理,初等列变换不改变矩阵的行秩;所以初等变换不改变矩阵的秩.
\end{proof}
\end{theorem}

\begin{theorem}\label{theorem:线性方程组.矩阵乘积的秩}
矩阵的乘积的秩\(r_{\A\B} \leqslant \min\{r_{\A},r_{\B}\}\).
\begin{proof}
设\(\C = \A\B\),将\(\C\)、\(\B\)分别进行列分块得\[
\C = (\AutoTuple{\g}{n}),
\qquad
\B = (\AutoTuple{\b}{n}),
\]则\[
(\AutoTuple{\g}{n}) = \A (\AutoTuple{\b}{n}) = (\A\b_1,\A\b_2,\dotsc,\A\b_n),
\]于是\[
\g_i = \A \b_i,
\quad i=1,2,\dotsc,n.
\]

若\(\rank\B = t\),%
则\(\B\)的任意\(t+1\)个列向量
\(\b_{k_1},\b_{k_2},\dotsc,\b_{k_{t+1}}\)
线性相关,%
存在不全为零的数
\(l_1,l_2,\dotsc,l_{t+1}\)
使得\[
l_1 \b_{k_1} + l_2 \b_{k_2} + \dotsb + l_{t+1} \b_{k_{t+1}} = \z,
\]
故\[
l_1 \g_{k_1} + l_2 \g_{k_2} + \dotsb + l_{t+1} \g_{k_{t+1}}
= \A(l_1 \b_{k_1} + l_2 \b_{k_2} + \dotsb + l_{t+1} \b_{k_{t+1}})
= \z,
\]
即\(\C\)的任意\(t+1\)个列向量线性相关,%
故\(\A\B\)的列秩小于或等于\(t\),%
即\(\rank(\A\B) \leqslant \rank\B\).
又\[
\rank(\A\B)
= \rank(\A\B)^T
= \rank(\B^T \A^T)
\leqslant \rank \A
= \rank \A^T,
\]所以
\(r_{\A\B}
\leqslant
\min\{r_{\A},r_{\B}\}\).
\end{proof}
\end{theorem}

\begin{corollary}
设\(\A\)是一个\(s \times n\)矩阵,\(\P\)、\(\Q\)分别是\(s\)阶和\(n\)阶可逆矩阵,则\[
r_{\A} = r_{\P \A} = r_{\A \Q} = r_{\P \A \Q}.
\]
\begin{proof}
因为\(\A = (\P^{-1} \P) \A = \P^{-1} (\P \A)\),由\cref{theorem:线性方程组.矩阵乘积的秩},有\[
r_{\A} = r_{\P^{-1} (\P \A)} \leqslant r_{\P \A} \leqslant r_{\A},
\]所以\(r_{\A} = r_{\P \A}\);同理可得\(r_{\A} = r_{\A \Q} = r_{\P \A \Q}\).
\end{proof}
\end{corollary}

\begin{theorem}
矩阵\(\A\)满足\(\rank\A=r\)的充要条件是:存在可逆矩阵\(\P,\Q\),使得\[
\P \A \Q = \begin{bmatrix}
\E_r & \z \\ \z & \z
\end{bmatrix} = \B.
\]\rm
称矩阵\(\B\)为\(\A\)的\DefineConcept{等价标准型}.
\begin{proof}
充分性.如果可逆矩阵\(\P,\Q\)使得\[
\P\A\Q = \begin{bmatrix}
\E_r & \z \\ \z & \z
\end{bmatrix},
\]
把上式等号左边的可逆矩阵\(\P\)、\(\Q\)分别视作对矩阵\(\A\)的初等行变换和初等列变换,%
那么,根据\cref{theorem:线性方程组.初等变换不变秩},%
所得矩阵\(\B\)的秩与原矩阵\(\A\)相同,%
即\[
\rank\A = \rank\B = r.
\qedhere
\]
%\cref{theorem:线性方程组.非零矩阵可经初等行变换化为若尔当阶梯形矩阵}
\end{proof}
\end{theorem}

\begin{theorem}
设\(\A\)与\(\B\)都是\(s \times n\)矩阵,则\(\A \cong \B\)的充要条件是:\(r_{\A} = r_{\B}\).
\begin{proof}
必要性.因为\(\A\)可经一系列初等变换化为\(\B\),根据\cref{theorem:线性方程组.初等变换不变秩},初等变换不改变矩阵的秩,所以\(r_{\A} = r_{\B}\).

充分性.已知\(r_{\A} = r_{\B} = r\).对\(\A\)作初等行变换可将其化简为仅前\(r\)行不为零的阶梯形矩阵\(\C\),同样对\(\C\)作初等列变换可化简为\(\A\)的等价标准型.
对\(\B\)也可作初等变换化为等价标准型.
那么存在\(s\)阶可逆矩阵\(\P_1\)和\(\P_2\),存在\(n\)阶可逆矩阵\(\Q_1\)和\(\Q_2\),使得\[
\P_1 \A \Q_1 = \P_2 \A \Q_2 = \begin{bmatrix} \E_r & \z \\ \z & \z \end{bmatrix},
\]令\(\P = \P_2^{-1} \P_1\),\(\Q = \Q_1 \Q_2^{-1}\),则\(\P\)和\(\Q\)可逆,\(\P \A \Q = \B\),从而\(\A \cong \B\).
\end{proof}
\end{theorem}

\begin{example}
试证:\[
\rank\begin{bmatrix} \A & \z \\ \z & \B \end{bmatrix} = \rank\A + \rank\B.
\]
\begin{proof}
设\(\rank\A = r_1, \rank\B = r_2\),则存在可逆矩阵\(\P_1,\Q_1,\P_2,\Q_2\),使得\[
\P_1 \A \Q_1 = \begin{bmatrix}
\E_{r_1} & \z \\
\z & \z
\end{bmatrix},
\qquad
\P_2 \B \Q_2 = \begin{bmatrix}
\E_{r_2} & \z \\
\z & \z
\end{bmatrix},
\]于是\[
\begin{bmatrix}
\P_1 & \z \\
\z & \P_2
\end{bmatrix} \begin{bmatrix}
\A & \z \\
\z & \B
\end{bmatrix} \begin{bmatrix}
\Q_1 & \z \\
\z & \Q_2
\end{bmatrix}
= \begin{bmatrix}
\P_1 \A \Q_1 & \z \\
\z & \P_2 \B \Q_2
\end{bmatrix}
= \begin{bmatrix}
\E_{r_1} \\
& \z \\
& & \E_{r_2} \\
& & & \z
\end{bmatrix}.
\]显然上式最右端矩阵的秩为\(r_1+r_2\),而最左端的矩阵\(\begin{bmatrix}
\P_1 & \z \\
\z & \P_2
\end{bmatrix}\)和\(\begin{bmatrix}
\Q_1 & \z \\
\z & \Q_2
\end{bmatrix}\)都是可逆矩阵,与之相乘不改变矩阵\(\begin{bmatrix} \A & \z \\ \z & \B \end{bmatrix}\)的秩,说明\(\rank\begin{bmatrix} \A & \z \\ \z & \B \end{bmatrix} = r_1 + r_2 = \rank\A + \rank\B\).
\end{proof}
\end{example}

\begin{example}
试证:\[
\rank\begin{bmatrix} \A & \C \\ \z & \B \end{bmatrix} \geqslant \rank\A + \rank\B,
\qquad
\rank\begin{bmatrix} \A & \z \\ \D & \B \end{bmatrix} \geqslant \rank\A + \rank\B.
\]
\end{example}

\begin{example}
设矩阵\(\A \in M_{s \times n}(P)\),矩阵\(\B \in M_{s \times m}(P)\),则\[
\rank(\A,\B) \leqslant \rank\A + \rank\B.
\]
\begin{proof}
\def\as{\AutoTuple{\a}{n}}
\def\bs{\AutoTuple{\b}{m}}
\def\asi{\a_{i_1},\dotsc,\a_{i_r}}
\def\bsj{\b_{j_1},\dotsc,\b_{j_t}}
设\(\rank\A = r\),\(\rank\B = t\).对\(\A\)、\(\B\)分别按列分块得\[
\A = (\as),
\qquad
\B = (\bs),
\]则\[
(\A,\B) = (\as,\bs),
\]且\[
\rank\{\as\} = r,
\quad
\rank\{\bs\} = t.
\]

由于\(\as\)可由其极大无关组\[
\asi
\]线性表出,\(\bs\)可由其极大无关组\[
\bsj
\]线性表出,故\[
V_1=\{\as,\bs\}
\]可由向量组\[
V_2=\{\asi,\bsj\}
\]线性表出,则\[
\rank V_1
\leqslant
\rank V_2
\leqslant
r+t;
\]
于是\(\rank(\A,\B) = \rank V_1 \leqslant r+t\).
\end{proof}
\end{example}

\begin{example}
设\(\A\)、\(\B\)都是\(s \times n\)矩阵,证明:\(\rank(\A+\B) \leqslant \rank\A + \rank\B\).
\begin{proof}
\def\asi{\a_{i_1},\a_{i_2},\dotsc,\a_{i_r}}
\def\bsj{\b_{j_1},\b_{j_2},\dotsc,\b_{j_t}}
设\(\rank\A = r\),\(\rank\B = t\).对\(\A\)、\(\B\)分别按列分块得\[
\A = (\v{\a}{n}), \qquad
\B = (\v{\b}{m}),
\]则\[
\A + \B = (\a_1 + \b_1,\a_2 + \b_2,\dotsc,\a_n + \b_n).
\]
由于\(\v{\a}{n}\)可由其极大无关组\(\asi\)线性表出,%
\(\v{\b}{m}\)可由其极大无关组\(\bsj\)线性表出,%
故\[
\a_1 + \b_1,\a_2 + \b_2,\dotsc,\a_n + \b_n
\]可由向量组\[
\asi,\bsj
\]线性表出,%
结论显然成立.
\end{proof}
\end{example}

\begin{example}
设\(\A\)是\(n\)阶矩阵,证明:\(n \leqslant \rank(\A + \E) + \rank(\A - \E)\).
\begin{proof}
可以证
\begin{align*}
\rank(\A + \E) + \rank(\A - \E)
&= \rank(\A + \E) + \rank[-(\A - \E)] \\
&\geqslant \rank\{(\A + \E) + [-(\A - \E)]\} \\
&= \rank(2\E) = \rank\E = n.
\qedhere
\end{align*}
\end{proof}
\end{example}

\begin{example}
设\(\A\)是\(s \times n\)矩阵(\(s \neq n\)),证明:\(\det(\A \A^T) \det(\A^T \A) = 0\).
\begin{proof}
由题有,\(\A^T \A\)是\(n\)阶矩阵,\(\A \A^T\)是\(s\)阶矩阵.

当\(s < n\)时,\(\rank(\A^T \A) \leqslant \min\{ \rank\A, \rank\A^T \} = \rank\A = \rank\A^T \leqslant \min\{s,n\} = s < n\),\(n\)阶方阵\(\A^T \A\)不满秩,那么\(\det(\A^T \A) = 0\).

同理,当\(n < s\)时,\(\rank(\A \A^T) \leqslant \min\{s,n\} = n < s\),那么\(\det(\A \A^T) = 0\).

综上所述,\(s \neq n\)时总有\(\det(\A \A^T) \det(\A^T \A) = 0\).
\end{proof}
\end{example}

\begin{example}
设\(\A\)是\(m \times n\)矩阵,\(\B\)是\(n \times m\)矩阵,\(\E\)是\(m\)阶单位矩阵.
已知\(\A\B = \E\),求\(\rank\A,\rank\B\).
\begin{solution}
假设\(m > n\),则由\cref{theorem:线性方程组.矩阵的秩的性质2} 可知\[
\rank\A,\rank\B \leqslant \min\{m,n\} = n.
\]再由\cref{theorem:线性方程组.矩阵乘积的秩} 可知\[
\rank(\A\B) \leqslant \min\{\rank\A,\rank\B\} \leqslant n < m.
\]但\(\rank(\A\B) = \rank\E = m\),矛盾!
由此可知,必有\(m \leqslant n\),那么\[
\rank\A,\rank\B \leqslant \min\{m,n\} = m,
\]因此\begin{align*}
m = \rank(\A\B) \leqslant \min\{\rank\A,\rank\B\} \leqslant m
&\implies
\min\{\rank\A,\rank\B\} = m \\
&\implies
m \leqslant \rank\A,\rank\B \leqslant m \\
&\implies \rank\A=\rank\B=m.
\end{align*}
\end{solution}
\end{example}

\subsection{柯西-比内公式}
\begin{theorem}
已知数域\(K\).
设矩阵\(\A \in M_{m \times n}(K),
\B \in M_{n \times m}(K)\).
那么\begin{enumerate}
\item 当\(m > n\)时,\(\abs{\A\B} = 0\).
\item 当\(m = n\)时,\(\abs{\A\B} = \abs{\A} \abs{\B}\).
\item 当\(m < n\)时,\begin{equation}\label{equation:线性方程组.柯西-比内公式}
\abs{\A\B}
= \sum\limits_{1 \leqslant i_1 < i_2 < \dotsb < i_m \leqslant n}
\MatrixMinor\A{
	1,2,\dotsc,m \\
	i_1,i_2,\dotsc,i_m
}
\MatrixMinor\B{
	i_1,i_2,\dotsc,i_m \\
	1,2,\dotsc,m
}.
\end{equation}
\rm\cref{equation:线性方程组.柯西-比内公式} 称为\DefineConcept{柯西-比内公式}.
\end{enumerate}
\begin{proof}
当\(m=n\)时,
根据\cref{theorem:行列式.矩阵乘积的行列式},
\(\abs{\A\B} = \abs{\A} \abs{\B}\)总成立.

当\(m>n\)时,
根据\cref{theorem:线性方程组.矩阵的秩的性质2},有\[\rank\A,\rank\B \leqslant \min\{m,n\} = n;\]
再根据\cref{theorem:线性方程组.矩阵乘积的秩},有\[\rank(\A\B) \leqslant \min\{\rank\A,\rank\B\} = n < m,\]
也就是说,矩阵\(\A\B\)不满秩;
那么根据\cref{theorem:线性方程组.矩阵的秩的性质4} 可知\(\abs{\A\B} = 0\).

当\(m<n\)时,
考虑\(m+n\)阶分块矩阵\[
	\begin{bmatrix}
		\E_n & \B \\
		\z & \A\B
	\end{bmatrix},
\]
由于\[
	\begin{vmatrix}
		\E_n & \B \\
		\z & \A\B
	\end{vmatrix}
	= \abs{\E_n} \abs{\A\B}
	= \abs{\A\B},
\]
所以\[
	\begin{bmatrix}
		\E_n & \B \\
		\z & \A\B
	\end{bmatrix}
	\to
	\begin{bmatrix}
		\E_n & \B \\
		-\A & \z
	\end{bmatrix}
	= \begin{bmatrix}
		\E_n & \z \\
		-\A & \E_m
	\end{bmatrix} \begin{bmatrix}
		\E_n & \B \\
		\z & \A\B
	\end{bmatrix},
\]\[
	\begin{vmatrix}
		\E_n & \B \\
		-\A & \z
	\end{vmatrix}
	= \begin{vmatrix}
		\E_n & \z \\
		-\A & \E_m
	\end{vmatrix} \begin{vmatrix}
		\E_n & \B \\
		\z & \A\B
	\end{vmatrix}
	= \begin{vmatrix}
		\E_n & \B \\
		\z & \A\B
	\end{vmatrix},
\]
根据\hyperref[theorem:行列式.拉普拉斯定理]{拉普拉斯定理},
把上式最左端行列式按后\(m\)行展开得\[
	\begin{vmatrix}
		\E_n & \B \\
		-\A & \z
	\end{vmatrix}
	= \sum\limits_{1 \leqslant i_1 < \dotsb < i_m \leqslant n}
	\MatrixMinor{(-\A)}{
		1,2,\dotsc,m \\
		i_1,i_2,\dotsc,i_m
	}
	(-1)^{[(n+1)+\dotsb+(n+m)]+(i_1+\dotsb+i_m)}
	\abs{(\e_{\mu_1},\dotsc,\e_{\mu_{n-m}},\B)},
\]
其中\(\Set{\mu_1,\dotsc,\mu_{n-m}}
= \Set{1,\dotsc,n}-\Set{i_1,\dotsc,i_s}\),
且\(\mu_1<\dotsb<\mu_{n-m}\).

把\(\abs{(\e_{\mu_1},\dotsc,\e_{\mu_{n-m}},\B)}\)
按前\(n-m\)行展开得\[
	\abs{(\e_{\mu_1},\dotsc,\e_{\mu_{n-m}},\B)}
	= \abs{\E_{n-m}}
	(-1)^{(\mu_1+\dotsb+\mu_{n-m})+[1+\dotsb+(n-m)]}
	\MatrixMinor{\B}{
		i_1,i_2,\dotsc,i_m \\
		1,2,\dotsc,m
	}.
\]
因此\begin{align*}
	\begin{vmatrix}
		\E_n & \B \\
		-\A & \z
	\end{vmatrix}
	&= \sum\limits_{1 \leqslant i_1 < \dotsb < i_m \leqslant n}
	(-1)^{m+m^2+n+n^2}
	\MatrixMinor\A{
		1,2,\dotsc,m \\
		i_1,i_2,\dotsc,i_m
	}
	\MatrixMinor\B{
		i_1,i_2,\dotsc,i_m \\
		1,2,\dotsc,m
	} \\
	&= \sum\limits_{1 \leqslant i_1 < \dotsb < i_m \leqslant n}
	\MatrixMinor\A{
		1,2,\dotsc,m \\
		i_1,i_2,\dotsc,i_m
	}
	\MatrixMinor\B{
		i_1,i_2,\dotsc,i_m \\
		1,2,\dotsc,m
	}.
\end{align*}
综上所述,\[
	\abs{\A\B}
	= \sum\limits_{1 \leqslant i_1 \leqslant i_2 \leqslant \dotsb \leqslant i_m \leqslant n}
	\MatrixMinor\A{
		1,2,\dotsc,m \\
		i_1,i_2,\dotsc,i_m
	}
	\MatrixMinor\B{
		i_1,i_2,\dotsc,i_m \\
		1,2,\dotsc,m
	}.
\]
\end{proof}
\end{theorem}

\begin{example}
计算行列式\(\det\D\),其中\[
\D = \begin{bmatrix}
1 & \cos(\alpha_1-\alpha_2) & \cos(\alpha_1-\alpha_3) & \dots & \cos(\alpha_1-\alpha_n) \\
\cos(\alpha_1-\alpha_2) & 1 & \cos(\alpha_2-\alpha_3) & \dots & \cos(\alpha_2-\alpha_n) \\
\cos(\alpha_1-\alpha_3) & \cos(\alpha_2-\alpha_3) & 1 & \dots & \cos(\alpha_3-\alpha_n) \\
\vdots & \vdots & \vdots & & \vdots \\
\cos(\alpha_1-\alpha_n) & \cos(\alpha_2-\alpha_n) & \cos(\alpha_3-\alpha_n) & \dots & 1
\end{bmatrix}.
\]
\begin{solution}
记\(\D = (d_{ij})_n\).
由\cref{equation:函数.三角函数.和积互化公式2} 可知\[
d_{ij} = \cos(\alpha_i-\alpha_j)
= \cos\alpha_i\cos\alpha_j+\sin\alpha_i\sin\alpha_j,
\quad i,j=1,2,\dotsc,n.
\]记\[
\A = \begin{bmatrix}
\cos\alpha_1 & \cos\alpha_2 & \dots & \cos\alpha_n \\
\sin\alpha_1 & \sin\alpha_2 & \dots & \sin\alpha_n
\end{bmatrix},
\]那么\(\D = \A^T \A\).

由\cref{theorem:线性方程组.矩阵的秩的性质2} 可知,%
\(\rank\A = \rank\A^T \leqslant \min\{n,2\}\).
由\cref{theorem:线性方程组.矩阵乘积的秩} 可知,\[
\rank(\A^T\A) \leqslant \min\left\{\rank\A^T,\rank\A\right\} = \rank\A.
\]

当\(n=1\)时,\(\abs{\D}=1\).

当\(n=2\)时,\[
\abs{\D} = \begin{vmatrix}
1 & \cos(\alpha_1-\alpha_2) \\
\cos(\alpha_1-\alpha_2) & 1
\end{vmatrix} = 1 - \cos^2(\alpha_1-\alpha_2).
\]

当\(n>2\)时,\(\rank(\A^T\A) \leqslant \rank\A \leqslant2\),\(\D = \A^T \A\)不满秩,故由\cref{theorem:线性方程组.矩阵的秩的性质4} 有\(\abs{\D}=0\).
\end{solution}
\end{example}

\begin{example}
设\(\A \in M_{m \times n}(\mathbb{R})\).
求证:\begin{equation}
	\rank\A = \rank(\A\A^T).
\end{equation}
\begin{proof}
设\(\rank\A=r\),%
由\hyperref[definition:线性方程组.矩阵的秩的定义]{矩阵的秩的定义}可知,
\(\A\)有\(r\)阶子式不等于零,即
\[
\MatrixMinor\A{
	\v{i}{r} \\
	\v{j}{r}
} \neq 0.
\eqno(1)
\]
由\hyperref[equation:线性方程组.柯西-比内公式]{柯西-比内公式}可知
\begin{align*}
\MatrixMinor{(\A\A^T)}{
	\v{i}{r} \\
	\v{i}{r}
}
&= \sum\limits_{1 \leqslant k_1 < k_2 < \dotsb < k_r \leqslant n}
\MatrixMinor\A{
	\v{i}{r} \\
	\v{k}{r}
}
\MatrixMinor{\A^T}{
	\v{k}{r} \\
	\v{i}{r}
} \\
&= \sum\limits_{1 \leqslant k_1 < k_2 < \dotsb < k_r \leqslant n}
\left[
	\MatrixMinor\A{
		\v{i}{r} \\
		\v{k}{r}
	}
\right]^2.
\tag2
\end{align*}
因为\(\A \in M_{m \times n}(\mathbb{R})\),%
所以\(\A\)的任意子式都是实数,于是有
\[
\left[
	\MatrixMinor\A{
		\v{i}{r} \\
		\v{k}{r}
	}
\right]^2
\geqslant 0.
\eqno(3)
\]
由(1)式可知,%
\[
\left[
	\MatrixMinor\A{
		\v{i}{r} \\
		\v{j}{r}
	}
\right]^2
> 0.
\eqno(4)
\]
由于在(2)式中,%
求和指标可以\((\v{k}{r})\)取到\((\v{j}{r})\),%
所以(4)式是(2)式中的一项.
综上所述
\[
\MatrixMinor{(\A\A^T)}{
	\v{i}{r} \\
	\v{i}{r}
}
=
\sum\limits_{1 \leqslant k_1 < k_2 < \dotsb < k_r \leqslant n}
\left[
	\MatrixMinor\A{
		\v{i}{r} \\
		\v{k}{r}
	}
\right]^2
> 0,
\]
这就说明\(\rank(\A\A^T) \geqslant r\).
又因为\(\rank(\A\A^T) \leqslant \rank\A = r\),%
所以\(\rank(\A\A^T) = \rank\A = r\).
\end{proof}
\end{example}

\begin{example}
设\(\A = (\B,\C) \in M_{n \times m}(\mathbb{R})\),其中\(\B \in M_{n \times s}(\mathbb{R})\),\(\C \in M_{n \times (m-s)}(\mathbb{R})\).
证明:\begin{equation}
\abs{\A^T \A} \leqslant \abs{\B^T \B} \abs{\C^T \C}.
\end{equation}
\end{example}

\begin{example}
设\(\A = (a_{ij})_n \in M_n(\mathbb{R})\).
证明:\begin{equation}\label{equation:线性方程组.Hadamard不等式}
\abs{\A}^2 \leqslant \prod\limits_{j=1}^n \sum\limits_{i=1}^n a_{ij}^2.
\end{equation}
\end{example}

\begin{example}
设\(\A = (a_{ij})_n \in M_n(\mathbb{R})\),且\(\abs{a_{ij}} < M\ (i,j=1,2,\dotsc,n)\).
证明:\begin{equation}
\abs{\det\A} \leqslant M^n n^{n/2}.
\end{equation}
\end{example}

\subsection{重要不等式}
\begin{theorem}[西尔维斯特不等式]
设\(\A\)是\(m \times n\)矩阵,\(\B\)是\(n \times t\)矩阵.证明:\begin{equation}\label{equation:线性方程组.西尔维斯特不等式}
\rank\A + \rank\B - n \leqslant \rank(\A \B).
\end{equation}
\begin{proof}
\def\AA{\P_1 \begin{bmatrix} \E_r & \z \\ \z & \z \end{bmatrix} \Q_1}
\def\BB{\P_2 \begin{bmatrix} \E_s & \z \\ \z & \z \end{bmatrix} \Q_2}
\def\CC#1{\C_{#1}}
设\(\rank\A = r\),\(\rank\B = s\),则存在可逆矩阵\(\P_1,\P_2,\Q_1,\Q_2\)使得\[
\A = \AA,
\quad
\B = \BB,
\]所以\[
\A \B = \AA \BB.
\]若\(\Q_1 \P_2 = \begin{bmatrix} \CC1 & \CC2 \\ \CC3 & \CC4 \end{bmatrix}\)(其中\(\CC1\)是\(r \times s\)矩阵),则\[
\A \B = \P_1 \begin{bmatrix} \CC1 & \z \\ \z & \z \end{bmatrix} \Q_2.
\]
注意到从任意一个矩阵中划去一行(或一列),\(\A\)的秩至多减少1,那么可以把\(\CC1\)看作将一个\(n\)阶方阵划去\(n-r\)行,并划去\(n-s\)列得到的,即\[
\rank\CC1 \geqslant n - (n-r) - (n-s) = r+s-n,
\]所以\(\rank\A + \rank\B - n \leqslant \rank(\A \B) = \rank\CC1\).
\end{proof}
\end{theorem}

上面的结果和\cref{theorem:线性方程组.矩阵乘积的秩} 一起可以确定矩阵乘积的秩的区间:\[
\rank\A + \rank\B - n \leqslant \rank(\A\B) \leqslant \min\{\rank\A,\rank\B\}.
\]
进一步,如果有\(\A\B=\z\),那么\[
\rank\A + \rank\B \leqslant n.
\]

我们还可以将\hyperref[equation:线性方程组.西尔维斯特不等式]{西尔维斯特不等式}进行如下的推广:\begin{equation}\label{equation:线性方程组.弗罗贝尼乌斯不等式}
\rank(\A\B\C) \geqslant \rank(\A\B) + \rank(\B\C) - \rank\B.
\end{equation}称其为\DefineConcept{弗罗贝尼乌斯不等式}.

\section{齐次线性方程组有非零解的条件及解的结构}
\begin{theorem}\label{theorem:线性方程组.齐次线性方程组有非零解的条件及解的结构}
设\(\A\)是\(s \times n\)矩阵,则齐次线性方程组\(\A\x=\z\)有非零解的充要条件为\[
r_{\A} < n.
\]
\begin{proof}
将\(\A\)按列分块得\(\A = (\v{\a}{n})\),则原方程可改写为\[
x_1 \a_1 + x_2 \a_2 + \dotsb + x_n \a_n = \z,
\]因此,原方程有非零解的充要条件是\(\v{\a}{n}\)线性相关,而其充要条件是\[
r_{\A} = \rank\{\v{\a}{n}\} < n.
\qedhere
\]
\end{proof}
\end{theorem}

上述\cref{theorem:线性方程组.齐次线性方程组有非零解的条件及解的结构}
的否命题“齐次线性方程组\(\A\x=\z\)只有零解的充要条件为\(r_{\A}=n\)”也成立.

\begin{theorem}\label{theorem:线性方程组.齐次线性方程组的解的线性组合也是解}
齐次线性方程组\(\A\x=\z\)的解的任意线性组合也是解.
\begin{proof}
设\(\X1\)与\(\X2\)是齐次线性方程组\(\A \x = \z\)的任意两个解,即\(\A \X1 = \z\),\(\A \X2 = \z\).又设\(k\)是任意常数,那么有\[
\A (\X1 + \X2) = \A \X1 + \A \X2 = \z + \z = \z,
\]\[
\A (k \X1) = k (\A \X1) = k \z = \z,
\]所以\(\X1 + \X2\)与\(k \X1\)都是\(\A \x = \z\)的解.
\end{proof}
\end{theorem}

\begin{definition}\label{definition:线性方程组.齐次线性方程组的解空间}
齐次线性方程组的解集\(V\)关于向量的加法、数乘构成一个线性空间,%
称这个线性空间为“齐次线性方程组的\DefineConcept{解空间}(space of solution)”.
\end{definition}

\begin{definition}\label{definition:线性方程组.齐次线性方程组的基础解系}
已知数域\(P\)上的齐次线性方程组\(\A\x=\z\).
如果向量组\[
X=\{\x_1,\x_2,\dotsc,\x_t\}
\]满足
\begin{enumerate}
\item \(X\)的每个向量都是\(\A\x=\z\)的解;
\item \(X\)线性无关;
\item \(\A\x=\z\)的任意一个解都可由向量组\(X\)线性表出,%
\end{enumerate}
那么称向量组\(X\)为“齐次线性方程组\(\A\x=\z\)的\DefineConcept{基础解系}(basic set of solutions)”.

\def\tongjie{ k_1\x_1+k_2\x_2+ \dotsb +k_t\x_t }
相应地,有齐次线性方程组\(\A\x=\z\)的解空间为\[
S = \Set{ \tongjie \given \v{k}{t} \in P }.
\]称\(\tongjie\)为“齐次线性方程组\(\A\x=\z\)的\DefineConcept{通解}(general solution)”.
上述基础解系\(X\)又称为“齐次线性方程组\(\A\x=\z\)的解空间的\DefineConcept{基}”.
\end{definition}

\begin{theorem}\label{theorem:线性方程组.齐次线性方程组的解向量个数}
设\(\A\)是\(s \times n\)矩阵,\(r_{\A} = r < n\),则齐次线性方程组\(\A\x=\z\)存在基础解系,且基础解系含\(n-r\)个解向量.
\begin{proof}
设\(\A\)经一系列初等行变换化为阶梯形矩阵\(\B\),则\(r_{\B} = r\),即\(\B\)的前\(r\)行向量不为零.
不失一般性,设\(\B\)的第\(i\)行非零首元为\(b_{ii}\ (i=1,2,\dotsc,r)\),\[
\A \to \B = \begin{bmatrix}
\B_1 & \B_2 \\
\z & \z
\end{bmatrix},
\]其中\[
\B_1 = \begin{bmatrix}
b_{11} & b_{12} & \dots & b_{1r} \\
& b_{22} & \dots & b_{2r} \\
& & \ddots & \vdots \\
& & & b_{rr}
\end{bmatrix},
\qquad
\B_2 = \begin{bmatrix}
b_{1,r+1} & \dots & b_{1n} \\
b_{2,r+1} & \dots & b_{2n} \\
\vdots & & \vdots \\
b_{r,r+1} & \dots & b_{rn}
\end{bmatrix}.
\]

记\[
\x = (x_1,x_2,\dotsc,x_r,x_{r+1},\dotsc,x_n)^T,
\]将自由未知量\(x_{r+1},x_{r+2},\dotsc,x_n\)的一组值\((1,0,\dotsc,0)\)代入\[
\B \x = \z,
\]去掉\(0 = 0\)的等式,移项得线性方程组\begin{gather}
\begin{bmatrix}
b_{11} & b_{12} & \dots & b_{1r} \\
& b_{22} & \dots & b_{2r} \\
& & \ddots & \vdots \\
& & & b_{rr}
\end{bmatrix}
\begin{bmatrix}
x_1 \\ x_2 \\ \vdots \\ x_r
\end{bmatrix}
=
\begin{bmatrix}
-b_{1,r+1} \\
-b_{2,r+1} \\
\vdots \\
-b_{r,r+1}
\end{bmatrix}
\tag1
\end{gather}系数行列式\(D = b_{11} b_{22} \dotsm b_{rr} \neq 0\).

由克拉默法则,(1)式有唯一解,于是得\(\A\x=\z\)的一个解\[
\X1 = (c_{11},c_{21},\dotsc,c_{r1},1,0,\dotsc,0)^T.
\]

同理,分别将\(x_{r+1},x_{r+2},\dotsc,x_n\)的值\((0,1,\dotsc,0),\dotsc,(0,0,\dotsc,1)\)代入\[
\B \x = \z,
\]求出\(\A\x=\z\)的相应的解\[
\begin{array}{rcl}
\X2 &=& (c_{12},c_{22},\dotsc,c_{r2},0,1,\dotsc,0)^T, \\
&\vdots& \\
\X{n-r} &=& (c_{1,n-r},c_{2,n-r},\dotsc,c_{r,n-r},0,0,\dotsc,1)^T.
\end{array}
\]

易见,\begin{enumerate}
\item \(\v{\x}{n-r}\)是\(\A\x=\z\)的解;

\item 考虑向量方程\(k_1 \X1 + k_2 \X2 + \dotsb + k_{n-r} \X{n-r} = \z\),即\[
(l_1,l_2,\dotsc,l_r,k_1,k_2,\dotsc,k_{n-r})^T
= (0,0,\dotsc,0,0,\dotsc,0)^T,
\]有\[
k_1 = k_2 = \dotsb = k_{n-r} = 0,
\]即\(\v{\x}{n-r}\)线性无关;

\item 设\(\x = (c_1,c_2,\dotsc,c_r,k_1,k_2,\dotsc,k_{n-r})^T\)是方程组\(\A\x=\z\)的任意一个解,则\[
\x - (k_1 \X1 + k_2 \X2 + \dotsb + k_{n-r} \X{n-r})
= (d_1,d_2,\dotsc,d_r,0,0,\dotsc,0)^T
\]是齐次方程组的解,代入\(\B\x=\z\),去掉\(0 = 0\)的等式,得\[
\begin{bmatrix}
b_{11} & b_{12} & \dots & b_{1r} \\
& b_{22} & \dots & b_{2r} \\
& & \ddots & \vdots \\
& & & b_{rr}
\end{bmatrix}
\begin{bmatrix}
d_1 \\ d_2 \\ \vdots \\ d_r
\end{bmatrix} = \begin{bmatrix}
0 \\ 0 \\ \vdots \\ 0
\end{bmatrix}.
\]因为系数行列式\(\abs{\B_1} \neq 0\),所以\(d_1 = d_2 = \dotsb = d_r = 0\).
于是\[
\x - (k_1 \X1 + k_2 \X2 + \dotsb + k_{n-r} \X{n-r}) = \z,
\]或\[
\x = k_1 \X1 + k_2 \X2 + \dotsb + k_{n-r} \X{n-r}.
\]
\end{enumerate}

综上所述,\(\v{\x}{n-r}\)是\(\A\x=\z\)的一个基础解系,含有\(n-r\)个解向量.
\end{proof}
\end{theorem}

\begin{corollary}
设齐次线性方程组\(\A\x=\z\)的系数矩阵\(\A\)是\(s \times n\)矩阵,若\(r_{\A} = r < n\),则\begin{enumerate}
\item \(\A\x=\z\)的每个基础解系都含有\(n-r\)个解向量;
\item \(\A\x=\z\)的任意\(n-r+1\)个解向量线性相关;
\item \(\A\x=\z\)的任意\(n-r\)个线性无关的解都是它的一个基础解系.
\end{enumerate}
\end{corollary}

\begin{example}
求齐次线性方程组\[
\left\{ \begin{array}{*{11}{r}}
x_1 &-& 2 x_2 &-& x_3 &+& 2 x_4 &+& 4 x_5 &=& 0 \\
2 x_1 &-& 2 x_2 &-& 3 x_3 && &+& 2 x_5 &=& 0 \\
4 x_1 &-& 2 x_2 &-& 7 x_3 &-& 4 x_4 &-& 2 x_5 &=& 0
\end{array} \right.
\]的通解.
\begin{solution}
写出系数矩阵\(\A\),并作初等行变换化简\begin{align*}
\A &= \begin{bmatrix}
1 & -2 & -1 & 2 & 4 \\
2 & -2 & -3 & 0 & 2 \\
4 & -2 & -7 & -4 & -2
\end{bmatrix} \\
&\xlongrightarrow{\begin{array}{c}
	-2\times\text{(1行)}+\text{(2行)} \\
	-4\times\text{(1行)}+\text{(3行)}
\end{array}} \begin{bmatrix}
1 & -2 & -1 & 2 & 4 \\
0 & 2 & -1 & -4 & -6 \\
0 & 6 & -3 & -12 & -18
\end{bmatrix} \\
&\xlongrightarrow{\begin{array}{c}
	-3\times\text{(2行)}+\text{(3行)} \\
	1\times\text{(2行)}+\text{(1行)}
\end{array}} \begin{bmatrix}
1 & 0 & -2 & -2 & -2 \\
0 & 2 & -1 & -4 & -6 \\
0 & 0 & 0 & 0 & 0
\end{bmatrix} = \B,
\end{align*}
因为\(\rank\A=\rank\B=2\),所以基础解系含\(5-2=3\)个向量.
分别将\(x_3,x_4,x_5\)的3组值\((2,0,0),(0,1,0),(0,0,1)\)代入\(\B\x=\z\),得基础解系:\[
\X1 = (4,1,2,0,0)^T, \quad
\X2 = (2,2,0,1,0)^T, \quad
\X3 = (2,3,0,0,1)^T.
\]

原方程组的通解为\(k_1 \X1 + k_2 \X2 + k_3 \X3\),其中\(k_1,k_2,k_3\)为任意常数.
\end{solution}
\end{example}

\cref{theorem:线性方程组.齐次线性方程组的解向量个数} 揭示矩阵\(\A\)的秩与方程\(\A\x=\z\)的基础解系所含向量个数的关系.
它不仅对\(\A\x=\z\)的求解有重要意义,而且可以用来解决矩阵的秩的一些问题.
\begin{example}
设\(\A\)为\(s \times n\)矩阵,\(\B\)为\(n \times m\)矩阵,\(\A\B=\z\).证明:\(r_{\A} + r_{\B} \leqslant n\).
\begin{proof}
设矩阵\(\B\)与\(\A\B=\z\)右端的零矩阵列分块矩阵分别为\[
\B=(\AutoTuple{\b}{m}),
\quad
\z=(\z,\dotsc,\z)
\]那么\[
\A(\AutoTuple{\b}{m})
= (\A\b_1,\A\b_2,\dotsc,\A\b_m)
= (\z,\z,\dotsc,\z)
\]或\[
\A\b_j = \z \quad(j=1,2,\dotsc,m)
\]即\(\beta=\{\AutoTuple{\b}{m}\}\)是齐次线性方程组\(\A\x=\z\)解向量组.

若\(r_{\A}=n\),则\(\A\x=\z\)只有零解,\(\B=\z\),\(r_{\B}=0=n-r_{\A}\).

若\(r_{\A}<n\),则\(X=\{\v{\x}{n-r}\}\)是\(\A\x=\z\)的一个基础解系,则\(\beta\)可由\(X\)线性表出,\(r_{\beta} \leqslant r_{X}\).
而\(r_{\beta}=r_{\B}\),\(r_{X}=n-r_{\A}\).

综上所述,\(r_{\A}+r_{\B} \leqslant n\)成立.
\end{proof}
\end{example}

\begin{example}
设\(\A\)为\(n\ (n>1)\)阶方阵,\(\A^*\)是\(\A\)的伴随矩阵.
证明:\[
r_{\A^*} = \left\{ \begin{array}{cl}
n, & r_{\A}=n, \\
1, & r_{\A}=n-1, \\
0, & r_{\A}<n-1.
\end{array} \right.
\]
\begin{proof}
我们根据矩阵\(\A\)的秩的取值分类讨论.
\begin{enumerate}
\item 当\(r_{\A} = n\)时,\(\A\)满秩,%
那么由\cref{theorem:线性方程组.矩阵的秩的性质4} 有\(\abs{\A} \neq 0\);%
根据恒等式 \labelcref{equation:行列式.伴随矩阵.恒等式1} 有\[
\A \A^* = \A^* \A = \abs{\A} \E;
\]
根据\cref{theorem:行列式.矩阵乘积的行列式} 有\[
\abs{\A \A^*} = \abs{\A} \abs{\A^*};
\]
根据\cref{theorem:行列式.性质2.推论2} 有\[
\abs{\abs{\A} \E} = \abs{\A}^n \abs{\E} = \abs{\A}^n;
\]
于是\[
\abs{\A^*}
= \frac{\abs{\A \A^*}}{\abs{\A}}
= \frac{\abs{\abs{\A} \E}}{\abs{\A}}
= \frac{\abs{\A}^n}{\abs{\A}}
= \abs{\A}^{n-1} \neq 0,
\]
进而有\[
r_{\A^*} = n.
\]

\item 当\(r_{\A} = n-1 < n\)时,%
\(\abs{\A} = 0\),\(\A \A^* = \z\),据上例,\[
r_{\A} + r_{\A^*} \leqslant n
\implies
r_{\A^*}
\leqslant n - r_{\A}
= n - (n-1)
= 1.
\]
又因为\(n > 1 \implies r_{\A} = n-1 > 0\),%
所以\(r_{\A} = 1\).

\item 当\(r_{\A} < n-1\)时,%
\(\A\)的所有\(n-1\)阶子式全为零,%
\(\A\)的任意元素代数余子式为零,%
\(\A^* = \z\),\(r_{\A^*} = 0\).
\end{enumerate}
\end{proof}
\end{example}

\begin{example}
设\(n\)阶矩阵\(\A\)满足\(\A^2 - 3 \A - 10 \E = \z\),证明:\[
\rank(\A - 5\E) + \rank(\A + 2\E) = n.
\]
\begin{proof}
因为\(\A\)满足\(\A^2 - 3 \A - 10 \E = (\A - 5\E)(\A + 2\E) = \z\),所以\[
\rank(\A - 5\E) + \rank(\A + 2\E) \leqslant n.
\]又因为\begin{align*}
&\rank(\A - 5\E) + \rank(\A + 2\E) \\
&\geqslant \max\{
	\rank[(\A - 5\E) + (\A + 2\E)],
	\rank[(5\E - \A) + (\A + 2\E)]
	\} \\
&= \max\{ \rank(2\A - 3\E),\rank(7\E) \}
= n,
\end{align*}所以\(\rank(\A - 5\E) + \rank(\A + 2\E) = n\).
\end{proof}
\end{example}

\begin{example}
已知向量\(\a_1 = \begin{bmatrix}
1 \\ 2 \\ 3
\end{bmatrix},
\a_2 = \begin{bmatrix}
2 \\ 1 \\ 1
\end{bmatrix},
\b_1 = \begin{bmatrix}
2 \\ 5 \\ 9
\end{bmatrix},
\b_2 = \begin{bmatrix}
1 \\ 0 \\ 1
\end{bmatrix}\).
若向量\(\g\)既可由\(\a_1,\a_2\)线性表出,也可由\(\b_1,\b_2\)线性表出,求\(\g\).
\begin{solution}
设\(\g = x_1 \a_1 + x_2 \a_2 = x_3 \b_1 + x_4 \b_2\),则\(x_1 \a_1 + x_2 \a_2 - x_3 \b_1 - x_4 \b_2 = \z\).
又由于\[
(\a_1,\a_2,-\b_1,-\b_2) = \begin{bmatrix}
1 & 2 & -2 & -1 \\
2 & 1 & -5 & 0 \\
3 & 1 & -9 & -1
\end{bmatrix} \to \begin{bmatrix}
1 & 0 & 0 & 3 \\
0 & 1 & 0 & -1 \\
0 & 0 & 1 & 1
\end{bmatrix},
\]解得\((x_1,x_2,x_3,x_4)^T = c (-3,1,-1,1)^T\ (c\in\mathbb{R})\).
因此\[
\g = c (x_3 \b_1 + x_4 \b_2)
= c \begin{bmatrix}
-1 \\ -5 \\ -8
\end{bmatrix}
= k \begin{bmatrix}
1 \\ 5 \\ 8
\end{bmatrix}
\quad(k\in\mathbb{R}).
\]
\end{solution}
\end{example}

\section{非齐次线性方程组有解的条件及解的结构}
\begin{definition}
与非齐次线性方程组\(\A\x=\b\)的系数矩阵相同的齐次线性方程组\(\A\x=\z\)称为前者的\DefineConcept{导出组}.
\end{definition}

\begin{theorem}\label{theorem:线性方程组.非齐次线性方程组有解的条件及解的结构}
已知数域\(P\),矩阵\(\A \in P^{s \times n}\).

对于线性方程组\(\A\x=\b\),其增广矩阵记为\(\widetilde{\A} = (\A,\b)\),则
\begin{enumerate}
\item \(\A\x=\b\ \text{有解} \iff \rank\A=\rank\widetilde{\A}\);

\item \(\rank\A=\rank\widetilde{\A}=n\)时,非齐次线性方程组有唯一解;

\item \(\rank\A=\rank\widetilde{\A}<n\)时,非齐次线性方程组有无穷多解.
设\[
\x_1,\x_2,\dotsc,\x_{n-r}
\]是导出组\(\A\x=\z\)的一个基础解系,那么通解为\[
\x_0+k_1\x_1+k_2\x_2+ \dotsb +k_{n-r}\x_{n-r},
\quad
k_1,k_2,\dotsc,k_{n-r} \in P,
\]其中\(\x_0\)是非齐次线性方程组的一个解,称为\DefineConcept{特解}.
\end{enumerate}
\end{theorem}

一般地,对\(\A\x=\b\)的增广矩阵\[
\widetilde{\A}=(\A,\b)
\]施行一系列初等行变换化为阶梯形矩阵\[
\widetilde{\B}=(\B,\g),
\]则\(\A\x=\b\)与\(\B\x=\g\)同解,\(\A\x=\z\)与\(\B\x=\z\)同解.

于是\(\A\x=\b\)有解的充要条件是\(\rank\B=\rank\widetilde{\B}\).

有解时,求出\(\B\x=\g\)的一个特解和\(\B\x=\z\)的一个基础解系,就可得出\(\A\x=\b\)的通解.

\begin{example}
求线性方程组\[
\left\{ \begin{array}{*{9}{r}}
x_1 &-& 2 x_2 &-& x_3 &+& 2 x_4 &=& 4 \\
2 x_1 &-& 2 x_2 &-& 3 x_3 && &=& 2 \\
4 x_1 &-& 2 x_2 &-& 7 x_3 &-& 4 x_4 &=& -2
\end{array} \right.
\]的通解.
\begin{solution}
写出增广矩阵\(\widetilde{\A}\),并作初等行变换化简\begin{align*}
\widetilde{\A}
&= \begin{bmatrix}
1 & -2 & -1 & 2 & 4 \\
2 & -2 & -3 & 0 & 2 \\
4 & -2 & -7 & -4 & -2
\end{bmatrix}
\xlongrightarrow{\begin{array}{c}
	-2\times\text{(1行)}+\text{(2行)} \\
	-4\times\text{(1行)}+\text{(3行)}
\end{array}}
\begin{bmatrix}
1 & -2 & -1 & 2 & 4 \\
0 & 2 & -1 & -4 & -6 \\
0 & 6 & -3 & -12 & -18
\end{bmatrix} \\
&\to \begin{bmatrix}
1 & -2 & -1 & 2 & 4 \\
0 & 2 & -1 & -4 & -6 \\
0 & 0 & 0 & 0 & 0
\end{bmatrix}
\to \begin{bmatrix}
1 & 0 & -2 & -2 & -2 \\
0 & 2 & -2 & -4 & -6 \\
0 & 0 & 0 & 0 & 0
\end{bmatrix}
= \widetilde{\B}
= (\B,\g),
\end{align*}
故\(\rank\A = \rank\widetilde{\A} = 2\),于是原方程组有解.

解同解方程组\[
\left\{ \begin{array}{*{9}{c}}
x_1 && &-& 2 x_3 &-& 2 x_4 &=& -2 \\
&& 2 x_2 &-& x_3 &-& 4 x_4 &=& -6
\end{array} \right..
\]令\(x_3 = 0\),\(x_4 = 0\),解得\(x_1 = -2\),\(x_2 = -3\),得特解\[
\x_0 = (-2,-3,0,0)^T.
\]

又因为导出组的基础解系含\(4 - r_{\A} = 2\)个向量.
将\(x_3,x_4\)的两组值\((2,0),(0,1)\)分别代入\[
\left\{ \begin{array}{*{9}{c}}
x_1 && &-& 2 x_3 &-& 2 x_4 &=& 0 \\
&& 2 x_2 &-& x_3 &-& 4 x_4 &=& 0
\end{array} \right.
\]得基础解系\(\x_1 = (4,1,2,0)^T\),\(\x_2 = (2,2,0,1)^T\).

于是原方程的通解为\[
\x = \x_0 + k_1 \x_1 + k_2 \x_2
= \begin{bmatrix} -2 \\ -3 \\ 0 \\ 0 \end{bmatrix}
+ k_1 \begin{bmatrix} 4 \\ 1 \\ 2 \\ 0 \end{bmatrix}
+ k_2 \begin{bmatrix} 2 \\ 2 \\ 0 \\ 1 \end{bmatrix},
\]其中\(k_1,k_2\)是任意常数.
\end{solution}
\end{example}

\begin{example}
写出线性方程组\[
\left\{ \begin{array}{l}
x_1 - x_2 = a_1 \\
x_2 - x_3 = a_2 \\
x_3 - x_4 = a_3 \\
x_4 - x_1 = a_4
\end{array} \right.
\]有解的充要条件,并求解.
\begin{solution}
对增广矩阵\(\widetilde{\A}\)作初等行变换化简\begin{align*}
\widetilde{\A}
&= \begin{bmatrix}
1 & -1 & 0 & 0 & a_1 \\
0 & 1 & -1 & 0 & a_2 \\
0 & 0 & 1 & -1 & a_3 \\
-1 & 0 & 0 & 0 & a_4
\end{bmatrix} \to \begin{bmatrix}
1 & -1 & 0 & 0 & a_1 \\
0 & 1 & -1 & 0 & a_2 \\
0 & 0 & 1 & -1 & a_3 \\
0 & 0 & 0 & 0 & \sum\limits_{i=1}^4 a_i
\end{bmatrix} \\
&\to \begin{bmatrix}
1 & 0 & 0 & -1 & a_1 + a_2 + a_3 \\
0 & 1 & 0 & -1 & a_2 + a_3 \\
0 & 0 & 1 & -1 & a_3 \\
0 & 0 & 0 & 0 & \sum\limits_{i=1}^4 a_i
\end{bmatrix}.
\end{align*}
可见\(\rank\A = 3\).
方程组有解的充要条件是\(\rank\widetilde{\A} = \rank\A = 3\),那么充要条件就是\(\sum\limits_{i=1}^4 a_i = 0\)

当方程组有解时,通解为\[
\begin{bmatrix}
a_1 + a_2 + a_3 \\ a_2 + a_3 \\ a_3 \\ 0
\end{bmatrix} + k \begin{bmatrix}
1 \\ 1 \\ 1 \\ 1
\end{bmatrix},
\]其中\(k\)为任意常数.
\end{solution}
\end{example}

至此,我们讨论了线性方程组的解的存在性、解的性质、解的结构及求解方法,建立起了线性方程组的完整理论.
解线性方程组是线性代数的基本问题之一,现代科学技术方面用到的数学问题也有很多要归结到解线性方程组.

\begin{example}
设\(\X0\)是非齐次线性方程组\(\A\x=\b\)的一个解,\(\v{\x}{n-r}\)是其导出组\(\A\x=\z\)的一个基础解系.证明:\(\X0,\v{\x}{n-r}\)线性无关.
\begin{proof}
因为\(\v{\x}{n-r}\)是其导出组\(\A\x=\z\)的一个基础解系,根据基础解系的定义,显然有\(\v{\x}{n-r}\)线性无关.
假设\(\X0,\v{\x}{n-r}\)线性相关,那么\(\X0\)可由\(\v{\x}{n-r}\)线性表出,即存在数\(k_1,k_2,\dotsc,k_{n-r}\)使得\[
\X0 = k_1 \X1 + k_2 \X2 + \dotsb + k_{n-r} \X{n-r},
\]进而有\begin{align*}
&\A\X0 = \A(k_1 \X1 + k_2 \X2 + \dotsb + k_{n-r} \X{n-r}) \\
&= k_1 \A\X1 + k_2 \A\X2 + \dotsb + k_{n-r} \A\X{n-r}
= \z + \z + \dotsb + \z = \z,
\end{align*}即\(\X0\)是\(\A\x=\z\)的一个解,这与\(\X0\)是\(\A\x=\b\neq\z\)的一个解矛盾,所以\(\X0,\v{\x}{n-r}\)线性无关.
\end{proof}
\end{example}

\begin{example}
\def\wA{\widetilde{\A}}
设线性方程组\(\A\x=\b\)的增广矩阵\(\wA = (\A,\b)\)是一个\(n\)阶可逆矩阵,证明:方程组无解.
\begin{proof}
因为\(\wA\)是\(n\)阶方阵,所以\(\A\)是\(n \times (n-1)\)矩阵,从而\(\rank\A \leqslant \min\{n-1,n\} = n-1\).
又因为\(\wA\)可逆,所以\(\rank\wA = n\).
因为\(\rank\wA = n > n-1 \geqslant \rank\A\),所以方程组\(\A\x=\b\)无解.
\end{proof}
\end{example}
