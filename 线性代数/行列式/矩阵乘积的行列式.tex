\section{矩阵乘积的行列式}
\begin{lemma}
设\(\A\)为\(n\)阶非零方阵,\(\B\)为\(m\)阶非零方阵,
\(\C\)为\(m \times n\)矩阵,\(\D\)为\(n \times m\)矩阵,
则\begin{align}
	\begin{vmatrix}
		\A & \z \\
		\C & \B
	\end{vmatrix}
	&= \begin{vmatrix}
		\A & \D \\
		\z & \B
	\end{vmatrix}
	= \abs{\A} \abs{\B}, \label{equation:行列式.广义三角阵的行列式1} \\
	\begin{vmatrix}
		\z & \A \\
		\B & \C
	\end{vmatrix}
	&= \begin{vmatrix}
		\D & \A \\
		\B & \z
	\end{vmatrix}
	= (-1)^{mn} \abs{\A} \abs{\B}. \label{equation:行列式.广义三角阵的行列式2}
\end{align}
\end{lemma}

\begin{theorem}[矩阵乘积的行列式定理]\label{theorem:行列式.矩阵乘积的行列式}
设\(\A\)、\(\B\)都是\(n\)阶矩阵,则\begin{equation}
\abs{\A\B} = \abs{\A} \abs{\B}.
\end{equation}
\end{theorem}

\begin{example}
设\(\A\)是\(n\)阶矩阵,且\(\A\A^T = \E\),\(\abs{\A}<0\).证明:\(\abs{\E+\A}=0\).
\begin{proof}
等式\(\A\A^T=\E\)的两端取行列式,\(\abs{\A} \abs{\A^T} = \abs{\E}\),由\(\abs{\A} = \abs{\A^T}\)与\(\abs{\A} < 0\),得\(\abs{\A}^2 = 1\),\(\abs{\A} = -1\).故\begin{align*}
\abs{\E+\A}
&= \abs{\A\A^T+\A}
= \abs{\A(\A^T+\E)}
= \abs{\A} \abs{\A^T+\E} \\
&= -\abs{(\A^T+\E)^T}
= -\abs{\A+\E},
\end{align*}所以\(\abs{\E+\A} = -\abs{\E+\A}\),\(\abs{\E+\A}=0\).
\end{proof}
\end{example}

\begin{example}
设\(\A^*\)是\(n\)阶矩阵\(\A\)的伴随矩阵,若\(\abs{\A} \neq 0\),证明:\(\abs{\A^*} \neq 0\)且\(\abs{\A^*}=\abs{\A}^{n-1}\).
\begin{proof}
因为\(\A \A^* = \abs{\A} \E\),%
所以\(\abs{\A \A^*} = \abs{\abs{\A} \E} = \abs{\A}^n \abs{\E} = \abs{\A}^n\).
又因为\(\abs{\A \A^*} = \abs{\A} \abs{\A^*}\),%
所以\(\abs{\A^*} = \abs{\A}^{n-1} \neq 0\).
\end{proof}
\end{example}

\begin{example}
设矩阵\(\A,\B\)满足\(\abs{\A}=3,\abs{\B}=2,\abs{\A^{-1}+\B}=2\),试求\(\abs{\A+\B^{-1}}\).
\begin{solution}
由于\(\abs{\E+\A\B} = \abs{\A(\A^{-1}+\B)} = \abs{\A} \abs{\A^{-1}+\B} = 6\),所以\[
\abs{\A+\B^{-1}}
= \frac{\abs{(\A+\B^{-1})\B}}{\abs{\B}}
= \frac{\abs{\A\B+\E}}{\abs{\B}}
= 3.
\]
\end{solution}
\end{example}

\begin{example}
用\(\abs{\A}^2 = \abs{\A} \abs{\A^T}\)的方法计算行列式\[
\abs{\A} = \begin{vmatrix}
a & b & c & d \\
-b & a & d & -c \\
-c & -d & a & b \\
-d & c & -b & a
\end{vmatrix}.
\]
\begin{solution}
因为\begin{align*}
\abs{\A}^2 &= \abs{\A} \abs{\A^T} = \abs{\A \A^T} \\
&= \abs{\begin{bmatrix}
a & b & c & d \\
-b & a & d & -c \\
-c & -d & a & b \\
-d & c & -b & a
\end{bmatrix} \begin{bmatrix}
a & -b & -c & -d \\
b & a & -d & c \\
c & d & a & -b \\
d & -c & b & a
\end{bmatrix}} \\
&= \abs{(a^2+b^2+c^2+d^2) \E}_4
= (a^2+b^2+c^2+d^2)^4,
\end{align*}所以\(\abs{\A} = \pm(a^2+b^2+c^2+d^2)^2\),再由\(\abs{\A}\)中含有项\(a^4\),得\[
\abs{\A} = (a^2+b^2+c^2+d^2)^2.
\]
\end{solution}
\end{example}

\begin{example}
计算:\[
	D = \begin{vmatrix}
		a & a & a & a \\
		a & a & -a & -a \\
		a & -a & a & -a \\
		a & -a & -a & a
	\end{vmatrix}.
\]
\begin{solution}
因为\[
	\begin{bmatrix}
		1 & 0 & 0 & 0 \\
		0 & 0 & 1 & 0 \\
		0 & 1 & 0 & 0 \\
		0 & 0 & 0 & 1
	\end{bmatrix} \begin{bmatrix}
		a & a & a & a \\
		a & a & -a & -a \\
		a & -a & a & -a \\
		a & -a & -a & a
	\end{bmatrix}
	= \begin{bmatrix}
		1 & 0 & 0 & 0 \\
		1 & 1 & 0 & 0 \\
		1 & 0 & 1 & 0 \\
		1 & 1 & 1 & 1
	\end{bmatrix} \begin{bmatrix}
		a & a & a & a \\
		0 & -2 a & 0 & -2 a \\
		0 & 0 & -2 a & -2 a \\
		0 & 0 & 0 & 4 a
	\end{bmatrix},
\]而\[
	\begin{vmatrix}
		1 & 0 & 0 & 0 \\
		0 & 0 & 1 & 0 \\
		0 & 1 & 0 & 0 \\
		0 & 0 & 0 & 1
	\end{vmatrix} = -1,
	\qquad
	\begin{vmatrix}
		1 & 0 & 0 & 0 \\
		1 & 1 & 0 & 0 \\
		1 & 0 & 1 & 0 \\
		1 & 1 & 1 & 1
	\end{vmatrix} = 1,
\]\[
	\begin{vmatrix}
		a & a & a & a \\
		0 & -2 a & 0 & -2 a \\
		0 & 0 & -2 a & -2 a \\
		0 & 0 & 0 & 4 a
	\end{vmatrix}
	= a\cdot(-2a)\cdot(-2a)\cdot(4a) = 16a^2,
\]
所以\[
	\begin{vmatrix}
		a & a & a & a \\
		a & a & -a & -a \\
		a & -a & a & -a \\
		a & -a & -a & a
	\end{vmatrix}
	= -16a^2.
\]
\end{solution}
\end{example}

\begin{example}
设\(s_k = a_1^k + a_2^k + a_3^k + a_4^k\ (k=1,2,3,4,5,6)\).
计算:\[
	D = \begin{vmatrix}
		4 & s_1 & s_2 & s_3 \\
		s_1 & s_2 & s_3 & s_4 \\
		s_2 & s_3 & s_4 & s_5 \\
		s_3 & s_4 & s_5 & s_6
	\end{vmatrix}.
\]
\begin{solution}
令矩阵\[
	\A = \begin{bmatrix}
		1 & 1 & 1 & 1 \\
		a_1 & a_2 & a_3 & a_4 \\
		a_1^2 & a_2^2 & a_3^2 & a_4^2 \\
		a_1^3 & a_2^3 & a_3^3 & a_4^3
	\end{bmatrix},
\]
显然\[
	D = \det(\A\A^T) = \abs{\A}^2.
\]
而根据\cref{equation:行列式.范德蒙德行列式},
\(\abs{\A}
= \prod\limits_{1 \leq j < i \leq n} (a_i - a_j)\),
故\(D = \prod\limits_{1 \leq j < i \leq n} (a_i - a_j)^2\).
\end{solution}
\end{example}
