\section{高斯--若尔当消元法}
\begin{definition}
在数域\(P\)上,设矩阵\(\A \in P^{s \times n}\),
向量\(\x \in P^n\),
\(\b \in P^s\).
由\(n\)元线性方程组\[
	\A \x = \b
\]的系数和常数项按原位置构成的\(s \times (n+1)\)矩阵\((\A,\b)\),
称为该\(n\)元线性方程组的\DefineConcept{增广矩阵}(augmented matrix),
记作\(\wA\).
\end{definition}

\begin{theorem}
对线性方程组\(\A \x = \b\)的增广矩阵\(\wA = (\A,\b)\)作初等行变换,
变为\(\widetilde{\C} = (\C,\g)\),
则相应的线性方程组\(\C \x = \g\)与原线性方程组同解.
\begin{proof}
显然存在可逆矩阵\(\P\),
使得\(\wA \to \widetilde{\C} = \P \wA = (\P\A,\P\b)\),
\(\C = \P\A\),\(\g = \P\b\).
如果\(\X0\)是原线性方程组的解,即\(\A \X0 = \b\),
用\(\P\)左乘等式两端得\(\C \X0 = \g\);
反之,若\(\X0\)满足\(\C \X0 = \g\),
用\(\P^{-1}\)左乘等式两端得\(\A \X0 = \b\),
故两方程组同解.
\end{proof}
\end{theorem}
这就是消元法解线性方程组的理论根据.
具体化简\(\wA\)时,
可用一系列初等行变换将其变成一个较为简单的“阶梯形矩阵”(或更简单的“若尔当阶梯形矩阵”).

\begin{definition}
称如下形式的\(s \times n\)矩阵\[
	\A = \begin{bmatrix}
		0 & \dots & a_{1 j_1} & \dots & a_{1 j_2} & \dots & a_{1 j_r} & \dots & a_{1n} \\
		0 & \dots & 0 & \dots & a_{2 j_2} & \dots & a_{2 j_r} & \dots & a_{2n} \\
		\vdots & & \vdots & & \vdots & & \vdots & & \vdots \\
		0 & \dots & 0 & \dots & 0 & \dots & a_{r j_r} & \dots & a_{rn} \\
		0 & \dots & 0 & \dots & 0 & \dots & 0 & \dots & 0 \\
		\vdots & & \vdots & & \vdots & & \vdots & & \vdots \\
		0 & \dots & 0 & \dots & 0 & \dots & 0 & \dots & 0 \\
	\end{bmatrix}
\]为\DefineConcept{阶梯形矩阵}(ladder matrix),
其中\(a_{1 j_1},a_{2 j_2},\dotsc,a_{r j_r}\)均不为零,
\(j_1 < j_2 < \dotsb < j_r\),
\(\A\)的后\(s-r\)行全为零.

元素全为\(0\)的行,称为\DefineConcept{零行}.
元素不全为\(0\)的行,称为\DefineConcept{非零行}.

在非零行中,从左数起,第一个不为\(0\)的元素\(a_{i j_i}\ (i=1,2,\dotsc,r)\),
称为\DefineConcept{主元}(pivot)或\DefineConcept{非零首元}.

以主元为系数的未知量\(x_{j_1},x_{j_2},\dotsc,x_{j_r}\),
称为\DefineConcept{主变量};
不以主元为系数的未知量,称为\DefineConcept{自由未知量}.
\end{definition}

\begin{definition}
若阶梯形矩阵\(\A\)的非零行的非零首元全为\(1\),
它们所在列的其余元素全为零,
则称\(\A\)为\DefineConcept{若尔当阶梯形矩阵}或\DefineConcept{行约化矩阵}.
\end{definition}
% 在Mathematica中可以用RowReduce对矩阵进行初等行变换化为若尔当阶梯形矩阵.

\begin{lemma}
任何一个非零矩阵都可经初等行变换化为阶梯形矩阵.
\begin{proof}
设\(\A_{s \times n} \neq \z\),则\(\A\)经0次或1次交换两行的变换化为\(\B\),即\[
	\A \to \B = \begin{bmatrix}
		0 & \dots & 0 & b_{1 j_1} & \dots & b_{1n} \\
		0 & \dots & 0 & b_{2 j_1} & \dots & b_{2n} \\
		\vdots & & \vdots & \vdots & & \vdots \\
		0 & \dots & 0 & b_{s j_1} & \dots & b_{sn}
	\end{bmatrix},
\]
其中\(b_{1 j_1} \neq 0\).
分别将\(\B\)的第一行的\(-b_{i j_1}/b_{1 j_1}\)倍加到第\(i\ (i=2,3,\dotsc,s)\)行,则\[
	\B \to \C = \begin{bmatrix}
		\z & b_{1 j_1} & \B_1 \\
		\z & \z & \C_1
	\end{bmatrix},
\]
其中\(\C_1\)是\((s-1)\times(n-j_1)\)矩阵,
再对\(\C\)的后面\(s-1\)行作类似的初等行变换化简.
因为矩阵行数有限,这样下去,最后总可化为阶梯形矩阵.
\end{proof}
\end{lemma}

\begin{corollary}\label{theorem:线性方程组.非零矩阵可经初等行变换化为若尔当阶梯形矩阵}
任何一个非零矩阵都可经初等行变换化为若尔当阶梯形矩阵.
\end{corollary}

\begin{theorem}
设\(\A\)为\(n\)阶方阵,
则齐次线性方程组\(\A\x = \z\)有非零解的充要条件是:
\(\abs{\A} = 0\).
\begin{proof}
必要性.
给定\(\X0 \neq \z\)满足\(\A \X0 = \z\).
假设\(\abs{\A} \neq 0\),
由克拉默法则可知方程组有唯一解\(\X0 = \A^{-1} \z = \z\),
矛盾,故\(\abs{\A} = 0\).

充分性.
用数学归纳法证明.
给定\(\abs{\A} = 0\).
当\(n=1\)时,\(\A = \z_{1 \times 1} = 0\),
\(0 \cdot x_1 = 0\)有非零解;
假设当\(n=k-1\geq1\)时,结论成立;
那么当\(n=k\)时,
设\(\A\)经初等行变换\(\P\)化为阶梯形矩阵\[
	\B = \begin{bmatrix}
		b & \g \\
		\z & \C \\
	\end{bmatrix} = \P \A,
\]
其中\(\C\)是\(n-1\)阶方阵,\(\P\)是\(n\)阶可逆矩阵.
取行列式得
\[
	\abs{\B} = \abs{\P} \abs{\A} = 0 = b \abs{\C}.
\]
解同解方程组\(\B \x = \z\).
若\(b = 0\)则\((1,0,\dotsc,0)^T\)是一个非零解;
若\(b \neq 0\),则\(\abs{\C} = 0\),由归纳假设,齐次线性方程组\[
	\C \begin{bmatrix} x_2 \\ x_3 \\ \vdots \\ x_n \end{bmatrix} = \z
\]有非零解\((k_2,k_3,\dotsc,k_n)^T\),
代入\(\B \x = \z\)的第一个方程,
因为\(x_1\)的系数\(b \neq 0\),可解出\(x_1\).
于是\((\AutoTuple{k}{n})^T\)是\(\A \x = \z\)的一个非零解.
\end{proof}
\end{theorem}

\begin{corollary}\label{theorem:线性方程组.方程个数少于未知量个数的齐次线性方程组必有非零解}
方程个数少于未知量个数的齐次线性方程组必有非零解.
\begin{proof}
设数域\(P\)上的线性方程组\(\A \x = \z\),
它的系数矩阵\(\A \in M_{s \times n}(P)\ (s<n)\).
在原方程组的后面添加\(n-s\)个\(0=0\)的方程,解不变,新方程组的系数矩阵为:
\[
	\B_n = \begin{bmatrix} \A_{s \times n} \\ \z_{(n-s) \times n} \end{bmatrix},
\]
由于\(s < n\),有\(\abs{\B} = 0\),故\(\B \x = \z\)有非零解,从而\(\A \x = \z\)有非零解.
\end{proof}
\end{corollary}
注:上述推论的逆命题不成立,即方程组有非零解,不能得出未知量个数与方程个数的关系.
