\section{克拉默法则}
\begin{theorem}[克拉默法则]\label{theorem:线性方程组.克拉默法则}
%@see: https://mathworld.wolfram.com/CramersRule.html
给定一个未知量个数与方程个数相同的线性方程组\[
	\left\{ \begin{array}{l}
		a_{11}x_1 + a_{12}x_2 + \dotsb + a_{1n}x_n = b_1, \\
		a_{21}x_1 + a_{22}x_2 + \dotsb + a_{2n}x_n = b_2, \\
		\hdotsfor{1} \\
		a_{n1}x_1 + a_{n2}x_2 + \dotsb + a_{nn}x_n = b_n.
	\end{array} \right.
	\eqno(1)
\]
如果它的系数行列式满足\[
	d=\abs{\A}=\begin{vmatrix}
	a_{11} & a_{12} & \dots & a_{1n} \\
	a_{21} & a_{22} & \dots & a_{2n} \\
	\vdots & \vdots & \ddots & \vdots \\
	a_{n1} & a_{n2} & \dots & a_{nn}
	\end{vmatrix} \neq 0,
\]
则线性方程组(1)存在唯一解:\[
	\x_0 = \left( \frac{d_1}{d},\frac{d_2}{d},\dotsc,\frac{d_n}{d} \right)^T,
\]
其中\[
	d_k = \begin{vmatrix}
		a_{11} & \dots & a_{1\ k-1} & b_1 & a_{1\ k+1} & \dots & a_{1n} \\
		a_{21} & \dots & a_{2\ k-1} & b_2 & a_{2\ k+1} & \dots & a_{2n} \\
		\vdots & & \vdots & \vdots & \vdots & & \vdots \\
		a_{n1} & \dots & a_{n\ k-1} & b_n & a_{n\ k+1} & \dots & a_{nn}
	\end{vmatrix},
	\quad k=1,2,\dotsc,n.
\]
\begin{proof}
上述原线性方程组可以改写为矩阵形式\(\A \x = \b\).
因为\(d = \abs{\A} \neq 0\),故\(\A\)可逆,那么线性方程组有唯一解\[
	\X0 = \A^{-1} \b = \frac{1}{d} \A^* \b,
\]其中\(\A^*\)是\(\A\)的伴随矩阵.
设\(A_{ij}\)表示\(\A\)中元素\(a_{ij}\ (i,j=1,2,\dotsc,n)\)的代数余子式,
则行列式\(d_k\)可按第\(k\)列展开,得\[
	d_k = b_1 A_{1k} + b_2 A_{2k} + \dotsb + b_n A_{nk},
	\quad k=1,2,\dotsc,n,
\]
则\[
	\X0 = \frac{1}{d} \A^* \b
	= \frac{1}{d} \begin{bmatrix}
		A_{11} & A_{21} & \dots & A_{n1} \\
		A_{12} & A_{22} & \dots & A_{n2} \\
		\vdots & \vdots & \ddots & \vdots \\
		A_{1n} & A_{2n} & \dots & A_{nn}
	\end{bmatrix} \begin{bmatrix} b_1 \\ b_2 \\ \vdots \\ b_n \end{bmatrix}
	= \frac{1}{d} \begin{bmatrix} d_1 \\ d_2 \\ \vdots \\ d_n \end{bmatrix}.
	\qedhere
\]
\end{proof}
\end{theorem}
