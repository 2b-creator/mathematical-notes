\section{线性方程组}
\subsection{线性方程组的概念}
我们把含有\(n\)个未知量\(\AutoTuple{x}{n}\)的一次方程组
\begin{equation}\label[equation-system]{equation:线性方程组.线性方程组的代数形式}
	\left\{ \begin{array}{l}
		a_{11} x_1 + a_{12} x_2 + \dotsb + a_{1n} x_n = b_1, \\
		a_{21} x_1 + a_{22} x_2 + \dotsb + a_{2n} x_n = b_2, \\
		\hdotsfor{1} \\
		a_{s1} x_1 + a_{s2} x_2 + \dotsb + a_{sn} x_n = b_s
	\end{array} \right.
\end{equation}
称为“\(n\)元\DefineConcept{线性方程组}(\emph{linear system of equations} in \(n\) variables)”.
这里,\(s\)为方程的个数\footnote{%
在\cref{equation:线性方程组.线性方程组的代数形式} 中,
方程的数目\(s\)可以等于未知数的数目\(n\),
也可以不相等(即\(s<n\)或\(s>n\)).
};
我们把数
\[
	a_{ij}
	\quad(i=1,2,\dotsc,s; j=1,2,\dotsc,n)
\]
称为“第\(i\)个方程中\(x_j\)的\DefineConcept{系数}(coefficient)”;
把数
\[
	b_i
	\quad(i=1,2,\dotsc,s)
\]
叫做“第\(i\)个方程的\DefineConcept{常数项}(constant term)”.

\subsection{线性方程组的表示}
\begin{definition}
将\cref{equation:线性方程组.线性方程组的代数形式} 的系数按原位置构成的\(s \times n\)矩阵\[
	\A = \begin{bmatrix}
		a_{11} & a_{12} & \dots & a_{1n} \\
		a_{21} & a_{22} & \dots & a_{2n} \\
		\vdots & \vdots & & \vdots \\
		a_{n1} & a_{n2} & \dots & a_{nn}
	\end{bmatrix}
\]叫做\DefineConcept{系数矩阵}(coefficient matrix).

特别地,如果\(s = n\)(即系数矩阵\(\A\)是一个方阵),
则系数矩阵的行列式\(\abs{\A}\)叫做\DefineConcept{系数行列式}.
\end{definition}

为使表述简明,常用向量、矩阵表示线性方程组.
若记\[
	\x=\begin{bmatrix}
		x_1 \\ x_2 \\ \vdots \\ x_n
	\end{bmatrix},
	\quad
	\b=\begin{bmatrix}
		b_1 \\ b_2 \\ \vdots \\ b_n
	\end{bmatrix},
	\quad
	\a_j=\begin{bmatrix}
		a_{1j} \\ a_{2j} \\ \vdots \\ a_{sj}
	\end{bmatrix},
	\quad
	j=1,2,\dotsc,n.
\]
\(\A\)的列分块阵为\(\A = (\a_1,\a_2,\dotsc,\a_n)\),
则\cref{equation:线性方程组.线性方程组的代数形式} 有以下两种等价表示:
\begin{enumerate}
	\item 矩阵形式\[
		\A \x = \b.
	\]
	\item 向量形式\[
		x_1 \a_1 + x_2 \a_2 + \dotsb + x_n \a_n = \b.
	\]
\end{enumerate}

\begin{definition}
我们把常数项全为零的线性方程组
称为\DefineConcept{齐次线性方程组},
把常数项中有非零数的线性方程组
称为\DefineConcept{非齐次线性方程组}.
\end{definition}

\begin{definition}
如果存在\(n\)个数\(\AutoTuple{c}{n}\)满足\cref{equation:线性方程组.线性方程组的代数形式},即\[
	a_{i1} c_1 + a_{12} c_2 + \dotsb + a_{in} c_n \equiv b_i
	\quad(i=1,2,\dotsc,s),
\]
则称“\cref{equation:线性方程组.线性方程组的代数形式} 有解”,
或“\cref{equation:线性方程组.线性方程组的代数形式} 是相容的”;
否则,称“\cref{equation:线性方程组.线性方程组的代数形式} 无解”,
或“\cref{equation:线性方程组.线性方程组的代数形式} 是不相容的”.

称这\(n\)个数\(\AutoTuple{c}{n}\)构成的列向量\((\AutoTuple{c}{n})^T\)为%
\cref{equation:线性方程组.线性方程组的代数形式} 的一个\DefineConcept{解}(solution)%
或\DefineConcept{解向量}(solution vector).
\cref{equation:线性方程组.线性方程组的代数形式} 的解的全体构成的集合,
称为“\cref{equation:线性方程组.线性方程组的代数形式} 的\DefineConcept{解集}”.
\end{definition}

\begin{definition}
元素全为零的解向量称为\DefineConcept{零解}.
元素不全为零的解向量称为\DefineConcept{非零解}.
\end{definition}

\begin{theorem}
任意一个齐次线性方程组都有零解.
\end{theorem}

\begin{definition}
解集相等的两个线性方程组称为\DefineConcept{同解方程组}.
\end{definition}

\begin{example}
设\(\A\in M_{s \times n}(\mathbb{R})\).
证明:齐次线性方程组\(\A\x=\z\)与\((\A^T\A)\x=\z\)同解.
\begin{proof}
\def\a{\vb{\xi}}
\def\b{\vb{\eta}}
设\(\a\)是\(\A\x=\z\)的任意一个解,
则\(\A\a=\z\),于是\[
	(\A^T\A)\a=\A^T(\A\a)=\A^T\z=\z,
\]
这就是说\(\a\)是\((\A^T\A)\x=\z\)的一个解.

又设\(\b\)是\((\A^T\A)\x=\z\)的任意一个解,
则\[
	(\A^T\A)\b=\z.
	\eqno(1)
\]
在(1)式等号两边同时左乘\(\b^T\)得\[
	\b^T(\A^T\A)\b=(\A\b)^T(\A\b)=0.
	\eqno(2)
\]

假设\(\A\b=(\AutoTuple{c}{s})^T\).
由(2)式有\[
	(\AutoTuple{c}{s}) (\AutoTuple{c}{s})^T
	= \AutoTuple{c}{n}[+][2]
	= 0.
\]
由于\(\AutoTuple{c}{n}\in\mathbb{R}\),
所以\(\AutoTuple{c}{n}[=]=0\),
\(\A\b=\z\),
这就是说\(\b\)是\(\A\x=\z\)的一个解.

综上所述,\((\A^T\A)\x=\z\)与\(\A\x=\z\)同解.
\end{proof}
\end{example}

需要注意的是,线性方程组的解集与数域有关.

\begin{theorem}
\(n\)元线性方程组的解的情况有且只有三种可能:
无解、有唯一解、有无穷多个解.
\end{theorem}
