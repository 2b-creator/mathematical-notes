\section{线性方程组}
我们把含有\(n\)个未知量\(\AutoTuple{x}{n}\)的一次方程组
\begin{equation}\label[equation-system]{equation:线性方程组.线性方程组的代数形式}
	\left\{ \begin{array}{l}
		a_{11} x_1 + a_{12} x_2 + \dotsb + a_{1n} x_n = b_1, \\
		a_{21} x_1 + a_{22} x_2 + \dotsb + a_{2n} x_n = b_2, \\
		\hdotsfor{1} \\
		a_{s1} x_1 + a_{s2} x_2 + \dotsb + a_{sn} x_n = b_s
	\end{array} \right.
\end{equation}
称为“\(n\)元\DefineConcept{线性方程组}(\emph{linear system of equations} in \(n\) variables)”.
这里,\(s\)为方程的个数\footnote{%
在\cref{equation:线性方程组.线性方程组的代数形式} 中,
方程的数目\(s\)可以等于未知数的数目\(n\),
也可以不相等(即\(s<n\)或\(s>n\)).};
我们把数\[
	a_{ij}
	\quad(i=1,2,\dotsc,s; j=1,2,\dotsc,n)
\]称为“第\(i\)个方程中\(x_j\)的\DefineConcept{系数}(coefficient)”;
把数\[
	b_i
	\quad(i=1,2,\dotsc,s)
\]叫做“第\(i\)个方程的\DefineConcept{常数项}(constant term)”.

\begin{definition}
我们把常数项全为零的线性方程组
称为\DefineConcept{齐次线性方程组},
把常数项中有非零数的线性方程组
称为\DefineConcept{非齐次线性方程组}.
\end{definition}

\begin{definition}
如果存在\(n\)个数\(\AutoTuple{c}{n}\)满足\cref{equation:线性方程组.线性方程组的代数形式},即\[
	a_{i1} c_1 + a_{12} c_2 + \dotsb + a_{in} c_n \equiv b_i
	\quad(i=1,2,\dotsc,s),
\]
则称“\cref{equation:线性方程组.线性方程组的代数形式} 有解”,
或“\cref{equation:线性方程组.线性方程组的代数形式} 是相容的”;
否则,称“\cref{equation:线性方程组.线性方程组的代数形式} 无解”,
或“\cref{equation:线性方程组.线性方程组的代数形式} 是不相容的”.

称这\(n\)个数\(\AutoTuple{c}{n}\)构成的列向量\((\AutoTuple{c}{n})^T\)为%
\cref{equation:线性方程组.线性方程组的代数形式} 的一个\DefineConcept{解}(solution)%
或\DefineConcept{解向量}(solution vector).
\cref{equation:线性方程组.线性方程组的代数形式} 的解的全体构成的集合,
称为“\cref{equation:线性方程组.线性方程组的代数形式} 的\DefineConcept{解集}”.
\end{definition}

\begin{definition}
元素全为零的解向量称为\DefineConcept{零解}.
元素不全为零的解向量称为\DefineConcept{非零解}.
\end{definition}

\begin{theorem}
任意一个齐次线性方程组都有零解.
\end{theorem}

\begin{definition}
解集相等的两个线性方程组称为\DefineConcept{同解方程组}.
\end{definition}

需要注意的是,线性方程组的解集与数域有关.

\begin{theorem}
\(n\)元线性方程组的解的情况有且只有三种可能:
无解、有唯一解、有无穷多个解.
\end{theorem}
