\section{解析函数与调和函数的关系}
\subsection{调和函数}
\begin{definition}
若二元实函数\(H(x,y)\)在平面区域\(D\)内有二阶连续偏导数,
且满足拉普拉斯方程\[
	\pdv[2]{H}{x} + \pdv[2]{H}{y} = 0,
\]
则称“\(H(x,y)\)是区域\(D\)内的\DefineConcept{调和函数}”.
\end{definition}

\begin{theorem}
若函数\(f(z)=u(x,y)+\iu v(x,y)\)在区域\(D\)内解析,
则其实部\(u(x,y)\)及其虚部\(v(x,y)\)都是\(D\)内的调和函数.
\end{theorem}

\subsection{共轭调和函数}
\begin{definition}
在区域\(D\)内的两个调和函数\(u\)和\(v\)满足C-R条件\[
	\pdv{u}{x}=\pdv{v}{y}, \qquad
	\pdv{u}{y}=-\pdv{v}{x},
\]
则把\(v\)称为“\(u\)在区域\(D\)内的\DefineConcept{共轭调和函数}”.
\end{definition}
注意,此定义中\(u\)和\(v\)是不能交换顺序的,
因为C-R条件中\(u\)和\(v\)不能交换顺序的.
当\(v\)是\(u\)在区域\(D\)内的共轭调和函数,
则只能说\(-u\)是\(v\)在\(D\)中的共轭调和函数.

\begin{theorem}
若\(u(x,y)\)是单连通区域\(D\)内的调和函数,
则由第二类曲线积分确定的函数\[
	v(x,y)
	= \int_{(x_0,y_0)}^{(x,y)}
		-\pdv{u}{y} \dd{x} + \pdv{u}{x} \dd{y} + C
\]
可使\(f(x+\iu y)=u(x,y)+\iu v(x,y)\)在\(D\)内解析,其中\(C\in\mathbb{C}\).
\end{theorem}
