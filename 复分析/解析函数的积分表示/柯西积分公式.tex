\section{柯西积分公式}
\subsection{柯西积分公式}
\begin{theorem}\label{theorem:解析函数的积分表示.柯西积分公式}
设区域\(D\)的边界\(C\)是围线(或复围线);
\(f(z)\)在\(D\)内解析,
在\(\overline{D}=D+C\)上连续,
则\begin{enumerate}
	\item 对\(D\)内任意一点\(z\),
	有\begin{equation}\label{equation:解析函数的积分表示.柯西积分公式}
		f(z)
		= \frac{1}{2\pi\iu}
		\int_C \frac{f(\zeta)}{\zeta-z} \dd{\zeta},
	\end{equation}
	上式称为\DefineConcept{柯西积分公式};

	\item \(f(z)\)在\(D\)内有各阶导数,
	且\begin{equation}\label{equation:解析函数的积分表示.柯西高阶导数公式}
		f^{(n)}(z)
		= \frac{n!}{2\pi\iu}
		\int_C \frac{f(\zeta)}{(\zeta-z)^{n+1}} \dd{\zeta}
		\quad (n=1,2,\dots),
	\end{equation}
	上式称为\DefineConcept{柯西高阶导数公式}.
\end{enumerate}
\begin{proof}
在区域\(D\)内任意取定一点\(z\),
以\(z\)为中心、\(r\)为半径作一小圆周\(C_r\),
使\(C_r\)及其内部都含于\(D\)内.
设\(D_1\)是由\(C\)及\(C_r\)所围成的区域.
显然,函数\(\frac{f(\zeta)}{\zeta-z}\)在\(D_1\)内解析,
在\(\overline{D_1}\)上连续.
由多连通区域上柯西积分定理,
有\[
	\int_C \frac{f(\zeta)}{\zeta-z} \dd{\zeta}
	= \int_{C_r} \frac{f(\zeta)}{\zeta-z} \dd{\zeta}.
\]
又由\(\lim_{\zeta \to z} (\zeta-z) \frac{f(\zeta)}{\zeta-z} = f(z)\),
有\[
	\int_{C_r} \frac{f(\zeta)}{\zeta-z} \dd{\zeta} = 2\pi\iu f(z).
\]
从而有\[
	f(z) = \frac{1}{2\pi\iu} \int_C \frac{f(\zeta)}{\zeta-z} \dd{\zeta}.
\]

当\(n=1\)时,
在\(D\)内任意取定一点\(z_0\),
在\(z_0\)的邻近任取一点\(z \in D\),
则\begin{align*}
	f(z)-f(z_0)
	&= \frac{1}{2\pi\iu}
		\int_C \left[
			\frac{f(\zeta)}{\zeta-z}
			- \frac{f(\zeta)}{\zeta-z_0}
		\right] \dd{\zeta} \\
	&= \frac{z-z_0}{2\pi\iu}
		\int_C \frac{f(\zeta)}{(\zeta-z)(\zeta-z_0)} \dd{\zeta},
\end{align*}
于是\begin{align*}
	&\hspace{-20pt}
		\frac{f(z)-f(z_0)}{z-z_0}
		- \frac{1}{2\pi\iu}
		\int_C \frac{f(\zeta)}{(\zeta-z_0)^2} \dd{\zeta} \\
	&=\frac{1}{2\pi\iu}
		\int_C f(\zeta)
		\left[
			\frac{1}{(\zeta-z)(\zeta-z_0)}-\frac{1}{(\zeta-z_0)^2}
		\right] \dd{\zeta} \\
	&=\frac{z-z_0}{2\pi\iu}
		\int_C \frac{f(\zeta)}{(\zeta-z)(\zeta-z_0)^2} \dd{\zeta}.
\end{align*}

由于\(\lim_{z \to z_0} z-z_0 = 0\),
要证明\[
	f'(z_0) = \lim_{z \to z_0} \frac{f(z)-f(z_0)}{z-z_0}
	= \frac{1}{2\pi\iu} \int_C \frac{f(\zeta)}{(\zeta-z_0)^2} \dd{\zeta},
\]
只需要证明\(\int_C \frac{f(\zeta)}{(\zeta-z)(\zeta-z_0)^2} \dd{\zeta}\)在点\(z_0\)的邻域内有界.

为此,由于\(f(z)\)在\(\overline{D}\)上连续,
设在\(C\)上有\(\abs{f(z)} \leq M\),
点\(z_0\)到\(C\)的距离为\(d\),
则\(\zeta \in C\)时,
\(\abs{\zeta-z_0} \geq d\).
若\(z \in N_{\frac{d}{2}}(z_0)\),
有\(\abs{z-z_0} \leq \frac{d}{2}\),
则对于\(\zeta \in C\)
有\begin{align*}
	\abs{\zeta-z}
	&= \abs{(\zeta-z_0)-(z-z_0)} \\
	&\geq \abs{ \abs{\zeta-z_0} - \abs{z-z_0} } \\
	&= \abs{\zeta-z_0} - \abs{z-z_0} \\
	&\geq d - \frac{d}{2} = \frac{d}{2},
\end{align*}
所以有\[
	\abs{\int_C \frac{f(\zeta)}{(\zeta-z)(\zeta-z_0)^2} \dd{\zeta}}
	\leq \frac{M}{(d/2) d^2} L
	= \frac{2ML}{d^3},
\]
其中\(L\)是围线\(C\)的长度.
因为\(z_0\)是\(D\)内任意一点,
于是得到当\(n=1\)时\[
	f'(z) = \frac{1}{2\pi\iu} \int_C{\frac{f(\zeta)}{(\zeta-z)^2}\dd{\zeta}}
\]成立.
利用数学归纳法可以进一步证明\(n=2,3,\dots\)情况下命题依然成立.
\end{proof}
\end{theorem}

\hyperref[equation:解析函数的积分表示.柯西积分公式]{柯西积分公式}表明,
一个在区域内解析并连续到边界的函数,
它在边界上的值决定了它在区域内任一点的值.
因此,人们称柯西积分公式为解析函数的积分表达式.
从柯西积分公式可以看出,
解析函数的函数值之间有着密切的联系.
这是解析函数不同于一般函数的一个显著特性.
积分是涉及函数整体性质的一个概念;
函数在一点的值应只涉及孤立点这一局部.
柯西积分公式把整体和局部联系起来,
可以帮助我们详细地研究解析函数的各种局部性质.

必须注意的是,\cref{theorem:解析函数的积分表示.柯西积分公式} 中的围线\(C\)可以是复围线,
这时\(C\)所围的区域\(D\)是多连通区域,
\cref{equation:解析函数的积分表示.柯西积分公式,equation:解析函数的积分表示.柯西高阶导数公式} 中的积分也是复围线上的积分.
其正面与在\cref{theorem:解析函数的积分表示.柯西积分公式} 中把\(C\)视作单围线时的证明方法完全一致.

\subsection{柯西积分公式的若干推论}
\begin{corollary}\label{theorem:解析函数的积分表示.柯西积分公式推论1}
若函数\(f(z)\)在区域\(D\)内解析,
则\(f(z)\)在\(D\)内有任意阶导数.
\end{corollary}
\cref{theorem:解析函数的积分表示.柯西积分公式推论1}
表明了我们曾多次提到的一个事实:
解析函数在其解析的区域内是无穷次可微的,
且求任意阶导数后所得到的函数仍是该区域内的解析函数.
这也就是我们曾经说过的:
对于复变函数来说,
在一个区域内一阶可微就蕴含着任意阶可微.
较之实函数,这是解析函数的一个最鲜明的特征.

\begin{corollary}\label{theorem:解析函数的积分表示.柯西积分公式推论2}
函数\(f(z)=u(x,y)+\iu v(x,y)\)在区域\(D\)内解析的充分必要条件为:
\begin{enumerate}
	\item 二元实函数\(u(x,y)\)和\(v(x,y)\)的偏导数
	\(u'_x\)、\(u'_y\)、\(v'_x\)和\(v'_y\)在区域\(D\)内连续;

	\item \(u(x,y)\)和\(v(x,y)\)在区域\(D\)内处处满足C-R条件.
\end{enumerate}
\end{corollary}
\cref{theorem:解析函数的积分表示.柯西积分公式推论2} 是刻画解析函数的第二个等价定理.

\begin{theorem}[莫雷拉定理]\label{theorem:解析函数的积分表示.莫雷拉定理}
若函数\(f(z)\)在单连通区域\(D\)内连续,
且对\(D\)内的任一围线\(C\)有\[
	\int_C f(z) \dd{z} = 0,
\]
则\(f(z)\)在\(D\)内解析.
\end{theorem}
对单连通区域内连续复变函数而言,
\hyperref[theorem:解析函数的积分表示.莫雷拉定理]{莫雷拉定理}是柯西积分定理的逆定理.

\begin{theorem}\label{theorem:解析函数的积分表示.柯西积分公式推论3}
函数\(f(z)\)在区域\(D\)内解析的充分必要条件为:
\begin{enumerate}
	\item \(f(z)\)在区域\(D\)内连续;

	\item 对任一围线\(C\),
	只要\(C\)及其内部全含于\(D\)内,
	就有\(\int_C f(z) \dd{z} = 0\).
\end{enumerate}
\end{theorem}
\cref{theorem:解析函数的积分表示.柯西积分公式推论3} 是刻画解析函数的第三个等价定理.

\begin{theorem}[柯西不等式]
若函数\(f(z)\)在圆\(K: \abs{z-a}<R\)内解析,
且\(\abs{f(z)} \leq M\),
则\begin{equation}\label{theorem:解析函数的积分表示.柯西不等式}
	\abs{f^{(n)}(a)} \leq \frac{n! M}{R^n}
	\quad(n=1,2,\dots).
\end{equation}
\end{theorem}
\cref{theorem:解析函数的积分表示.柯西不等式} 是对解析函数各阶导数模的估计式,
表明解析函数在解析点\(a\)的各阶导数模的估计与它的解析区域的大小密切相关.

\begin{definition}
在整个复平面上解析的函数称为\DefineConcept{整函数}(entire function).
\end{definition}
在有的书上,整函数是这样定义的:
若函数\(f(z)\)除了无穷远点外,
在\(\mathbb{C}\)上是全纯的,
则称\(f(z)\)是一个整函数.

\begin{example}
常函数、多项式函数、\(e^z\)、\(\cos z\)和\(\sin z\)都是整函数.
\end{example}

\begin{definition}
若函数\(f(z)\)除了无穷远点外,
在\(\mathbb{C}\)上只有极点(极点的可以是有限个,也可以是无限个),
则称\(f(z)\)是一个\DefineConcept{亚纯函数}(meromorphic function).
\end{definition}

\begin{example}
所有的整函数都是亚纯函数.
有理分式函数\(f(z) = \frac{P_n(z)}{Q_m(z)}\)也是亚纯函数.
\end{example}

\begin{theorem}[刘维尔定理]\label{theorem:解析函数的积分表示.刘维尔定理}
若\(f(z)\)是有界的整函数,
则\(f(z)\)必为常数.
\begin{proof}
因为\(f(z)\)有界,
所以\(\abs{f(z)} \leq M\)在\(z\)平面上恒成立,
在圆\(\abs{z-a}<R\)内自然也成立.
由柯西不等式得\[
	\abs{f'(z)} \leq \frac{M}{R}.
\]
令\(R\to+\infty\),
得到\(f'(z)=0\)恒成立,
即\(f(z)\)是常数.
\end{proof}
\end{theorem}
\hyperref[theorem:解析函数的积分表示.刘维尔定理]{刘维尔定理}的逆定理(即“常函数是有界整函数”)也成立.
刘维尔定理的逆否定理“非常数的整函数必无界”自然也成立.

\begin{theorem}[解析函数的平均值定理]\label{theorem:解析函数的积分表示.平均值定理}
若函数\(f(z)\)在圆\(\abs{z-z_0}<R\)内解析,
在闭圆\(\abs{z-z_0} \leq R\)上连续,
则\[
	f(z_0)
	= \frac{1}{2\pi} \int_0^{2\pi} f(z_0+Re^{\iu\phi}) \dd{\phi},
\]
即\(f(z)\)在圆心\(z_0\)的值等于它在圆周上的值的算术平均数.
\end{theorem}

\begin{theorem}[最大模定理]\label{theorem:解析函数的积分表示.最大模定理}
若函数\(f(z)\)在区域\(D\)内解析,
且不为常数,
则\(\abs{f(z)}\)在\(D\)内取不到最大值.
\end{theorem}

\begin{corollary}\label{theorem:解析函数的积分表示.最大模定理推论}
若函数\(f(z)\)在有界区域\(D\)内解析,
在闭区域\(\overline{D}\)上连续,
并且不为常数,
则\(\abs{f(z)}\)只能在\(D\)的边界上达到最大值.
\end{corollary}
