\chapter{解析函数的积分表示}
\section{复变函数的积分}
\subsection{积分的定义和计算方法}
为了叙述简便而又不妨碍实际应用,今后除特别声明外,我们谈到曲线时一律是指光滑或逐段光滑的曲线.
其中,逐段光滑的简单闭曲线简称为\DefineConcept{围线}或\DefineConcept{周线}.

\begin{definition}
设有向曲线\(C: z = z(t)\ (\alpha \leq t \leq \beta)\)以\(a = z(\alpha)\)为起点,\(b = z(\beta)\)为终点.
又设函数\(f(z)\)沿\(C\)有定义.
在\(C\)上沿着\(C\)从\(a\)到\(b\)的方向(此为实参数增大的方向,通常作为\(C\)的正方向)任取\(n-1\)个分点\[
z_0 = a,\ z_1,\dots,\ z_{n-1},\ z_n = b
\]把曲线\(C\)分成\(n\)个小弧段.
在每个小弧段\(\arc{z_{k-1} z_k}\)上任取一点\(\zeta_k\),求和得\[
S_n = \sum_{k=1}^n{f(\zeta_k) \increment z_k},
\]
其中\(\increment z_k = z_k - z_{k-1}\).
记\(\lambda = \max\{\abs{\increment z_1},\abs{\increment z_2},\dots,\abs{\increment z_n}\}\).
若\(\lambda\to0\)(分点无限增多,且这些弧段长度均趋于零)时,
上述和式的极限\(\lim_{n\to\infty}S_n\)存在且趋于确定的极限\(J\),
那么称函数\(f(z)\)沿\(C\) \DefineConcept{可积},
称\(J\)为函数\(f(z)\)沿\(C\)(从\(a\)到\(b\))的\DefineConcept{积分},
记作\(\int_C f(z) \dd{z}\),
即\[
\int_C f(z) \dd{z} = J = \lim_{\lambda\to0} \sum_{k=1}^n{f(\zeta_k) \increment z_k}.
\]其中\(C\)称为\DefineConcept{积分路径}.
\(\int_C f(z) \dd{z}\)表示沿\(C\)的正方向的积分,
而\(\int_{C^-}{f(z)\dd{z}}\)表示沿\(C\)的负方向的积分.
\end{definition}

\begin{theorem}
如果函数\(f(z)=u(x,y)+\iu v(x,y)\)沿曲线\(C\)连续,则\(f(z)\)沿\(C\)可积,且\[
\int_C f(z) \dd{z}
= \int_C{u(x,y)\dd{x} - v(x,y)\dd{y}} + \iu \int_C{v(x,y)\dd{x} + u(x,y)\dd{y}}.
\]

为了便于记忆,上式也可写成\[
\int_C f(z) \dd{z} = \int_C{(u+\iu v)(\dd{x}+\iu\dd{y})}.
\]
\end{theorem}

\begin{corollary}
如果\(\int_C f(z) \dd{z}\)沿有向光滑曲线\(C\)连续.曲线\(C\)的方程为\[
z = z(t) = x(t) + \iu y(t), \quad \alpha \leq t \leq \beta,
\]那么有\[
\int_C f(z) \dd{z} = \int_\alpha^\beta f[z(t)] z'(t) \dd{t}.
\]称上式为复积分的\DefineConcept{变量代换公式}.
\end{corollary}

\begin{example}
设\(n\in\mathbb{Z}\),\(C\)是以\(a\)为心,\(R\)为半径的圆周.求:\[
I = \int_C \frac{\dd{z}}{(z-a)^n}.
\]
\begin{solution}
由于圆周\(C\)的参数方程为\[
z-a=Re^{\iu\theta}, \quad 0 \leq \theta \leq 2\pi,
\]所以\(\dd{z}=\iu R e^{\iu\theta} \dd{\theta}\).

当\(n=1\)时,\[
\int_C \frac{\dd{z}}{z-a}
= \int_0^{2\pi}{\frac{\iu R e^{\iu\theta} \dd{\theta}}{R e^{\iu\theta}}}
= \iu \int_0^{2\pi}\dd{\theta}
= 2\pi\iu.
\]

当\(n \neq 1\)时,\begin{align*}
\int_C{\frac{\dd{z}}{(z-a)^n}}
&= \int_0^{2\pi}{
	\frac{
		\iu R e^{\iu\theta} \dd{\theta}
	}{
		R^n e^{\iu n \theta}
	}
}
= \frac{\iu}{R^{n-1}} \int_0^{2\pi}{e^{-\iu(n-1)\theta}\dd{\theta}} \\
&= \frac{\iu}{R^{n-1}} \left[
	\int_0^{2\pi}{\cos(n-1)\theta\dd{\theta}}
	-\iu \int_0^{2\pi}{\sin(n-1)\theta\dd{\theta}}
	\right]
= 0.
\end{align*}
也就是说,\begin{equation}\label{equation.解析函数的积分表示.重要积分1}
\int_C \frac{\dd{z}}{(z-a)^n} = \left\{ \begin{array}{cl}
2\pi\iu, & n=1, \\
0, & n\in\mathbb{Z}-\{1\}.
\end{array} \right.
\end{equation}
\end{solution}
\end{example}

\subsection{复积分的基本性质}
\begin{property}
设\(f(z)\)和\(g(z)\)沿曲线\(C\)连续,则\begin{enumerate}
\item \(\int_C a f(z) \dd{z} = a \int_C f(z) \dd{z}, a\in\mathbb{C}\);
\item \(\int_C [f(z) \pm g(z)] \dd{z} = \int_C f(z) \dd{z} \pm \int_C g(z) \dd{z}\);
\item \(\int_C f(z) \dd{z} = \int_{C_1} f(z) \dd{z} + \int_{C_2} f(z) \dd{z}\),其中\(C\)由曲线\(C_1\)和\(C_2\)衔接而成;
\item \(\int_{C^-} f(z) \dd{z} = -\int_C f(z) \dd{z}\).
\end{enumerate}
\end{property}

\begin{theorem}
设\(f(z)\)和\(g(z)\)沿曲线\(C\)连续,则\[
\abs{\int_C f(z) \dd{z}}
\leq \int_C{\abs{f(z)}\abs{\dd{z}}}
= \int_C{\abs{f(z)}\dd{s}},
\]其中\(\abs{\dd{z}}=\dd{s}=\sqrt{(\dd{x})^2+(\dd{y})^2}\)表示弧长的微分.
\end{theorem}

\begin{corollary}[积分估值定理]\label{theorem:解析函数的积分表示.积分估值定理}
若存在\(M > 0\),使在曲线\(C\)上\(\abs{f(z)} \leq M\),曲线\(C\)的长为\(L\),则有\begin{equation}\label{equation:解析函数的积分表示.长大不等式}
\abs{\int_C f(z) \dd{z}} \leq ML.
\end{equation}\rm
因为\cref{equation:解析函数的积分表示.长大不等式} 是关于弧长\(L\)和函数\(f\)的模(大小)\(\abs{f(z)}\)的不等式,所以又称之为\DefineConcept{长大不等式}.
\end{corollary}

\section{柯西积分定理}
\subsection{柯西积分定理}
\begin{theorem}[柯西积分定理]\label{theorem:解析函数的积分表示.柯西积分定理}
若\(f(z)\)在单连通区域\(D\)内解析,则对\(D\)内的任意一条围线\(C\),有\[
\int_C f(z) \dd{z}=0.
\]
\end{theorem}

\begin{theorem}\label{theorem:解析函数的积分表示.柯西积分定理.闭区域的情形}
设\(C\)是一条围线,\(D\)是\(C\)的内部区域,\(f(z)\)在闭区域\(\overline{D}=D \cup C\)上解析,则\[
\int_C f(z) \dd{z}=0.
\]
\end{theorem}

\begin{corollary}\label{theorem:解析函数的积分表示.柯西积分定理.非简单闭曲线的情形}
设\(f(z)\)在单连通区域\(D\)内解析,\(C\)为\(D\)内任一闭曲线(不必是简单闭曲线),则\[
\int_C f(z) \dd{z} = 0.
\]
\end{corollary}

\begin{corollary}\label{theorem:解析函数的积分表示.解析函数在解析区域内的积分与路径无关}
若\(f(z)\)在单连通区域\(D\)内解析,则\(f(z)\)在\(D\)内的积分与路径无关.
\end{corollary}

我们约定,当复积分\(\int_C f(z) \dd{z}\)与路径\(C\)无关,只与积分的起点\(z_0\)和终点\(z_1\)有关时,这个积分就可以表示为\[
\int_{z_0}^{z_1} f(z) \dd{z}.
\]

下面对柯西积分定理从两个方面进行推广:
一方面是被积函数的解析范围;
另一方面是解析区域的连通性.
这两个方面的推广分别表现在下面两个定理中.
\begin{theorem}
设\(C\)是一条围线,区域\(D\)是\(C\)的内部,\(f(z)\)在\(D\)内解析,在\(\overline{D}=D \cup C\)上连续\footnote{%
也称这种情况为“\(f(z)\)从\(D\)内连续到边界”.%
},则\[
\int_C f(z) \dd{z} = 0.
\]
\end{theorem}

\begin{definition}
设有\(n+1\)条围线\(C_0,C_1,\dots,C_n\),其中\(C_1,\dots,C_n\)中每一条都在其余各条的外部,而它们又全都在\(C_0\)的内部.
在\(C_0\)内部且在\(C_1,\dots,C_n\)外部的点集构成有界的多连通区域\(D\).
\(D\)以\(C_0,C_1,\dots,C_n\)为边界.
我们把区域\(D\)的边界称为\DefineConcept{复围线},记作\(C=C_0+C_1^-+\dots+C_n^-\).
规定:在外边界\(C_0\)上,\(C\)的正方向为逆时针方向;
在内边界\(C_1,\dots,C_n\)上,\(C\)的正方向为顺时针方向.
\end{definition}

\begin{theorem}[多连通区域的柯西积分定理]\label{theorem:解析函数的积分表示.多连通区域的柯西积分定理}
设\(C\)是由复围线\(C=C_0+C_1^-+\dots+C_n^-\)所围成的有界多连通区域,
\(f(z)\)在\(D\)内解析,在\(\overline{D}=D \cup C\)上连续,则\[
\int_C f(z) \dd{z} = 0
\]或\begin{equation}
\def\Ic#1{\int_{C_{#1}} f(z) \dd{z}}
\Ic{0} = \Ic{1} + \dots + \Ic{n}.
\end{equation}
\end{theorem}

在掌握上述柯西积分定理的两种等价形式、两个推论和两个推广结论后,可以使复积分的计算变得极为简便.

\begin{example}
设\(a\)是任意围线\(C\)内部一点,试证:\[
\int_C \frac{\dd{z}}{(z-a)^n} = \left\{ \begin{array}{cl}
2\pi\iu, & n=1, \\
0, & n\in\mathbb{Z}-\{1\}.
\end{array} \right.
\]
\begin{proof}
记围线\(C\)所围成的区域为\(D\).
取\(r\in\mathbb{R}^+\)使得圆周\(C_0: z - a = r e^{\iu\theta}\ (0 \leq \theta \leq 2\pi)\)所围成的区域\(D_0 \subseteq D\).
根据\hyperref[theorem:解析函数的积分表示.多连通区域的柯西积分定理]{多连通区域的柯西积分定理}有\[
\int_C \frac{\dd{z}}{(z-a)^n} = \int_{C_0} \frac{\dd{z}}{(z-a)^n};
\]再应用重要积分 \labelcref{equation.解析函数的积分表示.重要积分1} 即得要证结论.
\end{proof}
\end{example}
由本例结果还可以得到一个时常用到的重要公式:对任一围线\(C\),都有\begin{equation}\label{equation.解析函数的积分表示.重要积分2}
\frac{1}{2\pi\iu} \int_C \frac{\dd{z}}{z-a}
= \left\{ \begin{array}{ll}
1, & a \in I(C); \\
0, & a \in E(C).
\end{array} \right.
\end{equation}

\begin{example}
试讨论在不同围线\(C\)上的复积分\[
I = \frac{1}{2\pi\iu} \int_C \frac{\dd{z}}{(z-a)(z-b)}
\]的取值,其中\(a \neq b\)且\(a,b \notin C\).
\begin{solution}
由于\[
\frac{1}{(z-a)(z-b)}
= \frac{1}{a-b} \left(\frac{1}{z-a}-\frac{1}{z-b}\right),
\]应用\cref{equation.解析函数的积分表示.重要积分2},即得\[
I = \left\{\begin{array}{cl}
0, & \text{\(a,b\)同时在\(C\)内部或\(C\)外部}, \\
\frac{1}{a-b}, & \text{\(a\)在\(C\)内部而\(b\)在\(C\)外部}, \\
\frac{1}{b-a}, & \text{\(a\)在\(C\)外部而\(b\)在\(C\)内部}.
\end{array}\right.
\]
\end{solution}
\end{example}

\begin{example}%例3.2.4
试证:若函数\(f(z)\)在区域\(D: 0<\abs{z-a}<R\)内解析,且\[
\lim_{z \to a}(z-a)f(z)=A,
\]则\[
\int_{\gamma} f(z) \dd{z} = 2\pi\iu A,
\]其中\(\gamma(\theta) = a+re^{\iu\theta}\ (0 \leq \theta \leq 2\pi, 0 < r < R)\).
\begin{proof}
由\(\lim_{z \to a}(z-a)f(z) = A\),设\[
(z-a)f(z) = A + \epsilon(z),
\]其中\(\epsilon(z)\)满足\(\lim_{z \to a} \epsilon(z) = 0\).那么有\[
f(z) = \frac{1}{z-a}[A+\epsilon(z)].
\]

由于\(r\to0\)时,\(z=\gamma(\theta) \to a\),所以\[
\lim_{r\to0} \abs{\epsilon[\gamma(\theta)]} = 0,
\quad 0 \leq \theta \leq 2\pi.
\]

又因为\[
\abs{\frac{\epsilon(z)}{z-a}}
= \abs{\frac{\epsilon[\gamma(\theta)]}{\gamma(\theta)-a}}
= \frac{\abs{\epsilon[\gamma(\theta)]}}{\abs{re^{\iu\theta}}}
= \frac{\abs{\epsilon[\gamma(\theta)]}}{r},
\]则根据积分估值定理可得\[
\abs{\int_{\gamma} \frac{\epsilon(z)}{z-a} \dd{z}}
\leq \abs{\frac{\epsilon(z)}{z-a}} \cdot 2\pi r
= \frac{\abs{\epsilon[\gamma(\theta)]}}{r} \cdot 2\pi r
= 2\pi \abs{\epsilon[\gamma(\theta)]},
\]且由函数\(f(z) = \frac{1}{z-a}[A+\epsilon(z)]\)在区域\(D\)内解析可知函数\(\frac{\epsilon(z)}{z-a}\)也同样在区域\(D\)内解析,再根据\hyperref[theorem:解析函数的积分表示.多连通区域的柯西积分定理]{多连通区域的柯西积分定理}可得\[
\int_{\gamma} \frac{\epsilon(z)}{z-a} \dd{z}
= \lim_{r\to0} \int_{\gamma} \frac{\epsilon(z)}{z-a} \dd{z}
= 0.
\]

那么有
\[
\int_{\gamma} f(z) \dd{z}
= \int_{\gamma} \frac{A}{z-a} \dd{z}
+ \int_{\gamma} \frac{\epsilon(z)}{z-a} \dd{z}
= 2\pi\iu A.
\qedhere
\]
\end{proof}
\end{example}

\subsection{原函数}
在实积分中,我们学习过\hyperref[theorem:定积分.原函数存在定理]{原函数存在定理},现在我们将这个定理推广到复积分的情形.
\begin{theorem}\label{theorem:解析函数的积分表示.原函数1}
设\(f(z)\)在单连通区域\(D\)内连续,且对全含于\(D\)内的任一围线\(C\),有\[
\int_C f(z) \dd{z} = 0,
\]则由变上限积分所确定的函数\[
F(z) = \int_{z_0}^z f(\zeta) \dd{\zeta}
\]在\(D\)内解析,且\(F'(z) = f(z)\),其中\(z_0, z \in D\).
\begin{proof}
由于\(D\)是单连通区域,又由\(\int_C f(z) \dd{z}=0\)可知\(f(z)\)在\(D\)内从点\(z_0\)到\(z\)的积分与路径无关,因此\(F(z)\)是\(D\)内的单值函数.

以点\(z\)为心作一个含于\(D\)内的圆,在此圆内任取一点\(z+\increment z\),于是\[
\frac{F(z+\increment z)-F(z)}{\increment z}
= \frac{1}{\increment z}\left[
 \int_{z_0}^{z+\increment z} f(\zeta) \dd{\zeta}
 -\int_{z_0}^z f(\zeta) \dd{\zeta}
\right]
= \frac{1}{\increment z} \int_z^{z+\increment z} f(\zeta) \dd{\zeta}.
\]又因为\[
\frac{1}{\increment z} \int_z^{z+\increment z} f(z) \dd{\zeta}
= \frac{f(z)}{\increment z} \int_z^{z+\increment z} \dd{\zeta} = f(z),
\]所以\[
\frac{F(z+\increment z)-F(z)}{\increment z} - f(z)
= \frac{1}{\increment z} \int_z^{z+\increment z} [f(\zeta)-f(z)]\dd{\zeta}.
\]由于\(f(\zeta)\)在点\(z\)连续,故对任给的\(\epsilon > 0\),存在\(\delta > 0\),当\(\abs{\zeta - z} < \delta\)时,便有\(\abs{f(\zeta)-f(z)} < \epsilon\).这样,若取\(0<\abs{\increment z}<\delta\),则由\hyperref[theorem:解析函数的积分表示.积分估值定理]{积分估值定理}可得\[
\abs{\frac{F(z+\increment z)-F(z)}{\increment z} - f(z)}
= \frac{1}{\abs{\increment z}} \abs{ \int_z^{z+\increment z} [f(\zeta)-f(z)]\dd{\zeta} }
\leq \frac{1}{\abs{\increment z}} \epsilon \abs{\increment z}
= \epsilon.
\]也就是说\[
\lim_{\increment z\to0} \frac{F(z+\increment z)-F(z)}{\increment z} = f(z)
\]或\[
F'(z) = f(z).
\qedhere
\]
\end{proof}
\end{theorem}

\begin{corollary}\label{theorem:解析函数的积分表示.原函数2}
若\(f(z)\)在单连通区域\(D\)内解析,则由变上限积分确定的函数\[
F(z) = \int_{z_0}^z f(\zeta) \dd{\zeta}
\]在\(D\)内解析,且\(F'(z) = f(z)\).
\end{corollary}

\begin{definition}
若在区域\(D\)内有\(F'(z)=f(z)\),则称\(F(z)\)为\(f(z)\)在区域\(D\)内的一个\DefineConcept{原函数}.
\end{definition}

\begin{corollary}\label{theorem:解析函数的积分表示.原函数3}
若\(f(z)\)在单连通区域\(D\)内解析
(或在\cref{theorem:解析函数的积分表示.原函数1} 的前提条件下),
\(\Phi(z)\)为\(f(z)\)在\(D\)内的任一原函数,
则有牛顿--莱布尼茨公式
\begin{equation}\label{equation:解析函数的积分表示.牛顿莱布尼茨公式}
\int_{z_0}^z f(\zeta) \dd{\zeta}
= \eval{\Phi(\zeta)}_{z_0}^z
= \Phi(z)-\Phi(z_0)
\end{equation}成立.
\end{corollary}
事实上,由于\(F(z) = \int_{z_0}^z f(\zeta) \dd{\zeta}\)是\(f(z)\)在\(D\)内的一个原函数,按定义对任意\(z \in D\),有\([\Phi(z) - F(z)]'=0\),从而\[
\Phi(z) - F(z) = C,
\]即\[
\Phi(z) = F(z) + C.
\]令\(z=z_0\),得\(C = \Phi(z_0)\),于是\cref{equation:解析函数的积分表示.牛顿莱布尼茨公式} 成立.

\begin{example}
在单连通区域\(D: -\pi<\arg z<\pi\)内,函数\(\ln z\)是\(f(z) = \frac{1}{z}\)的一个原函数,而\(f(z)\)在单连通区域\(D\)内解析,故由\cref{theorem:解析函数的积分表示.原函数3} 有\[
\int_1^z \frac{\dd{\zeta}}{\zeta}
= \ln z - \ln 1
= \ln z \quad(z \in D).
\]
\end{example}

\begin{example}
计算积分\[
	\int_C (z^2 - \sin z) \dd{z},
\]
其中\(C\)为摆线\[
	\left\{ \begin{array}{l}
	x = a(\theta-\sin\theta) \\
	y = a(1\cos\theta)
	\end{array} \right.
	\quad(0\leq\theta\leq2\pi).
\]
\begin{solution}
因为被积函数\(f(z) = z^2 - \sin z\)在\(z\)平面\(\mathbb{C}\)上解析,\(\mathbb{C}\)是单连通区域,所以积分只与路径的起点、终点有关,而与路径无关.
当\(\theta=0\)时,\(z=0\);
当\(\theta=2\pi\)时,\(z=2\pi a\).
因此要么将沿\(C\)的积分简化为沿实轴的积分,要么直接应用%
\hyperref[equation:解析函数的积分表示.牛顿莱布尼茨公式]{牛顿--莱布尼茨公式}.
\begin{align*}
\int_C (z^2 - \sin z) \dd{z}
&= \int_0^{2\pi a} (x^2 + \sin x) \dd{x} \\
&= \left(\frac{1}{3} x^3 - \cos x\right)_0^{2\pi a} \\
&= \frac{8}{3} \pi^3 a^3 - \cos(3\pi a+1).
\end{align*}
\end{solution}
\end{example}

必须指出的是:\cref{theorem:解析函数的积分表示.原函数2,theorem:解析函数的积分表示.原函数3} 中,区域\(D\)的单连通性对于推论的结论来说是不可缺少的.
若函数\(f(z)\)的解析区域\(D\)是复连通区域,则\(f(z)\)沿\(D\)内任意围线上的积分可能不为零,因而若仍用\(\int_{z_0}^z f(\zeta) \dd{\zeta}\)来表示\(f(z)\)沿点\(z_0\)到点\(z\)的\(D\)内某曲线上的积分,则\(\int_{z_0}^z f(\zeta) \dd{\zeta}\)一般来说不仅与\(z_0,z\)在\(D\)中的位置有关,也与由\(z_0\)到\(z\)的积分路径有关.
随着\(z_0\)到\(z\)的积分路径不同,\(\int_{z_0}^z f(\zeta) \dd{\zeta}\)的值也不同,所以,一般来说\(\int_{z_0}^z f(\zeta) \dd{\zeta}\)是一个多值函数,\(f(z)\)在多连通区域\(D\)内就不可能有原函数.

若\(f(z)\)在\(D\)内有原函数\(F(z)\),而\(C(t)\ (\alpha\leq t \leq\beta)\)是\(D\)内从\(z_0\)到\(z\)的光滑曲线,则由\(F'(z) = f(z)\),\begin{align*}
\int_C f(z) \dd{z}
&= \int_\alpha^\beta F'[C(t)] \cdot C'(t) \dd{t}
= \eval{F[C(t)]}_\alpha^\beta \\
&= F[C(\beta)] - F[C(\alpha)]
\end{align*}
与路径无关,只与\(C(t)\)的起点\(z_0\)和终点\(z\)有关,这就产生矛盾.
例如,函数\(f(z) = \frac{1}{z}\)在\(D=\mathbb{C}-\{0\}\)内解析,但它在\(D\)内没有原函数.
\begin{example}%例3.2.7
\def\f{\frac{\dd{\zeta}}{\zeta}}
试证:在两连通区域\(G: z\neq0,\infty\)内\[
\int_1^z \f
= \Ln z
\quad (z \in G),
\]其中积分路径\(C\)是不经过原点,且连接点\(z_0=1\)和点\(z\)的任意逐段光滑曲线.
\begin{proof}
考虑这样两条路径的积分,其中一条\(L\)沿着正方向或负方向绕原点若干周,另一条则既不过原点也不穿过负实轴.
%由例3.2.5和例3.2.2可得
\[
\int_L \f
= \left(\int_{ABCbDA} + \int_{ADaCEA} + \int_{AEF}\right) \f
= 2 \int_{\gamma} \f + \int_l \f.
\]
一般,\begin{align*}
\int_L \f
&= \int_l \f + n \int_{\gamma} \f \\
&= \ln z + 2n\pi\iu \quad(n=0,\pm1,\dotsc) \\
&= \Ln z.
\end{align*}
因此,所给变上限积分\(F(z) = \int_1^z \f\)在两连通区域\(G\)内是对数函数\(\Ln z\)的一个积分表达式.
\end{proof}
\end{example}

\section{柯西积分公式}
\subsection{柯西积分公式}
\begin{theorem}\label{theorem:解析函数的积分表示.柯西积分公式}
设区域\(D\)的边界\(C\)是围线(或复围线);\(f(z)\)在\(D\)内解析,在\(\overline{D}=D+C\)上连续,则\begin{enumerate}
\item 对\(D\)内任意一点\(z\),有\begin{equation}\label{equation:解析函数的积分表示.柯西积分公式}
f(z) = \frac{1}{2\pi\iu} \int_C \frac{f(\zeta)}{\zeta-z} \dd{\zeta},
\end{equation}上式称为\DefineConcept{柯西积分公式};
\item \(f(z)\)在\(D\)内有各阶导数,且\begin{equation}\label{equation:解析函数的积分表示.柯西高阶导数公式}
f^{(n)}(z) = \frac{n!}{2\pi\iu} \int_C \frac{f(\zeta)}{(\zeta-z)^{n+1}} \dd{\zeta} \quad (n=1,2,\dots),
\end{equation}上式称为\DefineConcept{柯西高阶导数公式}.
\end{enumerate}
\begin{proof}
在区域\(D\)内任意取定一点\(z\),以\(z\)为中心、\(r\)为半径作一小圆周\(C_r\),使\(C_r\)及其内部都含于\(D\)内.设\(D_1\)是由\(C\)及\(C_r\)所围成的区域.显然,函数\(\frac{f(\zeta)}{\zeta-z}\)在\(D_1\)内解析,在\(\overline{D_1}\)上连续.由多连通区域上柯西积分定理,有\[
\int_C{\frac{f(\zeta)}{\zeta-z}\dd{\zeta}}
= \int_{C_r}{\frac{f(\zeta)}{\zeta-z}\dd{\zeta}}.
\]又由\(\lim_{\zeta \to z} (\zeta-z) \frac{f(\zeta)}{\zeta-z} = f(z)\),有\[
\int_{C_r}{\frac{f(\zeta)}{\zeta-z}\dd{\zeta}} = 2\pi\iu f(z).
\]从而有\[
f(z) = \frac{1}{2\pi\iu} \int_C{\frac{f(\zeta)}{\zeta-z}\dd{\zeta}}.
\]

当\(n=1\)时,在\(D\)内任意取定一点\(z_0\),在\(z_0\)的邻近任取一点\(z \in D\),则\begin{align*}
f(z)-f(z_0)
&= \frac{1}{2\pi\iu} \int_C{\left[\frac{f(\zeta)}{\zeta-z}-\frac{f(\zeta)}{\zeta-z_0}\right]\dd{\zeta}} \\
&= \frac{z-z_0}{2\pi\iu} \int_C{\frac{f(\zeta)}{(\zeta-z)(\zeta-z_0)}\dd{\zeta}},
\end{align*}
于是\begin{align*}
&\hspace{-20pt}\frac{f(z)-f(z_0)}{z-z_0} - \frac{1}{2\pi\iu} \int_C{\frac{f(\zeta)}{(\zeta-z_0)^2} \dd{\zeta}} \\
&=\frac{1}{2\pi\iu} \int_C{f(\zeta) \left[\frac{1}{(\zeta-z)(\zeta-z_0)}-\frac{1}{(\zeta-z_0)^2}\right]\dd{\zeta}} \\
&=\frac{z-z_0}{2\pi\iu} \int_C{\frac{f(\zeta)}{(\zeta-z)(\zeta-z_0)^2}\dd{\zeta}}.
\end{align*}

由于\(\lim_{z \to z_0} z-z_0 = 0\),要证明\[
f'(z_0) = \lim_{z \to z_0} \frac{f(z)-f(z_0)}{z-z_0}
= \frac{1}{2\pi\iu} \int_C{\frac{f(\zeta)}{(\zeta-z_0)^2}\dd{\zeta}},
\]只需要证明\(\int_C{\frac{f(\zeta)}{(\zeta-z)(\zeta-z_0)^2}\dd{\zeta}}\)在点\(z_0\)的邻域内有界.

为此,由于\(f(z)\)在\(\overline{D}\)上连续,设在\(C\)上有\(\abs{f(z)} \leq M\),点\(z_0\)到\(C\)的距离为\(d\),则\(\zeta \in C\)时,\(\abs{\zeta-z_0} \geq d\).若\(z \in N_{\frac{d}{2}}(z_0)\),有\(\abs{z-z_0} \leq \frac{d}{2}\),则对于\(\zeta \in C\)有\begin{align*}
\abs{\zeta-z}
&= \abs{(\zeta-z_0)-(z-z_0)} \\
&\geq \abs{ \abs{\zeta-z_0} - \abs{z-z_0} } \\
&= \abs{\zeta-z_0} - \abs{z-z_0} \\
&\geq d - \frac{d}{2} = \frac{d}{2},
\end{align*}
所以有\[
\abs{
 \int_C{\frac{f(\zeta)}{(\zeta-z)(\zeta-z_0)^2}\dd{\zeta}}
 } \leq \frac{M}{(d/2) d^2} L
= \frac{2ML}{d^3},
\]其中\(L\)是围线\(C\)的长度.因为\(z_0\)是\(D\)内任意一点,于是得到当\(n=1\)时\[
f'(z) = \frac{1}{2\pi\iu} \int_C{\frac{f(\zeta)}{(\zeta-z)^2}\dd{\zeta}}
\]成立.利用数学归纳法可以进一步证明\(n=2,3,\dots\)情况下命题依然成立.
\end{proof}
\end{theorem}

\hyperref[equation:解析函数的积分表示.柯西积分公式]{柯西积分公式}表明,一个在区域内解析并连续到边界的函数,它在边界上的值决定了它在区域内任一点的值.因此,人们称柯西积分公式为解析函数的积分表达式.从柯西积分公式可以看出,解析函数的函数值之间有着密切的联系.这是解析函数不同于一般函数的一个显著特性.积分是涉及函数整体性质的一个概念;函数在一点的值应只涉及孤立点这一局部.柯西积分公式把整体和局部联系起来,可以帮助我们详细地研究解析函数的各种局部性质.

必须注意的是,\cref{theorem:解析函数的积分表示.柯西积分公式} 中的围线\(C\)可以是复围线,这时\(C\)所围的区域\(D\)是多连通区域,\cref{equation:解析函数的积分表示.柯西积分公式} 和\cref{equation:解析函数的积分表示.柯西高阶导数公式} 中的积分也是复围线上的积分.其正面与在\cref{theorem:解析函数的积分表示.柯西积分公式} 中把\(C\)视作单围线时的证明方法完全一致.

\subsection{柯西积分公式的若干推论}
\begin{corollary}\label{theorem:解析函数的积分表示.柯西积分公式推论1}
若函数\(f(z)\)在区域\(D\)内解析,则\(f(z)\)在\(D\)内有任意阶导数.
\end{corollary}
\cref{theorem:解析函数的积分表示.柯西积分公式推论1} 表明了我们曾多次提到的一个事实:解析函数在其解析的区域内是无穷次可微的,且求任意阶导数后所得到的函数仍是该区域内的解析函数.这也就是我们曾经说过的:对于复变函数来说,在一个区域内一阶可微就蕴含着任意阶可微.较之实函数,这是解析函数的一个最鲜明的特征.

\begin{corollary}\label{theorem:解析函数的积分表示.柯西积分公式推论2}
函数\(f(z)=u(x,y)+\iu v(x,y)\)在区域\(D\)内解析的充分必要条件为:
\begin{enumerate}
\item 二元实函数\(u(x,y)\)和\(v(x,y)\)的偏导数\(u'_x\)、\(u'_y\)、\(v'_x\)和\(v'_y\)在区域\(D\)内连续;
\item \(u(x,y)\)和\(v(x,y)\)在区域\(D\)内处处满足C-R条件.
\end{enumerate}
\end{corollary}
\cref{theorem:解析函数的积分表示.柯西积分公式推论2} 是刻画解析函数的第二个等价定理.

\begin{theorem}[莫雷拉定理]\label{theorem:解析函数的积分表示.莫雷拉定理}
若函数\(f(z)\)在单连通区域\(D\)内连续,且对\(D\)内的任一围线\(C\)有\[
\int_C f(z) \dd{z} = 0,
\]则\(f(z)\)在\(D\)内解析.
\end{theorem}
对单连通区域内连续复变函数而言,
\hyperref[theorem:解析函数的积分表示.莫雷拉定理]{莫雷拉定理}是柯西积分定理的逆定理.

\begin{theorem}\label{theorem:解析函数的积分表示.柯西积分公式推论3}
函数\(f(z)\)在区域\(D\)内解析的充分必要条件为:
\begin{enumerate}
\item \(f(z)\)在区域\(D\)内连续;
\item 对任一围线\(C\),只要\(C\)及其内部全含于\(D\)内,就有\(\int_C f(z) \dd{z} = 0\).
\end{enumerate}
\end{theorem}
\cref{theorem:解析函数的积分表示.柯西积分公式推论3} 是刻画解析函数的第三个等价定理.

\begin{theorem}[柯西不等式]
若函数\(f(z)\)在圆\(K: \abs{z-a}<R\)内解析,且\(\abs{f(z)} \leq M\),则\begin{equation}\label{theorem:解析函数的积分表示.柯西不等式}
\abs{f^{(n)}(a)} \leq \frac{n! M}{R^n}
\quad(n=1,2,\dots).
\end{equation}
\end{theorem}
\cref{theorem:解析函数的积分表示.柯西不等式} 是对解析函数各阶导数模的估计式,表明解析函数在解析点\(a\)的各阶导数模的估计与它的解析区域的大小密切相关.

\begin{definition}
在整个复平面上解析的函数称为\DefineConcept{整函数}(entire function).
\end{definition}
在有的书上,整函数是这样定义的:若函数\(f(z)\)除了无穷远点外,在\(\mathbb{C}\)上是全纯的,则称\(f(z)\)是一个整函数.

\begin{example}
常函数、多项式函数、\(e^z\)、\(\cos z\)和\(\sin z\)都是整函数.
\end{example}

\begin{definition}
若函数\(f(z)\)除了无穷远点外,在\(\mathbb{C}\)上只有极点(极点的可以是有限个,也可以是无限个),则称\(f(z)\)是一个\DefineConcept{亚纯函数}(meromorphic function).
\end{definition}

\begin{example}
所有的整函数都是亚纯函数.
有理分式函数\(f(z) = \frac{P_n(z)}{Q_m(z)}\)也是亚纯函数.
\end{example}

\begin{theorem}[刘维尔定理]\label{theorem:解析函数的积分表示.刘维尔定理}
若\(f(z)\)是有界的整函数,则\(f(z)\)必为常数.
\begin{proof}
因为\(f(z)\)有界,所以\(\abs{f(z)} \leq M\)在\(z\)平面上恒成立,在圆\(\abs{z-a}<R\)内自然也成立.由柯西不等式得\[
\abs{f'(z)} \leq \frac{M}{R}.
\]令\(R\to+\infty\),得到\(f'(z)=0\)恒成立,即\(f(z)\)是常数.
\end{proof}
\end{theorem}
\hyperref[theorem:解析函数的积分表示.刘维尔定理]{刘维尔定理}的逆定理(即“常函数是有界整函数”)也成立.刘维尔定理的逆否定理“非常数的整函数必无界”自然也成立.

\begin{theorem}[解析函数的平均值定理]\label{theorem:解析函数的积分表示.平均值定理}
若函数\(f(z)\)在圆\(\abs{z-z_0}<R\)内解析,在闭圆\(\abs{z-z_0} \leq R\)上连续,则\[
f(z_0) = \frac{1}{2\pi} \int_0^{2\pi} f(z_0+Re^{\iu\phi}) \dd{\phi},
\]即\(f(z)\)在圆心\(z_0\)的值等于它在圆周上的值的算术平均数.
\end{theorem}

\begin{theorem}[最大模定理]\label{theorem:解析函数的积分表示.最大模定理}
若函数\(f(z)\)在区域\(D\)内解析,且不为常数,则\(\abs{f(z)}\)在\(D\)内取不到最大值.
\end{theorem}

\begin{corollary}\label{theorem:解析函数的积分表示.最大模定理推论}
若函数\(f(z)\)在有界区域\(D\)内解析,在闭区域\(\overline{D}\)上连续,并且不为常数,则\(\abs{f(z)}\)只能在\(D\)的边界上达到最大值.
\end{corollary}

\section{解析函数与调和函数的关系}
\begin{definition}
若二元实函数\(H(x,y)\)在平面区域\(D\)内有二阶连续偏导数,且满足拉普拉斯方程\[
\pdv[2]{H}{x} + \pdv[2]{H}{y} = 0,
\]则称\(H(x,y)\)是区域\(D\)内的调和函数.
\end{definition}

\begin{theorem}
若函数\(f(z)=u(x,y)+\iu v(x,y)\)在区域\(D\)内解析,则其实部\(u(x,y)\)及其虚部\(v(x,y)\)都是\(D\)内的调和函数.
\end{theorem}

\begin{definition}
在区域\(D\)内满足C-R条件\[
\pdv{u}{x}=\pdv{v}{y}, \qquad \pdv{u}{y}=-\pdv{v}{x}
\]的两个调和函数\(u\)和\(v\),称\(v\)为\(u\)在区域\(D\)内的\DefineConcept{共轭调和函数}.
\end{definition}
注意,此定义中\(u\)和\(v\)是不能交换顺序的,因为C-R条件中\(u\)和\(v\)不能交换顺序的.当\(v\)是\(u\)在区域\(D\)内的\DefineConcept{共轭调和函数},则只能说\(-u\)是\(v\)在\(D\)中的共轭调和函数.

\begin{theorem}
若\(u(x,y)\)是单连通区域\(D\)内的调和函数,则由第二类曲线积分确定的函数\[
v(x,y)=\int_{(x_0,y_0)}^{(x,y)}
-\pdv{u}{y} \dd{x} + \pdv{u}{x} \dd{y} + C
\]可使\(f(x+\iu y)=u(x,y)+\iu v(x,y)\)在\(D\)内解析,其中\(C\in\mathbb{C}\).
\end{theorem}
