\chapter{解析函数的级数表示}
这一章先介绍复级数(包括复幂级数)的一般概念及基本性质.
接着,从柯西积分公式这一解析函数的积分表达式出发,
给出解析函数的级数表示 --- 泰勒展开式和罗朗展开式.
然后以它们为工具,进一步研究解析函数的性质.

\section{常数项级数}
\subsection{常数项级数的概念}
\begin{definition}
设有复数列\[
\alpha_1,\alpha_2,\dotsc,\alpha_n,\dotsc,
\]其中\(\alpha_n=a_n+\iu b_n\ (n=1,2,\dotsc)\),则表达式\[
\alpha_1+\alpha_2+\dotsb+\alpha_n+\dotsb
\]称为\DefineConcept{复常数项无穷级数},
简称\DefineConcept{复数项级数}或\DefineConcept{复级数},
记作\(\sum\limits_{i=1}^\infty \alpha_i\),
即\[
\sum\limits_{i=1}^\infty \alpha_i = \alpha_1+\alpha_2+\dotsb+\alpha_n+\dotsb,
\]其中第\(n\)项\(\alpha_n\)叫做级数的\DefineConcept{一般项}.

作复数项级数的前\(n\)项的和\[
s_n = \alpha_1+\alpha_2+\dotsb+\alpha_n = \sum\limits_{i=1}^n{u_i},
\]称\(s_n\)为级数的\DefineConcept{部分和}.

如果级数\(\sum\limits_{i=1}^\infty \alpha_i\)的部分和数列\(\{s_n\}\)有极限\(s=a+\iu b\),
即\[
\lim\limits_{n\to\infty}s_n = s,
\]
则称无穷级数\(\sum\limits_{i=1}^\infty \alpha_i\) \DefineConcept{收敛},
这时极限\(s\)叫做这级数的\DefineConcept{和},并写成\[
s = \alpha_1+\alpha_2+\dotsb+\alpha_n+\dotsb;
\]
如果\(\{s_n\}\)没有极限,
则称无穷级数\(\sum\limits_{i=1}^\infty \alpha_i\) \DefineConcept{发散}.
\end{definition}

\subsection{常数项级数的性质}
\begin{theorem}\label{theorem:解析函数的级数表示.复级数与其实部及虚部级数的关系}
设\(\alpha_n=a_n+\iu b_n\ (n=1,2,\dots)\),则\[
\sum\limits_{i=1}^\infty \alpha_i = s = a + \iu b
\iff
\sum\limits_{i=1}^\infty a_i = a,
\sum\limits_{i=1}^\infty b_i = b.
\]
\end{theorem}

根据\cref{theorem:解析函数的级数表示.复级数与其实部及虚部级数的关系} 以及\cref{theorem:无穷级数.收敛级数性质1,theorem:无穷级数.收敛级数性质2,theorem:无穷级数.收敛级数性质3,theorem:无穷级数.收敛级数性质4,theorem:无穷级数.收敛级数性质5},可知收敛的复级数具有与实数项无穷级数相似的性质.
\begin{property}
收敛的复数项级数具有的性质包括:
\begin{enumerate}
\item \(\sum\limits_{i=1}^\infty \alpha_i\ \text{收敛}
\implies
\lim\limits_{n\to\infty}\alpha_n=0\);
\item \(\sum\limits_{i=1}^\infty \alpha_i\ \text{收敛}
\implies
\exists M > 0 : \abs{\alpha_n} \leq M \quad(n=1,2,\dotsc)\);
\item 若\(\sum\limits_{i=1}^\infty \alpha_i=S_1\),
\(\sum\limits_{i=1}^\infty \beta_i=S_2\),
则\[
\sum\limits_{i=1}^\infty (\alpha_i\pm\beta_i)=S_1+S_2,
\]\[
\sum\limits_{i=1}^\infty c\alpha_i
=c\sum\limits_{i=1}^\infty \alpha_i
=cS
\quad(c\in\mathbb{C});
\]
\item 在复级数\(\sum\limits_{i=1}^\infty \alpha_i\)中去掉、加上或改变有限项,不会改变其收敛性.
\end{enumerate}
\end{property}

\begin{definition}
若级数\(\sum\limits_{n=1}^\infty \abs{\alpha_n}\)收敛,
则称级数\(\sum\limits_{n=1}^\infty \alpha_n\) \DefineConcept{绝对收敛};
不绝对收敛的收敛级数,称为\DefineConcept{条件收敛级数}.
\end{definition}
参考\cref{theorem:无穷级数.绝对收敛级数必定收敛},
显然,在复级数中也有“绝对收敛级数一定收敛”.
级数\(\sum\limits_{n=1}^\infty \abs{\alpha_n}\)的各项均为非负实数,
因此\(\sum\limits_{n=1}^\infty \abs{\alpha_n}\)为实正项级数,
可按实正项级数的收敛性判别法则,
如\hyperref[theorem:无穷级数.正项级数的比较审敛法]{比较审敛法}、%
\hyperref[theorem:无穷级数.正项级数的比值审敛法]{比值审敛法}、%
\hyperref[theorem:无穷级数.正项级数的根值审敛法]{根值审敛法}等,判断其收敛性.

参考\cref{theorem:无穷级数.绝对收敛级数的可交换性},
和实级数一样,在复级数中也有“绝对收敛级数的各项可以重排顺序而不改变其绝对收敛性与和”.

\begin{definition}
设复级数\(\sum\limits_{n=1}^\infty \alpha_n = S_1\)和\(\sum\limits_{n=1}^\infty \beta_n = S_2\),称复级数\[
\sum\limits_{n=1}^\infty (
	\alpha_1 \beta_n + \alpha_2 \beta_{n-1} + \dotsc + \alpha_n \beta_1
)
= \sum\limits_{n=1}^\infty
	\sum\limits_{k=1}^n \alpha_k \beta_{n+1-k}
\]为级数\(\sum\limits_{n=1}^\infty \alpha_n\)和级数\(\sum\limits_{n=1}^\infty \beta_n\)的\DefineConcept{柯西乘积},记作\(\left( \sum\limits_{n=1}^\infty \alpha_n \right) \cdot \left( \sum\limits_{n=1}^\infty \beta_n \right)\).
\end{definition}

\begin{theorem}
若复级数\(\sum\limits_{n=1}^\infty \alpha_n\)和\(\sum\limits_{n=1}^\infty \beta_n\)绝对收敛且分别收敛到\(S_1\)和\(S_2\),则\[
\left( \sum\limits_{n=1}^\infty \alpha_n \right) \cdot \left( \sum\limits_{n=1}^\infty \beta_n \right) = S_1 \cdot S_2.
\]
\end{theorem}

\begin{example}[复等比级数]\label{example:解析级数的级数表示.复等比级数}
讨论复级数\(\sum\limits_{n=0}^\infty z^n\)的收敛性.
\begin{solution}
分两种情形讨论:\begin{enumerate}
\item 当\(\abs{z} < 1\)时,
正项级数\(\sum\limits_{n=0}^\infty \abs{z}^n\)收敛,
故原级数绝对收敛.
原级数的部分和为\[
S_n = 1 + z + z^2 + \dotsb + z^{n-1} + z^n = \frac{1-z^{n+1}}{1-z},
\]那么\[
\sum\limits_{n=0}^\infty z^n = \lim\limits_{n\to\infty} S_n = \lim\limits_{n\to\infty} \frac{1-z^{n+1}}{1-z} = \frac{1}{1-z}.
\]

\item 当\(\abs{z} \geq 1\)时,\(\abs{z}^n \geq 1\),
所以一般项\(z^n \not\to 0\ (n\to\infty)\),
级数发散.
\end{enumerate}
\end{solution}
\end{example}

\section{函数项级数}
\subsection{函数项级数的概念}
参照\hyperref[definition:无穷级数.实函数项级数的概念]{实函数项级数的定义},可以定义出复函数项级数的概念如下:
\begin{definition}\label{definition:解析函数的级数表示.收敛级数}
设点集\(E \subseteq \mathbb{C}\).如果给定一个定义在点集\(E\)上的复变函数列\[
f_1(z), f_2(z), \dotsc, f_n(z), \dotsc,
\]则由该函数列构成的表达式\[
f_1(z) + f_2(z) + \dotsb + f_n(z) + \dotsb
\]称为定义在点集\(E\)上的\DefineConcept{复函数项无穷级数},简称\DefineConcept{复函数项级数}.

如果\(\sum\limits_{i=1}^\infty f_i(z)\)%
对于\(\forall z_0 \in E\)%
都有\(\sum\limits_{i=1}^\infty f_i(z_0)\)收敛,
则称“\(\sum\limits_{i=1}^\infty f_i(z)\)在点集\(E\)上收敛”,
称复变函数\(F(z) = \sum\limits_{i=1}^\infty f_i(z)\)为它的\DefineConcept{和函数}.
\end{definition}
用“\(\epsilon-\delta\)”语言来说,级数\(\sum\limits_{i=1}^\infty f_i(z)\)在点\(z \in E\)处收敛于\(F(z)\)的充要条件是:\[
\forall \epsilon > 0, \exists N = N(\epsilon, z) \in \mathbb{N}^+ \left[
n > N \implies \abs{F(z) - \sum\limits_{i=1}^n f_i(z)} < \epsilon
\right].
\]

\subsection{级数的一致收敛}
通常来说,复函数项级数是否收敛依赖于其定义域\(E\).
\begin{definition}\label{definition:解析函数的级数表示.一致收敛级数}
设定义在点集\(E\)上的复函数项级数\(\sum\limits_{i=1}^\infty f_i(z)\)的部分和函数为\(S_n(z)\),若在点集\(E\)存在函数\(F(z)\),使得\[
\forall\epsilon>0,
\exists N \in \mathbb{N}^+,
\forall z \in E
\bigl[
	n>N \implies \abs{F(z) - S_n(z)} < \epsilon
\bigr],
\]则称\(\sum\limits_{i=1}^\infty f_i(z)\)在点集\(E\)上\DefineConcept{一致收敛}于\(F(z)\).
\end{definition}

比较\cref{definition:解析函数的级数表示.收敛级数} 和\cref{definition:解析函数的级数表示.一致收敛级数} 描述的两种收敛,前者定义中的\(N = N(\epsilon, z)\)不仅与\(\epsilon\)有关,通常也与所讨论的点\(z \in E\)有关;也就是说,当正数\(\epsilon\)取定后,随点\(z\)不同,正整数\(N\)通常也随之变化;后者定义中的\(N = N(\epsilon)\),却只随\(\epsilon\)变化而变化,并不随\(z\)变化而变化.因此,前者描述的收敛反映的是级数在\(E\)的局部性质,后者描述的一致收敛反映的是级数在\(E\)中的整体性质.

\begin{theorem}[柯西一致收敛准则]\label{theorem:无穷级数.柯西一致收敛准则}
复函数项级数\(\sum\limits_{i=1}^\infty f_i(z)\)在点集\(E\)上一致收敛于某函数的充要条件是:\[
\begin{array}{r}
\forall\epsilon>0,
\exists N \in \mathbb{N}^+, \\
\forall z \in E,
\forall p \in \mathbb{N}^+
\end{array}
\left[
\begin{array}{l}
n > N \implies \\
\hspace{20pt}
\abs{f_{n+1}(z) + f_{n+2}(z) + \dotsb + f_{n+p}(z)} < \epsilon
\end{array}
\right].
\]
\end{theorem}

类似于\cref{theorem:无穷级数.魏尔斯特拉斯判别法},由\hyperref[theorem:无穷级数.柯西一致收敛准则]{柯西一致收敛准则}可得复函数项级数一致收敛的一个充分条件.
\begin{corollary}[Weierstrass M - 判别法,优级数准则]\label{theorem:无穷级数.优级数准则}
若复函数序列\(\{f_n(z)\}\)在点集\(E\)上有定义,且存在正数列\(\{M_n\}\),使得对\(\forall z \in E\),有\[
\abs{f_n(z)} \leq M_n
\quad(n=1,2,\dotsc),
\]且正项级数\(\sum\limits_{i=1}^\infty M_i\)收敛,则复函数项级数\(\sum\limits_{i=1}^\infty f_i(z)\)在点集\(E\)上绝对收敛且一致收敛.
\begin{proof}
记复函数项级数\(\sum\limits_{i=1}^\infty f_i(z)\)的部分和为\(S_n(z)\).
取\(M > N\),那么部分和\(S_n(z)\)满足\[
\abs{S_M(z) - S_N(z)}
= \abs{\sum\limits_{i=N+1}^M f_i(z)}
\leq \sum\limits_{i=N+1}^M M_i.
\]因为正项级数\(\sum\limits_{i=1}^\infty M_i\)收敛,所以当\(N,M \to \infty\)时,\(\sum\limits_{i=N+1}^M M_i \to 0\),可知\(\{S_n\}\)是一个一致收敛的柯西序列,同时它也绝对收敛.
\end{proof}
\end{corollary}
这样的正项级数\(\sum\limits_{i=1}^\infty M_i\)称为复函数项级数\(\sum\limits_{i=1}^\infty f_i(z)\)的\DefineConcept{优级数}.

\begin{example}
证明:级数\(\sum\limits_{n=1}^\infty z^n\)在闭圆\(\abs{z} \leq r\ (r<1)\)上一致收敛.
\begin{proof}
显然成立,因为级数\(\sum\limits_{n=1}^\infty z^n\)有收敛的优级数\(\sum\limits_{n=1}^\infty r^n\).
\end{proof}
\end{example}

\begin{definition}
设函数\(f_n(z)\ (n=1,2,\dots)\)在区域\(D\)内有定义.
若\(\sum\limits_{n=1}^\infty f_n(z)\)在含于\(D\)内的任意一个有界闭区域\(d\)上都一致收敛,
则称级数\(\sum\limits_{n=1}^\infty f_n(z)\)在\(D\) \DefineConcept{内闭一致收敛}.
\end{definition}
显然,若\(\sum\limits_{i=1}^\infty f_i(z)\)在区域\(D\)内闭一致收敛,则\(\sum\limits_{i=1}^\infty f_i(z)\)在\(D\)内每一点都是收敛的,但不一定在\(D\)上一致收敛.自然,若\(\sum\limits_{i=1}^\infty f_i(z)\)在\(D\)上一致收敛,则\(\sum\limits_{i=1}^\infty f_i(z)\)在\(D\)内闭一致收敛.

\begin{theorem}\label{theorem:解析函数的级数表示.一致收敛级数的基本性质1}
设级数\(\sum\limits_{i=1}^\infty f_i(z)\)的各项在区域\(D\)内连续,并且一致收敛于\(F(z)\),则和函数\(F(z)\)也在\(D\)连续.
\end{theorem}

\begin{theorem}\label{theorem:解析函数的级数表示.一致收敛级数的基本性质2}
设级数\(\sum\limits_{i=1}^\infty f_i(z)\)的各项在曲线\(C\)上连续,并且在\(C\)上一致收敛于\(F(z)\),则沿\(C\)可以逐项积分:\[
\int_C F(z) \dd{z}
= \sum\limits_{i=1}^\infty \int_C f_i(z) \dd{z}.
\]
\end{theorem}

\begin{theorem}[魏尔斯特拉斯定理]\label{theorem:解析函数的级数表示.魏尔斯特拉斯定理}
设级数\(\sum\limits_{i=1}^\infty f_i(z)\)的各项在区域\(D\)内解析,且\(\sum\limits_{i=1}^\infty f_i(z)\)在\(D\)内闭一致收敛于\(F(z)\),则\begin{enumerate}
\item \(F(z)\)在\(D\)内解析;
\item 在\(D\)内可以逐项求任意阶导数:\[
F^{(m)}(z) = \sum\limits_{i=1}^\infty f_i^{(m)}(z)
\quad(m=1,2,\dotsc);
\]
\item \(\sum\limits_{i=1}^\infty f_i^{(m)}(z)\)在\(D\)内一致收敛于\(F^{(m)}(z)\).
\end{enumerate}
\begin{proof}
因为\(\sum\limits_{i=1}^\infty f_i(z)\)在\(D\)内收敛,所以和函数\(F(z)\)在\(D\)内有定义.根据\cref{theorem:解析函数的级数表示.一致收敛级数的基本性质1},\(F(z)\)在\(D\)内连续.

设\(K\)是\(D\)内的任一圆周,其内部属于\(D\).又设\(\gamma\)是\(K\)内部的任一围线,则由假定可知\(\sum\limits_{i=1}^\infty f_i(z)\)在\(\gamma\)上一致收敛;那么根据\cref{theorem:解析函数的级数表示.一致收敛级数的基本性质2},得\[
\int_{\gamma} F(z) \dd{z} = \sum\limits_{i=1}^\infty \int_{\gamma} f_i(z) \dd{z}.
\]因为\(f_i(z)\)在\(K\)的内部解析,根据\hyperref[equation:解析函数的积分表示.柯西积分公式]{柯西积分公式},有\(\int_{\gamma} f_i(z) \dd{z} = 0\),从而有\(\int_{\gamma} F(z) \dd{z} = 0\).根据\hyperref[theorem:解析函数的积分表示.莫雷拉定理]{莫雷拉定理}可知,\(f(z)\)在\(K\)的内部解析.因为\(K\)是\(D\)内的任一圆周,所以\(f(z)\)在\(D\)内解析.

\def\f#1{f\ifx\relax#1\relax\else_{#1}\fi^{(m)}(z_0) = \frac{m!}{2\pi\iu} \int_\Gamma \frac{f\ifx\relax#1\relax\else_{#1}\fi(\zeta)}{(\zeta-z_0)^{m+1}} \dd{\zeta}}
设\(z_0\)是\(D\)内任一点,则\[
\exists \rho > 0 : \Set{ z \given \abs{z-z_0} \leq \rho } \subseteq D.
\]根据上述结果,\(f(z)\)在闭圆\(\overline{K}\)上解析.应用\cref{equation:解析函数的积分表示.柯西高阶导数公式},得\[
\def\arraystretch{1.5}
\begin{array}{l}
\f{} \\
\f{n}
\end{array}
\quad(m=1,2,\dotsc),
\]其中\(\Gamma: \abs{\zeta-z_0}=\rho\).由于\(\sum\limits_{i=1}^\infty f_i(z)\)在\(D\)内闭一致收敛于\(f(z)\),从而在圆周\(\Gamma\)上\[
\frac{f(\zeta)}{(\zeta-z_0)^{m+1}}
= \sum\limits_{i=1}^\infty \frac{f_i(\zeta)}{(\zeta-z_0)^{m+1}}
\]是一致收敛的.于是由\cref{theorem:解析函数的级数表示.一致收敛级数的基本性质2} 得\[
\int_\Gamma \frac{f(\zeta)}{(\zeta-z_0)^{m+1}} \dd{\zeta}
= \sum\limits_{i=1}^\infty \int_\Gamma \frac{f_i(\zeta)}{(\zeta-z_0)^{m+1}} \dd{\zeta},
\]两端同乘以\(\frac{m!}{2\pi\iu}\)就得到所要证明的结果.
\end{proof}
\end{theorem}

\begin{example}
试证函数\(f(z) = \sum\limits_{n=1}^\infty \frac{1}{n^z}\)在区域\(D = \Set{ z \given \Re z > 1 }\)内解析.
\begin{proof}
显然\(\sum\limits_{n=1}^\infty \frac{1}{n^z}\)是解析函数项级数.任取有界闭区域\(D' \subseteq D\).设\(D'\)到直线\(\Re z = 1\)的距离为\(d\ (d > 0)\),于是\[
\forall z \in D' : \Re z > 1+d,
\]从而\[
\abs{n^{-z}} = \abs{e^{-z \ln n}}
= e^{-(\Re z) \ln n} = n^{-\Re z}
< n^{-(1+d)}.
\]因正项级数\(\sum\limits_{n=1}^\infty n^{-(1+d)}\)收敛,根据\hyperref[theorem:无穷级数.优级数准则]{优级数准则},级数\(\sum\limits_{n=1}^\infty n^{-z}\)在\(D' \subseteq D\)上一致收敛,从而在\(D\)内闭一致收敛.故由\hyperref[theorem:解析函数的级数表示.魏尔斯特拉斯定理]{魏尔斯特拉斯定理},\(f(z) = \sum\limits_{n=1}^\infty n^{-z}\)在\(D\)内解析.
\end{proof}
\end{example}

\section{幂级数}
\subsection{幂级数的概念}
\begin{definition}
各项均是幂函数的复函数项级数,称为\DefineConcept{幂级数}.其形式是\[
\sum\limits_{n=0}^\infty c_n z^n
= c_0 + c_1 z + c_2 z^2 + \dotsb,
\]其中复常数\(c_0,c_1,c_2,\dotsc\)称为幂级数的\DefineConcept{系数}.
\end{definition}

\subsection{阿贝尔定理}
\begin{theorem}[阿贝尔定理]\label{theorem:解析函数的级数表示.阿贝尔定理}
若幂级数\(\sum\limits_{n=0}^\infty c_n z^n\)在\(z_0 \neq 0\)处收敛,则它在圆\(\abs{z} < \abs{z_0}\)内绝对收敛且内闭一致收敛.
\begin{proof}
因\(\sum\limits_{n=0}^\infty c_n z_0^n\)收敛,故\(\lim\limits_{n\to\infty} c_n z_0^n = 0\),从而\[
\exists M > 0, \forall n \in \mathbb{N} : \abs{c_n z_0^n} \leq M,
\]进而有\[
\abs{c_n z^n}
= \abs{c_n z_0^n \cdot \frac{z^n}{z_0^n}}
\leq M \abs{\frac{z}{z_0}}^n.
\]当\(\abs{z} < \abs{z_0}\)时,有\(\abs{\frac{z}{z_0}} < 1\).把\(z\)视为固定的复数时,由\cref{example:无穷级数.等比级数的收敛性} 可知实等比级数\(\sum\limits_{n=0}^\infty \abs{\frac{z}{z_0}}^n\)收敛,所以根据\hyperref[theorem:无穷级数.正项级数的比较审敛法]{正项级数的比较审敛法}可知,实幂级数\(\sum\limits_{n=0}^\infty \abs{c_n z^n}\)在圆\(\abs{z} < \abs{z_0}\)内收敛,也就是说复幂级数\(\sum\limits_{n=0}^\infty c_n z^n\)在圆\(\abs{z} < \abs{z_0}\)内绝对收敛.

在闭圆\(\abs{z} \leq k \abs{z_0}\ (0<k<1)\)内,有\(\abs{c_n z^n} \leq M k^n\).因为实等比级数\(\sum\limits_{n=0}^\infty k^n\)收敛,所以根据\hyperref[theorem:无穷级数.优级数准则]{优级数准则}可知,复幂级数\(\sum\limits_{n=0}^\infty c_n z^n\)在闭圆\(\abs{z} \leq k \abs{z_0}\)一致收敛,故复幂级数\(\sum\limits_{n=0}^\infty c_n z^n\)在圆\(\abs{z} < \abs{z_0}\)内闭一致收敛.
\end{proof}
\end{theorem}
阿贝尔定理表明,若幂级数\(\sum\limits_{n=0}^\infty c_n z^n\)在点\(z = z_0 \neq 0\)处收敛,则它在圆\(\abs{z} < \abs{z_0}\)内绝对收敛;特别地,当\(0 \leq r < \abs{z_0}\)时正项级数\(\sum\limits_{n=0}^\infty \abs{c_n} r^n\)收敛.若幂级数\(\sum\limits_{n=0}^\infty c_n z^n\)在点\(z = z_0\)处发散,则它在圆\(\abs{z} > \abs{z_0}\)外发散;特别地,当\(r > \abs{z_0}\)时正项级数\(\sum\limits_{n=0}^\infty \abs{c_n} r^n\)发散.

我们可以看出,对于级数\(\sum\limits_{n=0}^\infty c_n z^n\),可能存在一个圆\[
K: \abs{z} < R,
\]使得该级数在\(K\)内绝对收敛,而在\(K\)外发散.这个圆\(K\)称为该幂级数的\DefineConcept{收敛圆},数\(R\)称为\DefineConcept{收敛半径}.

可以证明,复级数\(\sum\limits_{n=0}^\infty c_n z^n\)的收敛半径\(R\)恰好就是与之相应的实系数的幂级数\(\sum\limits_{n=0}^\infty \abs{c_n} r^n\)的收敛半径.

\begin{theorem}\label{theorem:解析函数的级数表示.复幂级数的收敛半径的求法}
若幂级数\(\sum\limits_{n=0}^\infty c_n z^n\)的系数\(c_n\)满足\[
\lim\limits_{n\to\infty} \abs{\frac{c_{n+1}}{c_n}} = \rho
\]或\[
\lim\limits_{n\to\infty} \sqrt[n]{\abs{c_n}} = \rho
\]或\[
\varlimsup\limits_{n\to\infty} \sqrt[n]{\abs{c_n}} = \rho,
\]则这幂级数的收敛半径为\[
\def\arraystretch{1.5}
R = \left\{ \begin{array}{ll}
\frac{1}{\rho}, & \rho \in (0,+\infty), \\
+\infty, & \rho = 0, \\
0, & \rho = +\infty \\
\end{array} \right.
\]
\end{theorem}
对于幂级数\(\sum\limits_{n=0}^\infty c_n (z-a)^n\),上述求收敛半径的定理仍然成立,其收敛圆为\[
\abs{z-a}<R.
\]

若\(0<R<+\infty\),那么当\(\abs{z-a} \leq r < R\)时,有\[
\abs{c_n (z-a)^n} \leq \abs{c_n} r^n,
\]级数\(\sum\limits_{n=0}^\infty \abs{c_n} r^n\)收敛,所以\(\sum\limits_{n=0}^\infty c_n (z-a)^n\)在\(\abs{z-a} \leq r\)上一致收敛,而\(r\)可以任意接近于\(R\).因此由\hyperref[theorem:解析函数的级数表示.魏尔斯特拉斯定理]{魏尔斯特拉斯定理},幂级数\(\sum\limits_{n=0}^\infty c_n (z-a)^n\)的和函数\(F(z)\)在圆\(\abs{z-a}<R\)内解析,而且\[
f^{(k)}(z) = k! c_k + (k+1)k\dotsm2c_{k+1}(z-a)+\dotsb.
\]特别地,令\(z=a\),得\[
c_k = \frac{1}{k!} f^{(k)}(a)
\quad(k=0,1,2,\dotsc).
\]

若\(R=+\infty\),则幂级数\(\sum\limits_{n=0}^\infty c_n (z-a)^n\)在整个复平面\(C\)上绝对收敛,在任意闭圆\(\abs{z} \leq r\)上一致收敛,和函数\(F(z)\)在\(C\)上解析,等等.综上所述,有以下结果:
\begin{theorem}\label{theorem:解析函数的级数表示.幂级数的和函数的性质}
设幂级数\(\sum\limits_{n=0}^\infty c_n (z-a)^n\)的和函数是\(F(z)\),收敛圆是\(K: \abs{z-a}<R\ (0<R<+\infty)\).\begin{enumerate}
\item 和函数\(F(z)\)在其收敛圆\(K\)内解析;

\item 在收敛圆\(K\)内,幂级数\[
F(z) = \sum\limits_{n=0}^\infty c_n (z-a)^n
\]可以逐项求任意阶导数,得\begin{equation}
f^{(p)}(z) = \sum\limits_{n=p}^\infty n(n-1)\dotsm(n-p+1) c_n (z-a)^{n-p}
\quad(p=0,1,2,\dotsc),
\end{equation}且其收敛半径不变;

\item 幂级数的系数\(c_n\)可以用和函数\(F(z)\)在收敛圆圆心\(z=a\)处的相应阶导数表出,即\begin{equation}
c_n = \frac{1}{n!} f^{(n)}(a)
\quad(n=0,1,2,\dotsc);
\end{equation}

\item 沿收敛圆\(K\)内的任一简单曲线\(\gamma \subseteq K\),可逐项积分,得\begin{equation}
\int_{\gamma} F(z) \dd{z}
= \sum\limits_{n=0}^\infty c_n \int_{\gamma} (z-a)^n \dd{z},
\end{equation}且收敛半径不变.
\end{enumerate}
\end{theorem}

\begin{example}
求幂级数\(\sum\limits_{n=0}^\infty \frac{z^n}{n}\)的收敛半径、收敛圆以及和函数.
\begin{solution}
收敛半径为\[
R = \lim\limits_{n\to\infty} \abs{\frac{c_n}{c_{n+1}}} = \lim\limits_{n\to\infty} \frac{n}{n+1} = 1.
\]收敛圆为\(\abs{z} < 1\).

设在\(\abs{z} < 1\)内,幂级数\(\sum\limits_{n=0}^\infty \frac{z^n}{n}\)的和函数\(F(z)\).由\cref{theorem:解析函数的级数表示.幂级数的和函数的性质} 可得\[
F'(z) = \sum\limits_{n=1}^\infty z^{n-1}
= \sum\limits_{n=0}^\infty z^n
= \frac{1}{1-z}
\quad(\abs{z} < 1).
\]令\(1-z = \zeta\),有\[
F'(1-\zeta) = \frac{1}{\zeta}
\quad(\abs{\zeta-1}<1).
\]在\(\abs{\zeta-1}<1\)内沿着\(1\)到\(\zeta\)的路径积分,可得\[
\int_1^{\zeta} F'(1-\zeta) \dd{\zeta}
= \int_1^{\zeta} \frac{1}{\zeta} \dd{\zeta},
\]\[
F(0) - F(1-\zeta)
= \ln \zeta,
\]其中\(-\pi < \arg \zeta < \pi\).由于\(F(0) = 0\),所以和函数\[
F(z) = -\ln(1-z)
\quad(\abs{z}<1, -\pi < \arg(1-z) < \pi).
\]
\end{solution}
\end{example}
注意,前面的讨论没有涉及到幂级数\(\sum\limits_{n=0}^\infty c_n (z-a)^n\)在收敛圆周\(\abs{z-a}=R\ (0<R<+\infty)\)上的收敛性.在\(\abs{z-a}=R\)上,幂级数\(\sum\limits_{n=0}^\infty c_n (z-a)^n\)既可以是点点收敛,也可以是点点发散,还可以在一部分点上收敛而在其余点上发散.例如,级数\(\sum\limits_{n=1}^\infty \frac{z^n}{n^2}\)的收敛半径\(R=1\);在\(\abs{z}=1\)上,\(\sum\limits_{n=1}^\infty \abs{\frac{z^n}{n^2}} = \sum\limits_{n=1}^\infty \frac{1}{n^2}\)收敛,因此\(\sum\limits_{n=1}^\infty \frac{z^n}{n^2}\)在\(\abs{z}=1\)上处处绝对收敛.又例如级数\(\sum\limits_{n=0}^\infty z^n\)在\(\abs{z}=1\)上点点发散,因为这时一般项\(z^n\)的模\(\abs{z^n}\)恒为\(1\)而不趋于零.再例如级数\(\sum\limits_{n=1}^\infty \frac{z^n}{n}\)的收敛半径\(R=1\);它在圆周\(\abs{z}=1\)上只在点\(z=1\)处发散,而在其他各点\(z = e^{\iu\theta}\ (0<\theta<2\pi)\)上,有\[
\sum\limits_{n=1}^\infty \frac{z^n}{n}
= \sum\limits_{n=1}^\infty \frac{\cos n\theta}{n}
+ \iu\sum\limits_{n=1}^\infty \frac{\sin n\theta}{n},
\]由于它的实部、虚部级数都收敛,因此它在圆周\(\abs{z}=1\)上除去点\(z=1\)外处处收敛.

但我们要特别之处的是:
纵使幂级数在其收敛圆周上处处收敛,其和函数在收敛圆周上仍然至少有一个奇点.
例如级数\(\sum\limits_{n=1}^\infty \frac{z^n}{n^2}\)虽然在\(\abs{z}=1\)上处处绝对收敛,从而在闭圆\(\abs{z}\leq1\)上一致收敛,且其一般项\(\frac{z^n}{n^2}\)在复平面上都是解析的,因此根据\hyperref[theorem:解析函数的级数表示.魏尔斯特拉斯定理]{魏尔斯特拉斯定理},级数\(\sum\limits_{n=1}^\infty \frac{z^n}{n^2}\)在\(\abs{z}<1\)内的和函数\(F(z)\)的导数为\[
F'(z) = 1 + \frac{z}{2} + \frac{z^2}{3} + \dotsb + \frac{z^{n-1}}{n} + \dotsb.
\]当\(z\)从单位元内沿实轴趋于\(1\)时,\(F'(z)\)趋于\(+\infty\).而我们知道,解析函数在其解析点处是无穷次可微的,所以\(z=1\)是和函数\(F(z)\)的一个奇点.

\section{解析函数的泰勒展开式}
\subsection{泰勒定理}
我们已经知道,任意一个收敛半径为正数的幂级数,其和函数在收敛圆内是解析的,下面的泰勒定理表明其逆命题也成立.
\begin{theorem}\label{theorem:解析函数的级数表示.泰勒定理}
\def\G{\Gamma_{\rho}}
设\(f(z)\)在区域\(D\)内解析,取\(a \in D\),只要开圆\(K: \abs{z-a}<R\)满足\(K \subseteq D\),则\(f(z)\)在圆\(K\)内能展开成幂级数\[
f(z) = \sum\limits_{n=0}^\infty c_n (z-a)^n,
\eqno{(1)}
\]其中\[
c_n = \frac{1}{2\pi\iu} \int_{\G} \frac{f(\zeta)}{(\zeta-a)^{n+1}} \dd{\zeta}
= \frac{f^{(n)}(a)}{n!}
\quad (n=0,1,2,\dotsc);
\eqno{(2)}
\]\[
\G: \abs{\zeta-a} = \rho \in (0,R).
\]而且上述展开式是唯一的,也就是说,如果\[
\forall r > 0 \left[
\abs{z-a} < r
\implies
f(z) = \sum\limits_{n=0}^\infty b_n (z-a)^n
= \sum\limits_{n=0}^\infty c_n (z-a)^n
\right],
\]那么\(\forall n\in\mathbb{N} : b_n = c_n\).

\rm
(1)式称为函数\(f(z)\)在点\(a\)的\DefineConcept{泰勒展式},(2)式称为展式的\DefineConcept{泰勒系数},而由(2)式确定系数的幂级数称为\DefineConcept{泰勒级数}.
\begin{proof}
\begin{figure}[ht]
\centering
\begin{tikzpicture}[rotate=10]
\draw circle(2cm);
\draw[dashed] circle(1.8cm);
\draw (1,.1)node{\(K\)};
\fill (0,0)circle(2pt)node[below]{\(a\)};
\fill (-1,1)circle(2pt)node[below]{\(z\)};
\pgfmathsetmacro{\ta}{40}
\pgfmathsetmacro{\xa}{2*sin(\ta)}
\pgfmathsetmacro{\ya}{-2*cos(\ta)}
\pgfmathsetmacro{\ua}{\ta-90}
\pgfmathsetmacro{\va}{\ta-160}
\pgfmathsetmacro{\ub}{\va-180}
\pgfmathsetmacro{\vb}{-70}
\draw (\xa,\ya)arc[start angle=\ua,end angle=\va,radius=-1]
	arc[start angle=\ub,end angle=\vb,radius=-3]
	arc[start angle=\vb,end angle=-49,radius=-13.5];
\draw (5,0)node[right]{\(D\)};
\end{tikzpicture}
\caption{解析函数\(f(z)\)在解析区域\(D\)内任一点\(a\)处的收敛圆\(K\)}
\label{figure:解析函数的级数表示.解析函数在解析区域内任一点处的收敛圆}
\end{figure}
设\(z\)是圆\(K\)内任意取定的点,总有一个圆周\(\G: \abs{\zeta-a} = \rho \in (0,R)\)可以使得点\(z\)含在\(\G\)(\cref{figure:解析函数的级数表示.解析函数在解析区域内任一点处的收敛圆} 中虚线表\(\G\))内部.由\hyperref[equation:解析函数的积分表示.柯西积分公式]{柯西积分公式}得\[
f(z) = \frac{1}{2\pi\iu} \int_{\G} \frac{f(\zeta)}{\zeta-z} \dd{\zeta}.
\]

为了得到\(f(z)\)的幂级数展式,
关键在于将\(\frac{1}{\zeta-z}\)(柯西核)展为几何级数.
为此,将\(\frac{1}{\zeta-z}\)变形为\[
\frac{1}{\zeta-z}
= \frac{1}{(\zeta-a)-(z-a)}
= \frac{1}{\zeta-a} \cdot \left(1 - \frac{z-a}{\zeta-a}\right)^{-1}.
\]当\(\zeta \in \G\)时,由于\(\abs{\frac{z-a}{\zeta-a}}<1\),那么\[
\frac{1}{1-u} = \sum\limits_{n=0}^\infty u^n
\quad(\abs{u}<1),
\]\[
\left(1 - \frac{z-a}{\zeta-a}\right)^{-1}
= \sum\limits_{n=0}^\infty \left(\frac{z-a}{\zeta-a}\right)^n.
\]因此\[
\frac{f(\zeta)}{\zeta-z}
= \frac{f(\zeta)}{\zeta-a} \cdot \left(1 - \frac{z-a}{\zeta-a}\right)^{-1}
= \frac{f(\zeta)}{\zeta-a} \cdot \sum\limits_{n=0}^\infty \left(\frac{z-a}{\zeta-a}\right)^n.
\]级数\(\sum\limits_{n=0}^\infty \left(\frac{z-a}{\zeta-a}\right)^n\)在\(\G\)上(关于\(\zeta\))是一致收敛的,与\(\G\)上的有界函数\(\frac{f(\zeta)}{\zeta-a}\)相乘,仍然是\(\G\)上的一致收敛级数,于是\(\frac{f(\zeta)}{\zeta-z}\)可表成在\(\G\)上一致收敛的级数:\[
\frac{f(\zeta)}{\zeta-z}
= \sum\limits_{n=0}^\infty (z-a)^n \cdot \frac{f(\zeta)}{(\zeta-a)^{n+1}}.
\]将上式沿\(\G\)积分,再乘以\(\frac{1}{2\pi\iu}\),根据\cref{theorem:解析函数的级数表示.一致收敛级数的基本性质2} 得\[
f(z) = \frac{1}{2\pi\iu} \int_{\G} \frac{f(\zeta)}{\zeta-z} \dd{\zeta}
= \sum\limits_{n=0}^\infty (z-a)^n \cdot \frac{1}{2\pi\iu} \int_{\G} \frac{f(\zeta)}{(\zeta-a)^{n+1}}.
\]由\cref{equation:解析函数的积分表示.柯西高阶导数公式} 可知,\[
\frac{1}{2\pi\iu} \int_{\G} \frac{f(\zeta)}{(\zeta-a)^{n+1}} \dd{\zeta}
= \frac{f^{(n)}(a)}{n!},
\]故\[
f(z) = \sum\limits_{n=0}^\infty c_n (z-a)^n.
\]

下面证明展式是唯一的.假设\[
f(z) = \sum\limits_{n=0}^\infty b_n (z-a)^n,
\quad z \in K: \abs{z-a}<R.
\]根据\cref{theorem:解析函数的级数表示.幂级数的和函数的性质},\[
b_n = \frac{f^{(n)}(a)}{n!} = c_n \quad (n\in\mathbb{N}),
\]故展式是唯一的.
\end{proof}
\end{theorem}
显然,幂级数\(\sum\limits_{n=0}^\infty c_n (z-a)^n\)的收敛半径应当大于或等于\(R\),否则\[
f(z) = \sum\limits_{n=0}^\infty c_n (z-a)^n
\]将不能在圆\(K\)内成立.
至于幂级数\(\sum\limits_{n=0}^\infty c_n (z-a)^n\)的收敛半径能取多大,当系数\(c_n\)确定后,可由\cref{theorem:解析函数的级数表示.复幂级数的收敛半径的求法} 确定其收敛半径.
另外,前面我们曾经指出:对收敛半径为有限正数的幂级数,它在收敛圆内的和函数在收敛圆周上至少有一个奇点.
由此我们得到确定幂级数收敛半径的另一个方法:

设\(f(z)\)在点\(a\)解析,又设\(f(z)\)在点\(a\)的某邻域内的幂级数展式为\(f(z) = \sum\limits_{n=0}^\infty c_n (z-a)^n\),再设点\(b\)是\(f(z)\)奇点中距离点\(a\)最近的一个奇点,则点\(a\)与点\(b\)间的距离\(\abs{a-b}\)就是幂级数\(\sum\limits_{n=0}^\infty c_n (z-a)^n\)的收敛半径.

根据上述求解收敛半径的方法,泰勒定理中圆\(K\)的半径\(R\),不论区域\(D\)是单连通区域还是多连通区域,最多可取为点\(a\)到解析区域\(D\)的边界的距离.
其实,为保证圆\(K\)含于\(D\),半径\(R\)最多也只能取为这个距离.

由泰勒展式的唯一性证明过程,及泰勒级数的定义,立即可以得到下述推论.
\begin{corollary}
任何收敛半径为正数的幂级数都是它的和函数在收敛圆内的泰勒展式.
\end{corollary}

综合\cref{theorem:解析函数的级数表示.幂级数的和函数的性质} 和\cref{theorem:解析函数的级数表示.泰勒定理},可以得出刻画解析函数的第四个等价定理:
\begin{theorem}
“函数\(f(z)\)在区域\(D\)内解析”的充要条件是“\(f(z)\)在\(D\)内任一点\(a\)的邻域内可展成\(z-a\)的幂级数”.
\end{theorem}

\subsection{常见初等函数的泰勒展式}
\subsubsection{直接法求泰勒展式}
\begin{example}
求\(f(z) = e^z\)在点\(z = 0\)处的泰勒展开式.
\begin{solution}
\(f(z) = e^z\)在\(z\)平面解析,\(f^{(n)}(z) = e^z\).在\(z = 0\)处的泰勒系数\[
c_n = \frac{1}{n!} f^{(n)}(0) = \frac{1}{n!}
\quad(n=0,1,2,\dotsc),
\]于是\(f(z) = e^z\)在点\(z = 0\)处的泰勒展式为\begin{equation}\label{equation:解析函数的级数表示.指数函数的泰勒展式}
e^z = \sum\limits_{n=0}^\infty \frac{z^n}{n!}
\quad(\abs{z} < +\infty).
\end{equation}
\end{solution}
\end{example}

\begin{example}
求\(f(z) = \Ln(1+z)\)的各个分支在点\(z = 0\)的泰勒展开式,并指出展开式成立范围.
\begin{solution}
多值函数\(f(z) = \Ln(1+z)\)以\(z = -1\)及\(z = \infty\)为支点,将\(z\)平面沿负实轴从\(-1\)到\(\infty\)割破.在这样得到的区域\(G\)内,\(\Ln(1+z)\)可以分出无穷多个单值解析分支:\[
f_k(z) = \ln(1+z) + 2k\pi\iu
\quad(k=0,\pm1,\pm2,\dotsc).
\]无论哪一个分支,与\(z = 0\)距离最近的奇点都是\(z = -1\).因此,泰勒定理中圆\(K\)的半径可取为\(1\).

先取\(\Ln(1+z)\)的主值\(f_0(z) = \ln(1+z)\),在单位圆\(\abs{z}<1\)内作泰勒展开.
为此先计算泰勒系数,由于\[
f'_0(z) = \frac{1}{1+z},
\dotsc,
f^{(n)}_0(z) = (-1)^{n-1} \frac{(n-1)!}{(1+z)^n},
\]所以泰勒系数为\[
c_n = \frac{1}{n!} f^{(n)}_0(0) = \frac{(-1)^{n-1}}{n}
\quad (n=1,2,\dotsc).
\]因为\(f_0(z) = \ln(1+z)\)是主值,即\(1+z\)取正实数时\(\ln(1+z)\)取实数值,于是有\(c_0 = f_0(0) = 0\).
这样,\(f_0(z) = \ln(1+z)\)在\(z = 0\)点的泰勒展式为\begin{equation}\label{equation:解析函数的级数表示.对数函数的泰勒展式}
\begin{split}
\ln(1+z) &= z - \frac{z^2}{2} + \frac{z^3}{3} - \dotsb + (-1)^{n-1} \frac{z^n}{n} + \dotsb \\
&= \sum\limits_{n=1}^\infty \frac{(-1)^{n-1}}{n} z^n
\end{split}
\quad (\abs{z} < 1).
\end{equation}故\(f(z) = \Ln(1+z)\)的各个单值解析分支在点\(z = 0\)处的泰勒展式为\[
f_k(z) = 2k\pi\iu + \sum\limits_{n=1}^\infty \frac{(-1)^{n-1}}{n} z^n
\quad (\abs{z} < 1; k=0,\pm1,\pm2,\dotsc).
\]
\end{solution}
\end{example}

\subsubsection{间接法求泰勒展式}
\begin{example}
将下列函数在\(z = 0\)展成幂级数:

(1)\(\cos z\); \hfill (2)\(\sin z\); \hfill (3)\(e^z \cos z\); \hfill (4)\(e^z \sin z\).
\begin{solution}
四个函数都在整个\(z\)平面上解析,将要求得的幂级数的收敛范围都应是\[
\abs{z}<+\infty.
\]

(1)利用\(e^z\)的展开式 \labelcref{equation:解析函数的级数表示.指数函数的泰勒展式},\[
\cos z = \frac{1}{2} (e^{\iu z} + e^{-\iu z})
= \frac{1}{2} \sum\limits_{n=0}^\infty \frac{(\iu z)^n}{n!} + \frac{1}{2} \sum\limits_{n=0}^\infty \frac{(-\iu z)^n}{n!},
\]注意到右端两个技术的奇次方项互相抵消,故得\begin{equation}\label{equation:解析函数的级数表示.余弦函数的泰勒展式}
\cos z = \sum\limits_{n=0}^\infty \frac{(-1)^n z^{2n}}{(2n)!} \quad(\abs{z}<+\infty).
\end{equation}

(2)与(1)类似可得\begin{equation}\label{equation:解析函数的级数表示.正弦函数的泰勒展式}
\sin z = \sum\limits_{n=0}^\infty \frac{(-1)^n z^{2n+1}}{(2n+1)!} \quad(\abs{z}<+\infty).
\end{equation}

(3)因为\[
e^z \cos z = e^z \cdot \frac{1}{2} (e^{\iu z} + e^{-\iu z})
= \frac{1}{2} \left[e^{(1+\iu)z} + e^{(1-\iu)z}\right],
\]利用\(e^z\)的展开式 \labelcref{equation:解析函数的级数表示.指数函数的泰勒展式},得\begin{align*}
e^z \cos z
&= \frac{1}{2} \left[
\sum\limits_{n=0}^\infty \frac{(1+\iu)^n}{n!} z^n
+ \sum\limits_{n=0}^\infty \frac{(1-\iu)^n}{n!} z^n
\right] \\
&= \frac{1}{2} \sum\limits_{n=0}^\infty \frac{1}{n!} [(1+\iu)^n+(1-\iu)^n] z^n
\quad(\abs{z}<+\infty).
\end{align*}
由于\(1+\iu=\sqrt{2} e^{\frac{\pi\iu}{4}}\),\(1-\iu=\sqrt{2} e^{-\frac{\pi\iu}{4}}\),代入上式得\begin{equation}
\begin{split}
e^z \cos z
&= \frac{1}{2} \sum\limits_{n=0}^\infty \frac{(\sqrt{2})^n}{n!} (e^{\frac{n\pi}{4}\iu}+e^{-\frac{n\pi}{4}\iu}) z^n \\
&= \sum\limits_{n=0}^\infty \frac{(\sqrt{2})^n \cos\frac{n\pi}{4}}{n!} z^n
\quad(\abs{z}<+\infty).
\end{split}
\end{equation}

(4)可用与(3)同样的方法求解.
但由于\(e^z\)及\(\sin z\)的已知泰勒展式均在\(\abs{z}<+\infty\)内是绝对收敛级数,故其柯西乘积也绝对收敛,所以我们用级数的乘法运算求解\(e^z \sin z\)较为方便.
由\[
e^z = \sum\limits_{n=0}^\infty \frac{z^n}{n!},
\qquad
\sin z = \sum\limits_{n=0}^\infty \frac{(-1)^n z^{2n+1}}{(2n+1)!},
\]可按对角线方法得出\[
e^z \sin z
= z + z^2 + \frac{1}{3} z^3 - \frac{1}{30} z^5 + \dotsb
\quad(\abs{z}<+\infty).
\]
\end{solution}
\end{example}

\subsection{解析函数零点的孤立性及唯一性定理}
下面利用泰勒展式研究解析函数的零点.
\begin{definition}\label{definition:解析函数的级数表示.零点}
设函数\(f(z)\)在点\(a\)的某邻域\(N(a)\)内解析.
若\(f(a)=0\),则称\(a\)是解析函数\(f(z)\)的\DefineConcept{零点}.
若\(f(a)=f'(a)=f''(a)=\dotsb=f^{(m-1)}(a)=0\),但\(f^{(m)}(a)\neq0\),则称\(a\)是解析函数\(f(z)\)的\(m\)阶零点.
特别地,\(m=1\)阶零点\(a\)(即\(f'(a)=0\)),又称为\(f(z)\)的简单零点.
\end{definition}

若\(f(z)\)在邻域\(N(a)\)内解析,不恒为零,则\(a\)为\(f(z)\)的\(m\)阶零点时,\(f(z)\)在\(N(a)\)内的泰勒展式形式为\[
f(z) = c_m (z-a)^m + c_{m+1} (z-a)^{m+1} + \dotsb \quad(c_m\neq0).
\]右端提取公因式\((z-a)^m\),并记\[
\phi(z) = c_m + c_{m+1} (z-a) + \dotsb,
\]容易证明\(\phi(z)\)在邻域\(N(a)\)内解析,且\(\phi(a)\neq0\).
于是在邻域\(N(a)\)内有\[
f(z) = (z-a)^m \phi(z).
\]反之,当\(f(z)\)在\(N(a)\)内解析,并能表成上述形式时,由泰勒定理中的关系式\[
c_n = \frac{1}{n!} f^{(n)}(a),
\]立即可知\(a\)是\(f(z)\)的\(m\)阶零点.
于是有下述定理:
\begin{theorem}\label{theorem:解析函数的级数表示.零点定理}
“不恒为零的解析函数\(f(z)\)以\(a\)为\(m\)阶零点”的充要条件是:
在\(a\)点的邻域\(N(a)\)内有\[
f(z) = (z-a)^m \phi(z)
\]成立,其中\(\phi(z)\)在邻域\(N(a)\)内解析,且\(\phi(a)\neq0\).
\end{theorem}

一个实变可微函数的零点不一定是孤立的.例如实变函数\[
f(x) = \left\{ \begin{array}{cl}
x^2 \sin\frac{1}{x}, & x\neq0, \\
0, & x=0
\end{array} \right.
\]在点\(x=0\)可微,且以\(x=0\)为一个零点.
但点列\(x = \pm\frac{1}{n\pi}\ (n=1,2,\dotsc)\)也是它的零点,并以\(x = 0\)为聚点,所以\(x = 0\)不是一个孤立的零点.

但在复变函数中,我们有如下定理:
\begin{theorem}\label{theorem:解析函数的级数表示.解析函数的零点的孤立性}
不恒为零的解析函数的零点必是孤立的.
\begin{proof}
设\(a\)为解析函数\(f(z)\)的\(m\)阶零点,则由\cref{theorem:解析函数的级数表示.零点定理},在邻域\(N_R(a)\)内有\[
f(z) = (z-a)^m \phi(z),
\]其中\(\phi(z)\)在\(N_R(a)\)内解析,且\(\phi(a)\neq0\).
从而由\(\phi(z)\)在点\(a\)的连续性可知,必定存在邻域\(N_r(a)\),在\(N_r(a)\)内\(\phi(z)\neq0\).
故\(f(z)\)在\(N_r(a)\)内没有异于点\(a\)的其他零点.
\end{proof}
\end{theorem}

由上可得以下推论:
\begin{corollary}
设函数\(f(z)\)在邻域\(N_R(a)\)内解析,且在\(N_R(a)\)内有\(f(z)\)的一列零点\(\{z_n\}\ (z_n \neq a)\)收敛于\(a\),则\(f(z)\)在\(N_R(a)\)内必恒为零.
\end{corollary}

\begin{theorem}[唯一性定理]\label{theorem:解析函数的级数表示.唯一性定理}
设函数\(f_1(z)\)和\(f_2(z)\)在区域\(D\)内解析,且在\(D\)内有一个收敛于\(a \in D\)的点列\(\{z_n\}\ (z_n \neq a)\),且\(f_1(z_n) = f_2(z_n)\),则在\(D\)内有\(f_1(z) \equiv f_2(z)\).
\end{theorem}

若\cref{theorem:解析函数的级数表示.唯一性定理} 中的点列\(\{z_n\}\)取\(D\)的一个子区域(或一段弧),定理的结论自然仍就成立.
从而一切在实轴上成立的恒等式,只要恒等式两边的函数在复平面\(\mathbb{C}\)上解析,则恒等式在整个复平面上就成立.

对于一个不加限制条件的分布函数,我们不能从其定义域中某一部分的取值情况来确定其他部分的值.
对于连续函数也只能说,相邻两点的函数值相差很小.
对于解析函数来说就完全不同了.
从上面的\hyperref[theorem:解析函数的级数表示.唯一性定理]{唯一性定理}我们看到,解析函数在其解析区域中某点邻域内的取值情况决定着它在其他部分的值,即在区域\(D\)内的局部值决定了函数在区域\(D\)内整体的值.
以前由\hyperref[equation:解析函数的积分表示.柯西积分公式]{柯西积分公式},曾经使我们知道,从解析函数在区域边界\(C\)上的值可以确定它在\(C\)的内部的一切值.
现在又知道,解析函数在区域内部的局部值确定了区域内整体的值.
因此\hyperref[theorem:解析函数的级数表示.唯一性定理]{唯一性定理}可以看成\hyperref[equation:解析函数的积分表示.柯西积分公式]{柯西积分公式}的补充定理,都揭示了解析函数的本质特性,都是解析函数论中最基本的定理.

\section{解析函数的罗朗展开式}

\subsection{罗朗级数、罗朗定理}
%@see: https://mathworld.wolfram.com/LaurentSeries.html
\begin{definition}
形如\begin{equation}\label{equation:解析函数的级数表示.罗朗级数}
\begin{split}
\sum\limits_{n=-\infty}^\infty c_n (z-a)^n
&\defeq \sum\limits_{n=-\infty}^{-1} c_n (z-a)^n + \sum\limits_{n=0}^\infty c_n (z-a)^n \\
&\defeq \sum\limits_{n=1}^\infty c_{-n} (z-a)^{-n} + \sum\limits_{n=0}^\infty c_n (z-a)^n
\end{split}
\end{equation}的级数称为\DefineConcept{罗朗级数},其中\(a\)及\(c_n\ (n=0,\pm1,\dotsc)\)是复常数.
\end{definition}
显然,当\(c_{-n}=0\ (n=1,2,\dotsc)\)时,形式上罗朗级数 \labelcref{equation:解析函数的级数表示.罗朗级数} 就化为了幂级数.
可以说罗朗级数是由两个幂级数组成的.
若级数\[
\sum\limits_{n=0}^\infty c_n (z-a)^n
\quad\text{和}\quad
\sum\limits_{n=1}^\infty c_{-n} (z-a)^{-n}
\]都在点\(z_0\)收敛,则称罗朗级数 \labelcref{equation:解析函数的级数表示.罗朗级数} 在点\(z_0\)收敛.
记罗朗级数 \labelcref{equation:解析函数的级数表示.罗朗级数} 的和函数为\(f(z)\).
称级数\[
\sum\limits_{n=0}^\infty c_n (z-a)^n
\]为\(f(z)\)在点\(a\)的\DefineConcept{解析部分}或\DefineConcept{正则部分},它的和函数记为\(\phi(z)\);称级数\[
\sum\limits_{n=1}^\infty c_{-n} (z-a)^{-n}
\]为\(f(z)\)在点\(a\)的\DefineConcept{主要部分}或\DefineConcept{奇异部分},它的和函数记为\(\psi(z)\).

设级数\(\sum\limits_{n=0}^\infty c_n (z-a)^n\)的收敛半径为\(R \in (0,+\infty)\),则它在圆\(\abs{z-a}<R\)内绝对收敛、内闭一致收敛,它的和函数\(\phi(z)\)在圆\(\abs{z-a}<R\)内解析.

记\(\zeta = \frac{1}{z-a}\).
设级数\(\sum\limits_{n=1}^\infty c_{-n} (z-a)^{-n} = \sum\limits_{n=1}^\infty c_{-n} \zeta^n\)的收敛半径为\(\lambda \in (0,+\infty)\),则它在圆\(\abs{\zeta} < \lambda\)内绝对收敛、内闭一致收敛.
换言之,级数\(\sum\limits_{n=1}^\infty c_{-n} (z-a)^{-n}\)在\(r = \frac{1}{\lambda} < \abs{z-a} < +\infty\)内绝对收敛、内闭一致收敛,它的和函数\(\psi(z)\)在\(r < \abs{z-a} < +\infty\)内解析.

结合上面的论述,我们来讨论一下级数 \labelcref{equation:解析函数的级数表示.罗朗级数} 的敛散性:
\begin{enumerate}
\item 当\(r > R\)时,级数 \labelcref{equation:解析函数的级数表示.罗朗级数} 处处发散;

\item 当\(r = R\)时,级数 \labelcref{equation:解析函数的级数表示.罗朗级数} 在区域\(\abs{z-a} \neq R\)上是发散的,而它在圆\(\abs{z-a}=R\)上则有三种可能性:\begin{enumerate}
\item 它在圆上处处收敛.
例如级数\(\sum\limits_{\substack{n=-\infty \\ n\neq0}}^\infty \frac{z^n}{n^2}\)在圆\(\abs{z}=1\)上处处收敛;

\item 它在圆上处处发散.
例如级数\(\sum\limits_{n=-\infty}^\infty z^n\)在圆\(\abs{z}=1\)上处处发散;

\item 它在圆上有的点收敛,有的点发散.
例如级数\(\sum\limits_{\substack{n=-\infty \\ n\neq0}}^\infty \frac{z^n}{n}\)在圆\(\abs{z}=1\)上除点\(z=1\)外处处收敛;
\end{enumerate}

\item 当\(r < R\)时,级数 \labelcref{equation:解析函数的级数表示.罗朗级数} 在圆环\(H: r < \abs{z-a} < R\)内绝对收敛且内闭一致收敛,在\(H\)外发散.
圆环\(H\)称为罗朗级数的\DefineConcept{收敛圆环}.

特别地,当\(r = 0\)且\(R = +\infty\)时,级数 \labelcref{equation:解析函数的级数表示.罗朗级数} 在复平面\(\mathbb{C}\)上除点\(a\)外处处收敛.

根据\hyperref[theorem:解析函数的级数表示.魏尔斯特拉斯定理]{魏尔斯特拉斯定理},罗朗级数 \labelcref{equation:解析函数的级数表示.罗朗级数} 的和函数在其收敛圆环\(H\)内是解析的,且可在\(H\)内逐项求任意阶导数.
\end{enumerate}

现在我们把罗朗级数 \labelcref{equation:解析函数的级数表示.罗朗级数} 在点\(a\)的解析部分与主要部分的和函数\(\phi(z)\)与\(\psi(z)\)分别表述为\[
\def\arraystretch{2}
\begin{array}{ll}
\phi(z) = \sum\limits_{n=0}^\infty c_n (z-a)^n & (\abs{z-a}<R), \\
\psi(z) = \sum\limits_{n=1}^\infty c_{-n} (z-a)^{-n} & (r < \abs{z-a} < +\infty).
\end{array}
\]显然,\(\phi(z)\)在\(\abs{z-a}<R\)内解析,\(\psi(z)\)在\(r<\abs{z-a}<+\infty\)内解析.
当\(r \leq R\)时,罗朗级数 \labelcref{equation:解析函数的级数表示.罗朗级数} 的和函数\(f(z)\)满足\[
f(z) = \phi(z) + \psi(z),
\]且\(f(z)\)在\(r<\abs{z-a}<R\)内解析.

综上所述,我们有以下定理.
\begin{theorem}
设罗朗级数 \labelcref{equation:解析函数的级数表示.罗朗级数} 的收敛圆环为\(H: r < \abs{z-a} < R\),则它在\(H\)内绝对收敛且内闭一致收敛,它的和函数\(f(z)\)在\(H\)内解析,且\[
f(z) = \sum\limits_{n=-\infty}^\infty c_n (z-a)^n
\]在\(H\)内可逐项求任意阶导数.
\end{theorem}

上述定理的逆定理也成立.
\begin{theorem}[罗朗定理]\label{theorem:解析函数的级数表示.罗朗定理}
在圆环\(H: 0 \leq r < \abs{z-a} < R < +\infty\)内解析的函数\(f(z)\)必定可以展成罗朗级数\begin{equation}\label{equation:解析函数的级数表示.罗朗展式}
f(z) = \sum\limits_{n=-\infty}^\infty c_n (z-a)^n,
\end{equation}其中\begin{equation}\label{equation:解析函数的级数表示.罗朗系数}
c_n = \frac{1}{2\pi\iu} \int_{\abs{z-a}=\rho} \frac{f(\zeta)}{(\zeta-a)^{n+1}} \dd{\zeta}
\quad(r<\rho<R).
\end{equation}
并且展式 \labelcref{equation:解析函数的级数表示.罗朗展式} 是唯一的(即\(f(z)\)和圆环\(H\)唯一地决定了系数\(c_n\)).
\end{theorem}
\cref{equation:解析函数的级数表示.罗朗展式} 称为函数\(f(z)\)在圆环\(H\)内的\DefineConcept{罗朗展式},\cref{equation:解析函数的级数表示.罗朗系数} 称为该展式的\DefineConcept{罗朗系数}.

在\cref{theorem:解析函数的级数表示.罗朗定理} 中,当给定函数\(f(z)\)在点\(a\)解析时,收敛圆环\(H\)就退化成收敛圆\(K: \abs{z-a}<R\).
这时,罗朗定理就退化为\hyperref[theorem:解析函数的级数表示.泰勒定理]{泰勒定理},而罗朗系数 \labelcref{equation:解析函数的级数表示.罗朗系数} 就是泰勒系数.
也只有这时,罗朗系数除了有积分形式以外,还有微分形式\[
c_n = \frac{f^{(n)}(a)}{n!}.
\]也只有这时,罗朗级数才退化为泰勒级数.
因此,泰勒级数是罗朗级数的特殊情形,即\(c_{-n} = 0\ (n=1,2,\dotsc)\)的情形.

在求一些初等函数的罗朗展式时,一般来说,不采用通过罗朗系数公式 \labelcref{equation:解析函数的级数表示.罗朗系数} 计算\(c_n\)的“直接法”,而主要采用“间接法”,即根据罗朗展式的唯一性,通过各种代数运算或分析运算以及变量代换等方法,应用已知的一些初等函数的泰勒展式,来求出所给函数的罗朗展式.
所以,把函数展开成罗朗级数时,泰勒级数仍然是基础.

\begin{example}
求函数\(f(z) = \frac{1}{(z-1)(z-2)}\)在适当区域内的展式.
\begin{solution}
\(f(z)\)在\(z\)平面上只有两个奇点:\(z=1\)及\(z=2\).
因此,就\(f(z)\)的解析区域来说,\(z\)平面可分成如下三个互不相交的解析区域:\begin{enumerate}
\item 圆\(D_1: \abs{z}<1\);
\item 圆环\(D_2: 1<\abs{z}<2\);
\item 圆环\(D_3: 2<\abs{z}<+\infty\).
\end{enumerate}
其中\(D_1\)是单连通区域,\(D_2\)、\(D_3\)都是以\(z=0\)为中心的不同的圆环.
下面在这三个区域内分别求\(f(z)\)的展开式.

首先将\(f(z)\)分解成部分分式:\[
f(z) = \frac{1}{z-2} - \frac{1}{z-1}.
\]那么\begin{enumerate}
\item 在\(D_1\)内,\(\abs{\frac{z}{2}}<1\),利用几何级数公式得\[
f(z) = \frac{1}{1-z} - \frac{1}{2 (1-z/2)}
= \sum\limits_{n=0}^\infty \left(1 - \frac{1}{2^{n+1}}\right) z^n,
\]这就是\(f(z)\)在\(D_1\)内的泰勒展式.

\item 在\(D_2\)内,\(\abs{\frac{1}{z}}<1\),\(\abs{\frac{z}{2}}<1\),\[
\begin{split}
f(z) &= -\frac{1}{2}\cdot\frac{1}{1-z/2} - \frac{1}{z}\cdot\frac{1}{1-1/z} \\
&= -\frac{1}{2} \sum\limits_{n=0}^\infty \frac{z^n}{2^n} - \frac{1}{z} \sum\limits_{n=1}^\infty \frac{1}{z^{n-1}} \\
&= -\sum\limits_{n=0}^\infty \frac{z^n}{2^{n+1}} - \sum\limits_{n=1}^\infty \frac{1}{z^n},
\end{split}
\]这就是\(f(z)\)在\(D_2\)内的罗朗展式.

\item 在\(D_3\)内,\(\abs{\frac{1}{z}}<1\),\(\abs{\frac{2}{z}}<1\),故\[
\begin{split}
f(z) &= \frac{1}{z}\cdot\frac{1}{1-2/z} - \frac{1}{z}\cdot\frac{1}{1-1/z} \\
&= \frac{1}{z} \sum\limits_{n=0}^\infty \frac{2^n}{z^n} - \frac{1}{z} \sum\limits_{n=0}^\infty \frac{1}{z^n} \\
&= \sum\limits_{n=2}^\infty \frac{2^{n-1}-1}{z^n},
\end{split}
\]这就是\(f(z)\)在\(D_3\)内的罗朗展式.
\end{enumerate}
\end{solution}
由此例我们看到,当函数\(f(z)\)及其解析的圆环取定以后,根据罗朗展式的唯一性,虽然其罗朗展式的系数\(c_n\ (n=0,\pm1,\dotsc)\)是唯一确定的,但计算\(c_n\)的方法并不唯一,不一定非得用\cref{equation:解析函数的级数表示.罗朗系数} 来计算罗朗系数.
\(c_n\)的唯一性保证我们可以用任何简便的方法来计算\(c_n\).
另外,此例还表明:同一函数在不同圆环内的罗朗展式不同.
\end{example}

\subsection{在孤立奇点去心邻域内的罗朗展式}
\begin{definition}
若\(f(z)\)在点\(a\)的某一去心邻域内解析,但在点\(a\)不解析,则称\(a\)为\(f(z)\)的\DefineConcept{孤立奇点}.
若\(a\)是\(f(z)\)的一个奇点,且在点\(a\)的任意邻域内\(f(z)\)总还有除点\(a\)以外的其他奇点,则称点\(a\)为\(f(z)\)的\DefineConcept{非孤立奇点}.
\end{definition}

\begin{example}
点\(z=0\)是函数\(f(z) = \frac{1}{z}\)的孤立奇点,是函数\(g(z) = \frac{1}{\sin(1/z)}\)的非孤立奇点.
\end{example}

\begin{example}
函数\(f(z) = \frac{1}{(z-1)(z-2)}\)在\(z\)平面只有两个孤立奇点\(z=1\)和\(z=2\).
分别求\(f(z)\)在这两个点的去心邻域内的罗朗展式.
\begin{solution}
在去心邻域\(0<\abs{z-1}<1\)内\[
f(z) = -\frac{1}{z-1} + \frac{1}{(z-1)-1} \\
= -\frac{1}{z-1} - \sum\limits_{n=0}^\infty (z-1)^n.
\]

在去心邻域\(0<\abs{z-2}<1\)内\[
f(z) = \frac{1}{z-2} - \frac{1}{1+(z-2)}
= \frac{1}{z-2} - \sum\limits_{n=0}^\infty (-1)^n (z-2)^n.
\]
\end{solution}
\end{example}

在用间接法进行罗朗展开时,常常要用到罗朗级数的加法和乘法:
\begin{theorem}[罗朗级数的加法]
设\(F(z)\)在环域\(H: r < \abs{z-a} < R\)内解析,且\(F(z) = f(z) + g(z)\),\(f(z)\)与\(g(z)\)在环域\(H\)内的罗朗展式分别为\[
f(z) = \sum\limits_{n=-\infty}^\infty a_n (z-a)^n
\quad\text{和}\quad
g(z) = \sum\limits_{n=-\infty}^\infty b_n (z-a)^n,
\]则在\(H\)内\[
F(z) = \sum\limits_{n=-\infty}^\infty (a_n+b_n) (z-a)^n.
\]
\end{theorem}

\begin{theorem}[罗朗级数的乘法]
设\(F(z) = f(z) g(z)\)在环域\(H: r < \abs{z-a} < R\)内解析,且\(f(z)\)与\(g(z)\)在\(H\)内的罗朗展式分别为\[
f(z) = \sum\limits_{n=-\infty}^\infty a_n (z-a)^n
\quad\text{和}\quad
g(z) = \sum\limits_{n=-\infty}^\infty b_n (z-a)^n,
\]则在\(H\)内\[
F(z) = \sum\limits_{n=-\infty}^\infty c_n (z-a)^n,
\]其中\[
c_n = \sum\limits_{k=-\infty}^\infty a_k b_{n-k},
\quad n=0,\pm1,\dotsc.
\]
\end{theorem}

\begin{example}
\def\fe{e^{\frac{t}{2} \left(z-\frac{1}{z}\right)}}
设\(t\)是实参数,求函数\(f(z) = \fe\)在点\(z=0\)的罗朗展式.
\def\s#1{\sum\limits_{#1}^\infty }%
\def\sk{\s{k=0} \frac{1}{k!} \left(\frac{t}{2}\right)^k z^k}%
\def\sl{\s{l=0} \frac{1}{l!} \left(-\frac{t}{2}\right)^l \left(\frac{1}{z}\right)^l}%
\begin{solution}
除\(z=0\)外,\(f\)在\(z\)平面解析,在去心邻域\(0<\abs{z}<+\infty\)内,\[
\fe = \left[ \sk \right] \cdot \left[ \sl \right].
\eqno(1)
\]
记\(A = \sk\),\(B = \sl\).

对于任意固定的\(t\),
(1)式右边的两个级数当\(\abs{z}>0\)都绝对收敛,
所以应用罗朗级数的乘法公式,可以任意方式合并同幂项.

为了得到乘积中某个正幂项\(z^n\ (n\geq0)\),
应将\(B\)中的各项分别与\(A\)中的第\(k=l+n\)项相乘,即得\[
\s{n=0} \left[ \s{l=0} \frac{(-1)^l}{l!(l+n)!} \left(\frac{t}{2}\right)^{2l+n} \right] z^n;
\]

而为了得到乘积中某个负幂项\(z^{-m}\ (m>0)\),
应将\(A\)中的各项分别与\(B\)中的第\(l=k+m\)项相乘,
即得\[
\s{m=1} \left[ (-1)^m \s{k=0} \frac{(-1)^k}{k! (k+m)!} \left(\frac{t}{2}\right)^{2k+m} \right] z^{-m}.
\]把\(-m\)改记为\(n\),\(k\)改记为\(l\),则\begin{align*}
\fe &= \s{n=0} \left[ \s{l=0} \frac{(-1)^l}{l!(l+n)!} \left(\frac{t}{2}\right)^{2l+n} \right] z^n \\
&\hspace{20pt} \sum\limits_{n=-1}^{-\infty} \left[ (-1)^n \s{l=0} \frac{(-1)^l}{l!(l-n)!} \left(\frac{t}{2}\right)^{2l-n} \right] z^n \\
&= \sum\limits_{n=-\infty}^\infty J_n(t) z^n,
\end{align*}其中\begin{equation}
J_n(t) = \left\{ \begin{array}{rl}
\s{l=0} \frac{(-1)^l}{l!(l+n)!} \left(\frac{t}{2}\right)^{2l+n}, & n=0,1,2,\dotsc, \\
(-1)^n \s{l=0} \frac{(-1)^l}{l!(l-n)!} \left(\frac{t}{2}\right)^{2l-n}, & n=-1,-2,\dotsc.
\end{array} \right.
\end{equation}
可以说\[
J_{-n}(t) = (-1)^n J_n(t), \quad n=1,2,\dotsc.
\]

\(J_n(t)\)和\(J_{-n}(t)\)分别称为\(n\)阶及\(-n\)阶\DefineConcept{贝塞尔函数}.
使用达朗贝尔公式,不难求得表示\(J_n(t)\)的幂级数的收敛半径为\(+\infty\).
这样,如果把\(t\)看成复变数,则\(J_n(t)\)和\(J_{-n}(t)\)都在全平面\(\mathbb{C}\)上解析.
\end{solution}
\end{example}

\section{解析函数在孤立奇点附近的性态}
\subsection{有限孤立奇点的情形}
\subsubsection{孤立奇点的分类}
现在我们利用罗朗展开对解析函数的孤立奇点进行分类,并讨论解析函数在各类孤立奇点附近的性态.
设\(a\)为函数\(f(z)\)的有限孤立奇点,\(f(z)\)在去心邻域\(0<\abs{z-a}<R\)内的罗朗展式为\[
f(z) = \sum\limits_{n=-\infty}^\infty c_n (z-a)^n
= \sum\limits_{n=0}^\infty c_n (z-a)^n
+ \sum\limits_{n=1}^\infty c_{-n} (z-a)^{-n}.
\]我们已经知道,级数\(\sum\limits_{n=0}^\infty c_n (z-a)^n\)称为\(f(z)\)在点\(a\)的解析部分,其和函数\(\phi(z)\)在包括点\(a\)的邻域\(\abs{z-a}<R\)内是解析的,故\(f(z)\)在点\(a\)的奇异性质完全体现在\(f(z)\)的罗朗展式的复幂项部分.
这就是为什么级数\(\sum\limits_{n=1}^\infty c_{-n} (z-a)^{-n}\)又被称为奇异部分.

现在根据奇异部分仅可能出现的三种情况,将\(f(z)\)的有限孤立奇点作如下分类:
\begin{definition}
设\(a\)是\(f(z)\)的有限孤立奇点.
\begin{enumerate}
\item 若\(f(z)\)在点\(a\)的奇异部分为零,则称\(a\)为\(f(z)\)的\DefineConcept{可去奇点}.

{\footnotesize
函数\(f(z)\)在可去奇点的去心邻域内的罗朗展式只有非负幂项的解析部分\[
c_0 + c_1 (z-a) + c_2 (z-a)^2 + \dotsb,
\]而这解析部分的和函数\(\phi(z)\)在包括点\(a\)在内的邻域\(K\)内解析,且\(\phi(a) = c_0\).
于是,在这去心邻域内,\(f(z) = \phi(z)\).
只要我们适当补充或改变\(f(z)\)在点\(a\)的定义,使得\(f(a) = \phi(a)\),那么\(f(z)\)在邻域\(N_R(a)\)内就没有奇点了.
}

\item 若\(f(z)\)在点\(a\)的奇异部分为有限多项,并假设最小复幂项为\((z-a)\)的\((-m)\)次幂项,即\[
\frac{c_{-m}}{(z-a)^m} + \frac{c_{-(m-1)}}{(z-a)^{m-1}} + \dotsb + \frac{c_{-1}}{z-a}
\quad(c_{-m}\neq0),
\]则称\(a\)是\(f(z)\)的\(m\)阶\DefineConcept{极点}.
特别地,一阶极点又称为\DefineConcept{简单极点}.

\item 若\(f(z)\)在点\(a\)的奇异部分有无限多项,则称\(a\)为\(f(z)\)的\DefineConcept{本性奇点}.
\end{enumerate}
\end{definition}

\begin{example}
点\(z=0\)是函数\(f(z) = \frac{\sin z}{z}\)的可去奇点.
只要补充定义\(f(z) = \lim\limits_{z\to0} \frac{\sin z}{z} = 1\),即规定\[
f(z) = \left\{ \begin{array}{cc}
\frac{\sin z}{z}, & z\neq0, \\
1, & z=0,
\end{array} \right.
\]则\(f(z)\)在\(z=0\)就解析了.
\end{example}

\subsubsection{可去奇点的判定}
\begin{theorem}\label{theorem:解析函数的级数表示.可去奇点的特征}
若点\(a\)为\(f(z)\)的孤立奇点,则下列几个命题是等价的:
\begin{enumerate}
\item 点\(a\)是\(f(z)\)的可去奇点,\(f(z)\)在点\(a\)的奇异部分为零;
\item \(\lim\limits_{z \to a} f(z) = c_0\ (\abs{c_0} < +\infty)\);
\item \(f(z)\)在点\(a\)的某去心邻域内有界.
\end{enumerate}
\end{theorem}

\subsubsection{极点的判定}
\begin{theorem}\label{theorem:解析函数的级数表示.极点的特征}
若\(a\)为\(f(z)\)的孤立奇点,则下列几个命题是等价的:
\begin{enumerate}
\item 点\(a\)是\(f(z)\)的\(m\)阶极点,\(f(z)\)在点\(a\)的奇异部分为\[
\frac{c_{-m}}{(z-a)^m} + \frac{c_{-(m-1)}}{(z-a)^{m-1}} + \dotsb + \frac{c_{-1}}{z-a}
\quad(c_{-m}\neq0);
\]

\item \(f(z)\)在点\(a\)的某去心邻域内能表成\[
f(z) = \frac{\lambda(z)}{(z-a)^m},
\]其中\(\lambda(z)\)在点\(a\)的邻域内解析,且\(\lambda(a)\neq0\);

\item 点\(a\)是\(g(z) = \frac{1}{f(z)}\)的可去奇点,将\(a\)作为\(g(z)\)的解析点看待时,点\(a\)为\(g(z)\)的\(m\)阶零点.
\end{enumerate}
\end{theorem}

\begin{corollary}\label{theorem:解析函数的级数表示.孤立奇点成为极点的充要条件1}
\(f(z)\)的孤立奇点\(a\)是极点的充要条件是:\[
\lim\limits_{z \to a} f(z) = \infty.
\]
\begin{proof}
“点\(a\)是\(f(z)\)的(\(m\)阶)极点”等价于“点\(a\)是\(g(z) = \frac{1}{f(z)}\)的(\(m\)阶)零点”,也就等价于“\(\lim\limits_{z \to a} f(z) = \infty\)”.
\end{proof}
\end{corollary}
\cref{theorem:解析函数的级数表示.孤立奇点成为极点的充要条件1} 的缺点是不能指明极点的阶数.

\subsubsection{本性奇点的判定}
\begin{theorem}
\(f(z)\)的孤立奇点\(a\)是本性奇点的充要条件是:
不存在有限或无限的极限\(\lim\limits_{z \to a} f(z)\).
\end{theorem}

\subsubsection{皮卡定理}
\begin{theorem}[皮卡定理]
解析函数在本性奇点的去心邻域内无穷多次地取到每个有限复值,至多可能除一个值(称为\DefineConcept{皮卡例外值}).
\end{theorem}
皮卡定理深刻、准确地揭示了解析函数在其本性奇点邻近取值的奇异性.
上述定理还有另一种表述方式,即“任何不为常数的整函数,除去可能的一个值外,取遍所有(有限)的复数值.”

又因为整函数是在复平面上不取无穷远点的亚纯函数,从而皮卡定理可以推广到任意的亚纯函数:\begin{theorem}
\(\mathbb{C}\)上的任意不为常数的亚纯函数,除去可能的两个值外,取遍\(\overline{\mathbb{C}}\)的所有值.
\end{theorem}

\begin{example}
点\(z=0\)是函数\(f(z) = e^{1/z}\)的一个孤立奇点.
在去心邻域\(0<\abs{z}<+\infty\)内,\(f(z) = e^{1/z}\)的罗朗展式为\[
e^{1/z} = 1 + \frac{1}{z} + \frac{1}{2!} \frac{1}{z^2} + \dotsb + \frac{1}{n!} \frac{1}{z^n} + \dotsb.
\]该展开式中含有无穷多个负次幂项,因此\(z=0\)是\(e^{1/z}\)的本性奇点.
又因为\(e^{1/z}\neq0\),所以\(z=0\)是\(e^{1/z}\)的皮卡例外值.
对任意有限非零复数\(A\),若取\(z_n = \frac{1}{\ln A + 2n\pi\iu}\),则\(\lim\limits_{n\to\infty} z_n = 0\),且\[
f(z_n) = A
\quad(n=1,2,\dotsc).
\]这表示在\(z=0\)附近,\(f(z) = e^{1/z}\)可以无穷多次地取到事先给定的任何有限非零复数\(A\).
于是可以想见函数\(e^{1/z}\)在\(z=0\)附近取值的奇异性了.
\end{example}

总之,上述定理,不仅反映了解析函数在孤立奇点附近的性态,同时也指出了判别奇点类型的方法.
下面就如何使用它们判别奇点类型举几个例子.

\begin{example}
证明\(z=0\)是\(f(z) = \frac{1}{z} (e^z-1)\)的可去奇点.
\begin{proof}[证法一]
用罗朗展开证明.

显然,\(z=0\)是\(f(z) = \frac{1}{z} (e^z-1)\)的孤立奇点.
在去心邻域\(0<\abs{z}<+\infty\)内将\(f(z)\)作罗朗展开,得\[
\frac{e^z-1}{z}
= \frac{1}{z} \left( \sum\limits_{n=0}^\infty \frac{z^n}{n!} - 1 \right)
= \frac{1}{z} \sum\limits_{n=1}^\infty \frac{z^n}{n!}
= \sum\limits_{n=1}^\infty \frac{z^{n-1}}{n!}
= 1 + \frac{z}{2!} + \frac{z^2}{3!} + \dotsb,
\]由此可见\(\frac{1}{z} (e^z-1)\)在点\(z=0\)的奇异部分为零,也就是说点\(z=0\)是它的可去奇点.
\end{proof}
\begin{proof}[证法二]
用极限关系证明.

由于\(\lim\limits_{z\to0} \frac{1}{z} (e^z-1) = \lim\limits_{z\to0} e^z = 1\)是有限复数,那么根据\cref{theorem:解析函数的级数表示.可去奇点的特征},\(z=0\)就是\(\frac{e^z-1}{z}\)的可去奇点.
\end{proof}
\end{example}

\subsection{解析函数在无穷远点附近的性态}
由于函数\(f(z)\)在无穷远点\(\infty\)总是没有定义的,所以无穷远点\(\infty\)总是函数\(f(z)\)的奇点.

\begin{definition}
若函数\(f(z)\)在无穷远点\(\infty\)的去心邻域\[
0 \leq r < \abs{z} < +\infty
\]内解析,则称点\(\infty\)为\(f(z)\)的一个孤立奇点.
若点\(\infty\)是\(f(z)\)的奇点的聚点,则称点\(\infty\)为\(f(z)\)的非孤立奇点.
\end{definition}

设点\(\infty\)是\(f(z)\)的孤立奇点,则作倒变换\(z = 1/\zeta\),得函数\[
\phi(\zeta) = f(1/\zeta) = f(z)
\]在\(\zeta\)平面上原点\(\zeta=0\)的去心邻域\(D: 0<\abs{\zeta}<\frac{1}{r}\)内解析,\(\zeta=0\)为函数\(\phi(\zeta)\)的孤立奇点.
于是,通过对函数\(\phi(\zeta)\)在有限孤立奇点\(\zeta=0\)的去心邻域\(D\)内性态的讨论,我们就可以得到函数\(f(z)\)在无穷远点\(\infty\)附近的性态.
\begin{definition}\label{definition:解析函数的级数表示.无穷远处孤立奇点的分类}
若\(\zeta=0\)为\(\phi(\zeta)=f(1/\zeta)\)的可去奇点(视为解析点),则称\(z=\infty\)为\(f(z)\)的可去奇点(解析点);
若\(\zeta=0\)为\(\phi(\zeta)\)的\(m\)阶极点,则称\(z=\infty\)为\(f(z)\)的\(m\)阶极点;
若\(\zeta=0\)为\(\phi(\zeta)\)的本性奇点,则称\(z=\infty\)为\(f(z)\)的本性奇点.
\end{definition}

前面在介绍分式线性变换时,我们曾按广义连续性来定义函数在点\(\infty\)处的值:
定义\(f(\infty) = \lim\limits_{z\to\infty} f(z)\).
但在点\(\infty\)没有定义差商,因此我们没有定义函数在无穷远点处的可微性.
现在依据\cref{definition:解析函数的级数表示.无穷远处孤立奇点的分类},我们可以将“函数\(f(z)\)在点\(\infty\)解析”定义为:
点\(\infty\)为\(f(z)\)的可去奇点,且定义\(f(\infty) = \lim\limits_{z\to\infty} f(z)\).

设由方程\[
\phi(\zeta) = f(1/\zeta) = f(z)
\]确定的函数\(\phi(\zeta)\)在去心邻域\(0<\abs{\zeta}<1/r\)内的罗朗展式为\[
\phi(\zeta) = \sum\limits_{n=-\infty}^\infty c_n \zeta^n.
\]将变量\(\zeta\)换为\(z\),就得到\(f(z)\)在无穷远点去心邻域\(r<\abs{z}<+\infty\)内的罗朗展式\[
f(z) = \sum\limits_{n=-\infty}^\infty c_n (1/z)^n
= \sum\limits_{n=-\infty}^\infty b_n z^n,
\]其中\(b_n = c_{-n}\ (n=0,\pm1,\pm2,\dotsc)\).
也就是说,\(\phi(\zeta)\)在原点去心邻域的展式中的复幂项系数,与\(f(z)\)在无穷远点去心邻域的展式中的相应正幂项系数相等,而前者展式中的正幂项系数与后者相应复幂项系数相等.
根据这个关系,应用对有限孤立奇点的讨论结果,我们知道:
\begin{enumerate}
\item \(z=\infty\)是\(f(z)\)的可去奇点 \(\iff\) \(f(z)\)的罗朗展式中不含\(z\)的正次幂 \(\iff\) \(\lim\limits_{z\to\infty} f(z) = A\)是有限复数.
\item \(z=\infty\)是\(f(z)\)的\(m\)阶极点 \(\iff\) \(f(z)\)的罗朗展式中只有有限个正次幂且其最高次幂是\(m\) \(\iff\) \(\lim\limits_{z\to\infty} f(z) = \infty\).
\item \(z=\infty\)是\(f(z)\)的本性奇点 \(\iff\) \(f(z)\)的罗朗展式中有无限多个正次幂 \(\iff\) \(\lim\limits_{z\to\infty} f(z)\)不存在.
\end{enumerate}

\begin{example}
多项式函数\[
P(z) = a_0 z^n + a_1 z^{n-1} + \dotsb + a_n
\quad(a_0\neq0)
\]以无穷远点\(\infty\)为\(n\)阶极点.
\end{example}

\begin{example}
有理分式函数\[
f(z) = \frac{a_0 z^n + a_1 z^{n-1} + \dotsb + a_n}{b_0 z^m + b_1 z^{m-1} + \dotsb + b_m}
\quad(a_0\neq0,b_0\neq0)
\]当\(m \geq n\)时,\(\lim\limits_{z\to\infty} f(z) = 0\)或\(\lim\limits_{z\to\infty} f(z) = \frac{a_0}{b_0}\),因此\(\infty\)是\(f(z)\)的解析点;
当\(m<n\)时,利用多项式的降幂除法,可求得\(f(z)\)在点\(\infty\)的罗朗展式为\[
f(z) = \frac{a_0}{b_0} z^{n-m} + \dotsb
\quad(\frac{a_0}{b_0}\neq0),
\]因此\(\infty\)是\(f(z)\)的\(n-m\)阶极点.
\end{example}

\begin{example}
无穷远点\(\infty\)是指数函数\(f(z) = e^z\)的本性奇点.
\end{example}
