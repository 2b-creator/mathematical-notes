\section{幂级数}
\subsection{幂级数的概念}
\begin{definition}
各项均是幂函数的复函数项级数,称为\DefineConcept{幂级数}.
其形式是\[
	\sum_{n=0}^\infty c_n z^n
	= c_0 + c_1 z + c_2 z^2 + \dotsb,
\]
其中复常数\(c_0,c_1,c_2,\dotsc\)称为“幂级数的\DefineConcept{系数}”.
\end{definition}

\subsection{阿贝尔定理}
\begin{theorem}[阿贝尔定理]\label{theorem:解析函数的级数表示.阿贝尔定理}
若幂级数\(\sum_{n=0}^\infty c_n z^n\)在\(z_0 \neq 0\)处收敛,
则它在圆\(\abs{z} < \abs{z_0}\)内绝对收敛且内闭一致收敛.
\begin{proof}
因\(\sum_{n=0}^\infty c_n z_0^n\)收敛,
故\(\lim_{n\to\infty} c_n z_0^n = 0\),
从而\[
	(\exists M > 0)
	(\forall n \in \mathbb{N})
	[\abs{c_n z_0^n} \leq M],
\]
进而有\[
	\abs{c_n z^n}
	= \abs{c_n z_0^n \cdot \frac{z^n}{z_0^n}}
	\leq M \abs{\frac{z}{z_0}}^n.
\]
当\(\abs{z} < \abs{z_0}\)时,
有\(\abs{\frac{z}{z_0}} < 1\).
把\(z\)视为固定的复数时,
由\cref{example:无穷级数.等比级数的收敛性} 可知
实等比级数\(\sum_{n=0}^\infty \abs{\frac{z}{z_0}}^n\)收敛,
所以根据\hyperref[theorem:无穷级数.正项级数的比较审敛法]{正项级数的比较审敛法}可知,
实幂级数\(\sum_{n=0}^\infty \abs{c_n z^n}\)
在圆\(\abs{z} < \abs{z_0}\)内收敛,
也就是说复幂级数\(\sum_{n=0}^\infty c_n z^n\)
在圆\(\abs{z} < \abs{z_0}\)内绝对收敛.

在闭圆\(\abs{z} \leq k \abs{z_0}\ (0<k<1)\)内,
有\(\abs{c_n z^n} \leq M k^n\).
因为实等比级数\(\sum_{n=0}^\infty k^n\)收敛,
所以根据\hyperref[theorem:无穷级数.优级数准则]{优级数准则}可知,
复幂级数\(\sum_{n=0}^\infty c_n z^n\)
在闭圆\(\abs{z} \leq k \abs{z_0}\)一致收敛,
故复幂级数\(\sum_{n=0}^\infty c_n z^n\)
在圆\(\abs{z} < \abs{z_0}\)内闭一致收敛.
\end{proof}
\end{theorem}
阿贝尔定理表明,
若幂级数\(\sum_{n=0}^\infty c_n z^n\)
在点\(z = z_0 \neq 0\)处收敛,
则它在圆\(\abs{z} < \abs{z_0}\)内绝对收敛;
特别地,当\(0 \leq r < \abs{z_0}\)时
正项级数\(\sum_{n=0}^\infty \abs{c_n} r^n\)收敛.
若幂级数\(\sum_{n=0}^\infty c_n z^n\)
在点\(z = z_0\)处发散,
则它在圆\(\abs{z} > \abs{z_0}\)外发散;
特别地,当\(r > \abs{z_0}\)时
正项级数\(\sum_{n=0}^\infty \abs{c_n} r^n\)发散.

我们可以看出,
对于级数\(\sum_{n=0}^\infty c_n z^n\),
可能存在一个圆\[
	K: \abs{z} < R,
\]使得该级数在\(K\)内绝对收敛,
而在\(K\)外发散.
这个圆\(K\)称为该幂级数的\DefineConcept{收敛圆},
数\(R\)称为\DefineConcept{收敛半径}.

可以证明,复级数\(\sum_{n=0}^\infty c_n z^n\)的收敛半径\(R\)
恰好就是与之相应的实系数的幂级数\(\sum_{n=0}^\infty \abs{c_n} r^n\)的收敛半径.

\begin{theorem}\label{theorem:解析函数的级数表示.复幂级数的收敛半径的求法}
若幂级数\(\sum_{n=0}^\infty c_n z^n\)的系数\(c_n\)满足\[
	\lim_{n\to\infty} \abs{\frac{c_{n+1}}{c_n}} = \rho
\]或\[
	\lim_{n\to\infty} \sqrt[n]{\abs{c_n}} = \rho
\]或\[
	\varlimsup_{n\to\infty} \sqrt[n]{\abs{c_n}} = \rho,
\]
则这幂级数的收敛半径为\[
	\def\arraystretch{1.5}
	R = \left\{ \begin{array}{ll}
		\frac{1}{\rho}, & \rho \in (0,+\infty), \\
		+\infty, & \rho = 0, \\
		0, & \rho = +\infty \\
	\end{array} \right.
\]
\end{theorem}
对于幂级数\(\sum_{n=0}^\infty c_n (z-a)^n\),
上述求收敛半径的定理仍然成立,
其收敛圆为\[
	\abs{z-a}<R.
\]

若\(0<R<+\infty\),
那么当\(\abs{z-a} \leq r < R\)时,
有\[
	\abs{c_n (z-a)^n} \leq \abs{c_n} r^n,
\]
级数\(\sum_{n=0}^\infty \abs{c_n} r^n\)收敛,
所以\(\sum_{n=0}^\infty c_n (z-a)^n\)
在\(\abs{z-a} \leq r\)上一致收敛,
而\(r\)可以任意接近于\(R\).
因此由\hyperref[theorem:解析函数的级数表示.魏尔斯特拉斯定理]{魏尔斯特拉斯定理},
幂级数\(\sum_{n=0}^\infty c_n (z-a)^n\)的和函数\(F(z)\)
在圆\(\abs{z-a}<R\)内解析,
而且\[
	f^{(k)}(z) = k! c_k + (k+1)k\dotsm2c_{k+1}(z-a)+\dotsb.
\]
特别地,令\(z=a\),得\[
	c_k = \frac{1}{k!} f^{(k)}(a)
	\quad(k=0,1,2,\dotsc).
\]

若\(R=+\infty\),则幂级数\(\sum_{n=0}^\infty c_n (z-a)^n\)在整个复平面\(C\)上绝对收敛,在任意闭圆\(\abs{z} \leq r\)上一致收敛,和函数\(F(z)\)在\(C\)上解析,等等.综上所述,有以下结果:
\begin{theorem}\label{theorem:解析函数的级数表示.幂级数的和函数的性质}
设幂级数\(\sum_{n=0}^\infty c_n (z-a)^n\)的和函数是\(F(z)\),
收敛圆是\(K: \abs{z-a}<R\ (0<R<+\infty)\).
\begin{enumerate}
	\item 和函数\(F(z)\)在其收敛圆\(K\)内解析;

	\item 在收敛圆\(K\)内,
	幂级数\[
		F(z) = \sum_{n=0}^\infty c_n (z-a)^n
	\]
	可以逐项求任意阶导数,
	得\begin{equation}
		f^{(p)}(z) = \sum_{n=p}^\infty n(n-1)\dotsm(n-p+1) c_n (z-a)^{n-p}
		\quad(p=0,1,2,\dotsc),
	\end{equation}
	且其收敛半径不变;

	\item 幂级数的系数\(c_n\)可以用和函数\(F(z)\)
	在收敛圆圆心\(z=a\)处的相应阶导数表出,
	即\begin{equation}
		c_n = \frac{1}{n!} f^{(n)}(a)
		\quad(n=0,1,2,\dotsc);
	\end{equation}

	\item 沿收敛圆\(K\)内的任一简单曲线\(\gamma \subseteq K\),
	可逐项积分,
	得\begin{equation}
		\int_\gamma F(z) \dd{z}
		= \sum_{n=0}^\infty c_n \int_\gamma (z-a)^n \dd{z},
	\end{equation}
	且收敛半径不变.
\end{enumerate}
\end{theorem}

\begin{example}
求幂级数\(\sum_{n=0}^\infty \frac{z^n}{n}\)的收敛半径、收敛圆以及和函数.
\begin{solution}
收敛半径为\[
	R = \lim_{n\to\infty} \abs{\frac{c_n}{c_{n+1}}}
	= \lim_{n\to\infty} \frac{n}{n+1}
	= 1.
\]
收敛圆为\(\abs{z} < 1\).

设在\(\abs{z} < 1\)内,
幂级数\(\sum_{n=0}^\infty \frac{z^n}{n}\)的和函数\(F(z)\).
由\cref{theorem:解析函数的级数表示.幂级数的和函数的性质} 可得\[
	F'(z) = \sum_{n=1}^\infty z^{n-1}
	= \sum_{n=0}^\infty z^n
	= \frac{1}{1-z}
	\quad(\abs{z} < 1).
\]
令\(1-z = \zeta\),
有\[
	F'(1-\zeta) = \frac{1}{\zeta}
	\quad(\abs{\zeta-1}<1).
\]
在\(\abs{\zeta-1}<1\)内沿着\(1\)到\(\zeta\)的路径积分,
可得\[
	\int_1^{\zeta} F'(1-\zeta) \dd{\zeta}
	= \int_1^{\zeta} \frac{1}{\zeta} \dd{\zeta},
\]\[
	F(0) - F(1-\zeta)
	= \ln \zeta,
\]
其中\(-\pi < \arg \zeta < \pi\).
由于\(F(0) = 0\),
所以和函数\[
	F(z) = -\ln(1-z)
	\quad(\abs{z}<1, -\pi < \arg(1-z) < \pi).
\]
\end{solution}
\end{example}
注意,前面的讨论没有涉及到幂级数\(\sum_{n=0}^\infty c_n (z-a)^n\)
在收敛圆周\(\abs{z-a}=R\ (0<R<+\infty)\)上的收敛性.
在\(\abs{z-a}=R\)上,
幂级数\(\sum_{n=0}^\infty c_n (z-a)^n\)
既可以是点点收敛,也可以是点点发散,
还可以在一部分点上收敛而在其余点上发散.
例如,级数\(\sum_{n=1}^\infty \frac{z^n}{n^2}\)的收敛半径\(R=1\);
在\(\abs{z}=1\)上,
\(\sum_{n=1}^\infty \abs{\frac{z^n}{n^2}}
= \sum_{n=1}^\infty \frac{1}{n^2}\)收敛,
因此\(\sum_{n=1}^\infty \frac{z^n}{n^2}\)
在\(\abs{z}=1\)上处处绝对收敛.
又例如级数\(\sum_{n=0}^\infty z^n\)
在\(\abs{z}=1\)上点点发散,
因为这时一般项\(z^n\)的模\(\abs{z^n}\)恒为\(1\)而不趋于零.
再例如级数\(\sum_{n=1}^\infty \frac{z^n}{n}\)的收敛半径\(R=1\);
它在圆周\(\abs{z}=1\)上只在点\(z=1\)处发散,
而在其他各点\(z = e^{\iu\theta}\ (0<\theta<2\pi)\)上,
有\[
	\sum_{n=1}^\infty \frac{z^n}{n}
	= \sum_{n=1}^\infty \frac{\cos n\theta}{n}
	+ \iu\sum_{n=1}^\infty \frac{\sin n\theta}{n},
\]
由于它的实部、虚部级数都收敛,
因此它在圆周\(\abs{z}=1\)上除去点\(z=1\)外处处收敛.

但我们要特别之处的是:
纵使幂级数在其收敛圆周上处处收敛,
其和函数在收敛圆周上仍然至少有一个奇点.
例如级数\(\sum_{n=1}^\infty \frac{z^n}{n^2}\)
虽然在\(\abs{z}=1\)上处处绝对收敛,
从而在闭圆\(\abs{z}\leq1\)上一致收敛,
且其一般项\(\frac{z^n}{n^2}\)在复平面上都是解析的,
因此根据\hyperref[theorem:解析函数的级数表示.魏尔斯特拉斯定理]{魏尔斯特拉斯定理},
级数\(\sum_{n=1}^\infty \frac{z^n}{n^2}\)
在\(\abs{z}<1\)内的和函数\(F(z)\)的导数为\[
	F'(z) = 1 + \frac{z}{2} + \frac{z^2}{3} + \dotsb + \frac{z^{n-1}}{n} + \dotsb.
\]
当\(z\)从单位元内沿实轴趋于\(1\)时,
\(F'(z)\)趋于\(+\infty\).
而我们知道,解析函数在其解析点处是无穷次可微的,
所以\(z=1\)是和函数\(F(z)\)的一个奇点.
