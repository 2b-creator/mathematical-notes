\section{数值分析的对象、作用与特点}
数值分析研究用计算机求解各种数学问题的数值计算方法及其理论与软件实现,
用计算机求解科学技术问题通常经历以下步骤:
\begin{enumerate}
	\item 根据实际问题建立数学模型.
	\item 由数学模型给出数值计算方法.
	\item 根据计算方法编制算法程序,在计算机上算出结果.
\end{enumerate}

能用计算机计算的数值问题,
是指输入数据(即问题中的自变量与原始数据)
与输出数据(结果)之间函数关系的一个确定而无歧义的描述,
输入输出数据可用有限维向量表示.

有的问题,例如求解线性方程组,属于数值问题.
而有的问题,例如给定常微分方程,求未知函数的解析表达式,就不属于数值问题.

数值计算的基本单位称为算法元,
它由算子、输入元和输出元组成.
算子可以是简单操作,
例如算术运算、逻辑运算;
也可以是宏操作,
例如向量运算、数组传输、基本初等函数求值等.
输入元和输出元通常视作向量.
有限个算法元的序列称为一个进程.
一个数值问题的算法是指按规定顺序执行一个或多个完整的进程,
通过它们将输入元变换称输出元.
按同时运行的进程个数,
可以将算法分为串行算法和并行算法两类;
其中,同一时间只有一个进程在运行的算法,称为串行算法;
反之,同一时间有若干个进程在运行的算法,称为并行算法.

一个给定的数值问题,可以有许多不同的算法,
它们都能给出近似答案,
但所需的计算量和得到的精确程度可能相差很大.
一个面向计算机、有可靠理论分析、计算简单的算法,就是一个好算法.
