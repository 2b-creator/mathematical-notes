\chapter{集族与测度}
\section{集族}
在测度论中,我们常把一个非空集合记为\(\Omega\),
研究它的子集、子集族的性质.
我们知道,只要任取\(\mathcal{A} \in \Powerset\Powerset\Omega\),
那么\(\mathcal{A}\)就成为\(\Omega\)的一个子集族.
我们把从自然数集\(\mathbb{N}\)到\(\mathcal{A}\)的映射
称为一个\DefineConcept{集合序列},
简称\DefineConcept{集列}.

\subsection{单调集列}
\begin{definition}
%@see: 《测度论讲义(第三版)》(严加安) P2. 1.1.4
%@see: 《实变函数论》(周民强) P9. 定义1.8
设\(\{A_n\}_{n\geq1}\)是一个集合序列.

若\[
	(\forall n\in\omega)
	[A_n \subseteq A_{n+1}],
\]
则称“\(\{A_n\}\)是\DefineConcept{单调增集列}”.

若\[
	(\forall n\in\omega)
	[A_n \supseteq A_{n+1}],
\]
则称“\(\{A_n\}\)是\DefineConcept{单调减集列}”.

我们将“单调增集列”与“单调减集列”统称为\DefineConcept{单调集列}.
\end{definition}

当我们把集合序列\(\{A_n\}_{n\geq1}\)看成映射时,
利用我们在\cref{section:集合论.指标集}所学的知识,
我们可以给出以下概念.
\begin{definition}
设\(\{A_n\}_{n\geq1}\)是一个集合序列.
定义:\begin{gather*}
	\bigcup_{k=n}^m A_k
	\defeq
	\bigcup_{n \leq k \leq m, k\in\mathbb{N}} A_k, \\
	\bigcup_{k=n}^\infty A_k
	\defeq
	\bigcup_{k \geq n, k\in\mathbb{N}} A_k, \\
	\bigcap_{k=n}^m A_k
	\defeq
	\bigcap_{n \leq k \leq m, k\in\mathbb{N}} A_k, \\
	\bigcap_{k=n}^\infty A_k
	\defeq
	\bigcap_{k \geq n, k\in\mathbb{N}} A_k.
\end{gather*}
\end{definition}

\begin{definition}
%@see: 《实变函数论》(周民强) P9. 定义1.8
设\(\{A_n\}_{n\geq1}\)是一个集列.
定义:\[
	\lim_{n\to\infty} A_n
	\defeq
	\left\{ \def\arraystretch{2} \begin{array}{cl}
		\bigcup_{n=1}^\infty A_n, & \text{$\{A_n\}$是单调增集列}, \\
		\bigcap_{n=1}^\infty A_n, & \text{$\{A_n\}$是单调减集列},
	\end{array} \right.
\]
称其为“\(\{A_n\}\)的\DefineConcept{极限}”.
\end{definition}

\begin{example}
%@see: 《实变函数论》(周民强) P9. 例5
设\(A_n = [n,+\infty)\ (n=1,2,\dotsc)\).
显然\(\{A_n\}\)是一个单调减集列,
容易得出\(\lim_{n\to\infty} A_n = \emptyset\).
\end{example}

\begin{example}
%@see: 《实变函数论》(周民强) P9. 例6
设函数族\(\{f_n\}_{n\geq1}\)满足
\begin{enumerate}
	\item 对\(\forall x\in\mathbb{R}\),
	总有\(f_n(x) \leq f_{n+1}(x)\ (n=1,2,\dotsc)\),
	\item 对\(\forall x\in\mathbb{R}\),
	总有\(\lim_{n\to\infty} f_n(x) = f(x)\).
\end{enumerate}
给定\(t\in\mathbb{R}\),
取集列\(\{E_n\}_{n\geq1}\),使之满足\[
	E_n = \Set{ x\in\mathbb{R} \given f_n(x) > t }
	\quad(n=1,2,\dotsc).
\]
那么就有\[
	E_1 \subseteq E_2 \subseteq \dotsb \subseteq E_n \subseteq \dotsb,
\]
并且还有\[
	\lim_{n\to\infty} E_n
	= \bigcup_{n=1}^\infty \Set{ x\in\mathbb{R} \given f_n(x) > t }
	= \Set{ x\in\mathbb{R} \given f(x) > t }.
\]
\end{example}

\subsection{集列的极限}
\begin{definition}
%@see: 《测度论讲义(第三版)》(严加安) P2. 1.1.4
设\(\{A_n\}_{n\geq1}\)是一个集合序列.

定义:\[
	\limsup_{n\to\infty} A_n
	\defeq
	\bigcap_{n=1}^\infty
	\bigcup_{k=n}^\infty
	A_k,
\]
称其为“\(\{A_n\}\)的\DefineConcept{上极限}”.

定义:\[
	\liminf_{n\to\infty} A_n
	\defeq
	\bigcup_{n=1}^\infty
	\bigcap_{k=n}^\infty
	A_k,
\]
称其为“\(\{A_n\}\)的\DefineConcept{下极限}”.
\end{definition}
容易看出,\(\limsup_{n\to\infty} A_n\)的任一元素属于无限多个\(A_n\),
而\(\liminf_{n\to\infty} A_n\)的任一元素至多不属于有限多个\(A_n\).

\begin{example}
%@see: 《实变函数论》(周民强) P10. 例7
设\(E,F\)都是集合,
集列\(\{A_k\}_{k\geq1}\)满足\[
	A_k = \left\{ \begin{array}{cl}
		E, & \text{$k$是奇数}, \\
		F, & \text{$k$是偶数}
	\end{array} \right.
	\quad(k=1,2,\dotsc),
\]
从而我们有\[
	\limsup_{k\to\infty} A_k
	= E \cup F, \qquad
	\liminf_{k\to\infty} A_k
	= E \cap F.
\]
\end{example}

\begin{proposition}
%@see: 《实变函数论》(周民强) P10.
设\(E\)是集合,\(\{A_n\}_{n\geq1}\)是集列,
则\begin{gather}
	E - \limsup_{n\to\infty} A_n = \liminf_{n\to\infty} (E - A_n); \\
	E - \liminf_{n\to\infty} A_n = \limsup_{n\to\infty} (E - A_n).
\end{gather}
\end{proposition}

\begin{proposition}
%@see: 《测度论讲义(第三版)》(严加安) P2. 1.1.4
设\(\{A_n\}_{n\geq1}\)是一个集合序列,
则\begin{equation}
	\liminf_{n\to\infty} A_n
	\subseteq
	\limsup_{n\to\infty} A_n.
\end{equation}
%@see: https://math.stackexchange.com/questions/107931/lim-sup-and-lim-inf-of-sequence-of-sets
\end{proposition}

\begin{theorem}
%@see: 《测度论讲义(第三版)》(严加安) P2. 1.1.4
设\(\{A_n\}_{n\geq1}\)是一个集列.
如果\[
	\liminf_{n\to\infty} A_n
	= \limsup_{n\to\infty} A_n
	= A,
\]
则\(\lim_{n\to\infty} A_n = A\).
\end{theorem}

\subsection{集族的封闭性}
%@see: 《测度论讲义(第三版)》(严加安) P2. 1.1.6
\begin{definition}[集族的封闭性]
设\(\mathcal{C}\)是一个非空集族.

如果\[
	(\forall A,B\in\mathcal{C})
	[A \cap B \in \mathcal{C}],
\]
则称“\(\mathcal{C}\)对有限交封闭”.

如果\[
	(\forall \{A_n\})
	(\forall n\geq1)
	\left[A_n\in\mathcal{C} \implies \bigcap_{k=1}^n A_k \in \mathcal{C}\right],
\]
则称“\(\mathcal{C}\)对可列交封闭”.

如果\[
	(\forall A,B\in\mathcal{C})
	[A \cup B \in \mathcal{C}],
\]
则称“\(\mathcal{C}\)对有限并封闭”.

如果\[
	(\forall \{A_n\})
	(\forall n\geq1)
	\left[A_n\in\mathcal{C} \implies \bigcup_{k=1}^n A_k \in \mathcal{C}\right],
\]
则称“\(\mathcal{C}\)对可列并封闭”.
\end{definition}

\begin{definition}
设\(\mathcal{C}\)是一个非空集族.
定义:\[
	\mathcal{C}_{\cap f}
	\defeq
	\Set*{
		A \given
		A = \bigcap_{k=1}^n A_k,
		A_k \in \mathcal{C}, i=1,\dotsc,n,
		n\geq1
	},
\]
称其为“用有限交运算封闭\(\mathcal{C}\)所得的集族”.
\end{definition}

\begin{proposition}
设\(\mathcal{C}\)是一个非空集族,
则\(\mathcal{C}_{\cap f}\)对有限交封闭.
\end{proposition}
