\chapter{集族与测度}
\section{集族}
在测度论中,我们常把一个非空集合记为\(\Omega\),
研究它的子集、子集族的性质.
我们知道,只要任取\(\mathcal{A} \in \Powerset\Powerset\Omega\),
那么\(\mathcal{A}\)就成为\(\Omega\)的一个子集族.
我们把从自然数集\(\mathbb{N}\)到\(\mathcal{A}\)的映射
称为一个\DefineConcept{集合序列},
简称\DefineConcept{集列}.

\subsection{单调集列}
%@see: 《测度论讲义(第三版)》(严加安) P2. 1.1.4
\begin{definition}
设\(\{A_n\}_{n\geq1}\)是一个集合序列.

若\[
	(\forall n\in\omega)
	[A_n \subseteq A_{n+1}],
\]
则称“\(\{A_n\}\)是\DefineConcept{单调增集列}”.

若\[
	(\forall n\in\omega)
	[A_n \supseteq A_{n+1}],
\]
则称“\(\{A_n\}\)是\DefineConcept{单调减集列}”.

我们将“单调增集列”与“单调减集列”统称为\DefineConcept{单调集列}.
\end{definition}

当我们把集合序列\(\{A_n\}_{n\geq1}\)看成映射时,
我们可以取得它的限制\[
	\{A_n\}_{s \leq n\leq t}
	\defeq
	\{A_n\}\upharpoonright([s,t]\cap\mathbb{N}),
\]
其中\(1\leq s<t\);
也可以对它的值域\(\ran\{A_n\}_{n\geq1}\)
或它的限制的值域\(\ran\{A_n\}_{s\leq n\leq t}\)求并集.
于是我们给出以下概念.
\begin{definition}
设\(\{A_n\}_{n\geq1}\)是一个集合序列.
定义:\begin{gather*}
	\bigcup_{k=n}^m A_k
	\defeq
	\bigcup \ran\{A_k\}_{n \leq k \leq m}, \\
	\bigcup_{n=1}^\infty A_n
	\defeq
	\bigcup \ran\{A_n\}_{n\geq1}, \\
	\bigcap_{k=n}^m A_k
	\defeq
	\bigcap \ran\{A_k\}_{n \leq k \leq m}, \\
	\bigcap_{n=1}^\infty A_n
	\defeq
	\bigcap \ran\{A_n\}_{n\geq1}.
\end{gather*}
\end{definition}

\subsection{集列的极限}
\begin{definition}
设\(\{A_n\}_{n\geq1}\)是一个集合序列.

定义:\begin{align*}
	\lim_{n\to\infty} A_n
	&\defeq \bigcup_{n=1}^\infty A_n \\
	&\defeq \bigcap_{n=1}^\infty A_n,
\end{align*}
称其为“\(\{A_n\}\)的\DefineConcept{极限}”.
\end{definition}

\begin{definition}
设\(\{A_n\}_{n\geq1}\)是一个集合序列.

定义:\[
	\limsup_{n\to\infty} A_n
	\defeq
	\bigcap_{n=1}^\infty
	\bigcup_{k=n}^\infty
	A_k,
\]
称其为“\(\{A_n\}\)的\DefineConcept{上极限}”.

定义:\[
	\liminf_{n\to\infty} A_n
	\defeq
	\bigcup_{n=1}^\infty
	\bigcap_{k=n}^\infty
	A_k,
\]
称其为“\(\{A_n\}\)的\DefineConcept{下极限}”.
\end{definition}

\begin{proposition}
设\(\{A_n\}_{n\geq1}\)是一个集合序列,
则\begin{equation}
	\liminf_{n\to\infty} A_n
	\subseteq
	\limsup_{n\to\infty} A_n.
\end{equation}
%@see: https://math.stackexchange.com/questions/107931/lim-sup-and-lim-inf-of-sequence-of-sets
\end{proposition}

\begin{proposition}
设\(\{A_n\}_{n\geq1}\)是一个集合序列.
若\[
	\liminf_{n\to\infty} A_n
	= \limsup_{n\to\infty} A_n,
\]
则\(\{A_n\}\)的极限存在,
且\[
	\lim_{n\to\infty} A_n
	= \liminf_{n\to\infty} A_n
	= \limsup_{n\to\infty} A_n.
\]
\end{proposition}

\subsection{集族的封闭性}
%@see: 《测度论讲义(第三版)》(严加安) P2. 1.1.6
\begin{definition}[集族的封闭性]
设\(\mathcal{C}\)是一个非空集族.

如果\[
	(\forall A,B\in\mathcal{C})
	[A \cap B \in \mathcal{C}],
\]
则称“\(\mathcal{C}\)对有限交封闭”.

如果\[
	(\forall \{A_n\})
	(\forall n\geq1)
	\left[A_n\in\mathcal{C} \implies \bigcap_{k=1}^n A_k \in \mathcal{C}\right],
\]
则称“\(\mathcal{C}\)对可列交封闭”.

如果\[
	(\forall A,B\in\mathcal{C})
	[A \cup B \in \mathcal{C}],
\]
则称“\(\mathcal{C}\)对有限并封闭”.

如果\[
	(\forall \{A_n\})
	(\forall n\geq1)
	\left[A_n\in\mathcal{C} \implies \bigcup_{k=1}^n A_k \in \mathcal{C}\right],
\]
则称“\(\mathcal{C}\)对可列并封闭”.
\end{definition}

\begin{definition}
设\(\mathcal{C}\)是一个非空集族.
定义:\[
	\mathcal{C}_{\cap f}
	\defeq
	\Set*{
		A \given
		A = \bigcap_{k=1}^n A_k,
		A_k \in \mathcal{C}, i=1,\dotsc,n,
		n\geq1
	},
\]
称其为“用有限交运算封闭\(\mathcal{C}\)所得的集族”.
\end{definition}

\begin{proposition}
设\(\mathcal{C}\)是一个非空集族,
则\(\mathcal{C}_{\cap f}\)对有限交封闭.
\end{proposition}
