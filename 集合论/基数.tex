\section{基数}
\subsection{等势}
我们时不时想要研究讨论下面的问题:
两个集合是否具有相同的大小?
或者说,这两个集合的元素的个数是否相同?
还是说,一个集合相比于另一个拥有更多的元素?

\begin{definition}
%@see: 《Elements of Set Theory》 P129. Definition
设\(A,B\)都是集合.
如果存在一个从\(A\)到\(B\)上的一一映射,
那么称“\(A\)与\(B\) \DefineConcept{等势}(\(A\) is \emph{equinumerous} to \(B\))”,
记作\(A \approx B\);
并把这个映射称为“\(A\)和\(B\)之间的\DefineConcept{一一对应}%
(\emph{one-to-one correspondence} between \(A\) and \(B\))”.
\end{definition}

\begin{example}
全体完全平方数组成的集合与自然数集\(\omega\)等势.
\end{example}

\begin{example}
全体偶数组成的集合与整数集\(\mathbb{Z}\)等势.
\end{example}

\begin{example}
\(\omega\times\omega\)与\(\omega\)等势,
这是因为存在从\(\omega\times\omega\)到\(\omega\)上的映射\(J\),
\[
	J(m,n) = \frac{1}{2} [(m+n)^2+3m+n].
\]
\end{example}

\begin{example}
\(\omega\)与\(\mathbb{Q}\)等势.
\end{example}

\begin{example}
开区间\((0,1)\)与\(\mathbb{R}\)等势.
\end{example}

\begin{example}
证明:\(A\)的幂集与从\(A\)到自然数\(2\)的映射空间等势,即\begin{equation}
	\Powerset A \approx 2^A.
\end{equation}
\begin{proof}
构造映射\(H\colon \Powerset A \to 2^A\),
使得对于\(A\)的任意子集\(B\),
\(H(B)\)是\(B\)的\DefineConcept{特征映射}(characteristic function),即\[
	H(B) = f_B(x) = \left\{ \begin{array}{cl}
		1, & x \in B, \\
		0, & x \in A - B.
	\end{array} \right.
\]
那么,\(2^A\)中的任一映射\(g\)都在\(\ran H\)中,这是因为\[
	g = H(\Set{ x \in A \given g(x) = 1 }).
\]
\end{proof}
\end{example}

\begin{theorem}\label{theorem:集合论.等势相当于等价关系}
%@see: 《Elements of Set Theory》 P132. Theorem 6A
对于任意集合\(A,B,C\):\begin{enumerate}
	\item \(A \approx A\).
	\item \(A \approx B \implies B \approx A\).
	\item \(A \approx B \land B \approx C \implies A \approx C\).
\end{enumerate}
\end{theorem}
\cref{theorem:集合论.等势相当于等价关系}
说明等势具有自反性、对称性、传递性,
但它不是一个等价关系,
这是因为它关注的是全体集合.

\begin{theorem}
%@see: 《Elements of Set Theory》 P132. Theorem 6B(a)
自然数集\(\omega\)与实数集\(\mathbb{R}\)不等势.
\end{theorem}

\begin{theorem}
%@see: 《Elements of Set Theory》 P132. Theorem 6B(b)
没有集合与它的幂集等势.
\end{theorem}

\subsection{有限集}
\begin{definition}
%@see: 《Elements of Set Theory》 P133. Definition
设\(A\)是集合.
当且仅当存在自然数\(n\)与之等势时,称“\(A\)是\DefineConcept{有限的}(finite)”;
否则称“\(A\)是\DefineConcept{无限的}(infinite)”,
即\begin{gather}
	\text{\(A\)是有限的}
	\defiff
	(\exists n\in\omega)[A \approx n]; \\
	\text{\(A\)是无限的}
	\defiff
	(\forall n\in\omega)\neg[A \approx n].
\end{gather}
\end{definition}

\begin{example}
任一自然数是有限集.
\end{example}

\begin{theorem}[鸽巢原理]
%@see: 《Elements of Set Theory》 P134. Pigeonhold Principle
没有自然数与它的真子集等势.
\end{theorem}
