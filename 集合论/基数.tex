\section{基数}
\subsection{等势}
我们时不时想要研究讨论下面的问题:
两个集合是否具有相同的大小?
或者说,这两个集合的元素的个数是否相同?
还是说,一个集合相比于另一个拥有更多的元素?

\begin{definition}
%@see: 《Elements of Set Theory》 P129. Definition
设\(A,B\)都是集合.
如果存在一个从\(A\)到\(B\)上的一一映射,
那么称“\(A\)与\(B\) \DefineConcept{等势}(\(A\) is \emph{equinumerous} to \(B\))”,
记作\(A \approx B\);
并把这个映射称为“\(A\)和\(B\)之间的\DefineConcept{一一对应}%
(\emph{one-to-one correspondence} between \(A\) and \(B\))”.
\end{definition}

\begin{example}
全体完全平方数组成的集合与自然数集\(\omega\)等势.
\end{example}

\begin{example}
\(\omega\times\omega\)与\(\omega\)等势,
这是因为存在从\(\omega\times\omega\)到\(\omega\)上的映射\(J\),
\[
	J(m,n) = \frac{1}{2} [(m+n)^2+3m+n].
\]
\end{example}

\begin{example}
\(\omega\)与\(\mathbb{Q}\)等势.
\end{example}

\begin{example}
开区间\((0,1)\)与\(\mathbb{R}\)等势.
\end{example}
