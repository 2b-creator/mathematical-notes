
\section{实数}
\subsection{实数的构造}
\subsubsection{第一种途径:利用柯西列构造实数}
\begin{definition}
%@see: 《Elements of Set Theory》 P112.
设映射\(s\colon \omega\to\mathbb{Q}\)满足\[
	(\forall\varepsilon\in\mathbb{Q}^+)(\exists k\in\omega)(\forall m>k)(\forall n>k)
	\abs{s(m) - s(n)}<\varepsilon,
\]
则称“\(s\)是一个\DefineConcept{柯西列}(Cauchy sequence)”.
\end{definition}

\begin{definition}
%@see: 《Elements of Set Theory》 P112.
我们把全体柯西列的集合记为\(S\).
定义关系\(\sim\):\[
	r \sim s
	\iff
	(\forall\varepsilon\in\mathbb{Q}^+)(\exists k\in\omega)(\forall n>k)
	\abs{r(n)-s(n)}<\varepsilon,
\]
称“柯西列\(r\)和\(s\)等价”.
\end{definition}

可以证明关系\(\sim\)是柯西列集\(S\)上的等价关系,
以此为基础我们可以确定一个等价类划分,把它定义为实数.
\begin{definition}
%@see: 《Elements of Set Theory》 P112.
称“\(S\)对\(\sim\)的商集”为\DefineConcept{实数集}(the set of real numbers),
记作\(\mathbb{R}\),即\[
	\mathbb{R} \defeq S/\kern-2pt\sim.
\]
\end{definition}
这种构造实数集的方法归功于康托.

\subsubsection{第二种途径:利用戴德金分割构造实数}
\begin{definition}
%@see: 《Elements of Set Theory》 P113. Definition
设集合\(x \subseteq \mathbb{Q}\).
如果\begin{enumerate}
	\item \(\emptyset \neq x \neq \mathbb{Q}\);
	\item \(x\)向下封闭(\(x\) is closed downward),即\[
		q \in x \land r < q \implies r \in x;
	\]
	\item \(x\)没有最大元素,
\end{enumerate}
那么称“\(x\)是一个\DefineConcept{戴德金分割}(Dedekind cut)”.
\end{definition}

于是,我们称每个戴德金分割都是一个\DefineConcept{实数}(real number).
全体戴德金分割的集合便是实数集.

在戴德金分割途径下,实数的序的定义非常简洁.
对于任意两个实数\(x\)和\(y\),定义“小于”关系\(<\):\[
	x < y \iff x \subseteq y.
\]
换句话说,\[
	<\ \defeq \Set*{ \opair{x,y}\in\mathbb{R}\times\mathbb{R} \given x \subseteq y }.
\]

\begin{theorem}
%@see: 《Elements of Set Theory》 P113. Theorem 5RA
关系\(<\)是实数集\(\mathbb{R}\)上的线性序.
\end{theorem}

考虑集合\(A \subseteq \mathbb{R}\).
如果实数\(x\)满足\[
	(\forall y \in A)
	y \leqslant x,
\]
那么称“\(x\)是\(A\)的\DefineConcept{上界}(\(x\) is an \emph{upper bound} of \(A\))”.
需要注意到,\(x\)本身可能不是\(A\)的元素.
如果存在实数\(y\),恰有\(y\)是\(A\)的上界,
那么称“\(A\)有上界(the set \(A\) is bounded above)”.
如果\(A\)的上界\(z\)比其他任何上界都要小,
那么称“\(z\)是\(A\)的\DefineConcept{最小上界}(least upper bound)%
或\DefineConcept{上确界}(supremum)”,记为\(\sup A\).
%@see: https://mathworld.wolfram.com/Supremum.html

%\begin{definition}
%设\(X\)是实数的有界子集.
%
%如果\(\forall x \in X\)都有\(x \geqslant m\),%
%且对于\(\forall \varepsilon > 0\)都\(\exists \alpha \in X\)使\(\alpha < m + \varepsilon\)成立,%
%则数\(m\)称为集合\(X\)的\DefineConcept{下确界},记作\(\inf{x}\).
%
%如果\(\forall x \in X\)都有\(x \leqslant M\),%
%且对于\(\forall \varepsilon > 0\)都\(\exists \beta \in X\)使\(\beta > M - \varepsilon\)成立,%
%则数\(M\)称为集合\(X\)的\DefineConcept{上确界},记作\(\sup{x}\).
%
%如果集合\(X\)下方无界,则记\[
%\inf{x} = -\infty;
%\]如果集合\(X\)上方无界,则记\[
%\sup{x} = +\infty.
%\]
%\end{definition}

\begin{theorem}
%@see: 《Elements of Set Theory》 P114. Theorem 5RB
实数集的任意一个有上界的非空子集总有一个实最小上界.
\end{theorem}
这个定理在数学分析中非常重要.
它可以用来证明在闭区间上的连续函数一定能取得最值.

对于\(\forall x,y\in\mathbb{R}\),
我们定义实数集上的加法运算:
\begin{equation}
	x + y = \Set{
		q + r \given q \in x \land r \in y
	}.
\end{equation}

\begin{lemma}
%@see: 《Elements of Set Theory》 P114. Lemma 5RC
若\(x,y\in\mathbb{R}\),那么\(x+y\in\mathbb{R}\).
\end{lemma}

\begin{theorem}
%@see: 《Elements of Set Theory》 P115. Theorem 5RD
以下命题恒成立:
\begin{enumerate}
	\item 加法交换律
	\begin{equation}
		\forall x,y\in\mathbb{R} \bigl(
			x+y=y+x
		\bigr).
	\end{equation}
	\item 加法结合律
	\begin{equation}
		\forall x,y,z\in\mathbb{R} \bigl(
			(x+y)+z=x+(y+z)
		\bigr).
	\end{equation}
\end{enumerate}
\end{theorem}

\begin{theorem}
%@see: 《Elements of Set Theory》 P116. Theorem 5RE
以下命题恒成立:
\begin{enumerate}
	\item \(0\defeq\Set{ r\in\mathbb{Q} \given r<0 }\)是实数.
	\item 任意实数加上实数\(0\)不变
	\begin{equation}
		\forall x\in\mathbb{R} \bigl(
			x+0=x
		\bigr).
	\end{equation}
\end{enumerate}
\end{theorem}

\begin{definition}
%@see: 《Elements of Set Theory》 P117.
设\(x\in\mathbb{R}\).
那么称集合\[
	\Set{ r \in \mathbb{Q} \given (\exists s>r) -s \notin x }
\]为“\(x\)的负元”,记作\((-x)\).
\end{definition}

\begin{theorem}
%@see: 《Elements of Set Theory》 P117.
设\(x\in\mathbb{R}\),那么\begin{enumerate}
	\item \(-x\in\mathbb{R}\).
	\item \(x+(-x)=0\).
\end{enumerate}
\end{theorem}

%TODO
