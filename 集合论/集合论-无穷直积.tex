
\section{无穷直积}
我们在前面学习了有限个集合的直积,
但让我们更好奇的是:
存不存在无限个集合的直积呢?
取集合\(I\)作为指标集,
设映射\(H\)的定义域包含\(I\),
那么对于\(I\)中的每个\(i\),总可得集合\(H(i)\).
我们定义:\[
	\BigTimes_{i \in I} H(i)
	\defeq
	\Set{ f \given f \ \text{是一个以\(I\)为定义域的映射},
	\forall i \in I \bigl[ f(i) \in H(i) \bigr] }.
\]
易见\(\BigTimes_{i \in I} H(i)\)的元素都是“\(I\)元组(\(I\)-tuples)”(即以\(I\)为定义域的映射),
这些“元组”的“第\(i\)坐标”(即\(i\)在这些映射下的像)是\(H(i)\)中的元素.

注意到\(\BigTimes_{i \in I} H(i)\)的元素都是从\(I\)到\(\bigcup_{i \in I} H(i)\)的映射,
显然这些元素也都是映射空间\[
	\mathcal{H} = \vphantom{.}^I\left(\bigcup_{i \in I} H(i)\right)
\]的元素,
于是集合\(\BigTimes_{i \in I} H(i)\)可以通过对映射空间\(\mathcal{H}\)使用子集公理构造得到.

\begin{example}
如果存在集合\(A\),对\(\forall i \in I\),有\(H(i) = A\),那么\[
	\BigTimes_{i \in I} H(i) = \vphantom{.}^IA.
\]
\end{example}

应该注意到,
如果某个\(H(i)\)是空集,
那么无穷直积\(\BigTimes_{i \in I} H(i)\)也将是空集.
反过来说,假设\(\forall i \in I \bigl( H(i) \neq \emptyset \bigr)\),
我们能不能说\(\BigTimes_{i \in I} H(i) \neq \emptyset\)呢?
为了得到这个无穷直积的一个元素\(f\),
我们需要从每个\(H(i)\)中选择一些元素,
令\(f(i)\)等于这些选定的元素.
这就需要用到选择公理,
而且实际上这也是选择公理的若干等价表述方式之一.

\begin{axiom}[选择公理(第二种形式)]
对于任意集合\(I\)和任意以\(I\)为定义域的映射\(H\),
如果\(H(i) \neq \emptyset\ (i \in I)\),
那么\(\BigTimes_{i \in I} H(i) \neq \emptyset\).
\end{axiom}
