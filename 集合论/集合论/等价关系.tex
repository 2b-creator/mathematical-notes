\section{等价关系}
\subsection{关系的性质}
\begin{definition}
%@see: 《Elements of Set Theory》 P56.
设\(\rel{R}\)是集合\(A\)上的二元关系.
\begin{enumerate}
	\item 若\[
		(\forall x \in A)
		[x\rel{R}x],
	\]
	则称“关系\(\rel{R}\)具有\DefineConcept{自反性}(\(\rel{R}\) is \emph{reflexive})”;
	否则称“关系\(\rel{R}\)不具有自反性(\(\rel{R}\) is \emph{irreflexive})”
	或“关系\(\rel{R}\)具有\DefineConcept{反自反性}”.

	\item 若\[
		(\forall x,y \in A)
		[x\rel{R}y \implies y\rel{R}x],
	\]
	则称“关系\(\rel{R}\)具有\DefineConcept{对称性}(\(\rel{R}\) is \emph{symmetric})”.

	\item 若\[
		(\forall x,y \in A)
		[x\rel{R}y \land y\rel{R}x \implies x = y],
	\]
	则称“关系\(\rel{R}\)具有\DefineConcept{反对称性}(antisymmetric)”.

	\item 若\[
		(\forall x,y,z \in A)
		[x\rel{R}y \land y\rel{R}z \implies x\rel{R}z],
	\]
	则称“关系\(\rel{R}\)具有\DefineConcept{传递性}(\(\rel{R}\) is \emph{transitive})”.
\end{enumerate}
\end{definition}

\begin{proposition}
设集合\(A\)上的二元关系\(\rel{R}\)同时具有对称性和传递性,
则\(\rel{R}\)具有自反性.
\begin{proof}
假设\(x\rel{R}y\).
由于\(\rel{R}\)具有对称性,
所以\(y\rel{R}x\).
又因为\(\rel{R}\)具有传递性,
那么由\(x\rel{R}y\)和\(y\rel{R}x\)可以推得\(x\rel{R}x\),
这就说明\(\rel{R}\)具有自反性.
\end{proof}
\end{proposition}

\begin{example}
%@see: 《Elements of Set Theory》 P61. Exercise 32.
设\(\rel{R}\)是集合\(A\)上的二元关系.
证明:\begin{enumerate}
	\item \(\rel{R}\)具有对称性的充分必要条件是
	\(\rel{R}^{-1} \subseteq \rel{R}\).
	\item \(\rel{R}\)具有传递性的充分必要条件是
	\(\rel{R}\circ\rel{R} \subseteq \rel{R}\).
	%@see: https://math.stackexchange.com/q/1386714/591741
\end{enumerate}
\begin{proof}
易见
\begin{align*}
	\rel{R}^{-1} \subseteq \rel{R}
	&\iff
	(\forall x,y \in A)[
		x\rel{R}^{-1}y \implies x\rel{R}y
	] \\
	&\iff
	(\forall x,y \in A)[
		y\rel{R}x \implies x\rel{R}y
	] \\
	&\iff
	\text{\(\rel{R}\)具有对称性}. \\
	\rel{R}\circ\rel{R} \subseteq \rel{R}
	&\iff
	(\forall x,z \in A)
	[
		x(\rel{R}\circ\rel{R})z
		\implies
		x\rel{R}z
	] \\
	&\iff
	(\forall x,z \in A)
	[
		(\exists y \in A)[x\rel{R}y \land y\rel{R}z]
		\implies
		x\rel{R}z
	] \\
	&\iff%FIXME 这里我还不太理解怎样才能把(\exists y \in A)改写成(\forall y \in A)
	(\forall x,y,z \in A)
	[
		x\rel{R}y \land y\rel{R}z
		\implies
		x\rel{R}z
	] \\
	&\iff
	\text{\(\rel{R}\)具有传递性}.
	\qedhere
\end{align*}
\end{proof}
\end{example}
实际上,由\cref{theorem:集合论.与逆相等的充分必要条件} 可知,
\(\rel{R}^{-1} = \rel{R}\)也是\(\rel{R}\)具有对称性的充分必要条件.

\begin{example}
%@see: 《Elements of Set Theory》 P61. Exercise 33.
设\(\rel{R}\)是集合\(A\)上的二元关系.
证明:\(\rel{R}\)同时具有对称性和传递性的充分必要条件是
\(\rel{R} = \rel{R}^{-1}\circ\rel{R}\).
%TODO
\begin{proof}
先证充分性.
由\cref{theorem:集合论.复合的逆.推论1}
可知\((\rel{R}^{-1}\circ\rel{R})^{-1}=\rel{R}^{-1}\circ\rel{R}\),
即\(\rel{R}^{-1}\circ\rel{R}\)具有对称性.
于是\[
	\rel{R}=\rel{R}^{-1}\circ\rel{R}
	\implies
	\text{\(\rel{R}\)具有对称性}.
\]
又因为\[
	\text{\(\rel{R}\)具有对称性} \iff \rel{R}^{-1}=\rel{R},
\]
所以\[
	\rel{R}=\rel{R}^{-1}\circ\rel{R}
	=\rel{R}\circ\rel{R}
	\implies
	\text{\(\rel{R}\)具有传递性}.
\]

再证必要性.
容易看出\begin{align*}
	\text{\(\rel{R}\)同时具有对称性和传递性}
	&\iff
	\rel{R}^{-1}=\rel{R}
	\land
	\rel{R}\circ\rel{R}\subseteq\rel{R} \\
	&\implies
	\rel{R}^{-1}\circ\rel{R}\subseteq\rel{R}.
\end{align*}
任取\(\opair{x,y}\in\rel{R}\).
因为\(\rel{R}\)具有对称性,
所以\(y\rel{R}x\),
从而\(x\rel{R}^{-1}y\).
又因为\(\rel{R}\)具有传递性,
于是\(\rel{R}\)具有自反性,
即\(y\rel{R}y\).
于是由\(x\rel{R}^{-1}y\)和\(y\rel{R}y\)
可得\(x(\rel{R}^{-1}\circ\rel{R})y\).
这就是说,由\(x\rel{R}y\)可得\(x(\rel{R}^{-1}\circ\rel{R})y\),
根据外延公理可知\(\rel{R}\subseteq\rel{R}^{-1}\circ\rel{R}\).
因此\(\rel{R}^{-1}\circ\rel{R}=\rel{R}\).
\end{proof}
%@see: https://math.stackexchange.com/a/3978349/591741
\end{example}

\subsection{等价关系}
\begin{definition}
设\(\rel{R}\)是集合\(A\)上的二元关系,即\(\rel{R} \subseteq A^2\).
如果\(\rel{R}\)同时具有自反性、对称性、传递性,
则称“\(\rel{R}\)是\(A\)上的\DefineConcept{等价关系}(equivalence relation)”.
\end{definition}

\subsection{等价类划分}
\begin{theorem}\label{theorem:集合论.划分集合获得等价关系}
%@see: 《Elements of Set Theory》 P56. Theorem 3M
如果关系\(\rel{R}\)具有对称性和传递性,
那么\(\rel{R}\)是\(\fld \rel{R}\)上的等价关系.
\begin{proof}
任意关系\(\rel{R}\)(不论它是三元的还是四元的)都是它的域上的二元关系,
这是因为\[
	\rel{R}
	\subseteq \dom \rel{R} \times \ran \rel{R}
	\subseteq \fld \rel{R} \times \fld \rel{R}.
\]
已知\(\rel{R}\)具有对称性和传递性,
要证\(\rel{R}\)是\(\fld \rel{R}\)上的等价关系,
只需证\(\rel{R}\)在\(\fld \rel{R}\)上具有自反性.
由于\begin{align*}
	x \in \dom \rel{R}
	&\implies
	(\exists y)[x\rel{R}y] \\
	&\implies
	(\exists y)[x\rel{R}y \land y\rel{R}x]
		\tag{对称性} \\
	&\implies
	x \rel{R} x,
		\tag{传递性}
\end{align*}
可知\(\rel{R}\)在\(\dom \rel{R}\)上具有自反性;
同理,\(x \in \ran \rel{R} \implies x \rel{R} x\),
即\(\rel{R}\)在\(\ran \rel{R}\)上具有自反性;
所以,\(\rel{R}\)在\(\dom \rel{R} \cup \ran \rel{R} = \fld \rel{R}\)上具有自反性.
\end{proof}
\end{theorem}
一般而言,如果\(\rel{R}\)是一个在\(A\)上兼具对称性和传递性的关系,
它可能不是在\(A\)上的等价关系.
根据\cref{theorem:集合论.划分集合获得等价关系} 我们知道,
这样的\(\rel{R}\)在\(\fld \rel{R}\)上具有自反性,
但\(\fld \rel{R}\)可能只是\(A\)的一个小小的子集.

利用\cref{theorem:集合论.划分集合获得等价关系},
我们学会通过对集合\(A\)的划分诱导出一个等价关系.
接下来我们来研究怎么逆转这个过程,
也就是说,已知\(A\)上的等价关系,求\(A\)的划分.

\begin{definition}
%@see: 《Elements of Set Theory》 P57. Definition
已知\(\rel{R}\)是一个等价关系.
对于\(x \in \fld \rel{R}\),集合\[
	\Set{ y \given x \rel{R} y }
\]
称为“\(x\)(在关系\(\rel{R}\)下)的\DefineConcept{等价类}%
(the \emph{equivalence class} of \(x\) (\emph{modulo} \(\rel{R}\)))”,
记作\(\rel{R}[x]\)或\([x]_{\rel{R}}\).
把等价类\(\rel{R}[x]\)中的任意一个元素\(y\)称为%
“\(\rel{R}[x]\)的\DefineConcept{代表}(representative)”.

对不强调关系\(\rel{R}\)时,
也可将上述等价类记为\(\overline{x}\)或\([x]\).
\end{definition}

虽然这里把\(\rel{R}[x]\)叫做等价“类”,
实际上它是实实在在的集合,
这一地位可以由子集公理确保无虞,
这是因为\(\rel{R}[x] \subseteq \ran \rel{R}\).

我们还可以进一步构造等价类的集合,例如\[
	\Set{ \rel{R}[x] \given x \in A },
\]
因为这个集合包含于\(\Powerset(\ran \rel{R})\).

根据等价类的定义,容易得到以下性质.
\begin{property}
设\(\rel{R}\)是\(A\)上的一个等价关系,则有:
\begin{enumerate}
	\item \(x \in A
	\implies \rel{R}[x] \neq \emptyset\).

	\item \(x,y \in A
	\implies \rel{R}[x] = \rel{R}[y]
	\lor \rel{R}[x] \cap \rel{R}[y] = \emptyset\).

	\item \((\forall x,y \in A)
	[
		\rel{R}[x] = \rel{R}[y]
		\implies
		x \rel{R} y
		\iff
		x \in \rel{R}[y] \land y \in \rel{R}[x]
	]\).

	\item \((\forall x,y \in A)
	[
		\rel{R}[x] \neq \rel{R}[y]
		\implies
		\rel{R}[x] \cap \rel{R}[y] = \emptyset
	]\).
\end{enumerate}
\end{property}


\begin{lemma}\label{theorem:集合论.相等的等价类的代表等价}
%@see: 《Elements of Set Theory》 P57. Theorem 3N
设\(\rel{R}\)是\(A\)上的一个等价关系,\(x,y \in A\),
那么\[
	\rel{R}[x] = \rel{R}[y]
	\iff
	x \rel{R} y.
\]
\begin{proof}
首先,设\(\rel{R}[x] = \rel{R}[y]\).
由于等价关系\(\rel{R}\)具有自反性,\(y \rel{R} y\),\(y \in \rel{R}[y]\);
那么由\(\rel{R}[x] = \rel{R}[y]\)就有\(y \in \rel{R}[x]\);
根据等价类\(\rel{R}[x]\)的定义,便得\(x \rel{R} y\).

然后,设\(x \rel{R} y\).
任取\(t\),若有\(t \in \rel{R}[y]\),
根据等价类\(\rel{R}[y]\)的定义,
必有\(y \rel{R} t\);
再根据假设条件\(x \rel{R} y\),以及等价关系\(\rel{R}\)具有传递性,
立即可得\(x \rel{R} t\);
那么根据等价类\(\rel{R}[x]\)的定义,
就有\(t \in \rel{R}[x]\);
因此,\(\rel{R}[y] \subseteq \rel{R}[x]\).
又因为\(\rel{R}\)具有对称性,
从\(x \rel{R} y\)还可得到\(y \rel{R} x\),
参照上面的推导过程,交换\(x\)和\(y\)符号,
不难得到\(\rel{R}[x] \subseteq \rel{R}[y]\),
所以\(\rel{R}[x] = \rel{R}[y]\).
\end{proof}
\end{lemma}
从\cref{theorem:集合论.相等的等价类的代表等价} 可以看出,
相等等价类的代表等价,代表等价的等价类相等.

\begin{definition}\label{definition:集合论.划分的定义}
%@see: 《Elements of Set Theory》 P57. Definition
设\(A,\Pi\)都是集合.
若\(\Pi\)满足:
\begin{enumerate}
	\item \(\Pi\)中的元素都是\(A\)的非空子集,即\[
		(\forall p \in \Pi)
		[
			p \neq \emptyset
			\land
			p \subseteq A
		].
	\]

	\item \(\Pi\)中的元素两两互斥,即\[
		(\forall p,q \in \Pi)[p \cap q = \emptyset].
	\]

	\item \(A\)中的元素是\(\Pi\)中某个元素的元素,即\[
		(\forall a \in A)
		(\exists p \in \Pi)
		[a \in p]
		\quad\text{或}\quad
		\bigcup\Pi = A.
	\]
\end{enumerate}
则称“\(\Pi\)是\(A\)的一个\DefineConcept{划分}(partition)”.
\end{definition}

\begin{theorem}
%@see: 《Elements of Set Theory》 P57. Theorem 3P
设\(\rel{R}\)是\(A\)上的一个等价关系,那么由所有等价类组成的集合\[
	\Pi = \Set{ \rel{R}[x] \given x \in A }
\]就是\(A\)的一个划分.
\begin{proof}
对于任一等价类\(\rel{R}[x]\),
由于总有\(x \in \rel{R}[x]\),
它永远不可能是空集;
又因为\(\rel{R}\)是\(A\)上的二元关系,\(\rel{R} \subseteq A^2\),
\(\rel{R}[x] \subseteq \ran \rel{R}\),
所以\(\rel{R}[x]\)一定是\(A\)的子集.
因此,\(\Pi\)满足\cref{definition:集合论.划分的定义} 中的第1条和第3条,
也就是说,在这里我们只需要证明第2条:\(\Pi\)中的元素是互不重叠的.
用反证法,设\(\rel{R}[x] \neq \rel{R}[y]\ (x,y \in A)\),
而且存在\(t \in \rel{R}[x] \cap \rel{R}[y]\),
于是有\[
	x \rel{R} t \land y \rel{R} t,
	\quad\text{即}\quad
	x \rel{R} y,
\]
再根据\cref{theorem:集合论.相等的等价类的代表等价},
必有\(\rel{R}[x] = \rel{R}[y]\),
即\(\rel{R}[x],\rel{R}[y]\)是同一个元素,矛盾!
\end{proof}
\end{theorem}

\begin{definition}
%@see: 《Elements of Set Theory》 P58.
设\(\rel{R}\)是\(A\)上的一个等价关系.
集合\[
	\Set{ \rel{R}[x] \given x \in A }
\]称为“\(A\)在\(\rel{R}\)下的\DefineConcept{划分}”,
或称为“\(A\)对\(\rel{R}\)的\DefineConcept{商集}(quotient set)”,
记作\(A/\rel{R}\),
读作“\(A\)余\(\rel{R}\)
(\(A\) modulo \(\rel{R}\))”.
把映射\[
	\phi\colon A \to A/\rel{R}, x \mapsto \rel{R}[x]
\]称为\DefineConcept{自然映射}(natural map)%
或\DefineConcept{典范映射}(canonical map).
\end{definition}
对于一个非空集合\(A\),通过建立\(A\)上的一个等价关系\(\rel{R}\),
得到\(A\)对于\(\rel{R}\)的商集\(A/\rel{R}\),
进而研究商集\(A/\rel{R}\)的性质,
这就是抽象代数的基本方法之一.

现在我们来研究如何在商集上定义映射.
具体而言,设\(\rel{R}\)是\(A\)上的一个等价关系,
映射\(F\colon A \to A\).
我们想知道是否存在一个对应的映射\(\hat{F}\colon A/\rel{R} \to A/\rel{R}\),
使得对于任意\(x \in A\),总有\[
	\hat{F}(\rel{R}[x]) = \rel{R}[F(x)].
\]
这里我们可以尝试依靠在等价类\(\rel{R}[x]\)中选择某个特定的元素\(x\),
定义等价类\(\rel{R}[x]\)在映射\(\hat{F}\)下的值.
不过,假如\(x_1,x_2\)都在同一个等价类中,
那么除非\(F(x_1),F(x_2)\)也都在同一个等价类中,
否则映射\(\hat{F}\)就不是良定的!

为了给出一个一般性结论,我们先了解这样一个概念:
如果\[
	(\forall x,y \in A)
	[
		x \rel{R} y
		\implies
		\opair{F(x),F(y)} \in \rel{R}
	],
\]
那么我们称“\(F\)和\(\rel{R}\) \DefineConcept{兼容}%
(\(F\) is \emph{compatible} with \(\rel{R}\))”.

\begin{theorem}\label{theorem:集合论.与等价关系兼容的映射的性质}
%@see: 《Elements of Set Theory》 P60. Theorem 3Q
设\(\rel{R}\)是\(A\)上的一个等价关系,映射\(F\colon A \to A\).
如果\(F\)和\(\rel{R}\)兼容,
那么存在一个唯一的映射\(\hat{F}\colon A/\rel{R} \to A/\rel{R}\),使得\[
	(\forall x \in A)
	[
		\hat{F}(\rel{R}[x]) = \rel{R}[F(x)]
	].
\]
如果\(F\)和\(\rel{R}\)不兼容,
那么不存在映射\(\hat{F}\)满足上述条件.
\begin{proof}
首先假设\(F\)和\(\rel{R}\)不兼容,即\[
	(\exists x,y \in A)
	[
		x \rel{R} y
		\land
		\opair{F(x),F(y)} \notin \rel{R}
	],
\]
也即\[
	(\exists x,y \in A)
	[
		\rel{R}[x] = \rel{R}[y]
		\land
		\rel{R}[F(x)] \neq \rel{R}[F(y)]
	].
\]
而要使\[
	(\forall x \in A)
	[
		\hat{F}(\rel{R}[x]) = \rel{R}[F(x)]
	]
\]成立,
必须有\[
	\hat{F}(\rel{R}[x])
	= \rel{R}[F(x)]
	\quad\text{和}\quad
	\hat{F}(\rel{R}[y])
	= \rel{R}[F(y)]
\]同时成立,
但这是不可能的,
毕竟上面两式的左边相等而右边不等.

接下来,我们假设\(F\)和\(\rel{R}\)兼容.
由于结论要求\(\opair{\rel{R}[x],\rel{R}[F(x)]} \in \hat{F}\),
所以我们可以令\[
	\hat{F} = \Set{ \opair{\rel{R}[x],\rel{R}[F(x)]} \given x \in A }.
\]
现在就需要证明关系\(\hat{F}\)是一个映射.
考虑\(\opair{\rel{R}[x],\rel{R}[F(x)]},
\opair{\rel{R}[y],\rel{R}[F(y)]} \in \hat{F}\),
由于\begin{align*}
	\rel{R}[x] = \rel{R}[y]
	&\implies
	x \rel{R} y
	\tag{\cref{theorem:集合论.相等的等价类的代表等价}} \\
	&\implies
	\opair{F(x),F(y)} \in \rel{R} \\
	&\implies
	\rel{R}[F(x)] = \rel{R}[F(y)],
	\tag{\cref{theorem:集合论.相等的等价类的代表等价}}
\end{align*}
\(\hat{F}\)是单值的,
可见\(\hat{F}\)确实是一个映射.
显然有\(\dom \hat{F} = A/\rel{R}\),\(\ran \hat{F} \subseteq A/\rel{R}\),
因此\(\hat{F}\)是从\(A/\rel{R}\)到\(A/\rel{R}\)的映射.
%TODO 没有给出唯一性的证明
\end{proof}
\end{theorem}
上述结论还可以推广到映射是\(F\colon A \times A \to A\)的情形.
