\section{映射}
\subsection{映射的概念}
\begin{definition}
设\(F\)是关系,如果\[
	(\forall x \in \dom F)(\exists! y)[\opair{x,y} \in F],
\]
则称“关系\(F\)是一个\DefineConcept{映射}(function)”.
\end{definition}
可以从映射的定义中看出,虽然映射也是关系,
但映射有一般的关系所没有的特殊性质:
映射是\DefineConcept{单值的}(single-valued).
换句话说,对于关系\(F\),每个\(x\)可能对应若干个\(y\);
但是,对于映射\(F\),每个\(x\)就只对应一个\(y\).
我们可以把\(x\)与\(y\)这两个元素之间的对应关系记为\(x \mapsto y\).

我们把使得\(xFy\)成立的\(y\)称为“\(x\)(在映射\(F\)下)的\DefineConcept{像}%
(the \emph{value} of \(F\) at \(x\))”,
记为\(F(x)\),即\[
	y = f(x);
\]
称\(x\)为“\(y\)(在映射\(F\)下)的一个\DefineConcept{原像}”.
这里用的\(F(x)\)符号是欧拉提出的,
我们仅当\(F\)是一个映射且\(x\in\dom F\)时使用这个记号.
不过,我们也可以定义:\[
	F(x) \defeq \bigcup\Set{ y \given \opair{x,y} \in F }.
\]
它对于任意\(F\)和\(x\)都有意义.

映射是如此重要,以至于各家对用于描述映射的术语没有达成统一.
以下是两种最常采用的术语.

设\(X,Y\)都是集合,
如果\(f\)是一个映射,且\(\dom f = X\),\(\ran f \subseteq Y\),
则称“\(f\)是从\(X\)到\(Y\)的\DefineConcept{映射}%
(\(f\) is a function \emph{from} \(X\) \emph{into} \(Y\))”,
或称“\(f\)将\(X\)映射到\(Y\)里%
(\(f\) \emph{maps} \(X\) \emph{into} \(Y\))”,
记作\[
	f\colon X \to Y.
\]
如果还有\(\ran f = Y\),
那么称“\(f\)是从\(X\)到\(Y\)上的映射%
(\(f\) is a function from \(X\) \emph{onto} \(Y\))”,
或称“\(f\)将\(X\)映射到\(Y\)上%
(\(f\) \emph{maps} \(X\) \emph{onto} \(Y\))”,
或称“\(f\)是\DefineConcept{满射}(surjective)”.
我们可以说“任意映射总将它的定义域映射到它的值域上”,
还可以说“任意映射总把它的定义域映射到以它的值域为子集的任意集合\(B\)里”.
注意到两种说法的区别,“上”字和“里”字的选用,
不光取决于映射\(f\)本身,还取决于我们讨论的集合\(B\).

如果\[
	(\forall y \in \ran f)
	(\exists! x)
	[\opair{x,y} \in f],
\]
那么称“映射\(f\)是\DefineConcept{一一映射}(one-to-one)”.

有时候我们希望把一一映射的概念套用到一般的关系上,
它们往往不是映射,因此“一一映射”这个用词就显得不那么恰当了.
为此,我们创造出“单根的”,类比于“单值的”.

\begin{definition}
如果集合\(R\)满足\[
	(\forall y \in \ran R)
	(\exists! x)
	[\opair{x,y} \in R],
\]
则称“\(R\)是\DefineConcept{单根的}(single-rooted)”.
\end{definition}

因此,我们可以说,“一个映射是单根的”当且仅当“这个映射是一一映射”.

如果\[
	(\forall x_1, x_2 \in \dom f)
	[x_1 \neq x_2 \implies f(x_1) \neq f(x_2)],
\]
那么称“\(f\)是\DefineConcept{单射}(injective)”.
由于映射本就是单值的,若它还是单根的,那么这个映射就是单射;
换句话说,一一映射和单射是相同的概念.

如果\(f\)既是单射,又是满射,即
那么称“\(f\)是\DefineConcept{双射}(bijective)\footnote{%
有的书可能会把双射称作一一映射,但是我们不采取这种说法.
}”.

\begin{example}
%@see: 《Elements of Set Theory》 P52. Exercise 11
设\(F,G\)都是映射,
\(\dom F = \dom G = X\),且\[
	(\forall x \in X)[F(x) = G(x)].
\]
证明:\(F=G\).
\begin{proof}
显然有
\begin{align*}
	F=G
	&\iff (\forall x \in X)(\exists!y)[
		\opair{x,y} \in F
		\iff
		\opair{x,y} \in G
	] \\
	&\iff (\forall x \in X)(\exists!y)[
		y = F(x) = G(x)
	] \\
	&\iff (\forall x \in X)[F(x) = G(x)].
	\qedhere
\end{align*}
\end{proof}
\end{example}

\begin{example}
%@see: 《Elements of Set Theory》 P52. Exercise 12
设\(f,g\)都是映射.
证明:\[
	f \subseteq g
	\iff
	\dom f \subseteq \dom g
	\land
	(\forall x \in \dom f)
	[f(x) = g(x)].
\]
\begin{proof}
直接有
\begin{align*}
	f \subseteq g
	&\iff (\forall x)(\exists!y)[\opair{x,y} \in f \implies \opair{x,y} \in g] \\
	&\iff (\forall x \in \dom f)[x \in \dom g]\land(\forall x \in \dom f)(\exists!y)[y=f(x)=g(x)] \\
	&\iff [\dom f \subseteq \dom g]\land(\forall x \in \dom f)[f(x)=g(x)].
	\qedhere
\end{align*}
\end{proof}
\end{example}

\begin{example}
%@see: 《Elements of Set Theory》 P53. Exercise 13
设\(f,g\)都是映射,\(f \subseteq g\)且\(\dom g \subseteq \dom f\).
证明:\(f=g\).
%TODO
\end{example}

\begin{example}
%@see: 《Elements of Set Theory》 P53. Exercise 14
设\(f,g\)都是映射.
证明:
\begin{enumerate}
	\item \(f \cap g\)是映射.
	\item \(f \cup g\)是映射的充分必要条件是\[
		(\forall x \in (\dom f)\cap(\dom g))[f(x)=g(x)].
	\]
\end{enumerate}
%TODO
\end{example}

\begin{example}
%@see: 《Elements of Set Theory》 P53. Exercise 15
\def\A{\mathscr{A}}%
设\(\A\)是一组映射,且\[
	(\forall f,g\in\A)[f \subseteq g \lor g \subseteq f].
\]
证明:\(\bigcup\A\)也是映射.
%TODO
\end{example}

\begin{example}
%@see: 《Elements of Set Theory》 P53. Exercise 16
证明:不存在一个集合,使得每个映射都属于它.
%TODO
\end{example}

\subsection{逆,复合,限制,像,原像}
以下定义的操作通常用在映射上,有时候也用于关系,但也可以用于任意集合.
\begin{definition}
设\(A,F,G\)都是集合.
\begin{enumerate}
	\item 称集合\[
		\Set*{ \opair{u,v} \given \opair{v,u} \in F }
	\]为“\(F\)的\DefineConcept{逆}%
	(the \emph{inverse} of \(F\))”,
	记作\(F^{-1}\).

	特别地,如果\(F^{-1}\)是映射,
	则称“\(F^{-1}\)是\(F\)的\DefineConcept{逆映射}”.

	\item 称集合\[
		\Set*{ \opair{u,v} \given (\exists t)[\opair{u,t} \in G \land \opair{t,v} \in F] }
	\]为“\(F\)和\(G\)的\DefineConcept{复合}%
	(the \emph{composition} of \(F\) and \(G\))”,
	记作\(F \circ G\).

	\item 称集合\[
		\Set*{ \opair{u,v} \given \opair{u,v} \in F \land u \in A }
	\]为“\(F\)在\(A\)上的\DefineConcept{限制}%
	(the \emph{restriction} of \(F\) to \(A\))”,
	记作\(F \upharpoonright A\).

	\item 称集合\[
		\Set*{ v \given (\exists u \in A)[\opair{u,v} \in F] }
	\]为“\(A\)在\(F\)下的\DefineConcept{像}%
	(the \emph{image} of \(A\) \emph{under} \(F\))”,
	记作\(F\ImageOfSetUnderRelation{A}\).
\end{enumerate}
\end{definition}

当\(F\)是一个映射,且\(A \subseteq \dom F\)时,
\(F\ImageOfSetUnderRelation{A}\)这个概念可能更容易理解,
因为这时候\[
	F\ImageOfSetUnderRelation{A}
	= \Set{ F(u) \given u \in A }.
\]

我们可以利用子集公理构造出上述定义下的所需集合的存在性.
特别地,\[
	F^{-1} \subseteq \ran F \times \dom F, \qquad
	F \circ G \subseteq \dom G \times \ran F,
\]\[
	F \upharpoonright A \subseteq F, \qquad
	F\ImageOfSetUnderRelation{A} \subseteq \ran F.
\]

例如,我们可以按如下方法正当化“关系\(F\)的逆”的定义:
根据子集公理,存在集合\(B\),使得对于任意\(x\),总有\begin{align*}
	x \in B
	&\iff
	[x \in \ran F \times \dom F]
	\land
	(\exists u)(\exists v)[x = \opair{u,v} \land \opair{v,u} \in F], \\
	&\iff
	(\exists u)(\exists v)[x = \opair{u,v} \land \opair{v,u} \in F].
\end{align*}
再根据外延公理,可以保证集合\(B\)的唯一性.
因此我们可以将集合\(B\)记为\(F^{-1}\).

\begin{theorem}
\(F \upharpoonright \emptyset = \emptyset\).
\end{theorem}

\begin{theorem}
\(F\ImageOfSetUnderRelation{A} = \ran(F \upharpoonright A)\).
\begin{proof}
根据值域的定义有\begin{align*}
	v \in \ran(F \upharpoonright A)
	&\iff
	(\exists u)[\opair{u,v} \in F \upharpoonright A] \\
	&\iff
	(\exists u)[\opair{u,v} \in F \land u \in A] \\
	&\iff
	v \in F\ImageOfSetUnderRelation{A}.
	\qedhere
\end{align*}
\end{proof}
\end{theorem}

\begin{definition}
设\(F\)是关系,\(A\)是集合,那么称集合\[
	\Set*{ x \in \dom F \given F(x) \in A }
\]为“集合\(A\)在关系\(F\)下的\DefineConcept{原像}%
(the \emph{inverse image} of \(A\) under \(F\))”,
记作\(F^{-1}\ImageOfSetUnderRelation{A}\).
\end{definition}

\begin{theorem}\label{theorem:集合论.关系的逆的定义域值域以及关系的二重逆}
%@see: 《Elements of Set Theory》 P46. Theorem 3E
设\(F\)是集合,则有\begin{gather}
	\dom F^{-1} = \ran F, \\
	\ran F^{-1} = \dom F.
\end{gather}

如果\(F\)是关系,则有\begin{equation}
	(F^{-1})^{-1} = F.
\end{equation}
\end{theorem}

\begin{theorem}\label{theorem:集合论.关系及其逆是映射的充分必要条件}
%@see: 《Elements of Set Theory》 P46. Theorem 3F
设\(F\)是集合,则“\(F^{-1}\)是映射”的充分必要条件是:\(F\)是单根的.

设\(F\)是关系,则“\(F\)是映射”的充分必要条件是:\(F^{-1}\)是单根的.
\end{theorem}

\begin{theorem}\label{theorem:集合论.逆映射的计算}
%@see: 《Elements of Set Theory》 P46. Theorem 3G
设\(F\)是一一映射.
\begin{enumerate}
	\item 如果\(x \in \dom F\),那么\[
		F^{-1}(F(x)) = x.
	\]

	\item 如果\(y \in \ran F\),那么\[
		F(F^{-1}(y)) = y.
	\]
\end{enumerate}
\begin{proof}
假设\(x \in \dom F\),
那么\(\opair{x,F(x)} \in F\),且\(\opair{F(x),x} \in F^{-1}\),
于是\(F(x) \in \dom F^{-1}\).
因为\(F\)是一一映射,是单值的,所以\(F^{-1}\)是映射,
从而\(x = F^{-1}(F(x))\).

如果\(y \in \ran F\),
那么根据本定理第1条,以及\((F^{-1})^{-1} = F\),可知\[
	F(F^{-1}(y)) = (F^{-1})^{-1}(F^{-1}(y)) = y.
	\qedhere
\]
\end{proof}
\end{theorem}

\begin{theorem}\label{theorem:集合论.映射的复合也是映射}
%@see: 《Elements of Set Theory》 P47. Theorem 3H
设\(F,G\)都是映射,则\(F \circ G\)是映射,且\[
	\dom(F \circ G)
	= \Set*{ x \in \dom G \given G(x) \in \dom F },
\]\[
	(\forall x \in \dom(F \circ G))
	[(F \circ G)(x) = F(G(x))].
\]
\begin{proof}
要证\(F \circ G\)是一个映射,
假设有\(\opair{x,y} \in F \circ G\)和\(\opair{x,z} \in F \circ G\)同时成立.
那么,\[
	(\exists p)[\opair{x,p} \in G \land \opair{p,y} \in F]
	\quad\text{和}\quad
	(\exists q)[\opair{x,q} \in G \land \opair{q,z} \in F]
\]同时成立.
既然\(G\)是映射,必有\(p = q\).
同理,\(F\)是映射,必有\(y = z\).
因此\(F \circ G\)是映射.

现在再假设\(x \in \dom G\)且\(G(x) \in \dom F\).
我们必须证明\[
	x \in \dom(F \circ G)
	\quad\text{和}\quad
	(F \circ G)(x) = F(G(x)).
\]
我们知道\[
	\opair{x,G(x)} \in G,
	\qquad
	\opair{G(x),F(G(x))} \in F.
\]
因此\(\opair{x,F(G(x))} \in F \circ G\).

反过来说,如果\(x \in \dom(F \circ G)\),
那么就有\[
	(\exists y)(\exists t)
	[\opair{x,t} \in G \land \opair{t,y} \in F].
\]
于是就有\(x \in \dom G\)和\(t = G(x) \in \dom F\).
\end{proof}
\end{theorem}

容易看出,映射的复合是有顺序的,
\(f \circ g\)有意义并不代表\(g \circ f\)也有意义.
即便两者都有意义,它们也未必相同.

\begin{example}
假设\(G\)是某个一一映射,
那么,根据\cref{theorem:集合论.映射的复合也是映射},
\(G^{-1} \circ G\)也是一个映射,
它的定义域为\[
	\Set{ x \in \dom G \given G(x) \in \dom G^{-1} }
	= \dom G,
\]
并且,对于\(\forall x \in \dom(G^{-1} \circ G)\),有\begin{align*}
	(G^{-1} \circ G)(x) &= G^{-1}(G(x)) \\
	&= x. \tag{\cref{theorem:集合论.逆映射的计算}}
\end{align*}
因此,\(G^{-1} \circ G\)就是\(I_{\dom G}\),
\(\dom G\)上的恒等映射.
同理,\(G \circ G^{-1}\)是\(I_{\ran G}\),
\(\ran G\)上的恒等映射.
\end{example}

\begin{theorem}
%@see: 《Elements of Set Theory》 P65. Exercise 53.
设\(R,S\)是集合,那么\begin{gather}
	(R \cup S)^{-1} = R^{-1} \cup S^{-1},
	\label{equation:集合论.并的逆等于逆的并} \\
	(R \cap S)^{-1} = R^{-1} \cap S^{-1},
	\label{equation:集合论.交的逆等于逆的交} \\
	(R - S)^{-1} = R^{-1} - S^{-1}.
	\label{equation:集合论.差的逆等于逆的差}
\end{gather}
\begin{proof}
对\cref{equation:集合论.并的逆等于逆的并} 证明如下:
\begin{align*}
	\opair{x,y} \in (R \cup S)^{-1}
	&\iff \opair{y,x} \in R \cup S \\
	&\iff \opair{y,x} \in R \lor \opair{y,x} \in S \\
	&\iff \opair{x,y} \in R^{-1} \lor \opair{x,y} \in S^{-1} \\
	&\iff \opair{x,y} \in R^{-1} \cup S^{-1}.
\end{align*}

对\cref{equation:集合论.交的逆等于逆的交} 证明如下:
\begin{align*}
	\opair{x,y} \in (R \cap S)^{-1}
	&\iff \opair{y,x} \in R \cap S \\
	&\iff \opair{y,x} \in R \land \opair{y,x} \in S \\
	&\iff \opair{x,y} \in R^{-1} \land \opair{x,y} \in S^{-1} \\
	&\iff \opair{x,y} \in R^{-1} \cap S^{-1}.
\end{align*}

对\cref{equation:集合论.差的逆等于逆的差} 证明如下:
\begin{align*}
	\opair{x,y} \in (R - S)^{-1}
	&\iff \opair{y,x} \in R - S \\
	&\iff \opair{y,x} \in R \land \opair{y,x} \notin S \\
	&\iff \opair{x,y} \in R^{-1} \land \opair{x,y} \notin S^{-1} \\
	&\iff \opair{x,y} \in R^{-1} - S^{-1}.
	\qedhere
\end{align*}
\end{proof}
\end{theorem}

\begin{theorem}
%@see: 《Elements of Set Theory》 P47. Theorem 3I
设\(F,G\)都是集合,那么\[
	(F \circ G)^{-1} = G^{-1} \circ F^{-1}.
\]
\begin{proof}
易知\((F \circ G)^{-1}\)和\(G^{-1} \circ F^{-1}\)都是关系,且\begin{align*}
	\opair{x,y} \in (F \circ G)^{-1}
	&\iff
	\opair{y,x} \in F \circ G \\
	&\iff
	(\exists t)[\opair{y,t} \in G \land \opair{t,x} \in F] \\
	&\iff
	(\exists t)[\opair{x,t} \in F^{-1} \land \opair{t,y} \in G^{-1}] \\
	&\iff
	\opair{x,y} \in G^{-1} \circ F^{-1}.
	\qedhere
\end{align*}
\end{proof}
\end{theorem}

\begin{axiom}[选择公理(第一种形式)]
对于任意关系\(R\),存在映射\(H\),满足\[
	H \subseteq R,
	\quad\text{且}\quad
	\dom H = \dom R.
\]
\end{axiom}

\begin{theorem}
%@see: 《Elements of Set Theory》 P48. Theorem 3J
设映射\(F\colon A \to B\),其中\(A\)是非空集合.
\begin{enumerate}
	\item “存在映射\(G\colon B \to A\)(称其为\DefineConcept{左逆}),
	使得\(G \circ F\)是\(A\)上的恒等映射\(I_A\)”是“\(F\)是一一映射”的充分必要条件.

	\item “存在映射\(H\colon B \to A\)(称其为\DefineConcept{右逆}),
	使得\(F \circ H\)是\(B\)上的恒等映射\(I_B\)”是“\(F\)是满射”的充分必要条件.
\end{enumerate}
\begin{proof}
\begin{enumerate}
	\item
	先证充分性.
	我们假设存在映射\(G\)使得\(G \circ F = I_A\).
	如果\(F(x) = F(y)\),那么\[
		x = G(F(x)) = G(F(y)) = y,
	\]
	于是\(F\)是一一映射.

	再证必要性.
	假设\(F\)是一一映射,
	那么根据\cref{theorem:集合论.关系的逆的定义域值域以及关系的二重逆,theorem:集合论.关系及其逆是映射的充分必要条件},
	\(F^{-1}\)是一个从\(\ran F\)到\(A\)上的映射.
	现在我们需要将\(F^{-1}\)延拓为以\(B\)为定义域的映射\(G\).
	因为\(A\)是非空集合,
	于是我们可以取定\(a \in A\),
	然后令\[
		G(x) = \left\{ \begin{array}{ll}
			F^{-1}(x), & x \in \ran F, \\
			a, & x \in B - \ran F,
		\end{array} \right.
	\]或者令\[
		G = F^{-1} \cup (B - \ran F) \times \Set{a}.
	\]
	这个构造出来的映射\(G\)是一个从\(B\)到\(A\)里的映射,
	且满足\[
		\dom(G \circ F) = A,
	\]
	以及\[
		(\forall x \in A)[G(F(x)) = F^{-1}(F(x)) = x],
	\]
	于是\(G \circ F = I_A\)成立.

	\item
	我们还是先证充分性.
	假设存在映射\(H\)使得\(F \circ H = I_B\).
	那么\[
		(\forall y \in B)[y = F(H(y))],
	\]
	从而\(y \in \ran F\),
	于是\(\ran F = B\).

	必要性的证明稍显困难.
	我们不能直接取\(H = F^{-1}\),
	因为一般而言\(F\)不会是一一映射,
	\(F^{-1}\)也不会是一个映射.
	假设\(F\)将\(A\)映射到\(B\)上,\(\ran F = B\).
	现在我们需要为每个\(y \in B\)选择某个\(x\),使得\(F(x) = y\),然后令\(H(y) = x\);
	考虑到\(y \in \ran F\),这样的\(x\)必定存在.
	虽然我们知道对于每个\(y\),存在一个合适的\(x\),
	但是我们无法据此构造所求映射\(H\).
	因此,我们需要引入选择公理.
	借助选择公理,我们可以令映射\(H\)满足\(H \subseteq F^{-1}\)且\(\dom H = \dom F^{-1} = B\).
	于是\(H\)满足\[
		(\forall y \in B)
		[
			\opair{y,H(y)} \in F^{-1}
			\iff
			\opair{H(y),y} \in F
			\iff
			F(H(y)) = y
		].
		\qedhere
	\]
\end{enumerate}
\end{proof}
\end{theorem}

\begin{theorem}
%@see: 《Elements of Set Theory》 P48. Theorem 3K
设\(A,B,F\)都是集合.
\def\F#1{F\ImageOfSetUnderRelation{#1}}
\begin{enumerate}
	\item 并的像是像的并:\begin{gather}
		\F{A \cup B}
		= \F{A} \cup \F{B},
		\label{equation:集合论.并的像与像的并的关系1} \\
		\F{\bigcup A}
		= \bigcup\Set{ \F{a} \given a \in A }.
		\label{equation:集合论.并的像与像的并的关系2}
	\end{gather}

	\item 交的像包含于像的交:\begin{gather}
		\F{A \cap B}
		\subseteq \F{A} \cap \F{B},
		\label{equation:集合论.交的像与像的交的关系1} \\
		\F{\bigcap A}
		\subseteq \bigcap\Set{ \F{a} \given a \in A }.
		\label{equation:集合论.交的像与像的交的关系2}
		\quad(A \neq \emptyset)
	\end{gather}
	若\(F\)是单根的,则以上两式取“=”号.

	\item 差的像包含像的差:\begin{equation}
		\F{A} - \F{B}
		\subseteq \F{A-B}.
		\label{equation:集合论.差的像与像的差的关系}
	\end{equation}
	若\(F\)是单根的,则上式取“=”号.
\end{enumerate}
\begin{proof}
\cref{equation:集合论.并的像与像的并的关系1} 证明如下:
\begin{align*}
	y \in \F{A \cup B}
	&\iff (\exists x \in A \cup B)[\opair{x,y} \in F] \\
	&\iff (\exists x \in A)[\opair{x,y} \in F]
			\lor (\exists x \in B)[\opair{x,y} \in F] \\
	&\iff y \in \F{A} \lor y \in \F{B}.
\end{align*}

\cref{equation:集合论.交的像与像的交的关系1} 证明如下:
\begin{align*}
	y \in \F{A \cap B}
	&\iff (\exists x \in A \cap B)[\opair{x,y} \in F] \\
	&\implies (\exists x \in A)[\opair{x,y} \in F]
		\land (\exists x \in B)[\opair{x,y} \in F] \\
	&\iff y \in \F{A} \land y \in \F{B}.
\end{align*}
注意到中间步骤的\(\implies\)不总是可逆的,
这时因为虽然有\[
	(\exists x_1 \in A)[\opair{x_1,y} \in F], \qquad
	(\exists x_2 \in B)[\opair{x_2,y} \in F],
\]
但是可能\[
	(\forall x \in A \cap B)[\opair{x,y} \notin F].
\]
不过,如果\(F\)是单根的,那么必有\(x_1 = x_2 \in A \cap B\),
这时候中间步骤的\(\implies\)是可逆的,可以改为\(\iff\).

\cref{equation:集合论.并的像与像的并的关系2,equation:集合论.交的像与像的交的关系2} 分别是%
\cref{equation:集合论.并的像与像的并的关系1,equation:集合论.交的像与像的交的关系1} 的简单推广,
故略去证明.

\cref{equation:集合论.差的像与像的差的关系} 证明如下:
\begin{align*}
	y \in \F{A} - \F{B}
	&\iff (\exists x \in A)[\opair{x,y} \in F]
		\land \neg[(\exists t \in B)[\opair{t,y} \in F]] \\
	&\implies (\exists x \in A - B)[\opair{x,y} \in F] \\
	&\iff y \in \F{A - B}.
\end{align*}
若\(F\)是单根的,则\[
	(\exists! x)[\opair{x,y} \in F].
\]
这种情况下,中间步骤的\(\implies\)可以改为\(\iff\).
\end{proof}
\end{theorem}

\begin{corollary}
%@see: 《Elements of Set Theory》 P48. Corollary 3L
设\(G\)是映射,\(A,B\)都是集合.
\def\G#1{G^{-1}\ImageOfSetUnderRelation{#1}}
\begin{gather}
	\G{\bigcup A} = \bigcup\Set*{ \G{a} \given a \in A },
	\label{equation:集合论.并的原像与原像的并的关系} \\
	\G{\bigcap A} = \bigcap\Set*{ \G{a} \given a \in A }, \quad A \neq \emptyset,
	\label{equation:集合论.交的原像与原像的交的关系} \\
	\G{A - B} = \G{A} - \G{B}.
	\label{equation:集合论.差的原像与原像的差的关系}
\end{gather}
\end{corollary}

\begin{example}
%@see: 《Elements of Set Theory》 P53. Exercise 22.(a)
\def\F#1{F\ImageOfSetUnderRelation{#1}}
证明:\begin{equation}
	A \subseteq B \implies \F{A} \subseteq \F{B}.
\end{equation}
\begin{proof}
因为\(A \subseteq B\),所以\(A \cap B = A\),
那么由\cref{equation:集合论.交的像与像的交的关系1} 可知,\[
	\F{A} = \F{A \cap B} \subseteq \F{A} \cap \F{B} \subseteq \F{B}.
	\qedhere
\]
\end{proof}
\end{example}

\begin{example}
%@see: 《Elements of Set Theory》 P53. Exercise 22.(b)
证明:\begin{equation}
	(F \circ G)\ImageOfSetUnderRelation{A}
	= F\ImageOfSetUnderRelation{G\ImageOfSetUnderRelation{A}}.
\end{equation}
%TODO
\end{example}

\begin{example}
%@see: 《Elements of Set Theory》 P53. Exercise 22.(c)
证明:\begin{equation}
	Q \upharpoonright (A \cup B)
	= (Q \upharpoonright A)\cup(Q \upharpoonright B).
\end{equation}
%TODO
\end{example}

\begin{example}
%@see: 《Elements of Set Theory》 P65. Exercise 59.
设\(A,B,Q\)是集合.
证明:\begin{gather}
	Q \upharpoonright (A \cap B)
	= (Q \upharpoonright A) \cap (Q \upharpoonright B), \\
	Q \upharpoonright (A - B)
	= (Q \upharpoonright A)
	- (Q \upharpoonright B).
\end{gather}
%TODO
\end{example}

\begin{example}
%@see: 《Elements of Set Theory》 P65. Exercise 60.
设\(A,R,S\)是集合.
证明:\begin{equation}
	(R \circ S) \upharpoonright A = R \circ (S \upharpoonright A).
\end{equation}
%TODO
\end{example}

\subsection{指标集}
\begin{definition}
设\(F\)是映射,\(I\)是集合,\(\dom F \supseteq I\).
定义:\[
	\bigcup_{i \in I} F(i) \defeq \bigcup\Set{ F(i) \given i \in I },
\]\[
	\bigcap_{i \in I} F(i) \defeq \bigcap\Set{ F(i) \given i \in I }.
\]
称\(I\)为“\(F\)的\DefineConcept{指标集}(\emph{index} set)”.
同时也称
“\(\bigcup_{i \in I} F(i)\)的指标集是\(I\)”
“\(\bigcap_{i \in I} F(i)\)的指标集是\(I\)”.
\end{definition}

\subsection{单调集列}
%@see: 《测度论讲义(第三版)》(严加安) P2. 1.1.4
\begin{definition}
设\(\{A_n\}_{n\geq1}\)是一个集合序列.

若\[
	(\forall n\in\omega)
	[A_n \subseteq A_{n+1}],
\]
则称“\(\{A_n\}\)是\DefineConcept{单调增集列}”.

若\[
	(\forall n\in\omega)
	[A_n \supseteq A_{n+1}],
\]
则称“\(\{A_n\}\)是\DefineConcept{单调减集列}”.

我们将“单调增集列”与“单调减集列”统称为\DefineConcept{单调集列}.
\end{definition}

\begin{definition}
设\(\{A_n\}_{n\geq1}\)是一个集合序列.

定义:\begin{align*}
	\lim_{n\to\infty} A_n
	&\defeq \bigcup_{n=1}^\infty A_n \\
	&\defeq \bigcap_{n=1}^\infty A_n,
\end{align*}
称其为“\(\{A_n\}\)的\DefineConcept{极限}”.

定义:\[
	\limsup_{n\to\infty} A_n
	\defeq
	\bigcap_{n=1}^\infty
	\bigcup_{k=n}^\infty
	A_k,
\]
称其为“\(\{A_n\}\)的\DefineConcept{上极限}”.

定义:\[
	\liminf_{n\to\infty} A_n
	\defeq
	\bigcup_{n=1}^\infty
	\bigcap_{k=n}^\infty
	A_k,
\]
称其为“\(\{A_n\}\)的\DefineConcept{下极限}”.
\end{definition}

\begin{proposition}
设\(\{A_n\}_{n\geq1}\)是一个集合序列,
则\begin{equation}
	\liminf_{n\to\infty} A_n
	\subseteq
	\limsup_{n\to\infty} A_n.
\end{equation}
%@see: https://math.stackexchange.com/questions/107931/lim-sup-and-lim-inf-of-sequence-of-sets
\end{proposition}

\begin{proposition}
设\(\{A_n\}_{n\geq1}\)是一个集合序列.
若\[
	\liminf_{n\to\infty} A_n
	= \limsup_{n\to\infty} A_n,
\]
则\(\{A_n\}\)的极限存在,
且\[
	\lim_{n\to\infty} A_n
	= \liminf_{n\to\infty} A_n
	= \limsup_{n\to\infty} A_n.
\]
\end{proposition}

%@see: 《测度论讲义(第三版)》(严加安) P2. 1.1.6
\begin{definition}[集族的封闭性]
设\(\mathcal{C}\)是一个非空集族.

如果\[
	(\forall A,B\in\mathcal{C})
	[A \cap B \in \mathcal{C}],
\]
则称“\(\mathcal{C}\)对有限交封闭”.

如果\[
	(\forall \{A_n\})
	(\forall n\geq1)
	\left[A_n\in\mathcal{C} \implies \bigcap_{k=1}^n A_k \in \mathcal{C}\right],
\]
则称“\(\mathcal{C}\)对可列交封闭”.

如果\[
	(\forall A,B\in\mathcal{C})
	[A \cup B \in \mathcal{C}],
\]
则称“\(\mathcal{C}\)对有限并封闭”.

如果\[
	(\forall \{A_n\})
	(\forall n\geq1)
	\left[A_n\in\mathcal{C} \implies \bigcup_{k=1}^n A_k \in \mathcal{C}\right],
\]
则称“\(\mathcal{C}\)对可列并封闭”.
\end{definition}

\begin{definition}
设\(\mathcal{C}\)是一个非空集族.
定义:\[
	\mathcal{C}_{\cap f}
	\defeq
	\Set*{
		A \given
		A = \bigcap_{k=1}^n A_k,
		A_k \in \mathcal{C}, i=1,\dotsc,n,
		n\geq1
	},
\]
称其为“用有限交运算封闭\(\mathcal{C}\)所得的集族”.
\end{definition}

\begin{proposition}
设\(\mathcal{C}\)是一个非空集族,
则\(\mathcal{C}_{\cap f}\)对有限交封闭.
\end{proposition}

\subsection{映射空间}
对于任意给定的集合\(A,X\),定义:\[
	X^A \defeq \Set{ F \given F\ \text{是从\(A\)到\(X\)的映射} }.
\]
我们把\(X^A\)称为“从\(A\)到\(X\)的\DefineConcept{映射空间}”.

因为\(F\colon A \to X\)必有\(F \subseteq A \times X\),\(F \in \Powerset(A \times X)\),
所以我们可以对集合\(\Powerset(A \times X)\)利用子集公理,构造包括全部从\(A\)到\(X\)的映射的集合.

%之所以采取这种表记方式,
%是因为当\(A\)和\(X\)是有限集,且\(\abs{A}=a,\abs{X}=x\)时,
%\(\abs{X^A}=x^a\).

容易看出,对于非空集合\(A\),总有\(\emptyset^A = \emptyset\);
这是因为没有哪个映射会同时有非空的定义域和空的值域.
另一方面,对于任意集合\(A\),总有\(A^\emptyset = \Set{\emptyset}\);
这是因为“空映射”\(\emptyset\colon \emptyset \to A\)的存在,
空映射是唯一的以空集为定义域的映射.
作为特例,我们还有\(\emptyset^\emptyset=\Set{\emptyset}\).
