\chapter{概率论基础}
\section{样本空间与随机事件}
\subsection{随机试验}
\begin{definition}
一个科学实验,或对一个自然现象和社会现象的观察,我们都称为一个试验.
如果一个试验满足以下三个特点,则称之为\DefineConcept{随机试验}:
\begin{enumerate}
	\item 可在相同条件下重复进行;
	\item 试验的所有可能结果不止一个,且试验前知道一切可能结果;
	\item 试验前不知哪一个可能结果出现,试验后能客观确定出现的哪一个结果.
\end{enumerate}
\end{definition}
随机试验以后简称为试验.

\subsection{样本空间与随机事件}
\begin{definition}
一个试验的所有可能结果的集合,称为该试验的\DefineConcept{样本空间},记作\(\Omega\).
这个试验的任何一个可能结果称为一个\DefineConcept{样本点}.
\end{definition}

样本空间是由试验确定的,它可能是有限集,也可能是无限集;
它可以是一维或多维的数集,也可以是抽象的集合.
有时为了数学处理方便,样本空间也可形式上扩大.
例如把样本空间\([a,b]\)扩大为\([a,+\infty)\),甚至是扩大为\((-\infty,+\infty)\).

\begin{definition}
样本空间的子集,称为\DefineConcept{随机事件},简称为\DefineConcept{事件}.
事件常用大写字母\(A,B,C\)等表示.

我们称事件\(A\)在一次试验中发生,当且仅当试验中出现的样本点\(\omega \in A\).
\end{definition}

\begin{definition}
\(\Omega\)本身是\(\Omega\)的子集,它包含所有样本点,
称为\DefineConcept{必然事件},
因为在任意一次试验时\(\Omega\)必然发生.

空集\(\emptyset\)是\(\Omega\)的子集,它不包含任何样本点,
称为\DefineConcept{不可能事件},
因为在任意一次试验时\(\emptyset\)必不发生.

仅含一个样本点\(\omega\)的事件\(B = \{\omega\}\)称为\DefineConcept{基本事件}.
\end{definition}

\subsection{事件的关系及运算}
随机事件是样本空间的一个子集,因而可以根据集合论的知识来讨论事件间的关系与运算.

\begin{definition}
设试验的样本空间是\(\Omega\).

\begin{enumerate}
\item 事件的包含与相等

若\(A \subseteq B\),称事件\(B\)包含事件\(A\),
其概率意义为“若事件\(A\)发生则事件\(B\)一定发生”
或“若事件\(B\)不发生则事件\(A\)一定不发生”.

若\(A \subseteq B\)且\(B \subseteq A\),
则称事件\(A\)与\(B\)相等,记为\(A = B\),
其概率意义为“事件\(A\)与\(B\)要么同时发生,要么同时不发生”.

\item 事件的和(并)

\(A \cup B\)称为“\(A\)与\(B\)的和事件”或“\(A\)与\(B\)的并事件”,
它是由事件\(A\)与\(B\)产生的一个新事件,
表示事件\(A\)与\(B\)至少有一个发生的事件.

和事件可以推广到\(\bigcup\limits_{i=1}^n A_i\)与\(\bigcup\limits_{i=1}^\infty A_i\),
它们分别表示“有限个事件\(A_1,A_2,\dotsc,A_n\)中至少有一个发生”
或“可数无穷个事件\(A_1,A_2,\dotsc,\)中至少有一个发生”的事件.

\item 事件的积(交)

\(A \cap B\)(或记为\(AB\))
称为“\(A\)与\(B\)的积事件”
或“\(A\)与\(B\)的交事件”,
它表示“事件\(A\)与\(B\)都发生”的事件.
同样地,积事件可以推广到\(\bigcap\limits_{i=1}^n A_i\)与\(\bigcap\limits_{i=1}^\infty A_i\),
它们分别表示“有限个事件\(A_1,A_2,\dotsc,A_n\)同时发生”
或“可数无穷个事件\(A_1,A_2,\dotsc,\)同时发生”的事件.

\item 事件的差

事件\(A-B\)称为事件\(A\)与\(B\)的差,
表示“\(A\)发生而\(B\)不发生”的事件.

\item 互斥(互不相容)事件

若\(AB = \emptyset\),即“事件\(A\)与\(B\)不可能同时发生”,
称事件\(A\)与\(B\)为\DefineConcept{互斥事件}(或\DefineConcept{互不相容事件}).
需要注意,基本事件之间是两两互斥的.

\item 互逆(互为独立)事件

若事件\(A\)与\(B\)有\(AB = \emptyset\)且\(A \cup B = \Omega\),
则称\(A\)与\(B\)为\DefineConcept{互逆事件}(或\DefineConcept{互为对立事件}),
因为此时“\(A\)与\(B\)不可能同时发生,但\(A\)与\(B\)必定有一个会发生”,
所以称\(B\)为\(A\)的\DefineConcept{逆事件}(或\DefineConcept{对立事件}),
记作\(\overline{A}\),即“\(A\)不发生”.
这时有\(B = \overline{A}\)和\(A = \overline{B}\).

\item 完备事件组

若事件\(A_1,A_2,\dotsc,A_n\)两两互斥,
且\(A_1 \cup A_2 \cup \dotsb \cup A_n = \Omega\),
则称\(n\)个事件\(A_1,A_2,\dotsc,A_n\)为一个\DefineConcept{完备事件组},
或称之为对样本空间\(\Omega\)的一个\DefineConcept{有限划分}.
可见,完备事件组是互为对立事件的一个延伸.
\end{enumerate}
\end{definition}

\begin{property}
\(A \overline{A} = \emptyset\).
\end{property}

\begin{property}
\(A \cup \overline{A} = \Omega\).
\end{property}

\begin{property}
\(A - B = A \overline{B}\).
\end{property}

\begin{property}
\(\overline{\overline{A}} = A\).
\end{property}

\begin{theorem}[事件的运算规律]
由于事件实质上是集合,有
\begin{enumerate}
	\item {\bf 交换律}
	\begin{gather}
		A \cup B = B \cup A, \\
		A B = B A;
	\end{gather}

	\item {\bf 结合律}
	\begin{gather}
		A \cup (B \cup C) = (A \cup B) \cup C, \\
		A \cap (B \cap C) = (A \cap B) \cap C;
	\end{gather}

	\item {\bf 分配律}
	\begin{gather}
		A \cup (B \cap C) = (A \cup B) \cap (A \cup C), \\
		A \cap (B \cup C) = (A \cap B) \cup (A \cap C);
	\end{gather}

	\item {\bf 对偶律}
	\begin{gather}
		\overline{A \cup B} = \overline{A}\ \overline{B}, \\
		\overline{AB} = \overline{A} \cup \overline{B}, \\
		\overline{\bigcup_i A_i} = \bigcap_i \overline{A_i}, \\
		\overline{\bigcap_i A_i} = \bigcup_i \overline{A_i}.
	\end{gather}
\end{enumerate}
\end{theorem}

\section{事件发生的概率}

\subsection{频率的概念及性质}
对于一个事件,除去必然事件与不可能事件外,它在一次试验中有可能发生,也有可能不发生.
为了揭示这些事件内在的统计规律性,往往需要知道这些事件在一次试验中发生的可能性的大小,以便更好地认识客观事物.
比如医学工作者在研制一种新药的过程中,需要做临床试验测试其是否有效,可否投入临床使用.

为了刻画事件在一次试验中发生的可能性,我们首先引入频率的概念.

\begin{definition}
在\(n\)次重复试验中,若事件\(A\)发生了\(k\)次,则称\(k\)为事件\(A\)发生的频数,称\(\frac{k}{n}\)为事件\(A\)发生的频率,记作\(f_n(A)\),即\[
f_n(A) = \frac{k}{n}.
\]
\end{definition}

\begin{property}
由定义可知,频率有如下性质:\begin{enumerate}
\item \(0 \leq f_n(A) \leq 1\);
\item \(f_n(\Omega) = 1\),\(f_n(\emptyset) = 0\);
\item 若\(A_1,A_2,\dotsc,A_r\)为\(r\)个两两互斥的事件,则\[
f_n\left( \bigcup_{i=1}^r A_i \right)
= \sum\limits_{i=1}^r f_n(A_i).
\]
\end{enumerate}
\end{property}

由于事件\(A\)在\(n\)次试验中发生的频率是\(A\)发生的频数与试验次数\(n\)之比,
频率大小表示了\(A\)发生的频繁程度.
频率越大,事件\(A\)在\(n\)次试验中发生得越频繁,
就意味着\(A\)在一次试验中发生的可能性越大.
因此,频率在一定程度上可以反映事件发生可能性的大小.

但是,另一方面频率具有不客观性.我们来看下面列出的数据:
\begin{example}
历史上,许多著名的统计学家做过“抛硬币”试验,得到如下数据:
\begin{center}
	\begin{tabular}{c|c|c|c}
	\hline
	试验者 & 抛硬币次数\(n\) & 正面朝上次数\(n_A\) & 频率\(f_n(A)\) \\ \hline
	Buffon & 4 040 & 2 048 & 0.506 9 \\
	Fisher & 10 000 & 4 979 & 0.497 9 \\
	Pearson & 12 000 & 6 019 & 0.501 6 \\
	Pearson & 24 000 & 12 012 & 0.500 5 \\ \hline
	\end{tabular}
\end{center}
\end{example}
可以看出,频率具有波动性.当试验次数\(n\)不同时,
频率不相同(事实上,即便试验次数\(n\)相同,不同的实验者得到的频率也未必相同).
进一步仔细观察这两组数据可以发现,当\(n\)较小时,频率波动较大;
而当\(n\)较大时,频率波动越来越小,且频率总稳定在一个客观数量附近,
例如“抛硬币”试验中频率的稳定值是\(0.5\).

\subsection{概率的公理化定义及性质}
在实践中,我们通常不可能,也无必要对每个事件做大量的试验来获取频率的稳定值.
历史上,数学家是在不同的概率模型下给出事件概率的计算公式,再抽象地公理化地定义概率.

\begin{definition}
设\(\Omega\)为一个试验的样本空间.
对\(\Omega\)中任意一个事件\(A\),
对应一个实数\(P(A)\).
若这个集合函数\(P\)满足以下三个条件,
则称“\(P(A)\)是事件\(A\)发生的\DefineConcept{概率}(probability)”:
\begin{enumerate}
	\item 非负性:
	\begin{equation}
	P(A) \geq 0;
	\end{equation}

	\item 规范性:
	\begin{equation}
	P(\Omega) = 1;
	\end{equation}

	\item 可列可加性:
	若\(A_1,A_2,\dotsc,A_n,\dotsc\)可列个两两互斥的事件,
	则\begin{equation}
		P\left(\bigcup_{i=1}^\infty A_i\right)
		= \sum\limits_{i=1}^\infty P(A_i).
	\end{equation}
\end{enumerate}
\end{definition}
这个概率的公理化定义是苏联科学家柯尔莫哥洛夫在1933年给出的.

由概率的定义,可得概率有如下性质:
\begin{property}
\begin{equation}
P(\emptyset) = 0.
\end{equation}
\end{property}

\begin{property}[有限可加性]
若\(n\)个事件\(A_1,A_2,\dotsc,A_n\)两两互斥,则\begin{equation}
P\left(\bigcup_{i=1}^n A_i\right)
= \sum\limits_{i=1}^n P(A_i).
\end{equation}
\end{property}

\begin{property}
\begin{equation}
P(\overline{A}) = 1 - P(A).
\end{equation}
\end{property}

\begin{property}[概率的减法]
\begin{equation}
P(A - B) = P(A) - P(AB).
\end{equation}

特别地,若\(B \subseteq A\),有
\begin{equation}
P(A - B) = P(A) - P(B),
\end{equation}
且
\begin{equation}
P(A) \geq P(B).
\end{equation}

\begin{proof}
由事件\(A\)满足:\[
A = A \Omega
= A(B+\overline{B})
= AB+A\overline{B},
\]故\[
P(A) = P(AB)+P(A\overline{B}),
\]进而有\[
P(A-B) = P(A\overline{B}) = P(A) - P(AB).
\]

当\(B \subseteq A\)时,
\(B = BB \subseteq AB\);
又由\(AB \subseteq B\),
所以\(AB = B\),
进而有\[
P(A-B) = P(A) - P(B).
\qedhere
\]
\end{proof}
\end{property}

\begin{property}
对任意事件\(A\),有\begin{equation}
P(A) \leq 1.
\end{equation}
\end{property}

\begin{theorem}[概率的加法]
对任意两事件\(A,B\),
有\begin{equation}
	P(A \cup B) = P(A) + P(B) - P(AB).
\end{equation}
\end{theorem}

\begin{corollary}
对任意三事件\(A,B,C\),
有\begin{equation}
	P(A \cup B \cup C)
	= P(A) + P(B) + P(C)
	- P(AB) - P(AC) - P(BC)
	+ P(ABC).
\end{equation}
\end{corollary}

\begin{theorem}
对任意两事件\(A,B\),
有\begin{equation}
	P(A \cup B) \leq P(A) + P(B).
\end{equation}
\end{theorem}

\begin{corollary}[布尔不等式]
对任意多个事件\(A_i\ (i=1,2,\dotsc)\),
不等式\begin{equation}\label{equation:概率论基础.布尔不等式}
	P\left(\bigcup_i A_i\right)
	\leq
	\sum_i P(A_i)
\end{equation}
成立,当且仅当“\(A_1,A_2,\dotsc\)两两互斥”时取“\(=\)”.
\end{corollary}

\section{等可能概型}
在实际问题中,具体找出符合概率公理化定义的集合函数\(P\),
再计算出事件\(A\)的概率\(P(A)\)通常是不容易的.
但在等可能概型下,\(P(A)\)的计算却十分简单.

等可能概型是指一个试验中所有的样本点都等可能出现的概率模型.
\subsection{古典概型}
\begin{definition}
若一个随机试验具有以下两个特点:
\begin{enumerate}
\item 试验只有有限个可能结果,即\[
\Omega = \{\omega_1, \omega_2, \dotsc, \omega_n\};
\]

\item 每个可能结果在试验中出现的可能性相等,即\[
P\{\omega_1\} = P\{\omega_2\} = \dotsb = P\{\omega_n\},
\]
\end{enumerate}
这样的随机试验的概率模型称为\DefineConcept{古典概率模型},简称\DefineConcept{古典概型}.
\end{definition}

因为\(\Omega = \{\omega_1\}\cup\{\omega_2\}\cup\dotsb\cup\{\omega_n\}\),且基本事件是两两互斥的,从而有\[
1 = P(\Omega) = P\{\omega_1, \omega_2, \dotsc, \omega_n\}
= \sum\limits_{i=1}^n P\{\omega_i\}
= n P\{\omega_1\},
\]得到\[
P\{\omega_1\} = P\{\omega_2\} = \dotsb = P\{\omega_n\} = \frac{1}{n}.
\]

对任一事件\(A\),为不失一般性,\(A\)总可表为\[
A = \{\omega_{i_1},\omega_{i_2},\dotsc,\omega_{i_k}\}
= \{\omega_{i_1}\}\cup\{\omega_{i_2}\}\cup\dotsb\cup\{\omega_{i_k}\},
\]于是有\[
P(A) = P\{\omega_{i_1}\} + P\{\omega_{i_2}\} + \dotsb + P\{\omega_{i_k}\}
= \frac{k}{n} = \frac{A \text{中的样本点总数}}{\Omega \text{中的样本点总数}}.
\]

这样的概率称为\DefineConcept{古典概率}.

计算古典概率时,应选取适当的随机试验以及样本空间,使其符合古典概率的两个特点.
比如掷一均匀硬币两次,考察出现的面(记正面为T,反面为H),
样本空间为\(\Omega_1 = \{ HH, HT, TH, TT \}\),四个基本事件出现的概率都是\(1/4\).
但若考察正面出现的次数,则样本空间为\(\Omega_2 = \{ 0,1,2 \}\),这便不是古典概型,
因为出现“0次正面”相当于第一个试验出现“TT”,其概率为\(1/4\);
而出现“1次正面”相当于第一个试验出现“HT”或“TH”,其概率为\(1/2\).

\begin{example}
设一个袋中有\(N\)个编号不同的小球.
从袋中\DefineConcept{有放回地}抽取\(r\)次,
每次一球,这时样本点总数为\(N^r\).
\end{example}

\begin{example}
设一个袋中有\(N\)个编号不同的小球.
从袋中\DefineConcept{不放回地}抽取\(r\)次,
每次一球,这时样本点总数为\[
A_N^r = N(N-1)\dotsb(N-r+1) = \frac{N!}{(N-r)!},
\quad
r \leq N.
\]
\end{example}

\begin{example}
设一个袋中有\(N\)个球,其中\(m\)个红球,余下是白球.
从袋中一次取\(n\)个球.这样抽取到的\(n\)个球是无序的,总的抽取结果有\(C_N^n\)种,
而取出的\(n\)个球中恰有\(k\ (k=0,1,\dotsc,m)\)个红球的样本点总数为\(C_m^k C_{N-m}^{n-k}\)种.
从而取出\(n\)个球中恰有\(k\)个红球的概率为\[
p_k = \frac{C_m^k C_{N-m}^{n-k}}{C_N^n},
\quad k=0,1,\dotsc,m.
\]
这个概率\(p_k\)称为\DefineConcept{超几何概率}.
\end{example}

\begin{example}[投球问题]\label{example:概率论基础.古典概型.投球问题}
有\(n\)个不同的球,将它们投入到\(N\ (n \leq N)\)个箱子内.
假设任意一个球被投入任意一个箱子的概率是\(\frac{1}{N}\),
且任意一个箱子可以容纳全部\(n\)个球.
求以下事件的概率:
\begin{enumerate}
	\item 设\(A\)表示“指定\(n\)个箱子,每个箱子里各有一球”;
	\item 设\(B\)表示“恰有\(n\)个箱子,其中各有一球”;
	\item 设\(C\)表示“指定某个箱子,其中恰有\(m\ (m \leq n)\)个球”;
	\item 设\(D\)表示“恰有\(k\)个箱子,其中有\(m\)个球”.
\end{enumerate}
\begin{solution}
由于每一个球有\(N\)种可能投法,所以样本点总数为\(N^n\).
\begin{enumerate}
\item
指定的\(n\)的箱子中各有一个球,即\(n\)个球分配在\(N\)箱子中的不同排列数共有\(n!\)种,所以\[
P(A) = \frac{n!}{N^n}.
\]

\item
由于未确定是哪几个箱子,而从\(N\)个箱子中选出\(n\)个箱子的方法有\(C_N^n\)种,
对于按这种方法选定的\(n\)个箱子,\(n\)个球投入其中且每个箱子各有一个球的投球方式共有\(n!\)种,
所以事件\(B\)所包含的样本点数为\(C_N^n n!\),从而\[
P(B) = \frac{C_N^n n!}{N^n} = \frac{A_N^n}{N^n}.
\]

\item
这个指定的箱子中有\(m\)个球须从\(n\)个球中选出,共有\(C_n^m\)种选法;
其余\(n-m\)个球可以任意投到其余的\(N-1\)个箱子中,共有\((N-1)^{n-m}\)种投法,
所以事件\(C\)所包含的样本点数为\(C_n^m (N-1)^{n-m}\),从而\[
	P(C) = \frac{C_n^m (N-1)^{n-m}}{N^n}
	= C_n^m \left(\frac{1}{N}\right)^m \left(1-\frac{1}{N}\right)^{n-m}.
\]

\item
“恰有\(k\)个箱子”是从\(N\)个箱子中任意选取的,有\(C_N^k\)种选法;
而\(m\)个球是从\(n\)个球中任意选出的,有\(C_n^m\)种选法;
由于选出的\(m\)个球中任意一个球都可投入到刚刚选出的\(k\)个箱子中的任意一个箱子中去,
所以事件\(D\)包含的样本点数为\(C_N^k C_n^m k^m\),从而\[
	P(D) = \frac{C_N^k C_n^m k^m}{N^n}.
\]
\end{enumerate}
\end{solution}
\end{example}

\begin{example}\label{example:概率论基础.古典概型.座位问题}
假设有\(n\)个人随机地坐在礼堂第1排的\(N\)个座位上,试求下列事件的概率:
\begin{enumerate}
	\item \(P(A)=\{\text{任何人都没有邻座}\}\);
	\item \(P(B)=\{\text{每人恰好有一个邻座}\}\);
	\item \(P(C)=\{\text{排在中央对称的两个座位至少有一个空着}\}\).
\end{enumerate}
{\small\it
这里要注意发现本例中对“座位”的描述和在\cref{example:概率论基础.古典概型.投球问题} 中对“箱子”的描述的差异.
投球问题并没有强调\(N\)个“箱子”有区别;
而在本例中,“座位”是依顺序排列摆放的、有区别的.
}
\begin{solution}
\(n\)个人随机地坐在礼堂第1排的\(N\)个座位上共有\(A_N^n\)种不同坐法,故样本点总数为\(A_N^n\).
\begin{enumerate}
\item
若任何人都没有邻座,则第1排至少得有\(n+n-1=2n-1\)个座位.
故当\(2n-1>N\)时,即\(n>(N+1)/2\)时,\(P(A)=0\).
当\(n\leq(N+1)/2\)时,要使任何人均无邻座,可以按以下方式安排他们的座位:
先从\(N\)个座位中搬走\(n-1\)个,然后将\(n\)个人随意安排在\(N-(n-1)\)个座位上,
再在每两个人之间插入\(1\)个座位;
从而事件\(A\)包含的样本点数为\(A_{N-n+1}^n\).
因此\[
	P(A) = \left\{ \begin{array}{cl}
		\frac{A_{N-n+1}^n}{A_N^n}, & n\leq\frac{N+1}{2} \\
		0, & n>\frac{N+1}{2}.
	\end{array} \right.
\]

\item
要使每人恰好只有一个邻座,那么\(n\)必须是偶数,且\(N
\geq n + \frac{n}{2} - 1
= \frac{3n}{2} - 1\);
也就是说,当\(N < \frac{3n}{2} - 1\)时,或当\(n\)是奇数时,\(P(B) = 0\).
当\(N \geq \frac{3n}{2} - 1\),且\(n\)是偶数时,可以按以下方式安排他们的座位:
先从\(N\)个座位中搬走\(2\left(\frac{n}{2}-1\right) = n-2\)个座位,
然后从\(n\)个人种随意地选出\(\frac{n}{2}\)个人并随意地安排在\(N-n+2\)个座位上,
从而有\(C_n^{n/2} A_{N-n+2}^{n/2}\)种放法;
继而再将搬走\(n-2\)个座位两个两个地插入每相邻的两人之间,
最后将剩下的\(\frac{n}{2}\)个人随机地安排到已经坐下的人的身边,
共有\((n/2)!\)种放法;
所以,将\(n\)个人随机安排在一排\(N\)个座位上,且要求每个人只有一个邻座,共有\(C_n^{n/2} A_{N-n+2}^{n/2} (n/2)!\)种放法;
从而事件\(B\)包含的样本点数为\([C_n^{n/2} (n/2)!] A_{N-n+2}^{n/2}
= A_n^{n/2} A_{N-n+2}^{n/2}\).
因此\[
P(B) = \left\{ \def\arraystretch{1.5} \begin{array}{cl}
\frac{A_n^{n/2} A_{N-n+2}^{n/2}}{A_N^n},
	& n\ \text{是偶数} \land N\geq\frac{3n}{2}-1, \\
0, & n\ \text{是奇数} \lor N<\frac{3n}{2}-1.
\end{array} \right.
\]

\item
当\(N\)为偶数时,将这一排座位看作一条线段,沿着它的垂直平分线将它对折重叠起来,
然后在其中一侧的\(\frac{N}{2}\)个座位上随机安排\(n\)个人(假设一侧的座位足够\(n\)个人坐下,即\(\frac{N}{2} \geq n\)),共有\(A_{N/2}^n\)种方法.
而每人又可视为在重叠起来的线段的上面一个或下面一个座位上,
故对每个人的安排都有两种方式,于是\(n\)个人就有\(2^n\)种不同方式.
所以当\(N\)为偶数时,事件\(C\)包含的样本点数为\(2^n A_{N/2}^n\).

当\(N\)为奇数时,还是将这一排座位折叠起来,特别地将位于这一排正中央的座位(折叠后变成半个座位了)视作一个完整的座位,然后在垂直平分线一侧的座位(包括正中央的座位)上,即在\(1+\frac{N-1}{2}\)个座位上,随机地安排\(n\)个人.
当中央位置不安排人时,共有\(2^n A_{(N-1)/2}^n\)种放法;
当中央位置安排指定的一个人时,共有\(n 2^{n-1} A_{(N-1)/2}^{n-1}\)种放法.
所以当\(N\)为奇数时,事件\(C\)包含的样本点数为\(2^n A_{(N-1)/2}^n + n 2^{n-1} A_{(N-1)/2}^{n-1}\).
因此\[
P(C) = \left\{ \def\arraystretch{1.5} \begin{array}{cl}
\frac{2^n A_{N/2}^n}{A_N^n},
	& N\ \text{是偶数}, \\
\frac{2^n A_{(N-1)/2}^n + n 2^{n-1} A_{(N-1)/2}^{n-1}}{A_N^n},
	& N\ \text{是奇数}.
\end{array} \right.
\]
\end{enumerate}
\end{solution}
\end{example}

\begin{example}[抽签问题]
袋中有\(m\)个白球,\(n\)个黑球,现从中不放回地依次取球,求第\(k\)次(\(1 \leq k \leq m+n\))取出的球是白球的概率.
\begin{solution}
设想将第\(i\)次取出的球放入第\(i\)号(\(i=1,2,\dotsc,m+n\))格子,则所求概率是第\(k\)号格子放白球的概率.所有\(m+n\)个格子放球的方式有\((m+n)!\)种,而当第\(k\)号格子放白球时,不妨先放第\(k\)号格子,再放余下\(m+n-1\)个格子,即共有放球方式\(m(m+n-1)!\)种,于是所求概率为\[
p = \frac{m (m+n-1)!}{(m+n)!} = \frac{m}{m+n}.
\]
\end{solution}
这个例子表明抽签的结果与抽签的顺序无关.
\end{example}

\begin{example}
30只元件中有27只一等品,3只二等品.随即将这30只元件均分装入三盒,求:\begin{enumerate}
\item 每盒有一只二等品的概率;
\item 有一盒有三只二等品的概率.
\end{enumerate}
\begin{solution}
30只元件平均分到三盒的总分法有\(C_{30}^{10} C_{20}^{10} C_{10}^{10} = \frac{30!}{10! 10! 10!}\)种.
\begin{enumerate}
\item 三只二等品均分到三个盒子的分法有\(3!\)种,再将27只一等品均分到三个盒子的分法有\(C_{27}^9 C_{18}^9 C_9^9 = \frac{27!}{9! 9! 9!}\)种,则\[
P(\text{每盒有一只二等品})=\frac{3! \cdot 27!}{9! 9! 9!} \bigg/ \frac{30!}{10! 10! 10!} \approx 0.2463.
\]
\item 指定一个盒子并将三只二等品装入这个盒子的指定方式有\(3\)种,其余27只一等品的分法(二等品所在盒7只,另外两盒各10只)有\(C_{27}^7 C_{20}^{10} C_{10}^{10} = \frac{27!}{7! 10! 10!}\)种,则\[
P(\text{有一盒有三只二等品}) = \frac{3 \cdot 27!}{7! 10! 10!} \bigg/ \frac{30!}{10! 10! 10!} \approx 0.0887.
\]
\end{enumerate}
\end{solution}
在本例中,我们反复利用以下结果:

将\(n\)个球分成\(k\)组,第\(i\)组恰有\(n_i\)个球(\(i=1,2,\dotsc,k\))且\(n_1+n_2+\dotsb+n_k=n\),则所有的分法总数为\begin{equation}
C_n^{n_1} C_{n-n_1}^{n_2} \dotsm C_{n-(n_1+n_2+\dotsb+n_{k-1})}^{n_k}
= \frac{n!}{n_1! n_2! \dotsm n_k!}.
\end{equation}
\end{example}

\subsection{几何概型}
\begin{definition}
一个随机试验,若其所有可能结果“等可能”地出现在一个有界的欧式区域\(\Omega\)内,
则称这个试验的概率模型为\DefineConcept{几何概型}.
\end{definition}
这时所有可能结果构成一个无限集,从而不能用计数的方法计算事件的概率.

\begin{definition}
如果把\(\Omega\)作为一般的欧式区域,\(m(A)\)作为\(A\)的度量(一维为长度,二维为面积,三维为体积……),
就得到几何概型下,一般事件\(A\)的概率计算公式:\[
P(A) = \frac{m(A)}{m(\Omega)},
\]称这个概率为\DefineConcept{几何概率}.
\end{definition}
由几何概率的定义,若事件\(B\)的度量为0,则\(B\)的几何概率\(P(B)=0\).但\(B\)不一定是不可能事件.

\begin{example}
随机地在单位圆内掷一点\(M\),求点\(M\)到原点距离小于\(1/4\)的概率.
\begin{solution}
因为点\(M\)“等可能”地出现在单位圆内,于是\(M\)出现在单位圆内任何一个面积相等的小区域\(A\)内因概率相等而与\(A\)的形态与位置无关.而且\(A\)的概率\(P(A)\)因与\(A\)的面积\(m(A)\)成正比,即\(P(A)=\lambda m(A)\).但是\(1 = P(\Omega) = \lambda m(\Omega)\),从而有\(\lambda = 1/m(\Omega)\),于是得到\(P(A) = m(A)/m(\Omega)\).这样,\(M\)到原点距离小于\(1/4\)的概率为\[
p = \frac{\pi (1/4)^2}{\pi \times 1^2} = \frac{1}{16}.
\]
\end{solution}
\end{example}

\begin{example}[会面问题]
甲、乙两人约定在早上7点到8点之间在某处会面,并约定先到者应等候另一人15min,过时即离去,求两人能会面的概率.
\begin{solution}
以\(x\)和\(y\)分别表示甲、乙两人到达约会地点的时间(从7点开始计时,单位:min),则两人所有可能到达时间为\[
\Omega = \Set{ (x,y) \given 0 \leq x \leq 60, 0 \leq y \leq 60 }.
\]

设事件\(A\)表示两人能会面,则\[
A = \Set{ (x,y)\in\Omega \given \abs{x-y} \leq 15 }.
\]

那么\[
P(A) = \frac{m(A)}{m(\Omega)} = \frac{60^2 - 45^2}{60^2} = \frac{7}{16}.
\]
\end{solution}
\end{example}

\begin{example}[布冯投针问题]
%@see: https://mathworld.wolfram.com/BuffonsNeedleProblem.html
设平面上有等距离的平行线.平行线的距离为\(a\).
向平面任意投掷一枚长为\(l\ (l<a)\)的针,
试求针与平行线相交的概率.
\begin{solution}
以\(x\)表示针的中点与最近一条平行线的距离,
又以\(\phi\)表示针与直线间的夹角,
则有\[
	\Omega = \Set{ (\phi,x) \given 0 \leq \phi \leq \pi, 0 \leq x \leq a/2 }.
\]
令\(A\)为“针与平行线相交”,
则\[
	A = \Set*{ (\phi,x)\in\Omega \given x \leq \frac{l}{2} \sin\phi },
\]
从而\[
	m(A) = \int_0^{\pi}\frac{l}{2} \sin\phi \dd{\phi},
\]\[
	m(\Omega) = \pi \cdot \frac{a}{2},
\]
于是\[
	P(A)
	= \frac{m(A)}{m(\Omega)}
	= \frac{2l}{\pi a}.
\]
\end{solution}
\end{example}

\section{条件概率}

\subsection{条件概率的概念}
\begin{definition}
设\(A\)、\(B\)是同一试验的两事件,
且\(P(B) > 0\),
我们把\[
	\frac{P(AB)}{P(B)}
\]称为“在事件\(B\)发生的条件下,事件\(A\)发生的概率”,
或简称“\(A\)对\(B\)的\DefineConcept{条件概率}”,
记作\(P(A \vert B)\),
即\begin{equation}
	P(A \vert B)
	\defeq
	\frac{P(AB)}{P(B)}.
\end{equation}
\end{definition}

\begin{property}
条件概率有如下性质:\begin{enumerate}
	\item \(P(A \vert B) \geq 0\);
	\item \(P(\Omega \vert B) = 1\);
	\item 若事件\(A_1,A_2,\dotsc,A_n,\dotsc\)两两互斥,则\[
		P\left(\bigcup_{i=1}^\infty {A_i} \Bigg\vert B\right)
		= \sum\limits_{i=1}^\infty {P(A_i \vert B)}.
	\]
\end{enumerate}
\begin{proof}
性质3推导如下:
当\(A_1,A_2,\dotsc,A_n,\dotsc\)两两互斥的,
则\(A_1 B,A_2 B,\dotsc,A_n B,\dotsc\)也两两互斥.
从而由条件概率定义及概率的可列可加性,
有\begin{align*}
	P\left(\bigcup_{i=1}^\infty {A_i} \Bigg\vert B\right)
	&= \frac{1}{P(B)} P\left[\left(\displaystyle\bigcup_{i=1}^\infty {A_i}\right) B\right]
	= \frac{1}{P(B)} P\left(\displaystyle\bigcup_{i=1}^\infty {A_i B}\right) \\
	&= \sum\limits_{i=1}^\infty {\frac{P(A_i B)}{P(B)}}
	= \sum\limits_{i=1}^\infty {P(A_i \vert B)}.
	\qedhere
\end{align*}
\end{proof}
\end{property}

当事件\(B\)给定时,\(P(A \vert B)\)是事件\(A\)的集合函数.而上述三个性质正好是概率定义中三个公理化条件,于是条件概率也是概率.这样,条件概率满足概率的一切性质,如:

\begin{property}
\(P(\overline{A} \vert B) = 1 - P(A \vert B)\).
\end{property}

\begin{property}
\(P(A \cup B \vert C) = P(A \vert C) + P(B \vert C) - P(AB \vert C)\).
\end{property}

\begin{example}
历史资料表明,某地区从某次特大洪水发生以后在30年内发生特大洪水的概率为80\%,在40年内发生特大洪水的概念为85\%.问现在已30年无特大洪水的该地区,在未来10年内将发生特大洪水的概率是多少?
\begin{solution}
设\(A\)表示“该地区从某次特大洪水后30年内无特大洪水”,\(B\)表示“该地区从某次特大洪水后40年内无特大洪水”,则\(P(A) = 0.2\),\(P(B) = 0.15\).因为\(B \subseteq A\)使得\(AB = B\),那么所求概率为\[
P(\overline{B} \vert A)
= 1 - P(B \vert A)
= 1 - \frac{P(AB)}{P(A)}
= 1 - \frac{P(B)}{P(A)}
= 1 - \frac{0.15}{0.2}
= 0.25.
\]
\end{solution}
\end{example}

\subsection{乘法公式}
\begin{theorem}[概率的乘法]
当\(P(A) > 0\),\(P(B) > 0\),由条件概率的定义有\begin{equation}
P(AB) = P(A) P(B \vert A) = P(B) P(A \vert B),
\end{equation}上式称为\DefineConcept{乘法公式}.

一般地,设\(A_1,A_2,\dotsc,A_n\)是\(n\)个事件,且\(P(A_1 A_2 \dotsm A_{n-1}) > 0\),则有乘法公式的一般形式为\begin{equation}
P(A_1 A_2 \dotsm A_n)
= P(A_1) P(A_2 \vert A_1) P(A_3 \vert A_1 A_2) \dotsm P(A_n \vert A_1 A_2 \dotsm A_{n-1}).
\end{equation}
\end{theorem}

\begin{example}
一个小组有10名同学,其中4名女同学.
每周依次有一名同学收作业.求第一、二周是女同学,第三、四周是男同学收作业的概率.
\begin{solution}
设\(A_i = \Set{\text{第\(i\)周由女同学收作业}}\ (i=1,2,3,4)\),
则有\begin{align*}
	P(A_1 A_2 \overline{A}_3 \overline{A}_4)
	&= P(A_1) P(A_2 \vert A_1)
	P(\overline{A}_3 \vert A_1 A_2) P(\overline{A}_4 \vert A_1 A_2 \overline{A}_3) \\
	&= \frac{4}{10} \times \frac{3}{9} \times \frac{6}{8} \times \frac{5}{7} = \frac{1}{14}.
\end{align*}
\end{solution}
\end{example}

\subsection{全概率公式与贝叶斯公式}
\begin{theorem}
设\(A_1,A_2,\dotsc,A_n\)为样本空间\(\Omega\)的一个完备事件组,且\[
P(A_i) > 0 \quad(i=1,2,\dotsc,n).
\]设\(B\)是任一事件,则:
\begin{enumerate}
\item {\bf 全概率公式}:\begin{equation}
P(B) = \sum\limits_{i=1}^n P(A_i) P(B \vert A_i);
\end{equation}
\item {\bf 贝叶斯公式}:若\(P(B) > 0\),还有\begin{equation}
P(A_i \vert B) = \frac{P(A_i) P(B \vert A_i)}{P(B)}
= \frac{P(A_i) P(B \vert A_i)}{\sum\limits_{j=1}^n P(A_j) P(B \vert A_j)},
\quad i = 1,2,\dotsc,n.
\end{equation}
\end{enumerate}
\begin{proof}
全概率公式和贝叶斯公式的成立是显然的.因为\(B = B \Omega = B(A_1 \cup A_2 \cup \dotsb \cup A_n) = A_1 B \cup A_2 B \cup \dotsb A_n B\),所以\[
P(B) = P\left(\bigcup\limits_{i=1}^n A_i B\right)
= \sum\limits_{i=1}^n P(A_i B)
= \sum\limits_{i=1}^n P(A_i) P(B \vert A_i).
\]

又因为\(P(A_i \vert B) = \frac{P(A_i B)}{P(B)}\),而\(P(B \vert A_i) = \frac{P(A_i B)}{P(A_i)}\),所以\[
P(A_i \vert B) = \frac{P(A_i) P(B \vert A_i)}{P(B)}.
\qedhere
\]
\end{proof}
\end{theorem}

构成完备事件组的每个事件\(A_i\)的发生,都有可能引起事件\(B\)的发生,
故可视\(A_i\)为引起事件\(B\)发生的“原因事件”,\(B\)视为“结果事件”.
只要知道各“原因事件”发生的概率,且知道“原因事件”\(A_i\)发生后引起“结果事件”\(B\)发生的条件概率,
则可求出\(B\)发生的概率.

如果我们\(P(A_i)\)称为\(A_i\)的\DefineConcept{先验概率},\(P(A_i \vert B)\)称为\(A_i\)的\DefineConcept{后验概率}.
这就是当我们已知“结果事件”\(B\)发生后,追究是由哪一个“原因事件”引发的概率,从而由后验概率做出贝叶斯决策.

在应用全概率公式与贝叶斯公式时,选择完备事件组较为常见的情形有两种:
\begin{enumerate}
\item 将某过程的第一个步骤的所有情况作为完备事件组;
\item 将某先决事件\(A\)与\(\overline{A}\)作为完备事件组.
\end{enumerate}

\begin{example}
一盒中装有12个球,其中8个是新球.第一次比赛从盒中任取2个球,使用后放入盒中.第二次比赛时,再从盒中任取两球,求\begin{enumerate}
\item 第二次取出2个新球的概率;
\item 已知第二次取出2个新球,而第一次仅取出一个新球的概率.
\end{enumerate}
\begin{solution}
设\(B\)表示“第二次取出2个新球”,\(A_i\)表示“第一次取出2个球中有\(i\)个新球”,\(i=0,1,2\),则\(A_0,A_1,A_2\)是第一次取球的完备事件组,且\[
P(A_0) = \frac{C_4^2}{C_{12}^2} = \frac{1}{11},
\quad
P(A_1) = \frac{C_4^1 C_8^1}{C_{12}^2} = \frac{16}{33},
\quad
P(A_2) = \frac{C_8^2}{C_{12}^2} = \frac{14}{33},
\]\[
P(B \vert A_0) = \frac{C_8^2}{C_{12}^2} = \frac{14}{33},
\quad
P(B \vert A_1) = \frac{C_7^2}{C_{12}^2} = \frac{7}{22},
\quad
P(B \vert A_2) = \frac{C_6^2}{C_{12}^2} = \frac{5}{22}.
\]

由全概率公式,有\begin{align*}
P(B) &= P(A_0) P(B \vert A_0)
	+ P(A_1) P(B \vert A_1)
	+ P(A_2) P(B \vert A_2) \\
&= \frac{1}{11} \times \frac{14}{33}
	+ \frac{16}{33} \times \frac{7}{22}
	+ \frac{14}{33} \times \frac{5}{22}
= 0.289\ 3.
\end{align*}

由贝叶斯公式,有\[
P(A_1 \vert B) = \frac{P(A_1) P(B \vert A_1)}{P(B)}
= \frac{\frac{16}{33} \times \frac{7}{22}}{0.289\ 3} = 0.533\ 3.
\]
\end{solution}
\end{example}

\section{事件的独立性及伯努利概型}
\subsection{事件的独立性}
从上一节可以发现,一般有\(P(A \vert B) \neq P(A)\),即事件\(B\)发生后会影响事件\(A\)发生的概率,但也有例外的情形.

例如一袋中有3个白球7个红球,有放回地取两次球,每次一球.\(A_i\)表示“第\(i\)次取红球”,\(i=1,2\).此时显然有\(P(A_2) = P(A_2 \vert A_1) = \frac{7}{10}\).这表明\(A_1\)是否发生不影响\(A_2\)的发生.

一般地,如果\(P(B) > 0\),且\(P(A \vert B) = P(A)\)时,则事件\(B\)的发生不影响事件\(A\)的发生.若还有\(P(A) > 0\),则必有\(P(B \vert A) = P(B)\).这是因为\(P(A \vert B) = \frac{P(AB)}{P(B)} = P(A)\),则有\(P(AB) = P(A) P(B)\),从而\(P(B \vert A) = \frac{P(AB)}{P(A)} = P(B)\).

可见事件彼此间的影响是相互的.事件\(A\)与\(B\)的某一个发生不影响另一个发火说呢过的情形称为事件\(A\)与\(B\)相互独立,这等价于\(P(AB) = P(A) P(B)\).于是我们定义:

\begin{definition}
对同一试验的任意两事件\(A\)、\(B\),若\[
	P(AB) = P(A) P(B),
\]称“事件\(A\)与\(B\) \DefineConcept{相互独立}”.
\end{definition}

注意,“\(A\)与\(B\)相互独立”和“\(A\)与\(B\)互斥”是两个不同的概念.事实上,只要\(P(A) > 0\),\(P(B) > 0\),“\(A\)与\(B\)相互独立”和“\(A\)与\(B\)互斥”这两种情况绝不会同时出现.这是因为当“\(A\)与\(B\)相互独立”时,\(P(AB) = P(A) P(B) > 0\),这就与“\(A\)与\(B\)互斥”时\(AB = \emptyset\),\(P(AB) = 0\)相矛盾.

设试验的样本空间为\(\Omega\),显然\(P(\Omega) = 1, P(\emptyset) = 0\).
因为对于任意事件\(A\),\(A \Omega = A\),\(A \emptyset = \emptyset\),那么\[
P(A \Omega) = P(A) = P(A) P(\Omega), \qquad
P(A \emptyset) = P(\emptyset) = P(A) P(\emptyset),
\]所以\(A\)与\(\Omega\)、\(A\)与\(\emptyset\)都相互独立.

\begin{theorem}
若事件\(A\)与\(B\)相互独立,则\(A\)与\(\overline{B}\),\(\overline{A}\)与\(B\),\(\overline{A}\)与\(\overline{B}\)也相互独立.
\begin{proof}
因为\(A\overline{B}=A(\Omega-B)=A-AB\),\(P(AB)=P(A)P(B)\),所以\[
P(A\overline{B})
= P(A) - P(AB)
= P(A) - P(A) P(B)
= P(A) [1 - P(B)]
= P(A) P(\overline{B}),
\]从而\(A\)与\(\overline{B}\)相互独立.
根据对称性易证\(\overline{A}\)与\(B\),\(\overline{A}\)与\(\overline{B}\)也都相互独立.
\end{proof}
\end{theorem}

相互独立的概念可推广到多个事件的情形.
\begin{definition}
设\(A_1,A_2,\dotsc,A_n\ (n \geq 2)\)是\(n\)个事件,
若任取两事件\(A_i,A_j\ (i \neq j)\),有\[
P(A_i A_j) = P(A_i) P(A_j),
\]则称“这\(n\)个事件\DefineConcept{两两独立}”.
\end{definition}

\begin{definition}
设\(A_1,A_2,\dotsc,A_n\ (n \geq 2)\)是\(n\)个事件,
若对其中任意\(k\ (2 \leq k \leq n)\)个事件\(A_{i_1},A_{i_2},\dotsc,A_{i_k}\)有\[
P(A_{i_1} A_{i_2} \dotsm A_{i_k})
= P(A_{i_1}) P(A_{i_2}) \dotsm P(A_{i_k}),
\]则称“这\(n\)个事件\DefineConcept{相互独立}”.
\end{definition}

\begin{theorem}
如果\(n\)个事件相互独立,则必有这\(n\)个事件两两独立;反之不然.
\end{theorem}

由定义判断\(n\)个事件的相互独立性,需要验证\(C_n^2+C_n^3+\dotsb+C_n^n=2^n-n-1\)个等式.
因此,在应用中,通常由实际意义判断事件的相互独立性.

概率论中,通常把概率小于0.05的事件叫做\DefineConcept{小概率事件}.
小概率事件有两个特点:
第一,小概率事件在一次试验中几乎不可能发生;
第二,小概率事件在大量重复试验中几乎必定会至少发生一次.

\begin{example}
设\(0<P(B)<1\).
证明:\(A\)与\(B\)相互独立的充要条件为\(P(A \vert B) = P(A \vert \overline{B})\).
\begin{proof}
必要性.
当\(A\)与\(B\)独立时,有\(P(A \vert B) = P(A)\).
而此时,\(A\)与\(\overline{B}\)也独立,有\(P(A \vert \overline{B}) = P(A)\).
因此\(P(A \vert B) = P(A \vert \overline{B})\).

充分性.
若\(P(A \vert B) = P(A \vert \overline{B})\),有\[
\frac{P(AB)}{P(B)}
= \frac{P(A\overline{B})}{P(\overline{B})}
= \frac{P(A)-P(AB)}{1-P(B)},
\]\[
P(AB) [1-P(B)] = P(B) [P(A)-P(AB)],
\]\[
P(AB) = P(A) P(B),
\]即\(A\)与\(B\)相互独立.
\end{proof}
\end{example}

\subsection{伯努利概型}
\begin{definition}
将随机试验重复进行\(n\)次,若各次试验的结果互不影响,即每次试验各可能结果出现的概率都不依赖其他各次试验的结果,这样的试验称为\(n\)重\DefineConcept{独立试验}.

特别地,\(n\)重\DefineConcept{独立试验}中,若每次试验结果只有两个,即\(A\)与\(\overline{A}\),且\(0 < P(A) < 1\),则这样的试验称为\(n\)重\DefineConcept{伯努利试验},相应的数学模型叫做\DefineConcept{伯努利概型}.
\end{definition}

对于伯努利概型,我们需要计算\(A\)在\(n\)次试验中恰好发生\(k\)次的概率.

\begin{theorem}[二项概率]
在\(n\)重伯努利试验中,设\(A\)在各次试验中发生的概率为\(p = P(A)\ (0 < p < 1)\),则在\(n\)次试验中\(A\)恰好发生\(k\)次的概率为\begin{equation}
P_n(k) = C_n^k p^k (1-p)^{n-k}, \quad k=0,1,\dotsc,n.
\end{equation}
\begin{proof}
设事件\(A_i\)表示“\(A\)在第\(i\)次试验发生”,则\(P(A_i)=p\),\(P(\overline{A_i})=1-p\),\(i=1,2,\dotsc,n\),且各\(A_i\)相互独立.
由此可知,\(A\)在\(n\)次试验中某指定\(k\)次(如前\(k\)次)试验中发生而在其余\(n-k\)次试验中不发生的概率为\[
P(A_1 \dotsm A_k \overline{A_{k+1}} \dotsm \overline{A_n})
= P(A_1) \dotsm P(A_k) P(\overline{A_{k+1}}) \dotsm P(\overline{A_n})
= p^k (1-p)^{n-k}.
\]由于\(A\)在\(n\)次试验中恰好发生\(k\)次共有\(C_n^k\)种指定\(k\)次试验\(A\)发生的方式,且这\(C_n^k\)种指定的方式是两两互斥的,从而有概率的可加性,知\[
P_n(k) = C_n^k p^k (1-p)^{n-k},
\quad k=0,1,\dotsc,n.
\qedhere
\]
\end{proof}
\end{theorem}
由于\(P_n(k)\)正好是二项式\([p+(1-p)]^n\)展开式中的第\(k+1\)项,所以通常称\(P_n(k)\)为\DefineConcept{二项概率}.
同时也可看出,二项概率\(P_n(k)\)满足\begin{equation}
\sum\limits_{k=0}^n P_n(k)
= \sum\limits_{k=0}^n C_n^k p^k (1-p)^{n-k} = 1.
\end{equation}

在\(n\)重独立试验中,若每次试验有\(k\)个结果\(A_1,A_2,\dotsc,A_k\),我们来求\(n\)次试验中\(A_i\)各发生\(r_i\)次概率.

\begin{theorem}[多项概率]
在\(n\)重独立试验中,每次试验可能的结果是\(A_1,A_2,\dotsc,A_k\),且\(0 < p_i = P(A_i) < 1\ (i=1,2,\dotsc,k)\),且\(p_1+p_2+\dotsb+p_k=1\),则\(A_1,A_2,\dotsc,A_k\)在\(n\)次试验中各发生\(r_1,r_2,\dotsc,r_k\)次的概率为\begin{equation}\label{equation:概率论基础.多项概率公式}
\frac{n!}{r_1! r_2! \dotsm r_k!} p_1^{r_1} p_2^{r_2} \dotsm p_k^{r_k},
\end{equation}其中\(r_1+r_2+\dotsb+r_k=n\).
\end{theorem}
\cref{equation:概率论基础.多项概率公式} 叫做\DefineConcept{多项概率公式}.
