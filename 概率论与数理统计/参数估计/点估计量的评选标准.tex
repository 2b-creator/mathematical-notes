\section{点估计量的评选标准}
我们已经知道,任何一个统计量都可以作为一个未知参数的估计量.
对于同一参数的多种估计量,采用哪一种更好呢?这就需要一定的标准来做评价.

\subsection{无偏性标准}
\begin{definition}
%@see: 《概率论与数理统计》(陈鸿建、赵永红、翁洋) P200. 定义8.3
若未知参数\(\theta\)的估计量\(\hat{\theta}=\hat{\theta}(\AutoTuple{X}{n})\)满足\[
	E(\hat{\theta})=\theta,
\]
则称“\(\hat{\theta}\)是参数\(\theta\)的\DefineConcept{无偏估计量}”.
否则称“\(\hat{\theta}\)是参数\(\theta\)的\DefineConcept{有偏估计量}”.

对有偏估计量\(\hat{\theta}\),
我们把\[
	E(\hat{\theta}) - \theta
\]称为“\(\hat{\theta}\)的\DefineConcept{偏差}(bias)”,
记作\(b(\hat{\theta})\).

若样本容量\(n\to\infty\)时,有\(b(\hat{\theta})\to0\),
则称“\(\hat{\theta}\)是\(\theta\)的\DefineConcept{渐进无偏估计量}”.
\end{definition}

\begin{example}
%@see: 《概率论与数理统计》(陈鸿建、赵永红、翁洋) P200. 例8.9
%@see: 《概率论与数理统计》(茆诗松、周纪芗、张日权) P230. 例5.2.2
设总体\(X\)服从任何分布,
且\(E(X)=\mu\),\(E(X^k)=m_k\),\(D(X)=\sigma^2\).
\(\AutoTuple{X}{n}\)是样本.
证明:样本均值\(\overline{X}\)、样本\(k\)阶原点矩\(A_k\)和样本方差\(S^2\)分别是
\(\mu\)、\(m_k\)和\(\sigma^2\)的无偏估计量.
\begin{proof}
因为\(E(A_k)
= E\left(\frac{1}{n} \sum_{i=1}^n X_i^k\right)
= \frac{1}{n} \sum_{i=1}^n E(X_i^k)
= E(X^k)
= m_k\),
所以\(A_k\)是\(m_k\)的无偏估计量.

特别地,
由\cref{equation:统计量.均值-1阶原点矩的关系},
有\(\overline{X} = A_1\),
于是\(E(\overline{X})
= E(A_1)
= m_1
= \mu\),
因此,\(\overline{X}\)是\(\mu\)的无偏估计量.

由\cref{equation:统计量.2阶中心矩-2阶原点矩-均值的关系}
有\(B_2 = A_2 - \overline{X}^2\);
由\cref{equation:统计量.方差-2阶中心矩的关系}
有\(S^2 = \frac{n}{n-1} B_2\);
所以\[
	E(S^2)
	= E\left(\frac{n}{n-1} B_2\right)
	= \frac{n}{n-1} E(A_2-\overline{X}^2)
	= \frac{n}{n-1}[E(A_2)-E(\overline{X}^2)].
\]
由于\(E(A_2)
= E(X^2)
= D(X) + [E(X)]^2
= \sigma^2 + \mu^2\),
所以\[
	E(\overline{X}^2)
	= D(\overline{X}) + [E(\overline{X})]^2
	= \frac{\sigma^2}{n} + \mu^2,
\]
那么有\[
	E(S^2)
	= \frac{n}{n-1} \left(\sigma^2 + \mu^2 - \frac{\sigma^2}{n} - \mu^2\right)
	= \sigma^2.
	\qedhere
\]
\end{proof}
\end{example}

\begin{example}
设总体\(X\)服从任何分布,
且\(E(X)=\mu\in[0,+\infty)\),
\(D(X)=\sigma^2\in(0,+\infty)\).
\(\AutoTuple{X}{n}\)是样本.
证明:样本2阶中心矩\(B_2\)是\(\sigma^2\)的渐进无偏估计量.
\begin{proof}
因为\[
	E(B_2)
	= E\left(\frac{n-1}{n} S^2\right)
	= \frac{n-1}{n} E(S^2)
	= \left(1-\frac{1}{n}\right) \sigma^2
	< \sigma^2,
\]
所以\(B_2\)是\(\sigma^2\)的有偏估计量.
又因为\[
	\lim_{n\to\infty} [E(B_2) - \sigma^2]
	= -\lim_{n\to\infty} \frac{\sigma^2}{n}
	= 0,
\]
所以\(B_2\)是\(\sigma^2\)的渐进无偏估计量.
\end{proof}
\end{example}

虽然我们已经证明了样本方差\(S^2\)是总体方差\(\sigma^2\)的无偏估计,
但是样本标准差\(S\)不是总体标准差\(\sigma\)的无偏估计.
下面的例子说明了这一点.
\begin{example}
%@see: 《概率论与数理统计》(茆诗松、周纪芗、张日权) P231. 例5.2.3
设\(\AutoTuple{X}{n}\)是来自正态总体\(N(\mu,\sigma^2)\)的一个样本,
其中\(\mu\)未知,
证明:样本标准差\(S\)的倍数\(\hat{\sigma} = C_n S\)是总体标准差\(\sigma\)的无偏估计,
其中\[
	C_n = \sqrt{\frac{n-1}2} \cdot \frac{\Gamma\left(\frac{n-1}2\right)}{\Gamma\left(\frac{n}2\right)}.
\]
\begin{proof}
令\[
	Y = \frac{(n-1)S^2}{\sigma^2},
	\eqno(1)
\]
则\(Y \sim \chi^2(n-1)\),
\(Y\)的密度函数为\[
	f_Y(y) = \frac1{2^{\frac{n-1}2} \Gamma\left(\frac{n-1}2\right)} y^{\frac{n-1}2-1} e^{-\frac{y}2},
	\quad y>0,
\]
从而\begin{align*}
	E(\sqrt{Y})
	&= \int_{-\infty}^{+\infty} y^{\frac12} f_Y(y) \dd{y} \\
	&= \frac1{2^{\frac{n-1}2} \Gamma\left(\frac{n-1}2\right)} \int_0^{+\infty} y^{\frac{n}2-1} e^{-\frac{y}2} \dd{y} \\
	&= \frac{2^{\frac{n}2} \Gamma\left(\frac{n}2\right)}
	{2^{\frac{n-1}2} \Gamma\left(\frac{n-1}2\right)}
	= \sqrt2 \cdot \frac{\Gamma\left(\frac{n}2\right)}{\Gamma\left(\frac{n-1}2\right)}.
\end{align*}

又由(1)式有\[
	\sqrt{Y} = \frac{\sqrt{n-1}}{\sigma} S,
\]
于是\[
	E(\sqrt{Y})
	= \frac{\sqrt{n-1}}{\sigma} E(S),
\]
那么\[
	E(S)
	= \frac{\sigma}{\sqrt{n-1}} E(\sqrt{Y})
	= \sqrt{\frac2{n-1}} \cdot \frac{\Gamma\left(\frac{n}2\right)}{\Gamma\left(\frac{n-1}2\right)} \sigma
	= \frac{\sigma}{C_n}.
\]
这就是说\(\hat{\sigma} = C_n S\)是\(\sigma\)的无偏估计.
\end{proof}
\end{example}
\begin{remark}
应该注意到,若令\(n\to\infty\),
则\[
	\lim_{n\to\infty} C_n
	= \lim_{n\to\infty} \sqrt{\frac{n-1}2} \cdot
		\frac{\Gamma\left(\frac{n-1}2\right)}{\Gamma\left(\frac{n}2\right)}
	= \lim_{x\to+\infty} \frac{\sqrt{x} \cdot \Gamma(x)}{\Gamma(x+\frac12)},
\]
于是由\cref{equation:反常积分.伽马函数.极限1}
可知\(\lim_{n\to\infty} C_n = 1\).
这就是说,样本标准差\(S\)是总体标准差\(\sigma\)的渐进无偏估计量.
\end{remark}

\begin{example}
设\(X_1,X_2\)是来自总体\(N(\mu,\sigma^2)\)的简单随机样本,其中\(\sigma\ (\sigma>0)\)是未知参数.
若\(\hat{\sigma} = a\abs{X_1-X_2}\)是\(\sigma\)的无偏估计量,求\(a\)的取值.
\begin{solution}
由正态分布的可加性,有\(X_1-X_2 \sim N(0,2\sigma^2)\).
令\(Y = X_1-X_2\),则\(Y\)的概率密度为\[
f(y) = \frac{1}{\sqrt{2\pi} \cdot \sqrt{2} \sigma} \exp(-\frac{y^2}{2 \cdot 2 \sigma^2}).
\]从而\[
	E(\hat{\sigma}) = E(a \abs{Y})
	= a \int_{-\infty}^{+\infty} \abs{y} f(y) \dd{y}
	= a \frac{2 \sigma}{\sqrt{\pi}}.
\]
因为\(\hat{\sigma} = a\abs{X_1-X_2}\)是\(\sigma\)的无偏估计量,
所以\(E(\hat{\sigma}) = \sigma\),代入上式,解得\(a = \sqrt{\pi}/2\).
\end{solution}
\end{example}

\begin{example}
%@see: 《概率论与数理统计》(茆诗松、周纪芗、张日权) P235. 习题5.2 2.
设总体\(X\)具有数学期望\(\mu\)与方差\(\sigma^2\),
\((X_{11},X_{12},\dotsc,X_{1n})\)与\((X_{21},X_{22},\dotsc,X_{2m})\)
是取自该总体的两个独立样本,
试证\[
	S^2 = \frac1{n+m-2} \left[
		\sum_{i=1}^n (X_{1i}-\overline{X}_1)^2
		+ \sum_{i=1}^m (X_{2i}-\overline{X}_2)^2
	\right]
\]是\(\sigma^2\)的无偏估计,
其中\(\overline{X}_1 = \frac1n \sum_{i=1}^n X_{1i},
\overline{X}_2 = \frac1m \sum_{i=1}^m X_{2i}\).
\begin{proof}
令\[
	S^2_1 = \frac1{n-1} \sum_{i=1}^n (X_{1i}-\overline{X}_1)^2, \qquad
	S^2_2 = \frac1{m-1} \sum_{i=1}^m (X_{2i}-\overline{X}_2)^2,
\]
则\(E(S^2_1)=E(S^2_2)=\sigma^2\)
且\[
	S^2 = \frac1{n+m-2} \left[
		(n-1) S^2_1
		+ (m-1) S^2_2
	\right],
\]
从而\begin{align*}
	E(S^2)
	&= \frac1{n+m-2} \left[
		(n-1) E(S^2_1)
		+ (m-1) E(S^2_2)
	\right] \\
	&= \frac1{n+m-2} (n+m-2) \sigma^2
	= \sigma^2.
\end{align*}
因此\(S^2\)是\(\sigma^2\)的无偏估计量.
\end{proof}
\end{example}

\subsection{有效性标准}
\begin{definition}
%@see: 《概率论与数理统计》(陈鸿建、赵永红、翁洋) P201. 定义8.4
设\(\hat{\theta}_1\)和\(\hat{\theta}_2\)是\(\theta\)的两个无偏估计量.
若\[
	D(\hat{\theta}_1) \leq D(\hat{\theta}_2),
\]
则称“\(\hat{\theta}_1\)比\(\hat{\theta}_2\)更\DefineConcept{有效}”.
\end{definition}

\begin{example}
%@see: 《概率论与数理统计》(陈鸿建、赵永红、翁洋) P202. 例8.10
设总体服从任何分布,\(\mu=E(X)\),\(\sigma^2=D(X)\)是未知参数.
\(\AutoTuple{X}{n}\)是样本,
\(\AutoTuple{a}{n}\)是常数.
证明:
\begin{enumerate}
	\item 若\(\sum_{i=1}^n a_i=1\),
	则\(\sum_{i=1}^n a_i X_i\)是\(\mu\)的无偏估计量;

	\item 在所有可能的\(\sum_{i=1}^n a_i X_i\)中,
	\(\overline{X} = \frac{1}{n} \sum_{i=1}^n X_i\)是最有效的无偏估计量.
\end{enumerate}
%TODO
\end{example}

\subsection{一致性标准}
容易看出,估计量\(\hat{\theta}(\AutoTuple{X}{n})\)与样本容量\(n\)有关,
故可将其看作一个关于\(n\)的函数\(\hat{\theta}(n)\).
对\(\hat{\theta}(n)\)的一个自然的要求是,
当\(n\)充分大时,\(\hat{\theta}(n)\)的取值与\(\theta\)的误差应充分小,
即估计量\(\hat{\theta}(n)\)的取值应稳定在参数\(\theta\)的一个充分小的邻域内.

\begin{definition}
%@see: 《概率论与数理统计》(陈鸿建、赵永红、翁洋) P202. 定义8.5
\(\hat{\theta}(n)\)是\(\theta\)的估计量,若\[
	\hat{\theta}(n) \toP \theta,
\]
则称“\(\hat{\theta}(n)\)是\(\theta\)的\DefineConcept{一致估计量}或\DefineConcept{相合估计量}”.
\end{definition}

\begin{theorem}\label{theorem:参数估计.一致估计量的连续函数也是一致估计量}
%@see: 《概率论与数理统计》(茆诗松、周纪芗、张日权) P235. 定理5.2.3
设\(\AutoTuple{\hat\theta}{k}\)分别是\(\AutoTuple{\theta}{k}\)的一致估计量.
若\(g(\AutoTuple{\theta}{k})\)是\(k\)个参数的连续函数,
则\(g(\AutoTuple{\hat\theta}{k})\)是\(g(\AutoTuple{\theta}{k})\)的一致估计量.
%TODO proof
\end{theorem}

\begin{corollary}
%@see: 《概率论与数理统计》(茆诗松、周纪芗、张日权) P235. 例5.2.6
矩估计量都是一致估计量.
\begin{proof}
首先,
由\hyperref[theorem:极限定理.大数律.辛钦大数律]{辛钦大数律},
可知当总体\(X\)有\(D(X^k)\)存在时,
样本\(k\)阶原点矩\(A_k = \frac1n \sum_{i=1}^k X_i^k\)
依概率收敛于总体\(k\)阶原点矩\(E(X^k) = \mu_k\),
即\(A_k \toP m_k\).
这就是说,样本\(k\)阶原点矩是总体\(k\)阶原点矩的一致估计量.
特别地,样本均值\(\overline{X}\)是总体数学期望\(\mu = E(X)\)的一致估计量.

此外,当样本方差\(S^2\)的方差\(D(S^2) \to 0\ (n\to\infty)\)时,
由\cref{theorem:极限定理.大数律.随机变量序列依概率收敛的充分条件}
有\(S^2 - E(S^2) = S^2 - \sigma^2 \toP 0\),
从而样本方差\(S^2\)是总体方差\(\sigma^2\)的一致估计量.

其次,由于总体\(k\)阶中心矩\(\nu_k\)
可以表示为总体的前\(k\)阶原点矩的连续函数:\[
	\nu_k = g(\AutoTuple{\mu}{k}),
\]
所以由\cref{theorem:参数估计.一致估计量的连续函数也是一致估计量}
可知\(g(\AutoTuple{A}{k})\)是\(g(\AutoTuple{\mu}{k})\)的一致估计量.
\end{proof}
\end{corollary}

\subsection{均方误差标准}
\begin{definition}
%@see: 《概率论与数理统计》(陈鸿建、赵永红、翁洋) P203. 定义8.6
对于总体\(X\)的未知参数\(\theta\),\(\hat{\theta}\)是\(\theta\)的估计量.
把\[
	M(\hat{\theta}) = E(\hat{\theta} - \theta)^2
\]称为“\(\hat{\theta}\)关于\(\theta\)的\DefineConcept{均方误差}”.
\end{definition}

\begin{definition}
%@see: 《概率论与数理统计》(陈鸿建、赵永红、翁洋) P203. 定义8.6
设\(\hat{\theta}_1\)和\(\hat{\theta}_2\)都是\(\theta\)的估计量,
\(M(\hat{\theta}_1)\)和\(M(\hat{\theta}_2)\)
分别是\(\hat{\theta}_1\)和\(\hat{\theta}_2\)关于\(\theta\)的均方误差.
若\[
	M(\hat{\theta}_1) \leq M(\hat{\theta}_2),
\]
则称“\(\hat{\theta}_1\)在均方误差下比\(\hat{\theta}_2\)更\DefineConcept{有效}”.
\end{definition}

\begin{theorem}\label{theorem:参数估计.估计量的均方误差-方差-偏差的关系}
%@see: 《概率论与数理统计》(陈鸿建、赵永红、翁洋) P203. 定理8.2
估计量\(\hat{\theta}\)的方差\(D(\hat{\theta})\)、
偏差\(b(\hat{\theta})\)
和它关于\(\theta\)的均方误差\(M(\hat{\theta})\)有以下关系:
\begin{equation}
	M(\hat{\theta}) = D(\hat{\theta}) + b^2(\hat{\theta})
	= D(\hat{\theta}) + [E(\hat{\theta}) - \theta]^2.
\end{equation}
\begin{proof}
直接计算得\begin{align*}
	M(\hat{\theta})
	&= E(\hat{\theta} - \theta)^2
	= E\left\{
		[\hat{\theta} - E(\hat{\theta})]
		+ [E(\hat{\theta}) - \theta]
	\right\}^2 \\
	&= E[\hat{\theta} - E(\hat{\theta})]^2
	+ [E(\hat{\theta}) - \theta]^2
	+ 2[E(\hat{\theta}) - \theta][E(\hat{\theta}) - E(\hat{\theta})] \\
	&= D(\hat{\theta}) + [E(\hat{\theta}) - \theta]^2
	= D(\hat{\theta}) + b^2(\hat{\theta}).
	\qedhere
\end{align*}
\end{proof}
\end{theorem}
从\cref{theorem:参数估计.估计量的均方误差-方差-偏差的关系} 可以看出,
若\(\hat{\theta}\)是\(\theta\)的无偏估计量,
则偏差\(b(\hat{\theta})=0\),
从而\(M(\hat{\theta})=D(\hat{\theta})\).
这样的\(\theta\)的无偏估计量
\(\hat{\theta}_1\)与\(\hat{\theta}_2\)之间均方误差的比较就成了方差的比较.
所以\emph{有效性标准}是\emph{均方误差标准}的特殊情况.
但用均方误差可以比较两个有偏估计量之间,
或一个无偏估计量与一个有偏估计量之间的有效性.

\begin{example}
%@see: 《概率论与数理统计》(陈鸿建、赵永红、翁洋) P204. 例8.11
设总体\(X \sim N(\mu,\sigma^2)\),其中\(\mu\)和\(\sigma^2\)是未知参数.
比较\(\hat\sigma_1^2=S^2\)和\(\hat\sigma_2^2=B_2\)这两个估计量关于\(\sigma^2\)的均方误差.
\begin{solution}
因为\(E(\hat{\sigma}_1^2)
= E(S^2)
= \sigma^2\),
所以\[
	M(\hat{\sigma}_1^2)
	= M(S^2)
	= D(S^2).
\]
但是由抽样分布定理可知\[
	\frac{(n-1) S^2}{\sigma^2}
	\sim
	\x(n-1),
\]
于是\[
	D\left[\frac{(n-1) S^2}{\sigma^2}\right]
	= 2(n-1),
\]
故得到\[
	M(\hat{\sigma}_1^2)
	= D(S^2)
	= \frac{2\sigma^4}{n-1}.
\]
又因为\[
	E(\hat{\sigma}_2^2)
	= E(B_2)
	= E\left(\frac{n-1}{n} S^2\right)
	= \frac{n-1}{n} \sigma^2,
\]\[
	D(\hat{\sigma}_2^2)
	= D(B_2)
	= D\left(\frac{n-1}{n} S^2\right)
	= \frac{(n-1)^2}{n^2} D(S^2)
	= \frac{2(n-1)}{n^2} \sigma^4,
\]
故\[
	M(\hat{\sigma}_2^2)
	= D(\hat{\sigma}_2^2) + b^2(\hat{\sigma}_2^2)
	= \frac{2n-1}{n^2} \sigma^4.
\]
由于\(\frac{2}{n-1} > \frac{2n-1}{n^2}\),所以\[
	M(\hat{\sigma}_2^2) < M(\hat{\sigma}_1^2).
\]
由此可见,在均方误差意义下,\(B_2\)比\(S^2\)更有效.
\end{solution}
\end{example}

\begin{example}
%@see: 《概率论与数理统计》(茆诗松、周纪芗、张日权) P236. 习题5.2 7.
设\(\AutoTuple{X}{n}\)是取自正态总体\(N(\mu,\sigma^2)\)的一个样本.
样本的偏差平方和\(Q=\sum_{i=1}^n (X_i-\overline{X})^2\)含有正态方差\(\sigma^2\)的信息.
现要求在形如\(c Q\)(其中\(c\)是正常数)的估计量中寻找\(c\),
使\(c Q\)在均方误差准则下是\(\sigma^2\)的最优估计.
%TODO
% \begin{solution}
% 因为\(Q = (n-1) S^2\),
% \(E(Q) = (n-1) \sigma^2\),
% 所以\begin{align*}
% 	D(Q)
% 	&= \sum_{i=1}^n D(X_i-\overline{X})^2 \\
% 	&= \sum_{i=1}^n [D(X_i^2) + 4 D(X_i \overline{X}) + D(\overline{X}^2)]
% \end{align*}
% \(E(Q^2) = D(Q) + E^2(Q)
% = \),
% 所以\begin{align*}
% 	M(c Q)
% 	&= E(cQ - \sigma^2)^2
% 	= c^2 E(Q^2) - 2c \sigma^2 E(Q) + \sigma^4 \\
% 	&= c^2 E(Q^2) + [1-2c(n-1)] \sigma^4
% \end{align*}
% \end{solution}
\end{example}

\begin{example}
%@see: 《概率论与数理统计》(茆诗松、周纪芗、张日权) P233. 例5.2.5
设\(\AutoTuple{X}{n}\)是来自正态总体\(N(\mu,\sigma^2)\)的一个样本.
利用\(\chi^2\)分布的性质可知其偏差平方和
\(Q = \sum_{k=1}^n (X_k - \overline{X})^2\)
的数学期望和方差分别是\[
	E(Q) = (n-1) \sigma^2, \qquad
	D(Q) = 2(n-1) \sigma^4.
\]
现在构造如下三个估计量:\[
	S^2 = \frac{Q}{n-1}, \qquad
	S_n^2 = \frac{Q}{n}, \qquad
	S_{n+1}^2 = \frac{Q}{n+1}.
\]
这三个估计量的偏差平方、方差、均方误差如\cref{table:参数估计.估计量在不同评选标准下的表现1}.
\begin{table}[ht]
	\centering
	\begin{tblr}{*4c}
		\hline
		& \(S^2\) & \(S_n^2\) & \(S_{n+1}^2\) \\ \hline
		\(\delta^2/\sigma^4\) & \(0\) & \(1/n^2\) & \(4/(n+1)^2\) \\
		\(D(\cdot)/\sigma^4\) & \(2/(n-1)\) & \(2(n-1)/n^2\) & \(2(n-1)/(n+1)^2\) \\
		\(MSE(\cdot)/\sigma^4\) & \(2/(n-1)\) & \((2n-1)/n^2\) & \(2/(n+1)\) \\
		\hline
	\end{tblr}
	\caption{}
	\label{table:参数估计.估计量在不同评选标准下的表现1}
\end{table}
可以看出,\(S^2\)虽然是\(\sigma^2\)的无偏估计量,
但他的方差和均方误差并不小,
故从均方误差标准看,它并不优良.
\(S_n^2\)和\(S_{n+1}^2\)都不是\(\sigma^2\)的无偏估计量,
但是在均方误差标准下都优于\(S^2\).

由此可见,从不同侧面取考察估计量的好坏会得出不同的结论.
因此我们在讨论估计浪的好坏时,必须明确我们所遵循的准则是什么.
至于具体采用哪个评选标准,则需要根据实际问题来确定.
\end{example}
