\section{区间估计}
从总体\(X\)中取得样本观测值后,由参数的点估计方法,可以求得未知参数\(\theta\)的估计值.
但由于样本的随机性,我们知道一般这个估计值与参数的真实值\(\theta\)有误差,但不知道误差在什么范围内.
因而我们希望能够找出一个区间,使得参数\(\theta\)落入该区间的概率是可以计算的.
这样我们就能在一定的可靠程度下得出估计值可能的最大误差.
这就是区间估计的思想.

\subsection{置信区间}
\begin{definition}\label{definition:参数估计.置信区间的定义}
%@see: 《概率论与数理统计》(陈鸿建、赵永红、翁洋) P205. 定义8.7
设总体\(X\)的分布函数\(F(x,\theta)\)中含有未知参数\(\theta\).
\(\AutoTuple{X}{n}\)是来自总体\(X\)的样本.
\(\hat{\theta}_1=\hat{\theta}_1(\AutoTuple{X}{n})\)%
和\(\hat{\theta}_2=\hat{\theta}_2(\AutoTuple{X}{n})\)是两个统计量.
若对给定的概率\(1-\alpha\ (0<\alpha<1)\),有\[
	P(\hat{\theta}_1<\theta<\hat{\theta}_2)=1-\alpha,
\]
则称随机区间\((\hat{\theta}_1,\hat{\theta}_2)\)为
“参数\(\hat{\theta}\)的置信度为\(1-\alpha\)的\DefineConcept{置信区间}”.
\(\hat{\theta}_1\)称为\DefineConcept{置信下限}.
\(\hat{\theta}_2\)称为\DefineConcept{置信上限}.
\(1-\alpha\)称为\DefineConcept{置信度}或\DefineConcept{置信水平}.
\end{definition}

若给定样本观测值\(\AutoTuple{x}{n}\),
代入\(\hat{\theta}_1,\hat{\theta}_2\),
得到的实数区间\((\hat{\theta}_1',\hat{\theta}_2')\)也叫置信区间.
要加以区分时,它们分别叫做\DefineConcept{随机置信区间}和\DefineConcept{实数置信区间}.

通常\(\alpha\)很小,使得\(\theta\)落在这个区间外的事件是一个小概率事件,
比如\(\alpha\)取0.01、0.05等.一般来说,若\(\alpha\)越小,
则\(\theta\)落在\((\hat{\theta}_1,\hat{\theta}_2)\)内的可靠程度越大,
但这个区间也就越宽,从而估计的误差也就越大.

根据\cref{definition:参数估计.置信区间的定义},
随机置信区间的意义是:\(\theta\)落入\((\hat{\theta}_1,\hat{\theta}_2)\)内的概率为\(1-\alpha\).
而实数置信区间的意义是:当样本容量\(n\)固定时,若我们做\(N\)次抽样,
第\(k\)次得到的样本观测值\(x_{1k},x_{2k},\dotsc,x_{nk}\),
这样随机地得到\(N\)个实数置信区间\((\hat{\theta}_{1k},\hat{\theta}_{2k})\ (k=1,2,\dots,N)\);
而这\(N\)个区间中,有的包含参数\(\theta\)的真值,有的不包含.
但当置信度为\(1-\alpha\)时,
这\(N\)个区间中包含有参数\(\theta\)的真值的区间大约占\(100(1-\alpha)\%\).
例如,取\(N=1~000,\alpha=0.05\),
那么在\(1~000\)个实数置信区间中,
不包含参数真值的大约有\(50\)个.

\subsection{确定置信区间的基本方法步骤}
我们不禁想要知道,
对于给定的置信度\(1-\alpha\),
怎样根据样本来确定参数\(\theta\)的置信区间\((\hat{\theta}_1,\hat{\theta}_2)\)?
这就是参数\(\theta\)的区间估计问题.
它的基本方法步骤是:
\begin{enumerate}
	\item 设\(\AutoTuple{X}{n}\)是来自总体\(X\)的样本,
	取一个\(\theta\)的较优的点估计量\(\hat{\theta}=\hat{\theta}(\AutoTuple{X}{n})\).
	习惯上,可取\(\theta\)的无偏估计量.

	\item 由\(\hat{\theta}\)出发,
	找一个样本函数\(W=W(\hat{\theta},\theta)\).
	我们要求\(W\)的分布已知,且只含一个未知参数\(\theta\),
	并且\(W\)的分位点应能从分布表中查到.

	\item 查表求得\(W\)的两个分位点\(W_{\frac{\alpha}{2}}\)和\(W_{1-\frac{\alpha}{2}}\),
	使得\(P(W_{\frac{\alpha}{2}}<W<W_{1-\frac{\alpha}{2}})=1-\alpha\).

	\item 从不等式\(W_{\frac{\alpha}{2}}<W<W_{1-\frac{\alpha}{2}}\)
	解得等价的不等式\(\hat{\theta}_1 < \theta < \hat{\theta}_2\),
	这时有\(P(\hat{\theta}_1 < \theta < \hat{\theta}_2) = 1-\alpha\).
	于是,\((\hat{\theta}_1,\hat{\theta}_2)\)
	就是\(\theta\)的置信度为\(1-\alpha\)的随机置信区间.

	\item 若还取得样本观测值\(\AutoTuple{x}{n}\),代入可得实数置信区间.
\end{enumerate}

这样求得的置信区间称为\DefineConcept{双侧置信区间}.

我们也可根据需要类似地求得\DefineConcept{单侧置信区间},
使\(P(\theta<\hat{\theta}_2)=1-\alpha\)或\(P(\hat{\theta}_1<\theta)=1-\alpha\).

\subsection{一个正态总体下的参数的置信区间}
本小节中,设总体\(X \sim N(\mu,\sigma^2)\),\(\AutoTuple{X}{n}\)是来自总体\(X\)的样本.
\begin{example}
已知\(\sigma^2=\sigma_0^2\),
则总体均值\(\mu\)的置信区间为\[
	\left( \overline{X} - u_{1-\frac{\alpha}{2}} \frac{\sigma_0}{\sqrt{n}},
	\overline{X} + u_{1-\frac{\alpha}{2}} \frac{\sigma_0}{\sqrt{n}} \right).
\]
\begin{proof}
\def\U{\frac{\overline{X}-\mu}{\sigma_0 / \sqrt{n}}}
由于\(\overline{X}\)是\(\mu\)的无偏估计,故可取样本函数\(U=\U\).
根据\cref{equation:抽样分布定理.一个正态总体的抽样分布2} 有\[
	U \sim N(0,1).
\]

对于给定的置信度\(1-\alpha\),
由\(P(u_{\frac{\alpha}{2}} < U < u_{1-\frac{\alpha}{2}})=1-\alpha\),
注意\(-u_{1-\frac{\alpha}{2}} = u_{\frac{\alpha}{2}}\),
有\[
	P(-u_{1-\frac{\alpha}{2}} < U < u_{1-\frac{\alpha}{2}}) = 1-\alpha.
\]

由不等式\[
	-u_{1-\frac{\alpha}{2}} < U=\U < u_{1-\frac{\alpha}{2}}
\]解得等价的不等式\[
	\overline{X} - u_{1-\frac{\alpha}{2}} \frac{\sigma_0}{\sqrt{n}}
	< \mu <
	\overline{X} + u_{1-\frac{\alpha}{2}} \frac{\sigma_0}{\sqrt{n}}.
	\qedhere
\]
\end{proof}
\end{example}

\begin{example}
\(\sigma^2\)未知,则总体均值\(\mu\)的置信区间为\[
	\left( \overline{X} - t_{1-\frac{\alpha}{2}}(n-1) \frac{S}{\sqrt{n}},
	\overline{X} + t_{1-\frac{\alpha}{2}}(n-1) \frac{S}{\sqrt{n}} \right).
\]
\begin{proof}
因为\(\sigma^2\)未知,
我们只能根据\cref{equation:抽样分布定理.一个正态总体的抽样分布4}
选择样本函数\[
	t = \frac{\overline{X}-\mu}{S / \sqrt{n}} \sim t(n-1).
\]
由不等式\[
	-t_{1-\frac\alpha2}(n-1)
	= t_{\frac\alpha2}(n-1)
	< \frac{\overline{X}-\mu}{S/\sqrt{n}}
	< t_{1-\frac\alpha2}(n-1)
\]
解得等价不等式\[
	\overline{X} - t_{1-\frac\alpha2}(n-1) \frac{S}{\sqrt{n}}
	< \mu < \overline{X} + t_{1-\frac\alpha2}(n-1) \frac{S}{\sqrt{n}}.
	\qedhere
\]
\end{proof}
\end{example}

\begin{example}
\(\mu\)未知,则总体方差\(\sigma^2\)的置信区间为\[
	\left( \frac{(n-1)S^2}{\chi_{1-\frac{\alpha}{2}}^2(n-1)},
	\frac{(n-1)S^2}{\chi_{\frac{\alpha}{2}}^2(n-1)} \right).
\]
\begin{proof}
因为\(\mu\)未知,而\(S^2\)是\(\sigma^2\)的无偏估计量,
那么根据\cref{equation:抽样分布定理.一个正态总体的抽样分布3}
选择样本函数\[
	\x = \frac{(n-1)S^2}{\sigma^2} \sim \x(n-1).
\]
由不等式\[
	\x_{\frac\alpha2}(n-1)
	< \frac{(n-1)S^2}{\sigma^2}
	< \x_{1-\frac\alpha2}(n-1)
\]
解得等价不等式\[
	\frac{(n-1)S^2}{\x_{1-\frac\alpha2}(n-1)}
	< \sigma^2
	< \frac{(n-1)S^2}{\x_{\frac\alpha2}(n-1)}.
	\qedhere
\]
\end{proof}
\end{example}

\subsection{两个正态总体下的参数的置信区间}
本小节中,设有两个正态总体\(X \sim N(\mu_1,\sigma_1^2)\),
\(Y \sim N(\mu_2,\sigma_2^2)\).
\(\AutoTuple{X}{n_1}\)和\(\AutoTuple{Y}{n_2}\)是分别来自\(X\)和\(Y\)的两个独立样本.
其样本均值和样本方差分别为\[
	\overline{X} = \frac{1}{n_1} \sum_{i=1}^{n_1} X_i,
	\quad
	\overline{Y} = \frac{1}{n_2} \sum_{i=1}^{n_2} Y_i,
	\]\[
	S_1^2 = \frac{1}{n_1-1} \sum_{i=1}^{n_1} (X_i-\overline{X})^2, \quad
	S_2^2 = \frac{1}{n_2-1} \sum_{i=1}^{n_2} (Y_i-\overline{Y})^2.
\]

\begin{example}
\(\sigma_1^2\)、\(\sigma_2^2\)都已知,
则总体均值差\(\mu_1-\mu_2\)的置信区间为\[
	\left(\overline{X}-\overline{Y}-\delta,\ \overline{X}-\overline{Y}+\delta\right),
\]
其中\[
	\delta = u_{1-\frac{\alpha}{2}} \sqrt{\frac{\sigma_1^2}{n_1}+\frac{\sigma_2^2}{n_2}}.
\]
\end{example}

\begin{example}
\(\sigma_1^2\)与\(\sigma_2^2\)都未知,
但\(\sigma_1^2=\sigma_2^2=\sigma^2\),
则\(\mu_1-\mu_2\)的置信区间为\[
	\left(\overline{X}-\overline{Y}-\delta,\ \overline{X}-\overline{Y}+\delta\right),
\]
其中\[
	\delta = t_{1-\frac{\alpha}{2}} S_w \sqrt{\frac{1}{n_1}+\frac{1}{n_2}}.
\]
\end{example}

\begin{example}
当\(\mu_1\)、\(\mu_2\)、\(\sigma_1^2\)、\(\sigma_2^2\)都未知时,
总体方差比\(\sigma_1^2/\sigma_2^2\)的置信区间为\[
	\left(
		\frac{S_1^2}{S_2^2 F_{1-\frac{\alpha}{2}}},
		\frac{S_1^2}{S_2^2 F_{\frac{\alpha}{2}}}
	\right).
\]
\end{example}

\subsection{自然指数分布族分布均值参数的置信区间}
如果总体\(X\)不是服从正态分布,那么样本函数的分布是难于确定的.
这样求总体分布中未知参数的置信区间就比较困难.
但当样本容量很大时,我们可以根据中心极限定理近似地求出置信区间.

设总体\(X\)服从某个分布,分布函数为\(F(x,\theta)\),
其中\(\theta\)是未知参数,
则总体均值\(E(X)=m(\theta)\)和总体方差\(D(X)=\sigma^2(\theta)\)都应该是\(\theta\)的函数.
抽取样本\(\AutoTuple{X}{n}\),由于\(X_i\ (i=1,2,\dotsc,n)\)与总体独立同分布,
由中心极限定理,当\(n\)充分大时(一般要求\(n \geq 50\)),样本函数\[
	U = \frac{\sum_{i=1}^n{X_i} - n m(\theta)}{\sqrt{n} \sigma(\theta)}
	= \frac{\overline{X} - m(\theta)}{\sigma(\theta) / \sqrt{n}}
\]近似地服从标准正态分布\(N(0,1)\).
对于给定的置信度\(1-\alpha\),有\[
	P(\abs{U}<u_{1-\frac{\alpha}{2}})
	= P\left(
		\abs{\frac{\overline{X} - m(\theta)}{\sigma(\theta) / \sqrt{n}}}<u_{1-\frac{\alpha}{2}}
	\right)
	\approx 1-\alpha.
\]
若能从不等式\[
	\abs{\frac{\overline{X} - m(\theta)}{\sigma(\theta) / \sqrt{n}}}<u_{1-\frac{\alpha}{2}}
\]
解得等价的不等式\[
	\hat{\theta}_1 < \theta < \hat{\theta}_2,
\]
则\((\hat{\theta}_1, \hat{\theta}_2)\)就是\(\theta\)的近似的\(1-\alpha\)置信区间.

对于非正态总体\(X\),较常见的是\(X\)服从自然指数分布族分布.
常见的参数情形是均值参数,则可从不等式\[
	\abs{\frac{\overline{X} - m}{\sqrt{V(m) / n}}}<u_{1-\frac{\alpha}{2}}
\]
解得\[
	\hat{m}_1 < m < \hat{m}_2.
\]

\subsection{单侧置信限}
\begin{definition}
设总体\(X\)的分布函数\(F(x,\theta)\)中含有未知参数\(\theta\),
\(\AutoTuple{X}{n}\)是来自总体\(X\)的一个样本.

若存在统计量\(\hat{\theta}_1(\AutoTuple{X}{n})\)使得\[
	P(\theta>\hat{\theta}_1)=1-\alpha,
\]
则称“\(\hat{\theta}_1\)是参数\(\theta\)的置信度为\(1-\alpha\)的\DefineConcept{单侧置信下限}”.

若存在统计量\(\hat{\theta}_2(\AutoTuple{X}{n})\)使得\[
	P(\theta<\hat{\theta}_2)=1-\alpha,
\]
则称“\(\hat{\theta}_2\)是参数\(\theta\)的置信度为\(1-\alpha\)的\DefineConcept{单侧置信上限}”.
\end{definition}
