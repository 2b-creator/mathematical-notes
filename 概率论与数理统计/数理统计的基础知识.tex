


\section{分布的分位点}
当随机变量\(X\)的分布已知时,在概率论中,对给定的实数\(a\),
常需要计算概率\(P(X \leq a) = p\).
而在数理统计中,常常需要对给定的\(0<p<1\),求出使\(P(X \leq a) = p\)成立的实数\(a\).

\begin{definition}
设随机变量\(X\)的分布函数是\(F\),\(0<p<1\).
若实数\(a_p\)满足\begin{equation}
	F(a_p) = P(X \leq a_p) = p,
\end{equation}
则称\(a_p\)为“\(X\)(所服从的分布)的\(p\) \DefineConcept{分位点}”.

特别地,\(a_{\frac{1}{2}}\)称为\DefineConcept{中位数}.
\end{definition}

\subsection{\texorpdfstring{\(N(0,1)\)分布的\(p\)分位点\(u_p\)}{标准正态分布的p分位点}}
当随机变量\(X \sim N(0,1)\)时,其\(p\)分位点\(u_p\)满足\[
\Phi(u_p)
= P(X \leq u_p)
= \int_{-\infty}^{u_p} \frac{1}{\sqrt{2\pi}} e^{-\frac{x^2}{2}} \dd{x}
= p.
\]由于\(\Phi(-x)=1-\Phi(x)\),可得\(\Phi(-u_p)=1-\Phi(u_p)=1-p=\Phi(u_{1-p})\),即\begin{equation}
-u_p=u_{1-p}.
\end{equation}

\subsection{\texorpdfstring{\(\x(n)\)分布的\(p\)分位点\(\x_p(n)\)}{卡方分布的p分位点}}
当随机变量\(X \sim \x(n)\)时,其\(p\)分位点\(\x_p(n)\)满足\[
P(X \leq \x_p(n)) = \int_0^{\x_p(n)} f(x,n) \dd{x} = p.
\]

当\(n\)充分大(通常只要求\(n>45\))时,\(\sqrt{2\x}\)近似地服从正态分布\(N(\sqrt{2n-1},1)\),因此,有近似公式\begin{equation}
\x_p(n) \approx \frac{1}{2}\left(u_p+\sqrt{2n-1}\right)^2.
\end{equation}

\subsection{\texorpdfstring{\(t(n)\)分布的\(p\)分位点\(t_p(n)\)}{t分布的p分位点}}
当随机变量\(X \sim t(n)\)时,其\(p\)分位点\(t_p\)满足\[
P(X \leq t_p(n))
= \int_{-\infty}^{t_p(n)} t(x,n) \dd{x} = p.
\]因为\(t\)分布的密度曲线对称于纵轴,从而\(t_p(n)\)有类似\(u_p\)的性质,即\begin{equation}
-t_p(n)=t_{1-p}(n).
\end{equation}

因为\(t\)分布的极限分布是标准正态分布,所以当\(n\to\infty\)(通常只要求\(n>45\))时,有\[
t_p(n) \approx u_p.
\]

\subsection{\texorpdfstring{\(F(n,m)\)分布的\(p\)分位点\(F_p(n,m)\)}{F分布的p分位点}}
当随机变量\(X \sim F(n,m)\)时,其\(p\)分位点\(F_p(n,m)\)满足\[
P(X \leq F_p(n,m)) = \int_0^{F_p(n,m)} f(x,n,m) \dd{x} = p.
\]

特别地,当\(p<\frac{1}{2}\)时,有\begin{equation}
F_p(n,m) = \frac{1}{F_{1-p}(m,n)}.
\end{equation}
这是因为,当\(X \sim F(n,m)\)时,\(\frac{1}{X} \sim F(m,n)\),从而\[
p = P(X \leq F_p(n,m))
= P\left(\frac{1}{X} \geq \frac{1}{F_p(n,m)}\right)
= 1 - P\left(\frac{1}{X} < \frac{1}{F_p(n,m)}\right),
\]即\[
P\left(\frac{1}{X} < \frac{1}{F_p(n,m)}\right) = 1 - p.
\]由\(p\)分位点定义可知,\[
\frac{1}{F_p(n,m)} = F_{1-p}(m,n).
\]
