\chapter{数理统计的基础知识}
在前六章,我们讨论了概率论的基础知识,即对随机变量所代表的客观随机现象的统计规律性的讨论.
在概率论中,随机变量\(X\)的分布都是已知的,或假定是已知的,从而可以求出其数字特征以及讨论有关的问题.
但在实际生活中却并非如此.

在本章,我们介绍数理统计方法(mathematical methods of statistics)的基本概念,即通过采集样本、进行试验,初步地研究服从未知分布的随机变量的相关问题.

\section{总体与样本}
\begin{definition}
在数理统计中,我们称全体被研究的对象为\textbf{总体},每个研究对象是一个\textbf{个体}.
如果总体中个体数是有限的,那么称这个总体为\textbf{有限总体};反之,称这个总体为\textbf{无限总体}.
\end{definition}

在数理统计中,我们研究的不是总体的全部属性,而是总体的某项数量指标.
这项数量指标可以用一个随机变量或其对应的一个分布来刻画.
因此约定以后提到总体时,将其表记为“总体\(X\)”或“总体\(F(x)\)”.

在抽样试验前,\(n\)个个体的特征\(\{X_n\}\)是与总体同分布的随机变量;
而在抽样试验后,是\(n\)个数据\(\v{x}{n}\).
而一般在总体中抽取的这一小部分个体要对总体有充分的代表性,需要满足以下两个条件:
\begin{enumerate}
\item \textbf{随机性},即总体中每个个体都有同等机会被抽取到,通常可用编号抽签的方法或用随机数表来实现;
\item \textbf{独立性},各次抽取必须是相互独立的,即各个个体是否被抽取到彼此独立的.
\end{enumerate}

\begin{definition}
设随机变量\(\v{X}{n}\)与总体\(X\)独立同分布,则称“\(\v{X}{n}\)为一样本容量为\(n\)的来自总体\(X\)的\textbf{(简单随机)样本}”.
而\(\v{X}{n}\)的取值\(\v{x}{n}\)叫做\textbf{样本观测值}.
\end{definition}

从总体中进行有放回地抽样,得到的显然是简单随机样本.
从有限总体中进行不放回地抽样,虽然不是简单随机样本,但当总体个体数\(N\)很大而样本容量\(n\)较小(通常要求\(n/N \leqslant 0.1\)),则可以近似地看作是有放回抽样,因而近似看作是简单随机抽样,即近似地得到简单随机样本.

若将样本看作一个\(n\)维随机变量\((\v{X}{n})\),则可求出其概率分布或密度.

当总体\(X\)是离散型随机变量,且有分布律\(P(X = x) = p(x)\),则样本\((\v{X}{n})\)的分布律为\[
P(X_1=x_1,X_2=x_2,\dotsc,X_n=x_n)
= p(x_1) p(x_2) \dotsm p(x_n)
= \prod_{i=1}^n p(x_i).
\]

当总体\(X\)是连续型随机变量,且有密度函数\(f(x)\),则样本\((\v{X}{n})\)的密度函数为\[
f(\v{x}{n}) = \prod_{i=1}^n f(x_i).
\]

\begin{example}
总体\(X \sim e(\lambda)\),\(\v{X}{n}\)为来自总体\(X\)的样本.
\def\P{P(\min\{\v{X}{n}\}>3)}
求\(\P\).
\begin{solution}
\def\intx{\int_3^{+\infty}}
\def\into{\intx \intx \dotso \intx}
\def\ddx{\dd{x_1} \dd{x_2} \dotsm \dd{x_n}}
\begin{align*}
&\hspace{-20pt}\P \\
&= P(X_1>3,X_2>3,\dotsc,X_n>3) \\
&= \into f(\v{x}{n}) \ddx \\
&= \into f(x_1) f(x_2) \dotsm f(x_n) \ddx \\
&= \left( \intx \lambda e^{-\lambda x_1} \dd{x_1} \right)^n
= e^{-3n\lambda}.
\end{align*}
\end{solution}
\end{example}

\section{常见统计分布}
\(\x\)分布、\(t\)分布和\(F\)分布在数理统计中与正态分布一起构成四个最基本、最重要的分布.
下面逐一介绍前三个分布.

\subsection{\texorpdfstring{\(\x\)}{卡方}分布}
\begin{definition}\label{definition:数理统计的基础知识.卡方分布的定义}
\(\v{X}{n}\)独立同服从标准正态分布\(N(0,1)\),称随机变量\[
\x = \sum\limits_{i=1}^n{X_i^2}
\]所服从的分布为自由度为\(n\)的 \textbf{ \(\x\)分布}\footnote{念作“卡方分布”.},记为\(\x \sim \x(n)\).
\end{definition}

\begin{theorem}\label{theorem:数理统计的基础知识.卡方分布的密度函数}
\(\x(n)\)分布等价于\(\Gamma\left(\frac{n}{2},\frac{1}{2}\right)\)分布,其密度函数为\[
f(x,n) = \left\{ \begin{array}{cl}
\frac{1}{2^{n/2} \Gamma(n/2)} x^{\frac{n}{2}-1} e^{-\frac{x}{2}}, & x > 0, \\
0, & x \leqslant 0.
\end{array} \right.
\]
\begin{proof}
首先求\(Z=X_1^2\)的分布.
由于\(Z\)非负,故当\(z \leqslant 0\)时,\(P(Z \leqslant z) = 0\);
当\(z > 0\)时,\(Z\)的分布函数为\[
F_Z(z) = P(Z \leqslant z)
= P(X_1^2 \leqslant z)
= P(-\sqrt{z} \leqslant X_1 \leqslant \sqrt{z})
= F_{X_1}(\sqrt{z}) - F_{X_1}(-\sqrt{z}),
\]相应的密度函数为\[
f_Z(z) = \frac{1}{2} z^{-\frac{1}{2}} \left[
f_{X_1}(\sqrt{z}) + f_{X_2}(-\sqrt{z})
\right],
\]其中\(f_{X_1}(x) = (2\pi)^{-\frac{1}{2}} \exp(-x^2/2)\)为标准正态分布的密度函数,代入即得\[
f_Z(z) = \left\{ \begin{array}{cl}
\frac{1}{\sqrt{2\pi}} z^{-\frac{1}{2}} e^{-\frac{z}{2}}, & z>0, \\
0, & z \leqslant 0.
\end{array} \right.
\]这正是\(\Gamma\left(\frac{1}{2},\frac{1}{2}\right)\)的密度函数.

又因为\(\v{X}{n}\)独立同分布,所以\(\v{X^2}{n}\)也独立同分布于\(\Gamma\left(\frac{1}{2},\frac{1}{2}\right)\).
再由\hyperref[theorem:多维随机变量及其分布.伽马分布的可加性1]{\(\Gamma\)分布的可加性}可知\(\x = X_1^2+X_2^2+\dotsb+X_n^2\)服从\(\Gamma\left(\frac{n}{2},\frac{1}{2}\right)\).
\end{proof}
\end{theorem}

\begin{corollary}\label{theorem:数理统计的基础知识.卡方分布的数字特征}
设\(\x \sim \x(n)\),则\(E(\x) = n\),\(D(\x) = 2n\).
\end{corollary}

\begin{theorem}[可加性]\label{theorem:数理统计的基础知识.卡方分布的可加性1}
设\(\x_1 \sim \x(n_1)\),\(\x_2 \sim \x(n_2)\),%
且\(\x_1\)与\(\x_2\)相互独立,则\[
\x_1 + \x_2 \sim \x(n_1+n_2).
\]
\end{theorem}

\begin{corollary}\label{theorem:数理统计的基础知识.卡方分布的可加性2}
设随机变量序列\(\x_k \sim \x(n_k)\ (k=1,2,\dotsc,n)\),%
且\(\x_1,\x_2,\dotsc,\x_n\)相互独立,则\[
\sum\limits_{k=1}^n{\x_k} \sim \x\left(\sum\limits_{k=1}^n{n_k}\right).
\]
\end{corollary}

\subsection{\texorpdfstring{\(t\)}{t}分布}
\begin{definition}
设随机变量\(X \sim N(0,1)\),\(Y \sim \x(n)\),%
且\(X\)与\(Y\)相互独立,称随机变量\[
t = \frac{X}{\sqrt{Y/n}}
\]所服从的分布为自由度为\(n\)的 \textbf{ \(t\)分布}\footnote{又作“学生氏分布”.},记为\(t \sim t(n)\).
\end{definition}

\begin{theorem}\label{theorem:数理统计的基础知识.学生氏分布的密度函数}
\(t\)分布的密度函数为\[
t(x,n) = \frac{ \Gamma\left(\frac{n+1}{2}\right) }{ \sqrt{n\pi} \Gamma\left(\frac{n}{2}\right) } \left(1+\frac{x^2}{n}\right)^{-\frac{n+1}{2}},
\quad x \in \mathbb{R}.
\]
\end{theorem}

由于\(t\)分布的密度函数\(t(x,n)\)是偶函数,容易验证:当\(n=1\)时\(E(t)\)不存在,而当\(n \geqslant 2\)时\(E(t)=0\).

当自由度\(n\to\infty\)时,有\[
\lim\limits_{n\to\infty} t(x,n) = \frac{1}{\sqrt{2\pi}} e^{-\frac{x^2}{2}},
\quad x \in \mathbb{R}.
\]即当\(n\)充分大时(通常需要\(n \geqslant 45\)),\(t\)分布就近似于标准正态分布.

\subsection{\texorpdfstring{\(F\)}{F}分布}
\begin{definition}
设随机变量\(X \sim \x(n)\),\(Y \sim \x(m)\),且\(X\)与\(Y\)独立,称随机变量\[
F=\frac{X/n}{Y/m}
\]所服从的分布为自由度为\((n,m)\)的 \textbf{ \(F\)分布},记为\(F \sim F(n,m)\),其中\(n\)称为\DefineConcept{第一自由度}(或\DefineConcept{分子自由度}),\(m\)称为\DefineConcept{第二自由度}(或\DefineConcept{分母自由度}).
\end{definition}

显然地,\(F \sim F(n,m) \implies \frac{1}{F} \sim F(m,n)\).

\begin{theorem}
\(F\)分布的密度函数为\[
f(x,n,m) = \left\{ \begin{array}{cl}
\frac{\Gamma\left(\dfrac{n+m}{2}\right)}{\Gamma\left(\dfrac{n}{2}\right) \Gamma\left(\dfrac{m}{2}\right)} \left(\frac{n}{m}\right)^{\frac{n}{2}} x^{\frac{n}{2}-1} \left(1+\frac{n}{m}x\right)^{-\frac{n+m}{2}}, & x > 0, \\
0, & x \leqslant 0.
\end{array} \right.
\]
\end{theorem}

\section{分布的分位点}
当随机变量\(X\)的分布已知时,在概率论中,对给定的实数\(a\),%
常需要计算概率\(P(X \leqslant a) = p\).
而在数理统计中,常常需要对给定的\(0<p<1\),求出使\(P(X \leqslant a) = p\)成立的实数\(a\).

\begin{definition}
设随机变量\(X\)的分布函数是\(F\),\(0<p<1\),若实数\(a_p\)满足\[
F(a_p) = P(X \leqslant a_p) = p,
\]则称“\(a_p\)为\(X\)(所服从的分布)的\(p\)\DefineConcept{分位点}”.

特别地,\(a_{\frac{1}{2}}\)称为\DefineConcept{中位数}.
\end{definition}

\subsection{\texorpdfstring{\(N(0,1)\)分布的\(p\)分位点\(u_p\)}{标准正态分布的p分位点}}
当随机变量\(X \sim N(0,1)\)时,其\(p\)分位点\(u_p\)满足\[
\Phi(u_p)
= P(X \leqslant u_p)
= \int_{-\infty}^{u_p} \frac{1}{\sqrt{2\pi}} e^{-\frac{x^2}{2}} \dd{x}
= p.
\]由于\(\Phi(-x)=1-\Phi(x)\),可得\(\Phi(-u_p)=1-\Phi(u_p)=1-p=\Phi(u_{1-p})\),即\begin{equation}
-u_p=u_{1-p}.
\end{equation}

\subsection{\texorpdfstring{\(\x(n)\)分布的\(p\)分位点\(\x_p(n)\)}{卡方分布的p分位点}}
当随机变量\(X \sim \x(n)\)时,其\(p\)分位点\(\x_p(n)\)满足\[
P(X \leqslant \x_p(n)) = \int_0^{\x_p(n)} f(x,n) \dd{x} = p.
\]

当\(n\)充分大(通常只要求\(n>45\))时,\(\sqrt{2\x}\)近似地服从正态分布\(N(\sqrt{2n-1},1)\),因此,有近似公式\begin{equation}
\x_p(n) \approx \frac{1}{2}\left(u_p+\sqrt{2n-1}\right)^2.
\end{equation}

\subsection{\texorpdfstring{\(t(n)\)分布的\(p\)分位点\(t_p(n)\)}{t分布的p分位点}}
当随机变量\(X \sim t(n)\)时,其\(p\)分位点\(t_p\)满足\[
P(X \leqslant t_p(n))
= \int_{-\infty}^{t_p(n)} t(x,n) \dd{x} = p.
\]因为\(t\)分布的密度曲线对称于纵轴,从而\(t_p(n)\)有类似\(u_p\)的性质,即\begin{equation}
-t_p(n)=t_{1-p}(n).
\end{equation}

因为\(t\)分布的极限分布是标准正态分布,所以当\(n\to\infty\)(通常只要求\(n>45\))时,有\[
t_p(n) \approx u_p.
\]

\subsection{\texorpdfstring{\(F(n,m)\)分布的\(p\)分位点\(F_p(n,m)\)}{F分布的p分位点}}
当随机变量\(X \sim F(n,m)\)时,其\(p\)分位点\(F_p(n,m)\)满足\[
P(X \leqslant F_p(n,m)) = \int_0^{F_p(n,m)} f(x,n,m) \dd{x} = p.
\]

特别地,当\(p<\frac{1}{2}\)时,有\begin{equation}
F_p(n,m) = \frac{1}{F_{1-p}(m,n)}.
\end{equation}
这是因为,当\(X \sim F(n,m)\)时,\(\frac{1}{X} \sim F(m,n)\),从而\[
p = P(X \leqslant F_p(n,m))
= P\left(\frac{1}{X} \geqslant \frac{1}{F_p(n,m)}\right)
= 1 - P\left(\frac{1}{X} < \frac{1}{F_p(n,m)}\right),
\]即\[
P\left(\frac{1}{X} < \frac{1}{F_p(n,m)}\right) = 1 - p.
\]由\(p\)分位点定义可知,\[
\frac{1}{F_p(n,m)} = F_{1-p}(m,n).
\]

\section{总结:常见分布的数字特征}
\begin{center}
\begin{tabular}{p{4cm}p{4cm}p{4cm}}
\hline
& 数学期望 & 方差 \\ \hline
几何分布\newline\(X \sim G(p)\) & \(E(X) = 1/p\) & \(D(X) = (1-p)/p^2\) \\ \hline
超几何分布\newline\(X \sim H(n,m,N)\) & \(E(X) = nm/N\) \\ \hline
二项分布\newline\(X \sim B(n,p)\) & \(E(X) = np\) & \(D(X) = np(1-p)\) \\ \hline
泊松分布\newline\(X \sim P(\lambda)\) & \(E(X) = \lambda\) & \(D(X) = \lambda\) \\ \hline
帕斯卡分布\newline\(X \sim NB(r,p)\) & \(E(X) = r/p\) & \(D(X) = r(1-p)/p^2\) \\ \hline
均匀分布\newline\(X \sim U(a,b)\) & \(E(X) = (a+b)/2\) & \(D(X) = (b-a)^2/12\) \\ \hline
指数分布\newline\(X \sim e(\lambda)\) & \(E(X) = 1/\lambda\) & \(D(X) = 1/\lambda^2\) \\ \hline
\(\Gamma\)分布\newline\(X \sim \Gamma(\alpha,\beta)\) & \(E(X) = \alpha/\beta\) & \(D(X) = \alpha/\beta^2\) \\ \hline
\(\x\)分布\newline\(X \sim \x(n)\) & \(E(X) = n\) & \(D(X) = 2n\) \\ \hline
\(t\)分布\newline\(X \sim t(n)\) & \(E(X) = 0\ (n>1)\) \\ \hline
\end{tabular}
\end{center}

\section{统计量和抽样分布定理}
\subsection{统计量}
抽取样本后要对总体的有关问题进行推断,首先需要对数据,即对样本观测值,进行整理、加工,也就是要构造样本的函数.

\begin{definition}
\def\g#1{g(\v{#1}{n})}
样本\(\v{X}{n}\)的一个连续函数\(\g{X}\)称为\textbf{样本函数}.
若\(\g{X}\)不含任何未知参数,则称\(\g{X}\)为一个\textbf{统计量}.
而代入样本观测值后\(\g{x}\)叫做\textbf{统计量的观测值}.
统计量的分布称为\textbf{抽样分布}.

例如,对于总体\(X \sim N(\mu,\sigma^2)\),\(\sigma^2\)已知,\(\mu\)未知,%
\(\v{X}{n}\)是来自总体\(X\)的样本,那么\(\sum\limits_{i=1}^n X_i, \sum\limits_{i=1}^n \frac{X_i^2}{\sigma^2}\)是统计量,而\(\sum\limits_{i=1}^n \frac{(X_i-\mu)^2}{\sigma^2}\)是样本函数.

常用的统计量如下:
\begin{center}
\begin{tabular}{*3c}
\hline
名称 & 统计量 & 观测值 \\ \hline
样本均值 & \(\overline{X} = \frac{1}{n} \sum\limits_{i=1}^n X_i\)
    & \(\overline{x} = \frac{1}{n} \sum\limits_{i=1}^n x_i\) \\[.5cm]
样本方差 & \(S^2 = \frac{1}{n-1} \sum\limits_{i=1}^n (X_i-\overline{X})^2\)
    & \(s^2 = \frac{1}{n-1} \sum\limits_{i=1}^n (x_i-\overline{x})^2\) \\[.5cm]
样本标准差 & \(S=\sqrt{S^2}\) & \(s=\sqrt{s^2}\) \\[.5cm]
样本\(k\)阶原点矩 & \(A_k=\frac{1}{n} \sum\limits_{i=1}^n X_i^k\)
    & \(a_k=\frac{1}{n} \sum\limits_{i=1}^n x_i^k\) \\[.5cm]
样本\(k\)阶中心距 & \(B_k=\frac{1}{n} \sum\limits_{i=1}^n (X_i-\overline{X})^2\)
    & \(b_k=\frac{1}{n} \sum\limits_{i=1}^n (x_i-\overline{x})^2\) \\[.5cm]
\hline
\end{tabular}
\end{center}
\end{definition}

与总体矩一样,样本\(k\)阶中心矩也可由各阶样本原点矩表示.如\begin{align*}
B_2 &= \frac{1}{n} \sum\limits_{i=1}^n{(X_i-\overline{X})^2}
= \frac{1}{n} \left(\sum\limits_{i=1}^n{X_i^2}-2\overline{X}\sum\limits_{i=1}^n{X_i}+n\overline{X}^2\right) \\
&= \frac{1}{n} \left(\sum\limits_{i=1}^n{X_i^2}-n\overline{X}^2\right)
= A_2 - \overline{X}^2.
\end{align*}

由上可知,样本均值\(\overline{X}\)是一阶样本原点矩\(A_1\),样本方差\(S^2\)不是二阶样本中心矩\(B_2\).
另外,统计量是随机变量,其观测值是一个实数.

此外,还有一些不常用到的统计量,也罗列于此.
\begin{definition}
设\(\v{X}{n}\)是来自某总体的一个样本,称\begin{equation}
SK = \frac{B_3}{(B_2)^{3/2}}
\end{equation}为\textbf{样本偏度}(skewness).
\end{definition}
样本偏度反映了总体分布密度曲线的对称性信息.
当\(SK > 0\)时,分布的形状是右尾长,称其为“正偏的”;
当\(SK < 0\)时,分布的形状是左尾长,称其为“负偏的”.

\begin{definition}
设\(\v{X}{n}\)是来自某总体的一个样本,称\begin{equation}
KU = \frac{B_4}{(B_2)^2} - 3
\end{equation}为\textbf{样本峰度}(kurtosis).
\end{definition}
样本峰度反映了总体分布密度曲线在其峰值附近的陡峭程度的信息.
当\(KU > 0\)时,分布密度曲线在其峰附近比正态分布来得更陡峭;
当\(KU < 0\)时,比正态分布来得更平坦.

\subsection{抽样分布定理}
从理论上说,当知道总体分布时,统计量与样本函数的分布都可以确定,但事实上一般确定统计量与样本函数的分布却十分困难.而当总体服从正态分布时,一些常用统计量与样本函数的分布则是容易确定的.我们把常用的统计量与样本函数的分布的结果叫做\textbf{抽样分布定理}.

\begin{example}
样本\(\v{X}{n}\)来自总体\(X\),其中\(X\)服从指数分布\(e(\lambda)\),求样本均值\(\overline{X}\)的分布.
\begin{solution}
记\(Y = \frac{1}{n} X\),则\(Y\)的值域为\(R_Y = (0,+\infty)\).
对于\(\forall y>0\),\(Y\)有分布函数\[
F_Y(y) = P(Y \leqslant y)
= P\left(\frac{1}{n} X \leqslant y\right)
= P(X \leqslant ny)
= \int_0^{ny} \lambda e^{-\lambda x} \dd{x}
= 1 - e^{-n\lambda y},
\]密度函数\[
f_Y(y) = F'_Y(y) = \left\{ \begin{array}{lc}
n\lambda e^{-n\lambda y}, & y>0, \\
0, & y \leqslant 0,
\end{array} \right.
\]即\(Y=\frac{1}{n}X \sim e(n\lambda)\),也即\(Y \sim \Gamma(1,n\lambda)\).

注意到\(\overline{X} = \frac{1}{n} X_1 + \frac{1}{n} X_2 + \dotsb + \frac{1}{n} X_n\),而\(\frac{1}{n} X_i\ (i=1,2,\dotsc,n)\)独立同服从于\(\Gamma(1,n\lambda)\)分布,那么根据\hyperref[theorem:多维随机变量及其分布.伽马分布的可加性1]{\(\Gamma\)分布的可加性},可知\(\overline{X} \sim \Gamma(n,n\lambda)\).
\end{solution}
\end{example}

\subsubsection{一个正态总体下的抽样分布定理}
\begin{theorem}
样本\(\v{X}{n}\)来自正态总体\(N(\mu,\sigma^2)\),则\begin{gather}
\overline{X} \sim N\left(\mu,\frac{\sigma^2}{n}\right), \\
\frac{\overline{X}-\mu}{\sigma / \sqrt{n}} \sim N(0,1).
\end{gather}
\begin{proof}
因为\[
E(\overline{X})=E\left(\frac{1}{n} \sum\limits_{i=1}^n X_i\right) = \frac{1}{n} \sum\limits_{i=1}^n E(X_i) = \mu,
\]\[
D(\overline{X})=D\left(\frac{1}{n} \sum\limits_{i=1}^n X_i\right) = \frac{1}{n^2} \sum\limits_{i=1}^n D(X_i) = \frac{\sigma^2}{n}.
\]

又由\hyperref[theorem:正态分布与自然指数分布族.正态分布的可加性2]{正态分布可加性}可得\[
\overline{X} = \frac{1}{n} \sum\limits_{i=1}^n X_i
\sim N\left(\mu,\frac{\sigma^2}{n}\right).
\qedhere
\]
\end{proof}
\end{theorem}

\begin{corollary}
样本\(\v{X}{n}\)来自任何总体,都有\begin{gather}
E(\overline{X}) = E(X), \\
D(\overline{X}) = \frac{D(X)}{n}.
\end{gather}
\end{corollary}

\begin{example}
样本\(\v{X}{n}\)来自任何总体.试证:\begin{equation}
E(S^2) = \sigma^2.
\end{equation}
\begin{proof}
显然\begin{align*}
E(S^2)
&= E\left[\frac{1}{n-1} \sum\limits_{i=1}^n (X_i-\overline{X})^2\right]
= \frac{1}{n-1} \sum\limits_{i=1}^n E(X_i-\overline{X})^2 \\
&= \frac{1}{n-1} \sum\limits_{i=1}^n \left[ E(X_i^2) + E(\overline{X}^2) - 2 E(\overline{X} X_i) \right],
\end{align*}
其中\[
E(X_i^2) = E(X^2) = D(X) + [E(X)]^2 = \sigma^2 + \mu^2,
\]\[
E(\overline{X}^2)
= D(\overline{X}) + [E(\overline{X})]^2
= \frac{\sigma^2}{n} + \mu^2,
\]\[
E(\overline{X} X_i)
= E\left(\frac{1}{n} \sum\limits_{j=1}^n X_j X_i\right)
= \frac{1}{n} \left[ E(X_i^2) + E\left(\sum\limits_{\substack{1 \leqslant j \leqslant n \\ j \neq i}} X_i X_j\right) \right],
\]\[
E\left(\sum\limits_{\substack{1 \leqslant j \leqslant n \\ j \neq i}} X_i X_j\right)
= \sum\limits_{\substack{1 \leqslant j \leqslant n \\ j \neq i}} E(X_i) E(X_j)
= (n-1) \mu^2,
\]因此\begin{align*}
E(S^2) &= \frac{1}{n-1} \sum\limits_{i=1}^n \left\{
\sigma^2 + \mu^2
+ \frac{1}{n} \sigma^2 + \mu^2
- 2 \frac{1}{n} \left[ \sigma^2 + \mu^2 + (n-1)\mu^2 \right]
\right\} \\
&= \frac{1}{n-1} n \cdot \frac{n-1}{n} \sigma^2
= \sigma^2.
\qedhere
\end{align*}
\end{proof}
\end{example}
上例也就说明了为什么样本方差\(S^2\)的定义式是\[
S^2 = \frac{1}{n-1} \sum\limits_{i=1}^n (X_i-\overline{X})^2,
\]而非\[
\frac{1}{n} \sum\limits_{i=1}^n (X_i-\overline{X})^2.
\]
不过,由于后者相当于\(\frac{n-1}{n} S^2\),其数学期望为\(\left(1-\frac{1}{n}\right) \sigma^2\),所以在工程上,当\(n\)足够大时,也可以将其作为总体方差的估计量.

虽然一般情况下样本方差的抽样分布不易精确得出,%
但是总体为\(N(\mu,\sigma^2)\)的样本方差的抽样分布可以精确求出.
\begin{theorem}\label{theorem:数理统计的基础知识.正态分布总体下样本方差的抽样分布}
样本\(\v{X}{n}\)来自正态总体\(N(\mu,\sigma^2)\),则\begin{equation}
\frac{(n-1)S^2}{\sigma^2} \sim \x(n-1),
\end{equation}
且\(\overline{X}\)与\(S^2\)相互独立.
\begin{proof}
对样本\((\v{X}{n})\)作线性变换,令\[
\left\{ \def\arraystretch{1.5} \begin{array}{l}
Z_1 = \frac{1}{\sqrt{2}} X_1 - \frac{1}{\sqrt{2}} X_2, \\
Z_2 = \frac{1}{\sqrt{2\cdot3}} (X_1+X_2) - \frac{2}{\sqrt{2\cdot3}} X_3, \\
Z_3 = \frac{1}{\sqrt{3\cdot4}} (X_1+X_2+X_3) - \frac{3}{\sqrt{3\cdot4}} X_4, \\
\hdotsfor{1} \\
Z_{n-1} = \frac{1}{\sqrt{(n-1)n}} (X_1+X_2+\dotsb+X_{n-1}) - \frac{n-1}{\sqrt{(n-1)n}} X_n, \\
Z_n = \frac{1}{\sqrt{n}} (X_1+X_2+\dotsb+X_n) = \sqrt{n} \cdot \overline{X}.
\end{array} \right.
\]由于\(\v{X}{n}\)独立同分布于\(N(\mu,\sigma^2)\),所以\[
Z_1,Z_2,\dotsc,Z_{n-1} \sim N(0,\sigma^2), \qquad
Z_n \sim N(\sqrt{n} \mu,\sigma^2),
\]且\(\Cov(Z_i,Z_j) = 0\ (i \neq j)\).
这就说明,\(\v{Z}{n}\)相互独立.

由于\[
\frac{1}{\sigma^2} \sum\limits_{i=1}^n (X_i-\overline{X})^2
= \frac{1}{\sigma^2} \left( \sum\limits_{i=1}^n X_i^2 - n \overline{X}^2 \right)
= \frac{1}{\sigma^2} \left( \sum\limits_{i=1}^n Z_i^2 - Z_n^2 \right)
= \sum\limits_{i=1}^{n-1} \left( \frac{Z_i}{\sigma} \right)^2,
\]且\(\v{Z}{n-1}\)相互独立,且均服从于\(N(0,\sigma^2)\),所以\(\frac{Z_1}{\sigma},\frac{Z_2}{\sigma},\dotsc,\frac{Z_{n-1}}{\sigma}\)仍相互独立,且均服从于\(N(0,1)\).
那么由\cref{definition:数理统计的基础知识.卡方分布的定义} 可知\[
\left( \frac{Z_1}{\sigma} \right)^2
+ \left( \frac{Z_2}{\sigma} \right)^2
+ \dotsb
+ \left( \frac{Z_{n-1}}{\sigma} \right)^2
\sim \x(n-1),
\]即\[
\frac{1}{\sigma^2} \sum\limits_{i=1}^n (X_i-\overline{X})^2 \sim \x(n-1).
\]又因为\(\v{Z}{n}\)相互独立,且\[
\frac{1}{\sigma^2} \sum\limits_{i=1}^n (X_i-\overline{X})^2
= \sum\limits_{i=1}^{n-1} \left( \frac{Z_i}{\sigma} \right)^2,
\qquad
\overline{X} = \frac{1}{\sqrt{n}} Z_n,
\]所以\(\frac{1}{\sigma^2} \sum\limits_{i=1}^n (X_i-\overline{X})^2\)与\(\overline{X}\)独立.
\end{proof}
\end{theorem}
注意\(\frac{(n-1) S^2}{\sigma^2}\)有两个等价的表达式,即\[
\frac{(n-1) S^2}{\sigma^2}
= \frac{1}{\sigma^2} \sum\limits_{i=1}^n (X_i - \overline{X})^2
= \frac{n B_2}{\sigma^2}.
\]

\begin{theorem}
样本\(\v{X}{n}\)来自正态总体\(N(\mu,\sigma^2)\),则\[
t = \frac{\overline{X}-\mu}{S / \sqrt{n}} \sim t(n-1).
\]
\begin{proof}
由于\[
U = \frac{\overline{X}-\mu}{\sigma/\sqrt{n}} \sim N(0,1),
\]\[
V = \frac{(n-1)S^2}{\sigma^2} \sim \x(n-1),
\]且\(\overline{X}\)与\(S^2\)相互独立,从而\(U\)与\(V\)相互独立.于是由\(t\)分布定义可得\[
\frac{U}{\sqrt{V/(n-1)}}
=\frac{\overline{X}-\mu}{S/\sqrt{n}}
=t \sim t(n-1).
\qedhere
\]
\end{proof}
\end{theorem}
由于t分布的极限分布是标准正态分布\(N(0,1)\),故在上述定理条件下,\(n\)充分大时,\(t = \frac{\overline{X}-\mu}{S / \sqrt{n}}\)近似地服从标准正态分布.
这个结论还可以推广到非正态总体的情形.
\begin{theorem}
对任何总体\(X\),\(E(X)=\mu\),\(D(X)=\sigma^2>0\),\(\v{X}{n}\)为来自总体\(X\)的样本,则当\(n\)充分大时,近似地有\begin{enumerate}
\item \(\frac{\overline{X}-\mu}{\sigma/\sqrt{n}} \sim N(0,1)\);
\item \(\frac{\overline{X}-\mu}{S/\sqrt{n}} \sim N(0,1)\).
\end{enumerate}
\end{theorem}

\subsubsection{两个正态总体下的抽样分布定理}
\begin{theorem}
若两个总体\(X \sim N(\mu_1,\sigma_1^2)\),\(Y \sim N(\mu_2,\sigma_2^2)\),则统计量\[
\overline{X}-\overline{Y} \sim N\left(\mu_1-\mu_2,\frac{\sigma_1^2}{n_1}+\frac{\sigma_2^2}{n_2}\right),
\]从而\[
U = \frac{(\overline{X}-\overline{Y})-(\mu_1-\mu_2)}{\sqrt{\frac{\sigma_1^2}{n_1}+\frac{\sigma_2^2}{n_2}}} \sim N(0,1).
\]
\end{theorem}

\begin{theorem}
若两个总体\(X \sim N(\mu_1,\sigma^2)\),\(Y \sim N(\mu_2,\sigma^2)\),则\[
T = \frac{(\overline{X}-\overline{Y})-(\mu_1-\mu_2)}{S_w \sqrt{\frac{1}{n_1}+\frac{1}{n_2}}} \sim t(n_1+n_2-2),
\]其中\[
S_w^2 = \frac{(n_1-1)S_1^2+(n_2-1)S_2^2}{n_1+n_2-2}.
\]
\end{theorem}

\begin{theorem}
若两个总体\(X \sim N(\mu_1,\sigma_1^2)\),\(Y \sim N(\mu_2,\sigma_2^2)\),则\[
F = \frac{S_1^2 / \sigma_1^2}{S_2^2 / \sigma_2^2} \sim F(n_1-1,n_2-1).
\]
\end{theorem}

\subsubsection{一个任何总体下的抽样分布定理}
\begin{theorem}
设样本\(\v{X}{n}\)来自任何总体%
\footnote{所称“任何总体”是指该总体的分布未知,也就是说,它可能是离散的,也可能连续的,可能是均匀分布,也可能是偏态分布.},%
该总体的均值、方差分别为\(\mu\)、\(\sigma^2\in(0,+\infty)\),则当样本量\(n\)充分大时,则\begin{equation}
\overline{X} \dotsim N\left(\mu,\frac{\sigma^2}{n}\right).
\end{equation}
\begin{proof}
由\hyperref[theorem:极限定理.林德伯格-列维中心极限定理]{林德伯格-列维中心极限定理}可知\[
\frac{\sum\limits_{i=1}^n X_i - n\mu}{\sqrt{n} \sigma} \dotsim N(0,1),
\]由此可知\[
X_1+X_2+\dotsb+X_n \dotsim N(n\mu,n\sigma^2),
\]\[
\overline{X} \dotsim N\left(\mu,\frac{\sigma^2}{n}\right).
\qedhere
\]
\end{proof}
\end{theorem}
这一定理表明,无论总体分布是什么,%
只要样本容量\(n\)充分大,%
则样本均值\(\overline{X}\)总可近似看作正态分布.
