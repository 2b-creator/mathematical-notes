\section{次序统计量及其分布}
次序统计量是另一类常用的统计量,
由它还可以派生出一些有用的统计量.

\subsection{次序统计量的概念}
%@see: 《概率论与数理统计》(茆诗松) P210.
设\(\AutoTuple{X}{n}\)是取自总体\(X\)的一个样本,
其观测值为\(\AutoTuple{x}{n}\).
我们可以按从小到大的顺序排列这些样本观测值:\[
	x_{(1)} \leq \dotsb \leq x_{(n)}.
\]
我们上面的第\(i\)个值\(x_{(i)}\)称为统计量\(X_{(i)}\)的观测值,
称\(X_{(1)},\dotsc,X_{(n)}\)为该样本的\DefineConcept{次序统计量}.
特别地,把\(X_{(1)}\)称为该样本的\DefineConcept{最小次序统计量},
把\(X_{(n)}\)称为该样本的\DefineConcept{最大次序统计量}.

可以看出,次序统计量与样本完全不相同,
具体体现在以下几个方面:
\begin{enumerate}
	\item 各个次序统计量的分布是不同的.
	\item 任意两个次序统计量的联合分布也是不同的.
	\item 任意两个次序统计量是不相互独立的.
\end{enumerate}

\subsection{一个次序统计量的抽样分布}
只要总体的分布已知,
那么若干个次序统计量的联合分布都是可以求出的.
下面仅就总体\(X\)是连续型分布的情况进行讨论.

设总体\(X\)的分布函数为\(F(x)\),
概率密度函数为\(f(x)\),
\(\AutoTuple{X}{n}\)是取自总体\(X\)的一个样本.
我们来求第\(k\)个次序统计量\(X_{(k)}\)的概率密度函数和概率分布函数.

注意到样本中有两个或两个以上分量落在无穷小区间\((x,x+\increment x]\)内的概率为\(0\),
因而考虑第\(k\)个次序统计量\(X_{(k)}\)落在无穷小区间\((x,x+\increment x]\)内这一事件,
它等价于“在容量为\(n\)的样本中,
有\(k-1\)个分量小于或等于\(x\),
有\(1\)个分量落在\((x,x+\increment x]\)内,
余下\(n-k\)个分量均大于\((x+\increment x]\)”.
又因为任一样本的分量小于或等于\(x\)的概率为\(F(x)\),
大于\((x+\increment x]\)的概率为\(1 - F(x+\increment x)\),
落在\((x,x+\increment x]\)内的概率为\(F(x+\increment x) - F(x)\),
而将这\(n\)个分量分成这样的三组,
总的分法有\(\frac{n!}{(k-1)! (n-k)!}\)种,
所以\(X_{(k)}\)落在\((x,x+\increment x]\)内的概率为
\begin{align*}
	&F_k(x+\increment x) - F_k(x) \\
	&= \frac{n!}{(k-1)! (n-k)!} [F(x)]^{k-1} [F(x+\increment x) - F(x)] [1-F(x+\increment x)]^{n-k},
\end{align*}
其中\(F_k\)表示\(X_{(k)}\)的分布函数;
在上式等号两边同除以\(\increment x\),
并令\(\increment x\to0\),
则有\begin{align*}
	f_k(x)
	&= \lim_{\increment x\to0} \frac{F_k(x+\increment x) - F_k(x)}{\increment x} \\
	&= \frac{n!}{(k-1)! (n-k)!} [F(x)]^{k-1} f(x) [1-F(x)]^{n-k}.
\end{align*}
这就是说,第\(k\)个次序统计量\(X_{(k)}\)的概率密度函数为
\begin{equation}\label{equation:次序统计量.第k个次序统计量的概率密度函数}
%@see: 《概率论与数理统计》(茆诗松) P212. 公式(4.3.1)
	f_k(x)
	= \frac{n!}{(k-1)! (n-k)!} [F(x)]^{k-1} [1-F(x)]^{n-k} f(x).
\end{equation}

特别地,
在\cref{equation:次序统计量.第k个次序统计量的概率密度函数} 中取\(k=n\),
可得最大次序统计量的概率密度函数为
\begin{equation}
%@see: 《概率论与数理统计》(茆诗松) P213. 公式(4.3.2)
	f_n(x)
	= n [F(x)]^{n-1} f(x),
\end{equation}
概率分布函数为
\begin{equation}
%@see: 《概率论与数理统计》(茆诗松) P213. 公式(4.3.3)
	F_n(x)
	= n [F(x)]^n.
\end{equation}
在\cref{equation:次序统计量.第k个次序统计量的概率密度函数} 中取\(k=1\),
可得最小次序统计量的概率密度函数为
\begin{equation}
%@see: 《概率论与数理统计》(茆诗松) P213. 公式(4.3.4)
	f_1(x)
	= n [1-F(x)]^{n-1} f(x),
\end{equation}
概率分布函数为
\begin{equation}
%@see: 《概率论与数理统计》(茆诗松) P213. 公式(4.3.5)
	F_1(x)
	= 1 - [1 - F(x)]^n.
\end{equation}

\subsection{最大次序统计量与最小次序统计量的联合抽样分布}
易见\(X_{(1)}\)与\(X_{(n)}\)的联合密度函数为
\begin{align*}
	f_{1n}(y_1,y_n)
	&= \lim_{\substack{
		\increment y_1\to0 \\
		\increment y_n\to0
	}} \frac{
		P(y_1 < X_{(1)} \leq y_1+\increment y_1,
		  y_n < X_{(n)} \leq y_n+\increment y_n)
	}{\increment y_1 \cdot \increment y_n}.
\end{align*}
这里,事件\((y_1 < X_{(1)} \leq y_1+\increment y_1,
y_n < X_{(n)} \leq y_n+\increment y_n)\)
等价于“容量为\(n\)的样本中,
有\(1\)个落在区间\((y_1,y_1+\increment y_1]\)内,
有\(1\)个落在区间\((y_n,y_n+\increment y_n]\)内,
其余\(n-2\)个落在区间\((y_1+\increment y_1,y_n]\)内”,
于是\begin{align*}
	&\hspace{-20pt}
	P(y_1 < X_{(1)} \leq y_1+\increment y_1,
	  y_n < X_{(n)} \leq y_n+\increment y_n) \\
	&= \frac{n!}{1! (n-2)! 1!}
		[F(y_1+\increment y_1) - F(y_1)]
		[F(y_n) - F(y_1+\increment y_1)]^{n-2}
		[F(y_n+\increment y_n) - F(y_n)].
\end{align*}
由于\(F\)可微,
所以\begin{align*}
	\lim_{\increment y_1\to0}
	\frac{F(y_1+\increment y_1) - F(y_1)}{\increment y_1}
	= f(y_1); \\
	\lim_{\increment y_n\to0}
	\frac{F(y_n+\increment y_n) - F(y_n)}{\increment y_n}
	= f(y_n);
\end{align*}
那么
\begin{equation}\label{equation:次序统计量.最大次序统计量与最小次序统计量的联合密度函数}
%@see: 《概率论与数理统计》(茆诗松) P214. 公式(4.3.6)
	f_{1n}(y_1,y_n)
	= n(n-1) f(y_1) [F(y_n) - F(y_1)]^{n-2} f(y_n).
\end{equation}

\subsection{样本极差}
\begin{definition}
%@see: 《概率论与数理统计》(茆诗松) P215. 定义4.3.2
样本最大次序统计量与样本最小次序统计量之差,
称为\DefineConcept{样本极差},
简称\DefineConcept{极差}.
\end{definition}
极差表示样本取值范围的大小,
也反映了总体取值分散于集中的程度.
一般说来,若总体的标准差\(\sigma\)较大,
从中取出的样本的极差也会大一些;
若总体的标准差\(\sigma\)较小,
那么从中取出的样本的极差也会小一些.
反过来也是如此,
若极差较大,表明总体取值较分散,
那么总体的标准差也较大;
若极差较小,表明总体取值相对集中一些,
从而总体的标准差较小.

在实际中,
极差常在小样本(容量\(n\leq10\))的场合使用,
而在大样本场合很少使用.
这是因为极差仅使用了样本中两个极端点的信息,
而把中间的信息都丢弃了,
当样本容量越大时,
丢弃的信息也就越多,
从而留下的信息就过少,
其使用价值就不大了.

当总体分布为正态分布\(N(\mu,\sigma^2)\)时,
由\cref{equation:次序统计量.最大次序统计量与最小次序统计量的联合密度函数}
可求出容量为\(n\)时,
样本极差
\begin{equation}
%@see: 《概率论与数理统计》(茆诗松) P215. 公式(4.3.7)
	R_n
	= X_{(n)} - X_{(1)}
\end{equation}
的密度函数为
\begin{equation}
%@see: 《概率论与数理统计》(茆诗松) P215. 公式(4.3.8)
	r_n(x)
	= \frac{n(n-1)}{2\pi\sigma^2}
	\int_{-\infty}^{+\infty} \left[
		\Phi\left(
			\frac{y+x-\mu}{\sigma}
		\right)
		- \Phi\left(
			\frac{y-\mu}{\sigma}
		\right)
	\right]^{n-2}
	e^{
		-\frac{(y+x-\mu)^2}{2\sigma^2}
		-\frac{(y-\mu)^2}{2\sigma^2}
	}
	\dd{y},
	\quad x>0,
\end{equation}
其中\(\Phi\)是标准正态分布函数.
此时\(R_n\)的数学期望为
\begin{equation}
%@see: 《概率论与数理统计》(茆诗松) P215. 公式(4.3.9)
	E(R_n)
	= \sigma d_n,
\end{equation}
其中\[
	d_n
	= \frac{n(n-1)}{2\pi}
	\int_0^{+\infty} u \dd{u}
	\int_{-\infty}^{+\infty}
	[\Phi(u+v) - \Phi(v)]^{n-2}
	\exp[
		-\frac{(u+v)^2+v^2}{2}
	]
	\dd{v}
\]是一个仅与\(n\)有关的常数.
