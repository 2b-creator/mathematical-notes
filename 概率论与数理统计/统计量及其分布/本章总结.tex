\section{本章总结}
本章学习的重心是样本函数、统计量以及抽样分布.

我们首先学习了卡方分布、\(t\)分布和\(F\)分布.
\begin{table}[ht]
	\centering
	\begin{tblr}{p{3.5cm}|p{4cm}p{4cm}}
		\hline
		& 数学期望\(E(X)\) & 方差\(D(X)\) \\ \hline
		几何分布\newline\(X \sim G(p)\)
			& \(1/p\)
			& \((1-p)/p^2\) \\ \hline
		超几何分布\newline\(X \sim H(n,m,N)\)
			& \(nm/N\) \\ \hline
		二项分布\newline\(X \sim B(n,p)\)
			& \(np\)
			& \(np(1-p)\) \\ \hline
		泊松分布\newline\(X \sim P(\lambda)\)
			& \(\lambda\)
			& \(\lambda\) \\ \hline
		帕斯卡分布\newline\(X \sim NB(r,p)\)
			& \(r/p\)
			& \(r(1-p)/p^2\) \\ \hline
		均匀分布\newline\(X \sim U(a,b)\)
			& \((a+b)/2\)
			& \((b-a)^2/12\) \\ \hline
		指数分布\newline\(X \sim e(\lambda)\)
			& \(1/\lambda\)
			& \(1/\lambda^2\) \\ \hline
		\(\Gamma\)分布\newline\(X \sim \Gamma(\alpha,\beta)\)
			& \(\alpha/\beta\)
			& \(\alpha/\beta^2\) \\ \hline
		\(\x\)分布\newline\(X \sim \x(n)\)
			& \(n\)
			& \(2n\) \\ \hline
		\(t\)分布\newline\(X \sim t(n)\)
			& \(0\ (n>1)\) \\ \hline
	\end{tblr}
	\caption{常见分布的数字特征}
\end{table}

常见的统计量包括:
样本均值\(\overline{X}\)、
样本方差\(S^2\)、
样本标准差\(S\)、
样本\(k\)阶原点矩\(A_k\)、
样本\(k\)阶中心矩\(B_k\).

\begin{table}[ht]
	\centering
	\begin{tblr}{*5c}
		\hline
		样本来源
			& \(\overline{X}\)
			& \(\frac{\overline{X}-\mu}{\sigma/\sqrt{n}}\)
			& \(\frac{(n-1)S^2}{\sigma^2}\)
			& \(\frac{\overline{X}-\mu}{S/\sqrt{n}}\)
			\\
		\hline
		\(X \sim N(\mu,\sigma^2)\)
			& \(N\left(\mu,\frac{\sigma^2}{n}\right)\)
			& \(N(0,1)\)
			& \(\x(n-1)\)
			& \(t(n-1)\)
			\\
		\(X \sim e(\lambda)\)
			& \(\Gamma(n,n\lambda)\)
			\\
		任何总体(\(n\)充分大)
			&
			& \(N(0,1)\)
			&
			& \(N(0,1)\)
			\\
		\hline
	\end{tblr}
	\caption{一个总体下的抽样分布}
\end{table}

\begin{table}[ht]
	\centering
	\begin{tblr}{*5c}
		\hline
		样本来源
			& \(\overline{X}-\overline{Y}\)
			& \(\frac{(\overline{X}-\overline{Y})-(\mu_1-\mu_2)}{\sqrt{(\sigma_1^2/n_1)+(\sigma_2^2/n_2)}}\)
			& \(\frac{S_1^2/\sigma_1^2}{S_2^2/\sigma_2^2}\)
			\\
		\hline
		\(\begin{array}{l}
			X \sim N(\mu_1,\sigma_1^2) \\
			Y \sim N(\mu_2,\sigma_2^2)
		\end{array}\)
			& \(N\left(\mu_1-\mu_2,\frac{\sigma_1^2}{n_1}+\frac{\sigma_2^2}{n_2}\right)\)
			& \(N(0,1)\)
			& \(F(n_1-1,n_2-1)\)
			\\
		\hline
	\end{tblr}
	\caption{两个总体下的抽样分布}
\end{table}
