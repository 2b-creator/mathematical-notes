\chapter{随机变量的数字特征}
\section{数学期望}
\subsection{数学期望的定义及计算}
\subsubsection{离散型随机变量的数学期望}
\begin{definition}
设离散型随机变量\(X\)的概率分布为\[
P(X=x_k) = p_k, \quad k=1,2,\dotsc,
\]若级数\(\sum\limits_{k=1}^\infty {x_k p_k}\)绝对收敛,%
则称这个级数为随机变量\(X\)的\DefineConcept{数学期望},简称为\DefineConcept{期望},%
记为\(E(X)\),即\begin{equation}\label{equation:随机变量的数字特征.数学期望的定义式}
E(X) = \sum\limits_{k=1}^\infty x_k p_k.
\end{equation}

由于数学期望是\(X\)取值的加权平均,我们也把\(E(X)\)叫做\(X\)的\DefineConcept{均值}.

若级数\(\sum\limits_{k=1}^\infty x_k p_k\)不绝对收敛,%
我们称\(X\)的数学期望不存在.
\end{definition}

\begin{theorem}\label{theorem:随机变量的数字特征.0-1分布的数学期望}
设\(X \sim B(1,p)\),则\(E(X) = p\).
\begin{proof}
由数学期望的定义有,\(E(X) = 0 \cdot (1-p) + 1 \cdot p = p\).
\end{proof}
\end{theorem}

\begin{theorem}\label{theorem:随机变量的数字特征.泊松分布的数学期望}
设\(X \sim P(\lambda)\),则\(E(X) = \lambda\).
\begin{proof}
由\(p_k = P(X=k) = \frac{\lambda^k}{k!} e^{-\lambda}\)(\(k=0,1,2,\dotsc\))可得
\begin{align*}
E(X) &= \sum\limits_{k=0}^\infty {k p_k}
= \sum\limits_{k=0}^\infty {k \cdot \frac{\lambda^k}{k!} e^{-\lambda}}
= \lambda e^{-\lambda} \sum\limits_{k=1}^\infty {\frac{\lambda^{k-1}}{(k-1)!}} \\
&\xlongequal{n=k-1} \lambda e^{-\lambda} \sum\limits_{k=0}^\infty {\frac{\lambda^{n}}{n!}}
= \lambda e^{-\lambda} e^{\lambda} = \lambda.
\qedhere
\end{align*}
\end{proof}
\end{theorem}

\begin{theorem}\label{theorem:随机变量的数字特征.几何分布的数学期望}
设\(X \sim G(p)\),则\(E(X) = \frac{1}{p}\).
\begin{proof}
\def\s{\sum\limits_{k=1}^\infty }%
记\(q = 1-p\),则\(p_k = pq^{k-1}\ (k=1,2,\dotsc)\),%
\begin{align*}
E(X) &= \s k p_k
= \s k p q^{k-1}
= p \s k q^{k-1}
= p \s \dv{q^k}{q} \\
&= p \dv{q}(\s q^k)
= p \dv{q}(\frac{q}{1-q})
= \frac{p}{(1-q)^2}
= \frac{1}{p}.
\qedhere
\end{align*}
\end{proof}
\end{theorem}

\begin{theorem}
设\(X \sim H(n,m,N)\),则\(E(X) = \frac{n m}{N}\).
\begin{proof}
\def\s#1{\sum\limits_{k=#1}^\infty }%
直接计算得\[
E(X) = \s0 k \frac{C_m^k C_{N-m}^{n-k}}{C_N^n}
= n \frac{m}{N} \s1 \frac{C_{m-1}^{k-1} C_{N-m}^{n-k}}{C_{N-1}^{n-1}}
= n \frac{m}{N}.
\qedhere
\]
\end{proof}
\end{theorem}

应该注意到,并非所有离散型分布都存在数学期望.
我们构造某个随机变量\(X\)的概率分布如下:\[
p_k = P(X=k) = \frac{6}{(\pi k)^2}
\quad(k=1,2,\dotsc).
\]容易验证该分布的非负性与归一性.
然而该分布的数学期望不存在,这是因为\[
\sum\limits_{k=1}^\infty k p_k
= \sum\limits_{k=1}^\infty k \frac{6}{(\pi k)^2}
= \sum\limits_{k=1}^\infty \frac{6}{\pi^2 k}
\]不收敛.

\subsubsection{连续型随机变量的数学期望}
\begin{definition}
设连续型随机变量\(X\)的密度为\(f(x)\),%
若反常积分\(\int_{-\infty}^{+\infty} x f(x) \dd{x}\)绝对收敛,%
则称这个积分为\(X\)的数学期望,记为\[
E(X) = \int_{-\infty}^{+\infty} x f(x) \dd{x}.
\]
\end{definition}

\begin{theorem}\label{theorem:随机变量的数字特征.伽马分布的数学期望}
设\(X \sim \Gamma(\alpha,\beta)\),则\(E(X)=\frac{\alpha}{\beta}\).
\begin{proof}
\begin{align*}
E(X)
&= \int_{-\infty}^{+\infty} x f(x) \dd{x}
= \int_0^{+\infty}{ x \frac{\beta^{\alpha}}{\Gamma(\alpha)} x^{\alpha-1} e^{-\beta x} \dd{x}} \\
&= \frac{1}{\beta \Gamma(\alpha)} \int_0^{+\infty} (\beta x)^{\alpha} e^{-(\beta x)} \dd(\beta x)
= \frac{\Gamma(\alpha + 1)}{\beta \Gamma(\alpha)}
= \frac{\alpha}{\beta}.
\qedhere
\end{align*}
\end{proof}
\end{theorem}

指数分布\(e(\lambda)\)等价于\(\Gamma\)分布\(\Gamma(1,\lambda)\),故有:
\begin{theorem}\label{theorem:随机变量的数字特征.指数分布的数学期望}
设\(X \sim e(\lambda)\),则\(E(X) = \frac{1}{\lambda}\).
\end{theorem}

同样应该注意到,并非所有连续型分布都存在数学期望.

以\DefineConcept{柯西-洛伦兹分布}(简称\DefineConcept{柯西分布})\(C(a,b)\)\footnote{\(C(0,1)\)称为标准柯西分布.可以注意到,如果随机变量\(X \sim U(-\pi,\pi)\),那么\(\tan X \sim C(0,1)\).}为例,其密度函数为\[
f(x) = \frac{1}{\pi} \frac{b}{(x-a)^2+b^2},
\]其中\(a\)是\DefineConcept{定位参数}(亦即\DefineConcept{位置参数}),\(b\)是\DefineConcept{尺度参数}(亦即\DefineConcept{尺寸参数}).
柯西分布不存在数学期望,即\(\int_{-\infty}^{+\infty} \abs{x} f(x) \dd{x}\)不收敛.

\subsubsection{随机变量的函数的数学期望}
\begin{theorem}\label{theorem:随机变量的数字特征.一维随机变量的函数的数学期望}
设\(X\)为随机变量,\(y=g(x)\)是\(x\)的(分段)连续函数或单调函数,则对\(Y=g(X)\),%
\begin{enumerate}
\item 若\(X\)是\DefineConcept{离散型}的,其分布律为\(p_k = P(X=x_k)\ (k=1,2,\dotsc)\)
且级数\(\sum\limits_{k=1}^\infty g(x_k) p_k\)绝对收敛,则有\[
E(Y) = E[g(X)] = \sum\limits_{k=1}^\infty {g(x_k) p_k};
\]
\item 若\(X\)是\DefineConcept{连续型}的,其密度函数为\(f(x)\),%
且反常积分\(\int_{-\infty}^{+\infty} g(x) f(x) \dd{x}\)绝对收敛,则有\[
E(Y) = E[g(X)] = \int_{-\infty}^{+\infty} g(x) f(x) \dd{x}.
\]
\end{enumerate}
\end{theorem}

\begin{theorem}\label{theorem:随机变量的数字特征.二维随机变量的函数的数学期望}
设\((X,Y)\)为二维随机变量,\(z=g(x,y)\)是\((x,y)\)的(分区域)连续函数,则对\(Z=g(X,Y)\),\begin{enumerate}
\item 若\((X,Y)\)为\DefineConcept{离散型},其二维概率分布为\[
p_{ij} = P(X=x_i,Y=y_j), \quad i,j=1,2,\dotsc,
\]且级数\(\sum\limits_i \sum\limits_j g(x_i,y_j) p_{ij}\)绝对收敛,则有\[
E(Z) = E[g(X,Y)] = \sum\limits_i \sum\limits_j g(x_i,y_j) p_{ij};
\]
\item 若\((X,Y)\)为\DefineConcept{连续型},其二维密度函数为\(f(x,y)\),且反常积分\[
\int_{-\infty}^{+\infty} \int_{-\infty}^{+\infty} g(x,y) f(x,y) \dd{x}\dd{y}
\]绝对收敛,则有\[
E(Z) = E[g(X,Y)] = \int_{-\infty}^{+\infty} \int_{-\infty}^{+\infty} g(x,y) f(x,y) \dd{x}\dd{y}.
\]
\end{enumerate}
\end{theorem}

\begin{example}
设随机变量\(X\)的密度函数为\[
f(x) = \left\{ \begin{array}{cl}
\frac{x}{a^2} \exp(-\frac{x^2}{2a^2}), & x>0, \\
0, & x \leqslant 0.
\end{array} \right.
\]又设\(Y = 1/X\),求\(E(Y)\).
\begin{solution}
当\(X>0\)时,\(Y>0\);
此时\(Y\)的分布函数为\[\begin{aligned}
F_Y(y)
&= P(Y \leqslant y)
= P(1/X \leqslant y)
= P(X \geqslant 1/y)
= 1 - P(X < 1/y) \\
&= 1 - \int_0^{1/y} \frac{x}{a^2} \exp(-\frac{x^2}{2a^2}) \dd{x}
= \exp(-\frac{1}{2a^2y^2});
\end{aligned}\]
而密度函数为\[
f_Y(y) = F_Y'(y)
= \frac{1}{a^2 y^3} \exp(-\frac{1}{2a^2y^2}).
\]那么\[\begin{aligned}
E(Y)
&= \int_{-\infty}^{+\infty} y f_Y(y) \dd{y}
= \int_0^{+\infty} \frac{1}{a^2 y^2} \exp(-\frac{1}{2a^2y^2}) \dd{y} \\
&= \int_{+\infty}^0 -\frac{1}{\sqrt{2} a} t^{-\frac{1}{2}} e^{-t} \dd{t}
= \frac{1}{\sqrt{2} a} \Gamma\left(\frac{1}{2}\right)
= \sqrt{\frac{\pi}{2a^2}}.
\end{aligned}\]
\end{solution}
\end{example}

\subsection{数学期望的性质}
\begin{property}\label{theorem:随机变量的数字特征.数学期望的性质1}
设\(C\)是常数,随机变量\(X\)存在数学期望\(E(X)\),则有:
\begin{enumerate}
\item \(E(C) = C\);
\item \(E(C X) = C E(X)\);
\item \(E(X+Y) = E(X)+E(Y)\).
\end{enumerate}
\end{property}

\begin{property}[线性性质]\label{theorem:随机变量的数字特征.数学期望的性质2}
设\(\v{X}{n}\)是\(n\)个随机变量,%
\(C_1,C_2,\dotsc,C_n,b\)是\(n+1\)个常数,则有\[
E\left(\sum\limits_{i=1}^n C_i X_i + b\right)
=\sum\limits_{i=1}^n C_i E(X_i) + b.
\]
\end{property}

\begin{property}\label{theorem:随机变量的数字特征.数学期望的性质3}
若随机变量\(X\)与\(Y\)相互独立,则\[
E(X Y) = E(X) E(Y).
\]

一般地,若随机变量\(\v{X}{n}\)相互独立,则\[
E\left( \prod_{i=1}^n{X_i} \right)
= \prod_{i=1}^n{E(X_i)}.
\]
\end{property}

\begin{theorem}
设\(X \sim B(n,p)\),则\(E(X) = np\).
\begin{proof}[证法一]
由数学期望的定义有\[
E(X) = \sum\limits_{k=0}^n k C_n^k p^k (1-p)^{n-k},
\]其中\[
k C_n^k = k \frac{n!}{k! (n-k)!}
= n \frac{(n-1)!}{(k-1)! (n-k)!}
= n C_{n-1}^{k-1}.
\]代回原式,得\begin{align*}
E(X) &= np \sum\limits_{k=1}^n C_{n-1}^{k-1} p^{k-1} (1-p)^{n-k} \\
&= np \sum\limits_{k=0}^n C_{n-1}^k p^k (1-p)^{n-1-k} \\
&= np[p+(1-p)]^{n-1} = np.
\qedhere
\end{align*}
\end{proof}
\begin{proof}[证法二]
令\(\v{X}{n}\)独立同服从于0-1分布\(B(1,p)\),由\hyperref[theorem:多维随机变量及其分布.二项分布的可加性3]{二项分布可加性},知\[
X = X_1 + X_2 + \dotsb + X_n.
\]那么由\cref{theorem:随机变量的数字特征.0-1分布的数学期望} 可知\[
E(X) = \sum\limits_{i=1}^n E(X_i) = np.
\qedhere
\]
\end{proof}
\end{theorem}

\begin{example}
设随机变量\(X\)服从参数为1的泊松分布,求\(E\abs{X-E(X)}\).
\begin{solution}
因为\(X \sim P(1)\),\(E(X) = 1\),所以\[
\abs{X-E(X)} = \left\{ \begin{array}{cl}
1, & X=0, \\
X - 1, & X=1,2,\dotsc.
\end{array} \right.
\]因此\begin{align*}
E\abs{X-E(X)}
&= 1 \cdot P(X=0)
+ \sum\limits_{k=1}^\infty (k-1) \cdot P(X=k) \\
&= 1 \cdot P(X=0)
+ \sum\limits_{k=0}^\infty (k-1) \cdot P(X=k)
- (0-1) \cdot P(X=0) \\
&= 2 \cdot P(X=0)
+ E(X-1).
\end{align*}
又因为\(P(X=0)=1/e\),\(E(X-1) = 0\),所以\(E\abs{X-E(X)} = 2/e\).
\end{solution}
\end{example}

\section{方差}
\subsection{方差的定义及计算}
\begin{definition}
若期望\(E[X-E(X)]^2\)存在,称为\(X\)的\DefineConcept{方差},记为\[
D(X) = E[X-E(X)]^2.
\]称\(\sqrt{D(X)}\)为\(X\)的\DefineConcept{均方差}或\DefineConcept{标准差}.
\end{definition}

\begin{theorem}
对于随机变量\(X\),%
\begin{enumerate}
\item 当\(X\)为\DefineConcept{离散型随机变量},其分布律为\(p_k = P(X=x_k)\)(\(k=1,2,\dotsc\)),则\[
D(X) = \sum\limits_{k=1}^\infty [x_k - E(X)]^2 p_k;
\]
\item 当\(X\)为\DefineConcept{连续型随机变量},有密度函数\(f(x)\),则\[
D(X) = \int_{-\infty}^{+\infty} [x - E(X)]^2 f(x) \dd{x}.
\]
\end{enumerate}
\end{theorem}

\begin{corollary}\label{theorem:随机变量的数字特征.常用的方差的计算式}
对于随机变量\(X\),有\begin{equation}
D(X) = E(X^2) - [E(X)]^2.
\end{equation}
\begin{proof}
利用二项式定理展开\([X-E(X)]^2\),注意到\(E(X)\)是固定的数而不再是随机变量,再依据\cref{theorem:随机变量的数字特征.数学期望的性质2} 即有:
\begin{align*}
D(X) &= E[X-E(X)]^2
= E\{X^2 - 2E(X) \cdot X + [E(X)]^2\} \\
&= E(X^2) - 2E(X) \cdot E(X) + [E(X)]^2 \\
&= E(X^2) - [E(X)]^2.
\qedhere
\end{align*}
\end{proof}
\end{corollary}

\begin{theorem}
设\(X \sim P(\lambda)\),则\(D(X) = \lambda\).
\begin{proof}
直接计算得
\begin{align*}
E[X(X-1)]
&= \sum\limits_{k=0}^\infty {k(k-1) \frac{\lambda^k}{k!} e^{-\lambda}}
= \lambda^2 e^{-\lambda} \sum\limits_{k=2}^\infty {\frac{\lambda^{k-2}}{(k-2)!}} \\
&\xlongequal{n=k-1} \lambda^2 e^{-\lambda} \sum\limits_{n=0}^\infty {\frac{\lambda^n}{n!}}
= \lambda^2 e^{-\lambda} e^{\lambda} = \lambda^2. \\
D(X)
&= E(X^2) - [E(X)]^2
= E[X(X-1)] + E(X) - [E(X)]^2 \\
&= \lambda^2 + \lambda - \lambda^2 = \lambda.
\qedhere
\end{align*}
\end{proof}
\end{theorem}

\begin{theorem}
设\(X \sim \Gamma(\alpha,\beta)\),则\(D(X) = \frac{\alpha}{\beta^2}\).
\begin{proof}
\def\inti{\int_0^{+\infty}}%
直接计算得
\begin{align*}
E(X^2) &= \int_0^{+\infty} x^2 \frac{\beta^{\alpha}}{\Gamma(\alpha)} x^{\alpha-1} e^{-\beta x} \dd{x} \\
&= \frac{1}{\beta^2 \Gamma(\alpha)} \int_0^{+\infty} (\beta x)^{\alpha+1} e^{-(\beta x)} \dd(\beta x) \\
&= \frac{\Gamma(\alpha+2)}{\beta^2 \Gamma(\alpha)}
= \frac{(\alpha+1) \alpha \Gamma(\alpha)}{\beta^2 \Gamma(\alpha)}
= \frac{(\alpha+1) \alpha}{\beta^2}. \\
D(X) &= E(X^2) - [E(X)]^2
= \frac{\alpha (\alpha+1)}{\beta^2} - \left( \frac{\alpha}{\beta} \right)^2
= \frac{\alpha}{\beta^2}.
\qedhere
\end{align*}
\end{proof}
\end{theorem}

\begin{theorem}
设\(X \sim e(\lambda)\),则\(D(X) = \frac{1}{\lambda^2}\).
\begin{proof}
既然\(e(\lambda) = \Gamma(1,\lambda)\),结果不言自喻.
\end{proof}
\end{theorem}

\subsection{方差的性质}
\begin{property}\label{theorem:随机变量的数字特征.方差的性质1}
设\(C,a,b\)是常数,随机变量\(X\)存在方差\(D(X)\),则有:
\begin{enumerate}
\item \(D(C) = 0\);
\item \(D(aX+b) = a^2 D(X)\).
\end{enumerate}
\end{property}

\begin{property}\label{theorem:随机变量的数字特征.方差的性质2}
设随机变量\(X\)、\(Y\)相互独立,则\[
D(X \pm Y) = D(X) + D(Y).
\]

一般地,若随机变量\(\v{X}{n}\)相互独立,%
\(C_1,C_2,\dotsc,C_n\)是\(n\)个常数,则\[
D\left( \sum_{i=1}^n{C_i X_i} \right)
= \sum_{i=1}^n{C_i^2 D(X_i)}.
\]
\end{property}

\begin{theorem}
设\(X \sim B(1,p)\),则\(D(X) = p(1-p)\).
\end{theorem}

\begin{theorem}
设\(X \sim B(n,p)\),则\(D(X) = np(1-p)\).
\end{theorem}

\begin{theorem}
设\(X \sim U(a,b)\),则\(E(X) = \frac{a+b}{2}\),\(D(X) = \frac{(b-a)^2}{12}\).
\begin{proof}
直接计算得
\begin{align*}
E(X) &= \int_{-\infty}^{+\infty} x f(x) \dd{x}
= \int_a^b x \frac{1}{b-a} \dd{x}
= \frac{1}{b-a} \frac{1}{2} (x^2)_a^b
= \frac{a+b}{2}, \\
E(X^2) &= \int_{-\infty}^{+\infty} x^2 f(x) \dd{x}
= \int_a^b x^2 \frac{1}{b-a} \dd{x}
= \frac{1}{b-a} \frac{1}{3} (x^3)_a^b
= \frac{a^2+ab+b^2}{3}, \\
D(X) &= E(X^2) - [E(X)]^2
= \frac{(a-b)^2}{12}.
\qedhere
\end{align*}
\end{proof}
\end{theorem}

\begin{theorem}\label{theorem:随机变量的数字特征.几何分布的方差}
设\(X \sim G(p)\),则\(D(X) = \frac{q}{p^2}\).
\begin{proof}
\def\s{\sum\limits_{k=1}^\infty }%
\def\l{\lim\limits_{n\to\infty}}%
记\(q = 1-p\),则\(p_k = pq^{k-1}\ (k=1,2,\dotsc)\).
因为\begin{align*}
\s (k+1)kpq^{k-1}
&= p \s \dv[2]{q}(q^{k+1})
= p \dv[2]{q}(\s q^{k+1}) \\
&= p \dv[2]{q}(\l \frac{q^2-q^{n+2}}{1-q})
= p \dv[2]{q}(\frac{q^2}{1-q})
= \frac{2}{p^2}, \\
\end{align*}所以\begin{align*}
E(X^2) &= \s k^2 pq^{k-1}
= \s [(k+1)k-k] pq^{k-1}
= \frac{2}{p^2} - \frac{1}{p}, \\
D(X) &= E(X^2) - [E(X)]^2
= \frac{2}{p^2} - \frac{1}{p} - \frac{1}{p^2}
= \frac{q}{p^2}.
\qedhere
\end{align*}
\end{proof}
\end{theorem}

\begin{example}
设随机变量\(X\)的概率分布为\(P(X=k) = \frac{C}{k!}\ (k=0,1,2,\dotsc)\),求\(E(X^2)\).
\begin{solution}
由\hyperref[theorem:随机变量及其分布.离散型随机变量的密度函数的性质]{规范性}和\cref{equation:无穷级数.幂级数展开式1} 可知\[
\sum\limits_{k=0}^\infty \frac{C}{k!}
= C \sum\limits_{k=1}^\infty \eval{\frac{x^k}{k!}}_{x=1}
= C \eval{e^x}_{x=1}
= C e = 1
\implies
C = e^{-1}.
\]那么随机变量\(X\)的分布为\[
P(X=k) = \frac{1^k}{k!} e^{-1} \quad(k=0,1,2,\dotsc)
\]将其与\hyperref[equation:随机变量及其分布.泊松分布的分布律]{泊松分布的分布律}作比较,可知\(X \sim P(1)\),因此\(E(X) = D(X) = 1\),\[
E(X^2) = D(X) + [E(X)]^2 = 1 + 1 = 2.
\]
\end{solution}
\end{example}

\section{变异系数、原点矩、中心距}
\begin{definition}
若随机变量\(X\)的期望、方差均存在,且\(E(X) \neq 0\),称\[
C_v = \frac{\sqrt{D(X)}}{\abs{E(X)}}
\]为\(X\)的\DefineConcept{变异系数}.

变异系数\(C_v\)衡量了\(X\)取值在\(E(X)\)周围的相对集中程度.
\end{definition}

\begin{definition}
若对非负整数\(k\),随机变量\(X^k\)存在期望,记作\[
m_k = E(X^k),
\]称\(m_k\)为\(X\)的\(k\)阶\DefineConcept{原点矩}.

若对非负整数\(k\),随机变量\([X-E(X)]^k\)存在期望,记作\[
\mu_k = E[X-E(X)]^k,
\]称\(\mu_k\)为\(X\)的\(k\)阶\DefineConcept{中心距}.

显然有\(E(X) = m_1\),\(D(X) = \mu_2\).
而\(m_0 = \mu_0 = 1\).
\end{definition}

\begin{theorem}
\begin{align*}
\mu_k &= E[X-E(X)]^k
= E\left[ \sum\limits_{r=0}^k{C_k^r X^r (-m_1)^{k-r}} \right] \\
&= \sum\limits_{r=0}^k{C_k^r E(X^r) (-m_1)^{k-r}}
= \sum\limits_{r=0}^k{C_k^r m_r (-m_1)^{k-r}}.
\end{align*}
\end{theorem}

\section{协方差和相关系数}
\subsection{协方差}
给定二维随机变量\((X,Y)\),一般\(X\)与\(Y\)之间存在一定的关系.
我们希望有一个数字特征可以用来描述\(X\)与\(Y\)的关系.
我们知道,当\(X\)与\(Y\)独立时,有\(E(XY) = E(X) E(Y)\)成立,从而\begin{align*}
&E\bigl\{[X-E(X)][Y-E(Y)]\bigr\} \\
&= E[XY - Y E(X) - X E(Y) + E(X) E(Y)] \\
&= E(XY) - E(Y) E(X) - E(X) E(Y) + E(X) E(Y) \\
&= E(XY) - E(X) E(Y) = 0.
\end{align*}
也就是说,当\(E\bigl\{[X-E(X)][Y-E(Y)]\bigr\} \neq 0\)时,\(X\)与\(Y\)就不独立.
这就说明,\(E\bigl\{[X-E(X)][Y-E(Y)]\bigr\}\)能在一定程度上刻画\(X\)与\(Y\)的关系.

\begin{definition}
对于二维随机变量\((X,Y)\),若\(E\bigl\{[X-E(X)][Y-E(Y)]\bigr\}\)存在,则称之为\(X\)与\(Y\)的\DefineConcept{协方差}(或\DefineConcept{相关矩}),记为\(\Cov(X,Y)\),即
\begin{equation}\label{equation:随机变量的数字特征.协方差的定义式}
\Cov(X,Y) = E\bigl\{[X-E(X)][Y-E(Y)]\bigr\}.
\end{equation}
\end{definition}

根据本节前面的推导,不难得到如下的性质:
\begin{property}\label{theorem:随机变量的数字特征.协方差的性质1}
对于二维随机变量\((X,Y)\),总有\begin{equation}\label{equation:随机变量的数字特征.协方差的计算式1}
\Cov(X,Y) = E(XY) - E(X) E(Y).
\end{equation}
\end{property}

进一步,令\(X = Y\),则还可以得到:
\begin{property}\label{theorem:随机变量的数字特征.协方差的性质2}
对于随机变量\(X\),总有\begin{equation}\label{equation:随机变量的数字特征.协方差与方差的联系1}
\Cov(X,X) = D(X).
\end{equation}
\end{property}

\begin{property}\label{theorem:随机变量的数字特征.协方差的性质3}
设\(X\)、\(Y\)、\(Z\)以及\(X_i\)、\(Y_j\)均为随机变量,则有:
\begin{enumerate}
\item \(\Cov(X,Y)=\Cov(Y,X)\);
\item \(\Cov(X,a)=0\)(\(a\)是常数);
\item \(\Cov(aX,bY)=ab\Cov(X,Y)\)(\(a\)、\(b\)是常数);
\item \(\Cov(X+Y,Z)=\Cov(X,Z)+\Cov(Y,Z)\);
\item \(\Cov\left(\sum\limits_{i=1}^n{a_i X_i},\sum\limits_{j=1}^m{b_j Y_j}\right)
= \sum\limits_{i=1}^n \sum\limits_{j=1}^m {a_i b_j \Cov(X_i,Y_j)}\)(\(a_i\)、\(b_j\)是常数);
\item 若\(X\)与\(Y\)相互独立,则\(\Cov(X,Y)=0\).
\end{enumerate}
\end{property}

\begin{theorem}
设\(D(X) = \sigma_1^2, D(Y) = \sigma_2^2\),则\begin{equation}
[\Cov(X,Y)]^2 \leqslant \sigma_1^2 \sigma_2^2.
\end{equation}
\end{theorem}

\begin{theorem}
设随机变量\(X\)、\(Y\)的方差\(D(X)\)、\(D(Y)\)存在,且协方差\(\Cov(X,Y)\)存在,那么\begin{equation}
D(X \pm Y) = D(X) + D(Y) \pm 2 \Cov(X,Y).
\end{equation}
\begin{proof}
由方差的定义可知\begin{align*}
D(X \pm Y)
&= E[(X \pm Y) - E(X \pm Y)]^2 \\
&= E\bigl\{[X - E(X)] \pm [Y - E(Y)]\bigr\}^2 \\
&= E\bigl\{
[X - E(X)]^2 + [Y - E(Y)]^2 - 2 [X - E(X)][Y - E(Y)]
\bigr\} \\
&= E[X - E(X)]^2 + E[Y - E(Y)]^2 - 2 E[X - E(X)][Y - E(Y)] \\
&= D(X) + D(Y) \pm 2 \Cov(X,Y).
\qedhere
\end{align*}
\end{proof}
\end{theorem}

我们可以进一步将上述定理推广到任意有限个随机变量相加的情形:
\begin{corollary}
设随机变量\(\v{X}{n}\)的方差\(D(X_i)\ (i=1,2,\dotsc,n)\)、协方差\(\Cov(X_i,X_j)\ (i,j=1,2,\dotsc,n)\)都存在,那么\begin{equation}
D\left( \sum\limits_{i=1}^n X_i \right)
= \sum\limits_{i=1}^n D(X_i)
+ 2 \sum\limits_{i<j} \Cov(X_i,X_j).
\end{equation}
\end{corollary}

\begin{example}
设二维随机向量\((X,Y)\)的联合密度函数为\[
p(x,y) = \left\{ \begin{array}{cl}
3x, & 0<y<x<1, \\
0, & \text{其他}.
\end{array} \right.
\]
求协方差\(\Cov(X,Y)\).
\begin{solution}
根据\cref{equation:随机变量的数字特征.协方差的计算式1} 可以直接计算得:
\begin{align*}
E(XY)
&= \int_0^1 \dd{x} \int_0^x xy \cdot 3x \dd{y}
= \frac{3}{10}, \\
E(X)
&= \int_0^1 \dd{x} \int_0^x x \cdot 3x \dd{y}
= \frac{3}{4}, \\
E(Y)
&= \int_0^1 \dd{x} \int_0^x y \cdot 3x \dd{y}
= \frac{3}{8}, \\
\Cov(X,Y)
&= E(XY) - E(X) E(Y)
= \frac{3}{160}.
\end{align*}
\end{solution}
\end{example}

\begin{example}[配对问题]
有\(n\)个人,每人将自己的礼品扔入同一箱中,把礼品充分混合后,每人再随机从中选取一个.
试求选中自己礼品的人数\(X\)的数学期望与方差.
\begin{solution}
设随机变量\[
X_i = \left\{ \begin{array}{ll}
1, & \text{当第\(i\)个人取出自己的礼品}, \\
0, & \text{当第\(i\)个人取出别人的礼品},
\end{array} \right.
\quad(i=1,2,\dotsc,n)
\]同分布于\[
\begin{bmatrix}
0 & 1 \\
\frac{1}{n} & 1-\frac{1}{n}
\end{bmatrix},
\]
即\[
P(X_i=1) = \frac{1}{n}, \qquad
P(X_i=0) = 1-\frac{1}{n}
\quad(i=1,2,\dotsc,n).
\]
那么其数学期望与方差分别为\[
E(X_i) = \frac{1}{n},
\qquad
D(X_i) = \frac{1}{n}\left(1-\frac{1}{n}\right)
\quad(i=1,2,\dotsc,n).
\]

在上述假设下,\(n\)个人种选中自己礼品的人数恰好为\[
X = X_1+X_2+\dotsb+X_n.
\]因此\[
E(X) = E(X_1)+E(X_2)+\dotsb+E(X_n)
= n \cdot \frac{1}{n}
= 1.
\]

由于\(X_i\ (i=1,2,\dotsc,n)\)不是相互独立的,所以\(X\)的方差为\[
D(X) = \sum\limits_{i=1}^n D(X_i) + 2 \sum\limits_{i<j} \Cov(X_i,X_j).
\]为了计算\(\Cov(X_i,X_j)\ (i \neq j)\),我们来考察\(X_i X_j\)的含义:\[
X_i X_j = \left\{ \begin{array}{ll}
1, & \text{当第\(i\)个人和第\(j\)个人都恰好取出各自的礼品}, \\
0, & \text{其他},
\end{array} \right.
\]于是\[
E(X_i X_j) = 1 \cdot P(X_i=1,X_j=1)
= P(X_i=1) \cdot P(X_j=1 \vert X_i=1)
= \frac{1}{n} \cdot \frac{1}{n-1},
\]因此\[
\Cov(X_i,X_j) = \frac{1}{n(n-1)} - \left(\frac{1}{n}\right)^2
= \frac{1}{n^2(n-1)},
\]\[
D(X) = n \cdot \frac{n-1}{n^2} + 2 \cdot \frac{n(n-1)}{2} \cdot \frac{1}{n^2(n-1)}
= 1.
\]
\end{solution}
由此可见,在配对问题,成对个数的均值与方差都是1,与人数\(n\)无关.
\end{example}

\subsection{均值向量\ 协方差阵}
\begin{definition}
对于二维随机变量\((X,Y)\),称向量\[
(E(X),E(Y))
\]为\((X,Y)\)的\DefineConcept{数学期望}或\DefineConcept{均值向量};
称矩阵\[
\mat{V} = \begin{bmatrix}
D(X) & \Cov(X,Y) \\
\Cov(Y,X) & D(Y)
\end{bmatrix}
\]为\((X,Y)\)的\DefineConcept{协方差阵}.

一般地,对\(n\)维随机变量\((\v{X}{n})\),称向量\[
(E(X_1),E(X_2),\dotsc,E(X_n))
\]为\((\v{X}{n})\)的\DefineConcept{均值向量};
记\[
\sigma_{ij} = \Cov(X_i,Y_j),
\quad i,j=1,2,\dotsc,n,
\]称矩阵\[
\mat{V} = \begin{bmatrix}
\sigma_{11} & \sigma_{12} & \dots & \sigma_{1n} \\
\sigma_{21} & \sigma_{22} & \dots & \sigma_{2n} \\
\vdots & \vdots & \ddots & \vdots \\
\sigma_{n1} & \sigma_{n2} & \dots & \sigma_{nn}
\end{bmatrix}
\]为\((\v{X}{n})\)的\DefineConcept{协方差阵}.
\end{definition}

易见\(\mat{V}\)总是对称阵.

\subsection{标准化随机变量}
\begin{definition}
设随机变量\(X\)与\(Y\)的期望、方差都存在,且\(D(X) > 0\),\(D(Y) > 0\),则将随机变量\[
X^* = \frac{X-E(X)}{\sqrt{D(X)}},
\qquad
Y^* = \frac{Y-E(Y)}{\sqrt{D(Y)}}
\]分别称为\(X\)与\(Y\)的\DefineConcept{标准化随机变量}.
\end{definition}

\begin{property}\label{theorem:随机变量的数字特征.标准化随机变量的数字特征}
任意随机变量\(X\)的标准化随机变量\(X^*\)具有以下性质:
\begin{enumerate}
\item \(E(X^*)=0\);
\item \(D(X^*)=1\).
\end{enumerate}
\begin{proof}
由\cref{theorem:随机变量的数字特征.数学期望的性质1} 、\cref{theorem:随机变量的数字特征.常用的方差的计算式} 有:\begin{align*}
E(X^*) &= \frac{1}{\sigma} E[X-E(X)] = 0, \\
D(X^*) &= E({X^*}^2) = \frac{1}{\sigma^2} E[X-E(X)] = 1.
\qedhere
\end{align*}
\end{proof}
\end{property}

\subsection{相关系数}
\begin{definition}\label{definition:随机变量的数字特征.相关系数}
设随机变量\(X\)、\(Y\)的标准化随机变量分别为\(X^*\)、\(Y^*\).
称“\(X^*\)与\(Y^*\)的协方差\(\Cov(X^*,Y^*)\)”为“\(X\)与\(Y\)的\DefineConcept{(线性)相关系数}”,记作\(R(X,Y)\),即\[
R(X,Y) = \Cov(X^*,Y^*).
\]
\end{definition}

\begin{theorem}\label{theorem:随机变量的数字特征.相关系数的性质1}
\(R(X,Y)=E(X^* Y^*)\).
\begin{proof}
根据\cref{theorem:随机变量的数字特征.协方差的性质1}和\cref{theorem:随机变量的数字特征.标准化随机变量的数字特征}立即可得.
\end{proof}
\end{theorem}

\begin{theorem}\label{theorem:随机变量的数字特征.相关系数的性质2}
\(R(X,Y)=\frac{\Cov(X,Y)}{\sqrt{D(X)} \sqrt{D(Y)}}\).
\end{theorem}

\begin{property}
相关系数具有以下性质:
\begin{enumerate}
\item \(R(X,Y)=R(Y,X)\);
\item \(\abs{R(X,Y)} \leqslant 1\);
\item \(\abs{R(X,Y)}=1 \iff \exists a \in \mathbb{R}^* \land \exists b \in \mathbb{R} : P(Y=aX+b)=1\).
\end{enumerate}
\end{property}

\begin{definition}
对于随机变量\(X\)、\(Y\),%
当满足\(R(X,Y)>0\)时,称“\(X\)与\(Y\) \DefineConcept{正(线性)相关}”;
当满足\(R(X,Y)<0\)时,称“\(X\)与\(Y\) \DefineConcept{负(线性)相关}”.
当\(R(X,Y)=0\)时,称“\(X\)与\(Y\) \DefineConcept{不相关}”,即“\(X\)与\(Y\)没有线性相关关系”.
当满足\(R(X,Y)=1\)时,称“\(X\)与\(Y\) \DefineConcept{完全正线性相关}”;
当满足\(R(X,Y)=-1\)时,称“\(X\)与\(Y\) \DefineConcept{完全负线性相关}”.
\end{definition}

显然地,当\(X\)与\(Y\)相互独立时,必有\(X\)与\(Y\)不相关.
但当\(X\)与\(Y\)不相关时,却不必然有\(X\)与\(Y\)相互独立.
例如,设\(X \sim N(0,1)\),\(Y=X^2\),于是有\[
E(X) = 0,
\qquad
E(Y) = E(X^2) = 1,
\qquad
E(XY) = E(X^3) = 0,
\]
可见\(\Cov(X,Y) = E(XY) - E(X) E(Y) = 0\),从而\(R(X,Y) = 0\).
但\(X\)与\(Y\)之间存在函数关系\(Y=X^2\),不能说\(X\)与\(Y\)独立.
