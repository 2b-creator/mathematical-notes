\chapter{自然指数分布族}

在数理统计中,最重要的分布就是正态分布.
正态分布的重要性在于:
在实际问题中有许多随机变量服从(或近似服从)正态分布,例如成年男子的身高、体重,工件的测量误差,大气的温度、湿度等;
正态分布的密度函数与分布函数具有许多良好的性质;
正态分布是许多分布的极限分布;
正态分布在数理统计中的基础作用.
所以,许多实际问题与理论问题的解决,都离不开正态分布.

另外,二项分布、泊松分布、指数分布、\(\Gamma\)分布、几何分布等分布也很重要.
由于它们与正态分布具有许多统一的概率性质,这些分布都被划归为自然指数分布族.

\section{自然指数分布族}
\begin{definition}
若存在\(H \subseteq \mathbb{R}\)上的实值函数\(\phi(\theta)\),
以及不依赖于\(\theta\)的函数\(h(x)\),
非退化的随机变量\(X\)有概率分布或概率密度函数\[
f(x,\theta) = \exp[\theta x - \phi(\theta)] h(x),
\quad x \in G,\,\theta \in H,
\]则称\(X\)服从\DefineConcept{自然指数分布族分布}.
其中\(\theta\)叫做自然参数,\(H\)叫做\DefineConcept{自然参数空间},
\(\phi(\theta)\)叫做\DefineConcept{累积量母函数},
\(G\)叫做\DefineConcept{支撑集},且\(G\)不依赖于\(\theta\).
\end{definition}

\begin{theorem}
若\(X\)服从自然指数分布族分布,则\[
E(X) = \phi'(\theta),
\quad
D(X) = \phi''(\theta),
\quad
\theta \in H.
\]
\end{theorem}

\begin{landscape}
	\begin{table}
		\centering
		\begin{tblr}{|*{5}{c|}}
			\hline
			密度函数或概率分布
				& 自然参数\(\theta\)
				& 累积量母函数\(\phi(\theta)\)
				& 均值参数\(m=\phi'(\theta)\)
				& 方差函数\(V(m) = \phi''(\theta)\) \\ \hline
			正态函数\(N(\mu,\sigma^2)\)
				& \(\frac{\mu}{\sigma^2}\)
				& \(\frac{\sigma^2 \theta^2}{2}\)
				& \(m=\mu=\theta\sigma^2\)
				& \(\sigma^2\) \\ \hline
			泊松分布\(P(\lambda)\)
				& \(\ln\lambda\)
				& \(e^{\theta}\)
				& \(m=\lambda=e^{\theta}\)
				& \(\lambda=m\) \\ \hline
			\(\Gamma\)分布\(\Gamma(\alpha,\beta)\)
				& \(-\beta\)
				& \(-\alpha\ln(-\theta)=-\alpha\ln\beta\)
				& \(m=\frac{\alpha}{\beta}=-\frac{\alpha}{\theta}\)
				& \(\frac{\alpha}{\beta^2}=\frac{m^2}{\alpha}\) \\ \hline
			二项分布\(B(n,p)\)
				& \(\ln\frac{p}{q}\)
				& \(n\ln(1+e^{\theta})=-n\ln{q}\)
				& \(m=np=\frac{n}{1+e^{-\theta}}\)
				& \(npq=-\frac{m^2}{n}+m\) \\ \hline
			负二项分布\(NB(r,p)\)
				& \(\ln{q}\)
				& \(r\ln(\frac{e^{\theta}}{1-e^{\theta}})=r\ln\frac{q}{p}\)
				& \(m=\frac{r}{q}=\frac{r}{1-e^{\theta}}\)
				& \(\frac{rq}{p^2}=\frac{m^2}{r}-m\) \\ \hline
		\end{tblr}
		\caption{常见自然指数分布族分布}
		\label{table:自然指数分布族分布.常见自然指数分布族分布}
	\end{table}
\end{landscape}

对于\cref{table:自然指数分布族分布.常见自然指数分布族分布} 有以下两点需要注意:
\begin{enumerate}
	\item 作为自然指数分布族分布,
	正态分布的参数\(\sigma^2\)和\(\Gamma\)分布的参数\(\alpha\)是作为已知的,
	且\(\alpha=1\)时\(\Gamma\)分布化为指数分布\(e(\beta)\).

	\item 当\(r=1\)时负二项分布化为几何分布\(G(p)\).
\end{enumerate}

\begin{definition}
称\(m\)为自然指数分布族的\DefineConcept{均值参数}.
称均值参数\(m\)的取值区间为\DefineConcept{均值空间},记为\(M\),即\[
M = \{ m \vert m = \phi'(\theta), \theta \in H \}.
\]

由于\(m\)与\(\theta\)存在一一对应关系,即\(\theta = \theta(m)\),则\[
D(X) = \phi''(\theta) = V(m),
\]称\(V(m)\)为\(X\)的\DefineConcept{方差函数}.
\end{definition}

\begin{theorem}
若\(\AutoTuple{X}{n}\)独立同服从于一个自然指数分布族分布,其方差函数为\(V(m)\),则\(Y=X_1+X_2+\dotsb+X_n\)也服从同一个自然指数分布族分布,且\[
E(Y)=n m,
\]\[
D(Y)=n V(m).
\]
\end{theorem}
