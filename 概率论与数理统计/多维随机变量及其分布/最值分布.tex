\section{最大值、最小值的分布}
\begin{theorem}
设随机变量\(\AutoTuple{X}{n}\)相互独立,
且\(X_i\)有分布函数\(F_i(x_i)\ (i=1,2,\dotsc,n)\),
则最大值\(M=\max\{\AutoTuple{X}{n}\}\)的分布函数为
\begin{equation}
	F_M(x) = F_1(x) F_2(x) \dotsm F_n(x);
\end{equation}
最小值\(N=\min\{\AutoTuple{X}{n}\}\)的分布函数为
\begin{equation}
	F_N(x) = 1 - [1-F_1(x)][1-F_2(x)]\dotsm[1-F_n(x)].
\end{equation}
\begin{proof}
显然有:
\begin{align*}
	F_M(x) &= P(\max\{\AutoTuple{X}{n}\} \leq x) \\
	&= P(X_1 \leq x,X_2 \leq x,\dotsc,X_n \leq x) \\
	&= P(X_1 \leq x) P(X_2 \leq x) \dotsm P(X_n \leq x) \\
	&= F_1(x) F_2(x) \dotsm F_n(x); \\
	F_N(x) &= P(\min\{\AutoTuple{X}{n}\} \leq x) \\
	&= 1 - P(\min\{\AutoTuple{X}{n}\} > x) \\
	&= 1 - P(X_1 > x,X_2 > x,\dotsc,X_n > x) \\
	&= 1 - P(X_1 > x) P(X_2 > x) \dotsm P(X_n > x) \\
	&= 1 - [1-F_1(x)][1-F_2(x)]\dotsm[1-F_n(x)].
	\qedhere
\end{align*}
\end{proof}
\end{theorem}

\begin{corollary}
%@see: 《概率论与数理统计》(茆诗松、周纪芗、张日权) P137 定理3.2.1
设随机变量\(\AutoTuple{X}{n}\)独立同分布,
它们的分布函数为\(F(x)\),密度函数为\(f(x)\).
那么这些随机变量的最大值\(M=\max\{\AutoTuple{X}{n}\}\)
和它们最小值\(N=\min\{\AutoTuple{X}{n}\}\)的分布函数分别为
\begin{gather}
	F_M(x) = [F(x)]^n, \\
	F_N(x) = 1-[1-F(x)]^n.
\end{gather}
\(M\)和\(N\)的密度函数分别为
\begin{gather}
	f_M(x) = n [F(x)]^{n-1} f(x), \\
	f_N(x) = n [1-F(x)]^{n-1} f(x).
\end{gather}
\end{corollary}

\begin{example}
设\(\AutoTuple{X}{n}\)独立同分布于\(U(0,1)\),求它们的最大值、最小值分布的分布函数.
\begin{solution}
均匀分布\(U(0,1)\)的分布函数为\[
F(x) = \left\{ \begin{array}{cc}
0, & x \leq 0, \\
x, & 0 < x < 1, \\
1, & x \geq 1.
\end{array} \right.
\]于是最大值\(M\)与最小值\(N\)的分布函数分别为\[
F_M(x) = \left\{ \begin{array}{cc}
0, & x \leq 0, \\
x^n, & 0 < x < 1, \\
1, & x \geq 1.
\end{array} \right.
\qquad
F_N(x) = \left\{ \begin{array}{cc}
0, & x \leq 0, \\
1-(1-x)^n, & 0 < x < 1, \\
1, & x \geq 1.
\end{array} \right.
\]
\end{solution}
\end{example}
