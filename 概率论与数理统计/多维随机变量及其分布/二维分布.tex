\section{二维随机变量及其分布函数}
\subsection{二维随机变量及其分布函数}
\begin{definition}
设\(X\)与\(Y\)是定义在同一样本空间\(\Omega\)上的两个随机变量,
则称\((X,Y)\)为二维\DefineConcept{随机变量}(或\DefineConcept{随机向量}).
\end{definition}

\begin{definition}
设\((X,Y)\)是二维随机变量,对任意实数\(x\),\(y\),称\begin{equation}\label{equation:多维随机变量及其分布.二维分布函数的定义式}
F(x,y) = P(X \leq x, Y \leq y)
\end{equation}为二维随机变量\((X,Y)\)的\DefineConcept{二维分布函数},
或称之为\(X\)与\(Y\)的\DefineConcept{联合分布函数}.
\end{definition}

\begin{property}
\(P(x_1 < X \leq x_2, y_1 < Y \leq y_2)
= F(x_2,y_2) - F(x_2,y_1) - F(x_1,y_2) + F(x_1,y_1)\).
\end{property}

\begin{property}
设\(F(x,y)\)为随机变量\((X,Y)\)的分布函数,则
\begin{enumerate}
	\item \(F(x,y)\)分别关于\(x\)及\(y\)单调不减,即
	当\(x_1 < x_2\)时,有\(F(x_1,y) \leq F(x_2,y)\);
	当\(y_1 < y_2\)时,有\(F(x,y_1) \leq F(x,y_2)\).

	\item \(F(-\infty,-\infty)=F(-\infty,y)=F(x,-\infty)=0\),\(F(+\infty,+\infty)=1\).

	\item \(F(x,y)\)关于\(x\)及\(y\)都右连续,即对任意实数\(x\)、\(y\)有
	\(F(x^+,y)=F(x,y)\)和\(F(x,y^+)=F(x,y)\)成立.

	\item 对任意\(x_1 < x_2\)、\(y_1 < y_2\)有\[
		P(x_1 < X \leq x_2, y_1 < Y \leq y_2)
		= F(x_2,y_2) - F(x_2,y_1) - F(x_1,y_2) + F(x_1,y_1)
		\geq 0.
	\]
\end{enumerate}
\end{property}

需要指出,上述四点也是二维分布函数的特征,也就是说,
任何一个二元函数只要满足这四点就是某二维随机变量的分布函数.

\section{二维离散型随机变量及其概率分布}

\subsection{二维离散型随机变量及其概率分布的概念与性质}
\begin{definition}
如果二维随机变量\((X,Y)\)只取有限个或可数无穷个点对\((x_i,y_i)\ (i,j=1,2,\dotsc)\),
则称\((X,Y)\)为\DefineConcept{二维离散型随机变量}.
\end{definition}

\begin{definition}
设二维离散型随机变量\((X,Y)\)所有可能取值为\((x_i,y_i)\ (i,j=1,2,\dotsc)\),记\[
p_{ij} = P(X = x_i, Y = y_j), \quad i,j = 1,2,\dotsc,
\]则称上式为\((X,Y)\)的\DefineConcept{二维概率分布}或\DefineConcept{二维概率分布律},
或称为\(X\)与\(Y\)的\DefineConcept{联合概率分布}.
\end{definition}

二维概率分布可以用\cref{table:多维随机变量及其分布.二维概率分布} 表示.

\begin{table}[ht]
\centering
\begin{tabular}{c|*5c}
	& \(y_1\) & \(y_2\) & \(\dots\) & \(y_j\) & \(\dots\) \\ \hline
\(x_1\) & \(p_{11}\) & \(p_{12}\) & \(\dots\) & \(p_{1j}\) & \(\dotsc\) \\
\(x_2\) & \(p_{21}\) & \(p_{22}\) & \(\dots\) & \(p_{2j}\) & \(\dotsc\) \\
\(\vdots\) & \(\vdots\) & \(\vdots\) & & \(\vdots\) \\
\(x_i\) & \(p_{i1}\) & \(p_{i2}\) & \(\dots\) & \(p_{ij}\) & \(\dotsc\) \\
\(\vdots\) & \(\vdots\) & \(\vdots\) & & \(\vdots\) \\
\end{tabular}
\caption{\((X,Y)\)的二维概率分布}
\label{table:多维随机变量及其分布.二维概率分布}
\end{table}

\begin{property}
二维离散型随机变量的概率分布有如下的性质:
\begin{enumerate}
	\item {\bf 非负性}:
	\(p_{ij} \geq 0\ (i,j=1,2,\dotsc)\);

	\item {\bf 规范性}:
	\(\sum_{i,j} p_{ij} = 1\).
\end{enumerate}
\end{property}

\begin{theorem}
对于任意一个二维点集\(G\),对任意二维离散型随机变量\((X,Y)\)可以求事件\(((X,Y) \in G)\)的概率,即\[
P\left[(X,Y) \in G\right] = \sum_{(x_i,y_j) \in G} p_{ij}.
\]

特别地,二维离散型随机变量\((X,Y)\)的二维分布函数可用概率分布求出,即\[
F(x,y) = \sum_{x_i \leq x}\sum_{y_j \leq y} p_{ij},
\]且有\[
p_{ij} = F(x_i,y_j) - F(x_i,y_{j-1}) - F(x_{i-1},y_j) + F(x_{i-1},y_{j-1}), \quad i,j = 1,2,\dotsc,
\]其中,规定\(x_0 = y_0 = -\infty\).
\end{theorem}

\subsection{重要的二维离散型分布 --- 三项分布}
\begin{definition}
在\(n\)重独立试验中,若每次试验只有\(A_1\)、\(A_2\)、\(A_3\)三个可能结果,
且\(0 < p_i = P(A_i) < 1\ (i=1,2,3)\),则\(p_1 + p_2 + p_3 = 1\).
令随机变量\(X\)及\(Y\)分别表示\(n\)次试验中\(A_1\)与\(A_2\)发生的次数,
则\(X\)与\(Y\)的联合概率分布为\[
P(X=k_1,Y=k_2) = \frac{n!}{k_1! k_2! (n-k_1-k_2)!} p_1^{k_1} p_2^{k_2} p_3^{n-k_1-k_2},
\]其中\(k_1+k_2 = 0,1,\dotsc,n\),\(k_1 \geq 0\),\(k_2 \geq 0\),
并称\((X,Y)\)服从参数为\(p _1\)、\(p_2\)、\(n\)的\DefineConcept{三项分布},记为\((X,Y) \sim T(n;p_1,p_2)\).
\end{definition}

\section{二维连续型随机变量及其密度函数}

\subsection{二维连续型随机变量及其密度函数的概念与性质}

\begin{definition}
设二维随机变量\((X,Y)\)有分布函数\(F(x,y)\),如果存在二元非负函数\(f(x,y)\),使得对任意实数\(x\)、\(y\)有\[
F(x,y) = \int_{-\infty}^x \int_{-\infty}^y f(u,v) \dd{u} \dd{v},
\]则称\((X,Y)\)是\DefineConcept{二维连续型随机变量},称\(f(x,y)\)为\((X,Y)\)的\DefineConcept{二维概率密度函数},
或称为\(X\)与\(Y\)的\DefineConcept{联合密度函数},简称为\DefineConcept{密度函数}或\DefineConcept{密度}.
\end{definition}

\begin{property}
二维连续型随机变量的密度函数有如下的性质:
\begin{enumerate}
	\item {\bf 非负性}:
	\((\forall (x,y)\in\mathbb{R}^2)[f(x,y) \geq 0]\);

	\item {\bf 规范性}:
	\(F(+\infty,+\infty) = \int_{-\infty}^{+\infty} \int_{-\infty}^{+\infty} f(x,y) \dd{x} \dd{y} = 1\).
\end{enumerate}
\end{property}

\begin{theorem}
设二维连续型随机变量\((X,Y)\)有密度函数\(f(x,y)\),则
\begin{enumerate}
	\item \(F(x,y)\)是连续函数且在\(f(x,y)\)的连续点\((x,y)\),有\[
		f(x,y) = \pdv{F(x,y)}{x}{y};
	\]

	\item 对平面上任意区域\(G \subseteq \mathbb{R}^2\),若\(f(x,y)\)在\(G\)上可积,有\[
		P\left[(X,Y) \in G\right] = \iint_G{f(x,y) \dd{x}\dd{y}};
	\]

	\item 对平面上任一条曲线\(L\),有\[
		P\left[(X,Y) \in L\right] = 0.
	\]
\end{enumerate}
\end{theorem}

\subsection{重要的二维连续型分布 --- 均匀分布}
\begin{definition}
令\(G\)是平面上一个有界区域,若二维随机变量\((X,Y)\)有密度函数\[
f(x,y) = \left\{ \begin{array}{ll}
\frac{1}{m(G)}, & (x,y) \in G, \\
0, & \text{其他}, \\
\end{array} \right.
\]其中\(m(G)\)为\(G\)的面积,则称\((X,Y)\)为在\(G\)上的\DefineConcept{均匀分布},记为\((X,Y) \sim U(G)\).
\end{definition}
