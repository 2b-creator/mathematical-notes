\section{边缘分布及随机变量的独立性}
\subsection{边缘分布函数与随机变量的独立性}
\begin{definition}
二维随机变量\((X,Y)\)的分量\(X\)、\(Y\)均可看作一维随机变量.
这两个分量各自的分布函数\(F_X(x)\)、\(F_Y(y)\),
相对于二维分布函数\(F(x,y)\)而被分别称为\(X\)与\(Y\)的\DefineConcept{边缘分布函数}.
\end{definition}

\begin{theorem}
设\(F(x,y)\)为二维随机变量\((X,Y)\)的二维分布函数,
则\(X\)与\(Y\)的边缘分布函数\(F_X(x)\)、\(F_Y(y)\)有
\begin{align*}
	F_X(x) &= F(x,+\infty), \quad x \in \mathbb{R}; \\
	F_Y(y) &= F(+\infty,y), \quad y \in \mathbb{R}.
\end{align*}
\end{theorem}

\begin{definition}
%@see: 《概率论与数理统计》(茆诗松、周纪芗、张日权) P132. 定义3.2.1
设\(\AutoTuple{X}{n}\)是\(n\)维随机变量.
若对任意\(n\)个实数\(\AutoTuple{x}{n}\),
\(n\)个事件\((X_1 \leq x_1),\dotsc,(X_n \leq x_n)\)相互独立,
即有\[
	P(X_1 \leq x_1,\dotsc,X_n \leq x_n)
	= P(X_1 \leq x_1) \dotsm P(X_n \leq x_n)
\]
或\[
	F(x_1,\dotsc,x_n)
	= F_1(x_1) \dotsm F_n(x_n),
\]
其中\(F\)是\(n\)维随机变量\(\AutoTuple{X}{n}\)的联合分布函数,
而\(F_1,\dotsc,F_n\)分别是\(X_1,\dotsc,X_n\)的边缘分布函数,
则称“\(n\)个随机变量\(\AutoTuple{X}{n}\)~\DefineConcept{相互独立}”;
否则称“\(n\)个随机变量\(\AutoTuple{X}{n}\)~\DefineConcept{不相互独立}”
或“\(n\)个随机变量\(\AutoTuple{X}{n}\)~\DefineConcept{相依}”.
\end{definition}

\begin{theorem}
设随机变量\(X\)与\(Y\)相互独立,且\(g(x)\)与\(h(y)\)均是连续函数,
则\(X_1 = g(X)\)与\(Y_1 = h(Y)\)也相互独立.
\end{theorem}

\subsection{二维离散型随机变量的边缘分布及独立性}
\begin{definition}
设\((X,Y)\)是二维离散型随机变量,有二维概率分布\[
p_{ij} = P(X=x_i,Y=y_j), \quad i,j=1,2,\dotsc.
\]显然此时\(X\)与\(Y\)都是一维离散型随机变量,各有分布律
\begin{align*}
p_{i*} &= P(X=x_i), \quad i=1,2,\dotsc; \\
p_{*j} &= P(Y=y_j), \quad j=1,2,\dotsc.
\end{align*}
相对于二维概率分布,\(X\)与\(Y\)各自的分布叫做\DefineConcept{边缘概率分布},简称\DefineConcept{边缘分布}.
\end{definition}

\begin{theorem}
设\((X,Y)\)是二维离散型随机变量,有二维概率分布\[
p_{ij} = P(X=x_i,Y=y_j), \quad i,j=1,2,\dotsc.
\]分量\(X\)与\(Y\)的边缘分布可由二维概率分布求出,即
\begin{align*}
p_{i*} = \sum_{j}{p_{ij}}, \quad i=1,2,\dotsc; \\
p_{*j} = \sum_{i}{p_{ij}}, \quad j=1,2,\dotsc.
\end{align*}
\end{theorem}

\begin{theorem}
设\((X,Y)\)是二维离散型随机变量,有二维概率分布\[
p_{ij} = P(X=x_i,Y=y_j), \quad i,j=1,2,\dotsc,
\]则随机变量\(X\)与\(Y\)相互独立的充分必要条件是:\[
p_{ij} = p_{i*} p_{*j}, \quad i,j=1,2,\dotsc.
\]
\end{theorem}

\subsection{二维连续型随机变量的边缘密度及独立性}
\begin{theorem}
设二维连续型随机变量\((X,Y)\)的二维密度为\(f(x,y)\),
\(X\)与\(Y\)的边缘密度分别为\(f_X(x)\)和\(f_Y(y)\),则
\begin{align*}
	f_X(x) = \int_{-\infty}^{+\infty} f(x,y) \dd{y}, \\
	f_Y(y) = \int_{-\infty}^{+\infty} f(x,y) \dd{x}.
\end{align*}

而\(X\)与\(Y\)相互独立的充分必要条件是:\[
	f(x,y) = f_X(x) f_Y(y).
\]在三个密度函数的公共连续点上成立.
\end{theorem}
