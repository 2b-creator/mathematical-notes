\section{矩}
\subsection{原点矩}
\begin{definition}
若对非负整数\(k\),随机变量\(X^k\)存在期望,记作\[
m_k = E(X^k),
\]称\(m_k\)为\(X\)的\(k\)阶\DefineConcept{原点矩}.
\end{definition}

\subsection{中心矩}
\begin{definition}
若对非负整数\(k\),随机变量\([X-E(X)]^k\)存在期望,记作\[
\mu_k = E[X-E(X)]^k,
\]称\(\mu_k\)为\(X\)的\(k\)阶\DefineConcept{中心矩}.
\end{definition}

显然有\(E(X) = m_1\),\(D(X) = \mu_2\).
而\(m_0 = \mu_0 = 1\).

原点矩和中心矩可以相互表示:
\begin{align*}
	\mu_k &= E[X-E(X)]^k
	= E\left[ \sum_{r=0}^k{C_k^r X^r (-m_1)^{k-r}} \right] \\
	&= \sum_{r=0}^k{C_k^r E(X^r) (-m_1)^{k-r}}
	= \sum_{r=0}^k{C_k^r m_r (-m_1)^{k-r}}.
\end{align*}
