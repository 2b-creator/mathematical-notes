\section{协方差}
\subsection{协方差}
给定二维随机变量\((X,Y)\),一般\(X\)与\(Y\)之间存在一定的关系.
我们希望有一个数字特征可以用来描述\(X\)与\(Y\)的关系.
我们知道,当\(X\)与\(Y\)独立时,
有\(E(XY) = E(X) E(Y)\)成立,
从而\begin{align*}
    &E((X-E(X))(Y-E(Y))) \\
    &= E(XY - Y E(X) - X E(Y) + E(X) E(Y)) \\
    &= E(XY) - E(Y) E(X) - E(X) E(Y) + E(X) E(Y) \\
    &= E(XY) - E(X) E(Y) = 0.
\end{align*}
也就是说,当\(E((X-E(X))(Y-E(Y))) \neq 0\)时,\(X\)与\(Y\)就不独立.
这就说明,\(E((X-E(X))(Y-E(Y)))\)能在一定程度上刻画\(X\)与\(Y\)的关系.

\begin{definition}
对于二维随机变量\((X,Y)\),
若\(E((X-E(X))(Y-E(Y)))\)存在,
则称之为\(X\)与\(Y\)的\DefineConcept{协方差}(或\DefineConcept{相关矩}),
记为\(\Cov(X,Y)\),即
\begin{equation}\label{equation:随机变量的数字特征.协方差的定义式}
    \Cov(X,Y) \defeq E((X-E(X))(Y-E(Y))).
\end{equation}
\end{definition}

根据本节开篇的推导,不难得到如下的性质:
\begin{property}\label{theorem:随机变量的数字特征.协方差的性质1}
对于二维随机变量\((X,Y)\),总有
\begin{equation}\label{equation:随机变量的数字特征.协方差的计算式1}
    \Cov(X,Y) = E(XY) - E(X) E(Y).
\end{equation}
\end{property}

进一步,令\(X = Y\),则还可以得到:
\begin{property}\label{theorem:随机变量的数字特征.协方差的性质2}
对于随机变量\(X\),总有
\begin{equation}\label{equation:随机变量的数字特征.协方差与方差的联系1}
    \Cov(X,X) = D(X).
\end{equation}
\end{property}

\begin{property}\label{theorem:随机变量的数字特征.协方差的性质3}
设\(X\)、\(Y\)、\(Z\)以及\(X_i\)、\(Y_j\)均为随机变量,则有:
\begin{enumerate}
    \item \(\Cov(X,Y)=\Cov(Y,X)\);
    \item \(\Cov(X,a)=0\)(\(a\)是常数);
    \item \(\Cov(aX,bY)=ab\Cov(X,Y)\)(\(a\)、\(b\)是常数);
    \item \(\Cov(X+Y,Z)=\Cov(X,Z)+\Cov(Y,Z)\);
    \item \(\Cov\left(\sum_{i=1}^n{a_i X_i},\sum_{j=1}^m{b_j Y_j}\right)
    = \sum_{i=1}^n \sum_{j=1}^m {a_i b_j \Cov(X_i,Y_j)}\)(\(a_i\)、\(b_j\)是常数);
    \item 若\(X\)与\(Y\)相互独立,则\(\Cov(X,Y)=0\).
\end{enumerate}
\end{property}

\begin{theorem}
%@see: 《概率论与数理统计》(茆诗松、周纪芗、张日权) P151 定理3.3.6
设\(D(X) = \sigma_1^2,
D(Y) = \sigma_2^2\),
则\begin{equation}\label{equation:随机变量的数字特征.协方差不等式1}
    [\Cov(X,Y)]^2 \leq \sigma_1^2 \sigma_2^2.
\end{equation}
\begin{proof}
如果\(\sigma_1^2=0\),
那么由\cref{theorem:大数律.方差为零的随机变量的性质} 可知,\(X\)几乎处处是常数;
而由\cref{theorem:随机变量的数字特征.协方差的性质3} 可知,常数与\(Y\)的协方差必为零;
于是\cref{equation:随机变量的数字特征.协方差不等式1} 两端都是零,
因此\cref{equation:随机变量的数字特征.协方差不等式1} 成立.

如果\(\sigma_1^2>0\),
因为\[
	E\{t[X-E(X)]+[Y-E(Y)]\}^2
	= t^2 \sigma_1^2
	+ 2t \Cov(X,Y)
	+ \sigma_2^2
	\geq 0,
\]
所以这个关于\(t\)的二次三项式的判别式非正,即\[
	[2\Cov(X,Y)]^2 - 4 \sigma_1^2 \sigma_2^2 \leq 0,
\]
移项便得\cref{equation:随机变量的数字特征.协方差不等式1}.
\end{proof}
\end{theorem}

现在我们对\cref{theorem:随机变量的数字特征.方差的性质2} 做一番推广.
\begin{theorem}
设随机变量\(X\)、\(Y\)的方差\(D(X)\)、\(D(Y)\)都存在,
且它们的协方差\(\Cov(X,Y)\)也存在,
那么\begin{equation}
    D(X \pm Y) = D(X) + D(Y) \pm 2 \Cov(X,Y).
\end{equation}
\begin{proof}
由方差的定义可知
\begin{align*}
    D(X \pm Y)
    &= E[(X \pm Y) - E(X \pm Y)]^2 \\
    &= E\bigl\{[X - E(X)] \pm [Y - E(Y)]\bigr\}^2 \\
    &= E\bigl\{
    [X - E(X)]^2 + [Y - E(Y)]^2 - 2 [X - E(X)][Y - E(Y)]
    \bigr\} \\
    &= E[X - E(X)]^2 + E[Y - E(Y)]^2 - 2 E[X - E(X)][Y - E(Y)] \\
    &= D(X) + D(Y) \pm 2 \Cov(X,Y).
    \qedhere
\end{align*}
\end{proof}
\end{theorem}

我们可以进一步将上述定理推广到任意有限个随机变量相加的情形:
\begin{corollary}
设随机变量\(\AutoTuple{X}{n}\)的方差\(D(X_i)\ (i=1,2,\dotsc,n)\)、
协方差\(\Cov(X_i,X_j)\ (i,j=1,2,\dotsc,n)\)都存在,
那么\begin{equation}
    D\left( \sum_{i=1}^n X_i \right)
    = \sum_{i=1}^n D(X_i)
    + 2 \sum_{i<j} \Cov(X_i,X_j).
\end{equation}
\end{corollary}

\begin{example}
设二维随机向量\((X,Y)\)的联合密度函数为\[
    p(x,y) = \left\{ \begin{array}{cl}
    3x, & 0<y<x<1, \\
    0, & \text{其他}.
    \end{array} \right.
\]
求协方差\(\Cov(X,Y)\).
\begin{solution}
根据\cref{equation:随机变量的数字特征.协方差的计算式1} 可以直接计算得:
\begin{align*}
    E(XY)
    &= \int_0^1 \dd{x} \int_0^x xy \cdot 3x \dd{y}
    = \frac{3}{10}, \\
    E(X)
    &= \int_0^1 \dd{x} \int_0^x x \cdot 3x \dd{y}
    = \frac{3}{4}, \\
    E(Y)
    &= \int_0^1 \dd{x} \int_0^x y \cdot 3x \dd{y}
    = \frac{3}{8}, \\
    \Cov(X,Y)
    &= E(XY) - E(X) E(Y)
    = \frac{3}{160}.
\end{align*}
\end{solution}
\end{example}

\begin{example}[配对问题]
有\(n\)个人,每人将自己的礼品扔入同一箱中,把礼品充分混合后,每人再随机从中选取一个.
试求选中自己礼品的人数\(X\)的数学期望与方差.
\begin{solution}
设随机变量\[
    X_i = \left\{ \begin{array}{ll}
    1, & \text{当第\(i\)个人取出自己的礼品}, \\
    0, & \text{当第\(i\)个人取出别人的礼品},
    \end{array} \right.
    \quad(i=1,2,\dotsc,n)
\]同分布于\[
    \begin{bmatrix}
    0 & 1 \\
    \frac{1}{n} & 1-\frac{1}{n}
    \end{bmatrix},
\]
即\[
    P(X_i=1) = \frac{1}{n}, \qquad
    P(X_i=0) = 1-\frac{1}{n}
    \quad(i=1,2,\dotsc,n).
\]
那么其数学期望与方差分别为\[
    E(X_i) = \frac{1}{n},
    \qquad
    D(X_i) = \frac{1}{n}\left(1-\frac{1}{n}\right)
    \quad(i=1,2,\dotsc,n).
\]

在上述假设下,\(n\)个人中选中自己礼品的人数恰好为\[
    X = X_1+X_2+\dotsb+X_n.
\]因此\[
    E(X) = E(X_1)+E(X_2)+\dotsb+E(X_n)
    = n \cdot \frac{1}{n}
    = 1.
\]

由于\(X_i\ (i=1,2,\dotsc,n)\)不是相互独立的,所以\(X\)的方差为\[
    D(X) = \sum_{i=1}^n D(X_i) + 2 \sum_{i<j} \Cov(X_i,X_j).
\]
为了计算\(\Cov(X_i,X_j)\ (i \neq j)\),我们来考察\(X_i X_j\)的含义:\[
    X_i X_j = \left\{ \begin{array}{ll}
    1, & \text{当第\(i\)个人和第\(j\)个人都恰好取出各自的礼品}, \\
    0, & \text{其他},
    \end{array} \right.
\]于是\[
    E(X_i X_j) = 1 \cdot P(X_i=1,X_j=1)
    = P(X_i=1) \cdot P(X_j=1 \vert X_i=1)
    = \frac{1}{n} \cdot \frac{1}{n-1},
\]
因此\[
    \Cov(X_i,X_j) = \frac{1}{n(n-1)} - \left(\frac{1}{n}\right)^2
    = \frac{1}{n^2(n-1)},
\]\[
    D(X) = n \cdot \frac{n-1}{n^2} + 2 \cdot \frac{n(n-1)}{2} \cdot \frac{1}{n^2(n-1)}
    = 1.
\]
\end{solution}
由此可见,在配对问题,成对个数的均值与方差都是1,与人数\(n\)无关.
\end{example}

\subsection{均值向量\ 协方差阵}
\begin{definition}
对于二维随机变量\((X,Y)\),称向量\[
    (E(X),E(Y))
\]
为“\((X,Y)\)的\DefineConcept{数学期望}或\DefineConcept{均值向量}”.
称矩阵\[
    \vb{V} = \begin{bmatrix}
    D(X) & \Cov(X,Y) \\
    \Cov(Y,X) & D(Y)
    \end{bmatrix}
\]为“\((X,Y)\)的\DefineConcept{协方差阵}”.

一般地,对\(n\)维随机变量\((\AutoTuple{X}{n})\),称向量\[
    (E(X_1),E(X_2),\dotsc,E(X_n))
\]
为“\((\AutoTuple{X}{n})\)的\DefineConcept{数学期望}或\DefineConcept{均值向量}”.
记\[
    \sigma_{ij} = \Cov(X_i,Y_j),
    \quad i,j=1,2,\dotsc,n;
\]
称矩阵\[
    \vb{V} = \begin{bmatrix}
    \sigma_{11} & \sigma_{12} & \dots & \sigma_{1n} \\
    \sigma_{21} & \sigma_{22} & \dots & \sigma_{2n} \\
    \vdots & \vdots & & \vdots \\
    \sigma_{n1} & \sigma_{n2} & \dots & \sigma_{nn}
    \end{bmatrix}
\]
为“\((\AutoTuple{X}{n})\)的\DefineConcept{协方差阵}”.
\end{definition}

易见协方差阵\(\vb{V}\)总是对称阵.
