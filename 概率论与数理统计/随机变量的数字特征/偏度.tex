\section{偏度}
\begin{definition}
%@see: 《概率论与数理统计》(茆诗松、周纪芗、张日权) P107 定义2.5.3
设随机变量\(X\)的前三阶矩存在.
把\[
	\frac{\mu_3}{\mu_2^{3/2}}
	=\frac{E[X-E(X)]^3}{[D(X)]^{3/2}}
\]称为“\(X\)的\DefineConcept{偏度}”,
记作\(\beta_s\).

当\(\beta_s>0\)时,称这个分布是\DefineConcept{正偏的}或\DefineConcept{右偏的};
当\(\beta_s<0\)时,称这个分布是\DefineConcept{负偏的}或\DefineConcept{左偏的};
当\(\beta_s=0\)时,称这个分布是\DefineConcept{不偏的}.
当一个分布的偏度不等于零时,我们称这个分布为\DefineConcept{偏态分布}.
\end{definition}

偏度\(\beta_s\)是描述分布偏离对称性程度的一个特征数.

当分布密度函数\(f(x)\)的图形关于它的数学期望\(E(X)\)定出的直线\(x=E(X)\)对称,
即\(f(E(X)-x)=f(E(X)+x)\)时,其三阶中心矩\(\mu_3\)必为零,从而它的偏度为零.
作为特例,正态分布\(N(\mu,\sigma^2)\)关于\(E(X)=\mu\)是对称的,故正态分布的偏度都是零.

偏度\(\beta_s\)是以分布的标准差的三次方\([D(X)]^{3/2}\)为单位,
来度量三阶中心矩大小的,从而消去了量纲,使其更具有可比性.
简单地说,分布的三阶中心矩\(mu_3\)决定偏度的符号,
而分布的标准差\(\sqrt{\mu_2}\)决定偏度的大小.

\begin{example}
%@see: 《概率论与数理统计》(茆诗松、周纪芗、张日权) P108 例2.5.2
讨论三个贝塔分布\(B(2,8),B(8,2),B(5,5)\)的偏度.
\begin{solution}
设随机变量\(X\)服从贝塔分布\(B(a,b)\),
那么\begin{gather*}
	E(X) = \frac{a}{a+b}, \\
	E(X^2) = \frac{a(a+1)}{(a+b)(a+b+1)}, \\
	E(X^3) = \frac{a(a+1)(a+2)}{(a+b)(a+b+1)(a+b+2)}.
\end{gather*}
于是,我们可以列出\cref{table:偏度.例表1}.
\begin{table}[ht]
	\centering
	\begin{tblr}{*4c}
		\hline
		\(X\) & \(B(2,8)\) & \(B(8,2)\) & \(B(5,5)\) \\
		\hline
		\(E(X)\) & \(1/5\) & \(4/5\) & \(1/2\) \\
		\(E(X^2)\) & \(3/55\) & \(36/55\) & \(3/11\) \\
		\(E(X^3)\) & \(1/55\) & \(6/11\) & \(7/44\) \\
		\(\mu_3\) & \(2/(55\times25)\) & \(-2/(55\times25)\) & 0 \\
		\(\sqrt{D(X)}\) & \(2/(5\sqrt{11})\) & \(2/(5\sqrt{11})\) & \(\sqrt{1/44}\) \\
		\(\beta_s\) & \(\sqrt{11}/4\approx0.8292\) & \(-\sqrt{11}/4\approx-0.8292\) & \(0\) \\
		\hline
	\end{tblr}
	\caption{}
	\label{table:偏度.例表1}
\end{table}
可以看出,\(B(2,8)\)是正偏的、右偏的,
\(B(8,2)\)是负偏的、左偏的,
\(B(5,5)\)是不偏的.
\end{solution}
\end{example}
