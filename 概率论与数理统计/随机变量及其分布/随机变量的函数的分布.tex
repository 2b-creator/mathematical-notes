\section{随机变量的函数的分布}
在实际应用中,常常遇到随机变量的函数.
如一个圆柱状工件的半径为\(X\),则工件截面积为\(\pi X^2\).
一般地,随机变量\(X\)的函数\(Y=g(X)\)是一个样本空间到实数域的复合函数,所以\(Y\)也是随机变量.
因此有依据\(X\)的分布求出\(Y\)的分布的问题.
这样的问题在离散的情形下较为简单,在连续的情形下较为复杂.

\subsection{求解离散型随机变量函数的概率分布的基本方法步骤}
设随机变量\(X\)的概率分布为\(p_k = P(X = x_k)\ (k=1,2,\dotsc)\),
则\(Y = g(X)\)的概率分布为\[
P(Y = y_j) = \sum_{g(x_k) = y_j} p_k
\quad(j=1,2,\dotsc).
\]

\begin{example}
设随机变量\(X\)的概率分布为\[
X \sim \begin{pmatrix}
-1 & 0 & 1 & 2 \\
0.2 & 0.3 & 0.3 & 0.2
\end{pmatrix},
\]\(Y=X^2\),\(Z=\frac{X^3+1}{2}\),求\(Y,Z\)的概率分布.
\begin{solution}
由于\(X\in\{-1,0,1,2\}\),所以\(Y\in\{0,1,4\}\),\begin{align*}
P(Y=0) &= P(X=0) = 0.3, \\
P(Y=1) &= P(X=-1 \lor X=1) \\
	&= P(X=-1) + P(X=1)
	= 0.5, \\
P(Y=4) &= P(X=2) = 0.2,
\end{align*}
从而\[
Y \sim \begin{pmatrix}
0 & 1 & 4 \\
0.3 & 0.5 & 0.2
\end{pmatrix}.
\]

同理,\(Z\in\Set{0,\frac{1}{2},1,\frac{9}{2}}\),且\begin{align*}
P(Z=0) &= P(X=-1) = 0.2, \\
P(Z=1/2) &= P(X=0) = 0.3, \\
P(Z=1) &= P(X=1) = 0.3, \\
P(Z=9/2) &= P(X=2) = 0.2,
\end{align*}
从而\[
Z \sim \begin{pmatrix}
0 & \frac{1}{2} & 1 & \frac{9}{2} \\
0.2 & 0.3 & 0.3 & 0.2
\end{pmatrix}.
\]
\end{solution}
\end{example}

\subsection{求解连续型随机变量函数的密度函数的基本方法步骤}
设随机变量\(X\)的密度函数为\(f_X(x)\),
要求\(Y = g(X)\)的密度函数,
\begin{enumerate}
	\item 首先依据\(X\)的取值区间,
	求出\(Y\)的值域\(R(Y)\).

	\item 然后求出\(Y\)的分布函数,
	即对\(\forall y \in R(Y)\),
	有\[
		F_Y(y) = P(Y \leq y)
		= P[g(X) \leq y]
		= P[X \in G(y)]
		= \int_{G(y)} f_X(x) \dd{x},
	\]
	其中\(G(y) = \Set{ x\in\mathbb{R} \given g(x) \leq y }\).
	对\(\forall y \notin R(Y)\),
	有\[
		F_Y(y) = 0
		\quad\text{或}\quad
		F_Y(y) = 1.
	\]

	\item 对\(Y\)的分布函数求导,得\[
		f_Y(y) = \left\{ \begin{array}{cl}
			F_Y'(y), & y \in R(Y), \\
			0, & y \notin R(Y).
		\end{array} \right.
	\]
\end{enumerate}
