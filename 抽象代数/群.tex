\section{群}
\subsection{群的概念}
\begin{definition}
假设在非空集合\(G\)上定义了一个代数运算\(\times\)且它满足:
\begin{enumerate}
    \item 结合律,即\[
        (\forall a \in G)(\forall b \in G)(\forall c \in G)\:
        (a \times b) \times c = a \times (b \times c).
    \]

    \item \(G\)中存在\DefineConcept{单位元},即\[
        (\exists e \in G)(\forall a \in G)\:
        e \times a = a \times e = a.
    \]

    \item 运算\(\times\)可逆,即\[
        (\forall a \in G)(\exists b \in G)\:
        b \times a = a \times b = e.
    \]
\end{enumerate}
那么称“\(G\)是一个\DefineConcept{群}(group)”.
\end{definition}

\begin{definition}
我们把满足\[
    (\forall a \in G)\:
    e \times a = a \times e = a
\]的元素\(e\)称为\(G\)的单位元.
\end{definition}

\begin{definition}
对于任意给定的\(a \in G\),如果\(b\)满足\[
    b \times a = a \times b = e,
\]
则称“\(b\)是\(a\)的\DefineConcept{逆元}”,记作\(a^{-1}\).
\end{definition}

\begin{property}
设\(G\)是一个群.
那么\(G\)的单位元唯一,\(G\)中每个元素\(a\)的逆元唯一,且\[
    (a^{-1})^{-1} = a.
\]
\end{property}

\subsection{交换群}
\begin{definition}
如果群\(G\)的乘法还满足交换律,
那么称“\(G\)是\DefineConcept{交换群}%
或\DefineConcept{阿贝尔群}(Abelian group)”.
\end{definition}
