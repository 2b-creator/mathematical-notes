\section{群}
\subsection{群的概念}
\begin{definition}
假设在非空集合\(G\)上定义了一个代数运算\(\times\)且它满足:
\begin{enumerate}
    \item 结合律,即\[
        (\forall a \in G)(\forall b \in G)(\forall c \in G)
        [(a \times b) \times c = a \times (b \times c)].
    \]

    \item \(G\)中存在\DefineConcept{单位元},即\[
        (\exists e \in G)(\forall a \in G)
        [e \times a = a \times e = a].
    \]

    \item 运算\(\times\)可逆,即\[
        (\forall a \in G)(\exists b \in G)
        [b \times a = a \times b = e].
    \]
\end{enumerate}
那么称“定义了\(\times\)运算的\(G\)集\(\opair{G,\times}\)是一个\DefineConcept{群}(group)”,
或称“集合\(G\)对于\(\times\)运算成群”.
\end{definition}

容易验证,自然数集\(\omega\)、整数集\(\mathbb{Z}\)对于加法、乘法分别成群,即\[
	\opair{\omega,+}, \qquad
	\opair{\omega,\times}, \qquad
	\opair{\mathbb{Z},+}, \qquad
	\opair{\mathbb{Z},\times}
\]都是群.
特别地,仅由一个自然数\(0\)组成的集合\(\Set{0}\)对于加法成群,但对乘法不成群;
而仅由一个自然数\(1\)组成的集合\(\Set{1}\)对于乘法成群,但对加法不成群.

\begin{definition}
设\(\opair{G,\times}\)是群.
我们把满足\[
    (\forall a \in G)
    [e \times a = a \times e = a]
\]的元素\(e\)称为\(\opair{G,\times}\)的单位元.
\end{definition}

\begin{definition}
设\(\opair{G,\times}\)是群.
对于任意给定的\(a \in G\),如果\(b\)满足\[
    b \times a = a \times b = e,
\]
则称“\(b\)是\(a\)的\DefineConcept{逆元}(inverse)”,记作\(a^{-1}\).
\end{definition}

\begin{property}
设\(\opair{G,\times}\)是群.
那么\(\opair{G,\times}\)的单位元唯一,
\(G\)中每个元素\(a\)的逆元唯一,且\[
    (a^{-1})^{-1} = a.
\]
\end{property}

\subsection{交换群}
\begin{definition}
如果群\(\opair{G,\times}\)的乘法还满足交换律,
那么称“\(\opair{G,\times}\)是\DefineConcept{交换群}(commutative group)%
或\DefineConcept{阿贝尔群}(Abelian group)”.
\end{definition}

由于自然数、整数的加法、乘法都满足交换律,所以\[
	\opair{\omega,+}, \qquad
	\opair{\omega,\times}, \qquad
	\opair{\mathbb{Z},+}, \qquad
	\opair{\mathbb{Z},\times}
\]都是交换群.
