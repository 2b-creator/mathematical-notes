\section{群}
\subsection{半群}
\begin{definition}\label{definition:抽象代数.半群的定义}
设\(G\)是非空集合,
\(\times\)是定义在\(G\)上的一个二元代数运算.
如果\(\times\)满足结合律,即\[
	(\forall a,b,c \in G)
	[(a \times b) \times c = a \times (b \times c)],
\]
那么称“定义了\(\times\)运算的\(G\)集\(\opair{G,\times}\)是一个\DefineConcept{半群}(semigroup)”,
或称“集合\(G\)对于\(\times\)运算成半群”.
\end{definition}

\subsection{单位元,幺半群}
下面我们给出左单位元、右单位元的定义,并证明左单位元、右单位元是同一的.
设\(G\)是非空集合,
\(\times\)是定义在\(G\)上的一个二元代数运算.
把满足\[
	e \in G
	\land
	(\forall a \in G)[e \times a = a]
\]的元素\(e\)称为“\(\opair{G,\times}\)的\DefineConcept{左单位元}(left identity)”;
把满足\[
	o \in G
	\land
	(\forall a \in G)[a \times o = a]
\]的元素\(o\)称为“\(\opair{G,\times}\)的\DefineConcept{右单位元}(right identity)”.

假设左单位元、右单位元都存在,
那么\[
	e = e \times o = o,
\]
这就是说\(\opair{G,\times}\)的左单位元就是它的右单位元;
这时候我们把\(\opair{G,\times}\)的左单位元、右单位元合称为%
“\(\opair{G,\times}\)的\DefineConcept{单位元}(identity)”.

\begin{example}\label{example:抽象代数.单位元.例1}
我们可以构造一个对集\(G = \Set{ a, b }\),其中\(a \neq b\).
再定义:\[
	\times = \Set{
		\opair{\opair{a,b},b},
		\opair{\opair{b,a},a},
		\opair{\opair{a,a},a},
		\opair{\opair{b,b},b}
	}.
\]
容易看出\(a,b\)都是\(G\)的左单位元.
由于\begin{align*}
	(a \times a) \times a = a = a \times (a \times a), \\
	(a \times a) \times b = b = a \times (a \times b), \\
	(a \times b) \times a = a = a \times (b \times a), \\
	(b \times a) \times a = a = b \times (a \times a), \\
	(a \times b) \times b = b = a \times (b \times b), \\
	(b \times a) \times b = b = b \times (a \times b), \\
	(b \times b) \times a = a = b \times (b \times a), \\
	(b \times b) \times b = b = b \times (b \times b),
\end{align*}
所以\(\times\)服从结合律.
综上所述,\(\opair{G,\times}\)是一个只有左逆元而没有右逆元的半群.
\end{example}

\begin{example}
我们可以把\cref{example:抽象代数.单位元.例1} 中定义的\(\times\)运算画成一个表格:
\begin{center}
\begin{tabular}{c|cc}
	\(\times\) & \(a\) & \(b\) \\ \hline
	\(a\) & \(a\) & \(b\) \\
	\(b\) & \(a\) & \(b\) \\
\end{tabular}
\end{center}
在阅读时,我们从\(\times\)算符下边任取一个元素,确定一行;
然后再从\(\times\)算符右边任取一个元素,确定一列;
行列相交之处的元素,就是上面我们取得的两个元素的运算结果.
例如,我们先取\(a\),这一行对过来的可能结果有\(a\)、\(b\)两个;
然后取\(b\),这一列对下去的可能结果只有重复的两个\(b\);
那么相交之处就是\(b\),也就是说\(a \times b = b\).
\end{example}

\begin{definition}\label{definition:抽象代数.幺半群的定义}
设\(\opair{G,\times}\)是半群,
且\(G\)中存在单位元,即\[
	(\exists e \in G)(\forall a \in G)
	[e \times a = a \times e = a].
\]
那么称“定义了\(\times\)运算的\(G\)集\(\opair{G,\times}\)是一个\DefineConcept{幺半群}(monoid)”,
或称“集合\(G\)对于\(\times\)运算成幺半群”.
\end{definition}

\begin{theorem}\label{theorem:抽象代数.群内单位元唯一}
幺半群的单位元唯一.
\begin{proof}
假设\(e_1\)和\(e_2\)都是幺半群\(\opair{G,\times}\)的单位元.
根据\hyperref[definition:抽象代数.幺半群的定义]{单位元的定义}有\[
	(\forall a \in G)[
		e_1 = e_1 \times e_2 = e_2 \times e_1 = e_2
	],
\]
也就是说,幺半群\(\opair{G,\times}\)的单位元是唯一的.
\end{proof}
\end{theorem}

\subsection{逆元,群}
下面我们给出左逆元、右逆元的定义,并证明幺半群中的左逆元与右逆元是同一的.
设幺半群\(\opair{G,\times}\)的单位元是\(e\),任意取定\(a \in G\).
把满足\[
	b \in G
	\land
	b \times a = e,
\]的元素\(b\)称为“\(a\)(在\(G\)中)的\DefineConcept{左逆元}(left inverse)”;
把满足\[
	c \in G
	\land
	a \times c = e,
\]的元素\(c\)称为“\(a\)(在\(G\)中)的\DefineConcept{右逆元}(right inverse)”.
因为\(\times\)服从结合律,再根据单位元的定义,可知\[
	b = b \times e = b \times (a \times c) = (b \times a) \times c = e \times c = c,
\]
这就是说\(a\)的左逆元就是它的右逆元,
于是我们可以将\(a\)的左逆元、右逆元统称为\DefineConcept{逆元}(inverse),记作\(a^{-1}\).
%同一个元素在不同集合中、不同运算下可能有不同的逆元

\begin{definition}\label{definition:抽象代数.群的定义}
设\(\opair{G,\times}\)是幺半群,
且运算\(\times\)可逆,即\[
	(\forall a \in G)(\exists b \in G)
	[b \times a = a \times b = e].
\]
那么称“定义了\(\times\)运算的\(G\)集\(\opair{G,\times}\)是一个\DefineConcept{群}(group)”,
或称“集合\(G\)对于\(\times\)运算成群”.
\end{definition}

\begin{theorem}\label{theorem:抽象代数.群内任一元的逆元唯一}
群内任一元素的逆元唯一.
\begin{proof}
假设\(\opair{G,\times}\)是群,\(e\)是它的单位元.
任意取定\(a \in G\).
又假设\(b\)和\(c\)都是\(a\)的逆元.
根据\hyperref[definition:抽象代数.群的定义]{逆元的定义},
有\[
    a \times b = b \times a = e,
\]\[
    a \times c = c \times a = e.
\]
再根据\hyperref[definition:抽象代数.幺半群的定义]{单位元的定义}和结合律,有\begin{align*}
    b &= b \times e
	= b \times (a \times c) \\
    &= (b \times a) \times c \\
    &= e \times c
	= c.
	\qedhere
\end{align*}
\end{proof}
\end{theorem}

由\cref{equation:集合论.自然数加法结合律,equation:集合论.自然数乘法结合律}
以及\cref{equation:集合论.整数加法结合律,equation:集合论.整数乘法结合律},
自然数集\(\omega\)、整数集\(\mathbb{Z}\)对于加法、乘法分别成半群,即\[
	\opair{\omega,+}, \qquad
	\opair{\omega,\times}, \qquad
	\opair{\mathbb{Z},+}, \qquad
	\opair{\mathbb{Z},\times}
\]都是半群.

由于\(0\)是自然数加法、整数加法的单位元,
而\(1\)是自然数乘法、整数乘法的单位元,
所以自然数集\(\omega\)、整数集\(\mathbb{Z}\)对于加法、乘法又分别成幺半群.

由\cref{equation:集合论.整数加法可逆} 可知,
整数集\(\mathbb{Z}\)对于加法成群.
但是,整数集\(\mathbb{Z}\)对于乘法不成群.
同样地,自然数集\(\omega\)对于加法、乘法不成群.

特别地,仅由一个自然数\(0\)组成的集合\(\Set{0}\)对于加法成群,但对乘法不成群;
而仅由一个自然数\(1\)组成的集合\(\Set{1}\)对于乘法成群,但对加法不成群.

一个群的单位元既与构成这个群的集合有关,又与定义在这个集合上的运算有关.
在同一个非空集合上定义的不同的代数运算对应的单位元有可能不同.
例如,对于\(\opair{\omega,+}\),自然数\(0\)是它的单位元;
然而,对于\(\opair{\omega,\times}\),自然数\(1\)是它的单位元.
又例如,对于\(\opair{\mathbb{Z},+}\)和\(\opair{\mathbb{Z},\times}\),
整数\(0\)、整数\(1\)分别是它们的单位元.

\subsection{交换群}
\begin{definition}
设\(G\)是非空集合,
\(\times\)是定义在\(G\)上的一个二元代数运算,
\(\times\)服从交换律,即\[
	(\forall a,b \in G)[a \times b = b \times a].
\]
那么\begin{enumerate}
	\item 如果\(\opair{G,\times}\)是半群,
	那么称“\(\opair{G,\times}\)是\DefineConcept{交换半群}(commutative semigroup)”.
	\item 如果\(\opair{G,\times}\)是群,
	那么称“\(\opair{G,\times}\)是\DefineConcept{交换群}(commutative group)%
	或\DefineConcept{阿贝尔群}(Abelian group)”.
\end{enumerate}
\end{definition}

由于自然数、整数的加法、乘法都满足交换律,所以\[
	\opair{\omega,+}, \qquad
	\opair{\omega,\times}, \qquad
	\opair{\mathbb{Z},+}, \qquad
	\opair{\mathbb{Z},\times}
\]都是交换群.

\begin{theorem}
设\(\opair{G,\times}\)是群.
那么\(n\)个元素\(\v{a}{n}\)的运算结果\[
	a_1 \times a_2 \times \dotsb \times a_n
\]由它们自身以及它们的顺序唯一确定.
\end{theorem}

\begin{corollary}
设\(\opair{G,\times}\)是交换群.
那么\(n\)个元素\(\v{a}{n}\)的运算结果\[
	a_1 \times a_2 \times \dotsb \times a_n
\]由它们自身唯一确定.
\end{corollary}
