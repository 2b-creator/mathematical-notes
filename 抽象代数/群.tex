\section{群}
\subsection{群的概念}
\begin{definition}
假设在非空集合\(G\)上定义了一个代数运算\(\times\),
且它满足结合律,即\[
	(\forall a,b,c \in G)
	[(a \times b) \times c = a \times (b \times c)].
\]
那么称“定义了\(\times\)运算的\(G\)集\(\opair{G,\times}\)是一个\DefineConcept{半群}(semigroup)”,
或称“集合\(G\)对于\(\times\)运算成半群”.
\end{definition}

\begin{definition}
设\(\opair{G,\times}\)是半群,
且\(G\)中存在\DefineConcept{单位元},即\[
	(\exists e \in G)(\forall a \in G)
	[e \times a = a \times e = a].
\]
那么称“定义了\(\times\)运算的\(G\)集\(\opair{G,\times}\)是一个\DefineConcept{幺半群}(monoid)”,
或称“集合\(G\)对于\(\times\)运算成幺半群”.
\end{definition}

\begin{definition}
设\(\opair{G,\times}\)是幺半群,
且运算\(\times\)可逆,即\[
	(\forall a \in G)(\exists b \in G)
	[b \times a = a \times b = e].
\]
那么称“定义了\(\times\)运算的\(G\)集\(\opair{G,\times}\)是一个\DefineConcept{群}(group)”,
或称“集合\(G\)对于\(\times\)运算成群”.
\end{definition}

由\cref{equation:集合论.自然数加法结合律,equation:集合论.自然数乘法结合律}
以及\cref{equation:集合论.整数加法结合律,equation:集合论.整数乘法结合律},
自然数集\(\omega\)、整数集\(\mathbb{Z}\)对于加法、乘法分别成半群,即\[
	\opair{\omega,+}, \qquad
	\opair{\omega,\times}, \qquad
	\opair{\mathbb{Z},+}, \qquad
	\opair{\mathbb{Z},\times}
\]都是半群.

由于\(0\)是自然数加法、整数加法的单位元,
而\(1\)是自然数乘法、整数乘法的单位元,
所以自然数集\(\omega\)、整数集\(\mathbb{Z}\)对于加法、乘法又分别成幺半群.

由\cref{equation:集合论.整数加法可逆} 可知,
整数集\(\mathbb{Z}\)对于加法成群.

特别地,仅由一个自然数\(0\)组成的集合\(\Set{0}\)对于加法成群,但对乘法不成群;
而仅由一个自然数\(1\)组成的集合\(\Set{1}\)对于乘法成群,但对加法不成群.

\begin{definition}
设\(\opair{G,\times}\)是群.
我们把满足\[
    (\forall a \in G)
    [e \times a = a \times e = a]
\]的元素\(e\)称为\(\opair{G,\times}\)的单位元.
\end{definition}

一个群的单位元既与构成这个群的集合有关,又与定义在这个集合上的运算有关.
在同一个非空集合上定义的不同的代数运算对应的单位元有可能不同.
例如,对于\(\opair{\omega,+}\),自然数\(0\)是它的单位元;
然而,对于\(\opair{\omega,\times}\),自然数\(1\)是它的单位元.
又例如,对于\(\opair{\mathbb{Z},+}\)和\(\opair{\mathbb{Z},\times}\),
整数\(0\)、整数\(1\)分别是它们的单位元.

\begin{definition}
设\(\opair{G,\times}\)是群.
对于任意给定的\(a \in G\),如果\(b\)满足\[
    b \times a = a \times b = e,
\]
则称“\(b\)是\(a\)的\DefineConcept{逆元}(inverse)”,记作\(a^{-1}\).
\end{definition}

\begin{property}
设\(\opair{G,\times}\)是群.
那么\(\opair{G,\times}\)的单位元唯一,
\(G\)中每个元素\(a\)的逆元唯一,且\[
    (a^{-1})^{-1} = a.
\]
\end{property}

\begin{definition}
如果半群\(\opair{G,\times}\)的乘法还满足交换律,
那么称“\(\opair{G,\times}\)是\DefineConcept{交换半群}(commutative semigroup).
\end{definition}

\begin{definition}
如果群\(\opair{G,\times}\)的乘法还满足交换律,
那么称“\(\opair{G,\times}\)是\DefineConcept{交换群}(commutative group)%
或\DefineConcept{阿贝尔群}(Abelian group)”.
\end{definition}

由于自然数、整数的加法、乘法都满足交换律,所以\[
	\opair{\omega,+}, \qquad
	\opair{\omega,\times}, \qquad
	\opair{\mathbb{Z},+}, \qquad
	\opair{\mathbb{Z},\times}
\]都是交换群.
