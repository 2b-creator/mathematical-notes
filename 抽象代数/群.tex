\section{群}
\subsection{群的概念}
\begin{definition}\label{definition:抽象代数.半群的定义}
假设在非空集合\(G\)上定义了一个代数运算\(\times\),
且它满足结合律,即\[
	(\forall a,b,c \in G)
	[(a \times b) \times c = a \times (b \times c)].
\]
那么称“定义了\(\times\)运算的\(G\)集\(\opair{G,\times}\)是一个\DefineConcept{半群}(semigroup)”,
或称“集合\(G\)对于\(\times\)运算成半群”.
\end{definition}

\begin{definition}\label{definition:抽象代数.幺半群的定义}
设\(\opair{G,\times}\)是半群,
且\(G\)中存在“单位元”,即\[
	(\exists e \in G)(\forall a \in G)
	[e \times a = a \times e = a].
\]
那么称“定义了\(\times\)运算的\(G\)集\(\opair{G,\times}\)是一个\DefineConcept{幺半群}(monoid)”,
或称“集合\(G\)对于\(\times\)运算成幺半群”.

这里我们把满足\[
	e \in G
	\land
    (\forall a \in G)
    [e \times a = a \times e = a]
\]的元素\(e\)称为“\(\opair{G,\times}\)的\DefineConcept{单位元}(identity)”.
\end{definition}

\begin{definition}\label{definition:抽象代数.群的定义}
设\(\opair{G,\times}\)是幺半群,
且运算\(\times\)可逆,即\[
	(\forall a \in G)(\exists b \in G)
	[b \times a = a \times b = e].
\]
那么称“定义了\(\times\)运算的\(G\)集\(\opair{G,\times}\)是一个\DefineConcept{群}(group)”,
或称“集合\(G\)对于\(\times\)运算成群”.

这里我们把满足\[
	b \in G
	\land
	(\forall a \in G)
    [b \times a = a \times b = e],
\]的元素\(b\)称为%
“\(a\)(在\(G\)中)的\DefineConcept{逆元}(inverse)”,%同一个元素在不同集合中、不同运算下可能有不同的逆元
记作\(a^{-1}\).
\end{definition}

由\cref{equation:集合论.自然数加法结合律,equation:集合论.自然数乘法结合律}
以及\cref{equation:集合论.整数加法结合律,equation:集合论.整数乘法结合律},
自然数集\(\omega\)、整数集\(\mathbb{Z}\)对于加法、乘法分别成半群,即\[
	\opair{\omega,+}, \qquad
	\opair{\omega,\times}, \qquad
	\opair{\mathbb{Z},+}, \qquad
	\opair{\mathbb{Z},\times}
\]都是半群.

由于\(0\)是自然数加法、整数加法的单位元,
而\(1\)是自然数乘法、整数乘法的单位元,
所以自然数集\(\omega\)、整数集\(\mathbb{Z}\)对于加法、乘法又分别成幺半群.

由\cref{equation:集合论.整数加法可逆} 可知,
整数集\(\mathbb{Z}\)对于加法成群.
但是,整数集\(\mathbb{Z}\)对于乘法不成群.
同样地,自然数集\(\omega\)对于加法、乘法不成群.

特别地,仅由一个自然数\(0\)组成的集合\(\Set{0}\)对于加法成群,但对乘法不成群;
而仅由一个自然数\(1\)组成的集合\(\Set{1}\)对于乘法成群,但对加法不成群.

一个群的单位元既与构成这个群的集合有关,又与定义在这个集合上的运算有关.
在同一个非空集合上定义的不同的代数运算对应的单位元有可能不同.
例如,对于\(\opair{\omega,+}\),自然数\(0\)是它的单位元;
然而,对于\(\opair{\omega,\times}\),自然数\(1\)是它的单位元.
又例如,对于\(\opair{\mathbb{Z},+}\)和\(\opair{\mathbb{Z},\times}\),
整数\(0\)、整数\(1\)分别是它们的单位元.

\begin{property}
设\(\opair{G,\times}\)是群.
那么\(\opair{G,\times}\)的单位元唯一,
\(G\)中每个元素\(a\)的逆元唯一,且\[
    (a^{-1})^{-1} = a.
\]
\end{property}

\begin{definition}
如果半群\(\opair{G,\times}\)的乘法还满足交换律,
那么称“\(\opair{G,\times}\)是\DefineConcept{交换半群}(commutative semigroup).
\end{definition}

\begin{definition}
如果群\(\opair{G,\times}\)的乘法还满足交换律,
那么称“\(\opair{G,\times}\)是\DefineConcept{交换群}(commutative group)%
或\DefineConcept{阿贝尔群}(Abelian group)”.
\end{definition}

由于自然数、整数的加法、乘法都满足交换律,所以\[
	\opair{\omega,+}, \qquad
	\opair{\omega,\times}, \qquad
	\opair{\mathbb{Z},+}, \qquad
	\opair{\mathbb{Z},\times}
\]都是交换群.

\subsection{群的性质}
\begin{theorem}\label{theorem:抽象代数.群内单位元唯一}
群的单位元唯一.
\begin{proof}
假设\(e_1\)和\(e_2\)都是群\(\opair{G,\times}\)的单位元.
根据\hyperref[definition:抽象代数.幺半群的定义]{单位元的定义}有\[
	(\forall a \in G)[
		e_1 = e_1 \times e_2 = e_2 \times e_1 = e_2
	],
\]
也就是说,群\(\opair{G,\times}\)的单位元是唯一的.
\end{proof}
\end{theorem}

\begin{theorem}\label{theorem:抽象代数.群内任一元的逆元唯一}
群内任一元素的逆元唯一.
\begin{proof}
假设\(\opair{G,\times}\)是群,\(e\)是它的单位元.
任意取定\(a \in G\).
又假设\(b\)和\(c\)都是\(a\)的逆元.
根据\hyperref[definition:抽象代数.群的定义]{逆元的定义},
有\[
    a \times b = b \times a = e,
\]\[
    a \times c = c \times a = e.
\]
再根据\hyperref[definition:抽象代数.幺半群的定义]{单位元的定义}和结合律,有\begin{align*}
    b &= b \times e
	= b \times (a \times c) \\
    &= (b \times a) \times c \\
    &= e \times c
	= c.
	\qedhere
\end{align*}
\end{proof}
\end{theorem}

\begin{theorem}
设\(\opair{G,\times}\)是群.
那么\(n\)个元素\(\v{a}{n}\)的运算结果\[
	a_1 \times a_2 \times \dotsb \times a_n
\]由它们自身以及它们的顺序唯一确定.
\end{theorem}

\begin{corollary}
设\(\opair{G,\times}\)是交换群.
那么\(n\)个元素\(\v{a}{n}\)的运算结果\[
	a_1 \times a_2 \times \dotsb \times a_n
\]由它们自身唯一确定.
\end{corollary}
