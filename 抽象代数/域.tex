
\section{域}
\begin{definition}
设\(F\)是一个有单位元\(e(\neq o)\)的交换环,
如果\(F\)中每个非零元都是可逆元,
那么称\(F\)是一个\DefineConcept{域}(field).
\end{definition}

%\section{数域}
%\begin{definition}
%一个数集\(K\)如果满足\begin{enumerate}
%\item \(0,1 \in K\);
%\item \(K\)对于加、减、乘、除这四种运算封闭,即
%\(\forall a,b \in K :
%a \pm b,
%a \cdot b \in K,
%\bigl( b \neq 0 \implies a/b \in K \bigr)
%\);
%\end{enumerate}
%则称\(K\)是一个\DefineConcept{数域}.
%\end{definition}
%容易验证,有理数集\(\mathbb{Q}\)、实数集\(\mathbb{R}\)和复数集\(\mathbb{C}\)都是数域.
%因此我们可以把\(\mathbb{Q}\)、\(\mathbb{R}\)、\(\mathbb{C}\)分别称为有理数域、实数域、复数域.
%
%因为整数集\(\mathbb{Z}\)对除法运算不封闭,
%即\(\exists a, b \bigl( b \neq 0 \land a/b \notin \mathbb{Z} \bigr)\),
%所以整数集不是数域.
%
%除了\(\mathbb{Q}\)、\(\mathbb{R}\)、\(\mathbb{C}\)以外,
%还有很多数域.
%例如,令{\def\Q{\mathbb{Q}(\sqrt{2})}
%\[
%    \Q
%    \defeq
%    \Set{ a + b \sqrt{2} \given a,b \in \mathbb{Q} }.
%\]
%显然\(0=0+0\times\sqrt{2}\in\Q\),
%\(1=1+0\times\sqrt{2}\in\Q\),
%并且容易验证\(\Q\)对于加、减、乘、除这四种运算封闭,
%因此,\(\Q\)是一个数域.
%}
%
%\begin{theorem}
%有理数域是最小的数域,即任意数域都包含有理数域.
%\begin{proof}
%设\(K\)是一个数域,则\(0,1 \in K\),从而
%\[
%    2 = 1 + 1 \in K,
%    3 = 2 + 1 \in K,
%    \dotsc,
%    n = (n-1) + 1 \in K.
%\]
%这就是说,任一正整数\(n \in K\).
%又由于\(-n = 0 - n \in K\),%
%因此任一负整数\(-n \in K\).
%由上可知,\(\mathbb{Z} \subseteq K\).
%于是,任一分数\[
%\frac{a}{b} \in K \quad(a,b\in\mathbb{Z} \land b\neq0).
%\]
%也就是说,\(\mathbb{Q} \subseteq K\).
%\end{proof}
%\end{theorem}
%
%\begin{theorem}
%复数域是最大的数域,即任意数域都包含于复数域.
%\end{theorem}

% TODO 什么是有序域?什么是无序域?
% 复数域是无序域.
% 有理数域、实数域是有序域.
