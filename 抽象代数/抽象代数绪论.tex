\chapter{抽象代数绪论}
\section{环}
\begin{definition}
在集合\(S\)上的一个\textbf{二元代数运算}是指\(S^2\)到\(S\)的一个映射.
\end{definition}

\begin{definition}
设\(R\)是一个非空集合,如果\(R\)上定义了两个代数运算,一个叫做加法(记作\(+\)),另一个叫做乘法(记作\(\cdot\)),并且满足下列6条运算法则\begin{enumerate}
\item 加法交换律,即\[
\forall a,b \in R \bigl(a+b = b+a\bigr);
\]
\item 加法结合律,即\[
\forall a,b,c \in R \Bigl[(a+b)+c = a+(b+c)\Bigr];
\]
\item \(R\)中存在一个元素\(o\),它满足\[
\forall a \in R \bigl( a+o = o+a = a \bigr),%
\]称其为\(R\)的\textbf{零元};
\item 任给\(a \in R\),都有\(b \in R\),使得\[
a+b = b+a = o,
\]把\(b\)称为\(a\)的\textbf{负元},记作\((-a)\);
\item 乘法结合律,即\[
\forall a,b,c \in R \bigl[ (a \cdot b) \cdot c = a \cdot (b \cdot c) \bigr];
\]
\item 左分配律,即\[
\forall a,b,c \in R \bigl[ a \cdot (b+c) = a \cdot b + a \cdot c \bigr],
\]和右分配律,即\[
\forall a,b,c \in R \bigl[ (b+c) \cdot a = b \cdot a + c \cdot a \bigr],
\]
\end{enumerate}那么称\(R\)是一个\textbf{环}(ring).

在环上可以定义二元运算\textbf{减法}如下:\[
a - b \mapsto a + (-b).
\]
\end{definition}

\begin{property}
任一环的零元唯一.
\begin{proof}
设环\(R\)中有两个零元\(o_1\)和\(o_2\).
由环的定义,有\[
o_1 = o_1 + o_2 = o_2 + o_1 = o_2.
\]也就是说,环\(R\)的零元是唯一的.
\end{proof}
\end{property}

\begin{property}
零元的负元是它本身.
\begin{proof}
设环\(R\)中的零元\(o\)的负元是\(x\).
由环的定义,有\[
x + o = o + x = x,
\]\[
x + o = o + x = o.
\]比较可得\[
x = o.
\]这就说明,零元的负元总是它本身.
\end{proof}
\end{property}

\begin{property}
两元素各自负元的和等于两元素和的负元.
\begin{proof}
从环\(R\)中任取两元素\(a\)和\(b\).
由加法交换律和加法结合律,得\[
[(-a) + (-b)] + [a + b]
= [(-a) + a] + [(-b) + b]
= o + o = o,
\]也就是说\((a+b)\)是\((-a) + (-b)\)的负元.
\end{proof}
\end{property}

\begin{property}
任一环中元素的负元唯一.
\begin{proof}
从环\(R\)中任取一元素\(a\),假设元素\(b\)与元素\(c\)都是\(a\)的负元.
由环的定义,有\[
b + a = o,
\]\[
c + a = o.
\]而\((-c) + (-a) = -(c + a) = -o = o\),那么\[
(b + a) + [-(c + a)]
= o + o = o.
\]又因为\[
(b + a) + [-(c + a)]
= (b + a) + [(-c) + (-a)]
= [b + (-c)] + [a + (-a)]
= b + (-c),
\]所以\[
b + (-c) = o,
\]\[
[b + (-c)] + c = b + [(-c) + c] = b = o + c = c,
\]也就是说,元素\(a\)的负元是唯一的.
\end{proof}
\end{property}

\begin{definition}
如果环\(R\)的乘法满足交换律,即\[
\forall a,b \in R \big( a \cdot b = b \cdot a \bigr),
\]那么称\(R\)是\textbf{交换环}.
\end{definition}

\begin{definition}
如果环\(R\)中有一个元素\(e\)满足\[
\forall a \in R \bigl( e \cdot a = a \cdot e = a \bigr),
\]那么称\(e\)是\(R\)的\textbf{单位元}(或\textbf{幺元}).
\end{definition}

\begin{property}
具有单位元的环的单位元唯一.
\begin{proof}
设环\(R\)具有单位元\(e_1\)和\(e_2\).
根据单位元的定义,对\(\forall a \in R\),有\[
e_1 = e_1 \cdot e_2 = e_2 \cdot e_1 = e_2,
\]也就是说,\(R\)的单位元是唯一的.
\end{proof}
\end{property}

\begin{theorem}[乘零定理]
在环\(R\)中,零元\(o\)满足\[
\forall a \in R \bigl( o \cdot a = a \cdot o = o \bigr).
\]
\begin{proof}
需要利用沟通加法和乘法的桥梁:
乘法对于加法的左、右分配律,即\[
o \cdot a = (o + o) \cdot a = o \cdot a + o \cdot a.
\]在上式等号两边加上\(o \cdot a\)的负元\(-o \cdot a\),得\[
o \cdot a + (- o \cdot a) = (o \cdot a + o \cdot a) + (- o \cdot a),
\]\[
o = o \cdot a.
\]同理,可证\(a \cdot o = o\).
\end{proof}
\end{theorem}

\begin{example}
设\(R\)是有单位元\(e(\neq o)\)的环.
证明:单位元的负元\((-e)\)满足\[
(-e)^2=(-e)\cdot(-e)=e.
\]
\begin{proof}
根据单位元的定义有\[
e \cdot (-e) = (-e) \cdot e = -e.
\]又由负元的定义有\[
e + (-e) = o,
\]在等号左右两边同时右乘\((-e)\),再由右分配律有\[
-e + (-e)^2 = e \cdot (-e) + (-e) \cdot (-e) = o \cdot (-e) = o,
\]移项得\((-e)^2 = e\).
\end{proof}
\end{example}

\begin{definition}
设\(R\)是有单位元\(e(\neq o)\)的环.
对于\(a \in R\),如果\[
\exists b \in R \bigl( a \cdot b = b \cdot a = e \bigr),
\]那么称\(a\)是一个\textbf{可逆元}(或\textbf{单位}),把\(b\)叫做\(a\)的\textbf{逆元},记作\(a^{-1}\).
\end{definition}

\begin{property}
若\(a\)是可逆元,则\(a\)的逆元唯一.
\begin{proof}
设环\(R\)具有单位元\(e(\neq o)\).
假设环\(R\)中的任一可逆元\(a\)的逆元为\(b\)和\(c\).由可逆元的定义,有\[
a \cdot b = b \cdot a = e,
\]\[
a \cdot c = c \cdot a = e.
\]又由单位元的定义,有\[
b = b \cdot e
= b \cdot (a \cdot c)
= (b \cdot a) \cdot c
= e \cdot c
= c,
\]这就说明,任一可逆元的逆元是唯一的.
\end{proof}
\end{property}

\begin{definition}
设\(R\)是一个环.对于\(a \in R\),%
如果存在\(c \in R\)且\(c \neq o\),%
使得\(a \cdot c = o\)(或\(c \cdot a = o\)),%
那么称\(a\)是一个\textbf{左零因子}(或\textbf{右零因子}).
左、右零因子统称为\textbf{零因子}.
\end{definition}

\begin{property}
在环\(R\)中,零元\(o\)是一个零因子.
\end{property}

\begin{theorem}
设\(R\)是有单位元\(e(\neq o)\)的环,则\(R\)的零因子不是可逆元.
\begin{proof}
设\(a\)是左零因子,则存在\(c \in R\)且\(c \neq o\),使得\[
a \cdot c = o.
\]

假如\(a\)是可逆元,则在上式两边左乘\(a^{-1}\),再根据乘零定理,得\[
a^{-1} \cdot (a \cdot c) = a^{-1} \cdot o = o.
\]根据乘法的结合律和逆元的性质,得\[
a^{-1} \cdot (a \cdot c) = (a^{-1} \cdot a) \cdot c = e \cdot c = c.
\]于是\(c = o\),矛盾,故左零因子\(a\)不是可逆元.同理可得,右零因子不是可逆元.
\end{proof}
\end{theorem}
上述定理的逆否命题是:设\(R\)是有单位元\(e(\neq o)\)的环,则\(R\)的可逆元不是零因子.

\begin{example}
要想让可逆元等于零元,唯一的办法是在由一个元素\(o\)构成的集合\(R\)上定义加法和乘法如下:\[
o + o \mapsto o,
\qquad
o \cdot o \mapsto o.
\]显然元素\(o\)是环\(R\)的单位元和零元.再将原本的可逆元的定义中对环\(R\)的单位元\(e \neq o\)的约束去掉,那么对于环\(R\)上唯一的元素\(o\),有\[
\forall x \in R \bigl( x = o \land x \cdot o = o \cdot x = o \bigr),
\]即元素\(o\)还是环\(R\)的“可逆元”,它的“逆元”是\(o\)本身.

这就说明,在上面对可逆元的定义中,我们必须强调环\(R\)上的单位元\(e\)不是零元.
\end{example}

\section{域}
\begin{definition}
设\(F\)是一个有单位元\(e(\neq o)\)的交换环,%
如果\(F\)中每个非零元都是可逆元,%
那么称\(F\)是一个\textbf{域}(field).
\end{definition}

\begin{definition}
一个数集\(K\)如果满足\begin{enumerate}
\item \(0,1 \in K\);
\item \(\forall a,b \in K : a \pm b, ab \in K\);
\item \(\forall a \in K, \forall b \in K - \{0\} : \frac{a}{b} \in K\),%
\end{enumerate}
则称\(K\)是一个\textbf{数域}.
\end{definition}
容易验证,有理数集\(\mathbb{Q}\)、实数集\(\mathbb{R}\)和复数集\(\mathbb{C}\)都是数域.

\begin{theorem}
有理数域是最小的数域,即任意数域都包含有理数域;
而复数域是最大的数域,即任意数域都包含于复数域.
\end{theorem}

\section{群}
\begin{definition}
设\(G\)是一个非空集合.如果在\(G\)上定义了一个代数运算\(\cdot\),该运算满足
\begin{enumerate}
\item 结合律,即\[
\forall a,b,c \in G \bigl[ (a \cdot b) \cdot c = a \cdot (b \cdot c) \bigr];
\]
\item \(G\)中有一个元素\(e\),使得\[
\forall a \in G \bigl( e \cdot a = a \cdot e = a \bigr),
\]称\(e\)是\(G\)的\textbf{单位元};
\item 对于\(G\)中每个元素\(a\),存在\(b \in G\),使得\[
b \cdot a = a \cdot b = e,
\]称\(a\)\textbf{可逆},把\(b\)称为\(a\)的\textbf{逆元},记作\(a^{-1}\),%
\end{enumerate}
那么称\(G\)是一个\textbf{群}(group).
\end{definition}

\begin{property}
群\(G\)的单位元唯一,\(G\)中每个元素\(a\)的逆元唯一,且\[
(a^{-1})^{-1} = a.
\]
\end{property}

\begin{definition}
如果群\(G\)的乘法还满足交换律,那么称\(G\)是\textbf{交换群}(或\textbf{Abel 群}).
\end{definition}

\begin{theorem}
设\(R\)是非空集合,如果\(R\)上定义了两种二元运算\(+\)和\(\cdot\),且满足\begin{enumerate}
\item \((R,+)\)是Abel群,\((R,\cdot)\)是半群;
\item 对任意\(a,b,c \in R\),对\(+\)和\(\cdot\)有分配律成立,即\[
a \cdot (b + c) = a \cdot b + a \cdot c,
\]\[
(b + c) \cdot a = b \cdot a + c \cdot a,
\]
\end{enumerate}则称\(R\)是环.
\end{theorem}
