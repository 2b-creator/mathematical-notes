\section{子群}
\subsection{子群}
在介绍子群的概念之前,让我们首先回顾\hyperref[definition:集合论.二元代数运算]{二元代数运算的定义}:
任意一个定义在集合\(A\)上的运算,本质上是一个从\(A \times A\)到\(A\)的映射.
现在我们来讨论更一般的情形,
假设我们有一个映射\(f\colon A \times A \to B\),其中\(B \supseteq A\),
如果\[
	(\forall x,y \in A)[f(x,y) \in A],
\]
那么实际上这个映射的值域为\(\ran f = A\);
在这种情况下,我们就说“集合\(A\)对运算\(f\) \DefineConcept{封闭}%
(set \(A\) is \emph{closed} under operation \(f\))”.
回过头来,我们可以看出,二元代数运算的定义保证了它的封闭性.
但是,假如我们从\(A\)中取出一个非空子集\(C\),
我们能否断言\(f\)在直积\(C \times C\)上的限制具有封闭性呢?

\begin{lemma}\label{theorem:抽象代数.群的运算在子集上的限制的不变性}
设\(\times\)是定义在\(G\)上的一个二元代数运算,
\(\emptyset \neq H \subseteq G\),
\(\otimes = (\times \upharpoonright(H \times H))\),
则\[
	(\forall a,b \in H)[a \otimes b = a \times b].
\]
\begin{proof}
根据限制的定义可知\[
	(\forall a,b,c)[
		\opair{\opair{a,b},c} \in \otimes
		\iff
		\opair{\opair{a,b},c} \in \times
		\land
		\opair{a,b} \in H \times H
	],
\]
于是\((\forall a,b \in H)[a \otimes b = a \times b]\).
\end{proof}
\end{lemma}
从\cref{theorem:抽象代数.群的运算在子集上的限制的不变性} 我们可以看出,
运算\(\times\)的限制\(\otimes\)在新的定义域\(H \times H\)上仍然具有和原本的运算\(\times\)一样的性质.
如果\(\times\)在\(G \times G\)上服从结合律,那么\(\otimes\)在\(H \times H\)上也服从结合律.
如果\(\times\)在\(G \times G\)上服从交换律,那么\(\otimes\)在\(H \times H\)上也服从交换律.

但是应该注意到,映射的限制只是限定了它的定义域,并没有限定它的值域,
因此我们可能会遇到值\(c\)落在\(H\)之外的情形,
也就是说\(a \otimes b \in(G - H)\)是可能的,
\(a \otimes b \in H\)不一定成立.

\begin{definition}\label{definition:抽象代数.子群的定义}
%@see: 《近世代数》(丘维声,2015) P40. 定义1
%@see: https://mathworld.wolfram.com/Subgroup.html
设\(\opair{G,\times}\)是群,
\(H\)是\(G\)的一个非空子集.
又设映射\(\otimes\)是\(\times\)在直积\(H \times H\)上的限制,
即\(\otimes = (\times \upharpoonright(H \times H))\).
如果\(H\)对于运算\(\otimes\)也成为一个群\(\opair{H,\otimes}\),
那么称“\(\opair{H,\otimes}\)是\(\opair{G,\times}\)的一个\DefineConcept{子群}(subgroup)”,
记作\(\opair{H,\otimes} \subseteq \opair{G,\times}\).
\end{definition}

一方面,仅由群\(\opair{G,\times}\)的单位元\(e\)%
组成的集合\(E=\{e\}\)对于运算\[
	\odot=(\times\upharpoonright(E\times E))
	= \Set{\opair{\opair{e,e},e}}
\]
也成为一个群\(\opair{E,\odot}\),
那么它就是\(\opair{G,\times}\)的一个子群.
在这个群中,虽然只有一个元素\(e\),但是它却满足\[
	(\forall a \in E)[a \times e = e \times a = e],
\]
于是\(e\)是\(\opair{E,\odot}\)的单位元.
因此,我们可以说群\(\opair{G,\times}\)和子群\(\opair{E,\odot}\)的单位元相同.

另一方面,\(\opair{G,\times}\)本身也是\(\opair{G,\times}\)的一个子群,
即\(\opair{G,\times}\subseteq\opair{G,\times}\);
同样地,群\(\opair{G,\times}\)和子群\(\opair{G,\times}\)的单位元相同.

我们不禁好奇,任意给定一个群,再从中任取一个子群,
是不是这个群的单位元与它的子群的单位元总是相同的?

\begin{theorem}\label{theorem:抽象代数.子群.群的单位元与其子群的单位元相同}
设\(\opair{G,\times}\)是群,
\(\opair{H,\otimes}\subseteq\opair{G,\times}\).
那么\(\opair{G,\times}\)的单位元与\(\opair{H,\otimes}\)的单位元相同.
\begin{proof}
设\(e'\)是\(\opair{H,\otimes}\)的单位元,
则\[
	e' \otimes e' = e'.
\]
根据子群的定义,\(\otimes = (\times \upharpoonright(H \times H))\);
再根据\cref{theorem:抽象代数.群的运算在子集上的限制的不变性},
\((\forall a,b \in H)[a \otimes b = a \times b]\);
于是\[
	e' \times e' = e'.
\]
设\(e\)是\(\opair{G,\times}\)的单位元.
假设\(e'\)在\(G\)中的逆元是\((e')^{-1}\),
即\(e' \times (e')^{-1} = e\),
那么在上式等号两边右乘\((e')^{-1}\),得\[
	(e' \times e') \times (e')^{-1} = (e') \times (e')^{-1};
\]
利用\(\times\)结合律可得\(e' \times e = e\);
又由于\(e' \times e = e'\),因此\(e = e'\);
这就说明,\(e\)是\(\opair{H,\otimes}\)的单位元,
\(\opair{G,\times}\)的单位元与\(\opair{H,\otimes}\)的单位元相同.
\end{proof}
\end{theorem}

\begin{definition}
设\(\opair{G,\times}\)是群,
\(e\)是\(\opair{G,\times}\)的单位元.
我们将\(\opair{\{e\},\times}\)和\(\opair{G,\times}\)
统称为“\(\opair{G,\times}\)的\DefineConcept{平凡子群}(trivial subgroup)”.
\end{definition}

\begin{definition}
设\(\opair{H,\times}\)是群\(\opair{G,\times}\)的子群.
如果\(H \subset G\),
那么称“\(\opair{H,\times}\)是\(\opair{G,\times}\)的\DefineConcept{真子群}(proper subgroup)”,
记作\(\opair{H,\times}\subset\opair{G,\times}\).
\end{definition}

从\cref{definition:抽象代数.子群的定义} 看出,
如果\(\opair{H,\otimes}\)是群\(\opair{G,\times}\)的一个子群,
那么根据群的定义以及二元代数运算的定义可知,
\(H\)对\(\otimes\)封闭,即\[
	(\forall a,b \in H)[a \otimes b \in H].
\]
由\cref{theorem:抽象代数.子群.群的单位元与其子群的单位元相同} 可知,
\(\opair{G,\times}\)的单位元与\(\opair{H,\times}\)的单位元相同,
不妨设它们的单位元为\(e\).
任给\(b \in H\),假设\(b\)在\(H\)中的逆元为\(c\),
则\[
	b \times c = c \times b = e;
\]
又假设\(b\)在\(G\)中的逆元为\(b^{-1}\),
则\[
	b \times b^{-1} = b^{-1} \times b = e;
\]
那么\[
	c = c \times e
	= c \times (b \times b^{-1})
	= (c \times b) \times b^{-1}
	= e \times b^{-1}
	= b^{-1};
\]
因此\(b^{-1} \in H\).
综上所述,如果\(\opair{H,\otimes}\)是群\(\opair{G,\times}\)的一个子群,
那么\[
	a,b \in H
	\implies
	a \times b^{-1} \in H.
\]
下面我们来证明,\([a,b \in H \implies a \times b^{-1} \in H]\)
是“\(\opair{H,\otimes}\)是\(\opair{G,\times}\)的子群”的充分条件.
\begin{theorem}
设\(\opair{G,\times}\)是群,
\(H\)是\(G\)的一个非空子集,
\(\otimes = (\times \upharpoonright(H \times H))\).
那么“\(\opair{H,\otimes}\)是\(\opair{G,\times}\)的子群”的充要条件是:\[
	a,b \in H \implies a \times b^{-1} \in H,
\]
其中\(b^{-1}\)是\(b\)在\(G\)中的逆元.
\begin{proof}
我们已经证得必要性,接下来只需证明充分性.
由于\(H\)不是空集,因此存在\(c \in H\).
假设\(e\)是\(\opair{G,\times}\)的单位元,
\(c^{-1}\)是\(c\)在\(G\)中的逆元,
由已知条件有\[
	c \times c^{-1} \in H;
\]
又因为\(c \times c^{-1} = e\),
于是有\(e \in H\).
再次利用已知条件可得\(e \times c^{-1} = c^{-1} \in H\),
这就是说\(H\)中任一元在\(G\)中的逆元实际上也在\(H\)中.

任给\(a,b \in H\),
由上面已证明的结论可知,\(b^{-1} \in H\);
再由已知条件可知,\[
	a \times (b^{-1})^{-1} \in H;
\]
考虑到\((b^{-1})^{-1} = b\),
于是\[
	a \times b = a \otimes b \in H,
\]
这就是说\(H\)对\(\otimes\)封闭,
或者说\(\otimes\)可以看成是定义在\(H\)上的一个二元代数运算.
由于运算\(\times\)服从结合律,
那么作为\(\times\)的限制,
根据\cref{theorem:抽象代数.群的运算在子集上的限制的不变性},
运算\(\otimes\)也服从结合律.
上面已证\(\opair{G,\times}\)的单位元\(e \in H\),
再次利用\cref{theorem:抽象代数.群的运算在子集上的限制的不变性},
就有\[
	(\forall a \in H)[
		a \otimes e = a \times e = a = e \times a = e \otimes a
	],
\]
这就说明\(e\)是\(\opair{H,\otimes}\)的单位元.
由上面已证明的结论可知,\(H\)中任一元\(b\)有逆元\(b^{-1}\).
综上所述,\(H\)是一个群,从而\(H\)是\(G\)的子群.
\end{proof}
\end{theorem}

鉴于\cref{theorem:抽象代数.群的运算在子集上的限制的不变性}
指出对运算的定义域的限制不会改变运算的性质,
于是我们可以把\cref{definition:抽象代数.子群的定义} 中%
子群\(\opair{H,\otimes}\)的符号改写为\(\opair{H,\times}\).
以后我们就不再特别强调\(\otimes\)是\(\times\)在\(H\times H\)上的限制.

%\subsection{循环群}
%\begin{definition}
%设\(\opair{G,\times}\)是群,\(H \subseteq G\).
%如果\[
%	(\forall \gamma \in G)
%	(\exists I \subseteq G)
%	[
%		(\gamma=\textstyle\prod_{a \in I} a)
%		\land
%		(\forall a \in I)[a \in H \lor a^{-1} \in H]
%	],
%\]
%那么称“\(H\)生成\(G\)(\(H\) generates \(G\))”;
%把\(\opair{G,\times}\)称为\DefineConcept{循环群}(cyclic group);
%把\(H\)的元素称为\(G\)的\DefineConcept{生成元}(generator).
%\end{definition}
%
%
%例如,给定元素\(a\),
%构造集合\(G = \Set{
%	\opair{a,n}
%	\given
%	n\in\mathbb{Z}
%}\),
%定义运算:\[
%	\otimes\colon G \times G \to G,
%	\opair{a,m}\otimes\opair{a,n}\mapsto\opair{a,m+n},
%\]
%并规定\(\opair{a,-n}=\opair{\opair{a,-1},n}\).
%我们已经熟知了整数集\(\mathbb{Z}\)上的加法运算,
%这里可以看出,运算\(\otimes\)与整数的加法一样,也服从结合律;
%\(\opair{a,0}\)是\(G\)的单位元;
%最后对于任意整数\(n\),\(\opair{a,n}\)与\(\opair{a,-n}\)互为逆元;
%由此可知\(G\)对\(\otimes\)成群.
%于是\(G\)是由\(a\)生成的循环群,\(a\)是\(G\)的生成元.
%
%再例如,整数加法\(\opair{\mathbb{Z},+}\)本身就是一个循环群,
%同时它也是唯一一个“无限”循环群.


%\subsection{陪集,拉格朗日定理}
%利用群\(\opair{G,\times}\)的子群\(H\)可以研究群\(\opair{G,\times}\)的结构,
%这时因为利用子群\(H\)可以给出集合\(G\)的一个划分.
%为了给出\(G\)的一个划分,需要在\(G\)上建立一个二元等价关系.
%于是我们定义:\[
%	(a \sim b)
%	\defiff
%	a \times b^{-1} \in H.
%\]
%首先,由于\(a \times a^{-1} = e \in H\),因此\(a \sim a\);
%这就说明\(\sim\)具有自反性.
%然后,由\(a \sim b\)可得\(a \times b^{-1} \in H\);
%那么由\(H \subseteq G\)可知\((a \times b^{-1})^{-1} \in H\);
%于是\(b \times a^{-1} \in H\),从而\(b \sim a\);
%这就说明\(\sim\)具有对称性.
%最后,由\(a \sim b\)和\(b \sim c\)得\(a \times b^{-1} \in H\)和\(b \times c^{-1} \in H\);
%由于\(H \subseteq G\),因此\((a \times b^{-1}) \times (b \times c^{-1}) \in H\);
%于是\(a \times c^{-1} \in H\),从而\(a \sim c\);
%这就说明\(\sim\)具有传递性.
%综上所述,\(\sim\)是\(G\)上的一个等价关系.
%
%任给\(a \in G\),
%其等价类为\begin{align*}
%	[a] &= \Set{ x \in G \given x \sim a }
%	= \Set{ x \in G \given x \times a^{-1} \in H }
%	= \Set{ x \in G \given x \times a^{-1} = h \land h \in H } \\
%	&= \Set{ x \in G \given x = h \times a \land h \in H }
%	= \Set{ h \times a \given h \in H }.
%\end{align*}
%我们称\([a]\)为\(H\)的一个\DefineConcept{右陪集},记为\([a]_\rightarrow\);
%称\(a\)为它的\DefineConcept{陪集代表}.
%于是\(H\)的全体右陪集组成的集合也就是\(G\)关于子群\(H\)的\DefineConcept{右商集},
%记为\((G/H)_\rightarrow\),它是\(G\)的一个划分.
%
%类似地,我们还可以重新定义:\[
%	(a \sim b)
%	\defiff
%	b^{-1} \times a \in H.
%\]
%同理可证新的\(\sim\)也是\(G\)上的一个等价关系.
%任给\(a \in G\),它的等价类为\begin{align*}
%	[a] &= \Set{ x \in G \given x \sim a }
%	= \Set{ x \in G \given a^{-1} \times x \in H }
%	= \Set{ x \in G \given a^{-1} \times x = h \land h \in H } \\
%	&= \Set{ x \in G \given x = a \times h \land h \in H }
%	= \Set{ a \times h \given h \in H }.
%\end{align*}
%我们称\([a]\)为\(H\)的一个\DefineConcept{左陪集},记为\([a]_\leftarrow\);
%称\(a\)为它的\DefineConcept{陪集代表}.
%于是\(H\)的全体左陪集组成的集合也就是\(G\)关于子群\(H\)的\DefineConcept{左商集},
%记为\((G/H)_\leftarrow\),它也是\(G\)的一个划分.
%
%现在定义映射\[
%	\sigma\colon (G/H)_\leftarrow \to (G/H)_\rightarrow,
%	[a]_\leftarrow \mapsto [a^{-1}]_\rightarrow.
%\]
%由于\[
%	[a]_\leftarrow = [c]_\leftarrow
%	\iff
%	c^{-1} \times a \in H
%	\iff
%	c^{-1} \times (a^{-1})^{-1} \in H
%	\iff
%	[c^{-1}]_\rightarrow = [a^{-1}]_\rightarrow,
%\]
%因此\(\sigma\)是单射.
%任给\([b]_\rightarrow \in (G/H)_\rightarrow\),
%有\[
%	\sigma([b^{-1}]_\leftarrow)
%	= ([b^{-1}]_\rightarrow)^{-1}
%	= [b]_\rightarrow,
%\]
%因此\(\sigma\)又是满射,
%从而\(\sigma\)是双射.
%于是左陪集与右陪集的基数相同,即\[
%	\abs{(G/H)_\leftarrow} = \abs{(G/H)_\rightarrow}.
%\]
%由此我们引出下述概念.
%\begin{definition}
%设\(H\)是群\(\opair{G,\times}\)的一个子群,
%把\((G/H)_\leftarrow\)(或\((G/H)_\rightarrow\))的基数%
%称为“\(H\)在\(G\)中的\DefineConcept{指数}”,记作\([G:H]\).
%\end{definition}
%
%假设群\(\opair{G,\times}\)的子群\(H\)在\(G\)中的指数为\([G:H]=r\),
%则\begin{equation}\label{equation:抽象代数.关于子群的左陪集分解式}
%	G = H \cup [a_1]_\leftarrow \cup [a_2]_\leftarrow \cup \dotsb \cup [a_{r-1}]_\leftarrow,
%\end{equation}
%其中\(H,[a_1]_\leftarrow,[a_2]_\leftarrow,\dotsc,[a_{r-1}]_\leftarrow\)两两互斥.
%我们把\cref{equation:抽象代数.关于子群的左陪集分解式}
%称为“群\(\opair{G,\times}\)关于子群\(H\)的\DefineConcept{左陪集分解式}”;
%把\(\Set{e,\AutoTuple{a}{r-1}}\)称为“\(H\)在\(G\)中的\DefineConcept{左陪集代表系}”.
%
%\begin{lemma}
%映射\[
%	\uptau\colon H \to [a]_\leftarrow, h \mapsto a \times h
%\]是双射.
%\end{lemma}
%由此可知,子群\(H\)与它的任一左陪集\([a]_\leftarrow\)的基数相同.
%
%\begin{theorem}[拉格朗日定理]
%设\(G\)是有限群,\(H\)是\(G\)的任一子群,则\[
%	\abs{G} = [G:H] \cdot \abs{H}.
%\]
%\end{theorem}
%拉格朗日定理说明:
%\(G\)的任一子群\(H\)的阶是\(G\)的阶的因数.
%
%\begin{corollary}
%设\(G\)是有限群,则\(G\)的任一元素\(a\)的阶是\(G\)的阶的因数,从而\(a^{\abs{G}}=e\).
%\end{corollary}

\section{正规子群}
\begin{definition}\label{definition:抽象代数.正规子群.子集乘积的定义}
设\(\opair{G,\otimes}\)是群,
\(H \subseteq G,
K \subseteq G\).
我们把集合\[
	\Set{ h \otimes k \given h \in H \land k \in K }
\]
称为“\(H\)与\(K\)的\DefineConcept{乘积}”,
记作\(H \otimes K\).
\end{definition}

\begin{example}
我们以整数乘群\(\opair{\mathbb{Z},\times}\)为例.
取\(H\)为正整数集,\(K\)为负整数集,即\[
	H = \Set{ h \in \mathbb{Z} \given h > 0 }, \qquad
	K = \Set{ k \in \mathbb{Z} \given k < 0 },
\]
由于\[
	(\forall h \in H)(\forall k \in K)[h \times k < 0],
\]
所以\(H\)与\(K\)的乘积恰好就是\(K\),
即\(H \times K = K \subset \mathbb{Z}\).
\end{example}

\begin{proposition}
设\(\opair{G,\otimes}\)是群,
\(H \subseteq G,
K \subseteq G\).
那么\(H \otimes K \subseteq G\).
\begin{proof}
根据\hyperref[definition:抽象代数.正规子群.子集乘积的定义]{子集乘积的定义},有\[
	(\forall h)
	(\forall k)
	[h \in H \land k \in K \iff h \otimes k \in H \otimes K].
	\eqno(1)
\]
再根据\hyperref[definition:集合论.子集的定义]{子集的定义},有\[
	(\forall h)
	[H \subseteq G \iff (h \in H \implies h \in G)],
	\qquad
	(\forall k)
	[K \subseteq G \iff (k \in K \implies k \in G)],
\]
所以\[
	(\forall h)
	(\forall k)
	[h \in H \land k \in K \implies h,k \in G \implies h \otimes k \in G].
	\eqno(2)
\]
于是由(1)(2)两式可知\(a \in h \otimes k \implies a \in G\),\(H \otimes K \subseteq G\).
\end{proof}
\end{proposition}

\begin{proposition}\label{theorem:抽象代数.正规子群.群中子集的乘积满足结合律}
群中子集的乘积满足结合律.
\begin{proof}
设\(\opair{G,\otimes}\)是群,
\(H \subseteq G,
K \subseteq G,
L \subseteq G\).
因为\[
	(\forall h \in H)
	(\forall k \in K)
	(\forall l \in L)
	[h \otimes (k \otimes l) = (h \otimes k) \otimes l],
\]
所以\(H \otimes (K \otimes L) = (H \otimes K) \otimes L\).
\end{proof}
\end{proposition}

\begin{proposition}\label{theorem:抽象代数.正规子群.子群与自身的乘积等于自身}
设\(\opair{G,\otimes}\)是群,
\(\opair{H,\otimes} \subseteq \opair{G,\otimes}\).
那么\[
	H \otimes H = H.
\]
\begin{proof}
由于\(\opair{H,\otimes}\subseteq\opair{G,\otimes}\),
所以\(H\)对\(\otimes\)封闭,即\[
	(\forall h,k \in H)
	[h \otimes k \in H];
\]
于是\(H \otimes H \subseteq H\).

设\(e\)是\(\opair{G,\otimes}\)的单位元.
因为\[
	(\forall h \in H)
	[h \otimes e = h],
\]
所以\(H = H \otimes \{e\} \subseteq H \otimes H\).

综上所述,\(H \otimes H = H\).
\end{proof}
\end{proposition}

应该注意到,\cref{theorem:抽象代数.正规子群.子群与自身的乘积等于自身} 的逆命题并不成立,
即当\(H \otimes H = H\)时,\(H\)不一定对\(\otimes\)成群.
譬如,在整数加群\(\opair{\mathbb{Z},+}\)中,正整数集\(\mathbb{Z}^+\)对加法\(+\)并不成群.

对于有限子群的乘积我们有如下定理.
\begin{theorem}
%@see: 《近世代数》(熊全淹) P35. 定理1
设\(H,K\)是群\(\opair{G,\otimes}\)的有限子群.
那么\[
	\card(H \otimes K)
	= \frac{\card H \cdot \card K}{\card(H \cap K)}.
\]
\begin{proof}
取定\(h_1,h_2 \in H\),\(k_1,k_2 \in K\),
使得\(h_1 \otimes k_1 = h_2 \otimes k_2\),
在等号两边同时左乘\(h_2^{-1}\)并右乘\(k_1^{-1}\),
便有\[
	h_2^{-1} \otimes h_1 = k_2 \otimes k_1^{-1};
\]
令\(d = h_2^{-1} \otimes h_1\),
那么\(d \in H \cap K\).
因此\[
	h_2 = h_1 \otimes d^{-1}, \qquad
	k_2 = d \otimes k_1.
\]

反过来,对于给定的\(h_1 \in H\)和\(k_1 \in K\),
任取\(H \cap K\)中的一个元素\(d\),
设\(h_2 = h_1 \otimes d^{-1},
k_2 = d \otimes k_1\),
就得到\(h_1 \otimes k_1 = h_2 \otimes k_2\),
这就是说,对于任意\(h_1,k_1\),
在\(H \otimes K\)中有\(\card(H \cap K)\)个与\(h_1 \otimes k_1\)相等的元素.
\end{proof}
\end{theorem}

特别地,当\(H \cap K = \{e\}\),即当\(\card(H \cap K) = 1\)时,
\(\card(H \otimes K) = \card H \cdot \card K\).

假设\(H,K\)都是\(\opair{G,\otimes}\)的子群,
\(H\)和\(K\)的乘积\(H \otimes K\)通常不一定成群.
于是我们想要知道,在什么条件下,\(H \otimes K\)也能成群?

\begin{theorem}\label{theorem:群的子集的乘积成群的充要条件}
%@see: 《近世代数》(熊全淹) P36. 定理2
群\(\opair{G,\otimes}\)的子群\(H,K\)的乘积\(H \otimes K\)成群的充要条件是:
\(H\)与\(K\)可交换.
\begin{proof}
首先我们假设\(H \otimes K\)成群,
任取\(h \in H\),任取\(k \in K\),
因为\(K \otimes H\)中的元素\(k \otimes h\)
是\(H \otimes K\)中元素\(h^{-1} \otimes k^{-1}\)的逆元,
所以\(k \otimes h \in H \otimes K\),
因此\(K \otimes H \subseteq H \otimes K\).
哟因为\((h\ otimes k)^{-1} \in H \otimes K\),
于是\(h \otimes k \in K \otimes H\),
所以\(H \otimes K \subseteq K \otimes H\).
综上,\(H \otimes K = K \otimes H\);
也就是说,假如\(H \otimes K\)成群,那么\(H\)与\(K\)可交换.

反过来,假如\(H \otimes K = K \otimes H\),
那么\(H \otimes K\)中任一元\(h \otimes k\)的逆元\(k^{-1} \otimes h^{-1}\)
在\(K \otimes H = H \otimes K\)中,
又因为\[
	H \otimes K \otimes H \otimes K
	= H \otimes H \otimes K \otimes K
	= H \otimes K,
\]
所以\(H \otimes K\)中任意两元的乘积仍然在\(H \otimes K\)中,
于是\(H \otimes K\)成群.
\end{proof}
\end{theorem}

应该注意到,\cref{theorem:群的子集的乘积成群的充要条件} 所说的“\(H\)与\(K\)可交换”
指的是\(H \otimes K\)和\(K \otimes K\)这两个集合相等,
并不意味着元素的乘积也同样相等:\[
	(\forall h \in H)(\forall k \in K)[h \otimes k = k \otimes h].
\]

\begin{corollary}
如果\(\opair{G,\otimes}\)是交换群,
那么它的子群\(H,K\)的乘积\(H \otimes K\)成群.
\end{corollary}
