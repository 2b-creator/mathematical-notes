\section{子群}
\subsection{子群}
\begin{definition}
%@see: 《近世代数》(丘维声,2015) P40. 定义1
%@see: https://mathworld.wolfram.com/Subgroup.html
设\(\opair{G,\times}\)是群,
\(H\)是\(G\)的一个非空子集.
又设映射\(\otimes\)是\(\times\)在\(H\)上的限制,
即\(\otimes = \times \upharpoonright H\).
如果\(H\)对于运算\(\otimes\)也成为一个群\(\opair{H,\otimes}\),
那么称“\(\opair{H,\otimes}\)是\(\opair{G,\times}\)的一个\DefineConcept{子群}”,
记作\(\opair{H,\otimes} \subseteq \opair{G,\times}\).
\end{definition}

显而易见的是,仅由群\(\opair{G,\times}\)的单位元\(e\)%
组成的集合\(\{e\}\)对于运算\(\odot\colon \{e\}\times\{e\}\to\{e\}\)%
也成为一个群\(\opair{\{e\},\odot}\);
它是\(\opair{G,\times}\)的一个子群.
同时,\(\opair{G,\times}\)本身也是\(\opair{G,\times}\)的一个子群.
我们将\(\opair{\{e\},\odot}\)和\(\opair{G,\times}\)%
统称为“\(\opair{G,\times}\)的\DefineConcept{平凡子群}”.

另外,我们还可以看出,
如果\(\opair{H,\otimes}\)是群\(\opair{G,\times}\)的一个子群,
那么根据群和二元代数运算的定义可知,
\(H\)对\(\otimes\)封闭,即
\[
	(\forall a,b \in H)[a \otimes b \in H].
\]
设\(e'\)是\(\opair{H,\otimes}\)的单位元,
则\[
	e' \otimes e' = e';
\]
假设\(e'\)在\(G\)中的逆元是\((e')^{-1}\),
即\[
	e' \times (e')^{-1} = e;
\]
那么\[
	(e' \otimes e') \times (e')^{-1} = (e') \times (e')^{-1};
\]
由此可得\(e' \times e = e\);
又由于\(e' \times e = e'\),因此\(e = e'\);
这就说明,群\(G\)的单位元\(e\)是\(G\)的子群\(H\)的单位元.
任给\(b \in H\),假设\(b\)在\(H\)中的逆元为\(c\),
则\[
	b \times c = c \times b = e;
\]
又假设\(b\)在\(G\)中的逆元为\(b^{-1}\),
则\[
	b \times b^{-1} = b^{-1} \times b = e;
\]
那么\(c = b^{-1}\);
因此\(b^{-1} \in H\).
综上所述,如果\(H\)是群\(G\)的一个子群,
那么从\(a,b \in H\)可推出\(a \times b^{-1} \in H\).
下面我们来证明,这个条件也是群\(G\)的非空子集\(H\)是\(G\)的子群的充分条件.
\begin{theorem}
群\(G\)的非空子集\(H\)是\(G\)的子群的充要条件是:\[
	a,b \in H \implies a \times b^{-1} \in H.
\]
\begin{proof}
我们已经证得必要性,接下来只需证明充分性.
由于\(H\)不是空集,因此存在\(c \in H\).
由已知条件\[
	e = c \times c^{-1} \in H;
\]
任给\(b \in H\),同样由已知条件得\(e \times b^{-1} \in H\),于是\(b^{-1} \in H\).

任给\(a,b \in H\),由已知条件和已证明的结论可知,\[
	a \times (b^{-1})^{-1} \in H;
\]于是\(a \times b \in H\).
因此群\(G\)的运算是\(H\)的运算.
由于群\(G\)的运算满足结合律,因此它在\(H\)中的限制也满足结合律.
上面已证\(G\)的单位元\(e \in H\),
从而\(H\)有单位元\(e\),
且任给\(b \in H\),有\(b^{-1} \in H\),因此\(b\)在\(H\)中有逆元\(b^{-1}\).

综上所述,\(H\)是一个群,从而\(H\)是\(G\)的子群.
\end{proof}
\end{theorem}

\subsection{陪集,拉格朗日定理}
利用群\(G\)的子群\(H\)可以研究群\(G\)的结构,
这时因为利用子群\(H\)可以给出集合\(G\)的一个划分.
为了给出\(G\)的一个划分,需要在\(G\)上建立一个二元等价关系.
于是我们定义:\[
	(a \sim b)
	\defiff
	a \times b^{-1} \in H.
\]
首先,由于\(a \times a^{-1} = e \in H\),因此\(a \sim a\);
这就说明\(\sim\)具有自反性.
然后,由\(a \sim b\)可得\(a \times b^{-1} \in H\);
那么由\(H \subseteq G\)可知\((a \times b^{-1})^{-1} \in H\);
于是\(b \times a^{-1} \in H\),从而\(b \sim a\);
这就说明\(\sim\)具有对称性.
最后,由\(a \sim b\)和\(b \sim c\)得\(a \times b^{-1} \in H\)和\(b \times c^{-1} \in H\);
由于\(H \subseteq G\),因此\((a \times b^{-1}) \times (b \times c^{-1}) \in H\);
于是\(a \times c^{-1} \in H\),从而\(a \sim c\);
这就说明\(\sim\)具有传递性.
综上所述,\(\sim\)是\(G\)上的一个等价关系.

任给\(a \in G\),
其等价类为\begin{align*}
	[a] &= \Set{ x \in G \given x \sim a }
	= \Set{ x \in G \given x \times a^{-1} \in H }
	= \Set{ x \in G \given x \times a^{-1} = h \land h \in H } \\
	&= \Set{ x \in G \given x = h \times a \land h \in H }
	= \Set{ h \times a \given h \in H }.
\end{align*}
我们称\([a]\)为\(H\)的一个\DefineConcept{右陪集},记为\([a]_\rightarrow\);
称\(a\)为它的\DefineConcept{陪集代表}.
于是\(H\)的全体右陪集组成的集合也就是\(G\)关于子群\(H\)的\DefineConcept{右商集},
记为\((G/H)_\rightarrow\),它是\(G\)的一个划分.

类似地,我们还可以重新定义:\[
	(a \sim b)
	\defiff
	b^{-1} \times a \in H.
\]
同理可证新的\(\sim\)也是\(G\)上的一个等价关系.
任给\(a \in G\),它的等价类为\begin{align*}
	[a] &= \Set{ x \in G \given x \sim a }
	= \Set{ x \in G \given a^{-1} \times x \in H }
	= \Set{ x \in G \given a^{-1} \times x = h \land h \in H } \\
	&= \Set{ x \in G \given x = a \times h \land h \in H }
	= \Set{ a \times h \given h \in H }.
\end{align*}
我们称\([a]\)为\(H\)的一个\DefineConcept{左陪集},记为\([a]_\leftarrow\);
称\(a\)为它的\DefineConcept{陪集代表}.
于是\(H\)的全体左陪集组成的集合也就是\(G\)关于子群\(H\)的\DefineConcept{左商集},
记为\((G/H)_\leftarrow\),它也是\(G\)的一个划分.

现在定义映射\[
	\sigma\colon (G/H)_\leftarrow \to (G/H)_\rightarrow,
	[a]_\leftarrow \mapsto [a^{-1}]_\rightarrow.
\]
由于\[
	[a]_\leftarrow = [c]_\leftarrow
	\iff
	c^{-1} \times a \in H
	\iff
	c^{-1} \times (a^{-1})^{-1} \in H
	\iff
	[c^{-1}]_\rightarrow = [a^{-1}]_\rightarrow,
\]
因此\(\sigma\)是单射.
任给\([b]_\rightarrow \in (G/H)_\rightarrow\),
有\[
	\sigma([b^{-1}]_\leftarrow)
	= ([b^{-1}]_\rightarrow)^{-1}
	= [b]_\rightarrow,
\]
因此\(\sigma\)又是满射,
从而\(\sigma\)是双射.
于是左陪集与右陪集的基数相同,即\[
	\abs{(G/H)_\leftarrow} = \abs{(G/H)_\rightarrow}.
\]
由此我们引出下述概念.
\begin{definition}
设\(H\)是群\(G\)的一个子群,
把\((G/H)_\leftarrow\)(或\((G/H)_\rightarrow\))的基数%
称为“\(H\)在\(G\)中的\DefineConcept{指数}”,记作\([G:H]\).
\end{definition}

假设群\(G\)的子群\(H\)在\(G\)中的指数为\([G:H]=r\),
则\begin{equation}\label{equation:抽象代数.关于子群的左陪集分解式}
	G = H \cup [a_1]_\leftarrow \cup [a_2]_\leftarrow \cup \dotsb \cup [a_{r-1}]_\leftarrow,
\end{equation}
其中\(H,[a_1]_\leftarrow,[a_2]_\leftarrow,\dotsc,[a_{r-1}]_\leftarrow\)两两互斥.
我们把\cref{equation:抽象代数.关于子群的左陪集分解式}
称为“群\(G\)关于子群\(H\)的\DefineConcept{左陪集分解式}”;
把\(\Set{e,\v{a}{r-1}}\)称为“\(H\)在\(G\)中的\DefineConcept{左陪集代表系}”.

\begin{lemma}
映射\[
	\uptau\colon H \to [a]_\leftarrow, h \mapsto a \times h
\]是双射.
\end{lemma}
由此可知,子群\(H\)与它的任一左陪集\([a]_\leftarrow\)的基数相同.

\begin{theorem}[拉格朗日定理]
设\(G\)是有限群,\(H\)是\(G\)的任一子群,则\[
	\abs{G} = [G:H] \cdot \abs{H}.
\]
\end{theorem}
拉格朗日定理说明:
\(G\)的任一子群\(H\)的阶是\(G\)的阶的因数.

\begin{corollary}
设\(G\)是有限群,则\(G\)的任一元素\(a\)的阶是\(G\)的阶的因数,从而\(a^{\abs{G}}=e\).
\end{corollary}
