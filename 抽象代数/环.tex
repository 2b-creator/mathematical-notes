
\section{环}
\begin{definition}
假设在非空集合\(R\)上定义了两个代数运算,
一个叫做加法(记作\(+\)),
另一个叫做乘法(记作\(\cdot\)),
并且它们满足下列6条运算法则:
\begin{enumerate}
    \item 加法交换律,即\[
        \forall a,b \in R \bigl(a+b = b+a\bigr).
    \]

    \item 加法结合律,即\[
        \forall a,b,c \in R \Bigl[(a+b)+c = a+(b+c)\Bigr].
    \]

    \item \(R\)中存在一个元素\(o\),它满足\[
        \forall a \in R \bigl( a+o = o+a = a \bigr);
    \]
    称其为\(R\)的\DefineConcept{零元}.

    \item 加法可逆,即\[
        (\forall a \in R)(\exists b \in R)
        a+b = b+a = o;
    \]
    把\(b\)称为\(a\)的\DefineConcept{负元},记作\((-a)\).

    \item 乘法结合律,即\[
        \forall a,b,c \in R
        \bigl[ (a \cdot b) \cdot c = a \cdot (b \cdot c) \bigr].
    \]

    \item 左分配律,即\[
        \forall a,b,c \in R \bigl[ a \cdot (b+c) = a \cdot b + a \cdot c \bigr],
    \]
    和右分配律,即\[
        \forall a,b,c \in R \bigl[ (b+c) \cdot a = b \cdot a + c \cdot a \bigr].
    \]
\end{enumerate}
那么称“\(R\)是一个\DefineConcept{环}(ring)”.
\end{definition}

在环上可以定义二元运算\DefineConcept{减法}如下:\[
    a - b \defeq a + (-b).
\]

\begin{property}
任意一个环的零元唯一.
\begin{proof}
设环\(R\)中有两个零元\(o_1\)和\(o_2\).
由环的定义,有\[
    o_1 = o_1 + o_2 = o_2 + o_1 = o_2.
\]
也就是说,环\(R\)的零元是唯一的.
\end{proof}
\end{property}

\begin{property}
零元的负元是它本身.
\begin{proof}
设环\(R\)中的零元\(o\)的负元是\(x\).
由环的定义,有\[
    x + o = o + x = x,
\]\[
    x + o = o + x = o.
\]比较可得\[
    x = o.
\]
这就说明,零元的负元总是它本身.
\end{proof}
\end{property}

\begin{property}
两元素各自负元的和等于两元素和的负元.
\begin{proof}
从环\(R\)中任取两元素\(a\)和\(b\).
由加法交换律和加法结合律,得\[
    [(-a) + (-b)] + [a + b]
    = [(-a) + a] + [(-b) + b]
    = o + o = o,
\]
也就是说\((a+b)\)是\((-a) + (-b)\)的负元.
\end{proof}
\end{property}

\begin{property}
任一环中元素的负元唯一.
\begin{proof}
从环\(R\)中任取一元素\(a\),假设元素\(b\)与元素\(c\)都是\(a\)的负元.
由环的定义,有\[
    b + a = o,
\]\[
    c + a = o.
\]
而\((-c) + (-a) = -(c + a) = -o = o\),那么\[
    (b + a) + [-(c + a)]
    = o + o = o.
\]又因为\[
    (b + a) + [-(c + a)]
    = (b + a) + [(-c) + (-a)]
    = [b + (-c)] + [a + (-a)]
    = b + (-c),
\]所以\[
    b + (-c) = o,
\]\[
    [b + (-c)] + c = b + [(-c) + c] = b = o + c = c,
\]
也就是说,元素\(a\)的负元是唯一的.
\end{proof}
\end{property}

\begin{definition}
如果环\(R\)的乘法满足交换律,即\[
    \forall a,b \in R \bigl( a \cdot b = b \cdot a \bigr),
\]
那么称\(R\)是\DefineConcept{交换环}.
\end{definition}

\begin{definition}
如果环\(R\)中有一个元素\(e\)满足\[
    \forall a \in R \bigl( e \cdot a = a \cdot e = a \bigr),
\]
那么称“\(e\)是\(R\)的\DefineConcept{单位元}或\DefineConcept{幺元}”.
\end{definition}

\begin{property}
具有单位元的环的单位元唯一.
\begin{proof}
设环\(R\)具有单位元\(e_1\)和\(e_2\).
根据单位元的定义,对\(\forall a \in R\),有\[
    e_1 = e_1 \cdot e_2 = e_2 \cdot e_1 = e_2,
\]
也就是说,\(R\)的单位元是唯一的.
\end{proof}
\end{property}

\begin{theorem}[乘零定理]
在环\(R\)中,零元\(o\)满足\[
    \forall a \in R \bigl( o \cdot a = a \cdot o = o \bigr).
\]
\begin{proof}
需要利用沟通加法和乘法的桥梁:
乘法对于加法的左、右分配律,即\[
    o \cdot a = (o + o) \cdot a = o \cdot a + o \cdot a.
\]在上式等号两边加上\(o \cdot a\)的负元\(-o \cdot a\),得\[
    o \cdot a + (- o \cdot a) = (o \cdot a + o \cdot a) + (- o \cdot a),
\]\[
    o = o \cdot a.
\]
同理,可证\(a \cdot o = o\).
\end{proof}
\end{theorem}

\begin{example}
设\(R\)是有单位元\(e(\neq o)\)的环.
证明:单位元的负元\((-e)\)满足\[
    (-e)^2=(-e)\cdot(-e)=e.
\]
\begin{proof}
根据单位元的定义有\[
    e \cdot (-e) = (-e) \cdot e = -e.
\]又由负元的定义有\[
    e + (-e) = o,
\]在等号左右两边同时右乘\((-e)\),再由右分配律有\[
    -e + (-e)^2 = e \cdot (-e) + (-e) \cdot (-e) = o \cdot (-e) = o,
\]
移项得\((-e)^2 = e\).
\end{proof}
\end{example}

\begin{definition}
设\(R\)是有单位元\(e(\neq o)\)的环.
对于\(a \in R\),如果\[
    \exists b \in R \bigl( a \cdot b = b \cdot a = e \bigr),
\]那么称“\(a\)是一个\DefineConcept{可逆元}”,
把\(b\)叫做\(a\)的\DefineConcept{逆元},记作\(a^{-1}\).
\end{definition}

\begin{property}
若\(a\)是可逆元,则\(a\)的逆元唯一.
\begin{proof}
设环\(R\)具有单位元\(e(\neq o)\).
假设环\(R\)中的任一可逆元\(a\)的逆元为\(b\)和\(c\).由可逆元的定义,有\[
    a \cdot b = b \cdot a = e,
\]\[
    a \cdot c = c \cdot a = e.
\]
又由单位元的定义,有\[
    b = b \cdot e
    = b \cdot (a \cdot c)
    = (b \cdot a) \cdot c
    = e \cdot c
    = c,
\]
这就说明,任一可逆元的逆元是唯一的.
\end{proof}
\end{property}

\begin{definition}
设\(R\)是一个环.
对于\(a \in R\),
如果存在\(c \in R\)且\(c \neq o\),
使得\(a \cdot c = o\)(或\(c \cdot a = o\)),
那么称\(a\)是一个\DefineConcept{左零因子}或\DefineConcept{右零因子}.
左、右零因子统称为\DefineConcept{零因子}.
\end{definition}

\begin{property}
在环\(R\)中,零元\(o\)是一个零因子.
\end{property}

\begin{theorem}
设\(R\)是有单位元\(e(\neq o)\)的环,则\(R\)的零因子不是可逆元.
\begin{proof}
设\(a\)是左零因子,则存在\(c \in R\)且\(c \neq o\),使得\[
    a \cdot c = o.
\]

假如\(a\)是可逆元,则在上式两边左乘\(a^{-1}\),再根据乘零定理,得\[
    a^{-1} \cdot (a \cdot c) = a^{-1} \cdot o = o.
\]
根据乘法的结合律和逆元的性质,得\[
    a^{-1} \cdot (a \cdot c) = (a^{-1} \cdot a) \cdot c = e \cdot c = c.
\]
于是\(c = o\),矛盾,故左零因子\(a\)不是可逆元.
同理可得,右零因子不是可逆元.
\end{proof}
\end{theorem}
上述定理的逆否命题是:
设\(R\)是有单位元\(e(\neq o)\)的环,则\(R\)的可逆元不是零因子.

\begin{example}
要想让可逆元等于零元,唯一的办法是在由一个元素\(o\)构成的集合\(R\)上定义加法和乘法如下:\[
    o + o \mapsto o,
    \qquad
    o \cdot o \mapsto o.
\]
显然元素\(o\)是环\(R\)的单位元和零元.
再将原本的可逆元的定义中对环\(R\)的单位元\(e \neq o\)的约束去掉,
那么对于环\(R\)上唯一的元素\(o\),有\[
    \forall x \in R \bigl( x = o \land x \cdot o = o \cdot x = o \bigr),
\]
即元素\(o\)还是环\(R\)的“可逆元”,它的“逆元”是\(o\)本身.

这就说明,在上面对可逆元的定义中,我们必须强调环\(R\)上的单位元\(e\)不是零元.
\end{example}

\begin{theorem}
假设在非空集合\(R\)上定义了两种二元运算\(+\)和\(\cdot\),
且满足\begin{enumerate}
    \item \((R,+)\)是阿贝尔群,\((R,\cdot)\)是半群.

    \item 对任意\(a,b,c \in R\),对\(+\)和\(\cdot\)有分配律成立,即\[
        a \cdot (b + c) = a \cdot b + a \cdot c,
    \]\[
        (b + c) \cdot a = b \cdot a + c \cdot a.
    \]
\end{enumerate}
那么称\(R\)是环.
\end{theorem}
