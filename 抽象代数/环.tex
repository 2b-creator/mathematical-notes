\section{环}
\subsection{环的概念}
\begin{definition}\label{definition:抽象代数.环的定义}
设\(R\)是非空集合,
加法\(+\)和乘法\(\times\)是定义在\(R\)上的两个二元代数运算.
如果这两种运算满足下列六条运算法则:
\begin{enumerate}
    \item 加法交换律,即\[
        (\forall a,b \in R)[a+b = b+a].
    \]

    \item 加法结合律,即\[
        (\forall a,b,c \in R)[(a+b)+c = a+(b+c)].
    \]

    \item \(R\)中存在零元,即\[
        (\exists o \in R)(\forall a \in R)[a+o = o+a = a].
    \]

	这里我们把满足\[
		o \in G
		\land
		(\forall a \in R)[a+o = o+a = a]
	\]的元素\(o\)称为“\(R\)的\DefineConcept{零元}”.

    \item 加法可逆,即\[
        (\forall a \in R)(\exists b \in R)[a+b = b+a = o];
    \]

	这里我们把满足\[
		b \in R
		\land
        (\forall a \in R)[a+b = b+a = o]
	\]的元素\(b\)称为“\(a\)的\DefineConcept{负元}”,记作\((-a)\).

    \item 乘法结合律,即\[
        (\forall a,b,c \in R)
        [ (a \times b) \times c = a \times (b \times c) ].
    \]

    \item 乘法对于加法的左分配律,即\[
        (\forall a,b,c \in R)[ a \times (b+c) = a \times b + a \times c ],
    \]
    和乘法对于加法的右分配律,即\[
        (\forall a,b,c \in R)[ (b+c) \times a = b \times a + c \times a ].
    \]
\end{enumerate}
那么称“\(\opair{R,+,\times}\)是一个\DefineConcept{环}(ring)”.
\end{definition}

\begin{theorem}
假设在非空集合\(R\)上定义了两种二元运算\(+\)和\(\times\),
且满足\begin{enumerate}
    \item \(\opair{R,+}\)是阿贝尔群,\(\opair{R,\times}\)是半群.

    \item 对任意\(a,b,c \in R\),对\(+\)和\(\times\)有分配律成立,即\[
        a \times (b + c) = a \times b + a \times c,
    \]\[
        (b + c) \times a = b \times a + c \times a.
    \]
\end{enumerate}
那么称\(\opair{R,+,\times}\)是环.
\end{theorem}

在环上可以定义二元运算\DefineConcept{减法}如下:\[
    a - b \defeq a + (-b).
\]

\begin{example}
整数集\(\mathbb{Z}\)、偶数集\(2\mathbb{Z}\)、任意一个数域\(K\)、
数域\(K\)上一元多项式集\(K[x]\)、
数域\(K\)上\(n\)阶全矩阵集\(M_n(K)\)都是环.
\end{example}

\subsection{环的性质}
\begin{property}
任意一个环的零元唯一.
\begin{proof}
设环\(\opair{R,+,\times}\)中有两个零元\(o_1\)和\(o_2\).
由环的定义,有\[
    o_1 = o_1 + o_2 = o_2 + o_1 = o_2.
\]
也就是说,环\(\opair{R,+,\times}\)的零元是唯一的.
\end{proof}
\end{property}

\begin{property}
环中零元的负元是它本身.
\begin{proof}
设环\(\opair{R,+,\times}\)中的零元\(o\)的负元是\(x\).
由环的定义,有\[
    x + o = o + x = x,
\]\[
    x + o = o + x = o.
\]比较可得\[
    x = o.
\]
这就说明,零元的负元总是它本身.
\end{proof}
\end{property}

\begin{property}
两元素各自负元的和等于两元素和的负元.
\begin{proof}
从环\(\opair{R,+,\times}\)中任取两元素\(a\)和\(b\).
由加法交换律和加法结合律,得\[
    [(-a) + (-b)] + [a + b]
    = [(-a) + a] + [(-b) + b]
    = o + o = o,
\]
也就是说\((a+b)\)是\((-a) + (-b)\)的负元.
\end{proof}
\end{property}

\begin{property}
任一环中元素的负元唯一.
\begin{proof}
从环\(\opair{R,+,\times}\)中任取一元素\(a\),假设元素\(b\)与元素\(c\)都是\(a\)的负元.
由环的定义,有\[
    b + a = o, \qquad
    c + a = o.
\]
而\((-c) + (-a) = -(c + a) = -o = o\),那么\[
    (b + a) + [-(c + a)]
    = o + o = o.
\]又因为\[
    (b + a) + [-(c + a)]
    = (b + a) + [(-c) + (-a)]
    = [b + (-c)] + [a + (-a)]
    = b + (-c),
\]所以\[
    b + (-c) = o,
\]\[
    [b + (-c)] + c = b + [(-c) + c] = b = o + c = c,
\]
也就是说,元素\(a\)的负元是唯一的.
\end{proof}
\end{property}

\subsection{交换环,单位元}
\begin{definition}
如果环\(\opair{R,+,\times}\)的乘法\(\times\)满足交换律,即\[
	(\forall a,b \in R)[a \times b = b \times a],
\]
那么称\(R\)是\DefineConcept{交换环}.
\end{definition}

\begin{definition}
如果环\(\opair{R,+,\times}\)中有一个元素\(e\)满足\[
	(\forall a \in R)[e \times a = a \times e = a],
\]
那么称“\(e\)是\(R\)的\DefineConcept{单位元}”;
这时候把\(\opair{R,+,\times}\)称为\DefineConcept{有单位元的环}.
\end{definition}

\begin{property}
具有单位元的环的单位元唯一.
\begin{proof}
设环\(\opair{R,+,\times}\)具有单位元\(e_1\)和\(e_2\).
根据单位元的定义,对\(\forall a \in R\),有\[
    e_1 = e_1 \times e_2 = e_2 \times e_1 = e_2,
\]
也就是说,\(R\)的单位元是唯一的.
\end{proof}
\end{property}

\subsection{乘零定理}
\begin{theorem}[乘零定理]
在环\(\opair{R,+,\times}\)中,零元\(o\)满足\[
    \forall a \in R \bigl( o \times a = a \times o = o \bigr).
\]
\begin{proof}
需要利用沟通加法和乘法的桥梁:
乘法对于加法的左、右分配律,即\[
    o \times a = (o + o) \times a = o \times a + o \times a.
\]在上式等号两边加上\(o \times a\)的负元\(-o \times a\),得\[
    o \times a + (- o \times a) = (o \times a + o \times a) + (- o \times a),
\]\[
    o = o \times a.
\]
同理,可证\(a \times o = o\).
\end{proof}
\end{theorem}

\begin{example}
设\(R\)是有单位元\(e(\neq o)\)的环.
证明:单位元的负元\((-e)\)满足\[
    (-e)^2=(-e)\times(-e)=e.
\]
\begin{proof}
根据单位元的定义有\[
    e \times (-e) = (-e) \times e = -e.
\]
又由负元的定义有\[
    e + (-e) = o,
\]
在等号左右两边同时右乘\((-e)\),再由右分配律有\[
    -e + (-e)^2 = e \times (-e) + (-e) \times (-e) = o \times (-e) = o,
\]
移项得\((-e)^2 = e\).
\end{proof}
\end{example}

\subsection{可逆元}
\begin{definition}
设\(R\)是有单位元\(e(\neq o)\)的环.
对于\(a \in R\),如果\[
	(\exists b \in R)[a \times b = b \times a = e],
\]
那么称“\(a\)是一个\DefineConcept{可逆元}”,
把\(b\)叫做\(a\)的\DefineConcept{逆元},记作\(a^{-1}\).
\end{definition}

\begin{proposition}
若\(a\)是可逆元,则\(a\)的逆元唯一.
\begin{proof}
设环\(\opair{R,+,\times}\)具有单位元\(e(\neq o)\).
假设环\(\opair{R,+,\times}\)中的任一可逆元\(a\)的逆元为\(b\)和\(c\).
由可逆元的定义,有\[
    a \times b = b \times a = e,
	\qquad
    a \times c = c \times a = e.
\]
又由单位元的定义,有\[
    b = b \times e
    = b \times (a \times c)
    = (b \times a) \times c
    = e \times c
    = c,
\]
这就说明,任一可逆元的逆元是唯一的.
\end{proof}
\end{proposition}

\subsection{零因子}
\begin{definition}
设\(R\)是一个环.
对于\(a \in R\),
如果存在\(c \in R\)且\(c \neq o\),
使得\(a \times c = o\)(或\(c \times a = o\)),
那么称“\(a\)是一个\DefineConcept{左零因子}或\DefineConcept{右零因子}”.
左、右零因子统称为\DefineConcept{零因子}.
\end{definition}

\begin{proposition}
在环\(\opair{R,+,\times}\)中,零元\(o\)是一个零因子.
\end{proposition}

\begin{definition}
设\(o\)是环\(\opair{R,+,\times}\)的零元.
我们把\(o\)称为\DefineConcept{平凡零因子},
把其余的零因子称为\DefineConcept{非平凡零因子}.
\end{definition}

\begin{definition}
如果环\(\opair{R,+,\times}\)中不存在非平凡零因子,
那么称“\(\opair{R,+,\times}\)是\DefineConcept{无零因子环}”.
\end{definition}

\begin{definition}
有单位元的无零因子的交换环,称为\DefineConcept{整环}.
\end{definition}

\begin{example}
整数环\(\mathbb{Z}\)、数域\(K\)上一元多项式环\(K[x]\)、数域\(K\)都是整环.
偶数环\(2\mathbb{Z}\)、数域\(K\)上\(n\)阶全矩阵环\(M_n(K)\)不是整环.
\end{example}

\begin{theorem}
设\(R\)是有单位元\(e(\neq o)\)的环,则\(R\)的零因子不是可逆元.
\begin{proof}
设\(a\)是左零因子,则存在\(c \in R\)且\(c \neq o\),使得\[
    a \times c = o.
\]

假如\(a\)是可逆元,则在上式两边左乘\(a^{-1}\),再根据乘零定理,得\[
    a^{-1} \times (a \times c) = a^{-1} \times o = o.
\]
根据乘法的结合律和逆元的性质,得\[
    a^{-1} \times (a \times c) = (a^{-1} \times a) \times c = e \times c = c.
\]
于是\(c = o\),矛盾,故左零因子\(a\)不是可逆元.
同理可得,右零因子不是可逆元.
\end{proof}
\end{theorem}

上述定理的逆否命题是:
\begin{corollary}
设\(R\)是有单位元\(e(\neq o)\)的环,则\(R\)的可逆元不是零因子.
\end{corollary}

\begin{example}
要想让可逆元等于零元,唯一的办法是在由一个元素\(o\)构成的集合\(R\)上定义加法和乘法如下:\[
    o + o \mapsto o,
    \qquad
    o \times o \mapsto o.
\]
显然元素\(o\)是环\(\opair{R,+,\times}\)的单位元和零元.
再将原本的可逆元的定义中对环\(\opair{R,+,\times}\)的单位元\(e \neq o\)的约束去掉,
那么对于环\(\opair{R,+,\times}\)上唯一的元素\(o\),有\[
    \forall x \in R \bigl( x = o \land x \times o = o \times x = o \bigr),
\]
即元素\(o\)还是环\(\opair{R,+,\times}\)的“可逆元”,它的“逆元”是\(o\)本身.

这就说明,在上面对可逆元的定义中,我们必须强调环\(\opair{R,+,\times}\)上的单位元\(e\)不是零元.
\end{example}
