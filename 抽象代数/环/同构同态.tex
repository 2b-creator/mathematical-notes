\section{同构,同态}
\begin{definition}
%@see: 《近世代数》(熊全淹) P77. 定义
设\(\opair{R,+,\times}\)和\(\opair{R',\oplus,\otimes}\)都是环,
\(\sigma\)是一个从\(R\)到\(R'\)的映射.
如果\[
	(\forall a,b\in R)[
		\sigma(a+b)=\sigma(a)\oplus\sigma(b)
		\land
		\sigma(a \times b)=\sigma(a)\otimes\sigma(b)
	],
\]
那么称“\(\sigma\)是一个从\(\opair{R,+,\times}\)到\(\opair{R',\oplus,\otimes}\)的同态”;
还称“\(\opair{R,+,\times}\)与\(\opair{R',\oplus,\otimes}\)同态”,
记为\(\opair{R,+,\times} \sim \opair{R',\oplus,\otimes}\).

如果\(\sigma\)是单射,则称\(\sigma\)是单同态.
如果\(\sigma\)是满射,则称\(\sigma\)是满同态.
如果\(\sigma\)是双射,则称\(\sigma\)是同构,
并记\(\opair{R,+,\times} \simeq \opair{R',\oplus,\otimes}\).
\end{definition}

任意一个环\(R\)都与由零元组成的零环同态,
因为我们把\(R\)中所有元都与零元对应就是零同态.
