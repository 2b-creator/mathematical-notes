\section{同构,同态}
\begin{definition}
%@see: 《近世代数》(熊全淹) P77 定义
设\(\opair{R,+,\times}\)和\(\opair{R',\oplus,\otimes}\)都是环,
\(\sigma\)是一个从\(R\)到\(R'\)的映射.
如果\[
	(\forall a,b\in R)[
		\sigma(a+b)=\sigma(a)\oplus\sigma(b)
		\land
		\sigma(a \times b)=\sigma(a)\otimes\sigma(b)
	],
\]
那么称“\(\sigma\)是一个从\(\opair{R,+,\times}\)到\(\opair{R',\oplus,\otimes}\)的同态”;
还称“\(\opair{R,+,\times}\)与\(\opair{R',\oplus,\otimes}\)同态”,
记为\(\opair{R,+,\times} \sim \opair{R',\oplus,\otimes}\).

如果\(\sigma\)是单射,则称\(\sigma\)是单同态.

如果\(\sigma\)是满射,则称\(\sigma\)是满同态.

如果\(\sigma\)是双射,则称\(\sigma\)是同构,
并记\(\opair{R,+,\times} \simeq \opair{R',\oplus,\otimes}\).
\end{definition}

任意一个环\(R\)都与由零元组成的零环同态,
因为我们把\(R\)中所有元都与零元对应就是零同态.

跟群同态、群同构一样,环同构是等价关系;
环同态不是等价关系,它只满足自反性、传递性,但不满足对称性.

两个同构的环除了记号外,构造完全一样,
也就是说,它保持原来环中用加法、乘法两种运算表示的一切代数性质,
所以我们有时也把同构的环看成是相同的环.
但环同态就不是这样,它不一定保持原有的性质.
譬如,当\(R \sim R'\)时,假如\(R\)有单位元,那么\(R'\)也有单位元,但反过来就不一定成立.
再假如\(R\)是交换环,那么\(R'\)也是交换环,但反过来也不一定成立.
又假如\(R\)是无零因子环,\(R'\)中可能有零因子;
反过来,假如\(R'\)是无零因子环,\(R\)中也可能有零因子;
因此,\(R\)是整环时,\(R'\)不一定是整环,反过来,\(R'\)是整环时,\(R\)也不一定是整环.

因为环同态是把环看成加群时的同态,
所以它把零元变成零元,把任一元的负元变为这元的像的负元,因此两元的差变为它们的像的差.
又由定义,我们容易得知环同态把幂零元变为幂零元,幂等元变为幂等元.
假如环有单位元,那么单位元也变为单位元,因此,任一元的逆又变为这元的像的逆.

跟群一样,环也有自同态、自同构.
我们容易得知,整数环及有理数域的自同构都只是恒等同构,即任一元与自己对应的同构.
映射\(a+b\iu \mapsto a-b\iu\)是复数域的自同构.
也跟群一样,自同构环也分内自同构和外自同构.

我们把对任意内同构不变的子环,称为\DefineConcept{不变子环}.
