\section{同态}
如果将上一节介绍的同构的定义中用到的“双射”换成一般的映射,
我们就能得到它的推广.

\begin{definition}
%@see: 《近世代数》(熊全淹) P56. 定义
%@see: 《近世代数》(丘维声,2015) P50. 定义1
设\(M\)和\(M'\)都是非空集合,
\(\times\)是定义在\(M\)上的运算,
\(\otimes\)是定义在\(M'\)上的运算,
\(\sigma\)是一个从\(M\)到\(M'\)的映射.
如果\[
	(\forall a,b \in M)[
		\sigma(a \times b) = \sigma(a) \otimes \sigma(b)
	],
\]
那么称“\(\sigma\)是一个%
从\(\opair{M,\times}\)到\(\opair{M',\otimes}\)的%
\DefineConcept{同态}(homomorphism)”;
还称“\(\opair{M,\times}\)与\(\opair{M',\otimes}\)同态”,
记为\(\opair{M,\times} \sim \opair{M',\otimes}\).
%@see: https://www.britannica.com/science/homomorphism
%@see: https://math.libretexts.org/Bookshelves/Abstract_and_Geometric_Algebra/Introduction_to_Algebraic_Structures_(Denton)/04%3A_Groups_III/4.01%3A_Homomorphisms
%@see: https://mathworld.wolfram.com/Homomorphism.html
\end{definition}

如果同态\(\sigma\)是满射,
则称“\(\sigma\)是\DefineConcept{满同态}(epimorphism)”.
%@see: https://mathworld.wolfram.com/Epimorphism.html
如果\(\sigma\)是单射,
则称“\(\sigma\)是\DefineConcept{单同态}(monomorphism)”
%@see: https://mathworld.wolfram.com/Monomorphism.html

显然,当\(\sigma\)是双射时,\(\sigma\)就是同构,
因此同构是同态的特例.

同态这个关系具有自反性、传递性;
但它不具有对称性,即\[
	\opair{M,\times}\sim\opair{M',\otimes}
	\notimplies
	\opair{M',\otimes}\sim\opair{M,\times}.
\]
因此同态不是等价关系.

\begin{property}
%@see: 《近世代数》(丘维声,2015) P51. 性质1
设\(\sigma\)是\(G\)到\(G'\)的同态,
\(e,e'\)分别是\(G,G'\)的单位元,
则\(\sigma(e)=e'\).
\end{property}

\begin{property}
%@see: 《近世代数》(丘维声,2015) P51. 性质2
设\(\sigma\)是\(G\)到\(G'\)的同态,
则\((\forall a \in G)[\sigma(a^{-1})=(\sigma(a))^{-1}]\).
\end{property}

\begin{property}
%@see: 《近世代数》(丘维声,2015) P51. 性质3
设\(\sigma\)是\(G\)到\(G'\)的同态,
则\(G\)的子群\(H\)在\(\sigma\)下的像\(\sigma(H)\)是\(G'\)的子群.
特别地,\(\sigma(G)\)是\(G'\)的子群.
\begin{proof}
设\(e,e'\)分别是\(G,G'\)的单位元.
由于\(\sigma(e)=e'\),因此\(e'\in\sigma(H)\).
从\(\sigma(H)\)中任取\(\sigma(a),\sigma(b)\),其中\(a,b\in H\).
由于\(\sigma(a) \otimes (\sigma(b))^{-1}
= \sigma(a) \otimes \sigma(b^{-1})
= \sigma(a \times b^{-1})\),
且\(a \times b^{-1} \in H\),
因此\(\sigma(a) \otimes (\sigma(b))^{-1} \in \sigma(H)\).
从而\(\sigma(H)\)是\(G'\)的一个子群.
\end{proof}
\end{property}

\begin{property}
%@see: 《近世代数》(丘维声,2015) P51. 性质4
设\(\sigma\)是\(G\)到\(G'\)的同态,
\(e,e'\)分别是\(G,G'\)的单位元.
若\(a \in G\)且\(a^n = e\),则\((\sigma(a))^n = e'\).
于是,若\(a\)是\(G\)的\(n\)阶元,则\(\sigma(a)\)的阶是\(n\)的一个因数.
\end{property}

\begin{definition}
%@see: 《近世代数》(丘维声,2015) P51. 定义2
设\(\sigma\)是群\(G\)到\(G'\)的一个同态.
我们把\(G'\)的单位元\(e'\)在\(\sigma\)下的原像集,
称为“\(\sigma\)的\DefineConcept{核}(kernel)”,
记作\(\ker\sigma\),即\[
	\ker\sigma \defeq \Set{ a \in G \given \sigma(a) = e' }.
\]
\end{definition}

\begin{definition}
设\(\sigma\)是一个从\(\opair{M,\times}\)到\(\opair{M',\otimes}\)的同态.
如果\(M' \subseteq M\),也就是说,\(\sigma\ImageOfSetUnderRelation{M}\subseteq M\),
那么称“\(\sigma\)是\(\opair{M,\times}\)的\DefineConcept{自同态}(endomorphism)”.
%@see: https://mathworld.wolfram.com/Endomorphism.html
\end{definition}

\begin{example}\label{example:抽象代数.零同态}
设\(\opair{G,\times}\)是一个群,\(e\)是它的单位元.
证明:映射\(\sigma\colon\opair{G,\times}\to\opair{\{e\},\times},a\mapsto e\)是一个同态.
\begin{proof}
由于\[
	(\forall a,b\in G)[
		\sigma(a \times b)
		= e
		= e \times e
		= \sigma(a) \times \sigma(b)
	],
\]
所以\(\sigma\)是同态.
\end{proof}
\end{example}
我们把\cref{example:抽象代数.零同态} 中的同态称为\DefineConcept{零同态}.

\begin{theorem}\label{theorem:抽象代数.群的同态象是群}
%@see: 《近世代数》(熊全淹) P57. 定理1
设\(\opair{G,\times}\)是群,
\(G'\)是非空集合,
\(\otimes\)是定义在\(G'\)上的运算.
如果\(\opair{G,\times}\sim\opair{G',\otimes}\),
那么\(\opair{G',\otimes}\)也是群.
\begin{proof}
设\(\sigma\)是从\(\opair{G,\times}\)到\(\opair{G',\otimes}\)的同态,
\(e\)是\(\opair{G,\times}\)的单位元.
任意取定\(a,b,c\in G\),
并假设\[
	\sigma(a)=a', \qquad
	\sigma(b)=b', \qquad
	\sigma(c)=c', \qquad
	\sigma(a^{-1})=(a^{-1})', \qquad
	\sigma(e)=e'.
\]
那么\[
	(a \times b)\times c = a \times(b \times c)
	\implies
	(a' \otimes b')\otimes c' = a' \otimes(b' \otimes c');
\]
可见,在\(G'\)中结合律成立.
又由\(e\times a=a\),
有\(e'\otimes a'=a'\);
再由\(a^{-1}\times a=e\),
有\((a^{-1})'\otimes a'=e'\);
这就是说,\(e'\)是\(\opair{G',\otimes}\)的单位元,
并且运算\(\otimes\)可逆.
综上所述\(\opair{G',\otimes}\)是群.
\end{proof}
\end{theorem}

\begin{theorem}
%@see: 《近世代数》(熊全淹) P57. 定理2
设\(\opair{G,\times}\)和\(\opair{G',\otimes}\)都是群,
\(\opair{G,\times}\sim\opair{G',\otimes}\),
那么\(\opair{G',\otimes}\)的单位元\(e'\)在\(G\)的完全象源\(E\)是\(G\)的正规子群.
\end{theorem}
