\section{同态}
如果将上一节介绍的同构的定义中用到的“双射”换成一般的映射,
我们就能得到它的推广.

\begin{definition}
设\(M\)和\(M'\)都是非空集合,
\(\times\)是定义在\(M\)上的运算,
\(\otimes\)是定义在\(M'\)上的运算,
\(\sigma\)是一个从\(M\)到\(M'\)的映射.
如果\[
	(\forall a,b \in M)[
		\sigma(a \times b) = \sigma(a) \otimes \sigma(b)
	],
\]
那么称“\(\sigma\)是一个%
从\(\opair{M,\times}\)到\(\opair{M',\otimes}\)的%
\DefineConcept{同态}(homomorphism)”;
还称“\(\opair{M,\times}\)与\(\opair{M',\otimes}\)同态”,
记为\(\opair{M,\times} \sim \opair{M',\otimes}\).
%@see: https://www.britannica.com/science/homomorphism
%@see: https://math.libretexts.org/Bookshelves/Abstract_and_Geometric_Algebra/Introduction_to_Algebraic_Structures_(Denton)/04%3A_Groups_III/4.01%3A_Homomorphisms
\end{definition}

显然,当\(\sigma\)是双射时,\(\sigma\)就是同构,
因此同构是同态的特例.

同态这个关系具有自反性、传递性;
但它不具有对称性,即\[
	\opair{M,\times}\sim\opair{M',\otimes}
	\notimplies
	\opair{M',\otimes}\sim\opair{M,\times}.
\]
因此同态不是等价关系.

\begin{definition}
设\(\sigma\)是一个从\(\opair{M,\times}\)到\(\opair{M',\otimes}\)的同态.
如果\(M' \subseteq M\),也就是说,\(\sigma\ImageOfSetUnderRelation{M}\subseteq M\),
那么称“\(\sigma\)是\(\opair{M,\times}\)的\DefineConcept{自同态}”.
\end{definition}

\begin{definition}
设\(\sigma\)是群\(\opair{M,\times}\)的\DefineConcept{自同态},
它满足\[
	(\forall a\in M)[\sigma(a)=e],
\]
其中\(e\)是\(\opair{M,\times}\)的单位元,
那么称“\(\sigma\)是\(\opair{M,\times}\)的\DefineConcept{零同态}”.
\end{definition}
