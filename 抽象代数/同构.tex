\section{同构}
\subsection{同构}
\begin{definition}
设\(M\)和\(M'\)都是非空集合,
\(\times\)是定义在\(M\)上的运算,
\(\otimes\)是定义在\(M'\)上的运算,
\(\sigma\)是一个从\(M\)到\(M'\)的双射.
如果\[
	(\forall a,b \in M)[\sigma(a \times b) = \sigma(a) \otimes \sigma(b)],
\]
那么称“\(\sigma\)是一个%
从\(\opair{M,\times}\)到\(\opair{M',\otimes}\)的%
\DefineConcept{同构}(isomorphism)”;
还称“\(\opair{M,\times}\)与\(\opair{M',\otimes}\)同构%
(\(\opair{M,\times}\) is \emph{isomorphic} to \(\opair{M',\otimes}\))”,
记为\(\opair{M,\times} \simeq \opair{M',\otimes}\).

特别地,当\(M=M'\)且\(\times=\otimes\)时,
称“\(\sigma\)是一个\(\opair{M,\times}\)的\DefineConcept{自同构}(automorphism)”.
%@see: https://mathworld.wolfram.com/Isomorphism.html
%@see: https://www.britannica.com/science/isomorphism-mathematics
%@see: https://mathworld.wolfram.com/Automorphism.html
\end{definition}

可以注意到,由于双射是等价关系,因此同构也是等价关系.

\begin{example}
设\(\opair{G,\times}\)是群,
\(u\in G\).
定义运算:\[
	(\forall a,b\in G)[
		a \cdot b
		= a \times u^{-1} \times b
	].
\]
证明:
\(\opair{G,\cdot}\)是与\(\opair{G,\times}\)同构的群,
\(u\)是\(\opair{G,\cdot}\)的单位元,
映射\(\sigma(a) = a \times u\)是从\(\opair{G,\times}\)到\(\opair{G,\cdot}\)的同构映射.
%TODO
\end{example}
%上例是1942年汤瑄真(1898--1951)发表的一个定理.

\begin{example}\label{example:抽象代数.恒等映射是自同构}
证明:恒等映射是自同构.
\begin{proof}
设\(M\)是非空集合,\(\times\)是定义在\(M\)上的运算.
因为\(\sigma\)是\(M\)的恒等映射,即\[
	(\forall a\in M)[\sigma(a) = a],
\]
所以\[
	(\forall a,b\in M)[
		\sigma(a \times b) = a \times b = \sigma(a) \times \sigma(b).
	],
\]
因此\(\sigma\)是\(M\)的自同构.
\end{proof}
\end{example}

\begin{definition}
设\(M\)是非空集合,\(\times\)是定义在\(M\)上的运算,
我们把\(M\)的恒等映射\(\sigma\)称为
“\(M\)的\DefineConcept{恒等同构}(identity automorphism)”.
\end{definition}

假如两个群\(\opair{G,\times}\)和\(\opair{G',\otimes}\)同构,
那么对于任意一条命题公式,
如果这条公式只与它们的元素和运算有关,
它们就都满足这条公式,
从而它们在本质上没有区别,
我们只靠群的性质是无法区分它们的,
于是同构的群就可以看成是相同的群.
因此,在研究群\(\opair{G,\times}\)的性质时,
我们可以寻找另一个群\(\opair{G',\otimes}\),
只要能够证明两者同构,
并且研究清楚后者的性质,
那么我们就可以说前者的性质也已经研究清楚了.
同构之所以重要,主要在此.
于是研究一个群时,我们常常把它分成若干个不同构的类来讨论.
但要注意的是,同构的群与相同的群是由区别的;
例如,全体整数的加群\(\opair{\mathbb{Z},+}\)与全体偶数的加群\(\opair{2\mathbb{Z},+}\)同构,
但是后者是前者的真子群.

\begin{example}
设\(\opair{G,\times}\)是交换群,
\(e\)是\(\opair{G,\times}\)的单位元,
映射\(\sigma\colon G\to G, a\mapsto a^{-1}\).
证明:\(\sigma\)是\(\opair{G,\times}\)的自同构.
\begin{proof}
根据\cref{theorem:抽象代数.群内任一元的逆元唯一,theorem:抽象代数.群内任一元的逆的逆是它本身},
可知\(\sigma\)是双射.
任意取定\(a,b\in G\),则有
\begin{align*}
	\sigma(a \times b)
	&= (a \times b)^{-1} \\
	&= b^{-1} \times a^{-1}
		&(\text{\cref{theorem:抽象代数.群内元素的乘积的逆}}) \\
	&= a^{-1} \times b^{-1}
		&(\text{\cref{definition:抽象代数.交换群的定义}}) \\
	&= \sigma(a) \times \sigma(b).
\end{align*}
综上所述,\(\sigma\)是\(\opair{G,\times}\)的自同构.
\end{proof}
\end{example}

\subsection{常见的同构}
我们在前面几节基于集合论构造了自然数集\(\omega\)和整数集\(\mathbb{Z}\),
尽管自然数集\(\omega\)并非整数集\(\mathbb{Z}\)的子集,
但是我们可以在整数集中找到一个子集\(\Set{ x \in \mathbb{Z} \given x \geqslant 0 }\),
它的行为表现都和\(\omega\)一模一样.
准确地说,我们定义映射\[
	E\colon \omega\to\mathbb{Z}, n\mapsto[\opair{n,0}].
\]
那么我们有\(E(0)=0\),\(E(1)=1\).
我们下面来证明,\(E\)是一个同构嵌入(isomorphic embedding),%@see: https://mathworld.wolfram.com/Embedding.html
它将\(\opair{\omega,+,\times,<}\)嵌入\(\opair{\mathbb{Z},+,\times,<}\).
\begin{theorem}
\(E\)是一一映射,对\(\forall m,n\in\omega\)总满足
\begin{gather}
	E(m+n)=E(m)+E(n),
	\label{equation:集合论.自然数集到整数集的同构嵌入1} \\
	E(m \times n)=E(m) \times E(n),
	\label{equation:集合论.自然数集到整数集的同构嵌入2} \\
	m<n \iff E(m)<E(n).
	\label{equation:集合论.自然数集到整数集的同构嵌入3}
\end{gather}
\begin{proof}
因为\begin{align*}
	E(m)=E(n)
	&\implies
	[\opair{m,0}]=[\opair{n,0}] \\
	&\implies
	\opair{m,0}\sim\opair{n,0} \\
	&\implies
	m=n,
\end{align*}
所以\(E\)是一一映射.

对\cref{equation:集合论.自然数集到整数集的同构嵌入1} 证明如下:
\begin{align*}
	E(m+n)
	&= [\opair{m+n,0}] \\
	&= [\opair{m,0}]+[\opair{n,0}] \\
	&= E(m)+E(n).
\end{align*}

对\cref{equation:集合论.自然数集到整数集的同构嵌入2} 证明如下:
\begin{align*}
	E(m \times n)
	&= [\opair{m \times n,0}] \\
	&= [\opair{m,0}]\times[\opair{n,0}] \\
	&= E(m) \times E(n).
\end{align*}

对\cref{equation:集合论.自然数集到整数集的同构嵌入3} 证明如下:
\begin{align*}
	m<n
	&\iff
	[\opair{m,0}]<[\opair{n,0}] \\
	&\iff
	E(m)<E(n).
	\qedhere
\end{align*}
\end{proof}
\end{theorem}

\subsection{自同构群}
\begin{theorem}\label{theorem:抽象代数.自同构的复合满足结合律}
自同构的复合满足结合律.
\begin{proof}
设\(G\)是非空集合,
\(\times\)是定义在\(G\)上的一个二元代数运算.
假设\(\rho,\sigma,\tau\)是\(\opair{G,\times}\)的自同构.
因为对于任意\(a\in G\),总有\[
	((\rho\circ\sigma)\circ\tau)(a)
	= (\rho\circ\sigma)(\tau(a))
	= \rho(\sigma(\tau(a)))
	= \rho((\sigma\circ\tau)(a))
	= (\rho\circ(\sigma\circ\tau))(a),
\]
也即\((\rho\circ\sigma)\circ\tau=\rho\circ(\sigma\circ\tau)\),
所以\(\circ\)满足结合律.
\end{proof}
\end{theorem}

\begin{theorem}\label{theorem:抽象代数.两个自同构的复合是自同构}
两个自同构的复合也是自同构.
\begin{proof}
设\(G\)是非空集合,
\(\times\)是定义在\(G\)上的一个二元代数运算.
假设\(\sigma,\tau\)都是\(\opair{G,\times}\)的自同构.
因为对于任意\(a,b\in G\),总有\[
	(\sigma\circ\tau)(a \times b)
	= \sigma(\tau(a \times b))
	= \sigma(\tau(a)\times\tau(b))
	= \sigma(\tau(a))\times\sigma(\tau(b))
	= (\sigma\circ\tau)(a)\times(\sigma\circ\tau)(b),
\]
所以\(\sigma\circ\tau\)是\(\opair{G,\times}\)的自同构.
\end{proof}
\end{theorem}

\begin{theorem}\label{theorem:抽象代数.自同构的逆是自同构}
自同构的逆也是自同构.
\begin{proof}
设\(G\)是非空集合,
\(\times\)是定义在\(G\)上的一个二元代数运算.
假设\(\sigma\)是\(\opair{G,\times}\)的自同构.
因为对于任意\(a,b\in G\),总有\begin{align*}
	\sigma^{-1}(a \times b)
	&= \sigma^{-1}((\sigma\circ\sigma^{-1})(a)\times(\sigma\circ\sigma^{-1})(b)) \\
	&= \sigma^{-1}(\sigma(\sigma^{-1}(a)\times\sigma^{-1}(b))) \\
	&= \sigma^{-1}(a)\times\sigma^{-1}(b),
\end{align*}
所以\(\sigma\)的逆\(\sigma^{-1}\)也是\(\opair{G,\times}\)的自同构.
\end{proof}
\end{theorem}

\begin{theorem}
%@see: 《近世代数》(熊全淹) P49. 定理2
自同构对复合成群.
\begin{proof}
设\(G\)是非空集合,
\(\times\)是定义在\(G\)上的一个二元代数运算,\[
	S = \Set{ \sigma \given \text{\(\sigma\)是\(\opair{G,\times}\)的自同构} }.
\]
由\cref{example:抽象代数.恒等映射是自同构},
恒等同构\(i\)是\(\opair{S,\circ}\)的单位元,即对于任意映射\(\sigma\),有\[
	i \circ \sigma = \sigma \circ i = \sigma.
\]
那么根据\cref{theorem:抽象代数.自同构的复合满足结合律,%
theorem:抽象代数.两个自同构的复合是自同构,%
theorem:抽象代数.自同构的逆是自同构},
\(\opair{S,\circ}\)是一个群.
\end{proof}
\end{theorem}

\subsection{内同构}
下面我们来介绍群的一种重要的自同构.

设\(\opair{G,\times}\)是群,\(a\in G\).
那么映射\(\sigma\colon G\to G, g\mapsto a\times g\times a^{-1}\)
是\(\opair{G,\times}\)的自同构.
这是因为,\(g\)的原像是\(\sigma^{-1}(g)=a^{-1}\times g\times a\).
如果\(a\times g\times a^{-1} = a\times h\times a^{-1}\),
那么\(g=h\),
所以\(\sigma\)是双射.
再从\(\sigma(g)=a\times g\times a^{-1}\),
\(\sigma(h)=a\times h\times a^{-1}\),
我们就有\[
	\sigma(g\times h)
	= a \times g \times h \times a^{-1}
	= (a \times g \times a^{-1}) \times (a \times h \times a^{-1})
	= \sigma(g) \times \sigma(h).
\]

像这样,由一个给定的元\(a\)决定的自同构\(\sigma\),
称为\DefineConcept{内自同构}(inner automorphism),
简称\DefineConcept{内同构}.
不是内自同构的自同构称为\DefineConcept{外自同构}(outer automorphism),
简称\DefineConcept{外同构}.
把\(\sigma(a)\)叫做“\(g\)用\(a\)得到的\DefineConcept{变形}”,
还称“\(g\)与\(\sigma(g)\) \DefineConcept{共轭}”.
%@see: https://mathworld.wolfram.com/InnerAutomorphism.html
%@see: https://mathworld.wolfram.com/OuterAutomorphism.html

\begin{theorem}
两个内同构的复合也是内同构.
\begin{proof}
设\(\opair{G,\times}\)是群,\(a,b\in G\),那么映射\[
	\sigma\colon G\to G, g\mapsto a\times g\times a^{-1}
	\quad\text{和}\quad
	\tau\colon G\to G, g\mapsto b\times g\times b^{-1}
\]是内同构.
因为\begin{align*}
	(\sigma\circ\tau)(g)
	&=\sigma(\tau(g))
	=\sigma(b\times g\times b^{-1}) \\
	&=a\times(b\times g\times b^{-1})\times a^{-1} \\
	&=(a\times b)\times g\times(b^{-1}\times a^{-1}) \\
	&=(a\times b)\times g\times(a\times b)^{-1},
\end{align*}
所以\(\sigma\circ\tau\)也是内同构.
\end{proof}
\end{theorem}

\subsection{内同构群}
\begin{theorem}
群的内同构对复合成群.
%TODO
\end{theorem}

\begin{theorem}
%@see: 《近世代数》(熊全淹) P51. 定理3
群的内同构群是它的自同构群的正规子群.
%TODO
\end{theorem}
