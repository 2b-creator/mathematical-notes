\section{同构}
\subsection{同构}
\begin{definition}
设\(M\)和\(M'\)都是非空集合,
\(\times\)是定义在\(M\)上的运算,
\(\otimes\)是定义在\(M'\)上的运算,
\(\sigma\)是一个从\(M\)到\(M'\)的双射,
且\[
	(\forall a,b \in M)[\sigma(a \times b) = \sigma(a) \otimes \sigma(b)],
\]
那么称“\(\sigma\)是一个从\(\opair{M,\times}\)到\(\opair{M',\otimes}\)的\DefineConcept{同构}”;
还称“\(\opair{M,\times}\)与\(\opair{M',\otimes}\)同构”,
记为\(\opair{M,\times} \simeq \opair{M',\otimes}\).

特别地,当\(M=M'\)时,
称“\(\sigma\)是一个从\(\opair{M,\times}\)到\(\opair{M,\otimes}\)的\DefineConcept{自同构}”.
\end{definition}

可以注意到,双射是等价关系,因此同构也是等价关系.
恒等映射是自同构.

\subsection{常见的同构}
我们在前面几节基于集合论构造了自然数集\(\omega\)和整数集\(\mathbb{Z}\),
尽管自然数集\(\omega\)并非整数集\(\mathbb{Z}\)的子集,
但是我们可以在整数集中找到一个子集\(\Set{ x \in \mathbb{Z} \given x \geqslant 0 }\),
它的行为表现都和\(\omega\)一模一样.
准确地说,我们定义映射\[
	E\colon \omega\to\mathbb{Z}, n\mapsto[\opair{n,0}].
\]
那么我们有\(E(0)=0\),\(E(1)=1\).
我们下面来证明,\(E\)是一个同构嵌入(isomorphic embedding),%@see: https://mathworld.wolfram.com/Embedding.html
它将\(\opair{\omega,+,\times,<}\)嵌入\(\opair{\mathbb{Z},+,\times,<}\).
\begin{theorem}
\(E\)是一一映射,对\(\forall m,n\in\omega\)总满足
\begin{gather}
	E(m+n)=E(m)+E(n),
	\label{equation:集合论.自然数集到整数集的同构嵌入1} \\
	E(m \times n)=E(m) \times E(n),
	\label{equation:集合论.自然数集到整数集的同构嵌入2} \\
	m<n \iff E(m)<E(n).
	\label{equation:集合论.自然数集到整数集的同构嵌入3}
\end{gather}
\begin{proof}
因为\begin{align*}
	E(m)=E(n)
	&\implies
	[\opair{m,0}]=[\opair{n,0}] \\
	&\implies
	\opair{m,0}\sim\opair{n,0} \\
	&\implies
	m=n,
\end{align*}
所以\(E\)是一一映射.

对\cref{equation:集合论.自然数集到整数集的同构嵌入1} 证明如下:
\begin{align*}
	E(m+n)
	&= [\opair{m+n,0}] \\
	&= [\opair{m,0}]+[\opair{n,0}] \\
	&= E(m)+E(n).
\end{align*}

对\cref{equation:集合论.自然数集到整数集的同构嵌入2} 证明如下:
\begin{align*}
	E(m \times n)
	&= [\opair{m \times n,0}] \\
	&= [\opair{m,0}]\times[\opair{n,0}] \\
	&= E(m) \times E(n).
\end{align*}

对\cref{equation:集合论.自然数集到整数集的同构嵌入3} 证明如下:
\begin{align*}
	m<n
	&\iff
	[\opair{m,0}]<[\opair{n,0}] \\
	&\iff
	E(m)<E(n).
	\qedhere
\end{align*}
\end{proof}
\end{theorem}
