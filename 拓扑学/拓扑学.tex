\chapter{拓扑学}
拓扑学是极限的推广,通常分为点集拓扑学和代数拓扑学.
本章主要介绍点集拓扑学,它为几何学提供了基本语言.

\section{度量空间}
根据\cref{definition:极限.函数在一点的连续性} 我们知道,
函数\(f\colon\mathbb{R}\to\mathbb{R}\)在点\(x_0\in\mathbb{R}\)处连续当且仅当\[
	(\forall\epsilon>0)
	(\exists\delta>0)
	[
		\abs{x - x_0}<\delta
		\implies
		\abs{f(x) - f(x_0)} < \epsilon
	].
\]
在这个定义中只涉及两个实数之间的距离(即两个实数之差的绝对值)这个概念.
为了验证一个函数在某点处的连续性往往只要用到关于上述距离的最基本的性质,而与实数的其他性质无关.
关于多元函数的连续性情形也完全类似.
在此之前,我们一直是依靠几何直觉理解“距离”的概念,从现在开始,我们要抽象出度量和度量空间的概念.

\subsection{度量与度量空间的概念}
\begin{definition}
%@see: 《数学分析》(卓里奇) P1. 定义1.
%@see: 《点集拓扑讲义》(熊金城) P45. 定义2.1.1
设\(X\)是一个集合,映射\(\rho\colon X \times X\to\mathbb{R}\).
如果对于\(\forall x,y,z \in X\),总有\begin{enumerate}
	\item \(\rho(x,y)=0 \iff x=y\);

	\item {\bf 对称性},即\[
		\rho(x,y) = \rho(y,x);
	\]

	\item {\bf 三角不等式},即\[
		\rho(x,z) \leq \rho(x,y) + \rho(y,z);
	\]
\end{enumerate}
那么称“映射\(\rho\)是集合\(X\)的一个\DefineConcept{度量}(metric)”;
称实数\(\rho(x,y)\)为“从点\(x\)到点\(y\)的\DefineConcept{距离}(distance)”.
\end{definition}
% TODO 注意将“度量”与“测度(measure)”作区别
%@see: https://math.stackexchange.com/questions/1402847/
%@see: https://mathworld.wolfram.com/Measure.html

我们指出,如果在度量的第三个条件的三角不等式中取\(x=z\),便得\[
	\rho(x,x)\leq\rho(x,y)+\rho(y,x),
\]
则利用度量的前两个条件可以得到\(\rho(x,x)=0,\rho(y,x)=\rho(x,y)\),
于是\(0\leq2\rho(x,y)\)
即\(\rho(x,y)\geq0\).
也就是说,对于任意两点\(x,y\),它们的距离是非负的.

应该注意到,给定任意一个集合,我们总可以找出无穷多个满足上述三个条件的映射.
例如,给定集合\(X\),对于\(x,y \in X\),令\[
	d(x,y) = \left\{ \begin{array}{cl}
		c, & x \neq y, \\
		0, & x=y,
	\end{array} \right.
\]
可以看出,对于任意\(c\in\mathbb{R}^+\),
映射\(d\)总是集合\(X\)的一个度量.

\begin{definition}
给定集合\(X\),如果映射\(\rho\)是集合\(X\)的一个度量,
那么称“集合\(X\)是一个对于度量\(\rho\)而言的\DefineConcept{度量空间}(metric space)”,
记为\((X,\rho)\).
%@see: https://mathworld.wolfram.com/MetricSpace.html
\end{definition}

当前文已经说明了度量\(\rho\),省略它不至于引起混淆时,可以简称“\(X\)是一个度量空间”.

\subsection{常见的度量空间}
\begin{example}[实数空间\(\mathbb{R}\)]
对于实数集\(\mathbb{R}\),
定义映射\(\rho\colon\mathbb{R}\times\mathbb{R}\to\mathbb{R}\)如下:\[
	\rho(x,y) = \abs{x-y},
	\quad x,y\in\mathbb{R}.
\]
显然\(\rho\)是\(\mathbb{R}\)的一个度量,
因此\((\mathbb{R},\rho)\)是一个度量空间.
特别地,这个度量空间被称为\DefineConcept{实数空间}或\DefineConcept{直线},
称度量\(\rho\)为“\(\mathbb{R}\)的\DefineConcept{通常度量}(usual metric)”.
\end{example}

我们可以假设一个定义在\([0,+\infty)\)上的非负函数\(f(x)\),
当且仅当\(x=0\)时\(f(x)=0\).
如果函数\(f(x)\)严格上凸,
则对于\(\forall x,y\in\mathbb{R}\),只要取\[
	d(x,y)=f(\abs{x-y}),
\]
就得到\(\mathbb{R}\)的一个度量.

特别地,可以取\(d(x,y)=\sqrt{\abs{x-y}}\)
或\(d(x,y)=\frac{\abs{x-y}}{1+\abs{x-y}}\).

可以验证,在度量\(d(x,y)=\frac{\abs{x-y}}{1+\abs{x-y}}\)下,
数轴上任意两点之间的距离都小于\(1\).

\begin{example}[欧式空间\(\mathbb{R}^n\)]
对于实数集\(\mathbb{R}\)的\(n\)重笛卡尔积\(\mathbb{R}^n\),
定义映射\(\rho\colon\mathbb{R}^n\times\mathbb{R}^n\to\mathbb{R}\)如下:\[
	\rho(\vb{x},\vb{y})
	= \sqrt{\sum\limits_{i=1}^n (x_i-y_i)^2},
	\quad \vb{x}=(\AutoTuple{x}{n}),\vb{y}=(\AutoTuple{y}{n})\in\mathbb{R}^n.
\]
显然\(\rho\)是\(\mathbb{R}^n\)的一个度量,
因此\((\mathbb{R}^n,\rho)\)是一个度量空间.
特别地,这个度量空间被称为(\(n\)维)\DefineConcept{欧式空间},
称度量\(\rho\)为\(\mathbb{R}^n\)的\DefineConcept{通常度量}.
2维欧式空间通常称为(欧式)平面.
\end{example}

\begin{example}
对于\(\mathbb{R}^n\),除了通常度量以外,我们还可以定义\[
	d_p(\vb{x},\vb{y})
	= \left(\sum\limits_{i=1}^n \abs{x_i-y_i}^p\right)^{\frac{1}{p}},
	\quad \vb{x}=(\AutoTuple{x}{n}),\vb{y}=(\AutoTuple{y}{n})\in\mathbb{R}^n,
\]
其中\(p\geq1\).
利用\hyperref[theorem:不等式.闵可夫斯基不等式]{闵可夫斯基不等式}%
可以证明\(d_p\)是\(\mathbb{R}^n\)的一个度量,
因此\((\mathbb{R}^n,d_p)\)也是一个度量空间,
把\(d_p\)称为\(\mathbb{R}^n\)的\DefineConcept{闵氏度量}.
\end{example}

\begin{example}
对于闭区间上的连续函数族\(C[a,b]\),任给其中两个函数\(f,g\),
定义:\[
	d(f,g)=\max_{a \leq x \leq b} \abs{f(x)-g(x)}.
\]
我们把\(d\)称为\(C[a,b]\)的\DefineConcept{一致度量}%
或\DefineConcept{一致收敛性度量}%
或\DefineConcept{切比雪夫度量}.
在利用多项式代替任意给定函数以所需精度进行近似计算时,
可以用度量\(d\)刻画近似计算得精度.

我们还可以定义\[
	d_p(f,g)=\left[\int_a^b\abs{f-g}^p(x)\dd{x}\right]^{\frac{1}{p}}.
\]
当\(p=1\)时,我们把\(d_p\)称为\DefineConcept{积分度量};
当\(p=2\)时,我们把它称为\DefineConcept{均方差度量};
当\(p=+\infty\)时,我们把它称为\DefineConcept{一致度量}.

我们常把度量空间\((C[a,b],d_p)\)简记为\(C_p[a,b]\),
把度量空间\((C[a,b],d)\)简记为\(C_\infty[a,b]\).
\end{example}

\begin{example}[希尔伯特空间\(\mathbb{H}\)]
构造由所有的平方收敛的实数序列构成的集合,并记为\[
	\mathbb{H}
	= \Set*{
		\vb{x}=(\AutoTuple{x}{0})
		\given
		x_i\in\mathbb{R},
		i\in\mathbb{N}^+;
		\sum\limits_{i=1}^\infty x_i^2<\infty
	}.
\]
定义映射\(\rho\colon\mathbb{H}\times\mathbb{H}\to\mathbb{R}\)如下:\[
	\rho(\vb{x},\vb{y}) = \sqrt{\sum\limits_{i=1}^\infty (x_i-y_i)^2},
	\quad \vb{x}=(\AutoTuple{x}{0}),\vb{y}=(\AutoTuple{y}{0})\in\mathbb{H}.
\]
可以证明\(\rho\)是\(\mathbb{H}\)的一个度量,
因此\((\mathbb{H},\rho)\)是一个度量空间.
特别地,这个度量空间被称为\DefineConcept{希尔伯特空间},
称度量\(\rho\)为\(\mathbb{H}\)的\DefineConcept{通常度量}.
\end{example}

\begin{example}[离散度量空间]
设\((X,\rho)\)是一个度量空间.
如果总有\[
	(\forall x \in X)
	(\exists \delta_x > 0)
	(\forall y \in X - \{x\})
	[\rho(x,y) > \delta_x]
\]成立,则称\((X,\rho)\)是离散的.

例如,设\(X\)是任意一个集合,映射\(\rho\colon X \times X\to\mathbb{R}\)满足\[
\rho(x,y) = \left\{ \begin{array}{ll}
0, & x=y, \\
1, & x\neq y.
\end{array} \right.
\]容易验证\(\rho\)是\(X\)的一个离散的度量,或者说度量空间\((X,\rho)\)是离散的.
\end{example}

\subsection{球形邻域的概念与性质}
\begin{definition}\label{definition:度量空间.球形邻域的概念}
设\((X,\rho)\)是一个度量空间,\(x \in X\).
对于\(\forall\epsilon>0\),集合\[
	\Set{ y \in X \given \rho(x,y) < \epsilon }
\]
称为“一个以\(x\)为\DefineConcept{中心}、以\(\epsilon\)为\DefineConcept{半径}的\DefineConcept{球形邻域}”
或“\(x\)的一个\(\epsilon\)-邻域”,
记作\(B(x,\epsilon)\)或\(B_{\epsilon}(x)\);
不特别强调球形邻域的半径\(\epsilon\)时,
可以简称其为“\(x\)的一个球形邻域”.
\end{definition}

\begin{theorem}\label{theorem:度量空间.球形邻域的性质}
设\((X,\rho)\)是一个度量空间.
取\(x \in X\).
\(x\)的球形邻域具有以下基本性质:
\begin{enumerate}
	\item 点\(x\)至少有一个球形邻域,
	并且点\(x\)属于它的每一个球形邻域;
	\item 对于点\(x\)的任意两个球形邻域,
	存在\(x\)的球形邻域同时包含于两者;
	\item 如果\(y \in X\)属于\(x\)的某一个球形邻域,
	则\(y\)有一个球形邻域包含于\(x\)的这个球形邻域.
\end{enumerate}
\end{theorem}

\subsection{开集的概念与性质}
\begin{definition}\label{definition:度量空间.开集的概念}
设\(A\)是度量空间\(X\)的一个子集.
如果\(A\)中的每一个点都有一个球形邻域包含于\(A\),
即\[
	(\forall a \in A)
	(\exists\epsilon>0)
	[B(a,\epsilon) \subseteq A],
\]
则称“\(A\)是度量空间\(X\)中的一个\DefineConcept{开集}(open set)”.
\end{definition}

\begin{example}
实数空间\(\mathbb{R}\)中,所有的开区间,不论是有限的还是无限的,都是开集;
闭区间或半开半闭区间都不是\(\mathbb{R}\)中的开集.
\end{example}

\begin{theorem}\label{theorem:度量空间.开集的性质}
度量空间\(X\)中的开集具有以下性质:
\begin{enumerate}
\item 集合\(X\)本身和空集\(\emptyset\)都是开集;
\item 任意两个开集的交也是一个开集;
\item 任意一个开集族的并是一个开集;
\item 任意一个球形邻域都是开集.
\end{enumerate}
\end{theorem}

\begin{proposition}
设\((X,\rho)\)是一个度量空间,\(x \in X\).
对于\(\forall\epsilon>0\),集合\[
	\Set{ y \in X \given \rho(x,y) > \epsilon }
\]总是度量空间\(X\)的一个开集.
\end{proposition}

有时候为了方便讨论问题,我们将球形邻域的概念稍稍作一点推广.
\begin{definition}\label{definition:度量空间.邻域的概念}
设\(x\)是度量空间\(X\)中的一个点,集合\(U \subseteq X\).
如果存在一个开集\(V\)满足条件\(x \in V \subseteq U\),
就称“\(U\)是点\(x\)的一个\DefineConcept{邻域}”.
\end{definition}
下面这个定理为邻域的定义提供了一个等价的说法,并且表明从球形邻域推广到邻域是自然的事情.
\begin{theorem}
设\(x\)是度量空间\(X\)中的一个点,则\(X\)的子集\(U\)是\(x\)的一个邻域的充要条件是:
\(x\)有某一个球形邻域包含于\(U\).
\end{theorem}

\subsection{度量空间的直积}
\begingroup
\def\A{X_1}\def\B{X_2}
\def\dA{d_1}\def\dB{d_2}
\def\X{(x_1,x_2)}
\def\Y{(x_1',x_2')}
\def\dAA{d_1(x_1,x_1')}
\def\dBB{d_2(x_2,x_2')}
设\((X_1,d_1)\)和\((X_2,d_2)\)是两个度量空间,
就可以在直积\(X_1 \times X_2\)中引入度量\(d\).
最常用的方法如下.

设\(\X,\Y\in X_1 \times X_2\).
取\[
	d(\X,\Y)
	\defeq
	\sqrt{[\dAA]^2+[\dBB]^2},
\]
或\[
	d(\X,\Y)
	\defeq
	\dAA+\dBB,
\]
或\[
	d(\X,\Y)
	\defeq
	\max\{\dAA,\dBB\}.
\]
容易看出,我们在上述每一种情形下都得到\(X_1 \times X_2\)上的度量.

\begin{definition}
设\((X_1,d_1)\)和\((X_2,d_2)\)是两个度量空间,
\(d\)是在\(X_1 \times X_2\)中引入的度量,
则称“度量空间\((X_1 \times X_2,d)\)是\((X_1,d_1)\)和\((X_2,d_2)\)的\DefineConcept{直积}”.
\end{definition}

\begin{example}
我们可以认为平面\(\mathbb{R}^2\)是两条直线\(\mathbb{R}\)的直积,
而欧式空间\(\mathbb{R}^3\)是\(\mathbb{R}^2\)与\(\mathbb{R}\)的直积.
\end{example}
\endgroup

\subsection{度量空间之间的连续映射}
现在我们把分析学中的连续函数的概念推广为度量空间之间的连续映射.

\begin{definition}\label{definition:度量空间.连续映射的概念}
设\(X\)和\(Y\)是两个度量空间.
映射\(f\colon X \to Y\).
取\(x_0 \in X\).
如果对于\(f(x_0)\)的任何一个球形邻域\(B(f(x_0),\epsilon)\),
存在\(x_0\)的某一个球形邻域\(B(x_0,\delta)\),
使得\(f(B(x_0,\delta)) \subseteq B(f(x_0),\epsilon)\),
则称“映射\(f\)在点\(x_0\)处是连续的”.

如果映射\(f\)在\(X\)的每一个点\(x \in X\)处连续,则称“\(f\)是一个\DefineConcept{连续映射}”.
\end{definition}
以上的这个定义是分析学中函数连续性定义的纯粹形式推广.
因为如果设\(\rho\)和\(\sigma\)分别是度量空间\(X\)与\(Y\)中的度量,
则“映射\(f\)在点\(x_0\)处连续”可以说成\[
\forall\epsilon>0,
\exists\delta>0
\bigl[
\rho(x,x_0)<\delta
\implies
\sigma(f(x),f(x_0))<\epsilon
\bigr].
\]

\begin{theorem}\label{theorem:度量空间.度量空间下的连续映射与邻域的联系}
设\(X\)、\(Y\)是两个度量空间.
映射\(f\colon X \to Y\).
取\(x_0 \in X\).
那么\[
\text{\(f\)在点\(x_0\)处是连续的}
\iff
\text{\(f(x_0)\)的每一个邻域的原像是\(x_0\)的一个邻域},
\]\[
\text{\(f\)是连续的}
\iff
\text{\(Y\)中的每一个开集的原像是\(X\)中的一个开集}.
\]
\end{theorem}

%现在我们就明白了,对于积分的每一种定义都基于一种度量:
%黎曼积分基于若尔当度量,
%勒贝格积分基于勒贝格度量.

\begingroup
\def\T{\mathfrak T}%
\def\oT{\overline{\mathfrak T}}%

\section{拓扑空间}
从\cref{theorem:度量空间.度量空间下的连续映射与邻域的联系} 可以看出:
度量空间之间的一个映射是否是连续的,或者在某一点处是否是连续的,
本质上只与度量空间中的开集有关(这是因为邻域是通过开集定义的).
这就导致我们可以抛弃度量这个概念,
参照\hyperref[theorem:度量空间.开集的性质]{度量空间中开集的基本性质},
抽象出拓扑空间和拓扑空间之间的连续映射的概念.
于是,\cref{theorem:度量空间.度量空间下的连续映射与邻域的联系}
成为了我们把度量空间和度量空间之间的连续映射的概念推广为拓扑空间和拓扑空间之间的连续映射的出发点.

\subsection{拓扑与拓扑空间的概念}
\begin{definition}\label{definition:拓扑学.开集公理定义的拓扑空间}
已知非空集合\(X\).
\(\T\)\footnote{\(\T\)是德文尖角体(Fraktur)的拉丁字母T.}
是\(X\)的一个子集族,
即\(\T \subseteq \Powerset X\).
若有\begin{enumerate}
	\item \(\emptyset,X \in \T\);
	\item \(\T\)的有限交仍然属于\(\T\),
	即\(A,B \in \T \implies A \cap B \in \T\);
	\item \(\T\)的任意并仍然属于\(\T\),
	即\(\T_1 \subseteq \T \implies \bigcup_{A \in \T_1} A \in \T\);
\end{enumerate}
则称“\(\T\)为\(X\)的一个\DefineConcept{拓扑}”,
称“集合\(X\)是一个(相对于拓扑\(\T\)而言的)\DefineConcept{拓扑空间}(topological space)”,
记作\((X,\T)\).
\(\T\)的任一元素都称为“拓扑空间\((X,\T)\)中的一个\DefineConcept{开集}(open set)”.
\end{definition}
如果我们将\cref{definition:拓扑学.开集公理定义的拓扑空间} 中的三个条件
与\cref{theorem:度量空间.开集的性质} 的三个结论对照一下,
并将“\(U\)属于\(\T\)”读作“\(U\)是一个开集”,
便会发现两者实际上是一样的.

只要经过简单的归纳立即可见,
\cref{definition:拓扑学.开集公理定义的拓扑空间} 中的第二个条件蕴含着以下结论:\[
	\AutoTuple{A}{n}\in\T\ (n\geq1)
	\implies
	A_1 \cap A_2 \cap \dotsb \cap A_n \in \T.
\]

此外,如果在\cref{definition:拓扑学.开集公理定义的拓扑空间} 中的第三个条件中令\(\T_1=\emptyset\),
就会得到\(\emptyset = \bigcup_{A\in\T_1} A \in \T\),
而这一点在第一个条件中已经做了规定.
因此我们在验证任意集合\(X\)的一个子集族是否可以是\(X\)的一个拓扑,
在验证第三个条件是否满足时,总可以假定\(\T_1\neq\emptyset\).

现在首先将度量空间纳入拓扑空间的范畴.

设\((X,\rho)\)是一个度量空间.
令\(\T_\rho\)是由\(X\)中的所有开集构成的集族.
根据\cref{theorem:度量空间.开集的性质} 可知\((X,\T_\rho)\)是\(X\)的一个拓扑.
我们称\(\T_\rho\)为“\(X\)的由度量\(\rho\)诱导出来的拓扑”
\footnote{%
以后如果没有特别说明,当提到度量空间\((X,\rho)\)的拓扑时,指的就是拓扑\(\T_\rho\);
在称度量空间\((X,\rho)\)为拓扑空间时,指的就是拓扑空间\((X,\T_\rho)\).
}.
此外,我们约定:
如果没有另外说明,当我们提到度量空间\((X,\rho)\)的拓扑时,
指的就是拓扑\(\T_\rho\);
在提到度量空间\((X,\rho)\)是拓扑空间时,
指的就是拓扑空间\((X,\T_\rho)\).

因此,实数空间\(\mathbb{R}\)、
\(n\)维欧式空间\(\mathbb{R}^n\)(特别是欧式平面\(\mathbb{R}^2\))
和希尔伯特空间\(\mathbb{H}\)都可以叫做拓扑空间,
它们各自的拓扑分别是由各自的通常度量所诱导出来的拓扑.

\subsection{常见的拓扑空间}
度量空间是拓扑空间中最为重要的一类.
于此,我们再举出一些拓扑空间的例子.

\begin{example}[平庸空间]
设\(X\)是一个集合.
令\(\T=\{X,\emptyset\}\).
容易验证,\(\T\)是\(X\)的一个拓扑,称其为\(X\)的\DefineConcept{平庸拓扑};
称拓扑空间\((X,\T)\)为一个\DefineConcept{平庸空间}.
在平庸空间\((X,\T)\)中,有且仅有两个开集,即\(X\)本身和空集\(\emptyset\).
\end{example}

\begin{example}[离散空间]
设\(X\)是一个集合.
令\(\T=\Powerset X\).
容易验证\(\T\)是\(X\)的一个拓扑,称其为\(X\)的\DefineConcept{离散拓扑};
称拓扑空间\((X,\T)\)为一个\DefineConcept{离散空间}.
在离散空间\((X,\T)\)中,\(X\)的每一个子集都是开集.
\end{example}

\begin{example}\label{example:拓扑学.常见的拓扑空间3}
设\(X = \{a,b,c\}\).
令\[
\T = \{
	\emptyset,
	\{a\},
	\{a,b\},
	\{a,b,c\}
\}.
\]
容易验证\(\T\)是\(X\)的一个拓扑,因此\((X,\T)\)是一个拓扑空间,但这个拓扑空间既不是平庸空间也不是离散空间.
\end{example}

\begin{example}[有限补空间]
设\(X\)是一个集合.
令\[
\T = \Set{ U \subseteq X \given \text{\((X-U)\)是\(X\)的一个有限子集} } \cup \{\emptyset\}.
\]

因为\(X \subseteq X\),\(X - X = \emptyset\)是\(X\)的一个有限子集,
所以\(X \in \T\).
再根据这里对\(\T\)的定义,还有\(\emptyset \in \T\).

设\(A,B\in\T\).
如果\(A\)和\(B\)之中有一个是空集,
则\(A \cap B = \emptyset \in \T\);
如果\(A\)和\(B\)都不是空集,
\(X - (A \cap B) = (X - A) \cup (X - B)\)是\(X\)的一个有限子集,
所以\(A \cap B \in \T\).

设\(\T_1 \subseteq \T\),令\(\T_2 = \T_1 - \{ \emptyset \}\).
显然有\[
\bigcup\limits_{A \in \T_1} A
= \bigcup\limits_{A \in \T_2} A.
\]
如果\(\T_2 = \emptyset\),则\[
\bigcup\limits_{A \in \T_1} A
= \bigcup\limits_{A \in \T_2} A
= \emptyset \in \T;
\]
如果\(\T_2 \neq \emptyset\),
则对\(\forall A_0 \in \T_2\),\[
X - \bigcup\limits_{A \in \T_1} A
= X - \bigcup\limits_{A \in \T_2} A
= \bigcap\limits_{A \in \T_2} (X - A)
\subseteq X - A_0
\]是\(X\)的一个有限子集;
所以\(\bigcup\limits_{A \in \T_1} A \in \T\).

综上所述,\(\T\)是\(X\)的一个拓扑,称其为\(X\)的\DefineConcept{有限补拓扑};
称拓扑空间\((X,\T)\)为一个\DefineConcept{有限补空间}.
\end{example}

\begin{example}[可数补空间]
设\(X\)是一个集合.
令\[
\T = \Set{ U \subseteq X \given \text{\((X-U)\)是\(X\)的一个可数子集} } \cup \{\emptyset\}.
\]
可以验证\(\T\)是\(X\)的一个拓扑,称其为\(X\)的\DefineConcept{可数补拓扑};
称拓扑空间\((X,\T)\)为一个\DefineConcept{可数补空间}.
\end{example}

\subsection{可度量化空间}
一个令人关心的问题是,
拓扑空间是否真的要比度量空间的范围更广一些?
是否每一个拓扑空间的拓扑都可以由某一个度量诱导出来?

\begin{definition}
设\((X,\T)\)是一个拓扑空间.
如果存在\(X\)的一个度量\(\rho\)使得拓扑\(\T\)就是由度量\(\rho\)诱导出来的拓扑\(\T_\rho\),则称“拓扑空间\((X,\T)\)是一个\DefineConcept{可度量化空间}”,或称“拓扑空间\((X,\T)\)是\DefineConcept{可度量化的}”.
\end{definition}

根据这个定义,前述问题即是:
是否每一个拓扑空间都是可度量化空间?
我们知道,每一个只含有限个点的度量空间作为拓扑空间都是离散空间.
然而一个平庸空间如果含有多余一个点的话,它肯定不是离散空间,
因此含有多余一个点的有限的平庸空间不是可度量化的.
\cref{example:拓扑学.常见的拓扑空间3} 给出的那个空间只含有三个点,但它既非离散空间也非可度量化空间.
由此可见,拓扑空间比度量空间的范围要更加广泛.
进一步的问题是,满足什么条件的拓扑空间是可度量化的?
这是点集拓扑学中的重要问题之一,以后我们将专门讨论.

\subsection{拓扑空间之间的连续映射}
下面我们参考\cref{definition:度量空间.连续映射的概念}
和\cref{theorem:度量空间.度量空间下的连续映射与邻域的联系},
将度量空间之间的连续映射的概念推广为拓扑空间之间的连续映射.

\begin{definition}\label{definition:拓扑学.拓扑空间之间的连续映射}
设\(X\)、\(Y\)是两个拓扑空间.
映射\(f\colon X \to Y\).
如果\(Y\)中每一个开集\(U\)的原像\(f^{-1}(U)\)是\(X\)中的一个开集,
则称“映射\(f\)是从\(X\)到\(Y\)的一个\DefineConcept{连续映射}”,
简称“映射\(f\) \DefineConcept{连续}”.
\end{definition}
结合\cref{definition:拓扑学.拓扑空间之间的连续映射} 和\cref{theorem:度量空间.度量空间下的连续映射与邻域的联系} 可知:
当\(X\)、\(Y\)是两个度量空间时,如果映射\(f\colon X \to Y\)是从度量空间\(X\)到度量空间\(Y\)的一个连续映射,那么它也是从拓扑空间\(X\)到拓扑空间\(Y\)的一个连续映射;反之亦然.
注意到这里提到的拓扑都是指诱导拓扑.

下面我们给出连续映射的最重要的性质.

\begin{theorem}\label{theorem:拓扑学.拓扑空间之间的连续映射的性质}
设\(X\)、\(Y\)、\(Z\)都是拓扑空间,那么
\begin{enumerate}
\item 恒同映射\(i_X\colon X \to X\)是一个连续映射;
\item 如果映射\(f\colon X \to Y\)和\(g\colon Y \to Z\)都是连续映射,则复合映射\(g \circ f\colon X \to Z\)也是连续映射.
\end{enumerate}
\begin{proof}
如果\(U\)是\(X\)的一个开集,则\(i_X^{-1}(U) = U\)当然也是\(X\)的开集,所以\(i_X\)连续.

设\(W\)是\(Z\)的一个开集.
由于\(g\)连续,\(g^{-1}(W)\)是\(Y\)的开集;
又由于\(f\)连续,故\(f^{-1}(g^{-1}(W))\)是\(X\)的开集;
因此,\[
(g \circ f)^{-1}(W) = f^{-1}(g^{-1}(W))
\]是\(X\)的开集,\(g \circ f\)连续.
\end{proof}
\end{theorem}

\subsection{同胚映射}
在数学的许多分支学科中都要涉及两种基本对象.
例如在线性代数中我们考虑线性空间和线性变换,
在群论中我们考虑群和同态,
在集合论中我们考虑集合和映射,
在不同的几何学中考虑各自的图形和各自的变换等.
并且对于后者都要提出一类来予以重点研究,
例如线性代数中的(线性)同构、%
群论中的同构、%
集合论中的一一映射%
以及欧式几何中的刚体运动(即平移、旋转)等.

既然我们已经提出了两种基本对象(即拓扑空间和连续映射),那么我们也要从连续映射中挑出重要的一类来进行特别研究.

\begin{definition}\label{definition:拓扑学.同胚映射的概念}
设\(X\)、\(Y\)是两个拓扑空间.
如果映射\(f\colon X \to Y\)是一个一一映射,
并且\(f\)和逆映射\(f^{-1}\colon Y \to X\)都是连续的,
那么称\(f\)为一个\DefineConcept{同胚映射}(homeomorphism),
简称\DefineConcept{同胚};
又称“\(X\)与\(Y\)是\DefineConcept{同胚的}(homeomorphic)”,
或“\(X\)与\(Y\)同胚”,或“\(X\)同胚于\(Y\)”,记作\(X \cong Y\).
%@see: https://mathworld.wolfram.com/Homeomorphism.html
\end{definition}

粗略地说,同胚的两个空间实际上就是两个具有相同拓扑结构的空间.

\begin{theorem}\label{theorem:拓扑学.同胚映射的性质}
设\(X\)、\(Y\)、\(Z\)都是拓扑空间,那么
\begin{enumerate}
\item 恒同映射\(i_X\colon X \to X\)是一个同胚;
\item 如果映射\(f\colon X \to Y\)是一个同胚,则逆映射\(f^{-1}\colon Y \to X\)也是同胚;
\item 如果映射\(f\colon X \to Y\)、\(g\colon Y \to Z\)都是同胚,则复合映射\(g \circ f\colon X \to Z\)也是同胚.
\end{enumerate}
\end{theorem}

\begin{theorem}\label{theorem:拓扑学.同胚关系是等价关系}
设\(X\)、\(Y\)、\(Z\)都是拓扑空间,那么
\begin{enumerate}
\item \(X\)与\(X\)同胚;
\item 如果\(X\)与\(Y\)同胚,则\(Y\)与\(X\)同胚;
\item 如果\(X\)与\(Y\)同胚、\(Y\)与\(Z\)同胚,则\(X\)与\(Z\)同胚.
\end{enumerate}
\end{theorem}
根据\cref{theorem:拓扑学.同胚关系是等价关系},我们可以说:
在任意给定的一个由拓扑空间组成的族中,两个拓扑空间是否同胚这一关系是一个等价关系
\footnote{%
之所以不说“在由全体拓扑空间组成的族中,
两个拓扑空间是否同胚这一关系是一个等价关系”,
是因为“由全体拓扑空间组成的族”这样一个概念会引起逻辑矛盾:
若记这个族为\(T\),令\(\widetilde{T} = \Set{ X \given (X,\T) \in T }\),
赋予\(\widetilde{T}\)以平庸拓扑\(\T_0\),于是\((\widetilde{T},\T_0) \in T\),
从而\(\widetilde{T} \in \widetilde{T}\).
这就产生了“一个集合是它自己的元素”的悖论.
}.
因此同胚关系将这个拓扑空间族分为互不相交的等价类,使得属于同一类的拓扑空间彼此同胚,属于不同类的拓扑空间彼此不同配.

拓扑空间的某种性质\(P\),如果为某一个拓扑空间所具有,则必为与其同胚的任何一个拓扑空间所具有,
那么称此性质\(P\)是一个\DefineConcept{拓扑不变性质}.
换言之,拓扑不变性质就是彼此同胚的拓扑空间所共有的性质.

{\color{red} 拓扑学的中心任务就是研究拓扑不变性质.}

\section{邻域与邻域系}
\begin{definition}\label{definition:拓扑学.邻域的概念}
%@see: 《点集拓扑讲义》(熊金城) P63. 定理2.3.1
设\((X,\T)\)是一个拓扑空间,任意取定一点\(x \in X\).
如果\(U \subseteq X\)满足:
\(\exists V \in \T\),
使得\(x \in V \subseteq U\),
则称\(U\)是“点\(x\)的一个\DefineConcept{邻域}”.

点\(x\)的所有邻域构成的\(X\)的子集族称为“点\(x\)的\DefineConcept{邻域系}”.
\end{definition}

易见,如果\(U\)是包含着点\(x\)的一个开集,
那么它一定是\(x\)的一个邻域,
于是我们称\(U\)为“点\(x\)的一个\DefineConcept{开邻域}”.

应当注意,当我们把一个度量空间看作拓扑空间时,
空间的拓扑是由度量诱导出来的拓扑,
而一个集合是不是一个某一个点的邻域,
无论是按\cref{definition:度量空间.邻域的概念},
还是按\cref{definition:拓扑学.邻域的概念},
都是一回事.

\begin{theorem}\label{theorem:拓扑学.成为开集的充要条件1}
%@see: 《点集拓扑讲义》(熊金城) P64. 定理2.3.1
设\((X,\T)\)是一个拓扑空间,\(U \subseteq X\).
\(U\)是开集的充要条件是:
\(U\)是它的每一点的邻域,即对于\(\forall x \in U\),\(U\)都是\(x\)的一个邻域.
\begin{proof}
充分性.
\begin{enumerate}
	\item 如果\(U\)是空集,当然\(U\)是一个开集.

	\item 如果\(U\neq\emptyset\),
	由于对于\(\forall x \in U\),\(\exists V_x \in \T\),
	使得\(x \in V_x \subseteq U\),
	所以\[
	U \equiv \bigcup\limits_{x \in U} \{ x \}
	\subseteq \bigcup\limits_{x \in U} V_x
	\subseteq U.
	\]
	故\(U = \bigcup\limits_{x \in U} V_x \subseteq \T\).
	根据\hyperref[definition:拓扑学.开集公理定义的拓扑空间]{拓扑的定义},\(U\)是一个开集.
	\qedhere
\end{enumerate}
\end{proof}
\end{theorem}

\begin{theorem}\label{theorem:拓扑学.邻域系的基本性质}
%@see: 《点集拓扑讲义》(熊金城) P64. 定理2.3.2
设\(X\)是一个拓扑空间.
设\(A_x\)是任意一点\(x \in X\)的邻域系,则
\begin{enumerate}
	\item \(A_x \neq \emptyset\);
	\item 如果\(U \in A_x\),则\(x \in U\);
	\item 如果\(U,V \in A_x\),则\(U \cap V \in A_x\);
	\item 如果\(U \in A_x\)且\(U \subseteq V\),则\(V \in A_x\);
	\item 如果\(U \in A_x\),则\(\exists V \in A_x\)满足:\[
		V \subseteq U
		\quad\land\quad
		\forall y \in V \bigl( V \in A_y \bigr).
	\]
\end{enumerate}
\end{theorem}

\begin{theorem}\label{theorem:拓扑学.从邻域系出发定义拓扑}
%@see: 《点集拓扑讲义》(熊金城) P65. 定理2.3.3
设\(X\)是一个集合.
又设对于\(\forall x \in X\),指定\(X\)的一个子集族\(A_x\),
并且它们满足\cref{theorem:拓扑学.邻域系的基本性质} 中的全部条件,
则\(X\)有唯一的一个拓扑\(\T\)使得对于\(\forall x \in X\),
子集族\(A_x\)恰是点\(x\)在拓扑空间\((X,\T)\)中的邻域系.
\end{theorem}

\cref{theorem:拓扑学.从邻域系出发定义拓扑}
表明,我们完全可以从邻域系的概念出发来建立拓扑空间理论.
这种做法在点集拓扑发展的早期常被采用,并且在一定程度上显得更加自然一些,
但不如现在流行的、从开集概念出发定义拓扑的做法来得简洁.

现在我们来讲度量空间之间的连续映射在一点处的连续性的概念推广到拓扑空间之间的映射中去.

\begin{definition}
%@see: 《点集拓扑讲义》(熊金城) P66. 定义2.3.2
设\(X\)和\(Y\)是两个拓扑空间.
映射\(f\colon X \to Y\).
取定一点\(x \in X\).
如果\(f(x) \in Y\)的每一个邻域\(U\)的原像\(f^{-1}(U)\)是\(x \in X\)的一个邻域,
则称映射\(f\)是“一个在点\(x\)处连续的映射”,
或称“映射\(f\)在点\(x\)处连续”.
\end{definition}

\begin{theorem}
%@see: 《点集拓扑讲义》(熊金城) P66. 定理2.3.4
设\(X,Y,Z\)都是拓扑空间,则
\begin{enumerate}
	\item 恒同映射\(i_X\colon X \to X\)在\(\forall x \in X\)处连续;
	\item 如果\(f\colon X \to Y\)在点\(x \in X\)处连续,
	\(g\colon Y \to Z\)在点\(f(x)\)处连续,
	则\(g \circ f\colon X \to Z\)在点\(x\)处连续.
\end{enumerate}
\end{theorem}

\begin{theorem}\label{theorem:拓扑学.连续性在局部与整体的连续}
%@see: 《点集拓扑讲义》(熊金城) P66. 定理2.3.5
设\(X\)和\(Y\)是拓扑空间,映射\(f\colon X \to Y\),
则映射\(f\)连续的充要条件是:
对于\(\forall x \in X\),映射\(f\)在点\(x\)处连续.
\end{theorem}

\cref{theorem:拓扑学.连续性在局部与整体的连续}
建立了“局部的”连续性概念和“整体的”连续性概念之间的联系.

\section{导集、闭集和闭包}
\begin{definition}\label{definition:拓扑学.聚点与孤立点的概念}
%@see: 《点集拓扑讲义》(熊金城) P67. 定义2.4.1
设\(X\)是拓扑空间,\(A \subseteq X\).
如果点\(x \in X\)的每一个邻域\(U\)中总有异于\(x\)而属于\(A\)的点,
即\(U \cap (A - \{x\}) \neq \emptyset\),
则称“点\(x\)是集合\(A\)的\DefineConcept{聚点}”.
集合\(A\)的所有聚点构成的集合称为\(A\)的\DefineConcept{导集},记作\(D(A)\).

如果\(x \in A\)不是\(A\)的聚点,
即存在\(x\)的一个邻域\(U\)使得\(U \cap (A - \{x\}) = \emptyset\),
则称\(x\)为“\(A\)的一个\DefineConcept{孤立点}”.
\end{definition}

在\cref{definition:拓扑学.聚点与孤立点的概念} 中,
聚点、导集以及孤立点的定义无一例外地依赖于它所在的拓扑空间的那个给定的拓扑.
因此,当我们在讨论问题时,
如果涉及了多个拓扑而又提及聚点或孤立点时,
我们必须明确说明所称的聚点或孤立点是相对于哪个拓扑而言,不容许产生任何混响.
由于我们将要定义的许多概念绝大多数都是依赖于给定拓扑的,
因此类似于这里谈到的问题,今后几乎时时刻刻都会发生,
即便以后不作特别说明,也请留意这一问题.

应该注意到,尽管在欧式空间中我们已经定义过聚点、孤立点的概念,
但绝不要以为某些在欧式空间中有效的聚点或孤立点的性质对一般的拓扑空间都有效.

\begin{example}[离散空间中的聚点]\label{example:拓扑学.离散空间中的聚点}
%@see: 《点集拓扑讲义》(熊金城) P67. 例2.4.1
设\(X\)是一个离散空间,\(A\)是\(X\)的一个任意子集.
由于\(X\)中的每一个单点集都是开集,因此如果\(x \in X\),
则\(x\)有一个邻域\(\{x\}\)使得\(\{x\}\cap(A-\{x\})=\emptyset\),
于是\(x\)不是\(A\)的聚点,\(A\)没有聚点,从而\(A\)的导集是空集,即\(D(A)=\emptyset\).
\end{example}

\begin{example}[平庸空间中的聚点]\label{example:拓扑学.平庸空间中的聚点}
%@see: 《点集拓扑讲义》(熊金城) P68. 例2.4.2
设\(X\)是一个平庸空间,\(A\)是\(X\)中的一个任意子集.
我们可以分三种情况讨论.
\begin{enumerate}
	\item 设\(\abs{A} = 0\).
	那么\(A = \emptyset\).
	这时\(A\)显然没有聚点,亦即\(D(A) = \emptyset\).

	\item 设\(\abs{A} = 1\).
	不妨设\(A = \{x_0\}\).
	如果\(x \in X\),\(x \neq x_0\),点\(x\)只有唯一的一个邻域\(X\).
	这时\(x_0 \in X \cap (A - \{x\})\),
	所以\(X \cap (A - \{x\}) \neq \emptyset\).
	因此\(x\)是\(A\)的一个凝聚点,即\(x \in D(A)\).
	然而对于\(x_0\)的唯一邻域\(X\),
	有\(X \cap (A - \{x_0\}) = \emptyset\),
	所以\(x_0 \notin D(A)\).
	于是\(D(A) = X - A\).

	\item 设\(\abs{A} > 1\).
	这时\(X\)中的每一个点都是\(A\)的聚点,即\(D(A) = X\).
\end{enumerate}
\end{example}

\begin{theorem}
%@see: 《点集拓扑讲义》(熊金城) P68. 定理2.4.1
设\(X\)是一个拓扑空间,\(A \subseteq X\),则
\begin{enumerate}
	\item \(D(\emptyset) = \emptyset\).
	\item \(A \subseteq B \implies D(A) \subseteq D(B)\).
	\item \(D(A \cup B) = D(A) \cup D(B)\).
	\item \(D(D(A)) \subseteq A \cup D(A)\).
\end{enumerate}
\end{theorem}

\begin{definition}\label{definition:拓扑学.闭集的概念}
%@see: 《点集拓扑讲义》(熊金城) P69. 定义2.4.2
设\(X\)是一个拓扑空间,\(A \subseteq X\).
如果\(A\)的每一个聚点都属于\(A\),即\(D(A) \subseteq A\),
则称\(A\)是“拓扑空间\(X\)中的一个\DefineConcept{闭集}”.
\end{definition}

例如,根据\cref{example:拓扑学.离散空间中的聚点,example:拓扑学.平庸空间中的聚点} 中的讨论可见,
离散空间中的任何一个子集都是闭集,而平庸空间中的任何一个非空真子集都不是闭集.

\begin{theorem}\label{theorem:拓扑学.成为闭集的充要条件1}
%@see: 《点集拓扑讲义》(熊金城) P69. 定理2.4.2
设\(X\)是一个拓扑空间,\(A \subseteq X\),
则\(A\)是一个闭集当且仅当\(A\)的补集\(X - A\)是一个开集.
\end{theorem}

\begin{example}[实数空间\(\mathbb{R}\)中的闭集]
%@see: 《点集拓扑讲义》(熊金城) P70. 例2.4.3
设\(a,b\in\mathbb{R}\),\(a<b\).
闭区间\([a,b]\)是实数空间\(\mathbb{R}\)中的一个闭集,
因为\([a,b]\)的补集\(\mathbb{R}-[a,b]
=(-\infty,a)\cup(b,+\infty)\)是一个开集.
同理,\((-\infty,a]\)、\([b,+\infty)\)和\((-\infty,+\infty)\)也都是闭集.
但是,开区间\((a,b)\)却不是闭集,这是因为\(a\)是\(a,b\)的一个聚点,但\(a\notin(a,b)\).
同理,\((a,b]\)、\([a,b)\)、\((-\infty,a)\)和\((b,+\infty)\)都不是闭集.
\end{example}

\begin{theorem}\label{theorem:拓扑学.闭集族的性质}
%@see: 《点集拓扑讲义》(熊金城) P70. 定理2.4.3
设\(X\)是一个拓扑空间,\(F\)为所有闭集构成的族,则
\begin{enumerate}
	\item \(\emptyset,X \in F\);
	\item \(A,B \in F \implies A \cup B \in F\);
	\item \(\emptyset \neq F_1 \subseteq F\)
	\footnote{%
		这里特别要求\(F_1 \neq \emptyset\)的原因在于
		当\(F_1 = \emptyset\)时所涉及的交运算没有定义.
	},则\(\bigcap\limits_{A \in F_1} A \in F\).
\end{enumerate}
\end{theorem}

\begin{definition}\label{definition:拓扑学.闭包的概念}
%@see: 《点集拓扑讲义》(熊金城) P71. 定义2.4.3
设\(X\)是一个拓扑空间,\(A \subseteq X\).
集合\(A\)及其导集\(D(A)\)的并\(A \cup D(A)\)称为“集合\(A\)的\DefineConcept{闭包}”,记作\(\overline{A}\).
\end{definition}

\begin{property}\label{theorem:拓扑学.一点属于闭包的充要条件}
设\(X\)是一个拓扑空间,\(A \subseteq X\).
\(x \in \overline{A}\)的充要条件是:
对\(x\)的任一邻域\(U\)有\(U \cap A \neq \emptyset\).
\end{property}

\begin{theorem}\label{theorem:拓扑学.成为闭集的充要条件2}
%@see: 《点集拓扑讲义》(熊金城) P71. 定理2.4.4
拓扑空间\(X\)的子集\(A\)是闭集的充要条件是:\(A = \overline{A}\).
\end{theorem}

\begin{theorem}\label{theorem:拓扑学.闭包的性质}
%@see: 《点集拓扑讲义》(熊金城) P71. 定理2.4.5
设\(X\)是一个拓扑空间,\(A,B \subseteq X\),有
\begin{enumerate}
	\item \(\overline{\emptyset} = \emptyset\).
	\item \(A \subseteq \overline{A}\).
	\item \(\overline{A \cup B} = \overline{A} \cup \overline{B}\).
	\item \(\overline{\overline{A}} = \overline{A}\).
\end{enumerate}
\end{theorem}

\begin{corollary}\label{theorem:拓扑学.拓扑空间子集闭包都是闭集}
%@see: 《点集拓扑讲义》(熊金城) P72. 定理2.4.6
拓扑空间\(X\)的任一子集\(A\)的闭包\(\overline{A}\)都是闭集.
\end{corollary}

\begin{theorem}\label{theorem:拓扑学.集合的闭包是含有该集的最小闭集}
%@see: 《点集拓扑讲义》(熊金城) P72. 定理2.4.7
设\(X\)是一个拓扑空间,\(F\)为所有闭集构成的族,
则对于\(X\)的每一个子集\(A\),有\[
\overline{A}
= \bigcap\limits_{B \in F, B \supseteq A} B
\equiv \bigcap\limits_{B \in \Set{ B_1 \in F \given B_1 \supseteq A}} B,
\]
也就是说,集合\(A\)的闭包等于包含\(A\)的所有闭集之交.
\end{theorem}
由此可见,\(\overline{A}\)是一个包含着\(A\)的闭集,
它又包含于任何一个包含\(A\)的闭集之中;
在这种意义下我们说:
一个集合的闭包乃是包含着这个集合的最小的闭集.

以上由一个集合求取它的闭包的步骤可以理解为空间\(X\)的幂集\(\Powerset X\)到自身的一个映射,
集合\(A \subseteq X\)在这个映射下的像便是\(A\)的闭包\(\overline{A}\).

\begin{definition}\label{definition:拓扑学.闭包运算的概念}
%@see: 《点集拓扑讲义》(熊金城) P73. 定义2.4.4
设\(X\)是一个集合.
如果映射\(c^*\colon \Powerset X \to \Powerset X\)满足条件:
对于\(\forall A,B \in \Powerset X\),\begin{enumerate}
	\item \(c^*(\emptyset) = \emptyset\);
	\item \(A \subseteq c^*(A)\);
	\item \(c^*(A \cup B) = c^*(A) \cup c^*(B)\);
	\item \(c^*(c^*(A)) = c^*(A)\),
\end{enumerate}
则称其为\(X\)的一个\DefineConcept{闭包运算}.
\end{definition}
\cref{definition:拓扑学.闭包运算的概念} 中给出的四个条件通常被称为“库拉托夫斯基闭包公理”.

根据\cref{theorem:拓扑学.闭包的性质},
将拓扑空间\(X\)的子集\(A\)映射为它的闭包\(\overline{A}\)的那个
从\(X\)的幂集\(\Powerset X\)到自身的映射便是一个闭包运算,
即这个映射满足库拉托夫斯基闭包公理.
不仅如此,下面的\cref{theorem:拓扑学.闭包公理与拓扑是等价的}
说明库拉托夫斯基闭包公理和我们定义拓扑的三个条件等价.
在一些点集拓扑发展的早期出现的文献就是从闭包运算出发来建立拓扑空间这一概念的.

\begin{theorem}\label{theorem:拓扑学.闭包公理与拓扑是等价的}
%@see: 《点集拓扑讲义》(熊金城) P73. 定理2.4.8
设\(X\)是一个集合,映射\(c^*\colon \Powerset X \to \Powerset X\)是集合\(X\)的一个闭包运算,
那么存在\(X\)的唯一一个拓扑\(\T\),使得在拓扑空间\((X,\T)\)中,
对于\(\forall A \subseteq X\),总有\(c^*(A) = \overline{A}\).
\end{theorem}

在度量空间中,集合的聚点、导集和闭包等概念都可以通过度量来刻画.

\begin{definition}\label{definition:拓扑学.点到点集的距离}
%@see: 《点集拓扑讲义》(熊金城) P75. 定义2.4.5
\def\d{\inf\limits_{y \in A} \rho(x,y)}%
设\((X,\rho)\)是一个度量空间,\(A\)是\(X\)的非空子集,取定一点\(x \in X\).
称实数\(\d\)
为“点\(x\)到\(A\)的\DefineConcept{距离}”,记作\(\rho(x,A)\),即\[
\rho(x,A) \defeq \d.
\]
\end{definition}

\begin{theorem}
设\((X,\rho)\)是一个度量空间,\(A\)是\(X\)的非空子集,取定一点\(x \in X\).
\(\rho(x,A) = 0\)的充要条件是:
对于\(\forall\epsilon\in\mathbb{R}^+\),\(\exists y \in A\),
使得\(\rho(x,y)<\epsilon\).
\end{theorem}

\begin{corollary}
设\((X,\rho)\)是一个度量空间,\(A\)是\(X\)的非空子集,取定一点\(x \in X\).
\(\rho(x,A) = 0\)的充要条件是:
对于\(x\)的任一邻域\(U\),总有\(U \cap A \neq \emptyset\).
\end{corollary}

\begin{theorem}
%@see: 《点集拓扑讲义》(熊金城) P75. 定理2.4.9
设\(A\)是度量空间\((X,\rho)\)中的一个非空子集,则
\begin{enumerate}
	\item \(x \in d(A)\)当且仅当\(\rho(x,A-\{x\})=0\);
	\item \(x \in \overline{A}\)当且仅当\(\rho(x,A)=0\).
\end{enumerate}
\end{theorem}

\begin{theorem}
%@see: 《点集拓扑讲义》(熊金城) P75. 定理2.4.10
设\(X\)和\(Y\)是两个拓扑空间,映射\(f\colon X \to Y\),则以下命题等价:
\begin{enumerate}
	\item \(f\)是一个连续映射;
	\item \(Y\)中的任何一个闭集的原像\(f^{-1}\ImageOfSetUnderRelation{B}\)是一个闭集;
	\item 对于\(X\)中的任何一个子集\(A\),\(A\)的闭包的像包含于\(A\)的像的闭包,
	即\(f\ImageOfSetUnderRelation{\overline{A}}
	\subseteq
	\overline{f\ImageOfSetUnderRelation{A}}\);
	\item 对于\(Y\)中的任何一个子集\(B\),\(B\)的闭包的原像包含\(B\)的原像的闭包,
	即\(f^{-1}\ImageOfSetUnderRelation{\overline{B}}
	\supseteq
	\overline{f^{-1}\ImageOfSetUnderRelation{B}}\).
\end{enumerate}
\end{theorem}

\section{内部、外部与边界}
\begin{definition}\label{definition:拓扑学.内部的概念}
%@see: 《点集拓扑讲义》(熊金城) P77. 定义2.5.1
设\(X\)是一个拓扑空间,\(A \subseteq X\).
如果\(A\)是点\(x \in A\)的一个邻域,则称“点\(x\)是集合\(A\)的一个\DefineConcept{内点}”.
集合\(A\)的所有内点构成的集合称为“\(A\)的\DefineConcept{内部}”,记作\(I(A)\).
\end{definition}

\begin{theorem}\label{theorem:拓扑学.内部与闭包的联系}
%@see: 《点集拓扑讲义》(熊金城) P78 定理2.5.1
设\(X\)是一个拓扑空间,\(A \subseteq X\),则\[
I(A) = X - \overline{X - A}, \qquad
\overline{A} = X - I(X - A).
\]
\end{theorem}

关于内部的基本性质,我们有与闭包的性质
(\cref{theorem:拓扑学.成为闭集的充要条件2,%
theorem:拓扑学.闭包的性质,%
theorem:拓扑学.拓扑空间子集闭包都是闭集,%
theorem:拓扑学.集合的闭包是含有该集的最小闭集})
完全对偶的一组定理,即以下的\cref{theorem:拓扑学.成为开集的充要条件2,%
theorem:拓扑学.内部的性质,%
theorem:拓扑学.拓扑空间子集内部都是开集,%
theorem:拓扑学.集合的内部是含于该集的最大开集}.
这些定理的证明过程都是将闭包的相应性质通过\cref{theorem:拓扑学.内部与闭包的联系}
转化为内部的性质.

\begin{theorem}\label{theorem:拓扑学.成为开集的充要条件2}
%@see: 《点集拓扑讲义》(熊金城) P78 定理2.5.2
拓扑空间\(X\)的子集\(A\)是开集的充要条件是:\(A = I(A)\).
\end{theorem}

\begin{theorem}\label{theorem:拓扑学.内部的性质}
%@see: 《点集拓扑讲义》(熊金城) P78 定理2.5.3
设\(X\)是一个拓扑空间,则对于\(\forall A,B \subseteq X\),有
\begin{enumerate}
	\item \(I(X) = X\);
	\item \(A \supseteq I(A)\);
	\item \(I(A \cap B) = I(A) \cap I(B)\);
	\item \(I(I(A)) = I(A)\).
\end{enumerate}
\end{theorem}

\begin{theorem}\label{theorem:拓扑学.拓扑空间子集内部都是开集}
%@see: 《点集拓扑讲义》(熊金城) P79 定理2.5.4
拓扑空间\(X\)的任一子集\(A\)的内部\(I(A)\)都是开集.
\end{theorem}

\begin{theorem}\label{theorem:拓扑学.集合的内部是含于该集的最大开集}
%@see: 《点集拓扑讲义》(熊金城) P79 定理2.5.5
设\((X,\T)\)是一个拓扑空间,则对于\(\forall A \subseteq X\),有\[
I(A) = \bigcup\limits_{B \in \T, B \subseteq A} B,
\]
即集合\(A\)的内部等于包含于\(A\)的所有开集之并.
\end{theorem}

由\cref{theorem:拓扑学.集合的内部是含于该集的最大开集}
可见,集合\(A\)的内部\(I(A)\)是一个包含于\(A\)的开集,
它又包含着任何一个包含于\(A\)的开集.
在这种意义下我们说
一个集合的内部乃是包含于这个集合的最大的开集.

与我们在前一节中处理闭包运算时的情形一样,
求取一个集合的内部也可以理解为从拓扑空间\(X\)的幂集到其自身的一个映射,
它将每一个\(A \in \Powerset X\)映射为\(I(A)\).
也同样可以像定义闭包运算一样定义内部运算,并由内部运算导出拓扑和拓扑空间的概念.
同样地,映射的连续性也可通过内部这个概念作出等价的描述.

\begin{definition}\label{definition:拓扑学.边界的概念}
设\(X\)是一个拓扑空间,\(A \subseteq X\).
取定一点\(x \in X\),
如果在\(x\)的任一邻域\(U\)中既有\(A\)中的点,又有\(X - A\)中的点,
即既有\(U \cap A \neq \emptyset\),又有\(U \cap (X-A) \neq \emptyset\),
则称“\(x\)是集合\(A\)的一个\DefineConcept{边界点}”.
集合\(A\)的全体边界点构成的集合称为集合\(A\)的边界,记作\(\partial A\).
\end{definition}

\endgroup
