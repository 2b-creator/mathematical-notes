\section{内部、外部与边界}
\begin{definition}\label{definition:拓扑学.内部的概念}
%@see: 《点集拓扑讲义》(熊金城) P77. 定义2.5.1
设\(X\)是一个拓扑空间,\(A \subseteq X\).
如果\(A\)是点\(x \in A\)的一个邻域,则称“点\(x\)是集合\(A\)的一个\DefineConcept{内点}”.
集合\(A\)的所有内点构成的集合称为“\(A\)的\DefineConcept{内部}”,记作\(I(A)\).
\end{definition}

\begin{theorem}\label{theorem:拓扑学.内部与闭包的联系}
%@see: 《点集拓扑讲义》(熊金城) P78 定理2.5.1
设\(X\)是一个拓扑空间,\(A \subseteq X\),则\[
I(A) = X - \overline{X - A}, \qquad
\overline{A} = X - I(X - A).
\]
\end{theorem}

关于内部的基本性质,我们有与闭包的性质
(\cref{theorem:拓扑学.成为闭集的充分必要条件2,%
theorem:拓扑学.闭包的性质,%
theorem:拓扑学.拓扑空间子集闭包都是闭集,%
theorem:拓扑学.集合的闭包是含有该集的最小闭集})
完全对偶的一组定理,即以下的\cref{theorem:拓扑学.成为开集的充分必要条件2,%
theorem:拓扑学.内部的性质,%
theorem:拓扑学.拓扑空间子集内部都是开集,%
theorem:拓扑学.集合的内部是含于该集的最大开集}.
这些定理的证明过程都是将闭包的相应性质通过\cref{theorem:拓扑学.内部与闭包的联系}
转化为内部的性质.

\begin{theorem}\label{theorem:拓扑学.成为开集的充分必要条件2}
%@see: 《点集拓扑讲义》(熊金城) P78 定理2.5.2
拓扑空间\(X\)的子集\(A\)是开集的充分必要条件是:\(A = I(A)\).
\end{theorem}

\begin{theorem}\label{theorem:拓扑学.内部的性质}
%@see: 《点集拓扑讲义》(熊金城) P78 定理2.5.3
设\(X\)是一个拓扑空间,则对于\(\forall A,B \subseteq X\),有
\begin{enumerate}
	\item \(I(X) = X\);
	\item \(A \supseteq I(A)\);
	\item \(I(A \cap B) = I(A) \cap I(B)\);
	\item \(I(I(A)) = I(A)\).
\end{enumerate}
\end{theorem}

\begin{theorem}\label{theorem:拓扑学.拓扑空间子集内部都是开集}
%@see: 《点集拓扑讲义》(熊金城) P79 定理2.5.4
拓扑空间\(X\)的任一子集\(A\)的内部\(I(A)\)都是开集.
\end{theorem}

\begin{theorem}\label{theorem:拓扑学.集合的内部是含于该集的最大开集}
%@see: 《点集拓扑讲义》(熊金城) P79 定理2.5.5
设\((X,\T)\)是一个拓扑空间,则对于\(\forall A \subseteq X\),有\[
I(A) = \bigcup_{B \in \T, B \subseteq A} B,
\]
即集合\(A\)的内部等于包含于\(A\)的所有开集之并.
\end{theorem}

由\cref{theorem:拓扑学.集合的内部是含于该集的最大开集}
可见,集合\(A\)的内部\(I(A)\)是一个包含于\(A\)的开集,
它又包含着任何一个包含于\(A\)的开集.
在这种意义下我们说
一个集合的内部乃是包含于这个集合的最大的开集.

与我们在前一节中处理闭包运算时的情形一样,
求取一个集合的内部也可以理解为从拓扑空间\(X\)的幂集到其自身的一个映射,
它将每一个\(A \in \Powerset X\)映射为\(I(A)\).
也同样可以像定义闭包运算一样定义内部运算,并由内部运算导出拓扑和拓扑空间的概念.
同样地,映射的连续性也可通过内部这个概念作出等价的描述.

\begin{definition}\label{definition:拓扑学.边界的概念}
设\(X\)是一个拓扑空间,\(A \subseteq X\).
取定一点\(x \in X\),
如果在\(x\)的任一邻域\(U\)中既有\(A\)中的点,又有\(X - A\)中的点,
即既有\(U \cap A \neq \emptyset\),又有\(U \cap (X-A) \neq \emptyset\),
则称“\(x\)是集合\(A\)的一个\DefineConcept{边界点}”.
集合\(A\)的全体边界点构成的集合称为集合\(A\)的边界,记作\(\partial A\).
\end{definition}

\endgroup
