\chapter{复数}
\section{复数的各种形式及代数运算}
\subsection{复数的代数形式、代数运算,复数域}
\begin{definition}[虚数单位]
规定:满足\(\iu^2=-1\)的数\(\iu\)称为\textbf{虚数单位}.
\end{definition}

\begin{definition}[复数的代数形式]
把由有序实数对\(\opair{x,y}\)作代数组合所确定的数\(z=x+\iu y\)称为代数形式的\textbf{复数}.实数\(x\)、\(y\)分别称为复数\(z=x+\iu y\)的\textbf{实部}、\textbf{虚部},记作\(x=\Re z\)、\(y=\Im z\).

特别地,当\(\Im z=0\)时,\(z=\Re z=x\)是实数;
当\(\Re z=0\)时,\(z=\iu\Im z=\iu y\),称为\textbf{纯虚数}.
\end{definition}

\begin{definition}
全体复数形成的数集\[
\Set{ x + \iu y \given x,y\in\mathbb{R} }
\]称为\textbf{复数集},记作\(\mathbb{C}\).
\end{definition}

\begin{definition}[代数形式下复数相等条件]
设\(z_1\)和\(z_2\)都是复数,则当\(\Re z_1 = \Re z_2\)和\(\Im z_1 = \Im z_2\)同时成立时,则称\(z_1 = z_2\).
特别地,对于复数\(z\),则当且仅当\(\Re z=0\)且\(\Im z=0\)时,\(z=0\).
\end{definition}

\begin{definition}[共轭复数]
设\(z=x + \iu y \in\mathbb{C}\),其中\(x,y\in\mathbb{C}\),称复数\(\complexconjugate{z}=x - \iu y\)为\(z\)的\textbf{共轭复数}(conjugate).
\end{definition}

\begin{definition}[复数的模]
设复数\(z = x + \iu y\),称\[
\abs{z} = \sqrt{x^2 + y^2}
\]为\(z\)的\textbf{模}或\textbf{绝对值}.
\end{definition}

\begin{definition}[复数加法]
设\(z_1=x_1+\iu y_1\),\(z_2=x_2+\iu y_2\),%
定义\[
z_1+z_2=(x_1+x_2)+\iu(y_1+y_2)
\]为复数\(z_1\)和\(z_2\)的加法运算.
\end{definition}

\begin{definition}[复数减法]
设\(z_1=x_1+\iu y_1\),\(z_2=x_2+\iu y_2\),定义\[
z_1-z_2=(x_1-x_2)+\iu(y_1-y_2)
\]为复数的\(z_1\)和\(z_2\)的减法运算.
显然,复数减法是加法的逆运算.
\end{definition}

\begin{definition}[复数乘法]
设\(z_1 = x_1 + \iu y_1\),\(z_2 = x_2 + \iu y_2\),定义\[
z_1 \cdot z_2
= (x_1 + \iu y_1)(x_2 + \iu y_2)
= (x_1 x_2 - y_1 y_2)+\iu(x_1 y_2 + x_2 y_1)
\]为复数\(z_1\)和\(z_2\)的乘法运算.
\end{definition}

\begin{definition}[复数除法]
设\(z_1 = x_1 + \iu y_1\),\(z_2 = x_2 + \iu y_2 \neq 0\),定义满足\[
z_1 = z \cdot z_2
\]的复数\(z = x + \iu y\)为复数\(z_1\)和\(z_2\)的商,记作\[
z = \frac{z_1}{z_2},
\]称为复数\(z_1\)和\(z_2\)的除法运算.

显然,复数的除法是乘法的逆运算.
\end{definition}

\begin{theorem}
设\(z_1 = x_1 + \iu y_1\),\(z_2 = x_2 + \iu y_2 \neq 0\),则\[
\frac{z_1}{z_2}
= \frac{z_1 \cdot \complexconjugate{z_2}}{z_2 \cdot \complexconjugate{z_2}}
= \frac{z_1 \cdot \complexconjugate{z_2}}{\abs{z_2}^2}
= \frac{x_1 x_2 + y_1 y_2}{x_2^2 + y_2^2}
+ \iu \frac{x_2 y_1 - x_1 y_2}{x_2^2 + y_2^2}.
\]
\begin{proof}
设\(z = x + \iu y = \frac{z_1}{z_2}\),则\[
z \cdot z_2 = (x + \iu y)(x_2 + \iu y_2)
= (x_2 x - y_2 y) + \iu(y_2 x + x_2 y)
= x_1 + \iu y_1,
\]从而有方程组\[
\left\{ \begin{array}{l}
x_2 x - y_2 y = x_1, \\
y_2 x + x_2 y = y_1,
\end{array} \right.
\]解得\[
x = \frac{x_1 x_2 + y_1 y_2}{x_2^2 + y_2^2},
\quad
y = \frac{x_2 y_1 - x_1 y_2}{x_2^2 + y_2^2}.
\]
\end{proof}
\end{theorem}

\begin{property}
复数集和实数域一样,对四则运算是封闭的,故复数集也是域.
但复数域和实数域、有理数域不同的是,复数没有大小之分,不能像实数、有理数那样比较大小,因此,复数域不是有序域,而是无序域.
\end{property}

\begin{property}
共轭复数以及复数的模具有以下性质:
\begin{enumerate}
\item \(\Re \complexconjugate{z} = \Re z\);
\item \(\Im \complexconjugate{z} = -\Im z\);
\item \(\complexconjugate{(\complexconjugate{z})} = z\);
\item \(z+\complexconjugate{z} = 2 \Re z\);
\item \(z-\complexconjugate{z} = 2\iu \Im z\);
\item \(z\complexconjugate{z} = \abs{z}^2\);
\item \(\abs{\complexconjugate{z}}=\abs{z}\);
\item \(\abs{z_1 z_2} = \abs{z_1} \abs{z_2}\);
\item \(\abs{\frac{z_1}{z_2}} = \frac{\abs{z_1}}{\abs{z_2}} \quad(z_2 \neq 0)\);
\item \(\complexconjugate{z_1 \pm z_2} = \complexconjugate{z_1} \pm \complexconjugate{z_2}\);
\item \(\complexconjugate{z_1 z_2} = \complexconjugate{z_1} \cdot \complexconjugate{z_2}\);
\item \(\complexconjugate{\left(\frac{z_1}{z_2}\right)} = \frac{\complexconjugate{z_1}}{\complexconjugate{z_2}} \quad (z_2 \neq 0)\);
\item \(-\abs{z} \leqslant \Re z \leqslant \abs{z} \leqslant \abs{\Re z} + \abs{\Im z}\);
\item \(-\abs{z} \leqslant \Im z \leqslant \abs{z} \leqslant \abs{\Re z} + \abs{\Im z}\).
\end{enumerate}
\end{property}

\begin{theorem}
若\(z,w \in \mathbb{C}\),则有\(\abs{z \pm w}^2 = \abs{z}^2 + \abs{w}^2 \pm 2 \Re(z \complexconjugate{w})\).
\begin{proof}
\(
\abs{z + w}^2
= (z + w) (\complexconjugate{z + w})
= z\complexconjugate{z} + z\complexconjugate{w} + w\complexconjugate{z} + w\complexconjugate{w}
= \abs{z}^2 + 2 \Re(z\complexconjugate{w}) + \abs{w}^2
\).
\end{proof}
\end{theorem}

\begin{theorem}
若\(z,w \in \mathbb{C}\),则有\(\abs{z + w}^2 \leqslant (\abs{z} + \abs{w})^2\).
\begin{proof}
\(
\abs{z + w}^2
= \abs{z}^2 + 2 \Re(z \complexconjugate{w}) + \abs{w}^2
\leqslant \abs{z}^2 + 2 \abs{z}\abs{\complexconjugate{w}} + \abs{w}^2
= (\abs{z} + \abs{w})^2
\).
\end{proof}
\end{theorem}

\begin{theorem}[三角不等式]
若\(z,w \in \mathbb{C}\),则有\(\abs{\abs{z}-\abs{w}} \leqslant \abs{z \pm w} \leqslant \abs{z} + \abs{w}\).
\begin{proof}
因为\[
\abs{z + w}^2 \leqslant (\abs{z} + \abs{w})^2,
\]所以\(\abs{z + w} \leqslant \abs{z} + \abs{w}\).

又因为\[
\abs{z} = \abs{z + w + (-w)} \leqslant \abs{z+w} + \abs{-w} = \abs{z+w} + \abs{w},
\]所以\(\abs{z}-\abs{w} \leqslant \abs{z+w}\).

同样地,有\(\abs{w}-\abs{z} \leqslant \abs{z+w}\).

综上所述,\(\abs{\abs{z}-\abs{w}}\leqslant\abs{z+w}\).
\end{proof}
\end{theorem}

\begin{example}
证明:\(\abs{z_1+z_2}^2 + \abs{z_1-z_2}^2 = 2 (\abs{z_1}^2 + \abs{z_2}^2)\).
\begin{proof}
记\(z_1 = x_1 + \iu y_1, z_2 = x_2 + \iu y_2\).那么\[
z_1+z_2 = (x_1+x_2) + \iu(y_1+y_2),
\]\[
\abs{z_1+z_2}^2 = (x_1+x_2)^2 + (y_1+y_2)^2;
\]同理\(\abs{z_1-z_2}^2 = (x_1-x_2)^2 + (y_1-y_2)^2\).那么\begin{align*}
\abs{z_1+z_2}^2 + \abs{z_1-z_2}^2
&= (x_1+x_2)^2 + (y_1+y_2)^2
+ (x_1-x_2)^2 + (y_1-y_2)^2 \\
&= 2 ( x_1^2 + x_2^2 + y_1^2 + y_2^2 )
= 2 ( \abs{z_1}^2 + \abs{z_2}^2 ).
\qedhere
\end{align*}
\end{proof}
\end{example}

\subsection{复数的几何表示}
\begin{definition}[复数在复平面上的几何表示]
在直角坐标系\(xOy\)上可以用点\(\opair{x,y}\)表示复数\(z=x+\iu y\),也可以用向量\((x,y)\)表示复数\(z=x+\iu y\).与复数建立了这种对应关系的坐标平面\(xOy\)称为\textbf{复平面},记作\(C\).
称\(x\)轴为复平面的\textbf{实轴}.称\(y\)轴为复平面的\textbf{虚轴}.

显然,表示复数\(z\)的点与表示其共轭复数\(\complexconjugate{z}\)的点关于实轴对称.
\end{definition}

\begin{definition}[复数在复球面上的几何表示]
在\(Ox_1x_2x_3\)坐标系下,考虑单位球面\(S\)(即球心位于原点、半径为1的球面):\[
x_1^2+x_2^2+x_3^2=1
\]点\((0,0,1)\)称为北极,记作\(N\),同时\(x_1Ox_2\)平面取为复平面\(C\).复平面\(C\)交球面\(S\)于单位球的赤道.

对于复平面\(C\)上的每一个点\(z\),它与\(N\)连接的直线必与\(S\)交且只交于一点\(Z \neq N\).
若\(\abs{z} < 1\),则点\(Z\)在下半球面上;
若\(\abs{z} > 1\),则点\(Z\)在上半球面上;
若\(\abs{z} = 1\),则点\(Z\)在赤道上.
反之,取球面上任意一点\(Z \neq N\),连接它与\(N\)的直线也只与复平面\(C\)交于一点\(z\).

可见,除北极\(N=(0,0,1)\)以外,复平面\(C\)和球面\(S\)上的点是一一对应的.并且当\(\abs{z} \to +\infty\)时,\(Z \to N\).那么可以假想一个模为无穷大的复数,称作\textbf{无穷远点},记作\(z = \infty\),作为复平面\(C\)上与复球面北极\(N\)对应的点.

加上无穷远点后的复平面称为\textbf{扩充复平面},记作\(C_{\infty}\),即\[
C_{\infty} = C \cup \{\infty\}.
\]
扩充复平面\(C_{\infty}\)又称为\textbf{闭平面}.对应地,复平面\(C\)因为不含无穷远点,所以又称为\textbf{开平面}.

复球面\(S\)与扩充复平面\(C_{\infty}\)上点之间的映射称为\textbf{球极射影}.\(S\)又称为\textbf{黎曼(Riemann)复球面}.

另外,对于\(\infty\)还有以下几点值得注意:
\begin{enumerate}
\item \(\infty\)的实部\(\Re\infty\)、虚部\(\Im\infty\)、辐角\(\Arg\infty\)均无意义,其模\(\abs{\infty}=+\infty\);
\item 运算\(\infty \pm \infty\)、\(0 \cdot \infty\)、\(\frac{\infty}{\infty}\)均无意义;
\item 设复数\(z \neq \infty\),有\(z \pm \infty = \infty \pm z = \infty\),\(\frac{z}{\infty} = 0\);
\item 设复数\(z \neq 0\),有\(z \cdot \infty = \infty \cdot z = \infty\),\(\frac{z}{0} = \infty\);
\item 设复数\(z \neq 0\)且\(z \neq \infty\),有\(\frac{\infty}{z} = \infty\);
\item 在扩充复平面\(C_{\infty}\)上,任一直线都是通过无穷远点\(\infty\)的.同时,没有一个半平面包含点\(\infty\).
\end{enumerate}
\end{definition}

\begin{theorem}
设复数\(z=x+\iu y\),其对应的复球面上的点为\(Z=\opair{x_1,x_2,x_3}\),并满足:

当\(z \neq \infty\)时,\(Z\)的坐标为\[
x_1 = \frac{z + \complexconjugate{z}}{\abs{z}^2 + 1}, \qquad
x_2 = \iu\frac{\complexconjugate{z} - z}{\abs{z}^2 + 1}, \qquad
x_3 = \frac{\abs{z}^2 - 1}{\abs{z}^2 + 1}
\]或\[
x_1 = \frac{2x}{x^2+y^2+1}, \quad
x_2 = \frac{2y}{x^2+y^2+1}, \quad
x_3 = \frac{x^2+y^2-1}{x^2+y^2+1}.
\]

当\(z = \infty\)时,\(Z\)的坐标为\(N = (0,0,1)\).
\end{theorem}

\begin{theorem}
已知复球面上一点\(Z=(x_1,x_2,x_3)\),则其对应的复平面上的点为\[
z = x+\iu y = \frac{x_1+\iu x_2}{1-x_3}
\]
\end{theorem}

\subsection{复数的三角形式、指数形式}
\begin{definition}[复数的三角形式]
由于平面直角坐标系上的点\(\opair{x,y}\)也可以用极坐标\(\opair{r,\theta}\)表示,因此根据极坐标和直角坐标间的关系,复数\(z=x+\iu y\)也可以表示成\[
z = r(\cos\theta+\iu\sin\theta).
\]这就是复数\(z\)的\textbf{三角形式}.其中\[
r = \sqrt{x^2+y^2} = \abs{z},
\]\[
\tan\theta = \frac{y}{x},
\]\(\theta\)称作复数\(z\)的\textbf{辐角},记作\(\theta = \Arg{z}\).

对于复数\(z=x+\iu y=r(\cos\theta+\iu\sin\theta)\),使得\(\tan\theta=\frac{y}{x}\)成立的\(\theta=\Arg{z}\)值有无穷多个.规定落在区间\((-\pi,\pi]\)内的辐角值为\(\Arg{z}\)的\textbf{主值},或称之为\textbf{主辐角},记作\(\arg{z}\),即\[
\Arg{z} = \arg{z} + 2k\pi, \quad k\in\mathbb{Z},
\]则主辐角\(\arg{z}\)是唯一确定的.
有时候主辐角\(\arg{z}\)被规定落在区间\([0,2\pi)\)内.

注意:\begin{enumerate}
\item 对复数\(z = x + \iu y\),若\(z \neq 0\)且\(z \neq \infty\),则满足\(\tan \theta = \frac{y}{x}\)的\(\theta\)的取值有无穷多个,因此辐角\(\Arg z\)有无穷多个值,但主辐角\(\arg z\)只有一个值;
\item 当复数\(z = 0\)或\(z = \infty\)时,辐角\(\Arg{z}\)、主辐角\(\arg{z}\)均无意义.
\end{enumerate}
\end{definition}

\begin{theorem}
设非零复数\(z=r(\cos\theta+\iu\sin\theta)\)的主辐角为\(\arg{z} \in (-\pi,\pi]\),则有\[
\def\arraystretch{1.5}
\arg{z} = \left\{ \begin{array}{lc}
\arctan{\frac{y}{x}}, &\quad x > 0, \\
\frac{\pi}{2}, &\quad x = 0 \land y > 0, \\
\arctan{\frac{y}{x}} + \pi, &\quad x < 0 \land y \geqslant 0, \\
\arctan{\frac{y}{x}} - \pi, &\quad x < 0 \land y < 0, \\
-\frac{\pi}{2}, &\quad x = 0 \land y < 0,
\end{array} \right.
\]
\end{theorem}

\begin{theorem}
设非零复数\(z=x+\iu y=r(\cos\theta+\iu\sin\theta)\)的主辐角为\(\arg{z} = \alpha\),则\[
\tan{\frac{\alpha}{2}}
= \frac{\sin\alpha}{1+\cos\alpha}
= \frac{r\sin\alpha}{r+r\cos\alpha}
= \frac{y}{r+x}
= \frac{y}{x+\sqrt{x^2+y^2}}
\]所以\[
\arg{z} = \alpha
= 2 \arctan{ \frac{y}{x+\sqrt{x^2+y^2}} }
\]
\end{theorem}

\begin{lemma}
设非零复数\(z_1=\cos\theta_1+\iu\sin\theta_1\)、\(z_2=\cos\theta_2+\iu\sin\theta_2\),则\[
z_1z_2 = (\cos\theta_1+\iu\sin\theta_1)(\cos\theta_2+\iu\sin\theta_2)
= \cos(\theta_1+\theta_2) + \iu\sin(\theta_1+\theta_2),
\]\[
\frac{z_1}{z_2} = \frac{\cos\theta_1+\iu\sin\theta_1}{\cos\theta_2+\iu\sin\theta_2}
= \cos(\theta_1-\theta_2) + \iu\sin(\theta_1-\theta_2),
\]
\end{lemma}

\begin{definition}[欧拉公式与复数的指数形式]
根据以上引理,只要定义\begin{align}
\cos\theta+\iu\sin\theta \equiv e^{\iu\theta},
\end{align}则有\[
e^{\iu\theta_1} e^{\iu\theta_2} = e^{\iu(\theta_1+\theta_2)}
\quad \text{和} \quad
\frac{e^{\iu\theta_1}}{e^{\iu\theta_2}} = e^{\iu(\theta_1-\theta_2)}
\]成立,继而可以将非零复数写成指数形式:\[
z = r e^{\iu\theta}.
\]

复数0的辐角无意义,不能写成指数形式.
\end{definition}

\begin{theorem}[指数形式下的复数的乘除法]
设非零复数\(z_1 = r_1 e^{\iu\theta_1}\)、\(z_2 = r_2 e^{\iu\theta_2}\),则\[
z_1 z_2 = r_1 e^{\iu\theta_1} \cdot r_2 e^{\iu\theta_2} = r_1 r_2 e^{\iu(\theta_1+\theta_2)}
\]\[
\frac{z_1}{z_2} = \frac{r_1 e^{\iu\theta_1}}{r_2 e^{\iu\theta_2}} = \frac{r_1}{r_2} e^{\iu(\theta_1-\theta_2)}
\]
\end{theorem}

\begin{property}
设复数\(z_1\)、\(z_2\),则有\begin{align*}
\Arg{z_1 z_2} &= \Arg{z_1} + \Arg{z_2} \\
\Arg{\frac{z_1}{z_2}} &= \Arg{z_1} - \Arg{z_2} \\
\Arg{\complexconjugate{z}} &= -\Arg{z}
\end{align*}
注意以上等式两边各是无穷多个角度值的集合.

同样地,存在\(k_1,k_2 \in \mathbb{Z}\),使得\begin{align*}
\arg{z_1 z_2} &= \arg{z_1} + \arg{z_2} + 2 k_1 \pi \\
\arg{\frac{z_1}{z_2}} &= \arg{z_1} - \arg{z_2} + 2 k_2 \pi
\end{align*}
成立.
\end{property}

\begin{theorem}[指数形式下复数相等条件]
非零复数\(z_1=r_1 e^{\iu\theta_1}\)、\(z_2=r_2 e^{\iu\theta_2}\)相等的充要条件是:\[
\left\{ \begin{array}{l}
r_1 = r_2 \\
\theta_1 = \theta_2 + 2k\pi
\end{array} \right. \quad (k \in \mathbb{Z})
\]
\end{theorem}

\begin{example}
证明:\[
\sum\limits_{k=1}^n \cos kx
= \frac{\sin \left(n+\frac{1}{2}\right)x}{2 \sin \frac{x}{2}} - \frac{1}{2}.
\]
\end{example}

\subsection{复数的乘幂与方根}
\begin{definition}
设\(n\)是正整数,则\(z^n\)表示\(n\)个\(z\)的乘积.
规定:\(0^n=0\),\(z^0=1\),\(z^{-n}=\frac{1}{z^n}\).
当\(z=re^{\iu\theta} \neq 0\)时,\[
z^n = r^n e^{in\theta} = r^n(\cos n\theta+\iu\sin n\theta),
\quad n \in \mathbb{Z}.
\]
\end{definition}

\begin{theorem}[棣莫弗(De Moivre)公式]
\begin{align}
(\cos\theta+i\sin\theta)^n = \cos{n\theta}+\iu\sin{n\theta}
\end{align}
\end{theorem}

\begin{definition}[复数的方根]
已知\(z\in\mathbb{C}\),关于复数\(w\)的方程\[
w^n = z \quad (n \geqslant 2, n \in \mathbb{Z})
\]的根称为\(z\)的\textbf{\(n\)次方根}.记所有根的总体为\(\sqrt[n]{z}\).

当\(z=0\)时,以上方程只有唯一解\(w = 0\).

当\(z \neq 0\)时,设\(z=re^{\iu\theta}\),\(w=\rho e^{\iu\varphi}\)(其中\(r,\theta,\rho,\varphi\in\mathbb{R}^+\))代入原方程,得\[
\rho^n e^{\iu n \varphi} = r e^{\iu\theta}
\]根据复数相等的充要条件,得\[
\left\{ \begin{array}{l}
\rho^n = r \\
n\varphi = \theta + 2k\pi
\end{array} \right. \quad (k \in \mathbb{Z})
\]由于\(\rho\)和\(r\)都是正实数,故可由\(\rho^n=r\)得出唯一确定的算术根\(\rho=\sqrt[n]{r}\);
又可由\(n\varphi=\theta+2k\pi\)得出\(\varphi=\frac{\theta+2k\pi}{n}\).
也就是说,\(z\)的\(n\)次方根为\[
w_k = (\sqrt[n]{z})_k = \sqrt[n]{r} e^{\iu \frac{\theta + 2k\pi}{n}} \quad (k=0,1,\dotsc,n-1)
\]
\end{definition}

在复平面上,\(\sqrt[n]{z}\)的不同值\(w_k\)表示为\[
w_k = w_0 e^{\iu\frac{2k\pi}{n}} \quad (k \in \mathbb{Z}),
\]对给定的复数\(z\),由\(z\)的模\(r\)和辐角\(\theta\)先在复平面上确定\(w_0 = \sqrt[n]{r} e^{\iu\frac{\theta}{n}}\),然后将\(w_0\)依次绕原点旋转\[
\def\f{\frac{2\pi}{n}}
\f,\,2\cdot\f,\,3\cdot\f,\,\dots,\,(n-1)\cdot\f,
\]包括\(w_0\)在内,就在复平面上得到\(\sqrt[n]{z}\)的总共\(n\)个值的几何表示.它们均匀地分布在中心在原点、半径为\(\sqrt[n]{r}\)的圆周上,即它们是内接于此圆周的正\(n\)角形的\(n\)个顶点.

\subsection{复平面上的几何图形}
复数乘除法的几何意义可由指数形式下的乘除法运算公式得到.复数\(z=z_1 z_2\)对应的向量是把复数\(z_1\)对应的向量先伸缩\(r_2 = \abs{z_2}\)倍,再旋转一个角度\(\theta_2 = \arg z_2\)得到的.

直角坐标平面上任意一条用隐函数方程\(F(x,y)=0\)表示的曲线,经过变量代换即可得到其复方程为\[
F\left(\frac{z+\complexconjugate{z}}{2},\frac{z-\complexconjugate{z}}{2\iu}\right)=0.
\]

\begin{example}[射线]
从点\(z_0\)出发,与正实轴夹角为\(\theta_0\)的射线的复变数方程为\[
\arg(z-z_0) = \theta_0.
\]
\end{example}

\begin{example}[线段]
连接\(z_1\)、\(z_2\)两点的线段的参数方程为\[
z = z_1 + t(z_2 - z_1), \qquad t \in [0,1].
\]
\end{example}

\begin{example}[直线]
过\(z_1\)、\(z_2\)两点的直线的参数方程为\[
z = z_1 + t(z_2 - z_1), \qquad t \in (-\infty,+\infty).
\]

实轴的方程为\(\Im z = 0\);
虚轴的方程为\(\Re z = 0\).
\end{example}

\begin{example}[三点共线的充要条件]
三点\(z_1\)、\(z_2\)、\(z_3\)共线的充要条件为\[
\frac{z_3 - z_1}{z_2 - z_1} = t \neq 0, \qquad t \in \mathbb{R}.
\]
\end{example}

\begin{example}[圆]
以\(z_0\)为圆心,\(R\)为半径的圆周的方程为\[
\abs{z - z_0} = R.
\]而\(\abs{z-z_0}<R\)表示圆的内部,\(\abs{z-z_0}>R\)表示圆的外部.

复平面上圆周的一般方程为\[
A z \complexconjugate{z} + \beta \complexconjugate{z} + \complexconjugate{\beta} z + C = 0
\]其中\(A,C\in\mathbb{R}\),\(A \neq 0\),\(\beta\in\mathbb{C}\),且\[
\abs{\beta}^2 > AC
\]

以\(z_0\)为圆心,\(R\)为半径的圆周的方程还可表示为\[
z - z_0 = R e^{\iu \theta}.
\]
\end{example}

\begin{definition}
设圆\(C: \abs{z - a} = R\).
如果点\(z_1,z_2\)都在从圆心\(a\)出发的同一条射线上,且满足\[
\abs{z_1 - a} \abs{z_2 - a} = R^2,
\]则称点\(z_1,z_2\)关于圆周\(C\)对称.

特别地,规定圆心\(a\)与无穷远点\(\infty\)关于圆周\(C\)对称.
\end{definition}

\begin{theorem}
点\(z_1,z_2\)关于圆周\(C: \abs{z - a} = R\)对称的充要条件是:\[
\complexconjugate{z_a - a} (z_2 - a) = R^2.
\]
\end{theorem}

\begin{theorem}
点\(z_1,z_2\)关于圆周\(C: \abs{z - a} = R\)对称的充要条件是:通过\(z_1,z_2\)的任意圆周都与\(C\)正交.
\end{theorem}
