\chapter{算术基础}

\section{比}
\subsection{比例数}
\begin{definition}
给定两个数\(a\)和\(b\),如果数\(c\)满足\(a=b \cdot c\),则称“\(c\)是\(a\)与\(b\)的\textbf{比例数}(简称\textbf{比})”,记作\(a \propto b\)或\(a:b=c\)或\(a/b=c\)或\(\frac{a}{b}=c\),并称\(a\)和\(b\)为这个比的\textbf{项},称\(a\)为\textbf{前项},称\(b\)为\textbf{后项}.
\end{definition}

\begin{property}
\(\frac{a}{b} = \frac{ma}{mb}\ (m\neq0)\).
\end{property}

\begin{definition}
相对于比\(a:b\),比\(a^2:b^2\)称为\(a:b\)的\textbf{二次比},\(a^3:b^3\)称为\(a:b\)的\textbf{三次比},\(a^{\frac{1}{2}}:b^{\frac{1}{2}}\)称为\(a:b\)的\textbf{平方根比}.
\end{definition}

\begin{example}
设\[
\frac{a}{b} = \frac{c}{d} = \frac{e}{f} = \dotsb = k,
\]证明:\[
\left(\frac{p a^n + q c^n + r e^n + \dotsb}{p b^n + q d^n + r f^n + \dotsb}\right)^{\frac{1}{n}} = k,
\]其中\(p,q,r,\dotsc\)和\(n\)都是任意常数.
\end{example}

\begin{example}
已知\(\frac{a}{b}=\frac{c}{d}=\frac{e}{f}\),证明:\(\frac{a^3b+2c^2e-3ae^2f}{b^4+2d^2f-3bf^3} = \frac{ace}{bdf}\).
\end{example}

\begin{example}
已知\(\frac{x}{a}=\frac{y}{b}=\frac{z}{c}\),证明:\[
\frac{x^2+a^2}{x+a}+\frac{y^2+b^2}{y+b}+\frac{z^2+c^2}{z+c}
= \frac{(x+y+z)^2+(a+b+c)^2}{x+y+z+a+b+c}.
\]
\end{example}

\begin{example}
已知方程\(7x=4y+8z, 3z=12x+11y\),求比\(x:y:z\).
\begin{solution}
方程移项,得\begin{gather*}
7x-4y-8z=0, \\
12x+11y-3z=0,
\end{gather*}
从每个方程的第二项开始写出系数,并利用\cemph{交叉相乘法}\[
\begin{array}{*4r}
-4, & -8, & 7, & -4, \\
11, & -3, & 12, & 11,
\end{array}
\]得到\begin{gather*}
(-4)\times(-3)-11\times(-8)=100, \\
(-8)\times12-(-3)\times7=-75, \\
7\times11-12\times(-4)=125,
\end{gather*}即\[
\frac{x}{100}=\frac{y}{-75}=\frac{z}{125}
\quad\text{或}\quad
\frac{x}{4}=\frac{y}{-3}=\frac{z}{5}.
\]
\end{solution}
\end{example}

\subsection{比例}
\begin{definition}
给定四个数\(a,b,c,d\),如果有\(\frac{a}{b}=\frac{c}{d}\),则称“\(a,b,c,d\)是\textbf{成比例的}”,记作\[
a:b :: c:d
\quad\text{或}\quad
a:b = c:d,
\]并称\(a\)和\(d\)两项为\textbf{外项},称\(b\)和\(c\)为\textbf{内项}.
\end{definition}

显然,当数\(a,b,c,d\)成比例时,\(\frac{a}{b}=\frac{c}{d}\),必有\(b,d\)均不为零,于是\[
bd \cdot \frac{a}{b} = bd \cdot \frac{c}{d},
\]\[
ad = bc,
\]也就是说,“外项之积等于内项之积.”

\begin{definition}
如果数\(a,b,c,d,\dotsc\)满足\[
\frac{a}{b} = \frac{b}{c} = \frac{c}{d} = \dotsb,
\]则称“\(a,b,c,d,\dotsc\)成\textbf{连比}”.

特别地,当\(a,b,c\)成连比(即\(a:b = b:c\))时,称\(b\)为\textbf{比例中项},称\(c\)为\(a\)与\(b\)的\textbf{第三比例项}.
\end{definition}

我们有以下几个平凡的结论:\begin{enumerate}
\item 若\(\frac{a}{b} = \frac{b}{c}\),则\(\frac{a}{c} = \frac{a^2}{b^2}\).
\item 若\(\frac{a}{b} = \frac{c}{d}\)且\(\frac{e}{f} = \frac{g}{h}\),则\(\frac{ae}{bf} = \frac{cg}{dh}\).
\item \textbf{交等定理}\footnote{%
交等定理、反比定理、交比定理、合比定理、分比定理和合分比定理这几个名称实际上取自欧几里得的《几何原本》.%
}.
若\(\frac{a}{b} = \frac{c}{d}\)且\(\frac{b}{x} = \frac{d}{y}\),则\(\frac{a}{x} = \frac{c}{y}\).
\item \textbf{反比定理}.
若\(\frac{a}{b} = \frac{c}{d}\),则\(\frac{b}{a} = \frac{d}{c}\).
\item \textbf{交比定理}.
若\(\frac{a}{b} = \frac{c}{d}\),则\(\frac{a}{c} = \frac{b}{d}\).
\item \textbf{合比定理}.
若\(\frac{a}{b} = \frac{c}{d}\),则\((a+b):b = (c+d):d\).
\item \textbf{分比定理}.
若\(\frac{a}{b} = \frac{c}{d}\),则\((a-b):b = (c-d):d\).
\item \textbf{合分比定理}.
若\(\frac{a}{b} = \frac{c}{d}\),则\(\frac{a+b}{a-b} = \frac{c+d}{c-d}\).
\end{enumerate}

\begin{example}
解方程\[
\frac{\sqrt{x+1}+\sqrt{x-1}}{\sqrt{x+1}-\sqrt{x-1}} = \frac{4x-1}{2}.
\]
\begin{solution}
注意到原方程中\(x\)的取值范围是\(x\geqslant1\).
利用合分比定理,有\[
\frac{\sqrt{x+1}}{\sqrt{x-1}} = \frac{4x+1}{4x-3},
\]所以\[
\frac{x+1}{x-1} = \frac{16x^2+8x+1}{16x^2-24x+9}.
\]再利用合分比定理,有\[
\frac{2x}{2} = \frac{32x^2-16x+10}{32x-8},
\]即\[
x = \frac{16x^2-8x+5}{16x-4},
\]解得\(x=\frac{5}{4} \geqslant1\).
\end{solution}
\end{example}

\subsection{变化率}
\begin{definition}
已知互相关联的两个量\(A\)与\(B\).
如果每当\(B\)变化了,就有\(A\)也随之变化一个相同的比率,那么称“量\(A\)与量\(B\)成\textbf{正比}”,记作\(A \propto B\).
\end{definition}

\begin{theorem}
如果\(A\)与\(B\)成正比,那么\(A\)等于\(B\)乘以一个常量.
\begin{proof}
设数\(\v{a}{0}\)和\(\v{b}{0}\)是\(A\)和\(B\)的对应的取值.
根据定义,有\[
\def\f#1{%
\frac{a_#1}{a_1} = \frac{b_#1}{b_1}%
}
\f2,\f3,\f4,\dotsc,
\]于是\[
\def\f#1{\frac{a_#1}{b_#1}}
\frac{A}{B} = \f1 = \f2 = \f3 = \f4 = \dotsb.
\qedhere
\]
\end{proof}
\end{theorem}

\begin{definition}
已知两个量\(A\)与\(B\).
如果\(A\)与\(B\)的倒数成正比,那么称“量\(A\)与量\(B\)成\textbf{反比}”.
\end{definition}

结合“正比”与“反比”的定义,如果\(A\)同\(B/C\)成正比,那么称“\(A\)同\(B\)成正比,而同\(C\)成反比”.
