\section{常见不等式}
\subsection{三角不等式}
\begin{theorem}[三角不等式I]\label{theorem:不等式.三角不等式1}
设\(a,b\in\mathbb{R}\),那么\begin{equation}
	\abs{a \pm b}
	\leq
	\abs{a} + \abs{b}.
\end{equation}
当且仅当\(ab\geq0\)时,
有\(\abs{a+b}=\abs{a}+\abs{b}\).
当且仅当\(ab\leq0\)时,
有\(\abs{a-b}=\abs{a}+\abs{b}\).
\begin{proof}
对于任意实数\(a,b\),总有\[
	-\abs{a} \leq a \leq \abs{a}, \qquad
	-\abs{b} \leq \pm b \leq \abs{b}.
\]
将上述两式相加,得\[
	-(\abs{a}+\abs{b}) \leq a \pm b \leq \abs{a}+\abs{b},
\]
即\(\abs{a \pm b} \leq \abs{a}+\abs{b}\).

下面来研究不等式\(\abs{a \pm b} \leq \abs{a} + \abs{b}\)取等号的充分必要条件.
我们可以按\(a>0,a=0,a<0\)以及\(b>0,b=0,b<0\)来分情况讨论.

当\(b=0\)时,
有\[
	\abs{a \pm b} = \abs{a} = \abs{a} + \abs{b}.
\]
当\(a=0\)时,
有\[
	\abs{a \pm b} = \abs{b} = \abs{a} + \abs{b}.
\]
当\(a>b>0\)时,
有\[
	\abs{a + b} = a + b = \abs{a} + \abs{b}, \qquad
	\abs{a - b} = a - b < a + b = \abs{a} + \abs{b}.
\]
当\(b>a>0\)时,
有\[
	\abs{a + b} = a + b = \abs{a} + \abs{b}, \qquad
	\abs{a - b} = b - a < a + b = \abs{a} + \abs{b}.
\]
当\(a<b<0\)时,
有\[
	\abs{a + b} = -a-b = \abs{a} + \abs{b}, \qquad
	\abs{a - b} = b-a < -a-b = \abs{a} + \abs{b}.
\]
当\(b<a<0\)时,
有\[
	\abs{a + b} = -a-b = \abs{a} + \abs{b}, \qquad
	\abs{a - b} = a-b < -a-b = \abs{a} + \abs{b}.
\]
当\(a>0>b\)时,
有\[
	\abs{a + b} \neq a - b = \abs{a} + \abs{b}, \qquad
	\abs{a - b} = a - b = \abs{a} + \abs{b}.
\]
当\(a<0<b\)时,
有\[
	\abs{a + b} \neq b - a = \abs{a} + \abs{b}, \qquad
	\abs{a - b} = b - a = \abs{a} + \abs{b}.
\]
综上所述,\[
	\abs{a + b} = \abs{a} + \abs{b} \iff ab \geq 0, \qquad
	\abs{a - b} = \abs{a} + \abs{b} \iff ab \leq 0.
	\qedhere
\]
\end{proof}
\end{theorem}

\begin{theorem}[三角不等式II]\label{theorem:不等式.三角不等式2}
设\(a,b\in\mathbb{R}\),那么\begin{equation}
	\abs{\abs{a}-\abs{b}}
	\leq
	\abs{a \pm b}.
\end{equation}
当且仅当\(ab\leq0\)时,
有\(\abs{\abs{a}-\abs{b}}=\abs{a+b}\).
当且仅当\(ab\geq0\)时,
有\(\abs{\abs{a}-\abs{b}}=\abs{a-b}\).
\begin{proof}
由\cref{theorem:不等式.三角不等式1},
有\[
	\abs{a}
	=\abs{(a-b)+b}
	\leq\abs{a-b}+\abs{b},
\]
移项得\[
	\abs{a}-\abs{b}\leq\abs{a-b}.
	\eqno(1)
\]
同理,有\[
	\abs{b}
	=\abs{(b-a)+a}
	\leq\abs{b-a}+\abs{a}
	=\abs{a-b}+\abs{a},
\]
移项得\[
	\abs{b}-\abs{a}\leq\abs{a-b},
\]
也即\[
	-\abs{a-b}\leq\abs{a}-\abs{b}.
	\eqno(2)
\]
因此,我们可以将(1)(2)两式合写为\[
	-\abs{a-b}\leq\abs{a}-\abs{b}\leq\abs{a-b},
\]
也即\[
	\abs{\abs{a}-\abs{b}}\leq\abs{a-b}.
	\eqno(3)
\]

由\cref{theorem:不等式.三角不等式1},
又有\[
	\abs{a}
	=\abs{(a+b)-b}
	\leq\abs{a+b}+\abs{b},
	\qquad
	\abs{b}
	=\abs{(b+a)-a}
	\leq\abs{b+a}+\abs{a},
\]
分别移项得\[
	\abs{a}-\abs{b}\leq\abs{a+b}, \qquad
	\abs{b}-\abs{a}\leq\abs{b+a},
\]
于是\[
	-\abs{a+b}\leq\abs{a}-\abs{b}\leq\abs{a+b},
\]
也即\[
	\abs{\abs{a}-\abs{b}}\leq\abs{a+b}.
	\eqno(4)
\]

最后,我们将(3)(4)两式合写为\[
	\abs{\abs{a}-\abs{b}}\leq\abs{a \pm b}.
	\qedhere
\]

下面来研究不等式\(\abs{\abs{a}-\abs{b}} \leq \abs{a \pm b}\)取等号的充分必要条件.

当\(a=0\)时,有\[
	\abs{\abs{a}-\abs{b}}
	= \abs{b}
	= \abs{a \pm b}.
\]
当\(b=0\)时,有\[
	\abs{\abs{a}-\abs{b}}
	= \abs{a}
	= \abs{a \pm b}.
\]
当\(a>b>0\)时,有\[
	\abs{\abs{a}-\abs{b}}
	= \abs{a-b}
	= a-b
	= \abs{a-b}.
\]
当\(b>a>0\)时,有\[
	\abs{\abs{a}-\abs{b}}
	= \abs{a-b}
	= b-a
	= \abs{a-b}.
\]
当\(a,b<0\)时,有\[
	\abs{\abs{a}-\abs{b}}
	= \abs{-a-(-b)}
	= \abs{b-a}
	= \abs{a-b}
	\neq \abs{a+b}.
\]
当\(a>0>b\)时,有\[
	\abs{\abs{a}-\abs{b}}
	= \abs{a-(-b)}
	= \abs{a+b}.
\]
当\(b>0>a\)时,有\[
	\abs{\abs{a}-\abs{b}}
	= \abs{(-a)-b}
	= \abs{a+b}.
\]
综上所述,\[
	\abs{\abs{a}-\abs{b}}=\abs{a-b} \iff ab \geq 0, \qquad
	\abs{\abs{a}-\abs{b}}=\abs{a+b} \iff ab \leq 0.
	\qedhere
\]
\end{proof}
\end{theorem}

现在我们可以把\cref{theorem:不等式.三角不等式1,theorem:不等式.三角不等式2} 合并写出如下不等式:
\begin{equation}
	\abs{\abs{a}-\abs{b}}\leq\abs{a \pm b}\leq\abs{a}+\abs{b}.
\end{equation}

\begin{corollary}
设\(\AutoTuple{a}{n}\)是实数,
则\[
	\abs{\sum_{i=1}^n a_i}
	\leq
	\sum_{i=1}^n \abs{a_i}.
\]
\begin{proof}
用数学归纳法.
当\(n=1\)时,
\(\abs{\sum_{i=1}^n a_i} = \abs{a_1}\)
而\(\sum_{i=1}^n \abs{a_i} = \abs{a_1}\),
命题成立.
假设当\(n=k\)时,
命题也成立,
即有\[
	\abs{\sum_{i=1}^k a_i}
	\leq
	\sum_{i=1}^k \abs{a_i}.
\]
那么当\(n=k+1\)时,
易得\begin{align*}
	\abs{\sum_{i=1}^{k+1} a_i}
	&= \abs{\left(\sum_{i=1}^k a_i\right) + a_{k+1}} \\
	&\leq \abs{\sum_{i=1}^k a_i} + \abs{a_{k+1}}
		\tag{\hyperref[theorem:不等式.三角不等式1]{三角不等式}} \\
	&\leq \left(\sum_{i=1}^k \abs{a_i}\right) + \abs{a_{k+1}}
		\tag{归纳假设} \\
	&= \sum_{i=1}^{k+1} \abs{a_i}.
\end{align*}
因此,命题对任意正整数均成立.
\end{proof}
\end{corollary}

\subsection{基本不等式}
\begin{theorem}\label{theorem:不等式.基本不等式1}
如果\(a,b\in\mathbb{R}\),
那么\(a^2 + b^2 \geq 2ab\)
(当且仅当\(a=b\)时取“\(=\)”号).
\begin{proof}
因为\(a,b\in\mathbb{R}\),
所以\(a-b\in\mathbb{R}\),
\((a-b)^2 = a^2 - 2ab + b^2 \geq 0\),
移项得\[
	a^2 + b^2 \geq 2ab.
\]

当\(a=b\)时,
有\(a^2+b^2
=2a^2
=2ab\),
因此\[
	a^2 + b^2 = 2ab
	\iff
	a = b.
	\qedhere
\]
\end{proof}
\end{theorem}

在\cref{theorem:不等式.基本不等式1} 中,
分别用\(\sqrt{a}\)和\(\sqrt{b}\)代\(a\)和\(b\),
立即得到如下推论.
\begin{corollary}\label{corollary:不等式.基本不等式2}
如果\(a,b\in\mathbb{R}^+\),
那么\(\frac{a+b}{2} \geq \sqrt{ab}\)
(当且仅当\(a=b\)时取“\(=\)”号).
\end{corollary}

类似地,我们还可以得到如下结论.
\begin{theorem}\label{theorem:不等式.基本不等式3}
如果\(a,b,c\in\mathbb{R}^+\),
那么\(a^3 + b^3 + c^3 \geq 3abc\)
(当且仅当\(a=b=c\)时取“\(=\)”号).
\begin{proof}
因为\[
	a^3+b^3+c^3-3abc
	= (a+b+c)(a^2+b^2+c^2-ab-bc-ca),
\]
所以\[
	a^3 + b^3 + c^3 \geq 3abc
	\iff
	a^2+b^2+c^2-ab-bc-ca
	= \frac12 \left[
		(a-b)^2+(b-c)^2+(c-a)^2
	\right]
	\geq 0.
\]
显然成立.
\end{proof}
\end{theorem}

\begin{corollary}\label{theorem:不等式.基本不等式4}
如果\(a,b,c\in\mathbb{R}^+\),
那么\(\frac{a+b+c}{3} \geq \sqrt[3]{abc}\)
(当且仅当\(a=b=c\)时取“\(=\)”号).
\end{corollary}

\begin{theorem}[均值不等式]\label{theorem:不等式.均值不等式}
如果\(\AutoTuple{x}{n}\in\mathbb{R}^+\),那么
\begin{equation}
n \left( \frac{1}{x_1} + \dotsb + \frac{1}{x_n} \right)^{-1}
\leq \sqrt[n]{x_1 \dotsm x_n}
\leq \frac{x_1 + \dotsb + x_n}{n}
\leq \sqrt{\frac{x_1^2 + \dotsb + x_n^2}{n}}.
\end{equation}
\rm
上述四个式子依次分别称为\DefineConcept{调和平均数}、\DefineConcept{几何平均数}、\DefineConcept{算术平均数}、\DefineConcept{平方平均数}.
\end{theorem}

\begin{corollary}\label{theorem:不等式.基本不等式6}
如果\(\AutoTuple{x}{n},\AutoTuple{p}{n}\in\mathbb{R}^+\),那么
\begin{equation}
x_1^{p_1} \dotsm x_n^{p_n}
\leq
\left( \frac{p_1 x_1 + \dotsb + p_n x_n}{p_1 + \dotsb + p_n} \right)^{p_1 + \dotsb + p_n}.
\end{equation}
\end{corollary}

\subsection{杨格不等式}
\begin{corollary}[杨格不等式]\label{theorem:不等式.杨格不等式}
设\(a,b\geq0\),\(p,q>1\)且\(\frac{1}{p}+\frac{1}{q}=1\),则
\begin{equation}
ab \leq \frac{a^p}{p} + \frac{b^q}{q}.
\end{equation}
\end{corollary}

\subsection{伯努利不等式}
\begin{theorem}[伯努利不等式]\label{theorem:不等式.伯努利不等式}
\begin{equation}
(1+x_1)(1+x_2)\dotsm(1+x_n) \geq 1+x_1+x_2+\dotsb+x_n,
\end{equation}
式中\(\AutoTuple{x}{n}\)是符号相同且大于\(-1\)的数.
\begin{proof}
用数学归纳法.
当\(n=1\)时两边相等,故结论成立.现假设\(n=k\)时结论成立,即\((1+x_1)(1+x_2)\dotsm(1+x_k) \geq 1+x_1+x_2+\dotsb+x_k\).那么当\(n=k+1\)时,\begin{align*}
&(1+x_1)(1+x_2)\dotsm(1+x_k)(1+x_{k+1}) \\
&\geq (1+x_1+x_2+\dotsb+x_k)(1+x_{k+1}) \\
&= (1+x_1+x_2+\dotsb+x_k+x_{k+1}) + x_{k+1}(x_1+x_2+\dotsb+x_k).
\end{align*}
若\(x_1,x_2,\dotsc,x_{k+1}\)都大于零,则\(x_{k+1}(x_1+x_2+\dotsb+x_k) > 0\),\begin{gather}
(1+x_1)(1+x_2)\dotsm(1+x_k)(1+x_{k+1}) \geq 1+x_1+x_2+\dotsb+x_k+x_{k+1}
\tag1
\end{gather}成立;若\(x_1,x_2,\dotsc,x_{k+1}\)都等于零,则\(x_{k+1}(x_1+x_2+\dotsb+x_k) = 0\),同样有(1)式成立;若\(x_1,x_2,\dotsc,x_{k+1}\in(-1,0)\),则\(x_1+x_2+\dotsb+x_k < 0\),\(x_{k+1}(x_1+x_2+\dotsb+x_k) > 0\),仍然有(1)式成立.
\end{proof}
\end{theorem}

\begin{example}
证明:当\(x > -1\)时,不等式\begin{equation}
(1+x)^n \geq 1+nx \quad (n>1)
\end{equation}成立,当且仅当\(x=0\)时取“\(=\)”.
\begin{proof}
用数学归纳法.
当\(n=2\)时,\((1+x)^2 = 1+2x+x^2 \geq 1+2x\)成立.
假设当\(n=k\)时有\((1+x)^k \geq 1+kx\)成立.
当\(n=k+1\)时,\begin{align*}
(1+x)^{k+1}
&= (1+x)^k (1+x) \\
&\geq (1+kx)(1+x) \\
&= 1 + (k+1)x + kx^2 \\
&\geq 1 + (k+1)x.
\qedhere
\end{align*}
\end{proof}
实际上,当\(-2 \leq x \leq -1\)时,上述不等式仍然成立.

\begin{enumerate}
\item 当\(x = -1\)时,左边\((1+x)^n \equiv 0\),右边\(1+nx = 1 - n < 0\),上述不等式成立.

\item 当\(x = -2\)时,左边\((1+x)^n = (-1)^n = \pm1\),右边\[
1+nx = 1-2n < -1,
\]即上述不等式成立.

\item 当\(-2 < x < -1\)时,\(-1 < 1+x < 0\),故\(1+x < (1+x)^n\);
又因为\(n>1\),\(nx<x\),故\(1+nx < 1+x < (1+x)^n\).
\end{enumerate}
综上所述,不等式\[
(1+x)^n \geq 1+nx \quad (n>1)
\]当\(-2 \leq x \leq -1\)时仍然成立.
\end{example}

\subsection{柯西不等式}
\begin{theorem}
设\(a,b,c,d\in\mathbb{R}\),则
\begin{equation}
	(ab+cd)^2
	\leq
	(a^2+b^2)(c^2+d^2).
\end{equation}
当且仅当\(ad=bc\)时,上式取“=”号.
\end{theorem}

\begin{theorem}[柯西不等式]\label{theorem:不等式.柯西不等式}
设\(\AutoTuple{a}{n},\AutoTuple{b}{n}\in\mathbb{R}\),则
\begin{equation}
	(a_1 b_1 + a_2 b_2 + \dotsb + a_n b_n)^2
	\leq
	(a_1^2 + a_2^2 + \dotsb + a_n^2) (b_1^2 + b_2^2 + \dotsb + b_n^2).
\end{equation}
当且仅当\(\frac{a_i}{b_i} = \lambda\ (i=1,2,\dotsc,n)\)
或\(\frac{b_i}{a_i} = \lambda\ (i=1,2,\dotsc,n)\),
上式取“\(=\)”号.
\begin{proof}
记\begin{align*}
	A &= a_1^2 + a_2^2 + \dotsb + a_n^2, \\
	B &= b_1^2 + b_2^2 + \dotsb + b_n^2, \\
	D &= a_1 b_1 + a_2 b_2 + \dotsb + a_n b_n.
\end{align*}
那么,要证\(D^2 \leq AB\),
即证\(D^2 - AB \leq 0\).
这个不等式的左边与二次三项式\(A x^2 + 2D x + B\)的判别式
\(4D^2-4AB=4(D^2-AB)\)只相差一个因子.
因此\begin{align*}
	D^2 - AB \leq 0
	&\iff
	\text{关于\(x\)的方程\(A x^2 + 2D x + B = 0\)要么没有实根,要么只有一个实根} \\
	&\iff
	(\forall x\in\mathbb{R})
	[A x^2 + 2D x + B \geq 0].
\end{align*}

任取\(x\in\mathbb{R}\),
总有\((a_i x + b_i)^2\geq0\ (i=1,2,\dotsc,n)\),
于是\begin{align*}
	\sum_{i=1}^n (a_i x + b_i)^2
	&= x^2 \sum_{i=1}^n a_i^2
	+ 2 x \sum_{i=1}^n a_i b_i
	+ \sum_{i=1}^n b_i^2 \\
	&= A x^2 + 2D x + B
	\geq 0.
\end{align*}

不难看出\begin{align*}
	a_i x + b_i = 0\ (i=1,2,\dotsc,n)
	&\iff A x^2 + 2D x + B = 0 \\
	&\iff \text{方程\(A x^2 + 2D x + B = 0\)只有一个实根} \\
	&\iff D^2 - AB = 0.
\end{align*}
这就是柯西不等式的取等条件.
\end{proof}
%@see: https://en.wikipedia.org/wiki/Cauchy%E2%80%93Schwarz_inequality
\end{theorem}

\begin{example}
证明:\(C_n^k < n^k\).
\begin{proof}
显然有\begin{align*}
C_n^k &= \frac{n \cdot (n-1) \dotsm (n-k+1)}{k \cdot (k-1) \dotsm 1} \\
&< n \cdot (n-1) \dotsm (n-k+1) \\
&< n \cdot n \dotsm n = n^k.
\qedhere
\end{align*}
\end{proof}
\end{example}

\subsection{赫尔德不等式}
\begin{theorem}[赫尔德不等式]\label{theorem:不等式.赫尔德不等式}
设\(\AutoTuple{x}{n},\AutoTuple{y}{n}\geq0\),\(p,q>1\),且满足\(\frac{1}{p}+\frac{1}{q}=1\),则
\def\s{\sum_{i=1}^n}%
\def\sp#1#2#3{\left( \s #1^#2 \right)^{#3/#2}}%
\begin{equation}
\s x_i y_i
\leq
\sp{x_i}{p}{1} \sp{y_i}{q}{1}.
\end{equation}
\begin{proof}
令\[
a_j = x_j \sp{x_i}{p}{-1}, \qquad
b_j = y_j \sp{y_i}{q}{-1}.
\]那么根据\cref{theorem:不等式.杨格不等式} 得到\[
\s a_i b_i \leq \s \left( \frac{a_i^p}{p} + \frac{b_i}{q} \right)
\leq \frac{1}{p} + \frac{1}{q} = 1.
\qedhere
\]
\end{proof}
\end{theorem}

\subsection{闵可夫斯基不等式}
\begin{theorem}[闵可夫斯基不等式]\label{theorem:不等式.闵可夫斯基不等式}
设\(\AutoTuple{x}{n},\AutoTuple{y}{n}\in\mathbb{R}\),\(p\geq1\),则
\def\s{\sum_{i=1}^n}%
\def\sumonly#1{\s \abs{#1}^p}%
\newcommand\sumpower[2][1]{\left( \sumonly{#2} \right)^{\frac{#1}{p}}}%
\begin{equation}
\sumpower{x_i+y_i} \leq \sumpower{x_i} + \sumpower{y_i}.
\end{equation}
\begin{proof}
利用\cref{theorem:不等式.赫尔德不等式} 就有\[\begin{aligned}
&\hspace{-20pt}\sumonly{x_i+y_i}
= \s \abs{x_i+y_i} \abs{x_i+y_i}^{p-1}
\leq \s (\abs{x_i}+\abs{y_i}) \abs{x_i+y_i}^{p-1} \\
&\leq \sumpower{x_i} \sumpower[p-1]{x_i+y_i}
+ \sumpower{y_i} \sumpower[p-1]{x_i+y_i}.
\end{aligned}\]
整理即得欲证不等式.
\end{proof}
\end{theorem}
