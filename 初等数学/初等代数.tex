\chapter{初等代数}
\section{排列组合}
\subsection{基本原理}
\begin{axiom}[加法原理]
如果做一件事,完成它可以有\(n\)类办法,在第一类办法中有\(m_1\)种不同的方法,在第二类办法中有\(m_2\)种不同的方法,……,在第\(n\)类办法中有\(m_n\)种不同的方法,那么完成这件事共有\[
N = m_1 + m_2 + \dotsb + m_n
\]种不同的方法.
\end{axiom}

\begin{axiom}[乘法原理]
如果做一件事,完成它需要分成\(n\)个步骤,做第一步有\(m_1\)种不同的方法,做第二部有\(m_2\)种不同的方法,……,做第\(n\)步有\(m_n\)种不同的方法,那么完成这件事共有\[
N = m_1 \times m_2 \times \dotsm \times m_n
\]种不同的方法.
\end{axiom}

\begin{theorem}[抽屉原理]\label{theorem:抽屉原理}
抽屉原理有以下几种形式:
\begin{enumerate}
\item 把\(n+1\)个元素放入\(n\)个集合内,则一定有一个集合里有两个或两个以上的元素.
\item 把\(m\)个元素任意放入\(n\ (n<m)\)个集合里,则一定有一个集合里至少有\(k\)个元素,其中\[
k = \left\{ \begin{array}{ll}
m/n, & m \pmod n = 0, \\
\floor{m/n}+1, & m \pmod n \neq 0.
\end{array} \right.
\]
\item 把无穷多个元素放入有限个集合里,则一定有一个集合里含有无穷多个元素.
\end{enumerate}
\end{theorem}
抽屉原理有时候也称作\DefineConcept{鸽巢原理};因它最先是由狄利克雷明确地提出来的,因此也称作\DefineConcept{狄利克雷原理}.

\subsection{排列、组合的基本概念}
\begin{definition}[排列]
从\(n\)个相异元素中取出\(k\)个元素,排列数为\[
A_n^k = \frac{n!}{(n-k)!} = n \cdot (n-1) \dotsm (n-k+1)
\]
\end{definition}

\begin{definition}[组合]
从\(n\)个相异元素中取出\(k\)个元素,组合数为\[
\binom{n}{k} =
C_n^k = \frac{n!}{k!(n-k)!} = \frac{n \cdot (n-1) \dotsm (n-k+1)}{k \cdot (k-1) \dotsm 1}
\]

特别地,规定:当\(n < k\)时,\(C_n^k = 0\).
\end{definition}

\subsection{组合数的性质}
\begin{property}\label{theorem:组合数性质1}
\(C_n^k = C_n^{n-k}\).
\begin{proof}
\(
\displaystyle
C_n^{n-k}
= \frac{n!}{(n-k)! [n-(n-k)]!}
= \frac{n!}{k! (n-k)!}
= C_n^k
\).
\end{proof}
\end{property}

\begin{property}\label{theorem:组合数性质2}
\(C_{n-1}^{k-1} + C_{n-1}^k = C_n^k\).
\begin{proof}
\begin{align*}
C_{n-1}^{k-1} + C_{n-1}^k
&= \frac{(n-1)!}{(k-1)! (n-k)!} + \frac{(n-1)!}{k! (n-k-1)!} \\
&= \frac{(n-1)!}{(k-1)! (n-k-1)!} \left( \frac{1}{n-k} + \frac{1}{k} \right) \\
&= \frac{(n-1)!}{(k-1)! (n-k-1)!} \frac{n}{(n-k)k}
= \frac{n!}{k! (n-k)!}
= C_n^k. \qedhere
\end{align*}
\end{proof}
\end{property}

\begin{property}\label{theorem:组合数性质3}
\(\sum\limits_{i=0}^n C_n^i = 2^n\).
\begin{proof}
当\(n=0\)时,\(\sum\limits_{i=0}^0 C_0^i = C_0^0 = \frac{0!}{0! \cdot 0!} = 1 = 2^0\)成立.
假设\(n=k\)时,结论仍成立,即\[
\sum\limits_{i=0}^k C_k^i
= C_k^0 + C_k^1 + \dotsb + C_k^k = 2^k.
\]那么\[
\begin{array}{*{14}{c}}
& C_k^0 &+& C_k^1 &+& \dotsb &+& C_k^{k-1} &+& C_k^k && &=& 2^k \\
+) & && C_k^0 &+& \dotsb &+& C_k^{k-2} &+& C_k^{k-1} &+& C_k^k &=& 2^k \\ \hline
& C_k^0 &+& C_{k+1}^1 &+& \dotsb &+& C_{k+1}^{k-1} &+& C_{k+1}^k &+& C_k^k &=& 2 \cdot 2^k
\end{array}
\]又因为\(C_k^0 = C_{k+1}^0 = C_k^k = C_{k+1}^{k+1} = 1\),所以\[
\sum\limits_{i=0}^{k+1} C_{k+1}^i = 2^{k+1}
\]成立.
\end{proof}
\end{property}

\begin{property}\label{theorem:组合数性质4}
\(\sum\limits_{k=0}^n (-1)^k C_n^k = 0\).
\end{property}

\begin{property}\label{theorem:组合数性质5}
\(\sum\limits_{k=0}^{\floor{n/2}} C_n^{2k} = 2^{n-1}\).
\end{property}
\begin{property}\label{theorem:组合数性质6}
\(\sum\limits_{k=0}^{\floor{(n-1)/2}} C_n^{2k+1} = 2^{n-1}\).
\end{property}

\begin{property}\label{theorem:组合数性质7}
\(C_{n+m}^k = \sum\limits_{r=0}^{k} C_n^r C_m^{k-r}\).
\end{property}

\begin{property}\label{theorem:组合数性质8}
\(\sum\limits_{k=0}^n (C_n^k)^2 = C_{2n}^n\).
\end{property}

\begin{property}\label{theorem:组合数性质9}
\(C_n^{r_1} C_{n-r_1}^{r_2} \dotsm C_{n-(r_1+r_2+\dotsb+r_{k-1})}^{r_k}
= \frac{n!}{r_1! r_2! \dotsm r_k!}\).
\end{property}

\section{常见代数公式}

\begin{theorem}[平方差、立方差公式]
\[
a^2 - b^2 = (a-b)(a+b),
\]\[
a^3 - b^3 = (a-b)(a^2+ab+b^2),
\]

推广一下可得,当\(n \in \mathbb{N}^+\)时,有\[
a^n - b^n = (a-b) \sum\limits_{k=0}^{n-1}{a^{n-1-k} b^k}
= (a-b)(a^{n-1} + a^{n-2} b + \dotsb + b^{n-1}).
\]
\end{theorem}

\begin{theorem}[完全平方公式、牛顿二项式定理]
\[
(a \pm b)^2 = a^2 \pm 2ab + b^2,
\]\[
(a \pm b)^3 = a^3 \pm 3 a^2 b + 3 a b^2 \pm b^3.
\]

推广一下可得,当\(n \in \mathbb{N}^+\)时,有\[
(x+y)^n = \sum_{k=0}^{n} {C_n^k x^{n-k} y^k}.
\]若令\(y=1\),则有\[
(1+x)^n = \sum_{k=0}^{n}{C_n^k x^k}.
\]
\end{theorem}

\section{初等代数方程}
\subsection{一元二次方程}
一元二次方程的\textbf{一般形式}为:\[
ax^2 + bx + c = 0, \quad a \neq 0. \eqno{(1)}
\]其中,\(ax^2\)是二次项,\(bx\)是一次项,\(c\)是常数项;
\(a\)、\(b\)、\(c\)被称作系数.

(1)式两端同除以\(a\),得\[
x^2 + \frac{b}{a} x + \frac{c}{a} = 0, \eqno{(2)}
\]配方,得\[
\left( x + \frac{b}{2a} \right)^2 + \left( \frac{c}{a} - \frac{b^2}{4a^2} \right) = 0,
\]移项,再开方,得\[
x = -\frac{b}{2a} \pm \sqrt{\frac{b^2}{4a^2} - \frac{c}{a}}
= -\frac{b}{2a} \pm \sqrt{\frac{b^2-4ac}{4a^2}}
= \frac{-b \pm \sqrt{b^2-4ac}}{2a}.
\]
于是我们得到一元二次方程\(ax^2 + bx + c = 0\ (a\neq0)\)的两个解\[
x_1 = \frac{-b + \sqrt{b^2-4ac}}{2a},
\qquad
x_2 = \frac{-b - \sqrt{b^2-4ac}}{2a}.
\]

记\(\Delta = b^2-4ac\),称之为方程(1)的\textbf{判别式}(discriminant).
当\(\Delta > 0\)时,它有两个不同的实根\[
x = \frac{-b \pm \sqrt{\Delta}}{2a};
\]当\(\Delta = 0\)时,它有两个相同的实根\[
x = -\frac{b}{2a};
\]当\(\Delta < 0\)时,它有一对共轭复根\[
x = \frac{-b \pm \iu \sqrt{-\Delta}}{2a}.
\]

\begin{theorem}[韦达定理]
设数\(x_1,x_2\)是一元二次方程{\rm(1)}的两个根,则有\[
x_1 + x_2 = -\frac{b}{a},
\qquad
x_1 \cdot x_2 = \frac{c}{a}.
\]
\begin{proof}
因为\(x_1,x_2\)是一元二次方程\(ax^2 + bx + c = 0\)的两个根,所以原方程可化为\[
a(x - x_1)(x - x_2) = 0
\quad\text{或}\quad
a x^2 - a (x_1 + x_2) x + a x_1 x_2 = 0.
\]将上式与方程(1)比较可得\[
b = -a (x_1 + x_2),
\qquad
c = a x_1 x_2.
\]整理得\(x_1 + x_2 = -\frac{b}{a}, x_1 \cdot x_2 = \frac{c}{a}\).
\end{proof}
\end{theorem}

\subsection{一元三次方程}
对于一般的一元三次方程\[
ax^3+bx^2+cx+d=0 \quad(a\neq0),
\]我们总可通过以下步骤将其化为标准形式.

首先在等号两边同除以\(a\),得\[
x^3+\frac{b}{a}x^2+\frac{c}{a}x+\frac{d}{a}=0,
\]再令\(x=y-\frac{b}{3a}\),得\[
\left(y-\frac{b}{3a}\right)^3+\frac{b}{a}\left(y-\frac{b}{3a}\right)^2+\frac{c}{a}\left(y-\frac{b}{3a}\right)+\frac{d}{a}=0,
\]整理得\[
y^3+py+q=0,
\]其中\[
p = \frac{3ac-b^2}{3a^2}, \qquad
q = \frac{2b^3}{27a^3}-\frac{bc}{3a^2}+\frac{d}{a}.
\]

\begin{theorem}[卡丹(Cardano)公式]
\def\a{-\frac{q}{2}}%
\def\d{\frac{q^2}{4}+\frac{p^3}{27}}
\def\b{\sqrt{\d}}%
\def\c#1{\sqrt[3]{\a#1\b}}%
形如\[
x^3 + px + q = 0 \quad (p,q \in \mathbb{C})
\]的一元三次方程的解为\[
x = \c{+}+\c{-}.
\]

令\[
\alpha=\c{+}, \qquad \beta=\c{-}.
\]总有\[
\alpha \beta = -\frac{p}{3}
\]成立.

当\(p,q\in\mathbb{R}\)时,判别式\[
\Delta = -108\left(\d\right) = -27q^2-4p^3
\]的正负号决定了\(x^3+px+q=0\)的根的性质:\begin{enumerate}
\item 当\(\Delta>0\)时,方程的三个根是各不相同的实根.
\item 当\(\Delta=0\)时,\begin{enumerate}
	\item 如果\(p=q=0\),则方程有三重实根;
	\item 如果\(p\neq0\)且\(q\neq0\),则方程有一个二重实根和一个与之不同的实根.
	\end{enumerate}
\item 当\(\Delta<0\)时,方程的三个根各不相同,其中一个是实根,两个是共轭复根.
\end{enumerate}
\end{theorem}
