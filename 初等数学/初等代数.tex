\chapter{初等代数}
\section{排列组合}
\subsection{排列、组合的基本概念}
从若干个元素中取出几个或全部的一种排法,称作是一个\DefineConcept{排列}.
例如,从\(a,b,c,d\)四个字母里一次取出两个的排列数有12个,即\[
	\begin{gathered}
	ab, \quad
	ac, \quad
	ad, \quad
	bc, \quad
	bd, \quad
	cd, \\
	ba, \quad
	ca, \quad
	da, \quad
	cb, \quad
	db, \quad
	dc;
	\end{gathered}
\]
其中每一个都代表两个字母的不同的排法.

从若干个元素中取出几个或全部的一种选法,称作是一个\DefineConcept{组合}.
例如,从\(a,b,c,d\)四个字母里一次取出两个的组合数有6个,即\[
	ab, \quad
	ac, \quad
	ad, \quad
	bc, \quad
	bd, \quad
	cd;
\]
其中每一个都代表两个字母的不同的选法.

从这些例子中我们看到,组合仅与每个选法所含元素的个数有关,
而排列还要考虑元素在每一个排法中的次序.
例如,从四个字母\(a,b,c,d\)中选三个字母可以得到\(abc\)这样一种组合,
可以作以下六种不同的排列:\[
	abc, \quad
	acd, \quad
	bca, \quad
	bac, \quad
	cab, \quad
	cba.
\]

\subsection{排列组合的基本原理}
在讨论排列与组合的一般性命题之前,我们先通过几个例子来说明两个重要的原则.

\subsubsection{乘法原理}
如果做一件事,完成它需要分成\(n\)个步骤,
做第一步有\(m_1\)种不同的方法,
做第二步有\(m_2\)种不同的方法,……,
做第\(n\)步有\(m_n\)种不同的方法,那么完成这件事共有\[
N = m_1 \times m_2 \times \dotsm \times m_n
\]种不同的方法.

\begin{example}
有10艘汽艇往返于利物浦与都柏林之间,问某人来回乘坐不同汽艇的方式有多少种?
\begin{solution}
去时有10种方式;对于其中每一种,因为不能乘坐同一条船,回来时有9种方式可选;
因此,一共有\(10 \times 9 = 90\)种方法来完成这两段路程.
\end{solution}
\end{example}

\begin{example}
3个旅行者来到一个小镇,镇上有4家旅店,若他们分别住不同的旅店,问投宿的方式有多少种?
\begin{solution}
第一个人可以有4种选择;
在他选定后,第二个人有3种选择;
从而这两个人共有\(4 \times 3\)种选择,
对于其中每一种选择,第三个人有2家旅店供选择;
所以,投宿的方式共有\(4 \times 3 \times 2 = 24\)种.
\end{solution}
\end{example}

我们现在来求从\(n\)个不同元素中一次取出\(k\)个的排列数.
这可以看成是用我们手上的\(n\)个不同元素去填满\(k\)个空位的不同方式的种数.

填第一个位置有\(n\)种方式,
因为可以取\(n\)个元素中的任意一个.
在这个位置用任意一种方式填好后,
填第二个位置有\(n-1\)种方式.
由于填第一个位置的每一种方式,都可与填第二个位置的每一种方式组合,
所以填前两个位置的方式有\(n(n-1)\)种.
前两个位置以任意一种方式填好后,
填第三个位置有\(n-2\)种方式.
同理,填前三个位置的方式共有\(n(n-1)(n-2)\)种.
继续这样的方法,我们注意到每填一个新的位置,就产生一个新的因子.
而且在每一步上,因子的个数总与剩余位置的个数相同;
于是,我们得出填满\(k\)个位置的方式种数等于\[
	\underbrace{n(n-1)(n-2)\dotsm[n-(k-1)]}_{k\ \text{个因子}}.
\]
这就是所求从\(n\)个元素一次取出\(k\)个元素的排列数,
我们可以利用阶乘记号将它表示为\[
	\frac{n!}{(n-k)!}.
\]

我们还可以推断出如下结论:
从\(n\)个元素一次取出全部\(n\)个元素的排列数为\(n!\).

以后我们用符号\(A_n^k\)表示从\(n\)个元素一次取出\(k\)个的排列数,即\begin{equation}
	A_n^k \defeq \frac{n!}{(n-k)!}.
\end{equation}
特别地,\(A_n^n \equiv n!\).

从\(n\)个元素一次取出\(k\)个元素的排列数也可用下面的思路求得.
按规定,我们用\(A_n^k\)代表从\(n\)个元素一次取出\(k\)个元素的排列数.
假设我们首先组成\(n\)个元素一次取出\(k-1\)个元素的所有排列,
那么这些排列一共有\(A_n^{k-1}\)种.
在这些排列的每一个的后面,我们放上剩下的\(n-(k-1)\)个元素中的任意一个;
每进行一次这样的操作,我们便得到\(n\)个元素一次取\(k\)个的一种排列,
因此,\(n\)个元素一次取\(k\)个的排列数为\(A_n^{k-1} \times (n-k+1)\)种,即\[
	A_n^k = A_n^{k-1} \times (n-k+1).
\]
在上式中用\(k-1\)代替\(k\),我们得\[
	A_n^{k-1} = A_n^{k-2} \times (n-k+2);
\]
依次递推,直到\[
	A_n^2 = A_n^1 \times (n-1),
\]\[
	A_n^1 = n.
\]
将上述各式的等号左右两边分别相乘,且约去两边相同的因子,便得\[
	A_n^k = n(n-1)\dotsm(n-k+2)(n-k+1).
\]

接下来我们想求从\(n\)个不同的元素中一次取出\(k\)的组合数.
我们用符号\(C_n^k\)表示所求组合数.
每一个这样的组合都由一组\(k\)个不同的元素组成,
而这组元素本身又可以组成\(k!\)个不同的排列.
因此\(C_n^k \times k!\)便等于\(n\)个元素一次取\(k\)个的排列数,即\[
	C_n^k \times k! \equiv A_n^k
	= n(n-1)(n-2)\dotsm(n-k+2)(n-k+1);
\]于是\begin{align}
	C_n^k &= \frac{n(n-1)(n-2)\dotsm(n-k+2)(n-k+1)}{k!} \\
	&= \frac{n!}{k! (n-k)!}.
\end{align}

需要注意到,当\(k=n\)时,有\[
	A_n^n = \frac{n!}{(n-n)!} = \frac{n!}{0!} = n!,
\]\[
	C_n^n = \frac{n!}{n! (n-n)!} = \frac{n!}{n! 0!} = \frac{1}{0!} = 1;
\]
这说明,符号\(0!\)的取值应该等于\(1\).
这就是为什么我们在\hyperref[definition:数列.阶乘的定义]{阶乘的定义}中特别规定\(0!\equiv1\).
%特别地,规定:当\(n < k\)时,\(C_n^k = 0\).

\begin{example}
将\(m+n\)颗小球分为两袋,一袋含\(m\)颗小球,另一袋含\(n\)颗小球,求可能的分法种数.
\begin{solution}
不难看出,这样的分法种数,等于从\(m+n\)颗小球中一次取\(m\)颗的组合数,
如此,可将取出的\(m\)颗小球装入一袋,将剩下的\((m+n)-m=n\)颗小球装入另一袋.
因此,所求的分法种数为\[
	\frac{(m+n)!}{m! n!}.
\]

特别地,当\(m=n\)时,两个袋子中所含小球颗数相同.
如果认为“两个袋子没有次序”或者说“袋子互换,分法种数不变”,那么分法种数为\[
	\frac{(2n)!}{(n!)^2 2!}.
\]
\end{solution}
\end{example}

\begin{example}
将\(m+n+p\)颗小球分为三袋,且各袋分别含\(m,n,p\)颗小球,求可能的分法种数.
\begin{solution}
首先将全部小球分到两个口袋里,各袋分别含\(m,n+p\)颗小球,分法种数为\[
	\frac{(m+n+p)!}{m!(n+p)!};
\]
然后将第二袋中的\(n+p\)颗小球再分为两袋,各袋分别含\(n,p\)颗小球,分法种数为\[
	\frac{(n+p)!}{n! p!};
\]
所以,全部\(m+n+p\)颗小球分成三袋,各袋分别含\(m,n,p\)颗小球的分法种数为\[
	\frac{(m+n+p)!}{m!(n+p)!} \cdot \frac{(n+p)!}{n! p!}
	= \frac{(m+n+p)!}{m! n! p!}.
\]

特别地,当\(m=n=p\)时,三个袋子中所含小球颗数相同.
如果认为“三个袋子有次序”或者说“袋子互换,就得到新的分法”,那么分法种类为\[
	\frac{(3n)!}{(n!)^3};
\]
反之,如果认为“三个袋子没有次序”,那么分法种类为\[
	\frac{(3n)!}{(n!)^3 3!}.
\]
\end{solution}
\end{example}

到目前为止,我们考虑的元素(如小球)常常被看作是不同的;
但有的时候,所给元素中有一部分元素是相同的.
相同的元素是无法区分的,不管怎么排布这些元素,都对排法种数没有影响.

\begin{example}
排列\(n\)颗小球,其中\(p\)颗是红球,\(q\)颗是黄球,\(r\)颗是蓝球,
剩余的\(n-p-q-r\)颗小球具有互不相同的彩色花纹,求可能的排列数.
\begin{solution}
当提到一组小球具有某种特征(如小球是红色的)时,
我们认为这组小球是相同的、无法区分的.
基于这条约定,我们来求解可能的排列数.

设所求排列数为\(x\).
如果用\(p\)颗花纹各异的小球代替上述\(p\)颗红球,
那么对于\(x\)个排列中的任一个,
不改变其他小球的位置,我们可以作\(p!\)个新排列;
于是如果对\(x\)个排列中的每一个,都作这样的替换,
我们就得到\(x \cdot p!\)种排列.
同理,如果在此基础上继续用\(q\)颗花纹各异的小球代替\(q\)颗黄球,会得到\(x \cdot p! \cdot q!\)种排列.
再用\(r\)颗花纹各异的小球代替\(r\)颗蓝球,会得到\(x \cdot p! \cdot q! \cdot r!\)种排列.
现在,这\(n\)颗小球全都互不相同了,它们的全排列数为\(n!\),于是有\[
	n! = x \cdot p! \cdot q! \cdot r!,
\]即有\[
	x = \frac{n!}{p! q! r!}.
\]
\end{solution}
\end{example}

\begin{example}
用数字\(1,2,3,4,3,2,1\)可以组成多少个七位数,且奇数总在奇数位上.
\begin{solution}
奇数\(1,3,3,1\)有\(\frac{4!}{2! 2!}\)种方式排列在它们的四个位置上;
偶数\(2,4,2\)有\(\frac{3!}{2!}\)种方式排列在它们的三个位置上;
奇数的每一种排列都能与偶数的每一种排列组合,因此,所求排列数为\[
	\frac{4!}{2! 2!} \times \frac{3!}{2!} = 18.
\]
\end{solution}
\end{example}

现在我们来求从\(n\)个元素中一次取出\(k\)个的排列数,
其中,取出的元素可以重复任意多次.
我们可以把这个问题考虑为这样的情况:
有\(n\)个不同的元素,去填满\(k\)个位置,并且每一个元素都可以任意次地重复使用.
这样的排列一共有多少种呢?
显而易见的是,填第一个位置有\(n\)种方式.
当第一个位置填好后,第二个位置仍有\(n\)种填法,
这是因为占据第一个位置的那个元素可以在第二个位置上重复使用.
因此,填满前两个位置的方式一共有\(n \times n = n^2\)种.
同理,填第三个位置还是有\(n\)种方式;
所以,填满前三个位置的方式一共有\(n^3\)种.
按这样的方法进行,我们注意到\(n\)的指数总与剩余位置的个数相同;
由此可知,一共有\(n^k\)种方式填满所给的\(k\)个位置.

\subsubsection{加法原理}
如果做一件事,完成它可以有\(n\)类办法,
在第一类办法中有\(m_1\)种不同的方法,
在第二类办法中有\(m_2\)种不同的方法,……,
在第\(n\)类办法中有\(m_n\)种不同的方法,那么完成这件事共有\[
N = m_1 + m_2 + \dotsb + m_n
\]种不同的方法.

%\begin{theorem}[抽屉原理]\label{theorem:排列组合.抽屉原理}
%抽屉原理有以下几种形式:
%\begin{enumerate}
%\item 把\(n+1\)个元素放入\(n\)个集合内,则一定有一个集合里有两个或两个以上的元素.
%\item 把\(m\)个元素任意放入\(n\ (n<m)\)个集合里,则一定有一个集合里至少有\(k\)个元素,其中\[
%k = \left\{ \begin{array}{ll}
%m/n, & m \pmod n = 0, \\
%\floor{m/n}+1, & m \pmod n \neq 0.
%\end{array} \right.
%\]
%\item 把无穷多个元素放入有限个集合里,则一定有一个集合里含有无穷多个元素.
%\end{enumerate}
%\end{theorem}
%\hyperref[theorem:排列组合.抽屉原理]{抽屉原理}有时候也称作\DefineConcept{鸽巢原理};
%因它最先是由狄利克雷明确地提出来的,因此也可称其为\DefineConcept{狄利克雷原理}.


\subsection{组合数的性质}
\begin{property}\label{theorem:组合数性质1}
\(C_n^k = C_n^{n-k}\).
\begin{proof}
\(
C_n^{n-k}
= \frac{n!}{(n-k)! [n-(n-k)]!}
= \frac{n!}{k! (n-k)!}
= C_n^k
\).
\end{proof}
\end{property}
这就是说,从\(n\)个元素中一次取出\(k\)个的组合数,
等于从\(n\)个元素中一次取出\(n-k\)个的组合数.
像这样的两类组合称为\textbf{互补}.

\cref{theorem:组合数性质1} 对于简化运算有很大用处.

\begin{property}\label{theorem:组合数性质2}
\(C_{n-1}^{k-1} + C_{n-1}^k = C_n^k\).
\begin{proof}
\begin{align*}
C_{n-1}^{k-1} + C_{n-1}^k
&= \frac{(n-1)!}{(k-1)! (n-k)!} + \frac{(n-1)!}{k! (n-k-1)!} \\
&= \frac{(n-1)!}{(k-1)! (n-k-1)!} \left( \frac{1}{n-k} + \frac{1}{k} \right) \\
&= \frac{(n-1)!}{(k-1)! (n-k-1)!} \frac{n}{(n-k)k}
= \frac{n!}{k! (n-k)!}
= C_n^k. \qedhere
\end{align*}
\end{proof}
\end{property}

\begin{property}\label{theorem:组合数性质3}
\(\sum\limits_{i=0}^n C_n^i = 2^n\).
\begin{proof}
当\(n=0\)时,\(\sum\limits_{i=0}^0 C_0^i = C_0^0 = \frac{0!}{0! \cdot 0!} = 1 = 2^0\)成立.
假设\(n=k\)时,结论仍成立,即\[
\sum\limits_{i=0}^k C_k^i
= C_k^0 + C_k^1 + \dotsb + C_k^k = 2^k.
\]那么\[
\begin{array}{*{14}{c}}
& C_k^0 &+& C_k^1 &+& \dotsb &+& C_k^{k-1} &+& C_k^k && &=& 2^k \\
+) & && C_k^0 &+& \dotsb &+& C_k^{k-2} &+& C_k^{k-1} &+& C_k^k &=& 2^k \\ \hline
& C_k^0 &+& C_{k+1}^1 &+& \dotsb &+& C_{k+1}^{k-1} &+& C_{k+1}^k &+& C_k^k &=& 2 \cdot 2^k
\end{array}
\]又因为\(C_k^0 = C_{k+1}^0 = C_k^k = C_{k+1}^{k+1} = 1\),所以\[
\sum\limits_{i=0}^{k+1} C_{k+1}^i = 2^{k+1}
\]成立.
\end{proof}
\end{property}

\begin{property}\label{theorem:组合数性质4}
\(\sum\limits_{k=0}^n (-1)^k C_n^k = 0\).
\end{property}

\begin{property}\label{theorem:组合数性质5}
\(\sum\limits_{k=0}^{\floor{n/2}} C_n^{2k} = 2^{n-1}\).
\end{property}
\begin{property}\label{theorem:组合数性质6}
\(\sum\limits_{k=0}^{\floor{(n-1)/2}} C_n^{2k+1} = 2^{n-1}\).
\end{property}

\begin{property}\label{theorem:组合数性质7}
\(C_{n+m}^k = \sum\limits_{r=0}^{k} C_n^r C_m^{k-r}\).
\end{property}

\begin{property}\label{theorem:组合数性质8}
\(\sum\limits_{k=0}^n (C_n^k)^2 = C_{2n}^n\).
\end{property}

\begin{property}\label{theorem:组合数性质9}
\(C_n^{r_1} C_{n-r_1}^{r_2} \dotsm C_{n-(r_1+r_2+\dotsb+r_{k-1})}^{r_k}
= \frac{n!}{r_1! r_2! \dotsm r_k!}\).
\end{property}

\section{常见代数公式}

\begin{theorem}[平方差、立方差公式]
\[
a^2 - b^2 = (a-b)(a+b),
\]\[
a^3 - b^3 = (a-b)(a^2+ab+b^2),
\]

推广一下可得,当\(n \in \mathbb{N}^+\)时,有\[
a^n - b^n = (a-b) \sum\limits_{k=0}^{n-1}{a^{n-1-k} b^k}
= (a-b)(a^{n-1} + a^{n-2} b + \dotsb + b^{n-1}).
\]
\end{theorem}

\begin{theorem}[完全平方公式、牛顿二项式定理]
\[
(a \pm b)^2 = a^2 \pm 2ab + b^2,
\]\[
(a \pm b)^3 = a^3 \pm 3 a^2 b + 3 a b^2 \pm b^3.
\]

推广一下可得,当\(n \in \mathbb{N}^+\)时,有\[
(x+y)^n = \sum_{k=0}^n C_n^k x^{n-k} y^k.
\]若令\(y=1\),则有\[
(1+x)^n = \sum_{k=0}^n C_n^k x^k.
\]
\end{theorem}

\section{初等代数方程}
\subsection{一元二次方程}
一元二次方程的\textbf{一般形式}为:\[
ax^2 + bx + c = 0, \quad a \neq 0. \eqno{(1)}
\]其中,\(ax^2\)是二次项,\(bx\)是一次项,\(c\)是常数项;
\(a\)、\(b\)、\(c\)被称作系数.

(1)式两端同除以\(a\),得\[
x^2 + \frac{b}{a} x + \frac{c}{a} = 0, \eqno{(2)}
\]配方,得\[
\left( x + \frac{b}{2a} \right)^2 + \left( \frac{c}{a} - \frac{b^2}{4a^2} \right) = 0,
\]移项,再开方,得\[
x = -\frac{b}{2a} \pm \sqrt{\frac{b^2}{4a^2} - \frac{c}{a}}
= -\frac{b}{2a} \pm \sqrt{\frac{b^2-4ac}{4a^2}}
= \frac{-b \pm \sqrt{b^2-4ac}}{2a}.
\]
于是我们得到一元二次方程\(ax^2 + bx + c = 0\ (a\neq0)\)的两个解\[
x_1 = \frac{-b + \sqrt{b^2-4ac}}{2a},
\qquad
x_2 = \frac{-b - \sqrt{b^2-4ac}}{2a}.
\]

记\(\Delta = b^2-4ac\),称之为方程(1)的\textbf{判别式}(discriminant).
当\(\Delta > 0\)时,它有两个不同的实根\[
x = \frac{-b \pm \sqrt{\Delta}}{2a};
\]当\(\Delta = 0\)时,它有两个相同的实根\[
x = -\frac{b}{2a};
\]当\(\Delta < 0\)时,它有一对共轭复根\[
x = \frac{-b \pm \iu \sqrt{-\Delta}}{2a}.
\]

\begin{theorem}[韦达定理]
设数\(x_1,x_2\)是一元二次方程{\rm(1)}的两个根,则有\[
x_1 + x_2 = -\frac{b}{a},
\qquad
x_1 \cdot x_2 = \frac{c}{a}.
\]
\begin{proof}
因为\(x_1,x_2\)是一元二次方程\(ax^2 + bx + c = 0\)的两个根,所以原方程可化为\[
a(x - x_1)(x - x_2) = 0
\quad\text{或}\quad
a x^2 - a (x_1 + x_2) x + a x_1 x_2 = 0.
\]将上式与方程(1)比较可得\[
b = -a (x_1 + x_2),
\qquad
c = a x_1 x_2.
\]整理得\(x_1 + x_2 = -\frac{b}{a}, x_1 \cdot x_2 = \frac{c}{a}\).
\end{proof}
\end{theorem}

\subsection{一元三次方程}
对于一般的一元三次方程\[
ax^3+bx^2+cx+d=0 \quad(a\neq0),
\]我们总可通过以下步骤将其化为标准形式.

首先在等号两边同除以\(a\),得\[
x^3+\frac{b}{a}x^2+\frac{c}{a}x+\frac{d}{a}=0,
\]再令\(x=y-\frac{b}{3a}\),得\[
\left(y-\frac{b}{3a}\right)^3+\frac{b}{a}\left(y-\frac{b}{3a}\right)^2+\frac{c}{a}\left(y-\frac{b}{3a}\right)+\frac{d}{a}=0,
\]整理得\[
y^3+py+q=0,
\]其中\[
p = \frac{3ac-b^2}{3a^2}, \qquad
q = \frac{2b^3}{27a^3}-\frac{bc}{3a^2}+\frac{d}{a}.
\]

\begin{theorem}[卡丹公式]
\def\a{-\frac{q}{2}}%
\def\d{\frac{q^2}{4}+\frac{p^3}{27}}
\def\b{\sqrt{\d}}%
\def\c#1{\sqrt[3]{\a#1\b}}%
形如\[
x^3 + px + q = 0 \quad (p,q \in \mathbb{C})
\]的一元三次方程的解为\[
x = \c{+}+\c{-}.
\]

令\[
\alpha=\c{+}, \qquad \beta=\c{-}.
\]总有\[
\alpha \beta = -\frac{p}{3}
\]成立.

当\(p,q\in\mathbb{R}\)时,判别式\[
\Delta = -108\left(\d\right) = -27q^2-4p^3
\]的正负号决定了\(x^3+px+q=0\)的根的性质:\begin{enumerate}
\item 当\(\Delta>0\)时,方程的三个根是各不相同的实根.
\item 当\(\Delta=0\)时,\begin{enumerate}
	\item 如果\(p=q=0\),则方程有三重实根;
	\item 如果\(p\neq0\)且\(q\neq0\),则方程有一个二重实根和一个与之不同的实根.
	\end{enumerate}
\item 当\(\Delta<0\)时,方程的三个根各不相同,其中一个是实根,两个是共轭复根.
\end{enumerate}
\end{theorem}
