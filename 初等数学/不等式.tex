\chapter{不等式}
\section{不等式的概念与性质}
\begin{definition}
设\(a,b\in\mathbb{R}\).
如果\(a-b\)是正数,则称\(a\) \DefineConcept{大于} \(b\),记作\(a>b\).
如果\(a-b\)是负数,则称\(a\) \DefineConcept{小于} \(b\),记作\(a<b\).
如果\(a-b\)是零,则称\(a\) \DefineConcept{等于} \(b\),记作\(a=b\).

如果\(a-b\)是非负数,则称\(a\) \DefineConcept{大于或等于} \(b\),记作\(a \geq b\).

如果\(a-b\)是非正数,则称\(a\) \DefineConcept{小于或等于} \(b\),记作\(a \leq b\).

如果\(a-b\)不是零,则称\(a\) \DefineConcept{不等于} \(b\),记作\(a \neq b\).
\end{definition}

\begin{property}
不等式具有以下性质:\begin{enumerate}
	\item {\bf 对称性} \(a>b \iff a<b\).

	\item {\bf 传递性} \begin{enumerate}
		\item \(a>b \land b>c \implies a>c\);
		\item \(a<b \land b<c \implies a<c\).
	\end{enumerate}
\end{enumerate}
\begin{proof}
\begin{enumerate}
\item 由于正数的相反数是负数,负数的相反数是正数,得\[
a > b \iff a-b > 0 \iff -(a-b) < 0 \iff b-a < 0 \iff b < a.
\]
\item 根据两个正数的和仍是正数,得\[
\left. \begin{array}{c}
a > b \iff a-b > 0 \\
b > c \iff b-c > 0
\end{array} \right\}
\implies (a-b)+(b-c) > 0
\implies a-c > 0
\implies a > c.
\]同理可得\(a<b \land b<c \implies a<c\).
\qedhere
\end{enumerate}
\end{proof}
\end{property}

\begin{theorem}
如果\(a>b\),那么\(a+c>b+c\).
\begin{proof}
显然有\[
a>b
\iff a-b>0
\iff (a+c)-(b+c)>0
\iff a+c>b+c.
\qedhere
\]
\end{proof}
\end{theorem}

\begin{corollary}
如果\(a+b>c\),那么\(a>c-b\).
\begin{proof}
显然有\[
a+b>c
\iff a+b+(-b)>c+(-b)
\iff a>c-b.
\qedhere
\]
\end{proof}
\end{corollary}
一般地说,不等式中任何一项的符号变成相反的符号后,应把它从一边移到另一边.

\begin{corollary}
如果\(a>b\)且\(c>d\),那么\(a+c>b+d\).
\begin{proof}
显然有\[
\left. \begin{array}{c}
a>b \iff a+c>b+c \\
c>d \iff b+c>b+d
\end{array} \right\}
\implies a+c>b+d.
\qedhere
\]
\end{proof}
\end{corollary}
这就是说,若干个同向不等式两边分别相加,所得不等式与原不等式同向.

\begin{example}
证明:如果\(a > b\)且\(c < d\),那么\(a - c > b - d\).
\begin{proof}
因为\(c < d\),所以\(-c > -d\).又因为\(a > b\),\(a + (-c) > b + (-d)\),所以\(a - c > b - d\).
\end{proof}
\end{example}

\begin{theorem}
设\(a>b\).如果\(c>0\),那么\(ac>bc\);如果\(c<0\),那么\(ac<bc\).
\begin{proof}
根据同号相乘得正,异号相乘得负,有\[
\left. \begin{array}{r}
a>b \iff a-b>0 \\
c>0
\end{array} \right\}
\implies (a-b)c>0
\iff ac-bc>0
\iff ac>bc;
\]同理有\[
\left. \begin{array}{r}
a>b \iff a-b> 0 \\
c<0
\end{array} \right\}
\implies (a-b)c<0
\iff ac-bc<0
\iff ac<bc.
\qedhere
\]
\end{proof}
\end{theorem}

\begin{corollary}
如果\(a>b>0\),\(c>d>0\),那么\(ac>bd>0\).
\begin{proof}
显然有\[
\left. \begin{array}{r}
a>b,c>0 \implies ac>bc \\
c>d,b>0 \implies bc>bd
\end{array} \right\}
\implies ac>bd.
\qedhere
\]
\end{proof}
\end{corollary}
这就是说,若干个两边都是正数的同向不等式两边分别相乘,所得不等式与原不等式同向.
由此,我们可以得到
\begin{corollary}
如果\(a>b>0\),那么\(a^n>b^n>0 \quad (n\in\mathbb{N}^+)\).
\end{corollary}

\begin{example}
证明:如果\(a > b > 0\)且\(c < d < 0\),那么\(ac < bd < 0\).
\begin{proof}
因为\(c < d < 0\),\(-c > -d > 0\),\(a(-c) > b(-d) > 0\),所以\(ac < bd < 0\).
\end{proof}
\end{example}

\begin{corollary}\label{corollary:不等式.正整数次幂的序}
设\(m,n\in\mathbb{N}^+\)且\(m>n\).
\begin{enumerate}
\item 当\(a>1\)时,\(a^m > a^n > 0\);
\item 当\(a=1\)时,\(a^m = a^n = 1\);
\item 当\(0<a<1\)时,\(0 < a^m < a^n < 1\);
\item 当\(a=0\)时,\(a^m = a^n = 0\).
\end{enumerate}
\begin{proof}
根据幂的定义,第2、4种情形是显然的.
现在来证第1种情形,因为\[
\left. \begin{array}{c}
a>1 \\
\Downarrow \\
a>0
\end{array} \right\}
\implies
a^2 = a \cdot a > 1 \cdot a = a
\implies
a^3 > a^2,
\]故以此类推,可得\[
\forall m,n\in\mathbb{N}^+ \bigl(
	a>1,m>n \implies a^m > a^n > 1
\bigr).
\]

再证第3种情形,因为\[
1>a>0
\implies
a = 1 \cdot a > a \cdot a = a^2
\implies
a^2 > a^3,
\]故以此类推,可得\[
\forall m,n\in\mathbb{N}^+ \bigl(
	0<a<1,m>n \implies 0 < a^m < a^n < 1
\bigr).
\qedhere
\]
\end{proof}
\end{corollary}

\begin{theorem}
如果\(a>b>0\),那么\(\sqrt[n]a > \sqrt[n]b \quad (n\in\mathbb{N}^+)\).
\begin{proof}
用反证法.
假设当\(a>b>0\)时,\(\sqrt[n]{a} \ngtr \sqrt[n]{b}\),
那么\[
	\sqrt[n]{a} < \sqrt[n]{b}
	\lor
	\sqrt[n]{a} = \sqrt[n]{b}
\]成立.
但是\[
\sqrt[n]{a} < \sqrt[n]{b} \implies a<b,
\]\[
\sqrt[n]{a} = \sqrt[n]{b} \implies a=b.
\]矛盾!
故\(\sqrt[n]{a}>\sqrt[n]{b}\)成立.
\end{proof}
\end{theorem}

\begin{example}
证明:如果\(a > b\)且\(ab > 0\),那么\(\frac{1}{a} < \frac{1}{b}\).
\begin{proof}
因为\(ab > 0\),\(\frac{1}{ab} > 0\),所以\(b \cdot \frac{1}{ab} < a \cdot \frac{1}{ab}\),\(\frac{1}{a} < \frac{1}{b}\).
\end{proof}
\end{example}

\begin{example}
证明:\(-\abs{a} \leq a \leq \abs{a}\).
\begin{proof}
我们可以按\(a\)的取值分为两种情况讨论:
\begin{itemize}
	\item 当\(a \geq 0\)时,\(\abs{a}=a\),
	原式化为\(-a \leq a \leq a\),成立.
	\item 当\(a < 0\)时,\(\abs{a}=-a\),
	原式化为\(a \leq a \leq -a\),成立.
	\qedhere
\end{itemize}
\end{proof}
\end{example}

\section{不等式的证明}
\subsection{作差比较法}
\begin{theorem}\label{theorem:不等式.作差比较法}
任给两个实数\(a\)和\(b\),有\begin{enumerate}
\item \(a - b > 0 \iff a > b\);
\item \(a - b = 0 \iff a = b\);
\item \(a - b < 0 \iff a < b\).
\end{enumerate}
\end{theorem}
\cref{theorem:不等式.作差比较法} 表述的不等式比较方法称为“作差比较法”.

\begin{example}\label{example:不等式.真分数的分子分母同加一个正数}
设\(b > a > 0\),\(m > 0\),证明:\(\frac{a}{b} < \frac{a+m}{b+m}\).
\begin{proof}
因为\[
\frac{a+m}{b+m} - \frac{a}{b}
= \frac{b(a+m) - a(b+m)}{b(b+m)}
= \frac{m(b-a)}{b(b+m)} > 0,
\]所以\[
\frac{a+m}{b+m} > \frac{a}{b}.
\qedhere
\]
\end{proof}
\end{example}
同理可证,若\(a > b > 0\),\(m > 0\),则\(\frac{a}{b} > \frac{a+m}{b+m}\).

\begin{example}\label{example:不等式.不同浓度的溶液的混合}
如果\(\frac{a_1}{b_1},\frac{a_2}{b_2},\dotsc,\frac{a_n}{b_n}\)是\(n\)个不相等的分数,且它们的分母的符号都相同,证明:分数\[
\frac{a_1+a_2+\dotsb+a_n}{b_1+b_2+\dotsb+b_n}
\]的值落在上述分数的最大值与最小值之间,即
\begin{equation}
\def\lu{\limits_{1 \leq k \leq n}}%
\min\lu\left\{ \frac{a_k}{b_k} \right\}
\leq
\frac{\sum\lu a_k}{\sum\lu b_k}
\leq
\max\lu\left\{ \frac{a_k}{b_k} \right\}.
\end{equation}
\begin{proof}
假设\(p=\frac{a_1}{b_1}<\frac{a_2}{b_2}<\dotsb<\frac{a_n}{b_n}=q\).
当\(b_1,b_2,\dotsc,b_n>0\)时,有\[
a_1 = p b_1,
a_2 > p b_2,
\dotsc,
a_n > p b_n,
\]相加得\[
a_1 + a_2 + \dotsb + a_n > p(b_1 + b_2 + \dotsb + b_n),
\]即\[
\frac{a_1+a_2+\dotsb+a_n}{b_1+b_2+\dotsb+b_n} > p = \frac{a_1}{b_1}.
\]同理可证\[
\frac{a_1+a_2+\dotsb+a_n}{b_1+b_2+\dotsb+b_n} < q = \frac{a_n}{b_n}.
\]

当\(b_1,b_2,\dotsc,b_n<0\)时,有相同的结论.
\end{proof}
\end{example}

\subsection{作商比较法}
\begin{theorem}\label{theorem:不等式.作商比较法}
任给两个实数\(a\)和\(b\),有\begin{enumerate}
\item \(\frac{a}{b} > 1 \iff ab > 0 \land \abs{a} > \abs{b}\);
\item \(\frac{a}{b} = 1 \iff a = b \neq 0\);
\item \(0 < \frac{a}{b} < 1 \iff ab > 0 \land \abs{a} < \abs{b}\);
\item \(\frac{a}{b} = 0 \iff a = 0 \land b \neq 0\);
\item \(-1 < \frac{a}{b} < 0 \iff ab < 0 \land \abs{a} < \abs{b}\);
\item \(\frac{a}{b} = -1 \iff a = -b \neq 0\);
\item \(\frac{a}{b} < -1 \iff ab < 0 \land \abs{a} > \abs{b}\).
\end{enumerate}
\end{theorem}
\cref{theorem:不等式.作商比较法} 表述的不等式比较方法称为“作商比较法”.

\section{常见不等式}
\begin{theorem}[三角不等式I]\label{theorem:不等式.三角不等式1}
设\(a,b\in\mathbb{R}\),那么\begin{equation}
	\abs{a \pm b}
	\leq
	\abs{a} + \abs{b}.
\end{equation}
当且仅当\(ab\geq0\)时,
有\(\abs{a+b}=\abs{a}+\abs{b}\).
当且仅当\(ab\leq0\)时,
有\(\abs{a-b}=\abs{a}+\abs{b}\).
\begin{proof}
对于任意实数\(a,b\),总有\[
	-\abs{a} \leq a \leq \abs{a}, \qquad
	-\abs{b} \leq \pm b \leq \abs{b}.
\]
将上述两式相加,得\[
	-(\abs{a}+\abs{b}) \leq a \pm b \leq \abs{a}+\abs{b},
\]即\(\abs{a \pm b} \leq \abs{a}+\abs{b}\).
\end{proof}
\end{theorem}

\begin{theorem}[三角不等式II]\label{theorem:不等式.三角不等式2}
设\(a,b\in\mathbb{R}\),那么\begin{equation}
	\abs{\abs{a}-\abs{b}}
	\leq
	\abs{a \pm b}.
\end{equation}
当且仅当\(ab\leq0\)时,
有\(\abs{\abs{a}-\abs{b}}=\abs{a+b}\).
当且仅当\(ab\geq0\)时,
有\(\abs{\abs{a}-\abs{b}}=\abs{a-b}\).
\end{theorem}

\begin{theorem}\label{theorem:不等式.基本不等式1}
如果\(a,b\in\mathbb{R}\),那么\(a^2 + b^2 \geq 2ab\)(当且仅当\(a=b\)时取“\(=\)”号).
\begin{proof}
因为\(a,b\in\mathbb{R}\),所以\(a-b\in\mathbb{R}\),\((a-b)^2 = a^2 - 2ab + b^2 \geq 0\),移项得\[
a^2 + b^2 \geq 2ab.
\qedhere
\]
\end{proof}
\end{theorem}

\begin{corollary}\label{corollary:不等式.基本不等式2}
如果\(a,b\in\mathbb{R}^+\),那么\(\frac{a+b}{2} \geq \sqrt{ab}\)(当且仅当\(a=b\)时取“\(=\)”号).
\begin{proof}
当\(a,b\in\mathbb{R}^+\)时,亦有\(\sqrt{a},\sqrt{b}\in\mathbb{R}^+\),
由\cref{theorem:不等式.基本不等式1} 有\[
a + b = \sqrt{a}^2 + \sqrt{b}^2 \geq 2\sqrt{a}\sqrt{b}.
\qedhere
\]
\end{proof}
\end{corollary}

\begin{theorem}\label{theorem:不等式.基本不等式3}
如果\(a,b,c\in\mathbb{R}^+\),那么\(a^3 + b^3 + c^3 \geq 3abc\)(当且仅当\(a=b=c\)时取“\(=\)”号).
\end{theorem}

\begin{corollary}\label{theorem:不等式.基本不等式4}
如果\(a,b,c\in\mathbb{R}^+\),那么\(\frac{a+b+c}{3} \geq \sqrt[3]{abc}\)(当且仅当\(a=b=c\)时取“\(=\)”号).
\end{corollary}

\begin{theorem}[均值不等式]\label{theorem:不等式.均值不等式}
如果\(\AutoTuple{x}{n}\in\mathbb{R}^+\),那么
\begin{equation}
n \left( \frac{1}{x_1} + \dotsb + \frac{1}{x_n} \right)^{-1}
\leq \sqrt[n]{x_1 \dotsm x_n}
\leq \frac{x_1 + \dotsb + x_n}{n}
\leq \sqrt{\frac{x_1^2 + \dotsb + x_n^2}{n}}.
\end{equation}
\rm
上述四个式子依次分别称为\DefineConcept{调和平均数}、\DefineConcept{几何平均数}、\DefineConcept{算术平均数}、\DefineConcept{平方平均数}.
\end{theorem}

\begin{corollary}\label{theorem:不等式.基本不等式6}
如果\(\AutoTuple{x}{n},\AutoTuple{p}{n}\in\mathbb{R}^+\),那么
\begin{equation}
x_1^{p_1} \dotsm x_n^{p_n}
\leq
\left( \frac{p_1 x_1 + \dotsb + p_n x_n}{p_1 + \dotsb + p_n} \right)^{p_1 + \dotsb + p_n}.
\end{equation}
\end{corollary}

\begin{corollary}[杨格不等式]\label{theorem:不等式.杨格不等式}
设\(a,b\geq0\),\(p,q>1\)且\(\frac{1}{p}+\frac{1}{q}=1\),则
\begin{equation}
ab \leq \frac{a^p}{p} + \frac{b^q}{q}.
\end{equation}
\end{corollary}

\begin{theorem}[伯努利不等式]\label{theorem:不等式.伯努利不等式}
\begin{equation}
(1+x_1)(1+x_2)\dotsm(1+x_n) \geq 1+x_1+x_2+\dotsb+x_n,
\end{equation}
式中\(\AutoTuple{x}{n}\)是符号相同且大于\(-1\)的数.
\begin{proof}
用数学归纳法.
当\(n=1\)时两边相等,故结论成立.现假设\(n=k\)时结论成立,即\((1+x_1)(1+x_2)\dotsm(1+x_k) \geq 1+x_1+x_2+\dotsb+x_k\).那么当\(n=k+1\)时,\begin{align*}
&(1+x_1)(1+x_2)\dotsm(1+x_k)(1+x_{k+1}) \\
&\geq (1+x_1+x_2+\dotsb+x_k)(1+x_{k+1}) \\
&= (1+x_1+x_2+\dotsb+x_k+x_{k+1}) + x_{k+1}(x_1+x_2+\dotsb+x_k).
\end{align*}
若\(x_1,x_2,\dotsc,x_{k+1}\)都大于零,则\(x_{k+1}(x_1+x_2+\dotsb+x_k) > 0\),\begin{gather}
(1+x_1)(1+x_2)\dotsm(1+x_k)(1+x_{k+1}) \geq 1+x_1+x_2+\dotsb+x_k+x_{k+1}
\tag1
\end{gather}成立;若\(x_1,x_2,\dotsc,x_{k+1}\)都等于零,则\(x_{k+1}(x_1+x_2+\dotsb+x_k) = 0\),同样有(1)式成立;若\(x_1,x_2,\dotsc,x_{k+1}\in(-1,0)\),则\(x_1+x_2+\dotsb+x_k < 0\),\(x_{k+1}(x_1+x_2+\dotsb+x_k) > 0\),仍然有(1)式成立.
\end{proof}
\end{theorem}

\begin{example}
证明:当\(x > -1\)时,不等式\begin{equation}
(1+x)^n \geq 1+nx \quad (n>1)
\end{equation}成立,当且仅当\(x=0\)时取“\(=\)”.
\begin{proof}
用数学归纳法.
当\(n=2\)时,\((1+x)^2 = 1+2x+x^2 \geq 1+2x\)成立.
假设当\(n=k\)时有\((1+x)^k \geq 1+kx\)成立.
当\(n=k+1\)时,\begin{align*}
(1+x)^{k+1}
&= (1+x)^k (1+x) \\
&\geq (1+kx)(1+x) \\
&= 1 + (k+1)x + kx^2 \\
&\geq 1 + (k+1)x.
\qedhere
\end{align*}
\end{proof}
实际上,当\(-2 \leq x \leq -1\)时,上述不等式仍然成立.

\begin{enumerate}
\item 当\(x = -1\)时,左边\((1+x)^n \equiv 0\),右边\(1+nx = 1 - n < 0\),上述不等式成立.

\item 当\(x = -2\)时,左边\((1+x)^n = (-1)^n = \pm1\),右边\[
1+nx = 1-2n < -1,
\]即上述不等式成立.

\item 当\(-2 < x < -1\)时,\(-1 < 1+x < 0\),故\(1+x < (1+x)^n\);
又因为\(n>1\),\(nx<x\),故\(1+nx < 1+x < (1+x)^n\).
\end{enumerate}
综上所述,不等式\[
(1+x)^n \geq 1+nx \quad (n>1)
\]当\(-2 \leq x \leq -1\)时仍然成立.
\end{example}

\begin{theorem}
设\(a,b,c,d\in\mathbb{R}\),则
\begin{equation}
	(ab+cd)^2
	\leq
	(a^2+b^2)(c^2+d^2).
\end{equation}
当且仅当\(ad=bc\)时,上式取“=”号.
\end{theorem}

\begin{theorem}[柯西不等式]\label{theorem:不等式.柯西不等式}
设\(\AutoTuple{a}{n},\AutoTuple{b}{n}\in\mathbb{R}\),则
\begin{equation}
	(a_1 b_1 + a_2 b_2 + \dotsb + a_n b_n)^2
	\leq
	(a_1^2 + a_2^2 + \dotsb + a_n^2) (b_1^2 + b_2^2 + \dotsb + b_n^2).
\end{equation}
\end{theorem}

\begin{example}
证明:\(C_n^k < n^k\).
\begin{proof}
显然有\begin{align*}
C_n^k &= \frac{n \cdot (n-1) \dotsm (n-k+1)}{k \cdot (k-1) \dotsm 1} \\
&< n \cdot (n-1) \dotsm (n-k+1) \\
&< n \cdot n \dotsm n = n^k.
\qedhere
\end{align*}
\end{proof}
\end{example}

\begin{theorem}[赫尔德不等式]\label{theorem:不等式.赫尔德不等式}
设\(\AutoTuple{x}{n},\AutoTuple{y}{n}\geq0\),\(p,q>1\),且满足\(\frac{1}{p}+\frac{1}{q}=1\),则
\def\s{\sum\limits_{i=1}^n}%
\def\sp#1#2#3{\left( \s #1^#2 \right)^{#3/#2}}%
\begin{equation}
\s x_i y_i
\leq
\sp{x_i}{p}{1} \sp{y_i}{q}{1}.
\end{equation}
\begin{proof}
令\[
a_j = x_j \sp{x_i}{p}{-1}, \qquad
b_j = y_j \sp{y_i}{q}{-1}.
\]那么根据\cref{theorem:不等式.杨格不等式} 得到\[
\s a_i b_i \leq \s \left( \frac{a_i^p}{p} + \frac{b_i}{q} \right)
\leq \frac{1}{p} + \frac{1}{q} = 1.
\qedhere
\]
\end{proof}
\end{theorem}

\begin{theorem}[闵可夫斯基不等式]\label{theorem:不等式.闵可夫斯基不等式}
设\(\AutoTuple{x}{n},\AutoTuple{y}{n}\in\mathbb{R}\),\(p\geq1\),则
\def\s{\sum\limits_{i=1}^n}%
\def\sumonly#1{\s \abs{#1}^p}%
\newcommand\sumpower[2][1]{\left( \sumonly{#2} \right)^{\frac{#1}{p}}}%
\begin{equation}
\sumpower{x_i+y_i} \leq \sumpower{x_i} + \sumpower{y_i}.
\end{equation}
\begin{proof}
利用\cref{theorem:不等式.赫尔德不等式} 就有\[\begin{aligned}
&\hspace{-20pt}\sumonly{x_i+y_i}
= \s \abs{x_i+y_i} \abs{x_i+y_i}^{p-1}
\leq \s (\abs{x_i}+\abs{y_i}) \abs{x_i+y_i}^{p-1} \\
&\leq \sumpower{x_i} \sumpower[p-1]{x_i+y_i}
+ \sumpower{y_i} \sumpower[p-1]{x_i+y_i}.
\end{aligned}\]
整理即得欲证不等式.
\end{proof}
\end{theorem}
