\section{函数的性质}
\subsection{函数的有界性}
\begin{definition}\label{definition:函数.函数的有界性}
设函数\(f(x)\)的定义域为\(D\),数集\(X \subseteq D\).

如果存在数\(K_1\),使得\(f(x) \leq K_1\)对任一\(x \in X\)都成立,即\[
	(\forall x \in X)
	(\exists K_1 \in \mathbb{R})[
		f(x) \leq K_1
	],
\]
则称“函数\(f(x)\)在\(X\)上有\DefineConcept{上界}”,
称“\(K_1\)是函数\(f(x)\)在\(X\)上的一个上界”.

如果存在数\(K_2\),使得\(f(x) \geq K_2\)对任一\(x \in X\)都成立,即\[
	(\forall x \in X)
	(\exists K_2 \in \mathbb{R})[
		f(x) \geq K_2
	],
\]
则称“函数\(f(x)\)在\(X\)上有\DefineConcept{下界}”,
称“\(K_2\)是函数\(f(x)\)在\(X\)上的一个下界”.

如果存在正数\(M\),使得\(\abs{f(x)} \leq M\)对任一\(x \in X\)都成立,即\[
	(\forall x \in X)
	(\exists M>0)[
		\abs{f(x)} \leq M
	],
\]
则称“函数\(f(x)\)在\(X\)上\DefineConcept{有界}”
或“函数\(f(x)\)是\(X\)上的\DefineConcept{有界函数}”.
反之如果这样的\(M\)不存在,即\[
	(\exists x_0 \in X)
	(\forall M>0)[
		\abs{f(x_0)} > M
	],
\]
就称“函数\(f(x)\)在\(X\)上\DefineConcept{无界}”.
\end{definition}

\begin{theorem}
设函数\(f(x)\)的定义域为\(D\),数集\(X \subseteq D\).
函数\(f(x)\)在\(X\)上有界的充要条件是它在\(X\)上既有上界又有下界.
\end{theorem}

\subsection{函数的单调性}
\begin{definition}
设函数\(f(x)\)的定义域为\(D\),区间\(I \subseteq D\).
如果对于区间\(I\)上任意两点\(x_1\)及\(x_2\),当\(x_1 < x_2\)时,恒有\[
	f(x_1) < f(x_2),
\]
则称“函数\(f(x)\)在区间\(I\)上是\DefineConcept{严格单调增加的}”;
如果对于区间\(I\)上任意两点\(x_1\)及\(x_2\),当\(x_1 < x_2\)时,恒有\[
	f(x_1) > f(x_2),
\]
则称“函数\(f(x)\)在区间\(I\)上是\DefineConcept{严格单调减少的}”.

单调增加的函数和单调减少的函数统称为\DefineConcept{单调函数}.
\end{definition}

\subsection{函数的奇偶性}
\begin{definition}
设函数\(f(x)\)定义域为\(D=(-l,l)\ (l>0)\).
若对于任一\(x \in D\)都有\[
	f(-x) = f(x)
\]恒成立,
则称“\(f(x)\)为\DefineConcept{偶函数}”;
若对于任一\(x \in D\)都有\[
	f(-x) = -f(x)
\]恒成立,
则称“\(f(x)\)为\DefineConcept{奇函数}”.
\end{definition}

\begin{property}
偶函数的图形是关于\(y\)轴对称的.
奇函数的图形是关于原点对称的.
\end{property}

\begin{property}
奇函数与奇函数之和、之差均为奇函数.
偶函数与偶函数之和、之差均为偶函数.
\begin{proof}
设\[
f(-x) = -f(x), \qquad g(-x) = -g(x).
\]令\(F(x) = f(x) \pm g(x)\),则\[
F(-x) = f(-x) \pm g(-x)
= [-f(x)] \pm [-g(x)]
= -[f(x) \pm g(x)]
= -F(x).
\qedhere
\]
\end{proof}
\end{property}

\begin{property}
奇函数与奇函数之积为偶函数.
\begin{proof}
设\[
f(-x) = -f(x), \qquad g(-x) = -g(x).
\]令\(F(x) = f(x) \cdot g(x)\),则\[
F(-x) = f(-x) \cdot g(-x)
= [-f(x)] \cdot [-g(x)]
= f(x) \cdot g(x)
= F(x).
\qedhere
\]
\end{proof}
\end{property}

\begin{property}
奇函数与偶函数之积为奇函数.
\begin{proof}
设\[
f(-x) = -f(x), \qquad g(-x) = g(x).
\]令\(F(x) = f(x) \cdot g(x)\),则\[
F(-x) = f(-x) \cdot g(-x)
= [-f(x)] \cdot g(x)
= - f(x) \cdot g(x)
= - F(x).
\qedhere
\]
\end{proof}
\end{property}

\begin{property}
偶函数与偶函数之积为偶函数.
\begin{proof}
设\[
f(-x) = f(x), \qquad g(-x) = g(x).
\]令\(F(x) = f(x) \cdot g(x)\),则\[
F(-x) = f(-x) \cdot g(-x) = f(x) \cdot g(x) = F(x).
\qedhere
\]
\end{proof}
\end{property}

\subsection{函数的周期性}
\begin{definition}
设函数\(f(x)\)的定义域为\(D\).
如果存在一个正数\(T\),使得对于任一\(x \in D\)有\((x + T) \in D\),且\[
	f(x+ T) = f(x)
\]恒成立,
则称“\(f(x)\)为\DefineConcept{周期函数}”,
称“\(T\)为\(f(x)\)的周期”.

已知周期为\(T\)的函数.
如果对于任意数\(a \in (0,T)\)都有\[
	f(x + a) \neq f(x),
\]
则称“\(T\)为\DefineConcept{最小正周期}”.
\end{definition}

\begin{example}
狄利克雷函数\[
	D(x) = \left\{ \begin{array}{ll}
		1, & x \in \mathbb{Q}, \\
		0, & x \in \mathbb{Q}^C.
	\end{array} \right.
\]是一个周期函数,任何正有理数\(r\)都是它的周期.
因为不存在最小的正有理数,所以它没有最小正周期.
\end{example}
