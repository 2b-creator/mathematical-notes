\chapter{不定积分}\label{chapter:不定积分}
\section{不定积分的概念与性质}
\subsection{原函数与不定积分的概念}
\begin{definition}
如果在区间\(I\)上,可导函数\(F(x)\)的导函数为\(f(x)\),即对任一\(x \in I\),都有\[
F'(x)=f(x) \quad\text{或}\quad \dd{F(x)}=f(x) \dd{x},
\]那么函数\(F(x)\)就称为\(f(x)\)(或\(f(x) \dd{x}\))在区间\(I\)上的\DefineConcept{原函数}.
\end{definition}

关于原函数,我们首先要问:一个函数具备什么条件,才能保证它的原函数一定存在?这个问题将在下一章中讨论,这里先介绍一个结论.
\begin{theorem}[原函数存在定理]%提前叙述该定理.参见\cref{theorem:定积分.原函数存在定理}
如果函数\(f(x)\)在区间\(I\)上连续,
那么在区间\(I\)上存在可导函数\(F(x)\),使对任一\(x \in I\)都有\[
	F'(x)=f(x).
\]
换言之,连续函数一定有原函数.
\end{theorem}
下面还要说明两点.

第一,如果\(f(x)\)在区间\(I\)上有原函数,即有一个函数\(F(x)\),使对任一\(x \in I\),都有\(F'(x) = f(x)\),那么,对任何常数\(C\),显然也有\[
[F(x) + C]' = f(x),
\]即对任何常数\(C\),函数\(F(x) + C\)也是\(f(x)\)的原函数.这说明,如果\(f(x)\)有一个原函数,那么\(f(x)\)就有无限多个原函数.

第二,如果在区间\(I\)上\(F(x)\)是\(f(x)\)的一个原函数,那么\(f(x)\)的其他原函数与\(F(x)\)有什么关系?

设\(\Phi(x)\)是\(f(x)\)的另一个原函数,即对任一\(x \in I\)有\[
\Phi'(x) = f(x),
\]于是\[
[\Phi(x) - F(x)]' = \Phi'(x) - F'(x) = f(x) - f(x) = 0.
\]在前面章节已经知道,在一个区间上导数恒为零的函数必为常数,所以\[
\Phi(x) - F(x) = C_0,
\]其中\(C_0\)是某个常数.这就表明\(\Phi(x)\)与\(F(x)\)只差一个常数.因此,当\(C\)为任意的常数时,表达式\(F(x) + C\)就可表示\(f(x)\)的任意一个原函数.也就是说,\(f(x)\)的全体原函数所组成的集合,就是函数族\[
\Set{ F(x) + C \given C \in (-\infty,\infty) }.
\]

\begin{definition}
在区间\(I\)上,函数\(f(x)\)的带有任意常数项的原函数称为\(f(x)\)(或\(f(x) \dd{x}\))在区间\(I\)上的\DefineConcept{不定积分},记作\[
\int f(x) \dd{x},
\]其中记号\(\int\)称为\DefineConcept{积分号},\(f(x)\)称为\DefineConcept{被积函数},\(f(x)\dd{x}\)称为\DefineConcept{被积表达式},\(x\)称为\DefineConcept{积分变量}.
\end{definition}
由此定义及前面的说明可知,如果\(F(x)\)是\(f(x)\)在区间\(I\)上的一个原函数,那么\(F(x) + C\)就是\(f(x)\)的不定积分,即\[
\int f(x) \dd{x} = F(x) + C.
\]
因而不定积分\(\int f(x) \dd{x}\)可以表示\(f(x)\)的任意一个原函数.


对初等函数来说,在其定义区间上,它的原函数一定存在,但原函数不一定都是初等函数,如\[
\int e^{-x^2} \dd{x}, \qquad
\int \frac{\sin x}{x} \dd{x}, \qquad
\int \frac{\dd{x}}{\ln{x}}, \qquad
\int \frac{\dd{x}}{\sqrt{1+x^4}}
\]
等等,它们的原函数就都不是初等函数.

\begin{example}
求\(\int \frac{1}{x} \dd{x}\).
\begin{solution}
当\(x > 0\)时,由于\((\ln x)' = \frac{1}{x}\),所以\(\ln x\)是\(\frac{1}{x}\)在区间\((0,+\infty)\)内的一个原函数.因此,在\((0,+\infty)\)内,\[
\int \frac{1}{x} \dd{x} = \ln x + C_1.
\]

当\(x < 0\)时,由于\([\ln(-x)]' = \frac{1}{-x} \cdot (-1) = \frac{1}{x}\),所以\(\ln(-x)\)是\(\frac{1}{x}\)在\((-\infty,0)\)内的一个原函数.因此,在\((-\infty,0)\)内,\[
\int \frac{1}{x} \dd{x} = \ln(-x) + C_2.
\]

把在\(x > 0\)及\(x < 0\)内的结果合起来,可写作\begin{equation}
\int \frac{1}{x} \dd{x} = \left\{ \begin{array}{lc}
\ln x + C_1, & x>0, \\
\ln(-x) + C_2, & x<0.
\end{array} \right.
\end{equation}
虽然常数\(C_1\)和\(C_2\)的取值可以是独立的,但在不严谨的情况下,为方便记忆,上式也可写作\begin{equation}
\int \frac{1}{x} \dd{x} = \ln\abs{x} + C.
\end{equation}
\end{solution}
\end{example}

\begin{definition}
函数\(f(x)\)的原函数的图形称为\(f(x)\)的积分曲线.
\end{definition}

从不定积分的定义,即可知以下关系:
\begin{align*}
	\text{\(\int f(x) \dd{x}\)是\(f(x)\)的原函数}
	&\iff
	\dv{x}\relax\left[ \int f(x) \dd{x} \right] = f(x) \\
	&\iff
	\dd\relax\left[\int f(x) \dd{x}\right] = f(x) \dd{x}. \\
	\text{\(F(x)\)是\(F'(x)\)的原函数}
	&\iff
	\int F'(x) \dd{x} = F(x) + C \\
	&\iff
	\int \dd{F(x)} = F(x) + C.
\end{align*}

由此可见,微分运算(以记号\(\dd\relax\)表示)
与求不定积分的运算(简称积分运算,以记号\(\int\)表示)是互逆的;
当记号\(\int\)与\(\dd\relax\)连在一起时,或者抵消,或者抵消后差一个常数.

\subsection{不定积分的性质}
\begin{property}
设函数\(f(x)\)及\(g(x)\)的原函数存在,则\[
\int \bigl[f(x) + g(x)\bigr] \dd{x} = \int f(x) \dd{x} + \int g(x) \dd{x}.
\]
\end{property}
这个性质对于有限个函数都是成立的.

\begin{example}
求\(\int \tan^2 x \dd{x}\).
\begin{solution}
根据三角恒等式\(\tan^2 x + 1 = \sec^2 x\),有\[
\int \tan^2 x \dd{x}
= \int (\sec^2 x - 1) \dd{x}
= \int \sec^2 x \dd{x} - \int \dd{x}
= \tan x - x + C.
\]
\end{solution}
\end{example}

\begin{property}
设函数\(f(x)\)的原函数存在,\(k\)为非零常数,则\[
\int k f(x) \dd{x} = k \int f(x) \dd{x}.
\]
\end{property}

\section{换元积分法}
利用基本积分表与积分的性质,我们可以计算的不定积分是非常有限的.
因此,有必要进一步来研究不定积分的求法.
本节把复合函数的微分法反过来用于求解不定积分,
利用中间变量的代换,得到复合函数的积分法,
称为\DefineConcept{换元积分法},简称\DefineConcept{换元法}.

\subsection{第一类换元法}
\begin{theorem}
设\(f(u)\)具有原函数,\(u=\phi(x)\)可导,则有换元公式\[
\int f\bigl[\phi(x)\bigr] \phi'(x) \dd{x} = \left[ \int f(u) \dd{u} \right]_{u=\phi(x)}.
\]
\begin{proof}
设\(f(u)\)的原函数是\(F(u)\),即\[
F'(u) = f(u),
\qquad
\int f(u) \dd{u} = F(u) + C.
\]因为\(u = \phi(x)\)可导,那么,根据复合函数微分法,有\[
\dd{F\bigl[\phi(x)\bigr]} = f\bigl[\phi(x)\bigr] \phi'(x) \dd{x},
\]从而根据不定积分的定义有\[
\int f\bigl[\phi(x)\bigr] \phi'(x) \dd{x}
= F\bigl[\phi(x)\bigr] + C
= \left[ \int f(u) \dd{u} \right]_{u=\phi(x)}.
\qedhere
\]
\end{proof}
\end{theorem}

\begin{example}
求\(\int \frac{1}{3+2x} \dd{x}\).
\begin{solution}
令\(u = 3+2x\),那么被积函数\(\frac{1}{3+2x} = \frac{1}{u}\).这里缺少\(\displaystyle\dv{u}{x}=2\)这样一个因子,但由于\(\displaystyle\dv{u}{x}\)是个常数,故可改变系数凑出这个因子:\[
\frac{1}{3+2x} = \frac{1}{2} \cdot \frac{1}{3+2x} \cdot 2 = \frac{1}{2} \cdot \frac{1}{3+2x} (3+2x)',
\]从而\begin{align*}
\int \frac{1}{3+2x} \dd{x}
&= \int \frac{1}{2} \cdot \frac{1}{3+2x} (3+2x)' \dd{x}
= \int \frac{1}{2} \frac{1}{u} \dd{u} \\
&= \frac{1}{2} \ln\abs{u} + C
= \frac{1}{2} \ln\abs{3+2x} + C.
\end{align*}
\end{solution}
\end{example}

一般地,对于积分\(\int f(ax+b) \dd{x}\),总可作变换\(u=ax+b\),把它化为\begin{align*}
\int f(ax+b) \dd{x}
&= \int \frac{1}{a} f(ax+b) \dd(ax+b) \\
&= \frac{1}{a} \left[ \int f(u) \dd{u} \right]_{u=ax+b}.
\end{align*}

\begin{example}
求\(\int 2x e^{x^2} \dd{x}\).
\begin{solution}
令\(u=x^2\),则被积函数\(2x e^{x^2} = e^{x^2} (x^2)' = e^u u'\),\[
\int 2x e^{x^2} \dd{x}
= \int e^u \dd{u}
= e^u + C
= e^{x^2} + C.
\]
\end{solution}
\end{example}

\begin{example}
求\(\int x \sqrt{1-x^2} \dd{x}\).
\begin{solution}
令\(u=1-x^2\),
则\(\dd{u} = -2x\dd{x}\),
\(-\frac{1}{2}\dd{u} = x\dd{x}\),
因此\begin{align*}
	\int x \sqrt{1-x^2} \dd{x}
	&= \int u^{\frac{1}{2}} \left(-\frac{1}{2}\right) \dd{u}
	= -\frac{1}{2} \frac{u^{\frac{3}{2}}}{\frac{3}{2}} + C \\
	&= -\frac{1}{3} u^{\frac{3}{2}} + C
	= -\frac{1}{3} (1-x^2)^{\frac{3}{2}} + C.
\end{align*}
\end{solution}
\end{example}

\begin{example}
求\(\int \frac{1}{a^2+x^2} \dd{x}\).
\begin{solution}
\begin{align*}
	\int \frac{1}{a^2+x^2} \dd{x}
	&= \int \frac{1}{a^2} \cdot \frac{1}{1+\left(\frac{x}{a}\right)^2} \dd{x} \\
	&= \frac{1}{a} \int \frac{1}{1+\left(\frac{x}{a}\right)^2} \dd(\frac{x}{a})
	= \frac{1}{a} \arctan\frac{x}{a} + C.
\end{align*}
\end{solution}
\end{example}

\begin{example}
求\(\int \frac{\dd{x}}{\sqrt{a^2-x^2}}\ (a>0)\).
\begin{solution}
\[
	\int \frac{\dd{x}}{\sqrt{a^2-x^2}}
	= \int \frac{1}{a} \frac{\dd{x}}{\sqrt{1-\left(\frac{x}{a}\right)^2}}
	= \int \frac{\dd(\frac{x}{a})}{\sqrt{1-\left(\frac{x}{a}\right)^2}}
	= \arcsin\frac{x}{a} + C.
\]
\end{solution}
\end{example}

\begin{example}
求\(\int \frac{1}{x^2 - a^2} \dd{x}\).
\begin{solution}
因为\[
	\frac{1}{x^2 - a^2}
	= \frac{1}{2a} \left(\frac{1}{x-a} - \frac{1}{x+a}\right),
\]
所以\begin{align*}
	\int \frac{1}{x^2 - a^2} \dd{x}
	&= \frac{1}{2a} \int \left(\frac{1}{x-a} - \frac{1}{x+a}\right) \dd{x} \\
	&= \frac{1}{2a} \left[ \int \frac{\dd(x-a)}{x-a} - \int \frac{\dd(x+a)}{x+a} \right] \\
	&= \frac{1}{2a} ( \ln\abs{x-a} - \ln\abs{x+a} ) + C \\
	&= \frac{1}{2a} \ln\abs{\frac{x-a}{x+a}} + C.
\end{align*}
\end{solution}
\end{example}

\begin{example}
求\(\int \sin mx \cos nx \dd{x}\ (m \neq n)\).
\begin{solution}
由\cref{equation:函数.三角函数.和积互化公式7},
\[
	\sin mx \cos nx
	= \frac{1}{2} [\sin(m+n)x + \sin(m-n)x],
\]
于是\begin{align*}
	\int \sin mx \cos nx \dd{x}
	&= \frac{1}{2} \left[
		\int \sin(m+n)x \dd{x}
		+ \int \sin(m-n)x \dd{x}
	\right] \\
	&= -\frac{\cos(m+n)x}{2(m+n)}
		- \frac{\cos(m-n)x}{2(m-n)}
		+ C.
\end{align*}
\end{solution}
\end{example}

\begin{example}
求\(\int \sin mx \sin nx \dd{x}\ (m \neq n)\).
\begin{solution}
由\cref{equation:函数.三角函数.和积互化公式10},
\[
	\sin mx \sin nx
	= -\frac12 [\cos(m+n)x - \cos(m-n)x],
\]
于是\begin{align*}
	\int \sin mx \sin nx \dd{x}
	&= -\frac12 \left[
		\int \cos(m+n)x \dd{x}
		- \int \cos(m-n)x \dd{x}
	\right] \\
	&= \frac{\sin(m-n)x}{2(m-n)}
		- \frac{\sin(m+n)x}{2(m+n)}
		+ C.
\end{align*}
\end{solution}
\end{example}

\begin{example}
求\(\int \cos mx \cos nx \dd{x}\ (m \neq n)\).
\begin{solution}
由\cref{equation:函数.三角函数.和积互化公式9},
\[
	\cos mx \cos nx
	= \frac12 [\cos(m+n)x + \cos(m-n)x],
\]
于是\begin{align*}
	\int \cos mx \cos nx \dd{x}
	&= \frac12 \left[
		\int \cos(m+n)x \dd{x}
		+ \int \cos(m-n)x \dd{x}
	\right] \\
	&= \frac{\sin(m+n)x}{2(m+n)}
		+ \frac{\sin(m-n)x}{2(m-n)}
		+ C.
\end{align*}
\end{solution}
\end{example}

\begin{example}
求\(\int \sin^3 x \dd{x}\).
\begin{solution}
\(\begin{aligned}[t]
	\int \sin^3 x \dd{x}
	&= \int \sin^2 x \cdot \sin x \dd{x}
	= -\int (1 - \cos^2 x) \dd(\cos x) \\
	&= -\cos x + \frac{1}{3} \cos^3 x + C.
\end{aligned}\)
\end{solution}
\end{example}

一般地,对于\(\sin^{2k+1} x \cos^n x\)或\(\sin^n x \cos^{2k+1} x\)
(其中\(k\in\mathbb{N}\))型函数的积分,
总可依次作变换\(u = \cos x\)或\(u = \sin x\),求得结果.

\begin{example}
求\(\int \tan x \dd{x}\).
\begin{solution}
\(\begin{aligned}[t]
	\int \tan x \dd{x}
	&= \int \frac{\sin x}{\cos x} \dd{x}
	= - \int \frac{1}{\cos x} \dd(\cos x) \\
	&= - \ln\abs{\cos x} + C.
\end{aligned}\)
\end{solution}
\end{example}

\begin{example}
求\(\int \cos^2 x \dd{x}\).
\begin{solution}
直接计算得
\begin{align*}
	\int \cos^2 x \dd{x}
	&= \int \frac{1 + \cos 2x}{2} \dd{x}
	= \frac{1}{2} \left( \int \dd{x} + \int \cos 2x \dd{x} \right) \\
	&= \frac{1}{2} \int \dd{x} + \frac{1}{4} \int \cos 2x \dd(2x) \\
	&= \frac{x}{2} + \frac{\sin 2x}{4} + C.
\end{align*}
\end{solution}
\end{example}

\begin{example}
求\(\int \sin^2 x \cos^4 x \dd{x}\).
\begin{solution}
\(\begin{aligned}[t]
	\int \sin^2 x \cos^4 x \dd{x}
	&= \frac{1}{8} \int (1 - \cos 2x) (1 + \cos 2x)^2 \dd{x} \\
	&= \frac{1}{8} \int (1 + \cos 2x - \cos^2 2x - \cos^3 2x) \dd{x} \\
	&= \frac{1}{8} \int (\cos 2x - \cos^3 2x) \dd{x}
		+ \frac{1}{8} \int (1 - \cos^2 2x) \dd{x} \\
	&= \frac{1}{8} \int \sin^2 2x \frac{1}{2} \dd(\sin 2x)
		+ \frac{1}{8} \int \frac{1}{2} (1 - \cos 4x) \dd{x} \\
	&= \frac{1}{48} \sin^3 2x + \frac{1}{16} x - \frac{1}{64} \sin 4x + C.
\end{aligned}\)
\end{solution}
\end{example}

一般地,对于\(\sin^{2k} x \cos^{2l} x\ (k,l\in\mathbb{N})\)型函数,
总可利用三角恒等式\(\sin^2 x = \frac{1}{2} (1 - \cos 2x)\),
\(\cos^2 x = \frac{1}{2} (1 + \cos 2x)\)把积分函数化为\(\cos 2x\)的多项式.

\subsection{第二类换元法}

\begin{theorem}
设\(x = \psi(t)\)是单调的、可导的函数,
并且\(\psi'(t) \neq 0\).又设\(f[\psi(t)] \psi'(t)\)具有原函数,
则有换元公式\[
	\int f(x) \dd{x} = \left[ \int f[\psi(t)] \psi'(t) \dd{t} \right]_{t=\psi^{-1}(x)},
\]
其中\(t=\psi^{-1}(x)\)是\(x=\psi(t)\)的反函数.
\begin{proof}
设\(f[\psi(t)] \psi'(t)\)的原函数为\(\Phi(t)\),
记\(\Phi[\psi^{-1}(x)] = F(x)\),
利用复合函数及反函数的求导法则,
得到\[
	F'(x) = \dv{\Phi}{t} \cdot \dv{t}{x}
	= f[\psi(t)] \psi'(t) \cdot \frac{1}{\psi'(t)}
	= f[\psi(t)] = f(x),
\]
即\(F(x)\)是\(f(x)\)的原函数,
所以有\begin{align*}
	\int f(x) \dd{x} &= F(x) + C
	= \Phi[\psi^{-1}(x)] + C \\
	&= \left[ \int f[\psi(t)] \psi'(t) \dd{t} \right]_{t=\psi^{-1}(x)}.
	\qedhere
\end{align*}
\end{proof}
\end{theorem}

\begin{example}
求\(\int \sqrt{a^2 - x^2} \dd{x}\).
\begin{solution}
设\(x = a \sin t\ (-\frac{\pi}{2} < x < \frac{\pi}{2})\),
那么\(t = \arcsin\frac{x}{a}\),
\(\sqrt{a^2 - x^2} = \sqrt{a^2 - a^2 \sin^2 t} = a \cos t\),
\(\dd{x} = a \cos t \dd{t}\),
于是根式化成了三角式,
所求积分化为\begin{align*}
	\int \sqrt{a^2 - x^2} \dd{x}
	&= \int a \cos t \cdot a \cos t \dd{t}
	= a^2 \int \cos^2 t \dd{t} \\
	&= a^2 \left( \frac{t}{2} + \frac{\sin 2t}{4} \right) + C \\
	&= \frac{1}{2} a^2 t + \frac{1}{2} a^2 \sin t \cos t + C.
\end{align*}
又因为\(\cos t = \sqrt{1 - \sin^2 t}
= \sqrt{1 - \left(\frac{x}{a}\right)^2}
= \frac{1}{a} \sqrt{a^2 - x^2}\),
于是所求积分为\[
	\int \sqrt{a^2 - x^2} \dd{x}
	= \frac{1}{2} a^2 \arcsin\frac{x}{a} + \frac{1}{2} x \sqrt{a^2 - x^2} + C.
\]
\end{solution}
\end{example}

\begin{example}
求\(\int \frac{\dd{x}}{\sqrt{x^2 + a^2}}\ (a>0)\).
\begin{solution}
设\(x = a \tan t\ (-\frac{\pi}{2} < t < \frac{\pi}{2})\),
那么\[
	\sqrt{x^2 + a^2}
	= \sqrt{a^2 + a^2 \tan^2 t}
	= a \sqrt{1 + \tan^2 t}
	= a \sec t,
\]\[
	\dd{x} = a \sec^2 t \dd{t},
\]
于是\[
	\int \frac{\dd{x}}{\sqrt{x^2 + a^2}}
	= \int \frac{a \sec^2 t}{a \sec t} \dd{t}
	= \int \sec t \dd{t}
	= \ln\abs{\sec t + \tan t} + C.
\]

因为\(\sec t = \frac{1}{a} \sqrt{x^2 + a^2}\),
且\(\sec t + \tan t > 0\),
所以\[
	\int \frac{\dd{x}}{\sqrt{x^2 + a^2}}
	= \ln( \frac{x}{a} + \frac{\sqrt{x^2 + a^2}}{a} ) + C
	= \ln(x + \sqrt{x^2 + a^2}) + C_1,
\]
其中\(C_1 = C - \ln a\).
\end{solution}
\end{example}

\begin{example}
求\(\int \frac{\dd{x}}{\sqrt{x^2 - a^2}}\ (a>0)\).
\begin{solution}
首先要注意被积函数的定义域是\((-\infty,-a)\cup(a,+\infty)\),
故须分别在这两个区间求解不定积分.

当\(x > a\)时,
设\(x = a \sec t\ (0 < t < \frac{\pi}{2})\),
那么\[
	\sqrt{x^2 - a^2} = \sqrt{a^2 \sec^2 t - a^2} = a \sqrt{\sec^2 t - 1} = a \tan t,
\]\[
	\dd{x} = a \sec t \tan t \dd{t},
\]
于是\begin{align*}
	\int \frac{\dd{x}}{\sqrt{x^2 - a^2}}
	&= \int \frac{a \sec t \tan t \dd{t}}{a \tan t}
	= \int \sec t \dd{t} \\
	&= \ln(\sec t + \tan t) + C.
\end{align*}
又因为\(\tan t = \frac{1}{a} \sqrt{x^2 - a^2}\),
所以\[
	\int \frac{\dd{x}}{\sqrt{x^2 - a^2}}
	= \ln( \frac{x}{a} + \frac{\sqrt{x^2 - a^2}}{a} ) + C
	= \ln( x + \sqrt{x^2 - a^2} ) + C_1,
\]
其中\(C_1 = C - \ln a\).

当\(x < -a\)时,
令\(x = -u\),
那么\(u > a\),
由上可知\begin{align*}
	\int \frac{\dd{x}}{\sqrt{x^2 - a^2}}
	&= -\int \frac{\dd{u}}{\sqrt{u^2 - a^2}}
	= -\ln(u + \sqrt{u^2 - a^2}) + C \\
	&= -\ln(-x + \sqrt{x^2 - a^2}) + C \\
	&= \ln\frac{1}{-x + \sqrt{x^2 - a^2}} + C \\
	&= \ln\frac{-x - \sqrt{x^2 - a^2}}{(-x)^2 - \sqrt{x^2 - a^2}^2} + C \\
	&= \ln\frac{-x - \sqrt{x^2 - a^2}}{a^2} + C \\
	&= \ln(-x - \sqrt{x^2 - a^2}) + C_1,
\end{align*}
其中\(C_1 = C - 2 \ln a\).

把\(x > a\)和\(x < -a\)的结果合起来,
可写作\[
	\int \frac{\dd{x}}{\sqrt{x^2 - a^2}}
	= \ln\abs{x + \sqrt{x^2 - a^2}} + C.
\]
\end{solution}
\end{example}

\begin{example}
求\(\int \frac{\sqrt{a^2 - x^2}}{x^4} \dd{x}\).
\begin{solution}
设\(x = \frac{1}{t}\),
那么\(\dd{x} = -\frac{\dd{t}}{t^2}\),
于是\[
	\int \frac{\sqrt{a^2 - x^2}}{x^4} \dd{x}
	= \int \frac{\sqrt{a^2 - \frac{1}{t^2}} \cdot \left( -\frac{\dd{t}}{t^2} \right)}{\frac{1}{t^4}}
	= -\int (a^2 t^2 - 1)^{1/2} \abs{t} \dd{t}.
\]
当\(x > 0\)时,\(t > 0\),
那么\begin{align*}
	\int \frac{\sqrt{a^2 - x^2}}{x^4} \dd{x}
	&= -\frac{1}{2a^2} \int (a^2 t^2 - 1)^{1/2} \dd(a^2 t^2 - 1) \\
	&= -\frac{(a^2 t^2 - 1)^{3/2}}{3 a^2} + C \\
	&= -\frac{(a^2 - x^2)^{3/2}}{3 a^2 x^3} + C,
\end{align*}
同理,当\(x < 0\)时,有相同的结果.
\end{solution}
\end{example}

\section{分部积分法}
\begin{theorem}[分部积分公式]
设函数\(u=u(x)\)及\(v=v(x)\)具有连续导数,那么\[
	\int u \dd{v} = uv - \int u' \dd{v}.
\]
\begin{proof}
因为函数\(u=u(x)\)及\(v=v(x)\)具有连续导数,
那么两个函数乘积的导数公式为\[
	(uv)' = u'v + uv',
\]
移项,得\[
	uv' = (uv)' - u'v.
\]

对这个等式两边求不定积分,得\[
	\int u v' \dd{x} = \int (uv)' \dd{x} - \int u' v \dd{x}
	= uv - \int u' v \dd{x}.
	\qedhere
\]
\end{proof}
\end{theorem}
如果直接求\(\int u v' \dd{x}\)有困难,
而求\(\int u' v \dd{x}\)时比较容易时,
分部积分公式就可以发挥作用了.

\begin{example}
求\(\int x \cos x \dd{x}\).
\begin{solution}
设\(u = x, \dd{v} = \cos x \dd{x}\),
那么\(\dd{u} = \dd{x}, v = \sin x\),得\[
	\int x \cos x \dd{x}
	= x \sin x - \int \sin x \dd{x},
\]
而\(\int v \dd{u} = \int \sin x \dd{x}\)容易积出,所以\[
	\int x \cos x \dd{x}
	= x \sin x + \cos x + C.
\]

求这个积分的时候,如果设\(u = \cos x, \dd{v} = x \dd{x}\),那么\[
\dd{u} = -\sin x \dd{x}, \qquad v = \frac{x^2}{2}.
\]于是\[
\int x \cos x \dd{x} = \frac{x^2}{2} \cos x + \int \frac{x^2}{2} \sin x \dd{x}.
\]上式右端的积分比原积分更不容易求出.
\end{solution}
\end{example}
由此可见,如果\(u\)和\(\dd{v}\)选取不当,就求不出结果,所以应用分部积分法时,恰当选取\(u\)和\(\dd{v}\)是一个关键.
选取\(u\)和\(\dd{v}\)一般要考虑下面两点:\begin{enumerate}
\item \(v\)要容易求得;
\item \(\int v \dd{u}\)要比\(\int u \dd{v}\)容易积出.
\end{enumerate}

\begin{example}
求\(\int x e^x \dd{x}\).
\begin{solution}
设\(u = x\),\(\dd{v} = e^x \dd{x}\),那么\(\dd{u} = \dd{x}\),\(v = e^x\),于是\[
\int x e^x \dd{x}
= \int x \dd(e^x)
= x e^x - \int e^x \dd{x}
= x e^x - e^x + C
= e^x (x - 1) + C.
\]
\end{solution}
\end{example}

\begin{example}
求\(\int x^2 e^x \dd{x}\).
\begin{solution}
设\(u = x^2\),\(\dd{v} = e^x \dd{x}\),那么\[
\int x^2 e^x \dd{x}
= \int x^2 \dd(e^x)
= x^2 e^x - \int e^x \dd{x^2}
= x^2 e^x - 2 \int x e^x \dd{x}.
\]

这里\(\int x e^x \dd{x}\)比\(\int x^2 e^x \dd{x}\)更容易积出,因为被积函数中\(x\)的幂次前者比后者降低了一次.由上例可知,对\(\int x e^x \dd{x}\)再使用一次分部积分就可以了,于是\[
\int x^2 e^x \dd{x} = e^x (x^2 -2x + 2) + C.
\]
\end{solution}
\end{example}

\begin{example}
求\(\int x \ln x \dd{x}\).
\begin{solution}
设\(u=\ln x\),\(\dd{v} = x \dd{x}\),那么\begin{align*}
\int x \ln x \dd{x}
&= \int \ln x \dd(\frac{x^2}{2})
= \frac{x^2}{2} \ln x - \int \frac{x^2}{2} \dd(\ln x) \\
&= \frac{x^2}{2} \ln x - \frac{1}{2} \int x \dd{x}
= \frac{x^2}{2} \ln x - \frac{x^2}{4} + C.
\end{align*}
\end{solution}
\end{example}

\begin{example}
求\(\int \arccos x \dd{x}\).
\begin{solution}
设\(u = \arccos x\),\(\dd{v} = \dd{x}\),那么\[
\int \arccos x \dd{x} = x \arccos x - \int x \dd(\arccos x),
\]其中\begin{align*}
\int x \dd(\arccos x)
&= -\int \frac{x}{\sqrt{1-x^2}} \dd{x}
= \frac{1}{2} \int \frac{\dd(1-x^2)}{(1-x^2)^{1/2}} \\
&= \sqrt{1-x^2} + C,
\end{align*}所以\[
\int \arccos x \dd{x} = x \arccos x - \sqrt{1-x^2} + C.
\]
\end{solution}
\end{example}

\begin{example}
求\(\int x \arctan x \dd{x}\).
\begin{solution}
设\(u = \arctan x\),\(\dd{v} = x \dd{x}\),那么\[
\int x \arctan x \dd{x}
= \frac{1}{2} \int \arctan x \dd(x^2)
= \frac{1}{2} \left( x^2 \arctan x
	- \int \frac{x^2}{1+x^2} \dd{x} \right),
\]其中\[
\int \frac{x^2}{1+x^2} \dd{x}
= \int \left(1-\frac{1}{1+x^2}\right) \dd{x}
= x - \arctan x + C,
\]所以\begin{align*}
\int x \arctan x \dd{x}
&= \frac{1}{2} \left[ x^2 \arctan x
	- (x - \arctan x + C) \right] \\
&= \frac{1}{2} (x^2+1) \arctan x - \frac{1}{2} x + C_1.
\end{align*}
\end{solution}
\end{example}

\begin{example}
计算\(I_1 = \int e^x \cos x\dd{x}\)和\(I_2 = \int e^x \sin x\dd{x}\).
\begin{solution}
因为\begin{align*}
I_1 &= \int e^x \cos x\dd{x}
= \int e^x \dd(\sin x) \\
&= e^x \sin x - \int \sin x \dd(e^x)
= e^x \sin x - I_2, \\
I_2 &= \int e^x \sin x\dd{x}
= -\int e^x \dd(\cos x) \\
&= -\left[ e^x \cos x - \int \cos x \dd(e^x) \right]
= I_1 - e^x \cos x.
\end{align*}解得\[
I_1 = \frac{1}{2} e^x (\sin x + \cos x) + C,
\qquad
I_2 = \frac{1}{2} e^x (\sin x - \cos x) + C.
\]
\end{solution}
\end{example}

\begin{example}
求\(\int \sec^3 x \dd{x}\).
\begin{solution}
由题有\begin{align*}
\int \sec^3 x \dd{x}
&= \int \sec x \dd(\tan x) \\
&= \sec x \tan x - \int \sec x \tan^2 x \dd{x} \\
&= \sec x \tan x - \int \sec x (\sec^2 x - 1) \dd{x} \\
&= \sec x \tan x - \int \sec^3 x \dd{x} + \int \sec x \dd{x} \\
&= \sec x \tan x + \ln\abs{\sec x + \tan x} - \int \sec^3 x \dd{x},
\end{align*}解得\[
\int \sec^3 x \dd{x}
= \frac{1}{2} \left(
	\sec x \tan x
	+ \ln\abs{\sec x + \tan x}
\right) + C.
\]
\end{solution}
\end{example}

\begin{example}
求\(\int e^{\sqrt{x}} \dd{x}\).
\begin{solution}
令\(t = \sqrt{x}\),则\(x = t^2\),\(\dd{x} = 2t\dd{t}\),于是\[
\int e^{\sqrt{x}} \dd{x}
= 2 \int t e^t \dd{t}
= 2 e^t (t-1) + C
= 2 e^{\sqrt{x}} (\sqrt{x}-1) + C.
\]
\end{solution}
\end{example}

\begin{example}
计算\(I_1 = \int \sin{\ln{x}} \dd{x}\)和\(I_2 = \int \cos{\ln{x}} \dd{x}\).
\begin{solution}
\begin{align*}
I_1
&= \int \sin{\ln x}\dd{x}
\xlongequal{u = \ln x} \int \sin u \dd(e^u) \\
&= \frac{1}{2} e^u (\sin u - \cos u) + C
= \frac{1}{2} x (\sin{\ln x} - \cos{\ln x}) + C, \\
I_2
&= \int \cos{\ln x}\dd{x}
\xlongequal{u = \ln x} \int \cos u \dd(e^u) \\
&= \frac{1}{2} e^u (\sin u + \cos u) + C
= \frac{1}{2} x (\sin{\ln x} + \cos{\ln x}) + C.
\end{align*}
\end{solution}
\end{example}

\begin{example}
求\(\displaystyle\int \frac{x \cos x}{(x + \cos x)^2}\dd{x}\).
\begin{solution}
因为\begin{align*}
\int \frac{x \cos x}{(x + \cos x)^2}\dd{x}
&= \int \frac{x \cos x}{(x + \cos x)^2} \frac{1 - \sin x}{1 - \sin x}\dd{x} \\
&= \int \frac{x \cos x}{(x + \cos x)^2} \frac{1}{1 - \sin x} \dd(x + \cos x) \\
&= -\int \frac{x \cos x}{1 - \sin x} \dd(\frac{1}{x + \cos x}) \\
&= -\frac{x \cos x}{1 - \sin x} \frac{1}{x + \cos x}
	+\int \frac{1}{x + \cos x} \dd(\frac{x \cos x}{1 - \sin x}),
\end{align*}
其中
\begin{align*}
&\int \frac{1}{x + \cos x} \dd(\frac{x \cos x}{1 - \sin x}) \\
&\qquad= \int \frac{1}{x + \cos x}
	\frac{(\cos x  - x \sin x)(1 - \sin x) - x \cos x (-\cos x)}{(1 - \sin x)^2}\dd{x} \\
&\qquad= \int \frac{1}{x + \cos x}
	\frac{\cos x - \sin x \cos x - x \sin x + x \sin^2 x + x \cos^2 x}{(1 - \sin x)^2}\dd{x} \\
&\qquad= \int \frac{1}{x + \cos x}
	\frac{\cos x - \sin x \cos x - x \sin x + x}{(1 - \sin x)^2}\dd{x} \\
&\qquad= \int \frac{1}{x + \cos x}
	\frac{(\cos x + x)(1 - \sin x)}{(1 - \sin x)^2}\dd{x} \\
&\qquad= \int \frac{1}{1 - \sin x}\dd{x} \\
&\qquad\xlongequal{u=\tan(x/2)}
	\int \frac{1}{1 - \frac{2u}{u^2 + 1}} \frac{2\dd{u}}{u^2 + 1}
= \int \frac{u^2 + 1}{(u - 1)^2} \frac{2\dd{u}}{u^2 + 1} \\
&\qquad= \int \frac{2}{(u - 1)^2}\dd{u}
\xlongequal{v=u-1} 2 \int v^{-2}\dd{v}
= 2 \cdot (-1) v^{-1} + C \\
&\qquad= \frac{2}{-v} + C
= \frac{2}{1 - u} + C
= \frac{2}{1 - \tan(x/2)} + C,
\end{align*}
所以\[
\int \frac{x \cos x}{(x + \cos x)^2}\dd{x}
= -\frac{x \cos x}{1 - \sin x} \frac{1}{x + \cos x}
	+\frac{2}{1 - \tan(x/2)} + C.
\]
\end{solution}
\end{example}

\section{有理函数的积分}
两个(不含公因式的)多项式的商\(\frac{P(x)}{Q(x)}\)
称为\DefineConcept{有理函数},
又称\DefineConcept{有理分式}.
当分子多项式\(P(x)\)的次数小于分母多项式\(Q(x)\)的次数时,
称这有理函数为\DefineConcept{真分式};
否则称为\DefineConcept{假分式}.

利用多项式的除法,总可以将一个假分式化为一个多项式与一个真分式之和的形式.
例如\[
	\frac{2x^4+x^2+3}{x^2+1}
	= 2x^2-1+\frac{4}{x^2+1}.
\]

对于真分式\(\frac{P(x)}{Q(x)}\),
如果分母可分解为两个多项式的乘积\[
	Q(x)=Q_1(x)Q_2(x),
\]
且\(Q_1(x)\)与\(Q_2(x)\)没有公因式,
那么它可分拆成两个真分式之和\[
	\frac{P(x)}{Q(x)} = \frac{P_1(x)}{Q_1(x)} + \frac{P_2(x)}{Q_2(x)},
\]
我们把上述步骤称为“将真分式化为\DefineConcept{部分分式}之和”.
如果部分分式的分母还能再分解成两个没有公因式的多项式的乘积,
那么就可再分拆成更简单的部分分式.
最后,有理函数的分解式中只出现三类函数:
\begin{enumerate}
	\item 多项式\(F(x)\);
	\item 部分分式\(\frac{P_1(x)}{(x-a)^k}\);
	\item 部分分式\(\frac{P_2(x)}{(x^2+px+q)^l}\),
\end{enumerate}
其中\(p^2-4q<0\),
\(P_1(x)\)为小于\(k\)次的多项式,
\(P_2(x)\)为小于\(2l\)次的多项式.

对于部分分式\(\frac{P_1(x)}{(x-a)^k}\),
我们总可运用以下公式:\[
	\int \frac{\dd{x}}{x-a} = \ln\abs{x-a} + C,
\]\[
	\int \frac{\dd{x}}{(x-a)^k} = \frac{(x-a)^{1-k}}{1-k} + C,
	\quad k>1.
\]

现在我们研究如何计算\[
	\int \frac{Ax+B}{(x^2+px+q)^l} \dd{x}.
\]
经过配方,得\[
	x^2+px+q = \left(x+\frac{p}{2}\right)^2 + q-\frac{p^2}{4}.
\]
令\(a^2=q-p^2/4\),利用\(u=x+p/2\)换元,得\[
	\int \frac{Ax+B}{(x^2+px+q)^l} \dd{x}
	= A \int \frac{u \dd{u}}{(a^2+u^2)^l}
	+ \left(B - \frac{Ap}{2}\right) \int \frac{\dd{u}}{(a^2+u^2)^l}.
\]

我们先求上式右边第一个不定积分\[
	J_l = \int \frac{u \dd{u}}{(a^2+u^2)^l}.
\]
当\(l=1\)时,\[
	J_1
	= \int \frac{u \dd{u}}{a^2+u^2}
	= \frac{1}{2} \ln(a^2+u^2) + C;
\]
当\(l>1\)时,\[
	J_l
	= \int \frac{u \dd{u}}{(a^2+u^2)^l}
	= \frac{1}{2(1-l)} (a^2+u^2)^{1-l} + C.
\]

我们再求右边第二个不定积分\[
	I_l = \int \frac{1}{(a^2+u^2)^l} \dd{u}.
\]
作分部积分,得\begin{align*}
	I_l &= \frac{u}{(a^2+u^2)^l} + 2l \int \frac{u^2}{(a^2+u^2)^{l+1}} \dd{u} \\
	&= \frac{u}{(a^2+u^2)^l} + 2l \int \frac{a^2+u^2-a^2}{(a^2+u^2)^{l+1}} \dd{u} \\
	&= \frac{u}{(a^2+u^2)^l} + 2l I_l - 2la^2 I_{l+1},
\end{align*}
由此推出\[
	I_{l+1} = \frac{1}{2la^2} \frac{u}{(a^2+u^2)^l} + \frac{2k-1}{2ka^2} I_l.
\]
这是一个递推公式.
反复利用这个公式可以把指标\(l\)降低,
最后归结为已知的不定积分.
最初的几个是\begin{align*}
	I_1 &= \int \frac{1}{a^2+u^2} \dd{u} = \frac{1}{a} \arctan\frac{u}{a} + C, \\
	I_2 &= \frac{1}{2a^2} \left( \frac{u}{a^2+u^2} + I_1 \right), \\
	I_3 &= \frac{1}{4a^2} \left[ \frac{u}{(a^2+u^2)^2} + 3 I_2 \right].
\end{align*}

\begin{example}
求\(\int \frac{x+1}{x^2-5x+6} \dd{x}\).
\begin{solution}
被积函数的分母可分解成\((x-3)(x-2)\),故可设\[
	\frac{x+1}{x^2-5x+6}
	= \frac{A}{x-3} + \frac{B}{x-2},
\]
其中\(A,B\)为待定系数.上式两端通分后,得\[
	x+1 = A(x-2)+B(x-3),
\]
即\[
	x+1 = (A+B)x -(2A+3B).
\]
比较上式两端同次幂的系数,即有\[
	\left\{ \begin{array}{l}
		A+B = 1, \\
		2A+3B = -1,
	\end{array} \right.
\]
从而解得\(A=4\),\(B=-3\).
于是\[
	\int \frac{x+1}{x^2-5x+6} \dd{x}
	= \int \left(\frac{4}{x-3} - \frac{3}{x-2}\right) \dd{x}
	= 4\ln\abs{x-3} - 3\ln\abs{x-2} + C.
\]
\end{solution}
\end{example}

\section{本章总结}
\subsection{基本积分表}
\begingroup
\def\intx#1{\int #1\dd{x}}%
\begin{gather}
\intx{k} = kx + C \\
\intx{x^\mu} = \frac{x^{\mu + 1}}{\mu + 1} + C \quad (\mu \neq -1) \\
\intx{\frac{1}{x}} = \ln\abs{x} + C \\
\intx{\frac{1}{1 + x^2}} = \arctan x + C \\
\intx{\frac{1}{\sqrt{1 - x^2}}} = \arcsin x + C \\
\intx{\cos x} = \sin x + C \\
\intx{\sin x} = -\cos x + C \\
\intx{\sec^2 x} = \tan x + C \\
\intx{\csc^2 x} = -\cot x + C \\
\intx{\sec x \tan x} = \sec x + C \\
\intx{\csc x \cot x} = -\csc x + C \\
\intx{e^x} = e^x + C \\
\intx{a^x} = \frac{a^x}{\ln a} + C \\
\intx{\sinh x} = \cosh x + C \\
\intx{\cosh x} = \sinh x + C \\
\intx{\tan x} = -\ln\abs{\cos x} + C \\
\intx{\cot x} = \ln\abs{\sin x} + C \\
\intx{\sec x} = \ln\abs{\sec x + \tan x} + C \\
\intx{\csc x} = \ln\abs{\csc x - \cot x} + C \\
\intx{\frac{1}{a^2 + x^2}} = \frac{1}{a} \arctan\frac{x}{a} + C \\
\intx{\frac{1}{x^2 - a^2}} = \frac{1}{2a} \ln\abs{\frac{x - a}{x + a}} + C \\
\intx{\frac{1}{\sqrt{a^2 - x^2}}} = \arcsin\frac{x}{a} + C \\
\intx{\frac{1}{\sqrt{x^2 + a^2}}} = \ln(x + \sqrt{x^2 + a^2}) + C \\
\intx{\frac{1}{\sqrt{x^2 - a^2}}} = \ln\abs{x + \sqrt{x^2 - a^2}} + C
\end{gather}
\endgroup

\subsection{积分技巧}
\begin{example}
一般的,对于积分\(\int f(ax+b) \dd{x}\),总可作变换\(u=ax+b\),把它作为\[
\int f(ax+b) \dd{x}
= \int \frac{1}{a} f(ax+b) \dd{(ax+b)} \\
= \frac{1}{a} \left[ \int f(u) \dd{u} \right]_{u=ax+b}.
\]
\end{example}

\begin{example}
一般的,对于\(\sin^{2k+1} x \cos^n x\)或\(\sin^n x \cos^{2k+1} x\ (k \in \mathbb{N})\)型函数的积分,总可依次作变换\(u=\cos x\)或\(u=\sin x\),利用恒等式\(\sin^2 x + \cos^2 x \equiv 1\)求得结果.
\begin{align*}
\int \sin^{2k+1} x \cos^n x \dd{x}
&= - \int \sin^{2k} x \cos^n x \cdot \sin x \dd{x} \\
&= - \int (1-\cos^2 x)^k \cos^n x \dd(\cos x), \\
\int \sin^n x \cos^{2k+1} x \dd{x}
&= \int \sin^n x \cos^{2k} x \cdot \cos x \dd{x} \\
&= \int \sin^n x (1-\sin^2 x)^k \dd(\sin x).
\end{align*}
\end{example}

\begin{example}
一般的,对于\(\sin^{2k} x \cos^{2l} x\ (k,l \in \mathbb{N})\)型函数,总可利用三角恒等式\(\sin^2 x = \frac{1}{2}(1-\cos 2x)\),\(\cos^2 x = \frac{1}{2}(1+\cos 2x)\)化成\(\cos 2x\)的多项式,求得结果.\[
\int \sin^{2k} x \cos^{2l} x \dd{x}
= \int \left[\frac{1}{2}(1-\cos 2x)\right]^k \left[\frac{1}{2}(1+\cos 2x)\right]^l \dd{x}
= \int f(\cos 2x) \dd{x}.
\]
\end{example}

\begin{example}
一般的,对于\(\tan^n x \sec^{2k} x\)或\(\tan^{2k+1} x \sec^n x\ (k \in \mathbb{N}^+)\)型函数的积分,可依次作变换\(u=\tan x\)或\(u=\sec x\),利用三角恒等式\(\sec^2 x = \tan^2 x + 1\)和微分公式\(\dd(\tan x) = \sec^2 x \dd{x}\),\(\dd(\sec x) = \sec x \tan x \dd{x}\),求得结果.
\begin{align*}
\int \tan^n x \sec^{2k} x \dd{x}
&=\int \tan^n x\sec^{2k-2} x \cdot \sec^2 x \dd{x} \\
&=\int \tan^n x(1+\tan^2 x)^{k-1} \dd(\tan x), \\
\int \tan^{2k+1} x \sec^n x \dd{x}
&=\int \tan^{2k} x \sec^{n-1} x \cdot \sec x\tan x\dd{x} \\
&=\int (\sec^2 x - 1)^k \sec^{n-1} x \dd(\sec x).
\end{align*}
\end{example}

\begin{example}
如果被积函数含有\(\sqrt{a^2 - x^2}\),可以作代换\(x = a \sin t\)化去根式;如果被积函数含有\(\sqrt{x^2 + a^2}\),可以作代换\(x=a \tan t\)化去根式;如果被积函数含有\(\sqrt{x^2 - a^2}\),可以作代换\(x=\pm a \sec t\)化去根式.

如果被积函数有高次多项式,可以首先利用平方差公式和平方和公式对其配方.

当被积函数含有\(\sqrt{x^2 \pm a^2}\)时,为了化去根式,除采用三角代换\(x = a \tan t\)或\(x = \pm a \sec t\)外,还可利用公式\(\cosh^2 t - \sinh^2 t = 1\),采用双曲代换\(x = a \sinh t\)和\(x = \pm a \cosh t\)来化去根式.
\end{example}

\begin{example}
如果被积函数是分式,且积分变量在分子中的最高幂次比其在分母中的最高幂次要低2次或3次,则可利用倒代换技巧,即令\(t=\frac{1}{x}\).
\end{example}

\begin{example}
如果被积函数是幂函数和正(余)弦函数或幂函数和指数函数的乘积,就可以考虑用分部积分法,并设幂函数为\(u\).这样每用一次分部积分法就可使幂次降低一次.

如果被积函数是幂函数和对数函数或幂函数和反三角函数的乘积,也可以考虑用分部积分法,并设对数函数或反三角函数为\(u\).
\end{example}

\begin{example}
如果被积函数中含有简单根式\(\sqrt[n]{ax+b}\)或\(\sqrt[n]{\frac{ax+b}{cx+d}}\),可以令这个简单根式为\(u\),即\begin{align*}
u=\sqrt[n]{ax+b} &\implies x=\frac{1}{a}(u^n-b) \\
u=\sqrt[n]{\frac{ax+b}{cx+d}} &\implies x=\frac{u^nd-b}{a-u^nc}
\end{align*}由于这样的变换具有反函数,且反函数是\(u\)的有理函数,因此原积分可以化为有理函数的积分.
\end{example}
