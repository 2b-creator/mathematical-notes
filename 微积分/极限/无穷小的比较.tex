\section{无穷小的比较}
在\cref{section:极限.极限的运算法则}中我们已经知道,
两个无穷小的和、差、积仍旧是无穷小.
但是,关于两个无穷小的商,却会出息不同的情况.
例如,当\(x\to0\)时,
\(3x\)、\(x^2\)、\(\sin x\)都是无穷小,
而\[
	\lim_{x\to0}\frac{x^2}{3x}=0, \qquad
	\lim_{x\to0}\frac{3x}{x^2}=\infty, \qquad
	\lim_{x\to0}\frac{\sin x}{3x}=\frac{1}{3}.
\]
两个无穷小之比的极限的各种不同情况,
反映了不同的无穷小趋于零的“快慢”程度.
就上面几个例子来说,
在\(x\to0\)的过程中,
\(x^2\to0\)比\(3x\to0\)要“快些”,
反过来说\(3x\to0\)比\(x^2\to0\)要“慢些”,
而\(\sin x\to0\)与\(x\to0\)“快慢相仿”.

下面,我们就无穷小之比的极限存在或为无穷大时,来说明两个无穷小之间的比较.
应当注意,下面的\(\alpha\)及\(\beta\)都是在同一个自变量的变化过程中的无穷小,
且\(\alpha\neq0\),\(\lim\frac{\beta}{\alpha}\)也是在这个变化过程中的极限.

\begin{definition}
%@see: 《高等数学(第六版 上册)》 P57 定义
\newcommand{\lf}[1][]{\lim \frac{\beta}{\alpha^{#1}}}
设\(\alpha\)和\(\beta\)都是在同一个自变量的变化过程中的无穷小,且\(\alpha\neq0\).
\begin{enumerate}
\item 如果\(\lf=0\),就说\(\beta\)是比\(\alpha\) \DefineConcept{高阶}的无穷小,记作\(\beta=o(\alpha)\);
\item 如果\(\lf=\infty\),就说\(\beta\)是比\(\alpha\) \DefineConcept{低阶}的无穷小.
\item 如果\(\lf=c\neq0\),就说\(\beta\)是比\(\alpha\) \DefineConcept{同阶}无穷小;
\item 如果\(\lf[k]=c\neq0\),\(k>0\),就说\(\beta\)是关于\(\alpha\)的\(k\) \DefineConcept{阶}无穷小.
\item 如果\(\lf=1\),就说\(\beta\)与\(\alpha\)是\DefineConcept{等价无穷小},记作\(\alpha\sim\beta\).
\end{enumerate}
\end{definition}
显然,等价无穷小是同阶无穷小的特殊情形,即\(c=1\)的情形.

应该注意到,记号\(o(\alpha)\)实际上是由所有满足\[
\lim \frac{\beta}{\alpha} = 0
\]的函数\(\beta\)构成的集合,即\[
o(\alpha) = \Set*{ \beta \given \lim \frac{\beta}{\alpha} = 0 }.
\]虽然在给定\(\lim \frac{\beta}{\alpha}\)时,我们本该用记号\(\beta \in o(\alpha)\)表示\(\beta\)是比\(\alpha\)高阶的无穷小,但我们在后面很快就会看到使用记号\(\beta = o(\alpha)\)的诸多好处.

\begin{example}
因为\(\lim_{x\to0} \frac{3x^2}{x} = 0\),所以当\(x\to0\)时,\(3x^2\)是比\(x\)高阶的无穷小,即\(3x^2 = o(x)\ (x\to0)\).

因为\(\lim_{n\to\infty} \frac{1/n}{1/n^2} = \infty\),所以当\(n\to\infty\)时,\(\frac{1}{n}\)是比\(\frac{1}{n^2}\)低阶的无穷小.

因为\(\lim_{x\to3} \frac{x^2-9}{x-3} = 6\),所以当\(x\to3\)时,\(x^2-9\)与\(x-3\)是同阶无穷小.

因为\(\lim_{x\to0} \frac{1-\cos x}{x^2} = \frac{1}{2}\),所以当\(x\to0\)时,\(1-\cos x\)是关于\(x\)的二阶无穷小.

因为\(\lim_{x\to0} \frac{\sin x}{x} = 1\),所以当\(x\to0\)时,\(\sin x\)与\(x\)是等价无穷小,即\(\sin x \sim x\ (x\to0)\).
\end{example}

\begin{example}
%@see: 《高等数学(第六版 上册)》 P58 例1
证明:当\(x\to0\)时,\(\sqrt[n]{1+x} - 1 \sim \frac{1}{n} x\).
\begin{proof}
因为\begin{align*}
\frac{\sqrt[n]{1+x} - 1}{\frac{1}{n} x}
&= \frac{(\sqrt[n]{1+x})^n - 1}{\frac{1}{n} x \left[ \sqrt[n]{(1+x)^{n-1}} + \sqrt[n]{(1+x)^{n-2}} + \dotsb + 1 \right]} \\
&= \frac{n}{\sqrt[n]{(1+x)^{n-1}} + \sqrt[n]{(1+x)^{n-2}} + \dotsb + 1},
\end{align*}
而\[
\lim_{x\to0} \sqrt[n]{(1+x)^m} = 1,
\]所以\[
\lim_{x\to0} \frac{\sqrt[n]{1+x} - 1}{\frac{1}{n} x} = \lim_{x\to0} \frac{n}{1 \cdot n} = 1,
\]也就是说\(\sqrt[n]{1+x} - 1 \sim \frac{1}{n} x \quad(x\to0)\).
\end{proof}
\end{example}

\begin{theorem}\label{theorem:极限.无穷小的比较1}
%@see: 《高等数学(第六版 上册)》 P58 定理1
\(\beta\)与\(\alpha\)是等价无穷小的充分必要条件为\[
	\beta=\alpha+o(\alpha).
\]
\begin{proof}
必要性.
设\(\alpha\sim\beta\),则\[
	\lim\frac{\beta-\alpha}{\alpha}
	=\lim\left(\frac{\beta}{\alpha}-1\right)
	=\lim\frac{\beta}{\alpha}-1 = 0,
\]
因此\(\beta-\alpha=o(\alpha)\),即\(\beta=\alpha+o(\alpha)\).

充分性.
设\(\beta=\alpha+o(\alpha)\),则\[
	\lim\frac{\beta}{\alpha}
	=\lim\frac{\alpha+o(\alpha)}{\alpha}
	=\lim\left(1+\frac{o(\alpha)}{\alpha}\right) = 1,
\]
因此\(\alpha\sim\beta\).
\end{proof}
\end{theorem}

\begin{theorem}\label{theorem:极限.无穷小的比较2}
%@see: 《高等数学(第六版 上册)》 P59 定理2
设\(\alpha\sim\alpha'\),\(\beta\sim\beta'\),
且\(\lim\frac{\beta'}{\alpha'}\)存在,
则\[
	\lim\frac{\beta}{\alpha}=\lim\frac{\beta'}{\alpha'}.
\]
\begin{proof}
由\(\alpha\sim\alpha'\)和\(\beta\sim\beta'\)有
\(\lim\frac{\alpha'}{\alpha} = 1\),
\(\lim\frac{\beta}{\beta'} = 1\),
因此\[
	\lim\frac{\beta}{\alpha}
	= \lim \left(
		\frac{\beta}{\beta'}
		\cdot \frac{\beta'}{\alpha'}
		\cdot \frac{\alpha'}{\alpha}
	\right)
	= \lim\frac{\beta}{\beta'}
		\cdot \lim \frac{\beta'}{\alpha'}
		\cdot \lim\frac{\alpha'}{\alpha}
	= 1 \cdot \lim \frac{\beta'}{\alpha'} \cdot 1
	= \lim \frac{\beta'}{\alpha'}.
	\qedhere
\]
\end{proof}
\end{theorem}
\cref{theorem:极限.无穷小的比较2} 表明,
求两个无穷小之比的极限时,
分子及分母都可用等价无穷小来代替.
因此,如果用来代替的无穷小选得适当的话,
可以使计算简化.

必须指出的是,等价无穷小实际上是一种等价关系.
\begin{property}\label{theorem:极限.无穷小的比较6}
设\(\alpha,\beta,\gamma\)都是在同一个自变量的变化过程中的无穷小,
那么\begin{enumerate}
	\item {\bf 自反性}:
	\(\alpha \sim \alpha\);

	\item {\bf 对称性}:
	\(\alpha \sim \beta \implies \beta \sim \alpha\);

	\item {\bf 传递性}:
	\(\alpha \sim \beta, \beta \sim \gamma \implies \alpha \sim \gamma\).
\end{enumerate}
\end{property}


\begin{proposition}
设\(\alpha,\beta,\gamma\)都是在同一个自变量的变化过程中的无穷小,
\(\alpha\neq0,
\gamma\neq0\),
且\[
	\lim\frac{\beta}{\alpha^p}=c_1\in(0,+\infty), \qquad
	\lim\frac{\gamma}{\alpha^q}=c_2\in(0,+\infty).
\]
那么:\begin{enumerate}
	\item 若\(p>q\),
	则\(\beta\)是比\(\gamma\)高阶的无穷小.

	\item 若\(p=q\),
	则\(\beta\)是比\(\gamma\)同阶的无穷小.

	\item 若\(p<q\),
	则\(\beta\)是比\(\gamma\)低阶的无穷小.
\end{enumerate}

\begin{proof}
容易得到\begin{align*}
	\lim\frac{\beta}{\gamma}
	&= \lim\left(
			\frac{\beta}{\alpha^p}
			\frac{\alpha^p}{\alpha^q}
			\frac{\alpha^q}{\gamma}
		\right) \\
	&= \lim\frac{\beta}{\alpha^p}
		\cdot \lim\frac{\alpha^p}{\alpha^q}
		\cdot \lim\frac{\alpha^q}{\gamma} \\
	&= c_1 \cdot \lim\alpha^{p-q} \cdot c_2^{-1}
	= \frac{c_1}{c_2} \lim\alpha^{p-q}.
	\tag1
\end{align*}
注意到\(\lim\alpha=0\).
当\(p>q\)时,(1)式等于\(0\),\(\beta\)是比\(\gamma\)高阶的无穷小,即\(\beta=o(\gamma)\).
当\(p=q\)时,(1)式等于\(\frac{c_1}{c_2}\),\(\beta\)是比\(\gamma\)同阶的无穷小.
当\(p<q\)时,(1)式等于\(\infty\),\(\beta\)是比\(\gamma\)低阶的无穷小,即\(\gamma=o(\beta)\).
\end{proof}
\end{proposition}
正因如此,我们常在求极限的时候,将某个数列或函数作为比较无穷小的阶的参照系.
例如,在讨论当\(n\to\infty\)时的无穷小时,
我们把数列\(\left\{\frac1n\right\}\)作为参照系,
计算极限\(\lim_{n\to\infty} n \sin\frac1n = 1\),
然后说“数列\(\left\{\sin\frac1n\right\}\)是(\(\frac1n\)的)1阶无穷小”.
又例如,在讨论当\(x\to0\)时的无穷小时,
我们把函数\(f(x)=x\)作为参考系,
计算极限\(\lim_{x\to0} \frac{3x^2+4x^5}{x^2} = 3\),
然后说“函数\(3x^2+4x^5\)是(\(x\)的)2阶无穷小”.

\begin{proposition}[和差取大规则]\label{theorem:极限.无穷小的比较3}
设\(\beta=o(\alpha)\),则\(\alpha\pm\beta\sim\alpha\).
\end{proposition}
\cref{theorem:极限.无穷小的比较3} 说明,
当高阶无穷小\(\beta\)和低阶无穷小\(\alpha\)相加或相减时,
它们的等价无穷小就是低阶无穷小\(\alpha\).
因为当自变量变化时,高阶无穷小比低阶无穷小更快地趋于零,
相对而言,低阶无穷小就显得更“大”一些,
因此我们把\cref{theorem:极限.无穷小的比较3} 称为“和差取大规则”.

\begin{proposition}[和差代替规则]\label{theorem:极限.无穷小的比较4}
设\(\alpha\sim\alpha'\),\(\beta\sim\beta'\),\(\beta\)与\(\alpha\)不是等价无穷小,则\[
	\alpha\pm\beta\sim\alpha'\pm\beta'.
\]
\end{proposition}

\begin{proposition}[因式代替规则]\label{theorem:极限.无穷小的比较5}
设\(\alpha\sim\beta\),且函数\(\phi\)有界或\(\lim\phi\)存在,则\[
	\alpha \phi \sim \beta \phi.
\]
\end{proposition}
