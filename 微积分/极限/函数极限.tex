\section{函数极限}
从\hyperref[definition:数列.数列的定义]{数列的定义}可以看出,
任意一个数列\(\{x_n\}\)只不过是一个以\(n\)为自变量的函数\(x_n = f(n)\).
我们只要把数列极限概念中
“自变量\(n\)的取值范围是整数的子集”
“自变量变化过程是\(n\to\infty\)”
等特殊性撇开,
就可以引出函数极限的一般概念.

\subsection{函数极限的定义}
\subsubsection*{自变量趋于有限值时函数的极限}
\begin{definition}\label{definition:极限.函数极限的定义1}
对于函数\(f\colon D\to\mathbb{R}\),
取\(x_0 \in D\),
如果存在常数\(A\in\mathbb{R}\),
对于任意给定的正数\(\epsilon\)(不论它多么小),
总存在正数\(\delta\),
使得当\(x\)满足不等式\(0 < \abs{x-x_0} < \delta\)时,
对应的函数值\(f(x)\)都满足不等式\(\abs{f(x)-A}<\epsilon\),
那么常数\(A\)就叫做“函数\(f(x)\)当\(x \to x_0\)时的\DefineConcept{极限}”,
记作
\[
	\lim_{\substack{x \to x_0 \\ x \in D}} f(x) = A,
	\quad\text{或}\quad
	f(x) \to A\ (x \to x_0; x \in D);
\]
有时候也简记为
\[
	\lim_{x \to x_0} f(x) = A,
	\quad\text{或}\quad
	f(x) \to A\ (x \to x_0).
\]
\end{definition}

上述对函数极限的定义可以简化为:
\[
	\lim_{\substack{x \to x_0 \\ x \in D}} f(x) = A
	\defiff
	(\forall \epsilon > 0)
	(\exists \delta > 0)
	(\forall x \in D)
	[0 < \abs{x - x_0} < \delta \implies \abs{f(x) - A} < \epsilon].
\]

现在我们来根据定义尝试计算一些函数的极限.
\begin{example}
证明:\(\lim_{x \to x_0} c = c\),其中\(c\)为常数.
\begin{proof}
这里\(\abs{f(x) - A} = \abs{c - c} = 0\),
因此\(\forall \epsilon > 0\),
可任取\(\delta > 0\),
当\(0 < \abs{x - x_0} < \delta\)时,
总能使不等式\[
	\abs{f(x) - A} = \abs{c - c} = 0 < \epsilon
\]成立,
所以\(\lim_{x \to x_0} c = c\).
\end{proof}
\end{example}

\begin{example}
证明:\(\lim_{x \to x_0} x = x_0\).
\begin{proof}
这里\(\abs{f(x) - A} = \abs{x - x_0}\),
因此\(\forall \epsilon > 0\),
总可取\(\delta = \epsilon\),
当\(0 < \abs{x - x_0} < \delta = \epsilon\)时,
能使不等式\(\abs{f(x) - A} = \abs{x - x_0} < \epsilon\)成立,
所以\(\lim_{x \to x_0} x = x_0\).
\end{proof}
\end{example}

\begin{example}
证明:\(\lim_{x\to1} 2x-1 = 1\).
\begin{proof}
由于\(\abs{f(x) - A} = \abs{(2x-1) - 1} = 2\abs{x-1}\),
为了使\(\abs{f(x) - A} < \epsilon\),
只要\(\abs{x-1}<\epsilon/2\).
因此\(\forall \epsilon > 0\),
可取\(\delta = \epsilon/2\),
则当\(x\)适合不等式\[
	0 < \abs{x-1} < \delta = \frac{\epsilon}{2}
\]时,
对应的函数值\(f(x)\)就满足不等式\[
	\abs{f(x) - 1} = \abs{(2x-1) - 1} = 2\abs{x-1} < \epsilon.
\]
从而\(\lim_{x\to1} 2x-1 = 1\).
\end{proof}
\end{example}

\begin{example}
证明:\(\lim_{x\to1} \frac{x^2-1}{x-1} = 2\).
\begin{proof}
这里,函数在点\(x=1\)处是没有定义的,
但是函数当\(x\to1\)时的极限存在或不存在与之无关.
事实上,\(\forall \epsilon > 0\),将不等式\[
	\abs{\frac{x^2-1}{x-1} - 2} < \epsilon
\]
约去非零因子\(x-1\)后,就化为\[
	\abs{(x+1)-2} = \abs{x-1} < \epsilon,
\]
因此,只要取\(\delta = \epsilon\),
那么当\(0 < \abs{x-1} < \delta\)时,就有\[
	\abs{\frac{x^2-1}{x-1} - 2} < \epsilon.
\]
所以\(\lim_{x\to1} \frac{x^2-1}{x-1} = 2\).
\end{proof}
\end{example}

\begin{example}\label{example:极限.根式函数在某一点的极限}
证明:当\(x_0 > 0\)时,
\(\lim_{x \to x_0}\sqrt{x} = \sqrt{x_0}\).
\begin{proof}
\(\forall \epsilon > 0\),
因为\[
	\abs{f(x) - A} = \abs{\sqrt{x} - \sqrt{x_0}}
	= \abs{\frac{x - x_0}{\sqrt{x} + \sqrt{x_0}}}
	\leq \frac{\abs{x - x_0}}{\sqrt{x_0}},
\]
那么要使\(0 < \abs{x - x_0} < \delta \implies \abs{f(x) - A} < \epsilon\),
只要\[
	0 < \abs{x - x_0} < \delta \implies \abs{x - x_0} < \sqrt{x_0} \epsilon,
\]
也即\[
	\delta \leq \sqrt{x_0} \epsilon.
	\eqno(1)
\]

由于我们还要确保\(f\)在\(x_0\)的去心邻域内有定义,
于是又有\[
	0 < \abs{x-x_0} < \delta \implies x\geq0
	\quad\text{或}\quad
	(x_0-\delta,x_0+\delta)\subseteq[0,+\infty),
\]
也即\[
	x_0-\delta\geq0
	\quad\text{或}\quad
	x_0\geq\delta.
	\eqno(2)
\]

综上所述,如果取\(\delta = \min\{x_0,\sqrt{x_0} \epsilon\}\),
则当\(0 < \abs{x - x_0} < \delta\)时,
对应的函数值\(\sqrt{x}\)就满足\[
	\abs{f(x) - A} = \abs{\sqrt{x} - \sqrt{x_0}} < \epsilon,
\]
所以\(\lim_{x \to x_0}\sqrt{x} = \sqrt{x_0}\).
\end{proof}
\end{example}

\begin{definition}\label{definition:极限.函数极限的定义2}
设函数\(f(x)\)在\(x < x_0\)时有定义.
如果存在常数\(A\),
对于任意给定的正数\(\epsilon\),
总存在正数\(\delta\),
使得当\(x\)满足不等式\(-\delta < x - x_0 < 0\)或\(x_0 - \delta < x < x_0\)时,
对应的函数值\(f(x)\)都满足不等式\(\abs{f(x) - A} < \epsilon\),
那么常数\(A\)就叫做“函数\(f(x)\)当\(x \to x_0\)时的\DefineConcept{左极限}”,
记作\[
\lim_{x \to x_0^-} f(x) = A
\quad\text{或}\quad
f(x_0^-) = A.
\]
\end{definition}

\begin{definition}\label{definition:极限.函数极限的定义3}
设函数\(f(x)\)在\(x > x_0\)时有定义.
如果存在常数\(A\),
对于任意给定的正数\(\epsilon\),
总存在正数\(\delta\),
使得当\(x\)满足不等式\(0 < x - x_0 < \delta\)或\(x_0 < x < x_0 + \delta\)时,
对应的函数值\(f(x)\)都满足不等式\(\abs{f(x) - A} < \epsilon\),
那么常数\(A\)就叫做“函数\(f(x)\)当\(x \to x_0\)时的\DefineConcept{右极限}”,
记作\[
\lim_{x \to x_0^+} f(x) = A
\quad\text{或}\quad
f(x_0^+) = A.
\]
\end{definition}
左极限与右极限统称为\DefineConcept{单侧极限}.

\begin{theorem}
函数\(f(x)\)当\(x \to x_0\)时极限存在的充分必要条件是:其左极限和右极限分别存在且相等,即\[
f(x_0^-) = f(x_0^+).
\]
\end{theorem}

\begin{example}
证明:函数\[
f(x) = \left\{ \begin{array}{lc}
x-1, & x<0, \\
0, & x=0, \\
x+1, & x>0.
\end{array} \right.
\]当\(x\to0\)时\(f(x)\)的极限不存在.
\begin{proof}
易证\[
\lim_{x\to0^-} f(x) = \lim_{x\to0^-} x-1 = -1,
\]而\[
\lim_{x\to0^+} f(x) = \lim_{x\to0^+} x+1 = 1,
\]因为左、右极限存在但不相等,所以\(\lim_{x\to0}\)不存在.
\end{proof}
\end{example}

\subsubsection*{自变量趋于无穷大时函数的极限}
\begin{definition}\label{definition:极限.函数极限的定义4}
设函数\(f(x)\)当\(\abs{x}\)大于某一正数时有定义.
如果存在常数\(A\),
对于任意给定的正数\(\epsilon\)(不论它多么小),
总存在正数\(X\),
使得当\(x\)满足不等式\(\abs{x} > X\)时,
对应的函数值\(f(x)\)都满足不等式\[
\abs{f(x) - A} < \epsilon,
\]那么常数\(A\)就叫做“函数\(f(x)\)当\(x \to \infty\)时的\DefineConcept{极限}”,
记作\[
\lim_{x \to \infty} f(x) = A
\quad\text{或}\quad
f(\infty) = A
\quad\text{或}\quad
f(x) \to A\ (x \to \infty).
\]
\end{definition}
\cref{definition:极限.函数极限的定义4} 可以简化为:
\[
	\lim_{x\to\infty} f(x) = A
	\defiff
	(\forall\epsilon>0)
	(\exists X>0)
	[
		\abs{x} > X
		\implies
		\abs{f(x) - A} < \epsilon
	].
\]

我们只要对\cref{definition:极限.函数极限的定义4} 中的条件稍加改变,
就可以得到以下两种不同的极限定义.
\begin{definition}\label{definition:极限.函数极限的定义5}
设函数\(f(x)\)在\(x > 0\)时有定义.
如果存在常数\(A\),
对于任意给定的正数\(\epsilon\),
总存在正数\(X\),
使得当\(x\)满足不等式\(x > X\)时,
对应的函数值\(f(x)\)都满足不等式\[
	\abs{f(x) - A} < \epsilon,
\]
那么常数\(A\)就叫做“函数\(f(x)\)当\(x \to +\infty\)时的\DefineConcept{极限}”,
记作\[
	\lim_{x \to +\infty} f(x) = A
	\quad\text{或}\quad
	f(+\infty) = A
	\quad\text{或}\quad
	f(x) \to A\ (x \to +\infty).
\]
\end{definition}

\begin{definition}\label{definition:极限.函数极限的定义6}
设函数\(f(x)\)在\(x < 0\)时有定义.
如果存在常数\(A\),
对于任意给定的正数\(\epsilon\),
总存在正数\(X\),
使得当\(x\)满足不等式\(x < -X\)时,
对应的函数值\(f(x)\)都满足不等式\[
\abs{f(x) - A} < \epsilon,
\]那么常数\(A\)就叫做“函数\(f(x)\)当\(x \to -\infty\)时的\DefineConcept{极限}”,
记作\[
\lim_{x \to -\infty} f(x) = A
\quad\text{或}\quad
f(-\infty) = A
\quad\text{或}\quad
f(x) \to A\ (x \to -\infty).
\]
\end{definition}

\begin{example}
\def\l{\lim_{x\to\infty}}
证明:\(\l \frac{1}{x} = 0\).
\begin{proof}
\(\forall\epsilon>0\),要证\(\exists X > 0\),当\(\abs{x}>X\)时,不等式\[
\abs{\frac{1}{x}-0}<\epsilon
\]成立.
因这个不等式相当于\(\frac{1}{\abs{x}}<\epsilon\)或\(\abs{x}>\frac{1}{\epsilon}\).
由此可知,如果取\(X=\frac{1}{\epsilon}\),
那么当\(\abs{x}>X=\frac{1}{\epsilon}\)时,
不等式\(\abs{\frac{1}{x}-0}<\epsilon\)成立,
这就证明了\(\l \frac{1}{x} = 0\).
\end{proof}
\end{example}

\subsection{函数极限的性质}
与收敛数列的性质相比较,可得函数极限的一些相应的性质.它们都可以根据函数极限的定义,运用类似于证明收敛数列性质的方法加以证明.由于函数极限的定义按自变量的变化过程不同有各种形式,下面仅以“\(\lim_{x \to x_0}f(x)\)”这种形式为代表给出关于函数极限性质的一些定理,并就其中的几个给出证明.至于其他形式的极限的性质及其证明,只要相应地做一些修改即可得出.

\begin{theorem}[函数极限的唯一性]\label{theorem:极限.函数极限的唯一性}
如果\(\lim_{x \to x_0}f(x)\)存在,那么这极限唯一.
\end{theorem}

\begin{theorem}[函数极限的局部有界性]\label{theorem:极限.函数极限的局部有界性}
如果\(\lim_{x \to x_0}f(x) = A\),那么存在常数\(M>0\)和\(\delta>0\),
使得当\(0<\abs{x-x_0}<\delta\)时,有\(\abs{f(x)} \leq M\).
\begin{proof}
因为\(\lim_{x \to x_0}f(x) = A\),所以若取\(\epsilon = 1\),则\(\exists \delta > 0\),当\(0 < \abs{x - x_0} < \delta\)时,有\[
\abs{f(x) - A} < 1 \implies \abs{f(x)} \leq \abs{f(x) - A} + \abs{A} < \abs{A} + 1.
\]记\(M = \abs{A} + 1\)即可.
\end{proof}
\end{theorem}

\begin{theorem}[函数极限的局部保号性]\label{theorem:极限.函数极限的局部保号性1}
如果\(\lim_{x \to x_0}f(x)=A\),且\(A \gtrless 0\),那么存在常数\(\delta>0\),使得当\(0<\abs{x-x_0}<\delta\)时,有\(f(x) \gtrless 0\).
\begin{proof}
就\(A > 0\)的情形证明.因为\(\lim_{x \to x_0}f(x) = A > 0\),所以若取\(\epsilon = \frac{A}{2} > 0\),则\(\exists \delta > 0\),当\(0 < \abs{x - x_0} < \delta\)时,有\[
\abs{f(x) - A} < \epsilon = \frac{A}{2}
\implies f(x) > A - \frac{A}{2} = \frac{A}{2} > 0.
\]类似地可以证明\(A < 0\)的情形.
\end{proof}
\end{theorem}

\begin{theorem}\label{theorem:极限.函数极限的局部保号性2}
%@see: 《高等数学(第六版 上册)》 P37 定理3'
设\(D\)是一个区间,
函数\(f\colon D\to\mathbb{R}\),
点\(x_0 \in D\).
如果\(\lim_{x \to x_0} f(x) = A \neq 0\),
那么\[
	(\exists \delta>0)
	(\forall x)
	\left[
		0<\abs{x-x_0}<\delta
		\implies
		\abs{f(x)}>\frac12\abs{A}.
	\right].
\]
\end{theorem}

\begin{corollary}\label{theorem:极限.函数极限的局部保号性3}
%@see: 《高等数学(第六版 上册)》 P37 推论
设\(D\)是一个区间,
函数\(f\colon D\to\mathbb{R}\),
点\(x_0 \in D\),
且\(\lim_{x \to x_0} f(x) = A\).
\begin{enumerate}
	\item 如果\((\forall x \in D)[f(x) \geq 0]\),则\(A \geq 0\).
	\item 如果\((\forall x \in D)[f(x) \leq 0]\),则\(A \leq 0\).
\end{enumerate}
\end{corollary}

\begin{example}
设函数\(f(x)\)在区间\((0,+\infty)\)上单调减少.
证明:若\(\lim_{x\to+\infty} f(x) = A\),
则\((\forall x\in(0,+\infty))[f(x)>A]\).
\begin{proof}
用反证法.
假设\((\exists x_0\in(0,+\infty))[f(x_0)<A]\),
那么\[
	(\forall x\in(x_0,+\infty))
	\left[
		\begin{array}{l}
			f(x)<f(x_0)<A \\
			\iff
			f(x) - A < f(x_0) - A < 0 \\
			\implies
			\abs{f(x) - A} > \abs{f(x_0) - A} > 0
		\end{array}
	\right].
\]
但是这与\[
	\lim_{x\to+\infty} f(x) = A
	\iff
	(\forall \epsilon>0)
	(\exists X>0)
	(\forall x>0)
	[
		x > X
		\implies
		\abs{f(x) - A} < \epsilon
	]
\]矛盾,
说明\((\forall x_0 > 0)[f(x_0) > A]\).
\end{proof}
\end{example}

\begin{theorem}[函数极限的保序性]\label{theorem:极限.函数极限的保序性}
若函数\(f(x)\)和\(g(x)\)在区间\(I\)上满足\(f(x) \leq g(x)\),那么有\[
	\varlimsup_{x \to a} f(x) \leq \varlimsup_{x \to a} g(x),
\]\[
	\varliminf_{x \to a} f(x) \leq \varliminf_{x \to a} g(x).
\]
\end{theorem}

\begin{example}
\def\l{\lim_{n\to+\infty}}
根据函数极限的定义证明:\(\l\frac{\sin x}{\sqrt x} = 0\).
\begin{proof}
\(\forall \epsilon>0\),要证\(\exists X > 0\),当\(x > X\)时,不等式\[
\abs{\frac{\sin x}{\sqrt x} - 0} = \frac{\abs{\sin x}}{\sqrt x}
\leq \frac{1}{\sqrt x} < \epsilon
\]成立.因这个不等式相当于\(\sqrt x > 1/\epsilon\)或\(x > 1/\epsilon^2\),由此可知,如果取\(X = 1/\epsilon^2\),那么当\(x > X = \frac{1}{\epsilon^2}\)时,不等式\(\abs{\frac{\sin x}{\sqrt x} - 0} < \epsilon\)成立,这就证明了\(\l\frac{\sin x}{\sqrt x} = 0\).
\end{proof}
\end{example}

\begin{example}
\def\l#1{\lim_{x\to#1}}
证明:\(\l0\abs{x}=0\).
\begin{proof}
先证\(\l{0^+}\abs{x}=0\).
\(\forall \epsilon>0\),要证\(\exists \delta > 0\),当\(0<x<\delta\)时,不等式\[
\abs{\abs{x} - 0} = x < \epsilon
\]成立,只需令\(\delta = \epsilon\)即可.

再证\(\l{0^-}\abs{x}=0\).
\(\forall \epsilon>0\),要证\(\exists \delta > 0\),当\(-\delta<x<0\)时,不等式\[
\abs{\abs{x} - 0} = -x < \epsilon
\quad \text{或} \quad
x > -\epsilon
\]成立,只需令\(\delta = \epsilon\)即可.

因为\(\l{0^+}\abs{x}=\l{0^-}\abs{x}=0\),所以\(\l0\abs{x}=0\).
\end{proof}
\end{example}

\subsection{子列极限与上下极限}
\begin{definition}\label{definition:极限.函数的子列极限和上下极限}
设函数\(f(x)\)在区间\(I\)上有定义.
如果数列\(\{x_n\}\)满足\[
	x_i \in I \quad (i=1,2,\dotsc,n),
\]且数列极限\(\lim_{n\to\infty}{x_n} = a\),
则称极限\(\lim_{n\to\infty}{f(x_n)}\)为
“函数\(f(x)\)在点\(a\)的\DefineConcept{子列极限}”.

这些子列极限中的最小值称作“函数\(f(x)\)在点\(a\)的\DefineConcept{下极限}”,
记作\[
	\varliminf_{x \to a} f(x)
	\quad\text{或}\quad
	\liminf_{x \to a} f(x).
\]

这些子列极限中的最大值称作“函数\(f(x)\)在点\(a\)的\DefineConcept{上极限}”,
记作\[
	\varlimsup_{x \to a} f(x)
	\quad\text{或}\quad
	\limsup_{x \to a} f(x).
\]
\end{definition}

\begin{property}
函数\(f(x)\)的上、下极限满足:\begin{gather}
\varlimsup_{x \to a} f(x) = \lim_{\delta\to0^+} \sup_{0<\abs{x-a}<\delta} f(x), \\
\varliminf_{x \to a} f(x) = \lim_{\delta\to0^+} \inf_{0<\abs{x-a}<\delta} f(x).
\end{gather}
\end{property}

\begin{theorem}
任意函数\(f(x)\)的上下极限总满足\[
\varliminf_{x \to a} f(x) \leq \varlimsup_{x \to a} f(x).
\]
\end{theorem}

\begin{theorem}
函数\(f(x)\)当\(x \to x_0\)时极限存在的充分必要条件是:其上极限和下极限相等,即\[
\varlimsup_{x \to a} f(x) = \varliminf_{x \to a} f(x) = A
\iff
\lim_{x \to a} f(x) = A.
\]
\end{theorem}
