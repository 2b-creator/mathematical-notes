\section{函数极限与数列极限的关系,海涅定理}
\begin{theorem}[海涅定理]\label{theorem:极限.海涅定理}
%@see: 《数学分析教程》(史济怀) P70 定理2.4.1
设函数\(f\colon D\to\mathbb{R}\),点\(x_0 \in D\).
函数\(f\)在点\(x_0\)有极限\(A\)的充分必要条件是:
对于任何一个收敛于\(x_0\)的数列\(\{x_n \neq x_0\ (n=1,2,\dotsc)\}\),
函数列\(\{f(x_n)\}\)有极限\(A\).
\end{theorem}
海涅定理也称作\DefineConcept{归结原则}.

这里要注意定理的条件\(x_n \neq x_0\),
因为只要没有这个条件就不能保证海涅定理的结果成立,下面我们举例说明.
设\[
	f(x) = \left\{ \begin{array}{cl}
		2, & x\neq0, \\
		1, & x=0.
	\end{array} \right.
\]
又设点\(x_0\)与数列\(\{x_n\}\)满足\(x_0=x_n=0\ (n=1,2,\dotsc)\).
那么有\[
	\lim_{n\to\infty} x_n = 0,
	\quad\text{和}\quad
	\lim_{x\to0} f(x) = 2,
\]
但是\[
	\lim_{n\to\infty} f(x_n) = \lim_{n\to\infty} f(0) = 1.
\]这就说明不总有\[
	\lim_{n\to\infty} f(x_n)
	= \lim_{x \to x_0} f(x)
\]成立.

\begin{example}
函数\(f(x)\)在\((-\infty,+\infty)\)内单调有界,
数列\(\{x_n\}\)收敛,
但函数列\(\{f(x_n)\}\)不一定收敛.
例如,取函数\[
	f(x) = \left\{ \begin{array}{rl}
		1 & x\geq0, \\
		-1, & x<0,
	\end{array} \right.
\]
和数列\(x_n = \frac{(-1)^n}{n}\),
容易看出\(\varlimsup_{n\to\infty} f(x_n) = 1\)
和\(\varliminf_{n\to\infty} f(x_n) = -1\),
于是\(\lim_{n\to\infty} f(x_n)\)发散.
\end{example}

\begin{example}
函数\(f(x)\)在\((-\infty,+\infty)\)内单调有界,
函数列\(\{f(x_n)\}\)单调或收敛,
但数列\(\{x_n\}\)不一定收敛.
例如,取\[
	f(x) = \arctan x,
\]
和数列\(x_n = n\),
容易看出\(\{f(x_n)\}\)单调且收敛,
但是\(\{x_n\}\)不收敛.
\end{example}
