\section{导数的基本概念}
\subsection{导数的定义}
\begin{definition}
设函数\(y=f(x)\)在点\(x_0\)的某个邻域内有定义,
当自变量\(x\)在\(x_0\)处取得增量\(\increment x\)(点\(x_0+\increment x\)仍在该邻域内)时,
相应的函数取得增量\(\increment y = f(x_0 + \increment x) - f(x_0)\);
如果\(\increment y\)与\(\increment x\)之比当\(\increment x\to0\)时的极限存在,
则称“函数\(y=f(x)\)在点\(x_0\)处\DefineConcept{可导}%
(\(f\) is \emph{differentiable} at \(x_0\))”,
或“\(f(x)\)在点\(x_0\)具有导数”,
或“\(f(x)\)在点\(x_0\)导数存在”;
并称这个极限为“函数\(y=f(x)\)在点\(x_0\)处的导数”,
记为\(f'(x_0)\),即
\begin{equation}
	f'(x_0)
	= \lim_{\increment x\to0} \frac{\increment y}{\increment x}
	= \lim_{\increment x\to0} \frac{f(x_0+\increment x)-f(x_0)}{\increment x}
\end{equation}
也可记作
\[
	y'\eval_{x=x_0}, \qquad
	\dv{y}{x}\eval_{x=x_0}, \qquad
	\dv{f(x)}{x}\eval_{x=x_0}, \quad\text{或}\quad
	\left[\dv{x} f(x)\right]_{x=x_0}.
\]
%@see: https://mathworld.wolfram.com/Derivative.html
\end{definition}

导数的定义式可取不同的形式,如\begin{equation}
	f'(x_0) = \lim_{x \to x_0}\frac{f(x) - f(x_0)}{x - x_0}.
\end{equation}

\begin{definition}
称极限\[
	\lim_{h\to0^-} \frac{f(x_0+h)-f(x_0)}{h}
\]为“函数\(f(x)\)在点\(x_0\)处的\DefineConcept{左导数}(left-sided derivative)”,
记作\(f'_-(x_0)\).

称极限\[
	\lim_{h\to0^+} \frac{f(x_0+h)-f(x_0)}{h}
\]为“函数\(f(x)\)在点\(x_0\)处的\DefineConcept{右导数}(right-sided derivative)”,
记作\(f'_+(x_0)\).

左导数和右导数统称\DefineConcept{单侧导数}(one-sided derivative).
\end{definition}

\begin{theorem}[导数存在的充分必要条件]
函数\(f(x)\)在点\(x_0\)处可导的充分必要条件是%
其左导数\(f'_-(x_0)\)和右导数\(f'_+(x_0)\)都存在且相等.
\end{theorem}

\begin{definition}
如果函数\(y = f(x)\)在开区间\((a,b)\)内的每点处都可导,
就称“函数\(f(x)\)在开区间\((a,b)\)内可导”.

如果函数\(f(x)\)在开区间\((a,b)\)内可导,
且\(f'_+(a)\)及\(f'_-(b)\)都存在,
就说“函数\(f(x)\)在闭区间\([a,b]\)上可导”.
\end{definition}

\begin{definition}
设函数\(y = f(x)\)在区间\(I\)内的每点处都可导,
即对于\(\forall x_0 \in I\),都对应着\(f(x_0)\)的一个确定的导数值\(f'(x_0)\).
这样就构成一个新的函数,
这个函数叫做原来函数\(y = f(x)\)的\DefineConcept{导函数}(derivative function),
简称导数,
记作\[
	y', \qquad
	f'(x), \qquad
	\dv{y}{x}, \qquad
	\dv{f(x)}{x}
	\quad\text{或}\quad
	\dv{x} f(x).
\]
\end{definition}

\begin{definition}\label{definition:函数族.可导函数族}
由区间\(I\)上全部的可导函数组成的集合,称作\DefineConcept{可导函数族},
记作\(D(I)\)\footnote{当\(I=(a,b)\)时,可改写为\(D(a,b)\),以此类推.},
即\[
	D(I)
	\defeq
	\Set*{
		f\in\mathbb{R}^I
		\given
		(\forall x \in I)
		[\text{\(f\)在点\(x\)处可导}]
	}.
\]
\end{definition}

\begin{proposition}
设\(f\colon\mathbb{R}\to\mathbb{R}\)在点\(x\)可导,
则\[
	\lim_{h\to0} \frac{f(x)-f(x-h)}{h}
	= f'(x).
\]
\begin{proof}
令\(t=-h\),
则\[
	\lim_{h\to0^+} \frac{f(x)-f(x-h)}{h}
	= \lim_{t\to0^-} \frac{f(x)-f(x+t)}{-t}
	= \lim_{t\to0^-} \frac{f(x+t)-f(x)}{t},
\]\[
	\lim_{h\to0^-} \frac{f(x)-f(x-h)}{h}
	= \lim_{t\to0^+} \frac{f(x)-f(x+t)}{-t}
	= \lim_{t\to0^+} \frac{f(x+t)-f(x)}{t},
\]
又因为\[
	\lim_{t\to0^-} \frac{f(x+t)-f(x)}{t}
	= \lim_{t\to0^+} \frac{f(x+t)-f(x)}{t}
	= f'(x),
\]
所以\[
	\lim_{h\to0} \frac{f(x)-f(x-h)}{h}
	= \lim_{h\to0^+} \frac{f(x)-f(x-h)}{h}
	= \lim_{h\to0^-} \frac{f(x)-f(x-h)}{h}
	= f'(x).
	\qedhere
\]
\end{proof}
\end{proposition}

\begin{proposition}
设\(f\colon\mathbb{R}\to\mathbb{R}\)在点\(x\)可导,
\(\lambda>0\)是常数,
则\[
	\lim_{h\to0} \frac{f(x+\lambda h)-f(x)}{h}
	= \lambda f'(x).
\]
\begin{proof}
直接计算得\begin{align*}
	\lim_{h\to0} \frac{f(x+\lambda h)-f(x)}{h}
	&= \lambda \lim_{h\to0} \frac{f(x+\lambda h)-f(x)}{\lambda h} \\
	&\xlongequal{t=\lambda h}
		\lambda \lim_{t\to0} \frac{f(x+t)-f(x)}{t}
	= \lambda f'(x).
	\qedhere
\end{align*}
\end{proof}
\end{proposition}

\begin{example}
求函数\(f(x) = C\)的导数,其中\(C\)为常数.
\begin{solution}
\(f'(x)
= \lim_{h\to0}\frac{C-C}{h}
= 0\).
\end{solution}
\end{example}

\begin{example}
求函数\(f(x) = x^n\)在\(x=a\)处的导数,其中\(a\in\mathbb{R},
n\in\mathbb{N}^+\).
\begin{solution}
\(f'(a)
= \lim_{x \to a}\frac{x^n-a^n}{x-a}
= \lim_{x \to a}(x^{n-1}+ax^{n-2}+\dotsb+a^{n-1})
= na^{n-1}\).
\end{solution}
\end{example}

更一般地,对于幂函数\(y=x^{\mu}\ (\mu\in\mathbb{R})\),有\begin{equation}
	(x^{\mu})' = \mu x^{\mu-1}.
\end{equation}

\begin{example}
求函数\(f(x) = \sin x\)的导数.
\begin{solution}
\(f'(x) = \lim_{h\to0}\frac{\sin(x+h)-\sin x}{h}
= \lim_{h\to0}{\cos(x+\frac{h}{2}) \frac{\sin(h/2)}{h/2}}
= \cos x\).
\end{solution}
\end{example}

\begin{example}
求函数\(f(x) = a^x\ (a > 0 \land a \neq 1)\)的导数.
\begin{solution}
\(f'(x)
= \lim_{h\to0}\frac{a^{x+h}-a^x}{h}
= a^x \lim_{h\to0}\frac{a^h-1}{h}
= a^x \ln a\).
\end{solution}

特别地,当\(a=e\)时,因\(\ln e = 1\),故有\[
	(e^x)' = e^x.
\]
\end{example}

\begin{example}
求函数\(f(x) = \log_a x\ (a > 0 \land a \neq 1)\)的导数.
\begin{solution}
\(f'(x)
= \lim_{h\to0}\frac{\log_a(x+h)-\log_a x}{h}
= \lim_{h\to0}{\frac{1}{h} \log_a\frac{x+h}{x}}
= \frac{1}{x} \lim_{h\to0}\frac{\log_a(1+h/x)}{h/x}
= \frac{1}{x \ln a}\).
\end{solution}
\end{example}

\begin{example}
求函数\(f(x) = \abs{x}\)在\(x=0\)处的导数.
\begin{solution}
\(\frac{f(0+h)-f(0)}{h} = \frac{\abs{h}-0}{h} = \frac{\abs{h}}{h}\).

当\(h < 0\)时,\(\frac{\abs{h}}{h} = -1\),
故\(\lim_{h\to0^-}\frac{f(0+h)-f(0)}{h}
= \lim_{h\to0^-}\frac{\abs{h}}{h} = -1\).

当\(h > 0\)时,\(\frac{\abs{h}}{h} = 1\),
故\(\lim_{h\to0^+}\frac{f(0+h)-f(0)}{h}
= \lim_{h\to0^+}\frac{\abs{h}}{h} = 1\).

综上,\(\lim_{h\to0}\frac{f(0+h)-f(0)}{h}\)不存在,即函数\(f(x) = \abs{x}\)在\(x = 0\)处不可导.
\end{solution}
\end{example}

\subsection{导数的几何意义}
\begin{theorem}
曲线\(y=f(x)\)在点\(M(x_0,y_0)\)处的\DefineConcept{切线方程}为\[
	y-y_0=f'(x_0)(x-x_0).
\]

过切点\(M(x_0,y_0)\)且与切线垂直的直线叫做曲线\(y=f(x)\)在点\(M\)处的\DefineConcept{法线}.
如果\(f'(x_0) \neq 0\),则法线的斜率为\(-\frac{1}{f'(x_0)}\),从而法线方程为\[
	y-y_0=-\frac{1}{f'(x_0)}(x-x_0);
\]
而如果\(f'(x_0) = 0\),则法线方程为\(x = x_0\).
\end{theorem}

\subsection{函数可导性与连续性的关系}
\begin{theorem}\label{theorem:导数与微分.函数可导性与连续性的关系}
如果函数\(y = f(x)\)在点\(x_0\)处可导,
则函数在该点必连续.
\begin{proof}
因为函数\(y = f(x)\)在点\(x_0\)处可导,所以极限\[
	f'(x_0) = \lim_{x \to x_0}\frac{f(x)-f(x_0)}{x-x_0} = A
\]存在,
也就是说\[
	(\forall \epsilon_1 > 0)
	(\exists \delta_1 > 0)
	\left[
		0 < \abs{x-x_0} < \delta_1
		\implies
		\abs{\frac{f(x)-f(x_0)}{x-x_0} - A} < \epsilon_1
	\right].
\]

由\(\abs{\frac{f(x)-f(x_0)}{x-x_0} - A} < \epsilon_1\)
可得\[
	\abs{\abs{f(x)-f(x_0)} - \abs{A(x-x_0)}}
	\leq \abs{f(x)-f(x_0) - A(x-x_0)}
	< \epsilon_1 \abs{x-x_0},
\]\[
	\abs{f(x)-f(x_0)} < (\abs{A} + \epsilon_1) \abs{x-x_0},
\]
只要取\(\epsilon_2 = (\abs{A} + \epsilon_1) \abs{x-x_0}\)
即有\(\abs{f(x)-f(x_0)} < \epsilon_2\).
\end{proof}
\end{theorem}
借用\cref{definition:函数族.连续函数族} 和\cref{definition:函数族.可导函数族} 的记号,
可以将\cref{theorem:导数与微分.函数可导性与连续性的关系} 描述为:\[
	D(I) \subseteq C(I).
\]

但是,一个函数在某点连续却不一定在该点可导.
\begin{example}
函数\(y=f(x)=\sqrt[3]x\)
在区间\((-\infty,+\infty)\)内连续,
但是在点\(x=0\)处不可导.
这是因为在点\(x=0\)处有\[
	\frac{f(0+h)-f(0)}{h}
	=\frac{\sqrt[3]{h}-0}{h}
	=\frac{1}{h^{2/3}}>0,
\]
因而,
\(\lim_{h\to0} \frac{f(0+h)-f(0)}{h}
=\lim_{h\to0} \frac{1}{h^{2/3}}
=\infty\),
即导数为无穷大(导数不存在).
这事实在图形中表现为:
曲线\(y=\sqrt[3]x\)在原点具有垂直于\(x\)轴的切线\(x=0\).
\end{example}

\begin{example}
函数\(y=\sqrt{x^2}\)(即\(y=\abs{x}\))
在\((-\infty,+\infty)\)内连续,
但是在\(x=0\)处不可导,
曲线\(y=\sqrt{x^2}\)在原点没有切线.
这是因为\[
	\lim_{x\to0^+} \frac{f(x)-f(0)}{x-0}
	= \lim_{x\to0^+} \frac{x-0}{x-0}
	= 1,
\]
而\[
	\lim_{x\to0^-} \frac{f(x)-f(0)}{x-0}
	= \lim_{x\to0^-} \frac{(-x)-0}{x-0}
	= -1.
\]
\end{example}

\begin{remark}
函数在某一点连续是函数在该点可导的必要不充分条件.
\end{remark}

% \begin{example}
% 函数\[
% 	W(x) = \lim_{n\to\infty} \sum_{k=0}^n a^k \cos(b^k \pi x),
% \]称为\DefineConcept{魏尔斯特拉斯函数},
% 它的参数\(a\)和\(b\)满足\(0<a<1, b\in\Set{ p \given p = 2q+1, q\in\mathbb{N} }, ab > 1+\frac{3}{2}\pi\).
% 可以证明,魏尔斯特拉斯函数在定义域上处处连续而又处处不可导.
% \end{example}

\subsection{导函数的间断点}
\begin{example}
对函数\[
	f(x) = \left\{ \begin{array}{cl}
		x+1, & x>0, \\
		x-1, & x\leq0
	\end{array} \right.
\]求导得\[
	f'(x) = 1
	\quad(x\neq0).
\]
显然点\(x=0\)是导函数\(f'\)的可去间断点.
\end{example}

\begin{example}
对函数\[
	f(x) = \abs{x}
\]求导得\[
	f'(x) = \left\{ \begin{array}{cl}
		1, & x>0, \\
		-1, & x<0.
	\end{array} \right.
\]
显然点\(x=0\)是导函数\(f'\)的跳跃间断点.
\end{example}

\begin{example}
对函数\[
	f(x) = \frac1x
\]求导得\[
	f'(x) = -\frac1{x^2}.
\]
显然点\(x=0\)是导函数\(f'\)的无穷间断点.
\end{example}

\begin{example}
对函数\[
	f(x) = \left\{ \begin{array}{cl}
		x^2 \sin\frac1x, & x \neq 0, \\
		0, & x = 0
	\end{array} \right.
\]求导得\[
	f'(x) = \left\{ \begin{array}{cl}
		2x \sin\frac1x - \cos\frac1x, & x \neq 0, \\
		0, & x = 0.
	\end{array} \right.
\]
由于点\(x=0\)是函数\(\phi(x)=\cos\frac1x\)的振荡间断点,
而\(\lim_{x\to0} x \sin\frac1x = 0\),
所以点\(x=0\)也是导函数\(f'\)的振荡间断点.
\end{example}

\begin{proposition}
如果函数\(f\)在区间\((a,b)\)内可导,
那么\(f\)的导函数\(f'\)在\((a,b)\)内
不可能存在可去间断点、跳跃间断点和无穷间断点,
只可能存在振荡间断点.
%TODO 需要证明、例子
\end{proposition}
