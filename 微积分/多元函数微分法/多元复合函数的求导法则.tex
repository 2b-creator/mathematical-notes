\section{多元复合函数的求导法则}
\subsection{一元函数与多元函数复合的情形}
\begin{theorem}
如果一元函数\(u=\phi(t)\)及\(v=\psi(t)\)都在点\(t\)可导,二元函数\(z=f(u,v)\)在对应点\((u,v)\)具有连续偏导数,则复合得到的一元函数\[
z = z(t) = f[\phi(t),\psi(t)]
\]在点\(t\)可导,且有\[
\dv{z}{t} = \pdv{z}{u}\dv{u}{t} + \pdv{z}{v}\dv{v}{t}.
\]
像这样的导数\(\dv{z}{t}\)称为\DefineConcept{全导数}.
\begin{proof}
\def\D#1#2{\frac{\increment #1}{\increment #2}}
设\(t\)获得增量\(\increment t\),这时\(u=\phi(t)\)和\(v=\psi(t)\)的对应增量为\(\increment u\)、\(\increment v\),由此函数\(z=f(u,v)\)相应地获得增量\(\increment z\).
按假定,函数\(z=f(u,v)\)在点\((u,v)\)具有连续偏导数,这时函数的全增量\(\increment z\)可以表示为
\[
\increment z = \left(\pdv{z}{u} + \epsilon_1\right) \increment u + \left(\pdv{z}{v} + \epsilon_2\right) \increment v,
\]这里,当\(\increment u\to0\),\(\increment v\to0\)时,\(\epsilon_1\to0\),\(\epsilon_2\to0\).

将上式两边各除以\(\increment t\),得\[
\D{u}{t} = \left(\pdv{z}{u} + \epsilon_1\right) \D{u}{t} + \left(\pdv{z}{v} + \epsilon_2\right) \D{v}{t}.
\]当\(\increment t\to0\)时,\(\increment u\to0\),\(\increment v\to0\),\(\D{u}{t} \to \dv{t}{t}\),\(\D{v}{t} \to \dv{v}{t}\),所以\[
\dv{z}{t} = \lim_{\increment t\to0}{\D{z}{t}}
= \pdv{z}{u} \dv{u}{t} + \pdv{z}{v} \dv{v}{t}.
\qedhere
\]
\end{proof}
\end{theorem}
用同样的方法,可把上述定理推广到复合函数的中间变量多于两个的情形.
例如,设三元函数\(z=f(u,v,w)\)与一元函数\(u=\phi(t)\)、\(v=\psi(t)\)、\(w=\omega(t)\)复合得到的一元函数\[
z = z(t) = f[\phi(t),\psi(t),\omega(t)],
\]则在与上述定理相类似的条件下,这复合函数在点\(t\)可导,且有\[
\dv{z}{t} = \pdv{z}{u}\dv{u}{t} + \pdv{z}{v}\dv{v}{t} + \pdv{z}{w}\dv{w}{t}.
\]

\subsection{多元函数与多元函数复合的情形}
\begin{theorem}
如果二元函数\(u=\phi(x,y)\)及\(v=\psi(x,y)\)都在点\((x,y)\)具有对\(x\)及对\(y\)的偏导数,二元函数\(z=f(u,v)\)在对应点\((u,v)\)具有连续偏导数,则复合得到的二元函数\(z=f[\phi(x,y),\psi(x,y)]\)在点\((x,y)\)的两个偏导数都存在,且有\[
\pdv{z}{x} = \pdv{z}{u}\pdv{u}{x} + \pdv{z}{v}\pdv{v}{x},
\]\[
\pdv{z}{y} = \pdv{z}{u}\pdv{u}{y} + \pdv{z}{v}\pdv{v}{y}.
\]
\end{theorem}
类似地,设\(u=\phi(x,y)\)、\(v=\psi(x,y)\)及\(w=\omega(x,y)\)都在点\((x,y)\)具有对\(x\)及对\(y\)的偏导数,函数\(z=f(u,v,w)\)在对应点\((u,v,w)\)具有连续偏导数,则复合函数\[
z = f[\phi(x,y),\psi(x,y),\omega(x,y)]
\]在点\((x,y)\)的两个偏导数都存在,且可用下列公式计算:\[
\pdv{z}{x} = \pdv{z}{u}\pdv{u}{x} + \pdv{z}{v}\pdv{v}{x} + \pdv{z}{w}\pdv{w}{x},
\]\[
\pdv{z}{y} = \pdv{z}{u}\pdv{u}{y} + \pdv{z}{v}\pdv{v}{y} + \pdv{z}{w}\pdv{w}{y}.
\]

\subsection{其他情形}
\begin{theorem}
如果函数\(u=\phi(x,y)\)在点\((x,y)\)具有对\(x\)及对\(y\)的偏导数,函数\(v=\psi(y)\)在点\(y\)可导,函数\(z=f(u,v)\)在对应点\((u,v)\)具有连续偏导数,则复合函数\[
z = f[\phi(x,y),\psi(y)]
\]在点\((x,y)\)的两个偏导数都存在,且有\[
\pdv{z}{x} = \pdv{z}{u} \pdv{u}{x},
\]\[
\pdv{z}{y} = \pdv{z}{u} \pdv{u}{y} + \pdv{z}{v} \dv{v}{y}.
\]
\end{theorem}
上述情形实际上是情形2的一种特例,即在情形2中,如变量\(v\)与\(x\)无关,从而\(\pdv{v}{x}=0\);在\(v\)对\(y\)求导时,由于\(v=\psi(y)\)是一元函数,故\(\pdv{v}{y}\)换成了\(\dv{v}{y}\),这就得上述结果.

在情形3中,还会遇到这样的情形:复合函数的某些中间变量本身又是复合函数的自变量.
例如,设\(z = f(u,x,y)\)具有连续偏导数,而\(u=\phi(x,y)\)具有偏导数,则复合函数\[
z = z(x,y) = f[\phi(x,y),x,y]
\]可看做情形2中当\(v=x\)、\(w=y\)的特殊情形.因此\[
\pdv{v}{x} = 1, \qquad \pdv{w}{x} = 0,
\qquad
\pdv{v}{y} = 0, \qquad \pdv{w}{y} = 1.
\]从而复合函数\(z = f[\phi(x,y),x,y]\)具有对自变量\(x\)及\(y\)的偏导数,且有\[
\pdv{z}{x} = \pdv{f}{u} \pdv{u}{x} + \pdv{f}{x},
\qquad
\pdv{z}{y} = \pdv{f}{u} \pdv{u}{y} + \pdv{f}{y}.
\]
注意:这里\(\pdv{z}{x}\)和\(\pdv{f}{x}\)是不同的.
\(\pdv{z}{x}\)是把复合函数\(z = f[\phi(x,y),x,y]\)中的\(y\)看做不变而对\(x\)的偏导数,
\(\pdv{f}{x}\)是把\(f(u,x,y)\)中的\(u\)及\(y\)看做不变而对\(x\)的偏导数.
\(\pdv{z}{y}\)和\(\pdv{f}{y}\)也有类似的区别.
有时候为了更准确地区分上述两个概念,按中间变量在外层函数的顺序从1开始编号:\[
\pdv{f}{u} = f_1', \qquad
\pdv{f}{v} = f_2', \qquad
\pdv{f}{w} = f_3'.
\]

\begin{example}
设\(w = f(x+y+z,xyz)\),\(f\)具有二阶连续偏导数,求\(\pdv{w}{x}\)及\(\pdv[2]{w}{x}{z}\).
\begin{solution}
记\(u = x+y+z\),\(v = xyz\),则\(w = f(u,v)\).
那么根据复合函数求导法则,有\[
\pdv{w}{x} = \pdv{f}{u} \pdv{u}{x} + \pdv{f}{v} \pdv{v}{x}
= \pdv{f}{u} + yz \pdv{f}{v},
\]\[
\pdv[2]{w}{x}{z} = \pdv{z}(\pdv{f}{u} + yz \pdv{f}{v})
= \pdv[2]{f}{u}{z} + y\pdv{f}{v} + yz \pdv[2]{f}{v}{z}.
\]又因为\[
\pdv[2]{f}{u}{z}
= \pdv{u}(\pdv{f}{u}) \pdv{u}{z} + \pdv{v}(\pdv{f}{u}) \pdv{v}{z}
= \pdv[2]{f}{u} + xy \pdv[2]{f}{u}{v},
\]\[
\pdv[2]{f}{v}{z}
= \pdv{u}(\pdv{f}{v}) \pdv{u}{z} + \pdv{v}(\pdv{f}{v}) \pdv{v}{z}
= \pdv[2]{f}{v}{u} + xy \pdv[2]{f}{v},
\]于是\begin{align*}
\pdv[2]{w}{x}{z}
&= \pdv[2]{f}{u} + xy \pdv[2]{f}{u}{v}
 + y\pdv{f}{v} + yz \left( \pdv[2]{f}{v}{u} + xy \pdv[2]{f}{v} \right) \\
&= \pdv[2]{f}{u}
 + xy \pdv[2]{f}{u}{v} + yz \pdv[2]{f}{v}{u}
 + x y^2 z \pdv[2]{f}{v}
 + y \pdv{f}{v} \\
&= \pdv[2]{f}{u}
 + y(x+z) \pdv[2]{f}{u}{v}
 + x y^2 z \pdv[2]{f}{v}
 + y \pdv{f}{v}.
\end{align*}
\end{solution}
\end{example}

\subsection{全微分形式不变性}
\begin{theorem}[全微分形式不变性]
设函数\(z=f(u,v)\)具有连续偏导数,则有全微分\[
\dd{z}=\pdv{z}{u}\dd{u}+\pdv{z}{v}\dd{v}.
\]

如果\(u\)、\(v\)又是中间变量,即\(u=\phi(x,y)\)、\(v=\psi(x,y)\),且这两个函数也具有连续偏导数,则复合函数\(z=f[\phi(x,y),\psi(x,y)]=g(x,y)\)的全微分为\[
\dd{z}=\pdv{z}{x}\dd{x}+\pdv{z}{y}\dd{y}.
\]又由\begin{gather*}
\pdv{z}{x} = \pdv{z}{u}\pdv{u}{x} + \pdv{z}{v}\pdv{v}{x}, \\
\pdv{z}{y} = \pdv{z}{u}\pdv{u}{y} + \pdv{z}{v}\pdv{v}{y},
\end{gather*}可知\begin{align*}
\dd{z}
&=\left(\pdv{z}{u}\pdv{u}{x} + \pdv{z}{v}\pdv{v}{x}\right)\dd{x}
+\left(\pdv{z}{u}\pdv{u}{y} + \pdv{z}{v}\pdv{v}{y}\right)\dd{y} \\
&=\pdv{z}{u}\left(\pdv{u}{x}\dd{x}+\pdv{u}{y}\dd{y}\right)
+\pdv{z}{v}\left(\pdv{v}{x}\dd{x}+\pdv{v}{y}\dd{y}\right) \\
&=\pdv{z}{u}\dd{u}+\pdv{z}{v}\dd{v}.
\end{align*}
\end{theorem}
由此可见,无论\(u\)和\(v\)是自变量还是中间变量,函数\(z=f(u,v)\)的全微分形式是一样的.
这个性质叫做\DefineConcept{全微分形式不变性}.

\begin{example}
若函数\(f(x,y)\)满足对任意\(t>0\)有\(f(tx,ty)=t^n f(x,y)\),则称\(f(x,y)\)是\(n\)次\DefineConcept{齐次函数}.
证明:若\(f(x,y)\)可微,则\(f(x,y)\)是\(n\)次齐次函数的充分必要条件是\[
x \pdv{f}{x} + y \pdv{f}{y} = n f(x,y).
\]
\end{example}
