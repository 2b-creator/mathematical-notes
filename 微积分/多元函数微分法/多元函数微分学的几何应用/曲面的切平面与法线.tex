\subsection{曲面的切平面与法线}
我们先讨论由隐式给出的曲面方程 \labelcref{equation:解析几何.曲面的一般方程}
\[
	F(x,y,z) = 0
\]的情形,
然后把由显式给出的曲面方程\(z = f(x,y)\)作为它的特殊情形.

设曲面\(\Sigma\)由方程 \labelcref{equation:解析几何.曲面的一般方程} 给出,
\(M(x_0,y_0,z_0)\)是曲面\(\Sigma\)上的一点,
并设函数\(F(x,y,z)\)的偏导数在该点连续且不同时为零.
在曲面\(\Sigma\)上,通过点\(M\)任意引一条曲线\(\Gamma\),
假设曲线\(\Gamma\)的参数方程为
\begin{equation}\label{equation:多元函数微分学的几何应用.曲面上的曲线.中间步骤1}
	x = \phi(t),
	y = \psi(t),
	z = \omega(t)
	\quad (\alpha \leq t \leq \beta),
\end{equation}
\(t = t_0\)对应于点\(M(x_0,y_0,z_0)\),
且\(\phi'(t_0)\)、\(\psi'(t_0)\)、\(\omega'(t_0)\)不全为零,
则由\cref{equation:多元函数微分学的几何应用.曲线的切线方程}
可得这曲线的切线方程为\[
	\frac{x-x_0}{\phi'(t_0)}
	=\frac{y-y_0}{\psi'(t_0)}
	=\frac{z-z_0}{\omega'(t_0)}.
\]

我们现在要证明,
在曲面\(\Sigma\)上通过点\(M\)且在点\(M\)处具有切线的任何曲线,
它们在点\(M\)处的切线都在同一个平面上.
事实上,因为曲线\(\Gamma\)完全在曲面\(\Sigma\)上,
即\(\Gamma \subseteq \Sigma\),
所以有恒等式\[
	F[\phi(t),\psi(t),\omega(t)] = 0
	\quad(\alpha \leq t \leq \beta),
\]
又因\(F(x,y,z)\)在点\((x_0,y_0,z_0)\)处有连续偏导数,
且\(\phi'(t_0)\)、\(\psi'(t_0)\)、\(\omega'(t_0)\)都存在,
所以这恒等式左边的复合函数在\(t = t_0\)时有全导数,
且这全导数等于零,
即\[
	\eval{\dv{t} F[\phi(t),\psi(t),\omega(t)]}_{t=t_0} = 0,
\]
即有\begin{equation}\label{equation:多元函数微分学的几何应用.曲面上的曲线.中间步骤2}
	F'_x(x_0,y_0,z_0) \phi'(t_0)
	+ F'_y(x_0,y_0,z_0) \psi'(t_0)
	+ F'_z(x_0,y_0,z_0) \omega'(t_0)
	= 0.
\end{equation}

引入向量\begin{equation}
	\vb{n}=(F'_x(x_0,y_0,z_0),F'_y(x_0,y_0,z_0),F'_z(x_0,y_0,z_0)),
\end{equation}
则\cref{equation:多元函数微分学的几何应用.曲面上的曲线.中间步骤2} 表示
曲线 \labelcref{equation:多元函数微分学的几何应用.曲面上的曲线.中间步骤1}
在点\(M\)处的切向量\[
	\vb{T} = (\phi'(t_0),\psi'(t_0),\omega'(t_0))
\]与向量\(\vb{n}\)垂直.
因为曲线 \labelcref{equation:多元函数微分学的几何应用.曲面上的曲线.中间步骤1}
是曲面上通过点\(M\)的任意一条曲线,
它们在点\(M\)的切线都与同一个向量\(\vb{n}\)垂直,
所以曲面上通过点\(M\)的一切曲线在点\(M\)的切线都在同一个平面\(\tau\)上.
这个平面\(\tau\)就是曲面\(\Sigma\)在点\(M\)的切平面.
这切平面\(\tau\)的方程是
\begin{equation}\label{equation:多元函数微分学的几何应用.曲面的切平面方程}
	F'_x(x_0,y_0,z_0) (x-x_0)
	+ F'_y(x_0,y_0,z_0) (y-y_0)
	+ F'_z(x_0,y_0,z_0) (z-z_0)
	= 0.
\end{equation}
曲面\(\Sigma\)在点\(M(x_0,y_0,z_0)\)的法线的方程是
\begin{equation}\label{equation:多元函数微分学的几何应用.曲面的法线方程}
	\frac{x-x_0}{F'_x(x_0,y_0,z_0)}
	=\frac{y-y_0}{F'_y(x_0,y_0,z_0)}
	=\frac{z-z_0}{F'_z(x_0,y_0,z_0)}.
\end{equation}
我们知道,
垂直于曲面上切平面的向量称为曲面的法向量.
向量\(\vb{n}\)就是曲面\(\Sigma\)在点\(M\)处的一个法向量.

现在来考虑曲面方程
\begin{equation}\label{equation:多元函数微分学的几何应用.曲面上的曲线.中间步骤3}
	z = f(x,y)
\end{equation}
令\(F(x,y,z)=f(x,y)-z\),
得\(F(x,y,z)=0\),
可见\[
	F'_x(x,y,z) = f'_x(x,y),
	\qquad
	F'_y(x,y,z) = f'_y(x,y),
	\qquad
	F'_z(x,y,z) = -1.
\]
于是,当函数\(f(x,y)\)的偏导数\(f'_x(x,y)\)、\(f'_y(x,y)\)在点\((x_0,y_0)\)连续时,
曲面 \labelcref{equation:多元函数微分学的几何应用.曲面上的曲线.中间步骤3}
在点\(M(x_0,y_0,z_0)\)处的法向量为\[
	\vb{n}=(f'_x(x,y),f'_y(x,y),-1),
\]
切平面方程为\begin{equation}\label{equation:多元函数微分学的几何应用.曲面的切平面方程.变式1}
	z-z_0 = f'_x(x_0,y_0) (x-x_0) + f'_y(x_0,y_0) (y-y_0).
\end{equation}
而法线方程为\begin{equation}\label{equation:多元函数微分学的几何应用.曲面的法线方程.变式1}
	\frac{x-x_0}{f'_x(x_0,y_0)}
	=\frac{y-y_0}{f'_y(x_0,y_0)}
	=\frac{z-z_0}{-1}.
\end{equation}
如果用\(\alpha,\beta,\gamma\)表示
曲面 \labelcref{equation:多元函数微分学的几何应用.曲面上的曲线.中间步骤3} 的方向角,
并假定法向量与\(z\)轴的正向所成的角\(\gamma\)是锐角,
则法向量的方向余弦为\[
	\cos\alpha=\frac{-f'_x}{\sqrt{1+(f'_x)^2+(f'_y)^2}}, \qquad
	\cos\beta=\frac{-f'_y}{\sqrt{1+(f'_x)^2+(f'_y)^2}}, \qquad
	\cos\alpha=\frac{1}{\sqrt{1+(f'_x)^2+(f'_y)^2}}.
\]

\begin{example}
求曲面\[
	F(x,y,z) = 0
	\quad\text{和}\quad
	G(x,y,z) = 0
\]的交线\(\Gamma\)在\(xOy\)面上的投影的切线方程.
\begin{solution}
因为两曲面在点\(P_0(x_0,y_0,z_0)\)处的法向量分别是
\[
	\vb{n}_F
	= (F'_x(P_0),F'_y(P_0),F'_z(P_0))
	\quad\text{和}\quad
	\vb{n}_G
	= (G'_x(P_0),G'_y(P_0),G'_z(P_0)),
\]
故交线在点\(P_0\)处的切向量为\[
	\vb{\tau}
	= \vb{n}_F \times \vb{n}_G
	= \begin{vmatrix}
		\vb{i} & \vb{j} & \vb{k} \\
		F'_x & F'_y & F'_z \\
		G'_x & G'_y & G'_z
	\end{vmatrix}_{P_0}.
\]
交线在\(xOy\)面的投影曲线\(\Gamma_{xy}\)
在点\(P_0\)处对应的投影点\(P'_0(x_0,y_0)\)处的切向量\(\vb{\tau}_{xy}\)
与\(\vb{\tau}\)相比缺少\(z\)分量,
也就是说\[
	\vb{\tau}_{xy}
	= \left(
			\begin{vmatrix}
				F'_y & F'_z \\
				G'_y & G'_z
			\end{vmatrix}_{P_0},
			\begin{vmatrix}
				F'_z & F'_x \\
				G'_z & G'_x
			\end{vmatrix}_{P_0}
		\right).
\]
那么所求投影的切线方程为\[
	C:
	\frac{x - x_0}{\begin{vmatrix}
		F'_y & F'_z \\
		G'_y & G'_z
	\end{vmatrix}_{P_0}}
	= \frac{y - y_0}{\begin{vmatrix}
		F'_z & F'_x \\
		G'_z & G'_x
	\end{vmatrix}_{P_0}}.
\]
\end{solution}
\end{example}

\begin{example}
求曲面\(\Gamma: z = x + 2y + \ln(1+x^2+y^2)\)在点\((0,0,0)\)处的切平面方程.
\begin{solution}
记\(F(x,y,z) = x + 2y - z + \ln(1+x^2+y^2)\),
那么曲面\(\Gamma\)在点\((x,y,z)\)处的法向量为\[
	\vb{n}
	= (F'_x,F'_y,F'_z)
	= \left(
		1 + \frac{2x}{1+x^2+y^2},
		2 + \frac{2y}{1+x^2+y^2},
		-1
	\right),
\]
即在点\((0,0,0)\)处的法向量为\((1,2,-1)\),
从而切平面方程为\(x+2y-z=0\).
\end{solution}
\end{example}
