\section{多元函数微分学的几何应用}
\subsection{空间曲线的切线与法平面}
假设空间曲线\(\Gamma\)满足参数方程 \labelcref{equation:解析几何.曲线的参数方程}
\[
	\vb{r}
	= \vb{f}(t)
	= (x(t),y(t),z(t)),
	\quad \alpha \leq t \leq \beta.
\]
其中\(\vb{f}\)的分量函数\(x(t),y(t),z(t)\)在\([\alpha,\beta]\)上都可导,
且它们的导数\(x'(t),y'(t),z'(t)\)不同时为零.

设点\(M(x_0,y_0,z_0)\)在曲线\(\Gamma\)上,对应的参数为\(t_0\).
由向量值函数的导向量的几何意义知,
向量\begin{equation}
	\vb{T}
	= \vb{f}'(t_0)
	= (x'(t_0),y'(t_0),z'(t_0)).
\end{equation}
就是曲线\(\Gamma\)在点\(M\)处的一个切向量,
从而曲线\(\Gamma\)在点\(M\)处的切线方程为
\begin{equation}
	\begin{bmatrix}
		x \\
		y \\
		z
	\end{bmatrix}
	= \lambda \begin{bmatrix}
		x'(t_0) \\
		y'(t_0) \\
		z'(t_0)
	\end{bmatrix}
	+ \begin{bmatrix}
		x_0 \\
		y_0 \\
		z_0
	\end{bmatrix},
	\quad \lambda\in\mathbb{R},
\end{equation}
或
\begin{equation}\label{equation:多元函数微分学的几何应用.曲线的切线方程}
	\frac{x-x_0}{x'(t_0)}
	=\frac{y-y_0}{y'(t_0)}
	=\frac{z-z_0}{z'(t_0)}.
\end{equation}

我们知道,通过点\(M\)且与切线垂直的平面,就是曲线\(\Gamma\)在点\(M\)处的法平面.
它是通过点\(M(x_0,y_0,z_0)\)且以\(\vb{T}=\vb{f}'(t_0)\)为法向量的平面,
因此法平面方程为\begin{equation}
	\vb{T}\cdot(x-x_0,y-y_0,z-z_0)=0
\end{equation}
或\begin{equation}
	x'(t_0) \cdot (x-x_0) + y'(t_0) \cdot (y-y_0) + z'(t_0) \cdot (z-z_0) = 0.
\end{equation}

若空间曲线\(\Gamma\)的方程以\[
	\left\{ \begin{array}{l}
		y = \phi(x), \\
		z = \psi(x)
	\end{array} \right.
\]的形式给出,
取\(x\)为参数,
它就可以表为参数方程的形式\[
	\left\{ \begin{array}{l}
		x = x, \\
		y = \phi(x), \\
		z = \psi(x).
	\end{array} \right.
\]
若\(\phi(x),\psi(x)\)都在\(x=x_0\)处可导,
那么根据上面的讨论可知,\[
	\vb{T} = (1,\phi'(x_0),\psi'(x_0)),
\]
因此曲线\(\Gamma\)在点\(M(x_0,y_0,z_0)\)处的切线方程为
\begin{equation}\label{equation:多元函数微分学的几何应用.曲线的切线方程.变式1}
	\frac{x-x_0}{1}
	=\frac{y-y_0}{\phi'(x_0)}
	=\frac{z-z_0}{\psi'(x_0)}.
\end{equation}
曲线\(\Gamma\)在点\(M(x_0,y_0,z_0)\)处的法平面方程为
\begin{equation}\label{equation:多元函数微分学的几何应用.曲线的法平面方程.变式1}
	(x-x_0) + \phi'(x_0) \cdot (y-y_0) + \psi'(x_0) \cdot (z-z_0) = 0
\end{equation}

假设空间曲线\(\Gamma\)满足一般方程 \labelcref{equation:解析几何.曲线的一般方程}
\[
	\left\{ \begin{array}{l}
		F(x,y,z) = 0, \\
		G(x,y,z) = 0,
	\end{array} \right.
\]
\(M(x_0,y_0,z_0)\)是曲线\(\Gamma\)上的一个点.
又设\(F\)、\(G\)有对各个变量的连续偏导数,
且\[
	\eval{\jacobi{F,G}{y,z}}_{(x_0,y_0,z_0)} \neq 0.
\]
这时方程组 \labelcref{equation:解析几何.曲线的一般方程}
在点\(M(x_0,y_0,z_0)\)的某一邻域内确定了一组函数\(y=\phi(x)\)和\(z=\psi(x)\).
要求曲线\(\Gamma\)在点\(M\)处的切线方程和法平面方程,
只要求出\(\phi'(x_0)\)、\(\psi'(x_0)\),
然后代入\cref{equation:多元函数微分学的几何应用.曲线的切线方程.变式1,equation:多元函数微分学的几何应用.曲线的法平面方程.变式1} 就行了.
为此,我们在恒等式\[
	F[x,\phi(x),\psi(x)] = 0,
	\quad\text{和}\quad
	G[x,\phi(x),\psi(x)] = 0
\]两边分别对\(x\)求全导数,
得\[
	\left\{ \def\arraystretch{1.5} \begin{array}{l}
		\pdv{F}{x} + \pdv{F}{y} \dv{y}{x} + \pdv{F}{z} \dv{z}{x} = 0, \\
		\pdv{G}{x} + \pdv{G}{y} \dv{y}{x} + \pdv{G}{z} \dv{z}{x} = 0.
	\end{array} \right.
\]
由假设可知,
在点\(M\)的某个邻域内\[
	J = \jacobi{F,G}{y,z} \neq 0,
\]
故可解得\[
	\dv{y}{x} = \frac{1}{J} \jacobi{F,G}{z,x},
	\qquad
	\dv{z}{x} = \frac{1}{J} \jacobi{F,G}{x,y}.
\]

于是\(\vb{T} = (1,\phi'(x_0),\psi'(x_0))\)是曲线\(\Gamma\)在点\(M\)处的一个切向量,
这里\[
	\phi'(x_0)
	= \frac{
			\begin{vmatrix}
				F'_z & F'_x \\
				G'_z & G'_x
			\end{vmatrix}_M
		}{
			\begin{vmatrix}
				F'_y & F'_z \\
				G'_y & G'_z
			\end{vmatrix}_M
		},
	\qquad
	\psi'(x_0)
	= \frac{
			\begin{vmatrix}
				F'_x & F'_y \\
				G'_x & G'_y
			\end{vmatrix}_M
		}{
			\begin{vmatrix}
				F'_y & F'_z \\
				G'_y & G'_z
			\end{vmatrix}_M
		},
\]
分子分母中带下标\(M\)的行列式表达行列式在点\(M(x_0,y_0,z_0)\)的值.
把上面的切向量\(\vb{T}\)乘以\(J\),得\begin{align*}
	\vb{T}_1
	&= \left(
		\begin{vmatrix}
			F'_y & F'_z \\
			G'_y & G'_z \\
		\end{vmatrix}_M,
		\begin{vmatrix}
			F'_z & F'_x \\
			G'_z & G'_x \\
		\end{vmatrix}_M,
		\begin{vmatrix}
			F'_x & F'_y \\
			G'_x & G'_y \\
		\end{vmatrix}_M
	\right) \\
	&= \begin{vmatrix}
		\vb{i} & \vb{j} & \vb{k} \\
		F'_x & F'_y & F'_z \\
		G'_x & G'_y & G'_z
	\end{vmatrix}
	,
\end{align*}
这也是曲线\(\Gamma\)在点\(M\)处的一个切向量.
由此可以写出曲线\(\Gamma\)在点\(M(x_0,y_0,z_0)\)处的切线方程为
\begin{equation}\label{equation:多元函数微分学的几何应用.曲线的切线方程.变式2}
	\frac{x-x_0}{\begin{vmatrix}
		F'_y & F'_z \\
		G'_y & G'_z \\
	\end{vmatrix}_M}
	=\frac{y-y_0}{\begin{vmatrix}
		F'_z & F'_x \\
		G'_z & G'_x \\
	\end{vmatrix}_M}
	=\frac{z-z_0}{\begin{vmatrix}
		F'_x & F'_y \\
		G'_x & G'_y \\
	\end{vmatrix}_M}.
\end{equation}
曲线\(\Gamma\)在点\(M(x_0,y_0,z_0)\)处的法平面方程为
\begin{equation}\label{equation:多元函数微分学的几何应用.曲线的法平面方程.变式2}
	\begin{vmatrix}
		F'_y & F'_z \\
		G'_y & G'_z \\
	\end{vmatrix}_M
	(x-x_0)
	+ \begin{vmatrix}
		F'_z & F'_x \\
		G'_z & G'_x \\
	\end{vmatrix}_M
	(y-y_0)
	+ \begin{vmatrix}
		F'_x & F'_y \\
		G'_x & G'_y \\
	\end{vmatrix}_M
	(z-z_0)
	= 0.
\end{equation}
如果\(\eval{\jacobi{F,G}{y,z}}_M = 0\),
而\(\eval{\jacobi{F,G}{z,x}}_M\)和\(\eval{\jacobi{F,G}{x,y}}_M\)中至少有一个不等于零,
我们可得同样的结果.

\subsection{曲面的切平面与法线}
我们先讨论由隐式给出的曲面方程 \labelcref{equation:解析几何.曲面的一般方程}
\[
	F(x,y,z) = 0
\]的情形,
然后把由显式给出的曲面方程\(z = f(x,y)\)作为它的特殊情形.

设曲面\(\Sigma\)由方程 \labelcref{equation:解析几何.曲面的一般方程} 给出,
\(M(x_0,y_0,z_0)\)是曲面\(\Sigma\)上的一点,
并设函数\(F(x,y,z)\)的偏导数在该点连续且不同时为零.
在曲面\(\Sigma\)上,通过点\(M\)任意引一条曲线\(\Gamma\),
假设曲线\(\Gamma\)的参数方程为
\begin{equation}\label{equation:多元函数微分学的几何应用.曲面上的曲线.中间步骤1}
	x = \phi(t),
	y = \psi(t),
	z = \omega(t)
	\quad (\alpha \leq t \leq \beta),
\end{equation}
\(t = t_0\)对应于点\(M(x_0,y_0,z_0)\),
且\(\phi'(t_0)\)、\(\psi'(t_0)\)、\(\omega'(t_0)\)不全为零,
则由\cref{equation:多元函数微分学的几何应用.曲线的切线方程}
可得这曲线的切线方程为\[
	\frac{x-x_0}{\phi'(t_0)}
	=\frac{y-y_0}{\psi'(t_0)}
	=\frac{z-z_0}{\omega'(t_0)}.
\]

我们现在要证明,
在曲面\(\Sigma\)上通过点\(M\)且在点\(M\)处具有切线的任何曲线,
它们在点\(M\)处的切线都在同一个平面上.
事实上,因为曲线\(\Gamma\)完全在曲面\(\Sigma\)上,
即\(\Gamma \subseteq \Sigma\),
所以有恒等式\[
	F[\phi(t),\psi(t),\omega(t)] = 0
	\quad(\alpha \leq t \leq \beta),
\]
又因\(F(x,y,z)\)在点\((x_0,y_0,z_0)\)处有连续偏导数,
且\(\phi'(t_0)\)、\(\psi'(t_0)\)、\(\omega'(t_0)\)都存在,
所以这恒等式左边的复合函数在\(t = t_0\)时有全导数,
且这全导数等于零,
即\[
	\eval{\dv{t} F[\phi(t),\psi(t),\omega(t)]}_{t=t_0} = 0,
\]
即有\begin{equation}\label{equation:多元函数微分学的几何应用.曲面上的曲线.中间步骤2}
	F'_x(x_0,y_0,z_0) \phi'(t_0)
	+ F'_y(x_0,y_0,z_0) \psi'(t_0)
	+ F'_z(x_0,y_0,z_0) \omega'(t_0)
	= 0.
\end{equation}

引入向量\begin{equation}
	\vb{n}=(F'_x(x_0,y_0,z_0),F'_y(x_0,y_0,z_0),F'_z(x_0,y_0,z_0)),
\end{equation}
则\cref{equation:多元函数微分学的几何应用.曲面上的曲线.中间步骤2} 表示
曲线 \labelcref{equation:多元函数微分学的几何应用.曲面上的曲线.中间步骤1}
在点\(M\)处的切向量\[
	\vb{T} = (\phi'(t_0),\psi'(t_0),\omega'(t_0))
\]与向量\(\vb{n}\)垂直.
因为曲线 \labelcref{equation:多元函数微分学的几何应用.曲面上的曲线.中间步骤1}
是曲面上通过点\(M\)的任意一条曲线,
它们在点\(M\)的切线都与同一个向量\(\vb{n}\)垂直,
所以曲面上通过点\(M\)的一切曲线在点\(M\)的切线都在同一个平面\(\tau\)上.
这个平面\(\tau\)就是曲面\(\Sigma\)在点\(M\)的切平面.
这切平面\(\tau\)的方程是\[
	F'_x(x_0,y_0,z_0) (x-x_0)
	+ F'_y(x_0,y_0,z_0) (y-y_0)
	+ F'_z(x_0,y_0,z_0) (z-z_0)
	= 0.
\]或\begin{gather}
	\vb{n}\cdot(x-x_0,y-y_0,z-z_0)=0.
	\tag{18}
\end{gather}

通过点\(M(x_0,y_0,z_0)\)且垂直于切平面(18)的直线称为曲面在该点的法线.
法线方程是\begin{gather}
	\frac{x-x_0}{F'_x(x_0,y_0,z_0)}
	=\frac{y-y_0}{F'_y(x_0,y_0,z_0)}
	=\frac{z-z_0}{F'_z(x_0,y_0,z_0)}.
	\tag{19}
\end{gather}

垂直于曲面上切平面的向量称为曲面的法向量.
向量\(\vb{n}\)就是曲面\(\Sigma\)在点\(M\)处的一个法向量.


现在来考虑曲面方程\begin{gather}
	z = f(x,y)
	\tag{20}
\end{gather}
令\(F(x,y,z)=f(x,y)-z=0\),可见\[
	F'_x(x,y,z) = f'_x(x,y),
	\qquad
	F'_y(x,y,z) = f'_y(x,y),
	\qquad
	F'_z(x,y,z) = -1.
\]
于是,当函数\(f(x,y)\)的偏导数\(f'_x(x,y)\)、\(f'_y(x,y)\)在点\((x_0,y_0)\)连续时,
曲面(20)在点\(M(x_0,y_0,z_0)\)处的法向量为\[
	\vb{n}=\opair{f'_x(x,y),f'_y(x,y),-1},
\]
切平面方程为\[
	\vb{n}\cdot(x-x_0,y-y_0,z-z_0)=0,
\]或\begin{gather}
	z-z_0 = f'_x(x_0,y_0) (x-x_0) + f'_y(x_0,y_0) (y-y_0).
	\tag{21}
\end{gather}
而法线方程为\begin{gather}
	\frac{x-x_0}{f'_x(x_0,y_0)}
	=\frac{y-y_0}{f'_y(x_0,y_0)}
	=\frac{z-z_0}{-1}.
	\tag{22}
\end{gather}
这里顺便指出,
方程(21)右端恰好是函数\(z = f(x,y)\)在点\((x_0,y_0)\)的全微分,
而左端是切平面上点的竖坐标的增量.
因此,函数\(z = f(x,y)\)在点\((x_0,y_0)\)的全微分,
在集合上表示曲面\(z = f(x,y)\)在点\(M\)出的切平面上点的竖坐标的增量.

\begin{example}
求曲面\[
	F(x,y,z) = 0
	\quad\text{和}\quad
	G(x,y,z) = 0
\]的交线\(\Gamma\)在\(xOy\)面上的投影的切线方程.
\begin{solution}
因为两曲面在点\(P_0(x_0,y_0,z_0)\)处的法向量分别是
\[
	\vb{n}_F
	= (F'_x(P_0),F'_y(P_0),F'_z(P_0))
	\quad\text{和}\quad
	\vb{n}_G
	= (G'_x(P_0),G'_y(P_0),G'_z(P_0)),
\]
故交线在点\(P_0\)处的切向量为\[
	\vb{\tau}
	= \vb{n}_F \times \vb{n}_G
	= \begin{vmatrix}
		\vb{i} & \vb{j} & \vb{k} \\
		F'_x & F'_y & F'_z \\
		G'_x & G'_y & G'_z
	\end{vmatrix}_{P_0}.
\]
交线在\(xOy\)面的投影曲线\(\Gamma_{xy}\)
在点\(P_0\)处对应的投影点\(P'_0(x_0,y_0)\)处的切向量\(\vb{\tau}_{xy}\)
与\(\vb{\tau}\)相比缺少\(z\)分量,
也就是说\[
	\vb{\tau}_{xy}
	= \left(
			\begin{vmatrix}
				F'_y & F'_z \\
				G'_y & G'_z
			\end{vmatrix}_{P_0},
			\begin{vmatrix}
				F'_z & F'_x \\
				G'_z & G'_x
			\end{vmatrix}_{P_0}
		\right).
\]
那么所求投影的切线方程为\[
	C:
	\frac{x - x_0}{\begin{vmatrix}
		F'_y & F'_z \\
		G'_y & G'_z
	\end{vmatrix}_{P_0}}
	= \frac{y - y_0}{\begin{vmatrix}
		F'_z & F'_x \\
		G'_z & G'_x
	\end{vmatrix}_{P_0}}.
\]
\end{solution}
\end{example}

\begin{example}
求曲面\(\Gamma: z = x + 2y + \ln(1+x^2+y^2)\)在点\((0,0,0)\)处的切平面方程.
\begin{solution}
记\(F(x,y,z) = x + 2y - z + \ln(1+x^2+y^2)\),
那么曲面\(\Gamma\)在点\((x,y,z)\)处的法向量为\[
	\vb{n}
	= (F'_x,F'_y,F'_z)
	= \left(
		1 + \frac{2x}{1+x^2+y^2},
		2 + \frac{2y}{1+x^2+y^2},
		-1
	\right),
\]
即在点\((0,0,0)\)处的法向量为\((1,2,-1)\),
从而切平面方程为\(x+2y-z=0\).
\end{solution}
\end{example}

\subsection{空间曲线的曲率}
我们在\cref{section:微分中值定理.曲率}中研究平面曲线时引入了“曲率”这样一个概念,
现在我们将它推广到空间曲线.

设空间曲线\(\Gamma: \vb{r} = \vb{r}(t)\ (\alpha \leq t \leq \beta)\)是一段光滑曲线.
当参数\(t = t_0\)时,曲线\(\Gamma\)上动点\(P\)的坐标为\(\vb{r}\),
而点\(P\)处的切向量为\(\vb{r}'(t_0)\)\footnote{向量\(\vb{r}'(t_0)\)不一定是单位向量!}.
记向量\(\vb{r}'(t_0)\)与\(\vb{r}'(t_0+\increment t)\)之间的夹角为\(\increment\theta\),
记\(\vb{r}(t_0)\)与\(\vb{r}(t_0+\increment t)\)之间的弧长为\(\increment s\).如果极限\[
	\lim_{\increment s \to 0} \abs{ \frac{\increment\theta}{\increment s} }
\]存在,
就称此极限为曲线\(\Gamma\)在点\(P\)处的\DefineConcept{曲率},
记为\(K(t_0)\),
即\[
	K(t_0)
	= \lim_{\increment s \to 0} \abs{ \frac{\increment\theta}{\increment s} }.
\]
因此\begin{align*}
	\norm{ \dv{\vb{\tau}}{s} }
	&= \norm{
			\lim_{\increment s \to 0}
				\frac{\vb{\tau}(s+\increment s) - \vb{\tau}(s)}{\increment s}
		} \\
	&= \lim_{\increment s \to 0}
		\frac{\norm{\vb{\tau}(s+\increment s) - \vb{\tau}(s)}}{\abs{\increment s}} \\
	&= \lim_{\increment s \to 0}
		\abs{ \frac{2 \sin(\increment\theta / 2)}{\increment s} } \\
	&= \lim_{\increment s \to 0}
		\left(
			\abs{ \frac{2 \sin(\increment\theta / 2)}{\increment\theta/2} }
			\cdot \abs{ \frac{\increment\theta}{\increment s} }
		\right) \\
	&= \lim_{\increment s \to 0}
		\abs{ \frac{\increment\theta}{\increment s} }.
\end{align*}
又因为\(\vb{\tau} = \dv{\vb{r}}{s}\),所以\(\norm{ \dv{\vb{\tau}}{s} } = \norm{ \dv{s}( \dv{\vb{r}}{s} ) } = \norm{ \dv[2]{\vb{r}}{s} }\),那么上式又可写作\begin{equation}
\norm{ \dv[2]{\vb{r}}{s} }
= \lim_{\increment s \to 0} \abs{ \frac{\increment\theta}{\increment s} }.
\end{equation}

假设空间曲线\(\Gamma\)由参数方程\(\vb{r} = \vb{r}(t)\)给出.由于位矢\(\vb{r}\)对弧长\(s\)的一阶、二阶导数分别为\[
\dv{\vb{r}}{s} = \dv{\vb{r}}{t} \dv{t}{s},
\qquad
\dv[2]{\vb{r}}{s} = \dv[2]{\vb{r}}{t} \left(\dv{t}{s}\right)^2 + \dv{\vb{r}}{t} \dv[2]{t}{s}.
\]将以上两式对应的两边分别作向量积,得\[
\dv{\vb{r}}{s} \times \dv[2]{\vb{r}}{s}
= \left(\dv{\vb{r}}{t} \times \dv[2]{\vb{r}}{t}\right) \cdot \left(\dv{t}{s}\right)^3.
\]由于\(\dv{\vb{r}}{s}\)是单位向量,所以\(\dv{\vb{r}}{s} \cdot \dv{\vb{r}}{s} = 1\),再对\(s\)求导,得\[
\dv{s}( \dv{\vb{r}}{s} \cdot \dv{\vb{r}}{s} )
= \dv[2]{\vb{r}}{s} \cdot \dv{\vb{r}}{s} + \dv{\vb{r}}{s} \cdot \dv[2]{\vb{r}}{s}
= 2 \left( \dv{\vb{r}}{s} \cdot \dv[2]{\vb{r}}{s} \right)
= 0,
\]也就是说向量\(\dv{\vb{r}}{s}\)与\(\dv[2]{\vb{r}}{s}\)互相正交,所以曲线\(\Gamma\)的曲率为\[
K(t) = \norm{ \dv[2]{\vb{r}}{s} }
= \norm{ \dv{\vb{r}}{s} \times \dv[2]{\vb{r}}{s} }
= \norm{ \dv{\vb{r}}{t} \times \dv[2]{\vb{r}}{t} } \cdot \abs{ \dv{t}{s} }^3.
\]但因\(\norm{ \dd{\vb{r}} } = \abs{ \dd{s} }\),\(\abs{ \dv{s}{t} } = \norm{ \dv{\vb{r}}{t} }\),所以\[
\abs{ \dv{t}{s} }^3 = \norm{ \dv{\vb{r}}{t} }^{-3}.
\]由此得出曲率为\begin{equation}
K(t) = \frac{ \norm{ \vb{r}'_t \times \vb{r}''_t } }{ \norm{ \vb{r}'_t }^3 }.
\end{equation}
