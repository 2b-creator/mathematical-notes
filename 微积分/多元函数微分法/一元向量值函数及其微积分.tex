\section{一元向量值函数及其微积分}
我们先在本节介绍一元向量值函数及其导数,
再在下一节讨论多元函数微分学的几何应用.

\subsection{向量值函数}
假设空间曲线\(\Gamma\)满足参数方程 \labelcref{equation:解析几何.曲线的参数方程}.
我们可以把方程 \labelcref{equation:解析几何.曲线的参数方程} 改写成向量形式.
若记\[
	\vb{r}=x\vb{i}+y\vb{j}+z\vb{k},
	\qquad
	\vb{f}(t)=x(t)\vb{i}+y(t)\vb{j}+z(t)\vb{k},
\]
则方程 \labelcref{equation:解析几何.曲线的参数方程}
就成为向量方程\begin{equation}\label{equation:解析几何.曲面的参数方程.向量形式}
	\vb{r}=\vb{f}(t),
	\quad \alpha \leq t \leq \beta.
\end{equation}

方程 \labelcref{equation:解析几何.曲面的参数方程.向量形式}
确定了一个映射\(\vb{f}\colon [\alpha,\beta]\to\mathbb{R}^3\).
由于这个映射将每一个\(t\in[\alpha,\beta]\)映成一个向量\(\vb{f}(t)\in\mathbb{R}^3\),
故称这映射为一元向量值函数.
一般地,有如下定义:
\begin{definition}
设数集\(D \subseteq \mathbb{R}\),
\(\vb{e}_1,\vb{e}_2,\dotsc,\vb{e}_n\)是\(\mathbb{R}^n\)的一组基.
称映射\(\vb{f}\colon D\to\mathbb{R}^n\)为\DefineConcept{一元向量值函数},
通常记为\begin{equation}\label{equation:一元实变向量值函数.向量形式}
	\vb{r}=\vb{f}(t),
	\quad t \in D,
\end{equation}
或\begin{equation}
	\vb{r}=f_1(t)\vb{e}_1+f_2(t)\vb{e}_2+\dotsb+f_n(t)\vb{e}_n,
	\quad t \in D,
\end{equation}
或\begin{equation}
	\vb{r}=(f_1(t),f_2(t),\dotsc,f_n(t)),
	\quad t \in D,
\end{equation}
其中数集\(D\)称为函数的\DefineConcept{定义域},
\(t\)称为\DefineConcept{自变量},
\(\vb{r}\)称为\DefineConcept{因变量},
\(f_n(t)\)称为\DefineConcept{分量函数}(component function).
\end{definition}
一元向量值函数是普通一元函数的推广.
现在,自变量\(t\)依然取实数值,但因变量\(\vb{r}\)不取实数值,而取值为\(n\)维向量.

为简单起见,以下将一元向量值函数简称为向量值函数,并把普通的实值函数称为数量函数.

假设变向量\(\vb{r}\)的起点取在坐标系的原点\(O\)处,终点在\(M\)处,即\(\vb{r}=\vec{OM}\).
当\(t\)改变时,\(\vb{r}\)跟着改变,从而终点\(M\)也随之改变.
终点\(M\)的轨迹(记作\(\Gamma\))
称为“向量值函数\(\vb{r}=\vb{f}(t)\ (t \in D)\)的\DefineConcept{终端曲线}”,
曲线\(\Gamma\)也称为“向量值函数\(\vb{r}=\vb{f}(t)\ (t \in D)\)的\DefineConcept{图形}”.

由于向量值函数\(\vb{r}=\vb{f}(t)\ (t \in D)\)与空间曲线\(\Gamma\)是一一对应的,
因此\(\vb{r}=\vb{f}(t)\ (t \in D)\)称为“曲线\(\Gamma\)的\DefineConcept{向量方程}”.

\subsection{向量值函数的极限}
根据\(\mathbb{R}^3\)中的向量的线性运算及向量的模的概念,
可以类似于定义数量函数的极限、连续、导数等概念的形式来定义向量值函数的相应概念:
\begin{definition}
设向量值函数\(\vb{f}(t)\)在点\(t_0\)的某一去心邻域内有定义,
如果存在一个常向量\(\vb{r}_0\),
对于任意给定的正数\(\epsilon\),
总存在正数\(\delta\),
使得当\(t\)满足\(0 < \abs{t-t_0} < \delta\)时,
对应的函数值\(\vb{f}(t)\)都满足不等式\[
	\norm{\vb{f}(t)-\vb{r}_0} < \epsilon,
\]
那么,常向量\(\vb{r}_0\)就叫做
“向量值函数\(\vb{f}(t)\)当\(t \to t_0\)时的\DefineConcept{极限}”,
记作\[
	\lim_{t \to t_0} \vb{f}(t) = \vb{r}_0,
	\quad\text{或}\quad
	\vb{f}(t) \to \vb{r}_0\ (t \to t_0).
\]
\end{definition}

\begin{theorem}
向量值函数\(\vb{f}(t)\)当\(t \to t_0\)时的极限存在的充分必要条件是:
\(\vb{f}(t)\)的全部分量函数\(f_1(t),f_2(t),\dotsc,f_n(t)\)
当\(t \to t_0\)的极限都存在.
\end{theorem}

\begin{theorem}
在函数\(\vb{f}(t)\)当\(t \to t_0\)时的极限存在时,
其极限为\begin{equation}\label{equation:一元实变向量值函数.向量值函数的极限与分量函数的极限的关系}
	\lim_{t \to t_0} \vb{f}(t)
	= \left(
			\lim_{t \to t_0} f_1(t),
			\lim_{t \to t_0} f_2(t),
			\dotsc,
			\lim_{t \to t_0} f_n(t)
		\right).
\end{equation}
\end{theorem}

\subsection{向量值函数的连续性}
\begin{definition}[向量值函数的连续性]
设向量值函数\(\vb{f}(t)\)在点\(t_0\)的某一邻域内有定义,
若\[
	\lim_{t \to t_0} \vb{f}(t) = \vb{f}(t_0),
\]
则称“向量值函数\(\vb{f}(t)\)在\(t_0\) \DefineConcept{连续}”.

若\(\vb{f}(t)\)在区域\(D\)中的每一点处都连续,
则称\(\vb{f}(t)\)在\(D\)上连续,
并称\(\vb{f}(t)\)是\(D\)上的\DefineConcept{连续函数}.
\end{definition}

\begin{theorem}
向量值函数\(\vb{f}(t)\)在\(t_0\)连续的充分必要条件为:
\(\vb{f}(t)\)的全部分量函数\[
	f_1(t),f_2(t),\dotsc,f_n(t)
\]都在\(t_0\)连续.
\end{theorem}

我们可以把一元向量值函数连续性的概念推广到变换上.
\begin{definition}
设区域\(D \subseteq \mathbb{R}^n\),
变换\(\vb{F}\colon D \to \mathbb{R}^m\).
如果\[
	\lim_{\substack{
		\vb{x}\to\vb{a} \\
		\vb{x} \in D
	}} \vb{F}(\vb{x})
	= \vb{F}(\vb{a}),
\]
则称“变换\(\vb{F}(\vb{x})\)在\(\vb{a}\) \DefineConcept{连续}”.
\end{definition}

\begin{theorem}
变换\(\vb{F}(\vb{x})\)在\(\vb{a}\)连续的充分必要条件为:
\(\vb{F}(\vb{x})\)的全部分量函数\[
	f_1(\vb{x}),f_2(\vb{x}),\dotsc,f_n(\vb{x})
\]都在\(\vb{a}\)连续.
\end{theorem}

\subsection{向量值函数的导数}
\begin{definition}
设向量值函数\(\vb{r}=\vb{f}(t)\)在点\(t_0\)的某一邻域内有定义,
如果\[
	\lim_{\increment t\to0}
		\frac{\increment \vb{r}}{\increment t}
	= \lim_{\increment t\to0}
		\frac{\vb{f}(t_0+\increment t)-\vb{f}(t_0)}{\increment t}
\]存在,
那么就称这个极限向量为
“向量值函数\(\vb{r}=\vb{f}(t)\)在\(t_0\)处的\DefineConcept{导数}
或\DefineConcept{导向量}”,
记作\[
	\vb{f}'(t_0)
	\quad\text{或}\quad
	\eval{\dv{\vb{r}}{t}}_{t=t_0}.
\]

若\(\vb{f}(t)\)在区域\(D\)中的每一点\(t\)处都存在导向量\(\vb{f}'(t)\),那么就称\(\vb{f}(t)\)在\(D\)上\DefineConcept{可导}.
\end{definition}

\begin{theorem}
向量值函数\(\vb{r}=\vb{f}(t)\)在\(t_0\)可导的充分必要条件是:
\(\vb{f}(t)\)的全部分量函数\[
f_1(t),f_2(t),\dotsc,f_n(t)
\]都在\(t_0\)可导.
\end{theorem}

\begin{theorem}
当\(\vb{f}(t)\)在\(t_0\)可导时,其导数为
\begin{equation}
	\vb{f}'(t_0)
	= (f_1'(t),f_2'(t),\dotsc,f_n'(t)).
\end{equation}
\end{theorem}

\begin{theorem}
设\(\vb{f}(t) = (f_1(t),f_2(t),f_3(t))\),
\(\vb{g}(t) = (g_1(t),g_2(t),g_3(t))\),
\[
	\lim_{t \to t_0} \vb{f}(t) = \vb{u},
	\qquad
	\lim_{t \to t_0} \vb{g}(t) = \vb{v},
\]
那么\[
	\lim_{t \to t_0} [\vb{f}(t) \times \vb{g}(t)]
	= \vb{u} \times \vb{v}.
\]
\end{theorem}

向量值函数的导数运算法则与数量函数的导数运算法则的形式相同:
\begin{theorem}[向量值函数的导数运算法则]
设\(\vb{u}(t)\)、\(\vb{v}(t)\)是可导的向量值函数,
\(\vb{C}\)是常向量,
\(c\)是任一常数,
\(\phi(t)\)是可导的数量函数,
则\begin{align}
	&\dv{t} \vb{C}
		= \vb{0}, \\
	&\dv{t} [c \vb{u}(t)]
		= c \vb{u}'(t), \\
	&\dv{t} [\vb{u}(t) \pm \vb{v}(t)]
		= \vb{u}'(t) + \vb{v}'(t), \\
	&\dv{t} [\phi(t) \vb{u}(t)]
		= \phi'(t) \vb{u}(t) + \phi(t) \vb{u}'(t), \\
	&\dv{t} [\vb{u}(t) \cdot \vb{v}(t)]
		= \vb{u}'(t) \cdot \vb{v}(t) + \vb{u}(t) \cdot \vb{v}'(t), \\
	&\dv{t} [\vb{u}(t) \times \vb{v}(t)]
		= \vb{u}'(t) \times \vb{v}(t) + \vb{u}(t) \times \vb{v}'(t), \\
	&\dv{t} \vb{u}[\phi(t)]
		= \phi'(t) \vb{u}'[\phi(t)].
\end{align}
\end{theorem}
