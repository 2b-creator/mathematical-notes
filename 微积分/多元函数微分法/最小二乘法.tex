\section{最小二乘法}
许多科学、工程问题常常需要根据两个或者多个变量的几次实验数据
来找出这些变量的(近似)关系式(称这些依据实验数据建立的近似关系式为经验公式).
在建立经验公式的过程中,经常使用线性数学模型拟合实验数据.
但是在测量时因为受到了各种条件的限制,我们往往得到的是一个不相容(无解)的线性方程组,
不过我们可以采用“最小二乘法”求出线性方程组的近似解.

我们首先考虑只有两个变量(一个作为自变量,一个作为因变量)的情形.

已知点列\(\{P_n\}\).
假设点列中的所有点\((x_k,y_k)\ (k=1,2,\dotsc,n)\)都近似满足直线方程\[
	f(x) = a x + b,
\]
再根据偏差的平方和\[
	M(a,b) = \sum_k [y_k - f(x_k)]^2 = \sum_k [y_k - (a x_k + b)]^2
\]为最小的条件来选择常数\(a,b\).

令\[
	\left\{ \begin{array}{l}
		\pdv{M}{a} = -2 \sum_k [y_k - (a x_k + b)] x_k = 0, \\
		\pdv{M}{b} = -2 \sum_k [y_k - (a x_k + b)] = 0,
	\end{array} \right.
\]
那么有\begin{equation}\label{equation:最小二乘法.关于a和b的代数方程组}
	\left\{ \begin{array}{l}
		\sum_k x_k y_k = a \sum_k x_k^2 + b \sum_n x_k, \\
		\sum_k y_k = a \sum_k x_k + b n. \\
	\end{array} \right.
\end{equation}
现在只要首先将\[
	\sum_k x_k, \qquad
	\sum_k x_k^2, \qquad
	\sum_k x_k y_k, \qquad
	\sum_k y_k
\]计算出来,
代入代数方程组 \labelcref{equation:最小二乘法.关于a和b的代数方程组},
即可解出合适的\(a\)和\(b\)使得\(M(a,b)\)取得最小值.

现在我们将上面的方法推广到有\(n\)个自变量和\(s\)个因变量的情形.

设\(\vb{A} \in M_{s \times n}(\mathbb{R})\),
\(\vb{\beta} \in \mathbb{R}^s\),
\(\vb{x} \in \mathbb{R}^n\),
线性方程组\(\vb{A}\vb{x}=\vb{\beta}\)无解,
即\(\rank(\vb{A},\vb{\beta})>\rank\vb{A}\),
定义函数\[
	M(\vb{x})
	\defeq
	\norm{\vb{A}\vb{x}-\vb{\beta}}^2,
\]
求得它的梯度\(\grad M(\vb{x})\),
再令\[
	\grad M(\vb{x}) = 0,
\]
解得\(\vb{x}=\vb{x}_0\),
它就可以使\(f(\vb{x})\)取得最小值.
我们把像这样的\(\vb{x}_0\)称为\DefineConcept{最小二乘解}.

\endgroup
