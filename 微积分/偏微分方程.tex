\chapter{偏微分方程}
从本章起,我们开始学习如何求解未知函数是多变量函数的微分方程,也就是\DefineConcept{偏微分方程}和\DefineConcept{全微分方程}.

\section{拉普拉斯方程}
\begin{definition}
形如\[
\sum\limits_{k=1}^n \pdv[2]{y}{x_k} = 0
\]的微分方程称为\DefineConcept{拉普拉斯方程}.
\end{definition}

\begin{example}
函数\(z=\ln\sqrt{x^2+y^2}\)是方程\[
\pdv[2]{z}{x} + \pdv[2]{z}{y} = 0
\]的一个解.
\end{example}

\begin{example}
函数\(u=\frac{1}{r}\)是方程\[
\pdv[2]{u}{x} + \pdv[2]{u}{y} + \pdv[2]{u}{z} = 0
\]的一个解,其中\(r=\sqrt{x^2+y^2+z^2}\).
\end{example}

\section{全微分方程}
\subsection{全微分方程的概念}
\begin{definition}
形如\[
P(x,y)\dd{x} + Q(x,y)\dd{y} = 0
\]的微分方程,如果恰好有\[
\dd{u(x,y)} = P(x,y)\dd{x} + Q(x,y)\dd{y},
\]则称其为\DefineConcept{全微分方程}(total differential equation)%
或\DefineConcept{恰当方程}(exact differential equation).
\end{definition}

\subsection{全微分方程的隐式通解}
设全微分方程的隐式通解为\[
u(x,y) = C,
\]其中\(C\)为常数.

如果\(P,Q\)在单连通区域\(G\)内具有一阶连续偏导数,
则曲线积分\(\int_L P\dd{x}+Q\dd{y}\)在\(G\)内与路径无关,
进而有全微分方程在\(G\)内的显式通解为\[
u(x,y) \equiv \int_{(x_0,y_0)}^{(x,y)} P\dd{x}+Q\dd{y} = C.
\]

已知\[
\pdv{u}{x} = P(x,y),
\qquad
\pdv{u}{y} = Q(x,y)
\]的情况下,首先对\(x\)积分,得\[
u = \int P \dd{x} + \phi(y);
\]再对\(y\)求偏导,将所得结果与\(\pdv{u}{y} = Q\)比较,可得\(\phi(y)\).

\subsection{积分因子法}
对于微分方程\[
P(x,y)\dd{x} + Q(x,y)\dd{y} = 0,
\]如果存在连续可微函数\(\mu=\mu(x,y)\),使得\(\mu P \dd{x} + \mu Q \dd{y} = 0\)成为全微分方程,即\(\mu P \dd{x} + \mu Q \dd{y} = \dd{u}\),则称函数\(\mu\)为该微分方程的\DefineConcept{积分因子}.

如果微分方程\[
P(x,y)\dd{x} + Q(x,y)\dd{y} = 0,
\]满足\(\frac{P}{Q}=-f\left(\frac{y}{x}\right)\),则积分因子为\[
\mu = \frac{1}{xP+yQ}.
\]
