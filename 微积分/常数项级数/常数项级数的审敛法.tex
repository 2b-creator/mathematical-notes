\section{常数项级数的审敛法}
\subsection{正项级数及其审敛法}
\subsubsection{正项级数的概念及其收敛条件}
一般的常数项级数,它的各项可以是正数、负数或零.
现在我们先讨论各项都是正数或零的级数,这种级数称为\DefineConcept{正项级数}.
这种级数特别重要,以后将看到许多级数的收敛性问题可归结为正项级数的收敛性问题.

\begin{theorem}\label{theorem:无穷级数.正项级数收敛的充要条件}
%@see: 《高等数学(第六版 下册)》 P256. 定理1
%@see: 《数学分析教程 (第三版 下册)》 P163. 定理14.2.1
正项级数收敛的充要条件是:它的部分和数列有界.
\begin{proof}
设\(\sum\limits_{n=1}^\infty u_n\)是一个正项级数,
它的部分和数列为\(\{s_n\}\).
显然,数列\(\{s_n\}\)是一个单调增加数列.
如果数列\(\{s_n\}\)有界,
那么根据\hyperref[theorem:极限.数列的单调有界定理]{单调有界定理},
数列\(\{s_n\}\)收敛,
也就是说,级数\(\sum\limits_{n=1}^\infty u_n\)收敛.

反之,如果正项级数\(\sum\limits_{n=1}^\infty u_n\)收敛于和\(s\),
即\(\lim\limits_{n\to\infty} s_n = s\),
根据\hyperref[theorem:极限.收敛数列的有界性]{收敛数列的有界性}可知,
数列\(\{s_n\}\)有界.
\end{proof}
\end{theorem}

我们可以写出\cref{theorem:无穷级数.正项级数收敛的充要条件} 的逆否命题.
\begin{proposition}
正项级数发散的充要条件是:它的部分和数列无界.
\end{proposition}

\begin{example}
%@see: 《数学分析教程 (第三版 下册)》 P163. 例1
设正项级数\(\sum\limits_{n=1}^\infty a_n\)的部分和是\(s_n\),证明:\[
	\sum\limits_{n=1}^\infty \frac{a_n}{s_n^2} < +\infty.
\]
\begin{proof}
显然\(\sum\limits_{n=1}^\infty \frac{a_n}{s_n^2}\)是正项级数.
由于对于任意正整数\(N\),有\begin{align*}
	\sum_{n=2}^N \frac{a_n}{s_n^2}
	&= \sum_{n=2}^N \frac{s_n-s_{n-1}}{s_n^2}
	\leq \sum_{n=2}^N \frac{s_n-s_{n-1}}{s_{n-1} s_n} \\
	&= \sum_{n=2}^N \left(
			\frac{1}{s_{n-1}} - \frac{1}{s_n}
		\right)
	= \frac{1}{s_1} - \frac{1}{s_N}
	< \frac{1}{a_1},
\end{align*}
也就是说\(\sum\limits_{n=1}^\infty \frac{a_n}{s_n^2}\)的部分和有界,
所以根据\cref{theorem:无穷级数.正项级数收敛的充要条件},该级数收敛.
\end{proof}
\end{example}

\subsubsection{比较审敛法}
利用\cref{theorem:无穷级数.正项级数收敛的充要条件} 直接证明某些级数的部分和有界不太容易,
因此我们需要一个判别级数敛散性的更简单的方法.

\begin{theorem}[比较审敛法]\label{theorem:无穷级数.正项级数的比较审敛法}
%@see: 《高等数学(第六版 下册)》 P256. 定理2
%@see: 《数学分析教程 (第三版 下册)》 P164. 定理14.2.2
设\(\sum\limits_{n=1}^\infty u_n\)
和\(\sum\limits_{n=1}^\infty v_n\)都是正项级数,且\[
	u_n \leq v_n
	\quad(n=1,2,\dotsc).
\]
若级数\(\sum\limits_{n=1}^\infty v_n\)收敛,
则级数\(\sum\limits_{n=1}^\infty u_n\)收敛;
反之,若级数\(\sum\limits_{n=1}^\infty u_n\)发散,
则级数\(\sum\limits_{n=1}^\infty v_n\)发散.
\begin{proof}
设级数\(\sum\limits_{n=1}^\infty v_n\)收敛于和\(\sigma\),
则级数\(\sum\limits_{n=1}^\infty u_n\)的部分和\[
	s_n = u_1 + u_2 + \dotsb u_n
	\leq
	v_1 + v_2 + \dotsb + v_n \leq \sigma
	\quad(n=1,2,\dotsc),
\]
即部分和数列\(\{s_n\}\)有界,
由\cref{theorem:无穷级数.正项级数收敛的充要条件} 知级数\(\sum\limits_{n=1}^\infty u_n\)收敛.
\end{proof}
\end{theorem}

\begin{example}
判断级数\(\sum\limits_{n=1}^\infty \frac{1}{1+a^n}\ (a>0)\)的收敛性.
\begin{solution}
显然有\[
	\frac{1}{1+a^n} < \frac{1}{a^n}.
\]
根据比较审敛法,如果级数\(\sum\limits_{n=1}^\infty \frac{1}{a^n}\)收敛,
那么级数\(\sum\limits_{n=1}^\infty \frac{1}{1+a^n}\)收敛;
而等比级数\(\sum\limits_{n=1}^\infty \frac{1}{a^n}\)
当且仅当其公比\(\abs{\frac{1}{a}} < 1\),即\(a > 1\)时收敛;
故当\(a > 1\)时,级数\(\sum\limits_{n=1}^\infty \frac{1}{1+a^n}\)收敛.

当\(0 < a \leq 1\)时,\(0 < a^n \leq 1\),\(1 < 1 + a^n \leq 2\),\[
	\frac{1}{1+a^n} \geq \frac{1}{2},
\]
而等差级数\(\sum\limits_{n=1}^\infty \frac{1}{2}\)发散,
故级数\(\sum\limits_{n=1}^\infty \frac{1}{1+a^n}\)发散.
\end{solution}
\end{example}

注意到级数的每一项同乘不为零的常数\(k\)
以及去掉级数前面部分的有限项不会影响级数的收敛性,
我们可得如下推论:
\begin{corollary}\label{theorem:无穷级数.正项级数的比较审敛法的推论}
%@see: 《高等数学(第六版 下册)》 P257. 推论
设\(\sum\limits_{n=1}^\infty u_n\)和\(\sum\limits_{n=1}^\infty v_n\)都是正项级数.
\begin{enumerate}
	\item 如果级数\(\sum\limits_{n=1}^\infty v_n\)收敛,
	且\[
		(\exists k>0)
		(\exists N\in\mathbb{N})
		(\forall n\in\mathbb{N})
		[
			n > N
			\implies
			u_n \leq k v_n
		],
	\]
	则级数\(\sum\limits_{n=1}^\infty u_n\)收敛.

	\item 如果级数\(\sum\limits_{n=1}^\infty v_n\)发散,
	且\[
		(\exists k>0)
		(\exists N\in\mathbb{N})
		(\forall n\in\mathbb{N})
		[
			n > N
			\implies
			u_n \geq k v_n
		],
	\]
	则级数\(\sum\limits_{n=1}^\infty u_n\)发散.
\end{enumerate}
\end{corollary}

\begin{example}
试证:级数\(\sum\limits_{n=1}^\infty \frac{1}{\sqrt{n(n+1)}}\)是发散的.
\begin{proof}
因为\(n(n+1) < (n+1)^2\),所以\(\frac{1}{\sqrt{n(n+1)}} > \frac{1}{n+1}\),而级数\(\sum\limits_{n=1}^\infty \frac{1}{n+1}\)是发散的,根据比较审敛法可知级数\(\sum\limits_{n=1}^\infty \frac{1}{\sqrt{n(n+1)}}\)是发散的.
\end{proof}
\end{example}

\subsubsection{p级数}
\begin{proposition}\label{example:无穷级数.p级数的收敛性}
讨论p级数\[
	1+\frac{1}{2^p}+\frac{1}{3^p}+\dotsb+\frac{1}{n^p}+\dotsb
\]的收敛性,
其中常数\(p>0\).
\begin{solution}
当\(p \leq 1\)时,p级数各项均不小于调和级数对应项,
即\(\frac{1}{n^p} \geq \frac{1}{n}\),
但调和级数发散,
故根据\cref{theorem:无穷级数.正项级数的比较审敛法} 可知,
当\(p \leq 1\)时p级数发散.

当\(p > 1\)时,
因为\(k-1
\leq x
\leq k \implies \frac{1}{k}
\leq \frac{1}{x} \implies \frac{1}{k^p}
\leq \frac{1}{x^p}\),
所以\[
	\frac{1}{k^p}
	= \int_{k-1}^k \frac{1}{k^p} \dd{x}
	\leq \int_{k-1}^k \frac{1}{x^p} \dd{x}
	\quad(k=2,3,\dotsc),
\]
从而级数的部分和
\begin{align*}
	s_n &= 1 + \sum\limits_{k=2}^n{\frac{1}{k^p}}
	\leq 1 + \sum\limits_{k=2}^n{ \int_{k-1}^k{\frac{1}{x^p}\dd{x}} }
	= 1 + \int_1^n{\frac{1}{x^p}\dd{x}} \\
	&= 1 + \frac{1}{p-1}\left(1-\frac{1}{n^{p-1}}\right)
	< 1 + \frac{1}{p-1}
	\quad(n=2,3,\dotsc),
\end{align*}
这表明数列\(\{s_n\}\)有界,因此p级数收敛.

综上所述,{\color{red} p级数\(\sum\limits_{n=1}^\infty \frac{1}{n^p}\)
当\(p > 1\)时收敛,
当\(p \leq 1\)时发散.}
\end{solution}
\end{proposition}
可以发现,p级数与\hyperref[example:定积分.p积分]{p积分}具有高度相似性.

\begin{example}
设正项级数\(\sum\limits_{n=1}^\infty a_n\)的部分和是\(s_n\),
证明:对于任意\(k>1\),有\[
	\sum_{n=1}^\infty \frac{a_n}{s_n^k} < +\infty.
\]
\begin{proof}
当\(k>1\)时,
有\begin{align*}
	\sum_{i=2}^n \frac{a_i}{s_i^k}
	&= \sum_{i=2}^n \frac{s_i-s_{i-1}}{s_i^k} \\
	&= \sum_{i=2}^n \frac{1}{s_i^k} \int_{s_{i-1}}^{s_i} \dd{x}
			&(\text{\cref{theorem:定积分.定积分性质4}}) \\
	&\leq \sum_{i=2}^n \int_{s_{i-1}}^{s_i} \frac{1}{x^k} \dd{x}
			&(\text{\cref{theorem:定积分.定积分性质6}}) \\
	&= \int_{s_1}^{s_n} \frac{1}{x^k} \dd{x}
			&(\text{\cref{theorem:定积分.定积分性质3}}) \\
	&\leq \int_{s_1}^{+\infty} \frac{1}{x^k} \dd{x}
	< +\infty,
			&(\text{\cref{example:定积分.p积分}})
\end{align*}
可见该级数的部分和有界,因此该级数收敛.
\end{proof}
\end{example}

\begin{example}
设正项级数\(\sum\limits_{n=1}^\infty a_n\)发散,证明:级数\(\sum\limits_{n=1}^\infty \frac{a_n}{n^3+a_n^2}\)收敛.
\begin{proof}
由基本不等式 \labelcref{theorem:不等式.基本不等式1} 可知
\(n^3+a_n^2\geq2\sqrt{n^3 a_n^2}=2n^{3/2}a_n\),
那么\(\frac{a_n}{n^3+a_n^2}\leq\frac{a_n}{2n^{3/2}a_n}=\frac{1}{2n^{3/2}}\).
由\cref{example:无穷级数.p级数的收敛性}
我们知道\(p=\frac{3}{2}>1\)时,p级数收敛;
那么根据\hyperref[theorem:无穷级数.正项级数的比较审敛法]{比较审敛法}可知
级数\(\sum\limits_{n=1}^\infty \frac{a_n}{n^3+a_n^2}\)收敛.
\end{proof}
\end{example}

\subsubsection{比较审敛法的极限形式}
\begin{theorem}[比较审敛法的极限形式]\label{theorem:无穷级数.正项级数的比较审敛法的极限形式}
%@see: 《高等数学(第六版 下册)》 P258. 定理3
设\(\sum\limits_{n=1}^\infty u_n\)和\(\sum\limits_{n=1}^\infty v_n\)都是正项级数,
记\[
	\rho
	\defeq
	\lim\limits_{n\to\infty} \frac{u_n}{v_n}.
\]
\begin{enumerate}
	\item 如果\(\rho\in[0,+\infty)\),
	且级数\(\sum\limits_{n=1}^\infty v_n\)收敛,
	则级数\(\sum\limits_{n=1}^\infty u_n\)收敛;

	\item 如果\(\rho\in(0,+\infty]\),
	且级数\(\sum\limits_{n=1}^\infty v_n\)发散,
	则级数\(\sum\limits_{n=1}^\infty u_n\)发散.
\end{enumerate}
\begin{proof}
如果\(\rho\in[0,+\infty)\),
那么由极限定义可知,
对\(\epsilon=1\),
存在正整数\(N\),
当\(n>N\)时,
有\[
	\frac{u_n}{v_n} < l+1,
\]
即\(u_n < (l+1) v_n\).
而级数\(\sum\limits_{n=1}^\infty v_n\)收敛,
根据\cref{theorem:无穷级数.正项级数的比较审敛法的推论} 可知,
级数\(\sum\limits_{n=1}^\infty u_n\)收敛.

如果\(\rho\in(0,+\infty]\),
那么极限\(\lim\limits_{n\to\infty} \frac{v_n}{u_n}\)存在且有限.
如果级数\(\sum\limits_{n=1}^\infty u_n\)收敛,
那么由上可知,级数\(\sum\limits_{n=1}^\infty v_n\)收敛;
但已知级数\(\sum\limits{n=1}^\infty v_n\)发散,矛盾!
因此级数\(\sum\limits_{n=1}^\infty u_n\)不可能收敛,
即级数\(\sum\limits_{n=1}^\infty u_n\)发散.
\end{proof}
\end{theorem}

极限形式的比较审敛法,在两个正项级数的一般项均趋于零的情况下,
其实是比较它们的一般项作为无穷小量的阶.
定理表明,当\(n \to \infty\)时,
如果\(u_n\)是与\(v_n\)同阶或是比\(v_n\)高阶的无穷小,
而级数\(\sum\limits_{n=1}^\infty v_n\)收敛,则级数\(\sum\limits_{n=1}^\infty u_n\)收敛;
如果\(u_n\)是与\(v_n\)同阶或是比\(v_n\)低阶的无穷小,
而级数\(\sum\limits_{n=1}^\infty v_n\)发散,则级数\(\sum\limits_{n=1}^\infty u_n\)发散.

\begin{example}
判断级数\(\sum\limits_{n=1}^\infty \sin\frac{1}{n}\)的收敛性.
\begin{solution}
因为\[
\lim\limits_{n\to\infty} \frac{\sin(1/n)}{1/n} = 1 > 0,
\]而级数\(\sum\limits_{n=1}^\infty \frac{1}{n}\)发散,可知级数\(\sum\limits_{n=1}^\infty \sin\frac{1}{n}\)发散.
\end{solution}
\end{example}

\begin{example}
判断级数\[
1 + \frac{1}{3} + \frac{1}{5} + \dotsb + \frac{1}{2n-1} + \dotsb
\]的收敛性.
\begin{solution}
记\(u_n = \frac{1}{2n-1}\),取\(v_n = \frac{1}{n}\).因为\[
\lim\limits_{n\to\infty} \frac{u_n}{v_n} = \lim\limits_{n\to\infty} \frac{n}{2n-1} = \lim\limits_{n\to\infty} \frac{1}{2-1/n} = \frac{1}{2} > 0,
\]而级数\(\sum\limits_{n=1}^\infty \frac{1}{n}\)发散,所以级数\(\sum\limits_{n=1}^\infty \frac{1}{2n-1}\)发散.
\end{solution}
\end{example}

\begin{example}
判断级数\[
1 + \frac{1+2}{1+2^2} + \frac{1+3}{1+3^2} + \dotsb + \frac{1+n}{1+n^2} + \dotsb
\]的收敛性.
\begin{solution}
记\(u_n = \frac{1+n}{1+n^2}\),取\(v_n = \frac{1}{1+n}\).
因为\[
\lim\limits_{n\to\infty} \frac{u_n}{v_n}
= \lim\limits_{n\to\infty} \frac{(1+n)^2}{1+n^2}
= \lim\limits_{n\to\infty} \frac{n^2 + 2n + 1}{n^2 + 1}
= 1 > 0,
\]而\(\sum\limits_{n=1}^\infty v_n\)发散,所以级数\(\sum\limits_{n=1}^\infty \frac{1+n}{1+n^2}\)发散.
\end{solution}
\end{example}

\begin{example}
判断级数\[
\frac{1}{2\cdot5} + \frac{1}{3\cdot6} + \dotsb + \frac{1}{(n+1)(n+4)} + \dotsb
\]的收敛性.
\begin{solution}
记\(u_n = \frac{1}{(n+1)(n+4)}\),取\(v_n = \frac{1}{n^2}\).
因为\[
\lim\limits_{n\to\infty} \frac{u_n}{v_n} = \lim\limits_{n\to\infty} \frac{n^2}{(n+1)(n+4)} = 1,
\]而级数\(\sum\limits_{n=1}^\infty v_n\)收敛,所以级数\(\sum\limits_{n=1}^\infty \frac{1+n}{1+n^2}\)收敛.
\end{solution}
\end{example}

\begin{example}
\newcommand\sinfrac[1][]{\sin\frac{\pi}{2^{#1}}}
判断级数\[
\sinfrac + \sinfrac[2] + \sinfrac[3] + \dotsb + \sinfrac[n] + \dotsb
\]的收敛性.
\begin{solution}
记\(u_n = \sin\frac{\pi}{2^n}\),取\(v_n = \frac{\pi}{2^n}\).
因为\[
\lim\limits_{n\to\infty} \frac{u_n}{v_n}
= \lim\limits_{n\to\infty} \frac{\sin(\pi/2^n)}{\pi/2^n} = 1,
\]而级数\(\sum\limits_{n=1}^\infty v_n\)收敛,所以级数\(\sum\limits_{n=1}^\infty \sin\frac{\pi}{2^n}\)收敛.
\end{solution}
\end{example}

\subsubsection{比值审敛法}
用比较审敛法审敛时,需要适当地选取一个已知其收敛性的级数\(\sum\limits_{n=1}^\infty v_n\)作为比较的基准.最常选用作为基准级数的是等比级数和p级数.

将所给正项级数与等比级数比较,我们能得到在实用上很方便的比值审敛法和根值审敛法.
\begin{theorem}[比值审敛法,达朗贝尔判别法]\label{theorem:无穷级数.正项级数的比值审敛法}
设\(\sum\limits_{n=1}^\infty u_n\)为正项级数,如果\[
\lim\limits_{n\to\infty} \frac{u_{n+1}}{u_n}=\rho,
\]则当\(\rho<1\)时,级数收敛;
当\(\rho>1\)时或当\(\lim\limits_{n\to\infty} \frac{u_{n+1}}{u_n}=\infty\)时,级数发散;
当\(\rho=1\)时,级数可能收敛也可能发散.
\begin{proof}
当\(\rho<1\).取一个适当小的正数\(\epsilon\),使得\(\rho+\epsilon=r<1\),根据极限定义,存在正整数\(m\),当\(n \geq m\)时有不等式\[
\frac{u_{n+1}}{u_n} < \rho + \epsilon = r.
\]因此\[
u_{m+1} < r u_m,
u_{m+2} < r u_{m+1} < r^2 u_m,
\dotsc,
u_{m+k} < r^k u_m,
\dotsc.
\]而因为公比\(r<1\),故等比级数\(\sum\limits_{k=1}^\infty r^k u_m\)收敛,根据\cref{theorem:无穷级数.正项级数的比较审敛法的推论} 可知,级数\(\sum\limits_{n=1}^\infty u_n\)收敛.

当\(\rho>1\).取一个适当小的正数\(\epsilon\),使得\(\rho-\epsilon>1\).
根据极限定义,当\(n \geq m\)时有不等式\[
\frac{u_{n+1}}{u_n} > \rho-\epsilon > 1,
\]也就是\(u_{n+1}>u_n\).所以当\(n \geq m\)时,级数的一般项\(u_n\)是逐渐增大的,从而\[
\lim\limits_{n\to\infty} u_n \neq 0.
\]根据\cref{theorem:无穷级数.收敛级数性质5} (即级数收敛的必要条件)可知,级数\(\sum\limits_{n=1}^\infty u_n\)发散.

类似地,可以证明当\(\lim\limits_{n\to\infty} \frac{u_{n+1}}{u_n} = \infty\)时,级数\(\sum\limits_{n=1}^\infty u_n\)发散.

当\(\rho = 1\)时,级数可能收敛也可能发散.例如p级数不论\(p\)为何值都有\[
\lim\limits_{n\to\infty} \frac{u_{n+1}}{u_n} = \lim\limits_{n\to\infty} \frac{1/(n+1)^p}{1/n^p} = 1.
\]但我们知道,当\(p>1\)时p级数收敛,当\(p\leq1\)时p级数发散,因此只根据\(\rho=1\)不能判定级数的收敛性.
\end{proof}
\end{theorem}

\begin{example}\label{example:无穷级数.常数e的级数表示}
证明级数\[
1+\frac{1}{1}+\frac{1}{1\cdot2}+\frac{1}{1\cdot2\cdot3}+\dotsb+\frac{1}{(n-1)!}+\dotsb
\]是收敛的,并估计以级数的部分和\(s_n\)近似代替和\(s\)所产生的误差.
\begin{solution}
因为\[
\lim\limits_{n\to\infty} \frac{u_{n+1}}{u_n} = \lim\limits_{n\to\infty} \frac{(n-1)!}{n!} = \lim\limits_{n\to\infty} \frac{1}{n} = 0 < 1,
\]根据比值审敛法可知,该级数收敛.

以该级数的部分和近似代替和\(s\)所产生的的误差为\[
\begin{split}
\abs{r_n} &= \frac{1}{n!} + \frac{1}{(n+1)!} + \frac{1}{(n+2)!} + \dotsb \\
&= \frac{1}{n!} \left[ 1 + \frac{1}{n+1} + \frac{1}{(n+1)(n+2)} + \dotsb \right] \\
&< \frac{1}{n!} \left( 1 + \frac{1}{n} + \frac{1}{n^2} + \dotsb \right) \\
&= \frac{1}{n!} \frac{1}{1-1/n}
= \frac{1}{(n-1)\cdot(n-1)!}.
\end{split}
\]
\end{solution}
\end{example}

\begin{example}
\newcommand\myfrac[1][]{
\def\a{\ifx\relax#1\relax1\else#1\fi}
\frac{3^{#1}}{\a\cdot2^{#1}}
}
判断级数\[
\myfrac + \myfrac[2] + \myfrac[3] + \dotsb + \myfrac[n] + \dotsb
\]的收敛性.
\begin{solution}
记\(u_n = \myfrac[n]\).因为\[
\lim\limits_{n\to\infty} \frac{u_{n+1}}{u_n}
= \lim\limits_{n\to\infty} \frac{3}{2}\cdot\frac{n}{n+1}
= \frac{3}{2} > 1,
\]所以级数发散.
\end{solution}
\end{example}

\begin{example}
判断级数\(\sum\limits_{n=1}^\infty \frac{n^2}{3^n}\)的收敛性.
\begin{solution}
记\(u_n = \frac{n^2}{3^n}\).因为\[
\lim\limits_{n\to\infty} \frac{u_{n+1}}{u_n}
= \lim\limits_{n\to\infty} \frac{1}{3} \cdot \frac{(n+1)^2}{n^2}
= \frac{1}{3} < 1,
\]所以级数收敛.
\end{solution}
\end{example}

\begin{example}
判断级数\(\sum\limits_{n=1}^\infty \frac{2^n \cdot n!}{n^n}\)的收敛性.
\begin{solution}
记\(u_n = \frac{2^n \cdot n!}{n^n}\).
因为\begin{align*}
	\lim\limits_{n\to\infty} \frac{u_{n+1}}{u_n}
	&= \lim\limits_{n\to\infty} 2(n+1) \cdot \frac{n^n}{(n+1)^{n+1}} \\
	&= 2 \cdot \lim\limits_{n\to\infty} \frac{n^n}{(n+1)^n}
	= 2 \cdot \lim\limits_{n\to\infty} \frac{1}{\left(1+\frac{1}{n}\right)^n} \\
	&= 2 \left[ \lim\limits_{n\to\infty} \left(1+\frac{1}{n}\right)^n \right]^{-1}
	= \frac{2}{e} < 1,
\end{align*}
所以级数收敛.
\end{solution}
\end{example}

\begin{example}
判断级数\(\sum\limits_{n=1}^\infty n \tan\frac{\pi}{2^{n+1}}\)的收敛性.
\begin{solution}
记\(u_n = n \tan\frac{\pi}{2^{n+1}}\).我们有\[
	\frac{u_{n+1}}{u_n}
	= \frac{n+1}{n} \frac{\tan(\frac{1}{2}\frac{\pi}{2^{n+1}})}{\tan\frac{\pi}{2^{n+1}}}.
\]
根据二倍角公式\[
	\tan2\theta = \frac{2\tan\theta}{1-\tan^2\theta},
	\qquad
	\frac{\tan\theta}{\tan2\theta} = \frac{1-\tan^2\theta}{2},
\]
有\[
	\frac{\tan(\frac{1}{2}\frac{\pi}{2^{n+1}})}{\tan\frac{\pi}{2^{n+1}}}
	= \frac{1}{2} \left(
		1-\tan^2\frac{\pi}{2^{n+2}}
	\right).
\]
于是\begin{align*}
	\lim\limits_{n\to\infty} \frac{u_{n+1}}{u_n}
	&= \lim\limits_{n\to\infty} \frac{n+1}{n} \frac{1}{2} \left(
		1-\tan^2\frac{\pi}{2^{n+2}}
	\right) \\
	&= \frac{1}{2} \cdot \lim\limits_{n\to\infty} \frac{n+1}{n} \cdot \left(
		1 - \lim\limits_{n\to\infty} \tan^2\frac{\pi}{2^{n+2}}
	\right) \\
	&= \frac{1}{2} \cdot 1 \cdot (1 - 0) = \frac{1}{2} < 1.
\end{align*}
所以级数收敛.
\end{solution}
\end{example}

\subsubsection{比值审敛法的上、下极限形式}
\begin{corollary}[比值审敛法的上、下极限形式]\label{theorem:无穷级数.正项级数的比值审敛法的上下极限形式}
\def\orho{\overline{\rho}}
\def\urho{\underline{\rho}}
设\(\sum\limits_{n=1}^\infty u_n\)为正项级数,
记\[
	\orho
	\defeq
	\varlimsup\limits_{n\to\infty} \frac{u_{n+1}}{u_n},
	\qquad
	\urho
	\defeq
	\varliminf\limits_{n\to\infty} \frac{u_{n+1}}{u_n}.
\]

如果\[
	\orho < 1,
\]
则级数\(\sum\limits_{n=1}^\infty u_n\)收敛.

如果\[
	\urho > 1,
\]
则级数\(\sum\limits_{n=1}^\infty u_n\)发散.

如果\[
	\orho \geq 1
	\quad\lor\quad
	\urho \leq 1,
\]
则级数可能收敛也可能发散.
\end{corollary}

\subsubsection{根值审敛法}
\begin{theorem}[根值审敛法,柯西判别法]\label{theorem:无穷级数.正项级数的根值审敛法}
设\(\sum\limits_{n=1}^\infty u_n\)为正项级数,如果\[
\lim\limits_{n\to\infty} \sqrt[n]{u_n}=\rho,
\]则当\(\rho<1\)时,级数收敛;
当\(\rho>1\)时或当\(\lim\limits_{n\to\infty} \sqrt[n]{u_n}=+\infty\)时,级数发散;
当\(\rho=1\)时,级数可能收敛也可能发散.
\end{theorem}
在根值审敛法中可以把\(\lim\limits_{n\to\infty}\)替换为\(\varlimsup\limits_{n\to\infty}\).

\begin{example}
判定级数\(\sum\limits_{n=1}^\infty \frac{2+(-1)^n}{2^n}\)的收敛性.
\begin{proof}
记\[
u_n = \frac{2+(-1)^n}{2^n},
\]显然有\[
\lim\limits_{n\to\infty} \sqrt[n]{u_n}
= \lim\limits_{n\to\infty} \frac{1}{2} \sqrt[n]{2+(-1)^n}
= \lim\limits_{n\to\infty} \frac{1}{2} \exp{\frac{1}{n} \ln[2+(-1)^n]},
\]因为\(\ln[2+(-1)^n] \in \{ 0, \ln3 \}\)有界,故\(\lim\limits_{n\to\infty} \frac{1}{n} \ln[2+(-1)^n] = 0\),从而\[
\lim\limits_{n\to\infty} \sqrt[n]{u_n} = \frac{1}{2} < 1.
\]根据根值审敛法可知,级数\(\sum\limits_{n=1}^\infty u_n\)收敛.
\end{proof}
\end{example}

\subsubsection{比值审敛法与根值审敛法之间的联系}
\begin{theorem}
对于正项数列\(\sum\limits_{n=1}^\infty u_n\),
总有\[
	\varliminf\limits_{n\to\infty} \frac{u_{n+1}}{u_n}
	\leq
	\varliminf\limits_{n\to\infty} \sqrt[n]{u_n}
	\leq
	\varlimsup\limits_{n\to\infty} \sqrt[n]{u_n}
	\leq
	\varlimsup\limits_{n\to\infty} \frac{u_{n+1}}{u_n}
\]成立.
\end{theorem}

\subsubsection{极限审敛法}
将所给正项级数与p级数作比较,可得在实用上较方便的极限审敛法.
\begin{theorem}[极限审敛法]\label{theorem:无穷级数.正项级数的极限审敛法}
设\(\sum\limits_{n=1}^\infty u_n\)为正项级数.
\begin{enumerate}
	\item 当\(\lim\limits_{n\to\infty} n^p u_n = l\in[0,+\infty)\),
	且\(p>1\)时,级数\(\sum\limits_{n=1}^\infty u_n\)收敛;
	\item 当\(\lim\limits_{n\to\infty} n^p u_n = l\in(0,+\infty]\),
	且\(p\leq1\)时,级数\(\sum\limits_{n=1}^\infty u_n\)发散.
\end{enumerate}
\end{theorem}

\begin{example}
判定级数\(\sum\limits_{n=1}^\infty \ln(1+\frac{1}{n^2})\)的收敛性.
\begin{solution}
因\(\ln(1+\frac{1}{n^2}) \sim \frac{1}{n^2}\ (n\to\infty)\),故\[
\lim\limits_{n\to\infty} n^2 u_n = \lim\limits_{n\to\infty} n^2 \ln(1+\frac{1}{n^2})
= \lim\limits_{n\to\infty} n^2 \cdot \frac{1}{n^2} = 1,
\]根据极限审敛法可知,所给级数收敛.
\end{solution}
\end{example}

\begin{example}
判定级数\(\sum\limits_{n=1}^\infty \sqrt{n+1} \left(1-\cos\frac{\pi}{n}\right)\)的收敛性.
\begin{solution}
因为\(1 - \cos x \sim \frac{1}{2} x^2\ (x\to0)\),故\[
\lim\limits_{n\to\infty} n^{3/2} \sqrt{n+1} \cdot \left(1-\cos\frac{\pi}{n}\right)
= \lim\limits_{n\to\infty} n^2 \sqrt{\frac{n+1}{n}} \cdot \frac{1}{2} \left(\frac{\pi}{n}\right)^2
= \frac{1}{2} \pi^2,
\]根据极限审敛法可知,所给级数收敛.
\end{solution}
\end{example}

\subsubsection{积分审敛法}
\begin{theorem}[积分审敛法]\label{theorem:无穷级数.积分审敛法}
设\(\sum\limits_{n=1}^\infty u_n\)为正项级数,又设非负函数\(f\)在\([1,+\infty)\)上单调递减,且\[
u_n = f(n)
\quad(n=1,2,\dotsc).
\]那么级数\(\sum\limits_{n=1}^\infty u_n\)收敛的充要条件是:反常积分\(\int_1^{+\infty} f(x) \dd{x}\)收敛.
\end{theorem}

\subsection{交错级数及其审敛法}
所谓\DefineConcept{交错级数}(alternating series)是这样的级数,
它的各项是正负交错的,从而可以写成下面的形式:\[
	u_1 - u_2 + u_3 - u_4 + \dotsb,
\]或\[
	-u_1 + u_2 - u_3 + u_4 - \dotsb,
\]
其中\(u_1,u_2,\dotsc\)都是正数.

\begin{theorem}[莱布尼茨定理]\label{theorem:无穷级数.莱布尼茨定理}
如果交错级数\(\sum\limits_{n=1}^\infty (-1)^{n-1} u_n\ (u_n>0)\)满足条件:
\begin{enumerate}
\item 级数的一般项的绝对值的数列是单调减少的,
即\(u_n \geq u_{n+1}\ (n=1,2,\dotsc)\);

\item 级数的一般项的绝对值的数列收敛于零,
即\(\lim\limits_{n\to\infty} {u_n}=0\),
\end{enumerate}
则级数收敛,且其和\(s \leq u_1\),
其余项\(r_n\)的绝对值\(\abs{r_n} \leq u_{n+1}\).
\end{theorem}

\begin{example}\label{example:无穷级数.交错级数1}
交错级数\[
1 - \frac{1}{2} + \frac{1}{3} - \frac{1}{4} + \dotsb + (-1)^{n-1} \frac{1}{n} + \dotsb
\]满足条件\begin{enumerate}
\item \(u_n = \frac{1}{n} > \frac{1}{n+1} = u_{n+1}\ (n=1,2,\dotsc)\);
\item \(\lim\limits_{n\to\infty} u_n = \lim\limits_{n\to\infty} \frac{1}{n} = 0\),
\end{enumerate}所以它是收敛的,且其和\(s < 1\).
如果取前\(n\)项的和\[
s_n = 1 - \frac{1}{2} + \frac{1}{3} - \dotsb + (-1)^{n-1} \frac{1}{n}
\]作为\(s\)的近似值,所产生的误差\(\abs{r_n} \leq \frac{1}{n+1}\).
\end{example}

\cref{theorem:无穷级数.莱布尼茨定理} 是判断交错级数收敛性的一个充分不必要定理.
\begin{example}
交错级数\[
\frac{1}{2} - \frac{1}{3}
+ \frac{1}{2^2} - \frac{1}{3^2}
+ \dotsm + \frac{1}{2^n} - \frac{1}{3^n}
\]是收敛的,这是因为它的前\(2n\)项和\begin{align*}
s_{2n} &= \frac{1}{2} - \frac{1}{3}
+ \frac{1}{2^2} - \frac{1}{3^2}
+ \dotsm + \frac{1}{2^n} - \frac{1}{3^n} \\
&= \left(\frac{1}{2} + \frac{1}{2^2} + \dotsm + \frac{1}{2^n}\right)
 - \left(\frac{1}{3} + \frac{1}{3^2} + \dotsm + \frac{1}{3^n}\right) \\
&= \left(1 - \frac{1}{2^n}\right)
 - \left(\frac{1}{2} - \frac{1}{2\cdot3^n}\right),
\end{align*}从而\[
\lim\limits_{n\to\infty} s_{2n} = \frac{1}{2}.
\]但是这级数在\(n>1\)时总有\[
u_{2n} = \frac{1}{3^n} < \frac{1}{2^{n+1}} = u_{2n+1},
\]不符合\cref{theorem:无穷级数.莱布尼茨定理} 的条件.
\end{example}

\begin{example}\label{example:交错级数.逐项平方以后发散的特例}
设\(u_n = \frac{(-1)^n}{\sqrt{n}}\).
试讨论级数\(\sum\limits_{n=1}^\infty u_n\)和\(\sum\limits_{n=1}^\infty u_n^2\)的敛散性.
\begin{proof}
因为\(u_n u_{n+1} < 0\),
\(\abs{u_n} > \abs{u_{n+1}}\),
\(\lim\limits_{n\to\infty} \abs{u_n} = 0\),
那么根据\hyperref[theorem:无穷级数.莱布尼茨定理]{莱布尼茨定理}可知
\(\sum\limits_{n=1}^\infty u_n\)收敛.
但是\(\sum\limits_{n=1}^\infty u_n^2
= \sum\limits_{n=1}^\infty \frac{1}{n}\)是调和级数,发散.
\end{proof}
\end{example}

从\cref{example:交错级数.逐项平方以后发散的特例} 可以得出以下结论.
\begin{proposition}
如果级数\(\sum\limits_{n=1}^\infty u_n\)和\(\sum\limits_{n=1}^\infty v_n\)都收敛,
那么级数\(\sum\limits_{n=1}^\infty u_n v_n\)可能收敛,也可能发散.
\end{proposition}

\subsection{绝对收敛与条件收敛}
\subsubsection{绝对收敛与条件收敛的概念}
\begin{definition}
设级数\(\sum\limits_{n=1}^\infty u_n\)的各项为任意实数.
如果级数\(\sum\limits_{n=1}^\infty u_n\)各项的绝对值所构成的正项级数
\(\sum\limits_{n=1}^\infty \abs{u_n}\)收敛,
则称级数\(\sum\limits_{n=1}^\infty u_n\) \DefineConcept{绝对收敛};
如果级数\(\sum\limits_{n=1}^\infty u_n\)收敛,
而级数\(\sum\limits_{n=1}^\infty \abs{u_n}\)发散,
则称级数\(\sum\limits_{n=1}^\infty u_n\) \DefineConcept{条件收敛}.
\end{definition}

\begin{theorem}\label{theorem:无穷级数.绝对收敛级数必定收敛}
如果级数\(\sum\limits_{n=1}^\infty u_n\)绝对收敛,
则级数\(\sum\limits_{n=1}^\infty u_n\)收敛.
\begin{proof}
令\[
	v_n = \frac{1}{2} (u_n + \abs{u_n})
	\quad(n=1,2,\dotsc).
\]
显然\(v_n \geq 0\)且\(v_n \leq \abs{u_n}\).
因级数\(\sum\limits_{n=1}^\infty \abs{u_n}\)收敛,
故由比较审敛法可知,
级数\(\sum\limits_{n=1}^\infty v_n\)收敛,
从而级数\(\sum\limits_{n=1}^\infty 2v_n\)也收敛.
而\(u_n = 2 v_n - \abs{u_n}\),
由收敛级数的基本性质可知\[
	\sum\limits_{n=1}^\infty u_n
	= \sum\limits_{n=1}^\infty 2v_n
	- \sum\limits_{n=1}^\infty \abs{u_n},
\]
所以级数\(\sum\limits_{n=1}^\infty u_n\)收敛.
\end{proof}
\end{theorem}

上面证明中引入的级数\(\sum\limits_{n=1}^\infty v_n\),其一般项\[
	v_n = \frac{1}{2} (u_n + \abs{u_n})
	= \left\{ \begin{array}{cl}
		u_n, & u_n > 0, \\
		0, & u_n \leq 0,
	\end{array} \right.
\]
可见级数\(\sum\limits_{n=1}^\infty v_n\)是
把级数\(\sum\limits_{n=1}^\infty u_n\)中的负项换成\(0\)得到的,
它也就是级数\(\sum\limits_{n=1}^\infty u_n\)中的全体正项所构成的级数.
类似可知,
令\[
	w_n = \frac{1}{2} (\abs{u_n} - u_n),
\]
则\(\sum\limits_{n=1}^\infty w_n\)为
级数\(\sum\limits_{n=1}^\infty u_n\)中全体负项的绝对值所构成的级数.

如果级数\(\sum\limits_{n=1}^\infty u_n\)绝对收敛,
则级数\(\sum\limits_{n=1}^\infty v_n\)和\(\sum\limits_{n=1}^\infty w_n\)都收敛;
如果级数\(\sum\limits_{n=1}^\infty u_n\)条件收敛,
则级数\(\sum\limits_{n=1}^\infty v_n\)和\(\sum\limits_{n=1}^\infty w_n\)都发散.

这就说明,对于一般的级数\(\sum\limits_{n=1}^\infty u_n\),
如果我们用正项级数的审敛法判定级数
\(\sum\limits_{n=1}^\infty \abs{u_n}\)收敛,
则原级数收敛.
这就使得一大类级数的收敛性判定问题,
转化成为正项级数的收敛性判定问题.

一般说来,
如果级数\(\sum\limits_{n=1}^\infty \abs{u_n}\)发散,
我们不能断定级数\(\sum\limits_{n=1}^\infty u_n\)也发散.
正如我们在\cref{example:无穷级数.交错级数1} 看到的那样,
虽然\(\sum\limits_{n=1}^\infty \frac{1}{n}\)发散,
但是\(\sum\limits_{n=1}^\infty \frac{(-1)^{n+1}}{n}\)收敛.
但是,如果我们用比值审敛法或根值审敛法,
根据\(\lim\limits_{n\to\infty} \abs{\frac{u_{n+1}}{u_n}} > 1\)
或\(\lim\limits_{n\to\infty} \sqrt[n]{\abs{u_n}} > 1\)
判定级数\(\sum\limits_{n=1}^\infty \abs{u_n}\)发散,
那么我们可以断定级数\(\sum\limits_{n=1}^\infty u_n\)必然发散.

\begin{theorem}\label{theorem:无穷级数.绝对发散的特殊情况}
当级数\(\sum\limits_{n=1}^\infty u_n\)满足\[
	\lim\limits_{n\to\infty} \abs{\frac{u_{n+1}}{u_n}} = \rho > 1
	\quad\text{或}\quad
	\lim\limits_{n\to\infty} \sqrt[n]{\abs{u_n}} = \rho > 1
\]时,
这个级数必定发散.
\begin{proof}
这是因为从\(\rho > 1\)可推知\(\abs{u_n} \not\to 0\ (n\to\infty)\),
从而\(u_n \not\to 0\ (n\to\infty)\),
因此级数\(\sum\limits_{n=1}^\infty u_n\)是发散的.
\end{proof}
\end{theorem}

\begin{example}
已知\(a_n < b_n\ (n=1,2,\dotsc)\),
若级数\(\sum\limits_{n=1}^\infty a_n\)与\(\sum\limits_{n=1}^\infty b_n\)均收敛,
证明:“\(\sum\limits_{n=1}^\infty a_n\)绝对收敛”是
“\(\sum\limits_{n=1}^\infty b_n\)绝对收敛”的充要条件.
\begin{proof}
由题可知级数\(\sum\limits_{n=1}^\infty (b_n - a_n)\)是收敛的正项级数,因而绝对收敛.

当级数\(\sum\limits_{n=1}^\infty a_n\)绝对收敛时,
由\hyperref[theorem:不等式.三角不等式1]{三角不等式}有\[
	\abs{b_n} = \abs{(b_n - a_n) + a_n}
	\leq \abs{b_n - a_n} + \abs{a_n},
\]
那么由\hyperref[theorem:无穷级数.正项级数的比较审敛法]{比较审敛法}可知,
\(\sum\limits_{n=1}^\infty \abs{b_n}\)收敛,
\(\sum\limits_{n=1}^\infty b_n\)绝对收敛.

同理,当级数\(\sum\limits_{n=1}^\infty b_n\)绝对收敛时,
亦有\(\sum\limits_{n=1}^\infty a_n\)绝对收敛.
\end{proof}
\end{example}

\subsubsection{绝对收敛级数的性质}
绝对收敛级数有很多性质是条件收敛级数所没有的.

\begin{property}[绝对收敛级数的可交换性]\label{theorem:无穷级数.绝对收敛级数的可交换性}
%@see: 《高等数学(第六版 下册)》 P265. 定理9
绝对收敛级数经改变项的位置后构成的级数也收敛,且与原级数有相同的和.
\end{property}

\begin{definition}\label{definition:无穷级数.绝对收敛级数的柯西乘积}
定义级数\(\sum\limits_{n=1}^\infty u_n\)
和\(\sum\limits_{n=1}^\infty v_n\)的\DefineConcept{柯西乘积}为\[
	\left( \sum\limits_{n=1}^\infty u_n \right)
	\cdot
	\left( \sum\limits_{n=1}^\infty v_n \right)
	\defeq
	\sum_{n=1}^\infty \sum_{k=1}^n u_k v_{n-k+1}.
\]
\end{definition}

\begin{theorem}\label{theorem:无穷级数.绝对收敛级数的柯西乘积必收敛}
%@see: 《高等数学(第六版 下册)》 P267. 定理10
设级数\(\sum\limits_{n=1}^\infty u_n\)和\(\sum\limits_{n=1}^\infty v_n\)都绝对收敛,
其和分别为\(s\)和\(\sigma\),
则它们的柯西乘积也是绝对收敛的,且其和为\(s \cdot \sigma\).
\end{theorem}

\begin{proposition}\label{theorem:绝对收敛.命题1}
若级数\(\sum\limits_{n=1}^\infty u_n\)绝对收敛,
级数\(\sum\limits_{n=1}^\infty v_n\)条件收敛,
则级数\(\sum\limits_{n=1}^\infty (u_n \pm v_n)\)条件收敛.
\begin{proof}[证法1]
因为级数\(\sum\limits_{n=1}^\infty u_n\)和\(\sum\limits_{n=1}^\infty v_n\)都收敛,
所以,由\cref{theorem:无穷级数.收敛级数性质2} 可知,
它们相加或相减所得级数\(\sum\limits_{n=1}^\infty (u_n \pm v_n)\)一定收敛.

因为级数\(\sum\limits_{n=1}^\infty v_n\)条件收敛,
所以级数\(\sum\limits_{n=1}^\infty \abs{v_n}\)发散;
由三角不等式 \labelcref{theorem:不等式.三角不等式1} 有\begin{align*}
	\abs{v_n}
	= \abs{(v_n + u_n) - u_n}
	\leq \abs{u_n + v_n} + \abs{u_n}, \\
	\abs{v_n}
	= \abs{(v_n - u_n) + u_n}
	\leq \abs{u_n - v_n} + \abs{u_n};
\end{align*}
于是由\cref{theorem:无穷级数.正项级数的比较审敛法} 可知
级数\(\sum\limits_{n=1}^\infty \left( \abs{u_n \pm v_n} + \abs{u_n} \right)\)发散.

又因为级数\(\sum\limits_{n=1}^\infty u_n\)绝对收敛,
也就是说级数\(\sum\limits_{n=1}^\infty \abs{u_n}\)收敛.
那么由\cref{theorem:无穷级数.收敛级数性质2.推论1} 可知,
级数\[
	\sum\limits_{n=1}^\infty \left[
		\left( \abs{u_n \pm v_n} + \abs{u_n} \right) - \abs{u_n}
	\right]
	= \sum\limits_{n=1}^\infty \abs{u_n \pm v_n}
\]发散.

综上所述,级数\(\sum\limits_{n=1}^\infty (u_n \pm v_n)\)条件收敛.
\end{proof}
\begin{proof}[证法2]
用反证法.
假设级数\(\sum\limits_{n=1}^\infty (u_n \pm v_n)\)绝对收敛,
那么级数\(\sum\limits_{n=1}^\infty \abs{u_n \pm v_n}\)收敛;
加之级数\(\sum\limits_{n=1}^\infty u_n\)绝对收敛,
或者说级数\(\sum\limits_{n=1}^\infty \abs{u_n}\)收敛;
所以由\cref{theorem:无穷级数.收敛级数性质2} 可知,
级数\(\sum\limits_{n=1}^\infty (\abs{u_n + v_n} + \abs{u_n})\)收敛.

又因为\[
	\abs{v_n}
	= \abs{(v_n + u_n) - u_n}
	\leq \abs{u_n + v_n} + \abs{u_n},
\]
所以由\cref{theorem:无穷级数.正项级数的比较审敛法} 可知,
级数\(\sum\limits_{n=1}^\infty \abs{v_n}\)收敛,
也就是说级数\(\sum\limits_{n=1}^\infty v_n\)绝对收敛,矛盾!
因此级数\(\sum\limits_{n=1}^\infty (u_n \pm v_n)\)条件收敛.
\end{proof}
\end{proposition}

\begin{proposition}\label{theorem:绝对收敛.命题2}
设正项级数\(\sum\limits_{n=1}^\infty u_n\)收敛,
数列\(\{v_n\}\)有界,
则级数\(\sum\limits_{n=1}^\infty u_n v_n\)绝对收敛.
\begin{proof}
设\(\abs{v_n} < M\ (n=1,2,\dotsc)\),
那么\[
	\abs{u_n v_n}
	= \abs{u_n} \abs{v_n}
	< M \abs{u_n}
	= M u_n
	\quad(n=1,2,\dotsc),
\]
于是根据\hyperref[theorem:无穷级数.正项级数的比较审敛法的推论]{正项级数的比较审敛法}可知,
级数\(\sum\limits_{n=1}^\infty u_n v_n\)绝对收敛.
\end{proof}
\end{proposition}

\begin{proposition}\label{theorem:绝对收敛.命题3}
设正项级数\(\sum\limits_{n=1}^\infty u_n\)和\(\sum\limits_{n=1}^\infty v_n\)都收敛,
则级数\(\sum\limits_{n=1}^\infty u_n v_n\)绝对收敛.
\end{proposition}

\begin{proposition}\label{theorem:绝对收敛.命题4}
设级数\(\sum\limits_{n=1}^\infty u_n\)收敛,
\(\lim\limits_{n\to\infty} v_n = 0\),
则级数\(\sum\limits_{n=1}^\infty u_n v_n\)绝对收敛.
\begin{proof}
根据\cref{theorem:极限.收敛数列的有界性},
数列\(\{v_n\}\)有界.
再根据\cref{theorem:绝对收敛.命题2} 可知
级数\(\sum\limits_{n=1}^\infty u_n v_n\)绝对收敛.
\end{proof}
\end{proposition}

\begin{proposition}\label{theorem:绝对收敛.命题5}
设级数\(\sum\limits_{n=1}^\infty u_n\)绝对收敛,
级数\(\sum\limits_{n=1}^\infty v_n\)条件收敛,
则级数\(\sum\limits_{n=1}^\infty u_n v_n\)绝对收敛.
\end{proposition}
