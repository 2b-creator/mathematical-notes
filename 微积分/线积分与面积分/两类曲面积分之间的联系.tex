\section{两类曲面积分之间的联系}
设有向曲面\(\Sigma\)由方程\(z = z(x,y)\)给出,\(\Sigma\)在\(xOy\)面上的投影区域为\(D_{xy}\),函数\(z = z(x,y)\)在\(D_{xy}\)上具有一阶连续偏导数,\(R(x,y,z)\)在\(\Sigma\)上连续.
如果\(\Sigma\)取上侧,则由对坐标的曲面积分计算公式有\[
\iint_\Sigma R(x,y,z) \dd{x}\dd{y} = \iint_{D_{xy}} R[x,y,z(x,y)] \dd{x}\dd{y}.
\]

另一方面,因上述有向曲面\(\Sigma\)的法向量的方向余弦为\[
\cos\alpha=\frac{-z'_x}{\sqrt{1+(z'_x)^2+(z'_y)^2}},
\]\[
\cos\beta=\frac{-z'_y}{\sqrt{1+(z'_x)^2+(z'_y)^2}},
\]\[
\cos\gamma=\frac{1}{\sqrt{1+(z'_x)^2+(z'_y)^2}},
\]故由对面积的曲面积分计算公式有\[
\iint_\Sigma R(x,y,z) \cos\gamma \dd{S}
= \iint_{D_{xy}} R[x,y,z(x,y)] \dd{x}\dd{y}.
\]由此可见,有\[
\iint_\Sigma R(x,y,z) \dd{x}\dd{y}
= \iint_\Sigma R(x,y,z) \cos\gamma \dd{S}.
\eqno{(1)}
\]

如果\(\Sigma\)取下侧,则\[
\iint_\Sigma R(x,y,z) \cos\gamma \dd{S}
= -\iint_{D_{xy}} R[x,y,z(x,y)] \dd{x}\dd{y}.
\]但这时\(\cos\gamma=\frac{-1}{\sqrt{1+(z'_x)^2+(z'_y)^2}}\),因此(1)式仍成立.

类似地可推得\[
\iint_\Sigma P(x,y,z) \dd{y}\dd{z}
= \iint_{D_{yz}} P(x,y,z) \cos\alpha \dd{S},
\eqno{(2)}
\]\[
\iint_\Sigma Q(x,y,z) \dd{z}\dd{x}
= \iint_{D_{zx}} Q(x,y,z) \cos\beta \dd{S}.
\eqno{(3)}
\]

合并(1)、(2)、(3)式,得两类曲面积分之间的如下联系:
\begin{equation}\label{equation:线积分与面积分.两类曲面积分之间的联系}
\iint_\Sigma P \dd{y}\dd{z} + Q \dd{z}\dd{x} + R \dd{x}\dd{y}
=\iint_\Sigma (P\cos\alpha+Q\cos\beta+R\cos\gamma) \dd{S},
\end{equation}
其中\(\cos\alpha\)、\(\cos\beta\)、\(\cos\gamma\)是有向曲面\(\Sigma\)在点\((x,y,z)\)处的法向量的方向余弦.

两类曲面积分之间的联系也可写成如下的向量形式:
\begin{equation}\label{equation:线积分与面积分.两类曲面积分之间的联系的向量形式}
\iint_\Sigma \vb{A} \cdot \dd{\vb{S}}
=\iint_\Sigma \vb{A} \cdot \vb{n} \dd{S},
\end{equation}
其中\(\vb{A}=\opair{P,Q,R}\),
\(\vb{n}=(\cos\alpha,\cos\beta,\cos\gamma)\)为有向曲面\(\Sigma\)在点\((x,y,z)\)处的单位法向量,
\(\dd{\vb{S}} = \vb{n}\dd{S} = \opair{\dd{y}\dd{z},\dd{z}\dd{x},\dd{x}\dd{y}}\)称为\DefineConcept{有向曲面元}.

\begin{example}
\def\ys{\iint_\Sigma (z^2+x) \dd{y}\dd{z} - z \dd{x}\dd{y}}%
计算曲面积分\(\ys\),其中\(\Sigma\)是旋转抛物面\(z = \frac{1}{2}(x^2+y^2)\)介于平面\(z=0\)和\(z=2\)之间的部分的下侧.
\begin{solution}
由两类曲面积分之间的联系 \labelcref{equation:线积分与面积分.两类曲面积分之间的联系},可得\[
\iint_\Sigma (z^2+x) \dd{y}\dd{z}
= \iint_\Sigma (z^2+x) \cos\alpha \dd{S}
= \iint_\Sigma (z^2+x) \frac{\cos\alpha}{\cos\gamma} \dd{x}\dd{y}.
\]在曲面\(\Sigma\)上,有\[
\cos\alpha
= \frac{x}{\sqrt{1+x^2+y^2}},
\qquad
\cos\gamma
= \frac{-1}{\sqrt{1+x^2+y^2}}.
\]故\[
\ys
= \iint_\Sigma [(z^2+x)(-x) - z] \dd{x}\dd{y}.
\]
再按对坐标的曲面积分的计算法,便得\[
\ys
= - \iint_{D_{xy}} \left\{
	\left[
		\frac{1}{4} (x^2+y^2)^2
		+ x
	\right] \cdot (-x)
	- \frac{1}{2} (x^2+y^2)
\right\} \dd{x}\dd{y}.
\]因为\(\iint_{D_{xy}} \frac{1}{4} x(x^2+y^2)^2 \dd{x}\dd{y} = 0\),所以\begin{align*}
\ys
&= \iint_{D_{xy}} \left[x^2+\frac{1}{2}(x^2+y^2)\right] \dd{x}\dd{y} \\
&= \int_0^{2\pi} \dd{\theta} \int_0^2 \left(\rho^2 \cos^2\theta + \frac{1}{2} \rho^2\right) \rho \dd{\rho}
= 8\pi.
\end{align*}
\end{solution}
\end{example}
