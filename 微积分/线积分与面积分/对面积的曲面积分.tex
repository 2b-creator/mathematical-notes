\section{对面积的曲面积分}
\subsection{对面积的曲面积分的概念}
\begin{definition}
设曲面\footnote{以后都假定曲面的边界曲线是分段光滑的闭曲线,且曲面有界.}%
\(\Sigma\)是光滑的\footnote{如果曲面上各点处都具有切平面,则称该曲面是光滑的.
如果曲面是由有限个光滑曲面所组成的,则称该曲面是分片光滑的.}.
函数\(f(x,y,z)\)在\(\Sigma\)上有界.
把\(\Sigma\)任意分成\(n\)小块\(\increment S_i\)
(\(\increment S_i\)同时也代表第\(i\)小块曲面的面积),
设\((\xi_i,\eta_i,\zeta_i)\)是\(\increment S_i\)上任取的一点,
作乘积\(f(\xi_i,\eta_i,\zeta_i) \increment S_i\ (i=1,2,\dotsc,n)\),
并作和\(\sum_{i=1}^n f(\xi_i,\eta_i,\zeta_i) \increment S_i\),
如果当各小块曲面的直径\footnote{%
曲面的直径是指曲面上任意两点间距离的最大者.}的最大值\(\lambda\to0\)时,
这和的极限总存在,
则称此极限为
“函数\(f(x,y,z)\)在曲面\(\Sigma\)上\DefineConcept{对面积的曲面积分}”
或“函数\(f(x,y,z)\)在曲面\(\Sigma\)上的\DefineConcept{第一类曲面积分}%
(type I surface integral)”,
记作\(\iint_\Sigma f(x,y,z) \dd{S}\),
即\[
	\iint_\Sigma f(x,y,z)\dd{S}
	\defeq
	\lim_{\lambda\to0} \sum_{i=1}^n f(\xi_i,\eta_i,\zeta_i) \increment S_i,
\]
其中\(f(x,y,z)\)叫做\DefineConcept{被积函数},
\(\Sigma\)叫做\DefineConcept{积分曲面}.

如果曲面\(\Sigma\)是分片光滑的,
我们规定:
函数在\(\Sigma\)上对面积的曲面积分,
等于函数在光滑的各片曲面上对面积的曲面积分之和.
\end{definition}
我们指出,当被积函数\(f(x,y,z)\)在光滑曲面\(\Sigma\)上连续时,对面积的曲面积分是存在的.

\subsection{对面积的曲面积分的性质}
由对面积的曲面积分的定义可知,它具有与对弧长的曲线积分相类似的性质,这里不再赘述.

\subsection{对面积的曲面积分的计算法}
\begin{theorem}
设积分曲面\(\Sigma\)由方程\(z=z(x,y)\)给出,
\(\Sigma\)在\(xOy\)面上的投影区域为\(D_{xy}\),
函数\(z=z(x,y)\)在\(D_{xy}\)上具有连续偏导数,
被积函数\(f(x,y,z)\)在\(\Sigma\)上连续.
那么有\[
	\iint_\Sigma f(x,y,z) \dd{S}
	=\iint_{D_{xy}} f[x,y,z(x,y)] \sqrt{1+[z'_x(x,y)]^2+[z'_y(x,y)]^2} \dd{x}\dd{y}.
\]
\end{theorem}
这就是把对面积的曲面积分化为二重积分的公式.
在计算时,只要把变量\(z\)换为\(z(x,y)\),\(\dd{S}\)换为
\(\sqrt{1+(z'_x)^2+(z'_y)^2} \dd{x}\dd{y}\),
再确定\(\Sigma\)在\(xOy\)面上的投影区域\(D_{xy}\),
这样就把对面积的曲面积分化为二重积分了.

如果积分曲面\(\Sigma\)由方程\(x=x(y,z)\)或\(y=y(z,x)\)给出,
也可类似地把对面积的曲面积分化为相应的二重积分.

可以将上述对面积的曲面积分的计算法总结为如下更一般的方法.
\begin{theorem}
设积分曲面\(\Sigma\)由二元向量值参数方程\[
	\vb{r}(u,v) = x(u,v) \vb{i} + y(u,v) \vb{j} + z(u,v) \vb{k},
	\quad (u,v) \in D
\]给出,
这里\(D\)是\(uv\)平面上具有分段光滑边界的有界闭区域,
函数\(\vb{r}(u,v)\)在\(D\)上具有连续偏导数,
被积函数\(f(x,y,z)\)在\(\Sigma\)上连续,
那么有\[
	\iint_\Sigma f(x,y,z) \dd{S}
	= \iint_D f(\vb{r}(u,v)) \cdot \abs{\vb{r}'_u \times \vb{r}'_v} \dd{\sigma}.
\]
\end{theorem}
这里,参数\(u\)、\(v\)可以代换为\(x\)、\(y\)、\(z\)中任意两个,
区域\(D\)可以代换为\(D_{xy}\)、\(D_{yz}\)或\(D_{zx}\).

\begin{example}
计算曲面积分\(\iint_\Sigma \frac{\dd{S}}{z}\),
其中\(\Sigma\)是球面\(x^2+y^2+z^2=a^2\)被平面\(z = h\ (0<h<a)\)截出的顶部.
\begin{solution}
\(\Sigma\)的方程为\(z = \sqrt{a^2-x^2-y^2}\).
\(\Sigma\)在\(xOy\)面上的投影区域\(D_{xy}\)为圆形闭区域\[
	\Set{(x,y) \given x^2+y^2 \leq a^2-h^2}.
\]
又\[
	\sqrt{1+(z'_x)^2+(z'_y)^2} = \frac{a}{\sqrt{a^2-x^2-y^2}},
\]
所以\[
	\iint_\Sigma \frac{\dd{S}}{z}
	= \iint_{D_{xy}} \frac{a\dd{x}\dd{y}}{a^2-x^2-y^2}.
\]
利用极坐标,得\[
	\iint_\Sigma \frac{\dd{S}}{z}
	= \iint_{D_{xy}} \frac{a\rho\dd{\rho}\dd{\theta}}{a^2-\rho^2}
	= a \int_0^{2\pi} \dd{\theta} \int_0^{\sqrt{a^2-h^2}} \frac{\rho\dd{\rho}}{a^2-\rho^2}
	= 2\pi a \ln\frac{a}{h}.
\]
\end{solution}
\end{example}

\begin{example}
计算\(\varoiint_\Sigma xyz \dd{S}\),
其中\(\Sigma\)是由平面\(x=0\)、\(y=0\)、\(z=0\)及\(x+y+z=1\)所围成的四面体的整个边界曲面.
\begin{solution}
整个边界曲面\(\Sigma\)在平面\(x=0\)、\(y=0\)、\(z=0\)及\(x+y+z=1\)上的部分
依次记为\(\AutoTuple{\Sigma}{4}\),于是\[
	\varoiint_\Sigma xyz \dd{S}
	= \varoiint_{\Sigma_1} xyz \dd{S}
	+ \varoiint_{\Sigma_2} xyz \dd{S}
	+ \varoiint_{\Sigma_3} xyz \dd{S}
	+ \varoiint_{\Sigma_4} xyz \dd{S}.
\]
由于在\(\AutoTuple{\Sigma}{3}\)上,被积函数\(f(x,y,z)=xyz\)均为零,所以\[
	\varoiint_{\Sigma_1} xyz \dd{S}
	= \varoiint_{\Sigma_2} xyz \dd{S}
	= \varoiint_{\Sigma_3} xyz \dd{S}
	= 0.
\]

在\(\Sigma_4\)上,\(z=1-x-y\),\(z'_x = z'_y = -1\),所以\[
	\sqrt{1+(z'_x)^2+(z'_y)^2}
	= \sqrt{1+(-1)^2+(-1)^2}
	= \sqrt{3},
\]
从而\[
	\varoiint_\Sigma xyz \dd{S}
	= \varoiint_{\Sigma_4} xyz \dd{S}
	= \iint_{D_{xy}} \sqrt{3} xy (1-x-y) \dd{x} \dd{y},
\]
其中\(D_{xy}\)是\(\Sigma_4\)在\(xOy\)面上的投影区域,
即由直线\(x=0\)、\(y=0\)及\(x+y=1\)所围成的闭区域.
因此\[
	\varoiint_\Sigma xyz \dd{S}
	= \sqrt{3} \int_0^1 x \dd{x} \int_0^{1-x} y (1-x-y) \dd{y}
	= \frac{\sqrt{3}}{120}.
\]
\end{solution}
\end{example}

\begin{example}
设点\(P\)是椭球面\(S: x^2 + y^2 + z^2 - yz = 1\)上的动点,
若\(S\)在点\(P\)处的切平面与\(xOy\)面垂直,求点\(P\)的轨迹\(C\),并计算曲面积分\[
	I = \iint_\Sigma \frac{(x+\sqrt{3}) \abs{y-2z}}{\sqrt{4+y^2+z^2-4yz}} \dd{S},
\]
其中\(\Sigma\)是椭球面\(S\)位于曲线\(C\)上方的部分.
\begin{solution}
设\(F(x,y,z) = x^2 + y^2 + z^2 - yz - 1\).
求导得\[
	F'_x = 2x, \qquad
	F'_y = 2y - z, \qquad
	F'_z = 2z - y,
\]
那么\(S\)在点\(P(x,y,z)\)处的切向量为\(\vb{n} = \opair{2x,2y-z,2z-y}\).
因为点\(P\)处的切平面与\(xOy\)面垂直,
所以法向量\(\vb{n}\)与\(xOy\)面的法向量\(\opair{0,0,1}\)垂直,\[
	2x\cdot0 + (2y-z)\cdot0+(2z-y)\cdot1=0
	\quad\text{或}\quad
	2z-y=0.
\]
因此,轨迹\(C\)的方程是\[
	\begin{cases}
		x^2 + y^2 + z^2 - yz = 1, \\
		2z-y=0.
	\end{cases}
\]
消去轨迹\(C\)方程中的\(z\),得到\(C\)在\(xOy\)面上的投影为\[
	C_{xy}: x^2+\frac{3}{4}y^2=1.
\]
又因为\[
	z'_x = - \frac{F'_x}{F'_z} = \frac{2x}{y-2z}, \qquad
	z'_y = - \frac{F'_y}{F'_z} = \frac{2y-z}{y-2z},
\]
\begin{align*}
	\dd{S} &= \sqrt{1+(z'_x)^2+(z'_y)^2} \dd{x}\dd{y} \\
	&= \sqrt{1+\left(\frac{2x}{y-2z}\right)^2+\left(\frac{2y-z}{y-2z}\right)^2} \dd{x}\dd{y} \\
	&= \frac{\sqrt{4x^2+5y^2+5z^2-8yz}}{\abs{y-2z}} \dd{x}\dd{y} \\
	&= \frac{\sqrt{4+y^2+z^2-4yz}}{\abs{y-2z}} \dd{x}\dd{y},
\end{align*}
所以\begin{align*}
	&\hspace{-20pt}
	\iint_\Sigma \frac{(x+\sqrt{3}) \abs{y-2z}}{\sqrt{4+y^2+z^2-4yz}} \dd{S} \\
	&= \iint_{x^2+\frac{3}{4}y^2\leq1} \frac{(x+\sqrt{3}) \abs{y-2z}}{\sqrt{4+y^2+z^2-4yz}}
		\cdot \frac{\sqrt{4+y^2+z^2-4yz}}{\abs{y-2z}} \dd{x}\dd{y} \\
	&= \iint_{x^2+\frac{3}{4}y^2\leq1} (x+\sqrt{3}) \dd{x}\dd{y} \\
	&= \sqrt{3} \iint_{x^2+\frac{3}{4}y^2\leq1} \dd{x}\dd{y}
	= 2\pi.
\end{align*}
\end{solution}
\end{example}

\subsection{利用对称性简化第一类曲面积分的计算}
设积分曲面\(\Sigma\)关于\(xOy\)面对称.
若被积函数\(f(x,y,z)\)是关于\(z\)的偶函数,
即\(f(x,y,-z) = f(x,y,z)\),则\[
	\iint_\Sigma f(x,y,z) \dd{S}
	= 2 \iint_{D_{xy}} f[x,y,z(x,y)] \sqrt{1+(z'_x)^2+(z'_y)^2} \dd{x}\dd{y};
\]
若被积函数\(f(x,y,z)\)是关于\(z\)的奇函数,
即\(f(x,y,-z) = -f(x,y,z)\),则\[
	\iint_\Sigma f(x,y,z) \dd{S} = 0.
\]

类似地,可以得出,在积分曲面关于\(yOz\)面或\(zOx\)面对称,
而被积函数\(f(x,y,z)\)是关于\(x\)或\(y\)的奇函数或偶函数的情形下,
第一类曲面积分的简化计算公式.

\begin{example}
计算\(\varoiint_\Sigma x^2 \dd{S}\),其中\(\Sigma: x^2+y^2+z^2=a^2\).
\begin{solution}
利用对称性可得\begin{align*}
	\varoiint_\Sigma x^2 \dd{S}
	&= \frac{1}{3} \varoiint_\Sigma (x^2+y^2+z^2) \dd{S} \\
	&= \frac{1}{3} \varoiint_\Sigma a^2 \dd{S}
	= \frac{a^2}{3} \varoiint_\Sigma \dd{S} \\
	&= \frac{a^2}{3} \cdot 4\pi a^2
	= \frac{4}{3} \pi a^4.
\end{align*}
\end{solution}
\end{example}
