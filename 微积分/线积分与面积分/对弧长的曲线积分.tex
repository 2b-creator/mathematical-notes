\section{对弧长的曲线积分}
\subsection{对弧长的曲线积分的概念}
\begin{definition}
设\(L\)为\(xOy\)面内一条光滑曲线弧,函数\(f(x,y)\)在\(L\)上有界.
在\(L\)上任意插入一点列\(M_1,M_2,\dotsc,M_{n-1}\)把\(L\)分成\(n\)个小段.
设第\(i\)个小段的长度为\(\increment s_i\).
又\(\opair{\xi_i,\eta_i}\)为第\(i\)个小段上任意取定的一点,作乘积\[
f(\xi_i,\eta_i) \increment s_i \quad(i=1,2,\dotsc,n),
\]并作和\(\sum_{i=1}^n f(\xi_i,\eta_i) \increment s_i\),如果当各小弧段的长度的最大值\(\lambda\to0\)时,这和的极限总存在,则称此极限为函数\(f(x,y)\)在曲线弧\(L\)上\DefineConcept{对弧长的曲线积分}或\DefineConcept{第一类曲线积分},记作\(\int_L f(x,y) \dd{s}\),即\[
\int_L f(x,y) \dd{s}
= \lim_{\lambda\to0} \sum_{i=1}^n f(\xi_i,\eta_i) \increment s_i,
\]其中\(f(x,y)\)叫做\DefineConcept{被积函数},\(L\)叫做\DefineConcept{积分弧段}.

类似地,函数\(f(x,y,z)\)在空间曲线弧\(\Gamma\)上对弧长的曲线积分定义为\[
\int_\Gamma f(x,y,z) \dd{s}
=\lim_{\lambda\to0} \sum_{i=1}^n f(\xi_i,\eta_i,\zeta_i) \increment s_i.
\]

如果平面曲线\(L\)或空间曲线\(\Gamma\)是分段光滑%
\footnote{%
所谓“分段光滑”,是指积分弧段\(L\)(或\(\Gamma\))可以分成有限段,而每一段都是光滑的.%
}%
的,我们规定函数在函数在曲线弧\(L\)或\(\Gamma\)上对弧长的曲线积分等于在光滑的各段上对弧长的曲线积分之和.
例如,设\(L\)可分成两段光滑曲线弧\(L_1\)及\(L_2\)(记作\(L=L_1+L_2\)),就规定
\[
\int_{L_1+L_2} f(x,y) \dd{s}
= \int_{L_1} f(x,y) \dd{s}
+ \int_{L_2} f(x,y) \dd{s}.
\]
有时候也将上式右端写成\[
\left( \int_{L_1} + \int_{L_2} \right) f(x,y) \dd{s}.
\]

如果\(L\)是闭曲线,那么函数\(f(x,y)\)在闭曲线\(L\)上对弧长的曲线积分记为\[
\oint_L f(x,y) \dd{s}.
\]
\end{definition}

\subsection{对弧长的曲线积分的性质}
\begin{property}\label{theorem:线积分与面积分.第一类曲线积分性质1}
设\(\alpha\)、\(\beta\)为常数,则\[
\int_L [\alpha f(x,y) + \beta g(x,y)] \dd{s}
= \alpha \int_L f(x,y) \dd{s}
+ \beta \int_L g(x,y) \dd{s}.
\]
\end{property}

\begin{property}\label{theorem:线积分与面积分.第一类曲线积分性质2}
若积分弧段\(L\)可分成两段光滑曲线弧\(L_1\)和\(L_2\),则\[
\int_L f(x,y) \dd{s}
=\int_{L_1} f(x,y) \dd{s}
+\int_{L_2} f(x,y) \dd{s}.
\]
\end{property}

\begin{property}\label{theorem:线积分与面积分.第一类曲线积分性质3}
设在\(L\)上\(f(x,y) \leq g(x,y)\),则\[
\int_L f(x,y) \dd{s}
\leq
\int_L g(x,y) \dd{s}.
\]

特别地,有\[
\abs{\int_L f(x,y) \dd{s}} \leq \int_L \abs{f(x,y)} \dd{s}.
\]
\end{property}

\subsection{对弧长的曲线积分的计算法}
\begin{theorem}
设二元函数\(f(x,y)\)在平面曲线弧\(L\)上有定义且连续,\(L\)的参数方程为\[
\left\{ \begin{array}{l}
x = \phi(t), \\
y = \psi(t) \\
\end{array} \right.
\qquad
(\alpha \leq t \leq \beta),
\]其中\(\phi(t)\)、\(\psi(t)\)在\([\alpha,\beta]\)上具有一阶连续导数,
且\([\phi'(t)]^2+[\psi'(t)]^2 \neq 0\),
则曲线积分\(\int_L f(x,y) \dd{s}\)存在,
且\begin{equation}\label{equation:线积分与面积分.第一类曲线积分的计算式1}
\int_L f(x,y) \dd{s}
= \int_\alpha^\beta f[\phi(t),\psi(t)] \sqrt{[\phi'(t)]^2+[\psi'(t)]^2} \dd{t}
\quad(\alpha<\beta).
\end{equation}
\begin{proof}
假定当参数\(t\)由\(\alpha\)变至\(\beta\)时,
\(L\)上的点\(M(x,y)\)依点\(A\)至点\(B\)的方向描出曲线\(L\).
在\(L\)上取一列点\[
A=M_0,M_1,M_2,\dotsc,M_{n-1},M_n=B,
\]它们对应于一列单调增加的参数值\[
\alpha=t_0<t_1<t_2<\dotsb<t_{n-1}<t_n=\beta.
\]根据对弧长的曲线积分的定义,有\[
\int_L f(x,y) \dd{s} = \lim_{\lambda\to0} \sum_{i=1}^n f(\xi_i,\eta_i) \increment s_i.
\]设点\(\opair{\xi_i,\eta_i}\)对应于参数值\(\tau_i\),
即\(\xi_i=\phi(\tau_i)\)、\(\eta_i=\psi(\tau_i)\),
这里\(t_{i-1}\leq\tau_i\leq t_i\).
由于\[
\increment s_i = \int_{t_{i-1}}^{t_i} \sqrt{[\phi'(t)]^2+[\psi'(t)]^2} \dd{t},
\]应用积分中值定理,有\[
\increment s_i = \sqrt{[\phi'(\tau'_i)]^2+[\psi'(\tau'_i)]^2} \increment t_i,
\]其中\(\increment t_i = t_i - t_{i-1}\),
\(t_{i-1} \leq \tau'_i \leq t_i\).
于是\[
\int_L f(x,y) \dd{s}
= \lim_{\lambda\to0} \sum_{i=1}^n f[\phi(\tau_i),\psi(\tau_i)] \sqrt{[\phi'(\tau'_i)]^2+[\psi'(\tau'_i)]^2} \increment t_i.
\]由于函数\(\sqrt{[\phi'(t)]^2+[\psi'(t)]^2}\)在闭区间\([\alpha,\beta]\)上连续,
我们可以把上式中的\(\tau'_i\)换成\(\tau_i\)
\footnote{这里利用了函数\(\sqrt{[\phi'(t)]^2+[\psi'(t)]^2}\)在闭区间\([\alpha,\beta]\)上的一致连续性.},
从而\[
\int_L f(x,y) \dd{s}
= \lim_{\lambda\to0} \sum_{i=1}^n f[\phi(\tau_i),\psi(\tau_i)] \sqrt{[\phi'(\tau_i)]^2+[\psi'(\tau_i)]^2} \increment t_i.
\]上式右端的和的极限就是函数\(f[\phi(t),\psi(t)] \sqrt{[\phi'(t)]^2+[\psi'(t)]^2}\)在区间\([\alpha,\beta]\)上的定积分,
由于这个函数在\([\alpha,\beta]\)上连续,
所以这个定积分是存在的,
因此上式左端的曲线积分\(\int_L f(x,y) \dd{s}\)也存在,
并且有\[
\int_L f(x,y) \dd{s}
=\int_\alpha^\beta
 f[\phi(t),\psi(t)]
 \sqrt{[\phi'(t)]^2+[\psi'(t)]^2}
 \dd{t}
\quad(\alpha<\beta).
\qedhere
\]
\end{proof}
\end{theorem}
\cref{equation:线积分与面积分.第一类曲线积分的计算式1} 表明,
计算对弧长的曲线积分\(\int_L f(x,y) \dd{s}\)时,
只要把\(x\)、\(y\)、\(\dd{s}\)依次换为\(\phi(t)\)、\(\psi(t)\)、\(\sqrt{[\phi'(t)]^2+[\psi'(t)]^2} \dd{t}\),
然后从\(\alpha\)到\(\beta\)作定积分就行了,
这里必须注意,定积分的下限\(\alpha\)一定要小于上限\(\beta\)!
这是因为,从上述推导中可以看出,由于小弧段的长度\(\increment s_i\)总是正的,
从而\(\increment t_i > 0\),所以定积分的下限\(\alpha\)一定小于上限\(\beta\).

如果曲线\(L\)由方程\[
y = \psi(x)
\quad(x_0 \leq x \leq X)
\]给出,那么可以把这种情形看做特殊的参数方程\[
x = t,
y = \psi(t)
\quad(x_0 \leq t \leq X)
\]的情形,从而由\cref{equation:线积分与面积分.第一类曲线积分的计算式1} 得出
\begin{equation}
\int_L f(x,y) \dd{s}
= \int_{x_0}^X f[x,\psi(x)] \sqrt{1+[\psi'(x)]^2} \dd{x}
\quad(x_0 < X).
\end{equation}

类似地,如果曲线\(L\)由方程\[
x = \phi(y)
\quad(y_0 \leq y \leq Y)
\]给出,则有
\begin{equation}
\int_L f(x,y) \dd{s}
= \int_{y_0}^Y f[\phi(y),y] \sqrt{1+[\phi'(y)]^2} \dd{y}
\quad(y_0 < Y).
\end{equation}

\cref{equation:线积分与面积分.第一类曲线积分的计算式1} 可推广到第一类曲线积分的积分弧段是空间曲线弧\(\Gamma\)的情形.
\begin{theorem}
设三元函数\(f(x,y,z)\)在空间曲线弧\(\Gamma\)上有定义且连续,\(\Gamma\)的参数方程为\[
\left\{ \begin{array}{l}
x = \phi(t), \\
y = \psi(t), \\
z = \omega(t) \\
\end{array} \right.
\qquad
(\alpha \leq t \leq \beta),
\]其中\(\phi(t)\)、\(\psi(t)\)、\(\omega(t)\)在\([\alpha,\beta]\)上具有一阶连续导数,
且\([\phi'(t)]^2+[\psi'(t)]^2+[\omega'(t)]^2 \neq 0\),
则曲线积分\(\int_\Gamma f(x,y,z) \dd{s}\)存在,
且\begin{equation}\label{equation:线积分与面积分.第一类曲线积分的计算式2}
\begin{aligned}
&\hspace{-20pt}
\int_\Gamma f(x,y,z) \dd{s} \\
&= \int_\alpha^\beta f[\phi(t),\psi(t),\omega(t)] \sqrt{[\phi'(t)]^2+[\psi'(t)]^2+[\omega'(t)]^2} \dd{t}
\quad(\alpha<\beta).
\end{aligned}
\end{equation}
\end{theorem}

\begin{example}
计算半径为\(R\)、中心角为\(2\alpha\)、线密度为\(\mu=1\)的圆弧\(L\)对于它的对称轴的转动惯量和质心坐标.
\begin{solution}
以圆弧\(L\)的圆心为极点,经过圆弧的中点作极轴,建立极坐标系,那么\[
	L: \left\{ \begin{array}{l}
		x = R\cos\theta, \\
		y = R\sin\theta
	\end{array} \right.
	\quad(-\alpha\leq\theta\leq\alpha).
\]

圆弧\(L\)的转动惯量为\[
	I = \int_L y^2 \dd{s}
	= \int_{-\alpha}^\alpha (R\sin\theta)^2 R\dd{\theta}
	= R^3 (\alpha - \sin\alpha \cos\alpha).
\]

又由\[
	M = \int_L \mu \dd{s} = L = 2\alpha R,
\]\[
	\overline{x}
	= \frac{1}{M} \int_L x \mu \dd{s}
	= \frac{1}{M} \int_{-\alpha}^\alpha R\cos\theta R\dd{\theta}
	= \frac{\sin\alpha}{\alpha}R,
\]\[
	\overline{y}
	= \frac{1}{M} \int_L y \mu \dd{s}
	= \frac{1}{M} \int_{-\alpha}^\alpha R\sin\theta R\dd{\theta}
	= 0,
\]
圆弧\(L\)在直角坐标系下的质心坐标为\[
	\left(
		\frac{\sin\alpha}{\alpha}R,0
	\right).
\]
\end{solution}
\end{example}

\begin{example}
设围线\(L\)是圆周\(x^2+y^2=a^2\),
直线\(y=x\)和\(x\)轴在第一象限内所围成的扇形的整个边界.
求\[
	\oint_L e^{\sqrt{x^2+y^2}} \dd{s}.
\]
\begin{solution}
将\(L\)分为\[
	L_1: \left\{ \begin{array}{l}
		x = t, \\
		y = 0
	\end{array} \right.
	\quad(0 \leq t \leq a),
\]\[
	L_2: \left\{ \begin{array}{l}
		x = a \cos t, \\
		y = a \sin t
	\end{array} \right.
	\quad(0 \leq t \leq \frac{\pi}{4}),
\]\[
	L_3: \left\{ \begin{array}{l}
		x = t, \\
		y = t
	\end{array} \right.
	\quad(0 \leq t \leq \frac{a}{\sqrt{2}})
\]三段,
分段积分,得\[
	\int_{L_1} e^{\sqrt{x^2+y^2}} \dd{s}
	= \int_0^a e^t \dd{t}
	= e^a-1,
\]\[
	\int_{L_2} e^{\sqrt{x^2+y^2}} \dd{s}
	= \int_0^{\pi/4} e^a a \dd{t}
	= a e^a \frac{\pi}{4},
\]\[
	\int_{L_3} e^{\sqrt{x^2+y^2}} \dd{s}
	= \int_0^{a/\sqrt{2}} e^{\sqrt{2}t} \sqrt{2} \dd{t}
	= e^a-1,
\]
所以\begin{align*}
	\oint_L e^{\sqrt{x^2+y^2}} \dd{s}
	&= \int_{L_1} e^{\sqrt{x^2+y^2}} \dd{s}
	+ \int_{L_2} e^{\sqrt{x^2+y^2}} \dd{s}
	+ \int_{L_3} e^{\sqrt{x^2+y^2}} \dd{s} \\
	&= e^a \left(\frac{\pi}{4}a + 2\right) - 2.
\end{align*}
\end{solution}
\end{example}

\begin{example}
设\[
L: \left\{ \begin{array}{l}
x = a(t - \sin t), \\
y = a(1 - \cos t)
\end{array} \right.
\quad(0 \leq t \leq 2\pi).
\]求\(\int_L y^2 \dd{s}\).
\begin{solution}
因为\(x'_t = a(1 - \cos t)\),\(y'_t = a \sin t\),所以\begin{align*}
\int_L y^2 \dd{s}
&= \int_0^{2\pi} a^2(1 - \cos t)^2 \sqrt{a^2(1 - \cos t)^2 + a^2 \sin^2 t} \dd{t} \\
&= \sqrt{2} a^3 \int_0^{2\pi} (1 - \cos t)^2 \sqrt{1 - \cos t} \dd{t} \\
&= 2\sqrt{2} a^3 \int_0^{2\pi}
 \left(2 \sin^2 \frac{t}{2}\right)^2 \sqrt{2} \sin\frac{t}{2} \dd(\frac{t}{2}) \\
&= -16 a^3 \int_0^{2\pi} \left(1 - \cos^2 \frac{t}{2}\right)^2 \dd(\cos\frac{t}{2}) \\
&\xlongequal{u=\cos(t/2)}
-16a^3 \int_1^{-1} (1-u^2)^2 \dd{u}
= \frac{256a^3}{15}.
\end{align*}
\end{solution}
\end{example}
