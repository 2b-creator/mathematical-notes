\section{对坐标的曲线积分}
\subsection{对坐标的曲线积分的概念}
\begin{definition}
设\(L\)为\(xOy\)面内从点\(A\)到点\(B\)的一条有向光滑曲线弧,函数\(P(x,y)\)和\(Q(x,y)\)在\(L\)上有界.
在\(L\)上沿\(L\)的方向任意插入一点列\[
M_1\opair{x_1,y_1},
M_2\opair{x_2,y_2},
\dotsc,
M_{n-1}\opair{x_{n-1},y_{n-1}},
\]把\(L\)分成\(n\)个有向小弧段\[
\arc{M_{i-1} M_i} \quad (i=1,2,\dotsc,n; M_0 = A, M_n = B).
\]
设\(\increment x_i = x_i - x_{i-1}\),\(\increment y_i = y_i - y_{i-1}\),
点\(\opair{\xi_i,\eta_i}\)为\(\arc{M_{i-1} M_i}\)上任意取定的点.
如果当各小弧段长度的最大值\(\lambda\to0\)时,\[
\sum\limits_{i=1}^n P(\xi_i,\eta_i) \increment x_i
\]的极限总存在,
则称此极限为“函数\(P(x,y)\)在有向曲线弧\(L\)上对坐标\(x\)的\DefineConcept{曲线积分}”,
记作\[\int_L P(x,y) \dd{x}.\]
类似地,如果\[
\lim\limits_{\lambda\to0} \sum\limits_{i=1}^n Q(\xi_i,\eta_i) \increment y_i
\]总存在,
则称此极限为“函数\(Q(x,y)\)在有向曲线弧\(L\)上对坐标\(y\)的\DefineConcept{曲线积分}”,
记作\[\int_L Q(x,y) \dd{y}.\]
这里,\(P(x,y)\)、\(Q(x,y)\)叫做\DefineConcept{被积函数},\(L\)叫做\DefineConcept{积分弧段}.
\end{definition}

我们可以将上述定义简单地总结为以下两个定义式:
\[
\int_L P(x,y) \dd{x}
\defeq \lim\limits_{\lambda\to0}
	\sum\limits_{i=1}^n P(\xi_i,\eta_i) \increment x_i.
\]\[
\int_L Q(x,y) \dd{y}
\defeq \lim\limits_{\lambda\to0}
	\sum\limits_{i=1}^n Q(\xi_i,\eta_i) \increment y_i.
\]

以上两个积分也称为\DefineConcept{第二类曲线积分}.

上述定义可以类似地推广到积分弧段为空间有向曲线弧\(\Gamma\)的情形:
\begingroup
\def\intgamma#1#2{\int_\Gamma #1(x,y,z) \dd{#2} = \lim\limits_{\lambda\to0} \sum\limits_{i=1}^n #1(\xi_i,\eta_i,\zeta_i) \increment #2_i}
\begin{gather*}
\intgamma{P}{x}, \\
\intgamma{Q}{y}, \\
\intgamma{R}{z}.
\end{gather*}
\endgroup

如果平面曲线\(L\)或空间曲线\(\Gamma\)是分段光滑的,我们规定函数在函数在有向曲线弧\(L\)或\(\Gamma\)上对坐标的曲线积分等于在光滑的各段上对坐标的曲线积分之和.

指向与有向曲线弧的方向一致的切向量称为\DefineConcept{有向曲线弧的切向量}.

一般地,可以将\[
\int_L P(x,y) \dd{x} + \int_L Q(x,y) \dd{y}
\]合并写作\[
\int_L P(x,y) \dd{x} + Q(x,y) \dd{y}.
\]也可将其写作向量形式\[
\int_L \vb{F}(x,y) \cdot \dd{\vb{r}},
\]其中\(\vb{F}(x,y) = P(x,y) \vb{i} + Q(x,y) \vb{j}\)为向量值函数,
\(\dd{\vb{r}} = \vb{i} \dd{x} + \vb{j} \dd{y}\).

类似地,把\[
\int_L P(x,y,z) \dd{x} + \int_L Q(x,y,z) \dd{y} + \int_L R(x,y,z) \dd{z}
\]合并写作\[
\int_L P(x,y,z) \dd{x} + Q(x,y,z) \dd{y} + R(x,y,z) \dd{z},
\]也可写作向量形式\[
\int_L \vb{F}(x,y,z) \cdot \dd{\vb{r}},
\]其中\(\vb{F}(x,y,z) = P(x,y,z)\vb{i} + Q(x,y,z)\vb{j} + R(x,y,z)\vb{k}\)为向量值函数,
\(\dd{\vb{r}} = \vb{i} \dd{x} + \vb{j} \dd{y} + \vb{k} \dd{z}\).

如果\(L\)(或\(\Gamma\))是分段光滑的,我们规定函数在有向曲线弧\(L\)(或\(\Gamma\))上对坐标的曲线积分等于在光滑的各段上对坐标的曲线积分之和.

\subsection{对坐标的曲线积分的性质}
根据上述曲线积分的定义,可以导出对坐标的曲线积分的一些性质.
为了表达简便起见,我们用向量形式表达,并假定其中的向量值函数在曲线\(L\)上连续%
\footnote{向量值函数\(\vb{F}(x,y)\)在曲线\(L\)上连续是指:%
对\(L\)上任意一点\(M_0\),当\(L\)上的动点\(M(x,y)\)沿\(L\)趋于\(M_0\)时,有\(\abs{\vb{F}(x,y)-\vb{F}(x_0,y_0)}\to0\).%
若\(\vb{F}(x,y)=P(x,y)\vb{i}+Q(x,y)\vb{j}\),则\(\vb{F}(x,y)\)在\(L\)上连续等价于\(P(x,y),Q(x,y)\)均在\(L\)上连续.}.

\begin{property}\label{theorem:线积分与面积分.第二类曲线积分性质1}
设\(\alpha\)、\(\beta\)为常数,则\[
\int_L [\alpha \vb{F}_1(x,y) + \beta \vb{F}_2(x,y)] \cdot \dd{\vb{r}}
= \alpha \int_L \vb{F}_1(x,y) \cdot \dd{\vb{r}}
+ \beta \int_L \vb{F}_2(x,y) \cdot \dd{\vb{r}}.
\]
\end{property}

\begin{property}\label{theorem:线积分与面积分.第二类曲线积分性质2}
若有向曲线弧\(L\)可分为两段光滑有向曲线弧\(L_1\)和\(L_2\),则\[
\int_L \vb{F}(x,y) \cdot \dd{\vb{r}}
= \int_{L_1} \vb{F}(x,y) \cdot \dd{\vb{r}}
+ \int_{L_2} \vb{F}(x,y) \cdot \dd{\vb{r}}.
\]
\end{property}

\begin{property}\label{theorem:线积分与面积分.第二类曲线积分性质3}
设\(L\)是有向光滑曲线弧,\(L^-\)是\(L\)的反向曲线弧,则\[
\int_{L^-} \vb{F}(x,y) \cdot \dd{\vb{r}}
= - \int_L \vb{F}(x,y) \cdot \dd{\vb{r}}.
\]
\end{property}
\cref{theorem:线积分与面积分.第二类曲线积分性质3} 表明,当积分弧段的方向改变时,对坐标的曲线积分要改变符号.
因此关于对坐标的曲线积分,我们必须注意积分弧段的方向.

这一性质是对坐标的曲线积分所特有的,对弧长的曲线积分不具有这一性质.
同时,对坐标的曲线积分也不具有对弧长的曲线积分所具有的\cref{theorem:线积分与面积分.第一类曲线积分性质3}.

\subsection{对坐标的曲线积分的计算法}
\begin{theorem}
设\(P(x,y)\)、\(Q(x,y)\)在有向曲线弧\(L\)上有定义且连续,
\(L\)的参数方程为\[
\left\{ \begin{array}{l}
x = \phi(t), \\
y = \psi(t), \\
\end{array} \right.
\]当参数\(t\)单调地由\(\alpha\)变到\(\beta\)时,
点\(M(x,y)\)从\(L\)的起点\(A\)沿\(L\)运动到终点\(B\),
\(\phi(t)\)、\(\psi(t)\)在以\(\alpha\)及\(\beta\)为端点的闭区间上具有一阶连续导数,
且\[
[\phi'(t)]^2+[\psi'(t)]^2 \neq 0,
\]则曲线积分\(\int_L{P(x,y)\dd{x} + Q(x,y)\dd{y}}\)存在,
且\begin{equation}\label{equation:线积分与面积分.第二类曲线积分的计算式1}
\int_L P(x,y) \dd{x} + Q(x,y) \dd{y}
= \int_\alpha^\beta \left\{ \def\arraystretch{.7}\begin{array}{l}
P[\phi(t),\psi(t)] \phi'(t) \\
\hspace{5pt}+ Q[\phi(t),\psi(t)] \psi'(t)
\end{array} \right\} \dd{t}.
\end{equation}
\end{theorem}
\cref{equation:线积分与面积分.第二类曲线积分的计算式1} 表明,计算对坐标的曲线积分\[
\int_L P(x,y) \dd{x} + Q(x,y) \dd{y}
\]时,只要把\(x\)、\(y\)、\(\dd{x}\)、\(\dd{y}\)依次换为\(\phi(t)\)、\(\psi(t)\)、\(\phi'(t) \dd{t}\)、\(\psi'(t) \dd{t}\),然后从\(L\)的起点所对应的参数值\(\alpha\)到\(L\)的终点所对应的参数值\(\beta\)作定积分就行了.
这里必须注意,下限\(\alpha\)对应于\(L\)的起点,上限\(\beta\)对应于\(L\)的终点,\(\alpha\)不一定小于\(\beta\).

如果\(L\)由方程\(y = \psi(x)\)或\(x = \phi(y)\)给出,
可以看作参数方程的特殊情形.
例如,当\(L\)由\(y = \psi(x)\)给出时,\cref{equation:线积分与面积分.第二类曲线积分的计算式1} 成为\[
\int_L P(x,y) \dd{x} + Q(x,y) \dd{y}
= \int_\alpha^\beta \left\{
	P[x,\psi(x)] + Q[x,\psi(x)] \psi'(x)
\right\} \dd{x}.
\]

\cref{equation:线积分与面积分.第二类曲线积分的计算式1} 可以推广到空间曲线\(\Gamma\)由参数方程\[
x=\phi(t), \qquad
y=\psi(t), \qquad
z=\omega(t)
\]给出的情形,这一便得到
\begin{equation}\label{equation:线积分与面积分.第二类曲线积分的计算式2}
\int_L \left[ \def\arraystretch{.7}\begin{array}{l}
	P(x,y,z)\dd{x} \\
	\hspace{5pt}+ Q(x,y,z)\dd{y} \\
	\hspace{10pt}+ R(x,y,z)\dd{z}
\end{array} \right]
= \int_\alpha^\beta \left\{\def\arraystretch{.7}\begin{array}{l}
P[\phi(t),\psi(t),\omega(t)] \phi'(t) \\
\hspace{5pt}+ Q[\phi(t),\psi(t),\omega(t)] \psi'(t) \\
\hspace{10pt}+ R[\phi(t),\psi(t),\omega(t)] \omega'(t)
\end{array} \right\} \dd{t}.
\end{equation}
这里下限\(\alpha\)对应\(\Gamma\)的起点,上限\(\beta\)对应\(\Gamma\)的终点.

\begin{example}
设\(L\)为抛物线\(y^2 = x\)上从点\(A\opair{1,-1}\)到点\(B\opair{1,1}\)的一段弧.
计算\(\int_L xy \dd{x}\).
\begin{solution}[解法一]
将所给积分化为对\(x\)的定积分来计算.
由于\(y = \pm\sqrt{x}\)不是单值函数,所以要把\(L\)分成\(AO\)和\(OB\)两部分.
在\(AO\)上,\(y=-\sqrt{x}\),\(x\)从1变到0;
在\(OB\)上,\(y=\sqrt{x}\),\(x\)从0变到1.
因此\begin{align*}
\int_L xy \dd{x} &= \int_{AO} xy \dd{x} + \int_{OB} xy \dd{x} \\
&= \int_1^0 x(-\sqrt{x}) \dd{x} + \int_0^1 x\sqrt{x} \dd{x} \\
&= 2\int_0^1 x^{3/2} \dd{x} = 4/5.
\end{align*}
\end{solution}
\begin{solution}[解法二]
将所给积分化为对\(y\)的定积分来计算,现在\(x=y^2\),\(y\)从\(-1\)变到\(1\).因此\[
\int_L xy \dd{x}
= \int_{-1}^1 y^2 \cdot y \cdot 2y \dd{y}
= 2\int_{-1}^1 y^4 \dd{y}
= \frac{2}{5} [y^5]_{-1}^1
= \frac{4}{5}.
\]
\end{solution}
\end{example}

\begin{example}
设\(L_1\)是以原点为圆心、以\(a\)为半径、按逆时针方向从\(A\opair{a,0}\)到\(B\opair{-a,0}\)的上半圆周,\(L_2\)是从点\(A\)到点\(B\)的线段,计算\(\int_{L_1} y^2 \dd{x}\)和\(\int_{L_2} y^2 \dd{x}\).
\begin{solution}
\(L_1\)可由参数方程\[
x = a \cos\theta, \qquad y = a \sin\theta
\]表出,当参数\(\theta\)从\(0\)变到\(\pi\),有\[
\int_{L_1} y^2 \dd{x}
= \int_0^{\pi} a^2 \sin^2 \theta \cdot (-a \sin\theta) \dd{\theta}
= -\frac{4}{3} a^3.
\]

而\(L_2\)的方程为\(y=0\),\(x\)从\(a\)变到\(-a\),有\[
\int_{L_2} y^2 \dd{x} = \int_a^{-a} 0 \dd{x} = 0.
\]
\end{solution}
\end{example}
从上例可以看出,在第二类曲线积分中,虽然被积函数相同,起点和终点也相同,但沿着不同路径得出的积分值并不相等.

\begin{example}
设\(L_1\)是抛物线\(y=x^2\)上从\(O(0,0)\)到\(B\opair{1,1}\)的一段弧;
\(L_2\)是抛物线\(x=y^2\)上从\(O(0,0)\)到\(B\opair{1,1}\)的一段弧;
\(L_3\)是有向折线\(OAB\),其中\(O,A,B\)依次是点\((0,0),(1,0),\opair{1,1}\).
计算\(\int_L 2xy\dd{x}+x^2\dd{y}\).
\begin{solution}
将\(\int_{L_1} 2xy\dd{x}+x^2\dd{y}\)化为对\(x\)的定积分,\(y=x^2, \dd{y}=2x\dd{x}\),\(x\)从0变到1,所以\[
\int_{L_1} 2xy\dd{x}+x^2\dd{y}
= \int_0^1 (2x \cdot x^2 + x^2) \dd{x}
= 4 \int_0^1 x^3 \dd{x} = 1.
\]

将\(\int_{L_2} 2xy\dd{x}+x^2\dd{y}\)化为对\(y\)的定积分,\(x=y^2, \dd{x}=2y\dd{y}\),\(y\)从0变到1,所以\[
\int_{L_2} 2xy\dd{x}+x^2\dd{y}
= \int_0^1 (2y^2 \cdot y \cdot 2y + y^4) \dd{y}
= 5 \int_0^1 y^4 \dd{y} = 1.
\]

显然\(\int_{L_3} 2xy\dd{x}+x^2\dd{y} = \int_{OA} 2xy\dd{x}+x^2\dd{y} + \int_{AB} 2xy\dd{x}+x^2\dd{y}\).
在\(OA\)上,\(y=0\),\(x\)从0变到1,所以\[
\int_{OA} 2xy\dd{x}+x^2\dd{y}
= \int_0^1 (2x\cdot0+x^2\cdot0) \dd{x} = 0.
\]
在\(AB\)上,\(x=1, \dd{x}=0\),\(y\)从0变到1,所以\[
\int_{AB} 2xy\dd{x}+x^2\dd{y}
= \int_0^1 (2y\cdot0+1) \dd{y} = 1.
\]从而\[
\int_{L_3} 2xy\dd{x}+x^2\dd{y} = 0 + 1 = 1.
\]
\end{solution}
\end{example}
从上例可以看出,虽然沿不同路径,曲线积分的值可以相等.

\begin{example}
设一个质点在点\(M(x,y)\)处受到力\(\vb{F}\)的作用,\(\vb{F}\)的大小与点\(M\)到原点\(O\)的距离成正比,\(\vb{F}\)的方向恒指向原点.
此质点由点\(A\opair{a,0}\)沿椭圆\(\frac{x^2}{a^2}+\frac{y^2}{b^2}=1\)按逆时针方向移动到点\(B\opair{0,b}\),求力\(\vb{F}\)所作的功\(W\).
\begin{solution}
记\(\vec{OM} = x\vb{i}+y\vb{j}\),那么\(\abs{\vec{OM}} = \sqrt{x^2+y^2}\).由假设有\[
\vb{F} = -k(x\vb{i}+y\vb{j}) \quad (k>0),
\]于是\[
W = \int_{AB} \vb{F} \cdot \dd\vb{r}
= \int_{AB} -kx\dd{x}-ky\dd{y}
= -k \int_{AB} x\dd{x}+y\dd{y}.
\]利用椭圆的参数方程\[
\begin{cases}
x = a \cos t, \\
y = b \sin t,
\end{cases}
\]起点\(A\)、终点\(B\)分别对应参数\(t = 0,\frac{\pi}{2}\).
于是\begin{align*}
W &= -k \int_0^{\pi/2} (-a^2 \cos t \sin t + b^2 \sin t \cos t) \dd{t} \\
&= \frac{k}{2}(a^2-b^2) \int_0^{\pi/2} \sin 2t \dd{t}
= \frac{k}{2}(a^2-b^2).
\end{align*}
\end{solution}
\end{example}
