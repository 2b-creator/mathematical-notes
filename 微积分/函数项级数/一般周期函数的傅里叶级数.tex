\section{一般周期函数的傅里叶级数}
\subsection{周期为2\texorpdfstring{\(l\)}{l}的周期函数的傅里叶级数}
\begin{theorem}
\def\f{\frac{n \pi x}{l}}
设周期为\(2l\)的周期函数\(f(x)\)满足\hyperref[theorem:无穷级数.傅里叶级数收敛的狄利克雷充分条件]{收敛定理}的条件,
则它的傅里叶级数展开式为\[
f(x) = \frac{a_0}{2} + \sum\limits_{n=1}^\infty \left( a_n \cos\f + b_n \sin\f \right)
\quad(x \in C),
\]其中\begin{gather*}
a_n = \frac{1}{l} \int_{-l}^{l} f(x) \cos\f \dd{x}
\quad(n=0,1,2,\dotsc), \\
b_n = \frac{1}{l} \int_{-1}^{l} f(x) \sin\f \dd{x}
\quad(n=1,2,3,\dotsc), \\
C = \Set*{ x \given f(x) = \frac{1}{2} [ f(x^-) + f(x^+) ] }.
\end{gather*}

特别地,当\(f(x)\)是奇函数时,\[
f(x) = \sum\limits_{n=1}^\infty b_n \sin\frac{n \pi x}{l}
\quad(x \in C),
\]其中\[
b_n = \frac{2}{l} \int_0^l f(x) \sin\f \dd{x}
\quad(n=1,2,3,\dotsc).
\]

当\(f(x)\)是偶函数时,\[
f(x) = \frac{a_0}{2} + \sum\limits_{n=1}^\infty a_n \cos\f
\quad(x \in C),
\]其中\[
a_n = \frac{2}{l} \int_0^l f(x) \cos\f \dd{x}
\quad(n=0,1,2,\dotsc).
\]
\begin{proof}
作变量代换\(z = \frac{\pi x}{l}\),于是区间\(-l \leq x \leq l\)就变换成\(-\pi \leq z \leq \pi\).
设函数\(f(x) = f\left(\frac{lz}{\pi}\right) = F(z)\),从而\(F(z)\)是周期为\(2\pi\)的周期函数,并且它满足\hyperref[theorem:无穷级数.傅里叶级数收敛的狄利克雷充分条件]{收敛定理}的条件,将\(F(z)\)展开成傅里叶级数\[
F(z) = \frac{a_0}{2} + \sum\limits_{n=1}^\infty (a_n \cos nz + b_n \sin nz),
\]其中\(a_n = \frac{1}{\pi} \int_{-\pi}^{\pi} F(z) \cos nz \dd{z},\ b_n = \frac{1}{\pi} \int_{-\pi}^{\pi} F(z) \sin nz \dd{z}\).

在以上式子中令\(z=\frac{\pi x}{l}\),并注意到\(F(z) = f(x)\),于是有\[
f(x) = \frac{a_0}{2} + \sum\limits_{n=1}^\infty \left( a_n \cos\frac{n\pi x}{l} + b_n \sin\frac{n\pi x}{l} \right),
\]而且\[
a_n = \frac{1}{l} \int_{-l}^l f(x) \cos\frac{n\pi x}{l} \dd{x},
\qquad
b_n = \frac{1}{l} \int_{-l}^l f(x) \sin\frac{n\pi x}{l} \dd{x}.
\]

类似地,可以证明定理的其余部分.
\end{proof}
\end{theorem}

要将定义在非对称区间\([a,b]\)上的函数\(f(x)\)展开成傅里叶级数,
既可以先作变换\(x = z + \frac{b+a}{2}\),
使得\(z \in \left[-\frac{b-a}{2},\frac{b-a}{2}\right]\),再进行周期延拓;
也可以先做变换\(x = z + a\),使得\(z \in [0,b-a]\),再进行奇延拓(或偶延拓).

\begin{example}
设周期函数\(f(x)\)的周期为\(2\),且\(f(x) = 1-x\ (0 \leq x \leq 1)\).
若\(f(x)\)可展为傅里叶级数\(\frac{a_0}{2} + \sum\limits_{n=1}^\infty a_n \cos n\pi x\),求\(\sum\limits_{n=1}^\infty a_{2n}\).
\begin{solution}
观察\(f(x)\)的傅里叶级数的形式可知,它是余弦级数,函数\(f(x)\)是偶函数,那么\begin{align*}
\frac{a_n}{2} &= \int_0^1 (1-x) \cos n\pi x \dd{x} \\
&= \int_0^1 \cos n\pi x \dd{x}
- \frac{1}{n\pi} \int_0^1 x \dd(\sin n\pi x) \\
&= \frac{1}{\pi} \int_0^{\pi} \cos nt \dd{t}
- \frac{1}{n\pi} \left[
(x \sin n\pi x)_0^1
- \frac{1}{\pi} \int_0^{\pi} \sin nt \dd{t}
\right] \\
&= \frac{1}{n^2\pi^2} \int_0^{\pi} \sin nt \dd(nt) \\
&= \frac{1}{n^2\pi^2} \left( 1 - \cos n\pi \right),
\end{align*}
于是\(a_{2n} = \frac{2}{n^2\pi^2} \left( 1 - \cos 2n\pi \right) = 0\),从而\(\sum\limits_{n=1}^\infty a_{2n} = 0\).
\end{solution}
\end{example}

\subsection{傅里叶级数的复数形式}
\begingroup
\def\f{\frac{n \pi x}{l}}
\def\eif#1{e^{#1\iu\f}}

傅里叶级数还可以用复数形式表示.
在电子技术中,经常应用这种形式.

设周期为\(2l\)的周期函数\(f(x)\)的傅里叶级数为\[
	\frac{a_0}{2} + \sum\limits_{n=1}^\infty \left( a_n \cos\f + b_n \sin\f \right),
\]
其中系数\(a_n,b_n\)为\begin{gather*}
	a_n = \frac{1}{l} \int_{-l}^{l} f(x) \cos\f \dd{x}
	\quad(n=0,1,2,\dotsc), \\
	b_n = \frac{1}{l} \int_{-1}^{l} f(x) \sin\f \dd{x}
	\quad(n=1,2,3,\dotsc).
\end{gather*}
利用欧拉公式\[
	\cos t = \frac{e^{\iu t}+e^{-\iu t}}{2},
	\qquad
	\sin t = \frac{e^{\iu t}-e^{-\iu t}}{2\iu},
\]
可将上述傅里叶级数化为\[
	\begin{split}
		&\frac{a_0}{2} + \sum\limits_{n=1}^\infty \left[
			\frac{a_n}{2} \left( \eif{} + \eif- \right)
			-\frac{\iu b_n}{2} \left( \eif{} - \eif- \right)
		\right] \\
		&= \frac{a_0}{2} + \sum\limits_{n=1}^\infty \left[
			\frac{a_n - \iu b_n}{2} \eif{}
			+ \frac{a_n + \iu b_n}{2} \eif-
		\right].
	\end{split}
\]
记\[
	\frac{a_0}{2} = c_0,
	\qquad
	\frac{a_n-\iu b_n}{2} = c_n,
	\qquad
	\frac{a_n+\iu b_n}{2} = c_{-n},
\]
其中\(n=1,2,\dotsc\),
则可进一步将傅里叶级数化为\[
	\begin{split}
		&c_0 + \sum\limits_{n=1}^\infty \left(
			c_n \eif{} + c_{-n} \eif-
		\right) \\
		&= (c_n \eif{})_{n=0}
		+ \sum\limits_{n=1}^\infty \left(
			c_n \eif{} + c_{-n} \eif-
	\right).
\end{split}
\]
即得\DefineConcept{傅里叶级数的复数形式}\footnote{这是一个罗朗级数.}为
\begin{equation}\label{equation:无穷级数.傅里叶级数的复数形式}
	\sum\limits_{n=-\infty}^{+\infty} c_n \eif{}.
\end{equation}
可以求出系数满足
\begin{equation}\label{equation:无穷级数.傅里叶系数的复数形式}
	c_n = \frac{1}{2l} \int_{-l}^l f(x) \eif- \dd{x}
	\quad(n=0,\pm1,\pm2,\dotsc).
\end{equation}
这就是\DefineConcept{傅里叶系数的复数形式}.

傅里叶级数的两种形式,本质上是一样的,但复数形式比较简洁,且只用一个算式计算系数.
\endgroup
