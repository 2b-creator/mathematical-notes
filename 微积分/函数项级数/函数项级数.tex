\section{函数项级数}
\subsection{函数项级数的概念}
\begin{definition}\label{definition:无穷级数.实函数项级数的概念}
给定一个定义在区间\(I \subseteq \mathbb{R}\)上的函数列\[
	u_1(x),u_2(x),\dotsc,u_n(x),\dotsc,
\]
则由该函数列构成的表达式\[
	u_1(x)+u_2(x)+\dotsb+u_n(x)+\dotsb
\]
称为“定义在区间\(I\)上的\DefineConcept{函数项无穷级数}(infinite series with function terms)”,
简称\DefineConcept{函数项级数},
或者进一步简称为\DefineConcept{级数};
记作\(\sum\limits_{n=1}^\infty u_n(x)\).

对于每一个确定的值\(x_0 \in I\),
级数\(\sum\limits_{n=1}^\infty u_n(x)\)
成为级数\(\sum\limits_{n=1}^\infty u_n(x_0)\).
如果级数\(\sum\limits_{n=1}^\infty u_n(x_0)\)收敛,
则称“点\(x_0\)是级数\(\sum\limits_{n=1}^\infty u_n(x)\)的\DefineConcept{收敛点}
(point of convergence)”;
反之,如果级数\(\sum\limits_{n=1}^\infty u_n(x_0)\)发散,
就称“点\(x_0\)是级数\(\sum\limits_{n=1}^\infty u_n(x)\)的\DefineConcept{发散点}
(point of divergence)”.
级数\(\sum\limits_{n=1}^\infty u_n(x)\)的收敛点的全体称为
“级数\(\sum\limits_{n=1}^\infty u_n(x)\)的\DefineConcept{收敛域}(domain of convergence)”,
级数\(\sum\limits_{n=1}^\infty u_n(x)\)的发散点的全体称为
“级数\(\sum\limits_{n=1}^\infty u_n(x)\)的\DefineConcept{发散域}(domain of divergence)”.

对应于收敛域内的任意一个数\(x\),级数成为一个收敛的级数,因而有一个确定的和\(s\).
这样,在收敛域上,级数的和是\(x\)的函数\(s(x)\),
通常称\(s(x)\)为“级数\(\sum\limits_{n=1}^\infty u_n(x)\)的\DefineConcept{和函数}”,
其定义域就是级数的收敛域,
并写成\[
	s(x) = u_1(x)+u_2(x)+\dotsb+u_n(x)+\dotsb.
\]

将级数\(\sum\limits_{n=1}^\infty u_n(x)\)的前\(n\)项的部分和记作\(s_n(x)\),
则在收敛域上有\[
	\lim\limits_{n\to\infty} s_n(x) = s(x).
\]
记\(r_n(x) = s(x)-s_n(x)\),
\(r_n(x)\)叫做“级数的\DefineConcept{余项}”%
\footnote{只有当\(x\)在收敛域上时\(r_n(x)\)才有意义.}.
\end{definition}

\begin{property}
设级数\(\sum\limits_{n=1}^\infty u_n(x)\)的收敛域为\(C\).
当\(x \in C\)时,级数的余项满足\[
	\lim\limits_{n\to\infty} r_n(x) = 0.
\]
\end{property}
