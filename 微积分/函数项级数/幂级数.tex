\section{幂级数}
我们在本节学习函数项级数中简单而常见的一类级数 --- “幂级数”.

\subsection{幂级数的概念}
\begin{definition}\label{definition:无穷级数.幂级数}
各项均是幂函数的函数项级数,
称为\DefineConcept{幂级数}(power series).

称级数\[
	\sum\limits_{n=0}^\infty a_n (x-x_0)^n
\]为“幂级数的\DefineConcept{一般形式}”.

称级数\[
	\sum\limits_{n=0}^\infty a_n x^n
\]为“幂级数的\DefineConcept{标准形式}”.

这里,我们把常数\(\AutoTuple{a}[0]{n},\dotsc\)称为
“幂级数的\DefineConcept{系数}”.
\end{definition}

由于幂级数的一般形式只要作变量代换\(t = x - x_0\)就可化为它的标准形式,
因此,即便我们取标准形式来讨论,也并不影响一般性.

\subsection{幂级数的收敛性}
现在我们来讨论:
对于一个给定的幂级数,
它的收敛域与发散域是怎样的?
即\(x\)取数轴上哪些点时幂级数收敛,
取哪些点时幂级数发散?
这就是幂级数的收敛性问题.

先看一个例子.
考察幂级数\[
	1+x^2+x^3+\dotsb+x^n+\dotsb
\]的收敛性.
我们已经知道,
当\(\abs{x}<1\)时,
该级数收敛于\(\frac{1}{1-x}\);
当\(\abs{x}\geq1\)时,
该级数发散.
因为,
这个幂级数的收敛域是开区间\((-1,1)\),
发散域是\((-\infty,-1]\cup[1,+\infty)\),
并有\[
	\frac{1}{1-x} = 1+x+x^2+\dotsb+x^n+\dotsb
	\quad(-1<x<1).
\]

在这个例子中我们看到,这个幂级数的收敛域是一个区间.
事实上,这个结论对于一般的幂级数也是成立的.
我们有如下的定理.

\begin{theorem}[阿贝尔定理]\label{theorem:无穷级数.阿贝尔定理}
%@see: 《高等数学(第六版 下册)》 P271. 定理1
如果级数\(\sum\limits_{n=0}^\infty a_n x^n\)当\(x=x_0\neq0\)时收敛,
则满足\(\abs{x}<\abs{x_0}\)的一切\(x\)可使该幂级数绝对收敛.
反之,如果级数\(\sum\limits_{n=0}^\infty a_n x^n\)当\(x=x_0\)时发散,
则满足\(\abs{x}>\abs{x_0}\)的一切\(x\)均使该幂级数发散.
\begin{proof}
先设\(x_0\)是幂级数\(\sum\limits_{n=0}^\infty a_n x^n\)的收敛点,
即常数项级数\[
	a_0 + a_1 x_0 + a_2 x_0^2 + \dotsb + a_n x_0^n + \dotsb
\]收敛.
根据级数收敛的必要条件,
这时有\[
	\lim\limits_{n\to\infty} a_n x_0^n = 0;
\]
于是\(\exists M > 0\),使得\[
	\abs{a_n x_0^n} \leq M
	\quad(n=0,1,2,\dotsc).
\]
这样幂级数\(\sum\limits_{n=0}^\infty a_n x^n\)的一般项的绝对值\[
	\abs{a_n x^n} = \abs{a_n x_0^n \cdot \frac{x^n}{x_0^n}}
	= \abs{a_n x_0^n} \cdot \abs{\frac{x}{x_0}}^n
	\leq M \abs{\frac{x}{x_0}}^n.
\]
因为当\(\abs{x}<\abs{x_0}\)时,
等比级数\(\sum\limits_{n=0}^\infty M \abs{\frac{x}{x_0}}^n\)收敛,
所以级数\(\sum\limits_{n=1}^\infty \abs{a_n x^n}\)收敛,
也就是级数\(\sum\limits_{n=0}^\infty a_n x^n\)绝对收敛.

定理的第二部分可用反证法证明.
假设幂级数当\(x=x_0\)时发散而有一点\(x_1\)适合\(\abs{x_1}>\abs{x_0}\)使级数收敛,
则根据本定理的第一部分,当\(x=x_0\)时级数应收敛,这与假设矛盾.
\end{proof}
\end{theorem}

\cref{theorem:无穷级数.阿贝尔定理} 表明,
如果幂级数在\(x=x_0\)处收敛,
则对\(\forall x\in(-\abs{x_0},\abs{x_0})\),
幂级数都收敛;
如果幂级数在\(x=x_0\)处发散,
则对\(\forall x\in(-\infty,-\abs{x_0})\cup(\abs{x_0},+\infty)\),
幂级数都发散.

设已给幂级数在数轴上既有收敛点(不仅是原点)也有发散点.
现在从原点出发沿数轴正方向走,最初只遇到收敛点,然后就只遇到发散点.
这两部分的界点可能是收敛点,也可能是发散点.
从原点出发沿数轴负方向走,情形相同.
利用\cref{theorem:无穷级数.阿贝尔定理} 可以证明:原点两侧的两个界点到原点的距离是相等的.
像这样,我们就得到以下重要推论.
\begin{corollary}\label{theorem:无穷级数.阿贝尔定理推论}
%@see: 《高等数学(第六版 下册)》 P271. 推论
如果幂级数\(\sum\limits_{n=0}^\infty a_n x^n\)不是仅在\(x=0\)一点收敛,
也不是在整个数轴上都收敛,
则必定存在正数\(R\),
使得:\begin{enumerate}
	\item 当\(\abs{x}<R\)时,幂级数绝对收敛;
	\item 当\(\abs{x}>R\)时,幂级数发散;
	\item 当\(\abs{x}=R\)时,幂级数可能收敛也可能发散.
\end{enumerate}
\end{corollary}

我们称\cref{theorem:无穷级数.阿贝尔定理推论} 中提到的正数\(R\)为“幂级数的\DefineConcept{收敛半径}”.
称开区间\((-R,R)\)为“幂级数的\DefineConcept{收敛区间}”.

在已知幂级数的收敛半径或收敛区间的情况下,
我们可以根据幂级数在点\(x = \pm R\)处的收敛性,
就可以决定其收敛域是\((-R,R)\)、\([-R,R)\)、\((-R,R]\)或\([-R,R]\)四个区间之一.

如果幂级数只在\(x=0\)处收敛,这时收敛域为\(\{0\}\),规定收敛半径\(R=0\).
如果幂级数对任意实数都收敛,则规定收敛半径\(R=+\infty\),收敛域为\((-\infty,+\infty)\).

关于幂级数的收敛半径的求法,有下面的定理.
\begin{theorem}\label{theorem:无穷级数.幂级数的收敛半径的求法}
如果\[
	\lim\limits_{n\to\infty} \abs{\frac{a_{n+1}}{a_n}} = \rho,
\]
其中\(a_n\)、\(a_{n+1}\)是
幂级数\(\sum\limits_{n=0}^\infty a_n x^n\)的相邻两项的系数,
则这幂级数的收敛半径为\[
	R = \left\{ \def\arraystretch{1.5} \begin{array}{ll}
		\frac{1}{\rho}, & \rho \in (0,+\infty), \\
		+\infty, & \rho = 0, \\
		0, & \rho = +\infty \\
	\end{array} \right.
\]或\[
	R = \lim\limits_{n\to\infty} \abs{\frac{a_n}{a_{n+1}}}.
\]
\begin{proof}
考察幂级数\(\sum\limits_{n=0}^\infty a_n x^n\)的各项取绝对值所成的级数\[
	\sum\limits_{n=1}^\infty \abs{a_n x^n}
	= \abs{a_0} + \abs{a_1 x} + \abs{a_2 x^2} + \dotsb + \abs{a_n x^n} + \dotsb.
\]
这级数相邻两项之比为\[
	\frac{\abs{a_{n+1} x^{n+1}}}{\abs{a_n x^n}}
	= \abs{\frac{a_{n+1}}{a_n}} \abs{x}.
\]

\begin{enumerate}
	\item 如果极限\(\lim\limits_{n\to\infty} \abs{\frac{a_{n+1}}{a_n}} = \rho\neq0\)存在,
	根据\hyperref[theorem:无穷级数.正项级数的比值审敛法]{比值审敛法},
	则当\(\rho \abs{x} < 1\)即\(\abs{x} < \frac{1}{\rho}\)时,
	级数\(\sum\limits_{n=1}^\infty \abs{a_n x^n}\)收敛,
	从而级数\(\sum\limits_{n=0}^\infty a_n x^n\)绝对收敛;
	再根据\cref{theorem:无穷级数.绝对发散的特殊情况},
	当\(\rho \abs{x} > 1\)即\(\abs{x} > \frac{1}{\rho}\)时,
	级数\(\sum\limits_{n=1}^\infty \abs{a_n x^n}\)发散,
	并且\[
		(\exists N\in\mathbb{N})
		(\forall n\in\mathbb{N})
		[
			n > N
			\implies
			\abs{a_{n+1} x^{n+1}} > \abs{a_n x^n}
		].
	\]

	\item 如果\(\rho=0\),
	则对\(\forall x\neq0\),
	有\(\lim\limits_{n\to\infty} \abs{\frac{a_{n+1} x^{n+1}}{a_n x^n}} = 0\),
	所以级数\(\sum\limits_{n=1}^\infty \abs{a_n x^n}\)收敛,
	从而级数\(\sum\limits_{n=0}^\infty a_n x^n\)绝对收敛.
	于是\(R=+\infty\).
		\item 如果\(\rho=+\infty\),
	则对\(\forall x\neq0\),
	级数\(\sum\limits_{n=0}^\infty a_n x^n\)必发散,
	否则由\cref{theorem:无穷级数.阿贝尔定理} 知道,
	\(\exists x\neq0\)使得\(\sum\limits_{n=1}^\infty \abs{a_n x^n}\)收敛.
	于是\(R=0\).
	\qedhere
\end{enumerate}
\end{proof}
\end{theorem}

\begin{example}
求幂级数\[
x-\frac{x^2}{2}+\frac{x^3}{3}-\dotsb+(-1)^{n-1}\frac{x^n}{n}+\dotsb
\]的收敛半径与收敛域.
\begin{solution}
因为\[
\rho = \lim\limits_{n\to\infty} \abs{\frac{a_{n+1}}{a_n}}
= \lim\limits_{n\to\infty} \frac{n}{n+1} = 1,
\]所以收敛半径\[
R = \frac{1}{\rho} = 1.
\]

对于端点\(x=1\),级数成为交错级数\[
1-\frac{1}{2}+\frac{1}{3}-\dotsb+(-1)^{n-1}\frac{1}{n}+\dotsb,
\]由\cref{example:无穷级数.交错级数1} 可知,此级数收敛;
对于端点\(x=-1\),级数成为\[
-1-\frac{1}{2}-\frac{1}{3}-\dotsb-\frac{1}{n}-\dotsb,
\]此级数发散.
综上所述,收敛域是\((-1,1]\).
\end{solution}
\end{example}
利用上例的结果可以计算出交错级数\[
1-\frac{1}{2}+\frac{1}{3}-\dotsb+(-1)^{n-1}\frac{1}{n}+\dotsb.
\]显然有
\begin{align*}
\sum\limits_{n=1}^\infty (-1)^{n-1} \frac{1}{n}
&= \eval{\sum\limits_{n=1}^\infty (-1)^{n-1} \frac{x^n}{n}}_{x=1} \\
&= \eval{\sum\limits_{n=1}^\infty (-1)^{n-1} \int_0^x \left(\frac{x^n}{n}\right)' \dd{x}}_{x=1} \\
&= \eval{\sum\limits_{n=1}^\infty (-1)^{n-1} \int_0^x x^{n-1} \dd{x}}_{x=1} \\
&= \eval{\int_0^x \sum\limits_{n=1}^\infty (-1)^{n-1} x^{n-1} \dd{x}}_{x=1} \\
&= \eval{\int_0^x \frac{1}{1+x} \dd{x}}_{x=1} \\
&= \eval{\ln(1+x)}_{x=1} = \ln2.
\end{align*}

\begin{example}
求幂级数\[
1+x+\frac{1}{2!}x^2+\dotsb+\frac{1}{n!}x^n+\dotsb
\]的收敛域.
\begin{solution}
因为\[
\rho = \lim\limits_{n\to\infty} \abs{\frac{a_{n+1}}{a_n}}
= \lim\limits_{n\to\infty} \frac{n!}{(n+1)!}
= \lim\limits_{n\to\infty} \frac{1}{n+1}
= 0,
\]所以收敛半径\(R = +\infty\),从而收敛域是\((-\infty,+\infty)\).
\end{solution}
\end{example}

\begin{example}
求幂级数\(\sum\limits_{n=0}^\infty n! x^n\)的收敛半径.
\begin{solution}
因为\[
	\rho
	= \lim\limits_{n\to\infty} \abs{\frac{a_{n+1}}{a_n}}
	= \lim\limits_{n\to\infty} \frac{(n+1)!}{n!}
	= \lim_{n\to\infty} (n+1)
	= +\infty,
\]
所以收敛半径\(R = 0\),
即级数仅在点\(x = 0\)处收敛.
\end{solution}
\end{example}

\begin{example}
求幂级数\(\sum\limits_{n=0}^\infty \frac{(2n)!}{(n!)^2} x^{2n}\)的收敛半径.
\begin{solution}
级数缺少奇次幂的项,\cref{theorem:无穷级数.幂级数的收敛半径的求法} 不能直接应用,
我们根据\cref{theorem:无穷级数.正项级数的比值审敛法的上下极限形式} 来求收敛半径:
\[
\lim\limits_{n\to\infty} \abs{{\frac{[2(n+1)]!}{[(n+1)!]^2} x^{2(n+1)}}\Bigg/{\frac{(2n)!}{(n!)^2} x^{2n}}}
= \lim\limits_{n\to\infty} \abs{\frac{(2n+2)(2n+1)}{(n+1)^2} x^2}
= 4 x^2.
\]

当\(4 x^2 < 1\)即\(\abs{x} < 1/2\)时级数收敛;
当\(4 x^2 > 1\)即\(\abs{x} > 1/2\)时级数发散.
所以收敛半径\(R = 1/2\).
\end{solution}
\end{example}

\begin{example}
求幂级数\(\sum\limits_{n=1}^\infty \frac{(x-1)^n}{2^n \cdot n}\)的收敛域.
\begin{solution}
令\(t = x-1\),上述级数变为\[
\sum\limits_{n=1}^\infty \frac{t^n}{2^n \cdot n}.
\]因为\[
\rho = \lim\limits_{n\to\infty} \abs{\frac{a_{n+1}}{a_n}} = \lim\limits_{n\to\infty} \frac{2^n \cdot n}{2^{n+1} \cdot (n+1)} = \frac{1}{2},
\]所以收敛半径\(R_t = 2\),而原级数的收敛区间为\(-1<x<3\).

当\(x=3\)时,级数成为\(\sum\limits_{n=1}^\infty \frac{1}{n}\),这级数发散;
当\(x=-1\)时,级数成为\(\sum\limits_{n=1}^\infty \frac{(-1)^n}{n}\),这级数收敛.
因此原级数的收敛域为\([-1,3)\).
\end{solution}
\end{example}

注意:当级数缺项时,不能直接运用以上定理求解幂级数的收敛半径,
而应该使用合适的审敛法(如比值审敛法、根值审敛法),或者对幂级数使用换元法.

\subsection{幂级数的运算}
\begin{definition}
设幂级数\(\sum\limits_{n=0}^\infty a_n x^n\)
和\(\sum\limits_{n=0}^\infty b_n x^n\)
分别在区间\((-R,R)\)和\((-R',R')\)内收敛.
定义:
\begin{gather}
	\left(\sum\limits_{n=0}^\infty a_n x^n\right)
	+ \left(\sum\limits_{n=0}^\infty b_n x^n\right)
	\defeq
	\sum\limits_{n=0}^\infty (a_n+b_n) x^n, \\
	\left(\sum\limits_{n=0}^\infty a_n x^n\right)
	- \left(\sum\limits_{n=0}^\infty b_n x^n\right)
	\defeq
	\sum\limits_{n=0}^\infty (a_n-b_n) x^n, \\
	\left(\sum\limits_{n=0}^\infty a_n x^n\right)
	\cdot \left(\sum\limits_{n=0}^\infty b_n x^n\right)
	\defeq
	\sum\limits_{n=0}^\infty \left(
		\sum\limits_{i=0}^n a_i b_{n-i}
	\right) x^n.
\end{gather}
\end{definition}

\begin{definition}
设幂级数\(\sum\limits_{n=0}^\infty a_n x^n\)
和\(\sum\limits_{n=0}^\infty b_n x^n\)
分别在区间\((-R,R)\)和\((-R',R')\)内收敛.

记\[
	\frac{
		\sum\limits_{n=0}^\infty a_n x^n
	}{
		\sum\limits_{n=0}^\infty b_n x^n
	}
	= \sum\limits_{n=0}^\infty c_n x^n,
\]
假设\(b_0 \neq 0\).
为了决定系数\(c_0,c_1,\dotsc,c_n,\dotsc\),
可以将级数\(\sum\limits_{i=0}^\infty b_i x^i\)
与\(\sum\limits_{i=0}^\infty c_i x^i\)相乘,
并令乘积中各项的系数分别等于级数\(\sum\limits_{n=0}^\infty a_n x^n\)中同次幂的系数,
即得\[
	\begin{cases}
		a_0 = b_0 c_0, \\
		a_1 = b_1 c_0 + b_0 c_1, \\
		a_2 = b_2 c_0 + b_1 c_1 + b_0 c_2, \\
		\hdotsfor{1} \\
	\end{cases}
\]
由这些方程就可以顺序地求出\(c_0,c_1,\dotsc,c_n,\dotsc\).

相除后所得的幂级数\(\sum\limits_{i=0}^\infty c_i x^i\)的收敛区间可能比原来的两级数的收敛区间小得多.
\end{definition}

\subsection{幂级数的和函数的性质}
\begin{property}\label{theorem:无穷级数.一致收敛的幂级数的性质}
如果幂级数\(\sum\limits_{n=0}^\infty a_n x^n\)的收敛半径为\(R>0\),
则此级数在\((-R,R)\)内闭一致收敛.
\end{property}
进一步还可证明,
如果幂级数\(\sum\limits_{n=0}^\infty a_n x^n\)在收敛区间的端点收敛,
则一致收敛的区间可扩大到包含端点.

\begin{property}\label{theorem:无穷级数.幂级数的和函数的性质1}
幂级数\(\sum\limits_{n=0}^\infty a_n x^n\)的和函数\(s(x)\)在其收敛域\(I\)上连续.
\end{property}

关于和函数的连续性及逐项可积的结论,
由\cref{theorem:无穷级数.一致收敛级数的基本性质1,%
theorem:无穷级数.一致收敛级数的基本性质3,%
theorem:无穷级数.一致收敛的幂级数的性质}
立即可得.

\begin{property}\label{theorem:无穷级数.幂级数的和函数的性质2}
幂级数\(\sum\limits_{n=0}^\infty a_n x^n\)的和函数\(s(x)\)在其收敛域\(I\)上可积,并有逐项积分公式\[
\int_0^x s(x) \dd{x}
=\int_0^x \left[\sum\limits_{n=0}^\infty a_n x^n\right] \dd{x}
=\sum\limits_{i=0}^\infty \int_0^x a_i x^i \dd{x}
=\sum\limits_{i=0}^\infty \frac{a_i}{i+1} x^{i+1}.
\]

逐项积分后所得到的幂级数和原级数有相同的收敛半径、收敛域.
\end{property}

\begin{property}\label{theorem:无穷级数.幂级数的和函数的性质3}
如果幂级数\(\sum\limits_{n=0}^\infty a_n x^n\)的收敛半径为\(R>0\),
则其和函数\(s(x)\)在\((-R,R)\)内可导,且有逐项求导公式\[
	s'(x)
	= \left( \sum\limits_{n=1}^\infty a_n x^n \right)'
	= \sum\limits_{n=0}^\infty (a_n x^n)'
	= \sum\limits_{n=1}^\infty n a_n x^{n-1}.
\]
逐项求导后所得到的幂级数与原级数具有相同的收敛半径.
\end{property}
逐项求导后所得到的幂级数虽然和原级数有相同的收敛半径,
但可能有不同的收敛域,
这是因为逐项求导后所得到的幂级数在边界点处的收敛性可能发生改变.

反复利用上述结论可得:
幂级数\(\sum\limits_{n=0}^\infty a_n x^n\)的和函数\(s(x)\)
在其收敛区间\((-R,R)\)内具有任意阶导数.

\begin{example}
求幂级数\(\sum\limits_{n=1}^\infty \frac{x^n}{n+1}\)的和函数.
\begin{solution}
先求收敛域.
由\[
	\lim\limits_{n\to\infty} \abs{\frac{a_{n+1}}{a_n}}
	= \lim\limits_{n\to\infty} \frac{n+1}{n+2}
	= 1,
\]
得收敛半径\(R=1\).

在端点\(x = -1\)处,
幂级数成为\(\sum\limits_{n=1}^\infty \frac{(-1)^n}{n+1}\),
是收敛的交错级数;
在端点\(x = 1\)处,
幂级数成为\(\sum\limits_{n=1}^\infty \frac{1}{n+1}\),
是发散的.
因此收敛域是\(I = [-1,1)\).

设和函数为\(s(x)\),即\[
	s(x) = \sum\limits_{n=1}^\infty \frac{x^n}{n+1},
	\quad x\in[-1,1).
\]
于是\[
	x s(x) = \sum\limits_{n=1}^\infty \frac{x^{n+1}}{n+1}.
\]

利用\cref{theorem:无穷级数.幂级数的和函数的性质3},逐项求导,并由\[
	\frac{1}{1-x} = 1+x+x^2+\dotsb+x^n+\dotsb
	\quad(-1<x<1),
\]
得\[
	[x s(x)]'
	= \sum\limits_{n=1}^\infty \left( \frac{x^{n+1}}{n+1} \right)'
	= \sum\limits_{n=1}^\infty x^n
	= \frac{1}{1-x}
	\quad(\abs{x}<1).
\]
对上式积分,
得\[
	x s(x) = \int_0^x \frac{1}{1-x} \dd{x} = -\ln(1-x)
	\quad(-1 \leq x < 1).
\]
于是,当\(x\neq0\)时,有\(s(x) = -\frac{1}{x} \ln(1-x)\).

而\(s(0)\)可由\(s(0) = a_0 = 1\)得出,
或者由和函数的连续性得到,即\[
	s(0)
	= \lim\limits_{x\to0} s(x)
	= \lim\limits_{x\to0} \left[ -\frac{1}{x} \ln(1-x) \right]
	= 1.
\]
故\[
	s(x) = \left\{ \begin{array}{cl}
		-\frac{1}{x} \ln(1-x), & x\in[-1,0)\cup(0,1), \\
		1, & x=0.
	\end{array} \right.
\]
\end{solution}
\end{example}
